\documentclass[main.tex]{subfiles}
\begin{document}

\section*{1 October 2019}

Figure 22.8 in some PDF: run of adiabatic, radiation gradients vs \(\log T \).

We compute \(\nabla_{\text{ad}}\) and \(\nabla_{\text{rad}}\) and see whether the region is convective or radiative.

We can move from the Eulerian and Lagrangian formalisms using the continuity equation. In the Eulerian formalism:

\begin{equation}
  m(r) = \int_0^\infty 4 \pi r^2 \rho(x) \dd{x} \,.
\end{equation}

We define the mean molecular weight \(\mu \) with:
%
\begin{equation}
  \mu^{-1} = \sum _{i}  (1 + \nu_e (i)) \frac{X_i}{A_i}
\,,
\end{equation}
where \(\nu_{e} (i)\) is the number of free electrons coming from element \(i\), \(X_{i}\) is the abundance by mass fraction of the element \(i\), and \(A_{i}\) is its mass number. 

The variables \(X\), \(Y\) and \(Z\) represent the abundances of \(\ce{H}\), \(\ce{He}\) and metals, and satisfy \(X+Y+Z=1\).

We may need to know the metal mixture inside \(Z\), but often we can approximate it as the Sun's distribution.

The time evolution of the various elements' fractions is given by
%
\begin{equation}
  \pdv{X_i}{t} = \frac{m_i}{\rho} \sum _j \qty( r_{ji} - r_{ij})
\end{equation}

\subsubsection{Classification of stars}

\begin{enumerate}
  \item Low mass stars have between \(0.8\) and \(2\) solar masses. They develop an electron-degenerate core after their time on the Main Sequence. 
  \item Intermediate mass stars have masses between \(2\) and \(8\) \(M_{\odot}\). They start burning helium in a non-degenerate core, then they develop a degenerate \ce{C-O} core. 
  \item Massive stars have masses of over \(8 M_{\odot}\). They start burning carbon in a non-degenerate core. 
\end{enumerate}

\todo[inline]{What does \emph{degenerate} mean in this context?}

\todo[inline]{What are Hayashi lines?}

% Today we will look at typical time-scales, the period-mean density relation, the energy equation and perturbation theory w/ linearization.

% No derivations of the equations in the exam.

\subsection{Time-scales}

The \emph{free fall} time scale is:

\begin{equation}
  \tau_{\text{dyn}} \sim \qty(\frac{R}{g} )^{1/2} = \qty(\frac{R^3}{GM} )^{1/2}\,.
\end{equation}

It is associated with pulsation. It is calculated using the travel time of a mass in free fall across the stellar radius accelerated by constant acceleration equal to surface acceleration.

We note that \(\tau_{\text{dyn}} \propto \overline{\rho} ^{-1/2} \), since \(\overline{\rho} \propto M / R^3\). For the Sun it is about \SI{1.6e3}{s}.

The \emph{thermal} time scale is the relaxation time of deviations from thermal equilibrium:
%
\begin{equation}
  \tau_{\text{th}} \sim E_{\text{th}} / L\,.
\end{equation}

It is calculated as the time required for a star to irradiate all its energy. 

\begin{proof}
We use the virial theorem to estimate the thermal (so, local kinetic) energy \(T\): we know that \(T = - V/2\) where \(V\) is the total potential energy of the star. \(V\) can be computed as 
%
\begin{align}
V = - \int \frac{Gm}{r(m)} \dd{m}
\,,
\end{align}
%
and since \(r(m) \approx \sqrt[3]{3m / 4 \pi \rho }\) if the density is constant, we get 
%
\begin{align}
V &= - \int Gm \sqrt{\frac{4 \pi \rho }{3}} m^{-1/3} \dd{m}  \\
&=- \frac{3}{5} G \sqrt[3]{\frac{4 \pi \rho }{3}} M^{5/3} = -\frac{3}{5} \frac{GM^2}{R} \sim - \frac{GM^2}{R}
\,.
\end{align}

Therefore, \(T \sim - GM^2 / 2 R\).  
\end{proof}

Typically, \(\tau_{\text{th}} \sim G M^2 /(LR) \sim 10^7 M^2/(LR) \) years.  
For the Sun we have \(\tau_{\text{th}} \sim \SI{5e14}{s}\). 
It is much larger than the dynamic time scale.

The \emph{nuclear} time scale is even longer: it is calculated using the efficiency of nuclear fusion of \ce{H} to \ce{He}, which is about \(\epsilon \sim \num{.7} \% \), and the fraction of hydrogen in the star, which is about \(M_H = \num{10} \% \times M_{\odot} \). Using these numbers, we get for the Sun:
%
\begin{align}
\tau_{\text{nuc}} = \frac{\epsilon c^2  M_H}{L_{\odot}} \approx \SI{3e17}{s}
\,,
\end{align}
%


This allows us to say that oscillations will not be heavily affected by thermal conduction, and even less by nuclear processes: the pulsations will be almost \emph{adiabatic}.

The best candidate for these oscillations are \emph{sound waves}: is the adiabatic speed of sound roughly right?

The speed of sound is given by:
%
\begin{equation}
  v_s^2 = \pdv{P}{\rho } = \frac{P}{\rho} \pdv{\log P }{\log
  \rho } = \Gamma_1 \frac{P}{\rho}\,,
\end{equation}
%
where \(\Gamma_1 \) is the adiabatic exponent, such that \(P = \rho^{\Gamma_1 }\).  

If the gas follows the perfect equation of state

\begin{equation}
  \frac{P}{\rho} = \frac{k_B}{m_H} \frac{T}{\mu}  \,,
\end{equation}
%
where \(\mu \) is the average mass of a gas particle in atomic mass units (units of \(m_H\)), so we have \(\rho = \mu N m_H / V\). 
Substituting this in, we get

\begin{equation}
  v_s^2 = \frac{\Gamma_1 k_B T}{m_H \mu}  \,.
\end{equation}

The mean molecular weight is calculated taking account of the fact that the electrons appear in the count of particles, but we can neglect their mass: therefore, we get a \(+1\) in the formula. So the value becomes 
%
\begin{align}
\mu = \qty(\sum_{j} \frac{X_j}{A_j} (1+Z_j))^{-1} \approx \qty(2X + 3Y/4 + Z/2)^{-1} \sim 0.6
\,,
\end{align}
%
where we approximated that for each metal \((1+Z) / A \sim \frac[i]{1}{2} \) and we used the values of the mass fractions of the Sun: \(X =\num{.7381}\), \(Y = \num{.2485}\) and \(Z = \num{.0134}\).\footnote{See \url{https://arxiv.org/abs/0909.0948}} 

Typical values for the other parameters are \(\Gamma_1 = 5/3\), and \(T_{\text{He}} \sim \SI{4.5e4}{K} \).

So we get \(v_s \sim \SI{32.2}{km/s}\).

The timescale for a perturbation to go from one side of the star to the other is \(\Pi \sim 2R/v_s \sim \SI{22}{d}\), while the observed value is \(\Pi_{\text{obs}} = \SI{5.336}{d} \), so in terms of orders of magnitude it works. 
We could say that from a more thorough analysis we would see that the vibration does not actually go all the way from an edge of the star to the other. 

We can use the equation for the sound speed in the virial theorem, which for a star relates the gravitational potential energy \(\Omega \) to the integral of the pressure:
%
\begin{align}
\Omega  = - 3 \int P \dd{V} = - 3 \int \frac{P}{\rho } \dd{m}
\,,
\end{align}
%
and substitute in \(P/\rho = v_s^2 / \Gamma_1 \):
%
\begin{equation}
\Omega = - 3 \frac{\int _{M}  v_s^2 / \Gamma_1 \dd{m}}{\int_M \dd{m}}M = -3 \expval{\frac{v_s^2}{\Gamma_1}} M
\,,
\marginnote{We multiply and divide by \(M = \int \dd{m}\)}
\end{equation}
%
where the brackets denote an average weighted by the mass distribution. 

If \(\Gamma_1 \) and \(v_s\) are independent, we can compute their averages separately: we approximate and do this. Then, we can substitute in our expression for the average \(\expval{v_s^2} \sim \expval{v_s}^2\) into \(\Pi \sim 2R / v_s\): we get  
%
\begin{align}
\expval{v_s}^2 = - \frac{\Omega \Gamma_1 }{3M}
\,,
\end{align}
%
so
%
\begin{align}
\Pi \sim \frac{2R}{\expval{v_s}} = \sqrt{\frac{-4R^2 \times 3M }{\Omega \Gamma_1 }}
\,,
\end{align}
%

This means we are writing the period \(\Pi  \) with respect to something resembling the moment of inertia, \(I = \int_r r^2 \dd{m}(r) \sim R^2M\):

\begin{equation}
  \Pi \sim \qty(\frac{I_{\text{osc}}}{-\Omega} )^{1/2}
\end{equation}

This is further evidence that we are dealing with a dynamical phenomenon.

We can refine our model: the speed of sound changes throughout the interior of the star.
We compute the period as the travel time of sound waves throughout the diameter:
%
\begin{equation}
  \Pi = 2 \int _{0}   ^{R} \dd{t(r)} = 2 \int_0^R \frac{\dd{r} }{\sqrt{\Gamma_1(r) P(r) / \rho(r)}}\,,
\end{equation}
%
since \(\dd{t} = \dd{r} / v_s\); the factor of 2 comes from the fact that the sound wave must go from one side of the star to the other.

We also integrate the hydrostatic balance equation, which reads 
%
\begin{align}
\dv{P}{r} = - \frac{Gm \rho }{r^2} 
= - \frac{G \rho^2 4 \pi r}{3}
\,,
\end{align}
%
with respect to \(r\): then we get 
%
\begin{align}
P(r) - \underbrace{P(R)}_{=0} = \int_{R}^{r} \dv{P}{r} \dd{r}  = \frac{2 \pi \rho^2 G}{3} \qty(R^2 - r^2)
\,,
\end{align}
%
which we can plug into our formula for the period to find: 
%
\begin{align}
\Pi &= 2 \int \dd{r} \qty(\Gamma_1 \frac{P}{\rho })^{-1/2}  \\
&= 2 \int \dd{r} \qty(\Gamma_1 \frac{2 \pi \rho G}{3} \qty(R^2 - r^2))^{-1/2}  \\
&= \sqrt{\frac{6}{\Gamma_1 \pi G \overline{\rho}}} \underbrace{\int \frac{ \dd{r}}{\sqrt{R^2- r^2}}}_{\pi /2}  \\
&= \sqrt{\frac{3 \pi }{2\Gamma_1 G \overline{\rho} }}
\,,
\end{align}
%
which confirms Ritter's relation \(\Pi \propto \overline{\rho}^{-1/2} \).
Since the product \(\Pi \overline{\rho}^{1/2}\) is approximately constant, we give it the name 
%
\begin{align}
\mathcal{Q} = \Pi \sqrt{\overline{\rho}} \approx 
\sqrt{\frac{3 \pi }{2 \Gamma_1 G}} 
\,.
\end{align}

Ritter's relation is consistent with the statement that the period of the oscillations is of the order of the dynamical characteristic time of the star, since 
%
\begin{align}
\tau _{\text{dyn}} = \sqrt{\frac{R^3}{GM}} \qquad \text{while} \qquad \overline{\rho}^{-1/2} \approx \qty(\frac{M}{4 \pi R^3 / 3})^{-1/2} = \sqrt{\frac{4 \pi R^3}{3M}}
\,.
\end{align}

This gives an estimate of \(\Pi \sim \SI{8.5}{d}\) for a \(\delta \)-Cephei star, which we have to compare to the observed period of \(\Pi _{\text{obs}} = \SI{5.466}{d}\). We have definitely improved our estimate. 
The prediction of this model is that \emph{dense stars pulstate faster}. 

This works for acoustic modes, such as those found in \(\delta \)-Cephei, \(o\)-Ceti and SX-Phe stars, but if we consider non-radial g-modes such as those found in variables of type ZZ Ceti it stops working. Then, the estimate given by this model can be off by three orders of magnitude. 

\subsection{The energy equations}



\end{document}
