\documentclass[main.tex]{subfiles}
\begin{document}

\section*{1 October 2019}

Figure 22.8 in some PDF: run of adiabatic, radiation gradients vs \(\log T \).

We compute \(\nabla_{\text{ad}}\) and \(\nabla_{\text{rad}}\) and see whether the region is convective or radiative.

We can move from the Eulerian and Lagrangian formalisms using the continuity equation. In the Eulerian formalism:

\begin{equation}
  m(r) = \int_0^\infty 4 \pi r^2 \rho(x) \dd{x} \,.
\end{equation}

We define the mean molecular weight \(\mu \) with:
%
\begin{equation}
  \mu^{-1} = \sum _{i}  (1 + \nu_e (i)) \frac{X_i}{A_i}
\,,
\end{equation}
where \(\nu_{e} (i)\) is the number of free electrons coming from element \(i\), \(X_{i}\) is the abundance by mass fraction of the element \(i\), and \(A_{i}\) is its mass number. 

The variables \(X\), \(Y\) and \(Z\) represent the abundances of \(\ce{H}\), \(\ce{He}\) and metals, and satisfy \(X+Y+Z=1\).

We may need to know the metal mixture inside \(Z\), but often we can approximate it as the Sun's distribution.

The time evolution of the various elements' fractions is given by
%
\begin{equation}
  \pdv{X_i}{t} = \frac{m_i}{\rho} \sum _j \qty( r_{ji} - r_{ij})
\end{equation}

\subsubsection{Classification of stars}

\begin{enumerate}
  \item Low mass stars have between \(0.8\) and \(2\) solar masses. They develop an electron-degenerate core after their time on the Main Sequence. 
  \item Intermediate mass stars have masses between \(2\) and \(8\) \(M_{\odot}\). They start burning helium in a non-degenerate core, then they develop a degenerate \ce{C-O} core. 
  \item Massive stars have masses of over \(8 M_{\odot}\). They start burning carbon in a non-degenerate core. 
\end{enumerate}

\todo[inline]{What does \emph{degenerate} mean in this context?}

\todo[inline]{What are Hayashi lines?}

% Today we will look at typical time-scales, the period-mean density relation, the energy equation and perturbation theory w/ linearization.

% No derivations of the equations in the exam.

\subsection{Time-scales}

The \emph{free fall} time scale is:

\begin{equation}
  \tau_{\text{dyn}} \sim \qty(\frac{R}{g} )^{1/2} = \qty(\frac{R^3}{GM} )^{1/2}\,.
\end{equation}

It is associated with pulsation. It is calculated using the travel time of a mass in free fall across the stellar radius accelerated by constant acceleration equal to surface acceleration.

We note that \(\tau_{\text{dyn}} \propto \overline{\rho} ^{-1/2} \), since \(\overline{\rho} \propto M / R^3\). For the Sun it is about \SI{1.6e3}{s}.

The \emph{thermal} time scale is the relaxation time of deviations from thermal equilibrium:
%
\begin{equation}
  \tau_{\text{th}} \sim E_{\text{th}} / L\,.
\end{equation}

It is calculated as the time required for a star to irradiate all its energy. 

\begin{proof}
We use the virial theorem to estimate the thermal (so, local kinetic) energy \(T\): we know that \(T = - V/2\) where \(V\) is the total potential energy of the star. \(V\) can be computed as 
%
\begin{align}
V = - \int \frac{Gm}{r(m)} \dd{m}
\,,
\end{align}
%
and since \(r(m) \approx \sqrt[3]{3m / 4 \pi \rho }\) if the density is constant, we get 
%
\begin{align}
V &= - \int Gm \sqrt{\frac{4 \pi \rho }{3}} m^{-1/3} \dd{m}  \\
&=- \frac{3}{5} G \sqrt[3]{\frac{4 \pi \rho }{3}} M^{5/3} = -\frac{3}{5} \frac{GM^2}{R} \sim - \frac{GM^2}{R}
\,.
\end{align}

Therefore, \(T \sim - GM^2 / 2 R\).  
\end{proof}

Typically, \(\tau_{\text{th}} \sim G M^2 /(LR) \sim 10^7 M^2/(LR) \) years.  
For the Sun we have \(\tau_{\text{th}} \sim \SI{5e14}{s}\). 
It is much larger than the dynamic time scale.

The \emph{nuclear} time scale is even longer: it is calculated using the efficiency of nuclear fusion of \ce{H} to \ce{He}, which is about \(\epsilon \sim \num{.7} \% \), and the fraction of hydrogen in the star, which is about \(M_H = \num{10} \% \times M_{\odot} \). Using these numbers, we get for the Sun:
%
\begin{align}
\tau_{\text{nuc}} = \frac{\epsilon c^2  M_H}{L_{\odot}} \approx \SI{3e17}{s}
\,,
\end{align}
%


This allows us to say that oscillations will not be heavily affected by thermal conduction, and even less by nuclear processes: the pulsations will be almost \emph{adiabatic}.

The best candidate for these oscillations are \emph{sound waves}: is the adiabatic speed of sound roughly right?

The speed of sound is given by:
%
\begin{equation}
  v_s^2 = \pdv{P}{\rho } = \frac{P}{\rho} \pdv{\log P }{\log
  \rho } = \Gamma_1 \frac{P}{\rho}\,,
\end{equation}
%
where \(\Gamma_1 \) is the adiabatic exponent, such that \(P = \rho^{\Gamma_1 }\).  

If the gas follows the perfect equation of state

\begin{equation}
  \frac{P}{\rho} = \frac{k_B}{m_H} \frac{T}{\mu}  \,,
\end{equation}
%
where \(\mu \) is 
we get

\begin{equation}
  v_s^2 = \frac{\Gamma_1 k_B T}{m_H \mu}  \,.
\end{equation}

Typical values are \(\mu \sim \qty(2X + 3Y/4 + Z/2)^{-2} \sim 0.6\), \(\Gamma_1 = 5/3\), and \(T_{\text{He}} \sim \SI{4.5e4}{K} \).

So we get \(v_s \sim \SI{32.2}{km/s}\).

The timescale is \(\Pi \sim 2R/v_s \sim \SI{22}{d}\), while \(\Pi_{\text{obs}} = \SI{5.336}{d} \), in terms of orders of magnitude it works.

We can use the equation for the sound speed in the virial theorem, and get:

\begin{equation}
  \Omega = - 3 \frac{\int _{M}  v_s^2 / \Gamma_1 \dd{m}}{\int_M \dd{m}}M = -3 \expval{\frac{v_s^2}{\Gamma_1}} M
\end{equation}

If \(\Gamma_1 \) and \(v_s\) are independent, we can compute their averages separately.

This allows us to write \(\Pi  \) wrt the moment of inertia, \(I = \int_r r^2 \dd{m}(r)\):

\begin{equation}
  \Pi \sim \qty(\frac{I_{\text{osc}}}{-\Omega} )^{1/2}
\end{equation}

This is further evidence that we are dealing with a dynamical phenomenon.

Of course the speed of sound changes throughout the interior of the star.
We compute the period as the travel time of sound waves throughout the diameter:
%
\begin{equation}
  \Pi = 2 \int _{0}   ^{R} \dd{t(r)} = 2 \int_0^R \frac{\dd{r} }{\sqrt{\Gamma_1(r) P(r) / \rho(r)}}\,,
\end{equation}
%
since \(\dd{t} = \dd{r} / v_s\).

We also rewrite the differential equation for \(P\) substituting \(m=\rho r\). Doing this we get

\begin{equation}
  \Pi  \overline{\rho}^{1/2}= \sqrt{\frac{3 \pi}{2 \Gamma_1 G}}
\end{equation}

Which confirms Ritter's relation \(\Pi \propto \overline{\rho}^{-1/2} \) .

This works for acoustic modes, but if we consider non-radial g-modes it stops working, such as variables of type ZZ Ceti.

\subsection{The energy equations}

Just an overview for now: we will consider the star as a thermodynamic engine.

\end{document}
