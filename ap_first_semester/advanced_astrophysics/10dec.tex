\documentclass[main.tex]{subfiles}
\begin{document}

\section*{Tue Dec 10 2019}

We continue with the evolution of a \(15 M_{\odot}\) star. 
We use Kippenhahn diagrams. 

There are four major burning phases: 
%
\begin{enumerate}
    \item Hydrogen;
    \item Helium;
    \item Carbon;
    \item Silicon.
\end{enumerate}
\todo[inline]{Check}

There can be different burning stages simultaneously, in different shells. 

Stellar winds are of general application in the study of stellar evolution. 

We can make an H-R diagram using, instead of the luminosity \(L\), the bolometric magnitude 
%
\begin{align}
  M _{\text{bol}} = -2.5 \log L + 4.73
\,,
\end{align}
%
where the luminosity is measured in solar luminosities. 

The Eddington luminosity is calculated by  setting
%
\begin{align}
  \Gamma_{\text{Edd}} = \frac{a _{\text{rad}}}{g} = \frac{\kappa L}{4 \pi R^2 c} \frac{R^2}{GM} =1
\,,
\end{align}
%
which implies 
%
\begin{align}
  L _{\text{Edd}} = \frac{cGM 4 \pi }{\kappa_e}
\,.
\end{align}

Wolf-Rayet stars have high effective temperatures and high luminosities, they expel a great quantity off mass through stellar winds. 

The spectra of WR stars have little of no \ce{H}, and have an abundance of either \ce{He + N} or \ce{C + O}.

We distinguigh them into 
\begin{enumerate}
    \item WNL;
    \item WNE;
    \item WC;
    \item WO.
\end{enumerate}

These are actually evolutionary stages: the outer layers are successively stripped by winds, and the inner ones are exposed.
We observe lots of \ce{^{14}N} in WNE stars, since the Nitrogen burning phase since that is the slowest process. 

We have these stars when \(\log T _{\text{eff}} > 4\), and \(M > 30 M_{\odot}\). 

Depending on the mass of the progenitor red supergiant, we have different behaviours: below \(30 M_{\odot} \) the star explodes as a red supergiant; more massive stars will experience more mass loss and explode as blue supergiants, further left on the HR diagram. 

We see a simulation, a \(15 M_{\odot}  \) star starts off on the high part of the Main Sequence, then quickly moves right becoming a red supergiant when its core collapses.

A \(3 M_{\odot}\) star instead moves away from the MS more slowly, and then after having moved right it quickly moves far left and cools, as a white dwarf. 

Now, let us discuss the engine of the explosion of these massive stars. The explosion is triggered by an implosion: we are at the end of the Silicon-burning phase. 
The pressure of the core is mantained by the degenerate electrons. 

The degenerate iron core starts off with \(\rho \approx
\SI{e9}{g cm^{-3}}\), \(T \approx \SI{e10}{K}\), \(M _{\text{Fe}} \approx \num{1.5} M_{\odot}\). 

What triggers the collapse of the iron core? The adiabatic exponent 
%
\begin{align}
  \gamma _{\text{ad}} = \qty(\pdv{\log P}{\log \rho })_{\text{ad}}
\,
\end{align}
%
falls below the critical value of \(4/3\). 
In general, if it falls below this value we expect a dynamical instability. It is a local quantity, it can be defined for each layer; we can use a global parameter to measure the instability: 
%
\begin{align}
  \int \qty(\gamma _{\text{ad}} - \frac{4}{3}) \frac{P}{\rho } \dd{m}
\,,
\end{align}
%
which will tell us whether the system as a whole is stable. 
\todo[inline]{Why the factor \(P/\rho \) in the average?}

There is a very important weak process: electron capture, or \emph{neutronization}: \(\ce{p+ + e- \rightarrow n +} \nu_{e}\). This decreases the electron pressure!

The Chandrasekar mass scales as the square of the mean molecular weight: 
%
\begin{align}
  M _{\text{ch}} = \frac{5.83}{\mu_{e}^2} \sim 1.26 M_{\odot}
\,
\end{align}
%
for the iron core. A huge amount of neutrinos are produced. 
Now we follow a work by Janka et al (2007) at the Max Planck, which outlines the steps of the collapse. 

As the density of the core increases, it becomes opaque to neutrinos: we have the scattering cross section 
%
\begin{align}
  \sigma_{\nu } \approx \num{e-49} A^2 \qty(\frac{\rho }{\mu_{e}})^{2/3} \SI{}{cm^2}
\,,
\end{align}
%
and in the meantime the collapse is proceding, until it reaches values of \(\rho \sim \SI{e14}{g cm^{-3}}\), comparable to the nuclear density, at which point the material becomes incompressible. 

A proto-neutron star is formed, and a shock front travels back. 

\todo[inline]{What triggers the initial infall and compression in the first place? }

It is unclear whether the energy of the shock is enough to blow off the envelope. The energy of the delayed neutrinos which are still trapped in the nucleus will contribute\dots

Now we start an interesting new part: the advanced burning stages, after the \ce{He}-burning phases. 

As the initial mass increases, the fraction of carbon produced decreases from around \num{.35} to \num{.15} as \(M\) goes from 10 to 120 solar masses. 

What about the density structure in the pre-supernova phase? 
The starting mass of the iron core increases with the stellar mass: as it goes from 10 to 120 solar masses it goes from 1.2 to 1.8 solar masses. 

As the total mass increases, at a fixed radius we have more internal mass: the density \emph{increases} with mass. 

Some of the material near the core, when the shock occcurs, falls back on it if it is inside a certain critical radius, if it is outside that radius it is thrown out. 
How much of it depends on the physical properties of the core. 

What does the remnant look like? It can either be a black hole or a neutron star. The possibilities 
\begin{enumerate}
    \item Explosion and Neutron Star;
    \item Implosion and Black Hole;
    \item (rarely) Explosion and Black Hole.
\end{enumerate}

There is no clear-cut law. One thing to look at is the bounce-compactness parameter: 
%
\begin{align}
  \xi_{M^*} = \abs{\frac{M^{*} / M_{\odot}}{R(M^{*}) / \SI{e7}{m}}}_{t = t _{\text{bounce}}}
\,,
\end{align}
%
and it seems like this paramter being high is correlated to a black hole being formed. 
We have also the parameter 
%
\begin{align}
  \mu_{4}  = \abs{\frac{ \dd{m} / M_{\odot}}{ \dd{\tau } / \SI{e7}{m}}}_{s=4}
\,,
\end{align}
%
the normalized mass inside a dimensionless entripy per nucleon of \(s=4\). 
We can make a plot of the various final fates of the parts of the initial mass: some of it is expelled by winds, some of it is expelled through the supernova explosion, some of it ends up in the final remnant.

The mass of the iron core is never very much larger than a couple of solar masses, but BHs can become more massive through fallback. 

Then it seems that WR stars of more than \(30 M_{\odot}\) will not explode: they will not eject material as supernovas. 

\todo[inline]{Simulations show that there is not a clear-cut boundary between neutron star and BH formation, but then we go on as if there were\dots}

The behaviour of LA89 massive stars is quite different, NS are more favoured towards higher masses. 

\end{document}