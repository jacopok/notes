\documentclass[main.tex]{subfiles}
\begin{document}

\section*{Mon Nov 11 2019}

Tomorrow we have the meeting at 13.30 in room C.

On the 19th the lecture is in room A (still @ Specola).

\section{Line driven winds}

Now we deal with line driven winds.
The main mechanism is line scattering.

\subsection{P-Cygni profiles}

They are spectra characterized  by a blue-shifted absorption and a red-shifted emission.

We consider a source which emits a spherically symmetric wind.
The region of the wind which is directed at us corresponds to a blue-shifted absorption.
We basically have emission which is symmetric centered at \(v=0\) (since most of the radiation comes from the photosphere), and absorption which is caused by atoms moving towards us at velocities \(0 \leq v \leq + v_\infty\): so the absorption is something like \(- k [0 \leq v \leq v_\infty]\).

We can gather data from these profiles: for example, the maximum Doppler shift of the absorption corresponds to the maximum velocity.

We can also infer the number densities of the chemical species in the wind: 
%
\begin{align}
  n_i  = \frac{X_i \rho }{A_i m_u}
\,,
\end{align}
%
assuming some parametric velocity profile \(v(r)\) and density profile \(\rho (r)\), by this we derive the mass loss rate.

The opacity of the spectral lines is several orders of magnitude larger than the continuum opacity: for example, the opacity of the IV line of Carbon is \(\sim \num{e6} \kappa_{es}\).

When a photon is absorbed, the momentum increases by \(\Delta p = h \nu / c \). 
The increase in velocity when a typical metal absorbs an UV photon with wavelength \SI{e-7}{m} is of the order \(\Delta v \approx \SI{20}{cm/s}\). Typically, due to the redistribution of momentum among atoms, this is something like four orders of magnitude less.
All the gas is then accelerated by the radiation.

In order to accelerate the gas to \SI{2000}{km/s} we'd need \num{e11} photons.
If the distance to accelerate to terminal velocity is about three sun radii, the time to accelerate is of the order \SI{e4}{s}.
So we need around \num{e7} photons per second: the typical lifetime of the transition is of the order \SI{e-7}{s}, 

So only transitions with oscillator strengths \(f \gtrsim 0.01\) will contribute significantly.

\todo[inline]{What is oscillator strength? for \(\Delta t \approx \SI{e-7}{s}\) we have \(\Delta E = \hbar / \Delta t \approx \SI{6}{neV}\)...}

The largest contribution is not the energy of the photons, but their momentum input.

Let us calculate the radiation pressure due to one line. The momentum equation is 
%
\begin{align}
  v \dv[]{v}{r} = - \frac{1}{\rho } \dv[]{P}{r} - \frac{GM}{r^2} + f(r)
\,,
\end{align}
%
and we want to compute the line-driven \(f(r)\).

Say we have a line @ wavelength \(\lambda_0 \),  which we assume coincides with the peak of the Planckian function for the star.
We also assume that the line is so strong that it absorbs or scatters \emph{all} the photons at its wavelength.

\todo[inline]{So, photons at that wavelength have 0 mean free path?

It seems like that is not necessary actually.}

How much mass loss can one optically thick line produce? 

The emitted wavelengths which are absorbed are all the ones between \(\lambda_0\) and \(\lambda_0 (1 - v_{\infty}/c)\).

If our velocity range goes from \(v=0\) ar \(r = R\) to \(v = v_{\infty}\) at \(r = \infty\), then the total energy absorbed is: 
%
\begin{align}
  L _{\text{line}} = \int _{\nu_0 }^{\nu_0 (1 + v_{\infty}/c)} \underbrace{F_\nu 4 \pi R^2}_{{L_{\nu}}} \dd{\nu } \approx L_{\nu_0 } \Delta \nu = L_{\nu_0 } \nu_0 v_{\infty} /c
\,,
\end{align}
%
where \(F_{\nu }\) is the monochromatic flux (specific spectral intensity in \(\SI{}{\erg\per\square\centi\metre\per\second\per\hertz}\)) at the photosphere, whose radius is \(R\).
We approximate the flux to be almost constant with respect to \(\nu \).

\(L_{\nu_0 }\) is the specific luminosity per unit frequency at \(\nu_0 \).

The variation of radiative momentum is 
%
\begin{align}
  \dv[]{p _{\text{rad}}}{t} = \frac{L _{\text{line}}}{c}
  = \frac{L_{\nu_0 } \nu_0 v_{\infty}}{c^2}
\,,
\end{align}
%
and the total momentum loss in the wind per second is 
%
\begin{align}
  \dv[]{p _{\text{wind}}}{t} = \dot{M} v_{\infty} 
\,,
\end{align}
%
so if we approximate the wind to be momentum wind, we can equate these two terms: then we get 
%
\begin{align}
  \frac{L_{\nu_0 } \nu_0}{c^2} = \dot{M} 
\,.
\end{align}

Now, the Planck function has the property that \(L_{\nu_0  } \nu_0 = 0.62 L\).
Then we see that the mass loss rate of one strong line is linear in the luminosity of the star.

These are additive: if we can approximate the mass loss rate and luminosity with other means, we can find the total number of spectral lines.

\end{document}