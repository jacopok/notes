\documentclass[main.tex]{subfiles}
\begin{document}

\section*{Mon Nov 11 2019}

Tomorrow we have the meeting at 13.30 in room C.

On the 19th the lecture is in room A (still @ Specola).

\section{Line driven winds}

Now we deal with line driven winds.
The main mechanism is line scattering.

\subsection{Spectral lines and P-Cygni profiles}

The spectrum of a star can be roughly approximated as a blackbody spectrum: Wien's law, then, tells us that the maximum of the spectral intensity occurs for a frequency which is proportional to the temperature of the star.

\begin{bluebox}
Specifically, the blackbody spectrum has the shape: 
%
\begin{align}
I_\nu \propto \frac{\nu^3}{\exp(\frac{h \nu }{k_B T}) - 1}
\,,
\end{align}
%
so its maximum can be found by differentiating: setting \(x = h \nu / k_B T\), this yields the equation \(x/3 = 1- e^{-x}\), which can be solved numerically to see that the solutions are \(x=0\) and \(x \approx\num{2.82143937}\).
We keep the positive solution: then we can see that 
%
\begin{align}
\nu _{\text{max}} \approx \num{2.82} \frac{k_B T}{h}
\,.
\end{align}

This does \emph{not} correspond to the maximum of the intensity expressed in terms of the wavelength: that formula looks like 
%
\begin{align}
I_\lambda \propto \frac{\lambda^{-5}}{\exp(\frac{hc}{\lambda k_B T}) - 1}
\,,
\end{align}
%
whose maximum can be found similarly to before: if we set \(x = \lambda k_B T / hc\), the equation to solve becomes 
%
\begin{align}
-5 x \qty(e^{1/x} -1) + e^{1/x} = 0
\,,
\end{align}
%
which can be solved numerically to yield \(x \approx 1/\num{4.96511423174}\), so 
%
\begin{align}
\frac{1}{\lambda _{\text{max}}} \approx 4.97 \frac{k_B T}{hc}
\,.
\end{align}
%
These are both artefacts of the fact that we are plotting probability distributions: the variable matters, since to go from one to the other we must multiply by \(\abs{\dv*{\nu }{\lambda }}\).
A figure which is more representative since it is variable-independent is the \emph{average} energy of a photon: 
%
\begin{align}
E _{\text{mean}} \approx 3.832 k_B T
\,.
\end{align}
\end{bluebox}

Many of the strongest absorption lines of abundant elements such as Carbon, Nitrogen, Oxygen, Silicon, Sulfur and Iron are in the UV, where the flux of stars is usually low. However, if the star is very hot, then the peak of the Planckian is near or in the UV band: specifically, the peak of the distribution in wavelength moves into the UV when the temperature is somewhere between \SI{7000}{K} and \SI{8000}{K}.

The spectral lines in winds are characterized by a large width: the wavelength shift is due to the Doppler effect from the outflowing motion.
\subsubsection{Spectral line formation}

\paragraph{Line scattering}
A process like: 
%
\begin{align}
\ket{0} + \gamma _{\text{in}} \rightarrow \ket{1}
\rightarrow \ket{0} + \gamma _{\text{out}}
\,,
\end{align}
%
the energies of the in and out photons are almost the same, they may be slightly different due to the Doppler shift due to thermal motion.
The direction of the photons may be different. If \(\ket{0}\) is indeed the ground state of the system, this is called \emph{resonance scattering}. 

\paragraph{Emission by recombination}
An electron collides with an ion (which is in a high-energy continuum state), and causes it to recombine into a less energetic state: 
%
\begin{align}
e^{-} + \ket{\text{cont}}_{\text{ion}} 
% \qty(\rightarrow
% \ket{n} + \gamma 
% ) 
\rightarrow
\ket{0} + \gamma 
\,,
\end{align}
%
it may pass through one or several excited state(s) on its way to the ground state.
For each deexcitation we have a spectral line photon.
We can see this as emission lines in a spectrum.
\todo[inline]{What is HEX emission?}

\paragraph{Emission from collisional or photo-excitation}
Collisional excitation is a process like:
%
\begin{align}
\ket{0} \overset{\text{collision}}{\rightarrow} \ket{2} \rightarrow \ket{1} + \gamma 
\rightarrow \ket{0} + 2 \gamma 
\,
\end{align}
%
with \(m < n\). The collisional excitation is efficient at high temperatures and densities, in plasmas: because of this process we see emission lines from hot coronas.

Photo excitation is a process like: 
%
\begin{align}
\ket{0} + \gamma \rightarrow
\ket{2} 
\rightarrow
\ket{1} + \gamma 
\rightarrow
\ket{0} + 2 \gamma 
\,:
\end{align}
%
it occurs if a photon excites the ground state into a higher state, but the descent to lower energy states occurs in more then one step: then a higher-energy photon is absorbed and two or more lower-energy ones are emitted.
In principle we could see emission lines from this process in stellar winds, but in practice the most likely thing to happen is simply resonance scattering, which does not significantly change the energy of the photon.

\paragraph{Pure absorption}
A process like 
%
\begin{align}
\ket{1} + \gamma \rightarrow \ket{2} 
\rightarrow \ket{0}  + \gamma 
\,,
\end{align}
%
which is not relevant for stellar winds since it needs an excited atom to start and there are few, the vast majority are in their ground state.

\paragraph{Masering by stimulated emission}
A process like 
%
\begin{align}
\ket{1} + \gamma 
\rightarrow \ket{0} + 2 \gamma 
\,,
\end{align}
%
where the two emitted photons are equal both in energy and direction to the incoming one. 

If there is no velocity gradient, this process can generate a strong and narrow emission line. 
A velocity gradient inhibits it because of the Doppler effect.
This process is relevant for the winds of cool stars.

\subsubsection{P-Cygni profiles}

We have components due both to absorption and emission: these make up the so-called P-Cygni profile: a spectral line characterized  by a blue-shifted absorption and a red-shifted emission with respect to the base spectrum profile.

They are mainly found for the resonance lines of C IV, N V, Si II and Mg II. 

We consider a source which emits a spherically symmetric wind, continuum radiation and the gas has an absorption line at some wavelength \(\lambda_0 \). 

This wavelength is shifted to \(\lambda = \lambda_0 (1 + v)\), where \(v\) is the velocity of the gas relative to the emission: it is positive if the gas is receding away from the emitter, and negative otherwise.

The region of the wind which is directed at us corresponds to a blue-shifted absorption.
We basically have emission which is symmetric centered at \(v=0\) (since most of the radiation comes from the photosphere), and absorption which is caused by atoms moving towards us at velocities \(0 \leq v \leq + v_\infty\): so the absorption is something like \(- k [0 \leq v \leq v_\infty]\).

We can gather data from these profiles: for example, the maximum Doppler shift of the absorption corresponds to the maximum velocity.

We can also infer the number densities of the chemical species in the wind: 
%
\begin{align}
  n_i  = \frac{X_i \rho }{A_i m_u}
\,,
\end{align}
%
assuming some parametric velocity profile \(v(r)\) and density profile \(\rho (r)\), by this we derive the mass loss rate.

The opacity of the spectral lines is several orders of magnitude larger than the continuum opacity: for example, the opacity of the IV line of Carbon is \(\sim \num{e6} \kappa_{es}\).

When a photon is absorbed, the momentum increases by \(\Delta p = h \nu / c \). 
The increase in velocity when a typical metal absorbs an UV photon with wavelength \SI{e-7}{m} is of the order \(\Delta v \approx \SI{20}{cm/s}\). Typically, due to the redistribution of momentum among atoms, this is something like four orders of magnitude less.
All the gas is then accelerated by the radiation.

In order to accelerate the gas to \SI{2000}{km/s} we'd need \num{e11} photons.
If the distance to accelerate to terminal velocity is about three sun radii, the time to accelerate is of the order \SI{e4}{s}.
So we need around \num{e7} photons per second: the typical lifetime of the transition is of the order \SI{e-7}{s}, 

So only transitions with oscillator strengths \(f \gtrsim 0.01\) will contribute significantly.

\todo[inline]{What is oscillator strength? for \(\Delta t \approx \SI{e-7}{s}\) we have \(\Delta E = \hbar / \Delta t \approx \SI{6}{neV}\)...}

The largest contribution is not the energy of the photons, but their momentum input.

Let us calculate the radiation pressure due to one line. The momentum equation is 
%
\begin{align}
  v \dv[]{v}{r} = - \frac{1}{\rho } \dv[]{P}{r} - \frac{GM}{r^2} + f(r)
\,,
\end{align}
%
and we want to compute the line-driven \(f(r)\).

Say we have a line @ wavelength \(\lambda_0 \),  which we assume coincides with the peak of the Planckian function for the star.
We also assume that the line is so strong that it absorbs or scatters \emph{all} the photons at its wavelength.

\todo[inline]{So, photons at that wavelength have 0 mean free path?

It seems like that is not necessary actually.}

How much mass loss can one optically thick line produce? 

The emitted wavelengths which are absorbed are all the ones between \(\lambda_0\) and \(\lambda_0 (1 - v_{\infty}/c)\).

If our velocity range goes from \(v=0\) ar \(r = R\) to \(v = v_{\infty}\) at \(r = \infty\), then the total energy absorbed is: 
%
\begin{align}
  L _{\text{line}} = \int _{\nu_0 }^{\nu_0 (1 + v_{\infty}/c)} \underbrace{F_\nu 4 \pi R^2}_{{L_{\nu}}} \dd{\nu } \approx L_{\nu_0 } \Delta \nu = L_{\nu_0 } \nu_0 v_{\infty} /c
\,,
\end{align}
%
where \(F_{\nu }\) is the monochromatic flux (specific spectral intensity in \(\SI{}{\erg\per\square\centi\metre\per\second\per\hertz}\)) at the photosphere, whose radius is \(R\).
We approximate the flux to be almost constant with respect to \(\nu \).

\(L_{\nu_0 }\) is the specific luminosity per unit frequency at \(\nu_0 \).

The variation of radiative momentum is 
%
\begin{align}
  \dv[]{p _{\text{rad}}}{t} = \frac{L _{\text{line}}}{c}
  = \frac{L_{\nu_0 } \nu_0 v_{\infty}}{c^2}
\,,
\end{align}
%
and the total momentum loss in the wind per second is 
%
\begin{align}
  \dv[]{p _{\text{wind}}}{t} = \dot{M} v_{\infty} 
\,,
\end{align}
%
so if we approximate the wind to be momentum wind, we can equate these two terms: then we get 
%
\begin{align}
  \frac{L_{\nu_0 } \nu_0}{c^2} = \dot{M} 
\,.
\end{align}

Now, the Planck function has the property that \(L_{\nu_0  } \nu_0 = 0.62 L\).
Then we see that the mass loss rate of one strong line is linear in the luminosity of the star.

These are additive: if we can approximate the mass loss rate and luminosity with other means, we can find the total number of spectral lines.

\end{document}