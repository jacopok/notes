\documentclass[main.tex]{subfiles}
\begin{document}

\section*{Tue Oct 15 2019}

\subsubsection{\(\epsilon \) mechanism}

We now concern ourselves with the energy generation term of equation \eqref{eq:kappa-driving-mechanisms}.

In general the effective energy generation per unit mass is given by 
%
\begin{align}
\epsilon _{\text{eff}} = \epsilon _{\text{nuc}} - \epsilon _{\nu }
\,,
\end{align}
%
but we assume that the neutrino loss of energy contribution in the perturbation is negligible, so when we perturb we will have \(\delta \epsilon _{\text{eff}} = \delta \epsilon _{\text{nuc}}\). 
We also have the approximate relation 
%
\begin{align}
\frac{ \delta T}{T} = (\Gamma_3 -1 ) \frac{ \delta \rho }{\rho }
\,,
\end{align}
%
which implies, since \(\Gamma_3 >1\), that the temperature and density relative perturbations are concordant in sign.
\todo[inline]{Why can we use this relation? Does it have something to do with working ``at first order in \(\dv*{Q}{t}\)''?}

Now, the line of reasoning goes like this: the nuclear energy generation perturbation is concordant in sign with the temperature perturbation. So, the term 
%
\begin{align}
\qty(\frac{ \delta T}{T} )_{\text{sp}} \delta \epsilon _{\text{eff}} \geq 0
\,,
\end{align}
%
which means that the contribution to \(\kappa \) will be negative (because of the minus sign in equation \eqref{eq:kappa-driving-mechanisms}): so, since the dependence of the radial perturbation looks like \(\zeta \propto \exp(-\kappa t)\), the perturbation is \emph{amplified}: this is a driving layer. 

\todo[inline]{The reasoning in slide 6.08 seems way too contorted: plus, it mentions the layers ``absorbing energy'', but we cannot actually talk about that without considering the luminosity gradient perturbation\dots}

Are the fluctuations in \(\epsilon _{\text{nuc}}\) actually enough to power pulsations? \emph{No}. 
An illustrative example is given by RR-Lyrae stars. There, we can look at the relative magnitude of the realative temperature variation, \(\delta T / T\), at various layers in the star: it is of the order \num{e-1} or even more at the surface, but quickly drops as we move towards the center; in the Hydrogen burning shell and in the Helium burning core it is of the order \(\num{e-9} \divisionsymbol \num{e-7}\), which implies that the actual temperature fluctuations are of the order \(\delta T \sim \SI{1}{K}\): way too little to amplify pulsations. 

\subsubsection{\(\kappa \)-\(\gamma \) mechanism}

Today we look at the \(\kappa \)-\(\gamma \)-mechanism, which is about the term 
%
\begin{equation}
\int_M \qty(\frac{\delta T}{T})_{\text{sp}} \pdv{\delta L}{m} \dd{m} 
\,,
\end{equation}
%
and we will have driving layers (\(\kappa <0\)) in the regions in which the two terms multiplied in the integrand are discordant. 

The mean Rosseland opacity is approximated by a law in the form: 
%
\begin{equation}
  \kappa _R \approx \overline{\kappa }_R \rho^{n} T^{-s} 
\,,
\end{equation}
%
with \(n \approx 1\), \(s \approx 7/2\) in the case of free-free absorption (that is, inverse \& direct bremsstrahlung) in a non-degenerate, totally ionized gas.
This seems to be a good approximation for \(4 < \log_{10}T < 8\). 

This allows us to relate the perturbations in \(\kappa_{R} \) to those in \(T\): 
%
\begin{subequations}
\begin{align}
\frac{ \delta \kappa_R}{ \kappa_R} &\approx
n \frac{ \delta \rho }{\rho } - s \frac{ \delta T}{T}  \\
&\approx \qty(n - s (\Gamma_3 -1)) \frac{ \delta \rho }{\rho }
\,,
\end{align}
\end{subequations}
%
and if we substitute in \(\Gamma_3 \approx 5/3\) with the other terms we get 
%
\begin{align}
\frac{ \delta \kappa_R}{\kappa _R} \approx - \frac{4}{3} \frac{ \delta \rho }{\rho }
\,,
\end{align}
%
which means that the relative mean opacity perturbation is \emph{discordant} in sign with the pressure one, and thus with the temperature one. 

So, up to a positive constant, the integrand looks like 
%
\begin{align}
-\frac{ \delta \kappa _R}{\kappa _R} \pdv{ \delta L}{m}
\,,
\end{align}
%
and we want it to be \(<0\). 
However, if under the perturbation the layer is \emph{more} opaque than usual (\(\delta \kappa _R > 0\)) then it will let through \emph{less} light than usual, so we will have \(\delta (\pdv*{L}{m}) < 0\).
This can be rendered more formal by considering the luminosity equation \eqref{eq:luminosity-equation}: the dependence of \(L\) on \(\kappa_R \) is inverse.

So, the global contribution to \(\kappa \) will be positive, therefore our layer will be \emph{damping}.

\paragraph{\(\kappa \)-mechanism}

However, layers for which the contribution of this term is negative are present in certain regions: where there is ionization, we can have regions for which the gradient \(\pdv*{\kappa_R}{T}\) reverses: it is usually negative, but it can become locally positive.
This provides an additional channel for energy stocking: it is like a \emph{dam} for energy.
The ionization energy is then released through mechanical work.
This is called the \(\kappa \)-mechanism.

\paragraph{\(\gamma \)-mechanism}

The \(\gamma \)-mechanism, on the other hand, involves a decrease of \(\Gamma_3\) which brings it near to 1: recall, 
%
\begin{align}
\Gamma_3 -1 = \eval{\pdv{\log T}{\log \rho }}_{\text{adiabatic}}
\,,
\end{align}
%
so if it is small that means that the change in temperature, i.e.\ internal energy, associated with a compression (\(\delta \rho >0\)) is small: this can happen in the outer layers of the star, which have partial ionization regions. The work of the compression is partially absorbed in order to ionize the atoms in these layers: if \(\Gamma_3 \) is close enough to 1, we can have 
%
\begin{align}
n - s (\Gamma_3 -1) >0 
\,
\end{align}
%
even with \(s > 0\).

\paragraph{\(\kappa \)-\(\gamma \)-mechanism}

We look at the linearized equations of continuity and radiative transfer for an expression for the gradient of \(\delta L\), neglecting the temperature perturbation gradient term for simplicity: then, starting from equation \eqref{eq:linearized-RT} we have 
%
\begin{align}
\frac{ \delta L}{L} =
4 \zeta 
- n \frac{ \delta \rho }{\rho }
+ (s+4) \frac{ \delta T}{T}
\,,
\end{align}
%
while from equation \eqref{eq:linearized-cont} we get
%
\begin{align}
\frac{ \delta \rho }{\rho } = - 3 \zeta 
\,.
\end{align}

Combining these, and using \(\delta T / T = (\Gamma_3 -1 ) \delta \rho / \rho \) we find: 
%
\begin{align}
\frac{ \delta L}{L }= \qty(-\frac{4/3 +n}{\Gamma_3 -1} + s+4) \frac{ \delta T}{T}
\,,
\end{align}
%
and now we make the following consideration: on average, in the outer layers of the stars we have no nuclear energy generation (\(\epsilon _{\text{nuc}} \approx 0\)) and the heat variation is negligible (\(\dv*{Q}{t}\approx 0 \)).\todo{Why?} So, we have \(\pdv*{L}{m} \approx 0\) by the energy conservation equation \eqref{eq:energy-conservation}: using this, we can freely bring \(L\) outside of derivatives with respect to \(m\), in order to write 
%
\begin{align}
\pdv{ \delta L}{m} = L \pdv{}{m} \qty[\qty(s+4- \frac{4/3 +n}{\Gamma_3 -1}) \frac{ \delta T}{T}]
\,.
\end{align}

Outside the regions of partial ionization, typical values for the parameters are \(\Gamma_3 \sim 1.6\), \(s \sim 7/2\), \(n \sim 1\). Therefore, we get 
%
\begin{align}
\pdv{ \delta L}{m} \approx \num{3.61} \times L \pdv{}{m} \qty(\frac{ \delta T}{T})
\,,
\end{align}
%
\todo[inline]{and a function and its derivative have the same sign, right? so we approximate the derivative of \(\pdv*{}{m}\qty( \delta T / T)\) with \(\delta T / T\)}
which means that 
%
\begin{align}
\pdv{ \delta L}{m} \sim 3.6 L \frac{ \delta T}{T}
\,,
\end{align}
%
so they are concordant in sign, so the layer is a damping layer. 


In partial ionization regions, instead, \(\kappa _R \propto \rho^{n}T^{-s}\) with negative \(s\): so it is much easier for the terms to be discordant. Keeping \(n =1\) and \(\Gamma_3 = 1.6  \), we need \(s < \num{-.11}\) in order to have a driving layer. However, \(\Gamma_3\) also decreases in these layers, so the threshold for \(s\) is higher: for example, with \(n=1\) and \(\Gamma_3 = 1.4\) we only need \(s < \num{1.83}\). 

\todo[inline]{So, the \(\kappa \) in the \(\kappa\)-\(\gamma \)-mechanism refers to the decrease in \(s\) while the \(\gamma \) refers to the decrease in \(\Gamma_3\), right?}

\paragraph{Opacity bump mechanism}

The \(\kappa \)-\(\gamma \)-mechanism cannot explain the pulsation of hotter stars than RR-Lyr, Cepheids, \(\delta \)-Scuti or SX-Phe, such as \(\beta\)-Ceph, GW Vir and sdBV stars: in these, the stellar material is much more ionized on average: in the regions which, for cooler stars, are partial ionization regions the gas is fully ionized, while the partial ionization regions have moved further out towards the surface, if they have not disappeared entirely.

One could consider the partial second ionization layers: this has been thoroughly considered, and it has been found that they cannot provide a significant enough contribution.

There is an opacity bump at \(5<\log_{10}(T)< 6.5 \), which was found in the eighties by Simon: the old graph for \(\kappa (T)\) had spikes around \(T \sim \num{e4}\) and \(T \sim \num{e4.7}\), and a new spike was found at \(T \sim \num{e5.4}\). 
Do note that these are temperatures reached when looking a bit inside the star, not at the surface (although close to it).

\paragraph{Convective blocking}
Convection is slow to adapt, so it cannot move away the heat brought by the pulsations. Then, the heat must be turned into work. 
This happens when the period of the pulsations is of the order of the thermal timescale?
\todo[inline]{Not sure about this}

\paragraph{Convective driving}
Convective driving is called the \(\delta \)-mechanism.
It happens when the convective timescale is much shorter than the period of the pulsation, so the convection can mix the gas. 
\todo[inline]{And this drives pulsations?}

\paragraph{Stochastic driving}
There also are \emph{stochastically driven} oscillations: they happen in stars which are intrinsically stable, but the pulsations may be fed by convective turbulence. 
This can happen in Sun-like Main Sequence stars, or in solar-like red giants, such as \(\xi \)-Hydrae stars.

\paragraph{Strange modes}
There are very luminous \emph{strange modes}, very dim convectively driven modes.

\subsubsection{The classical instability strip}

The pulsation region has boundaries:
for the \(\kappa \)-mechanism:
\begin{itemize}
    \item if the star is too hot, the regions of partial ionization get too close to the surface;
    \item if the star is too cold, they are too far in: in their outer region the pulsation is damped by convective heat transfer.
\end{itemize}

So, there is an optimal region for the partial ionization layers to be.

What is the optimal region? We will only look  at the fundamental mode. We define
%
\begin{equation}
  \phi (m) \equiv \frac{1}{L(\Pi / 2 \pi )}\int_m^M c_V T \dd{m} 
\,,
\end{equation}
%
which represents the thermal balance of a single oscillation: \(L(\Pi / 2 \pi )\) is the luminosity radiated in a radian of the pulsation cycle (\(1/ 2 \pi \) cycles), 
while \(c_V T = T \qty(\pdv*{Q}{T}) \approx Q\), the heat per unit mass, which when integrated from \(m\) to \(M\), the whole mass of the star, gives us the global heat variation of the layers above \(m\). 

We are in the helium ionization region: there are no nuclear reactions, so we set energy generation to zero in equation \eqref{eq:linearized-E2}: we get 
%
\begin{align}
\pdv{}{t} \qty(\frac{ \delta T}{T}) =
(\Gamma_3 -1) \pdv{}{t} \qty(\frac{ \delta \rho }{\rho })
- \frac{1}{c_V T} \pdv{ \delta L}{m}
\,.
\end{align}

Now we assume that the temperature, density and luminosity perturbations are oscillating, so they have a certain starting value, and the temporal dependence is absorbed in a factor \(e^{i \sigma t}\). Then, we get 
%
\begin{subequations}
\begin{align}
i \sigma \frac{ \delta T}{T} &= (\Gamma_3 -1) i \sigma \frac{ \delta \rho }{\rho } - \frac{1}{c_V T} \pdv{ \delta L}{m} \\
\frac{ \delta T}{T} &= (\Gamma_3 -1) \frac{ \delta \rho }{\rho } + \frac{i}{\sigma  c_V T} \pdv{ \delta L}{m} 
\,.
\end{align}
\end{subequations}

Now, consider the following: with our definition of \(\phi \), we can write: 
%
\begin{align}
\pdv{}{\phi } = \pdv{m}{\phi } \pdv{}{m}
= - \frac{L(\Pi /2\pi )}{c_V T} \pdv{}{m}
\,;
\end{align}
%
do note that \(\phi \) is adimensional, because of the way we define \(L(\Pi / 2 \pi )\):
%
\begin{align}
L(\Pi / 2\pi ) = L / \sigma 
\,,
\end{align}
%
which is an energy. With these facts in mind, plus the fact that approximately \(\pdv*{L}{m} = 0\), we can do the following manipulation: 
%
\begin{subequations}
\begin{align}
\frac{i}{\sigma c_V T} \pdv{ \delta L}{m}
&= - \frac{i }{\sigma L(\Pi / 2 \pi ) } \pdv{ \delta L }{\phi }  \\
&= -i \pdv{}{\phi } \qty(\frac{ \delta L}{L})
\,.
\end{align}
\end{subequations}

Now, we can use the chain rule on the derivative in \(\partial \phi \) using any auxiliary variable we like: we choose \(x = r/ R\), the fractionary radius, so we get 
%
\begin{equation} \label{eq:temperature-evolution-transition-region}
  \frac{\delta T}{T} = (\Gamma_3 -1 ) \frac{\delta \rho }{\rho } - i \qty(\pdv{\phi }{x} )^{-1} \pdv{}{x} \qty(\frac{\delta L}{L})
\,,
\end{equation}
%
where \(\delta T\) is actually complex, since the perturbations are out of phase.

We can make some considerations on the variations of the terms in the equation: 
%
\begin{align}
\phi \approx \frac{1}{L(\Pi / 2 \pi )}
\frac{4}{3} \pi R^3 \qty(1 - x^{3}) \rho c_V T
\propto (1-x^3) \rho c_V T
\,,
\end{align}
%
so 
%
\begin{align}
\pdv{\phi }{x} \propto - x^2 \rho c_V T
\,.
\end{align}

The \(x^2\) dependence, however, is not very important: if the internal energy of a layer does not change, we can write the first law of thermodynamics as 
%
\begin{align}
\dd{Q} = P \dd{\qty(\frac{1}{\rho })} 
\implies 
P \sim \dv{Q}{ \qty(\rho^{-1})} \sim Q \times \rho 
\,,
\end{align}
%
and \(c_V T \sim Q\), so we have: 
%
\begin{align}
\pdv{\phi }{x} \propto - x^2 P
\,,
\end{align}
%
and while \(x\) goes to zero in the center, the pressure increases a lot. 

\begin{bluebox}
From what I could gather in 5 minutes, the Sun's pressure profile is very roughly given by 
%
\begin{align}
P(x) \sim \SI{2.5e11}{Pa} \exp( - \lambda x)
\,,
\end{align}
%
where \(\lambda \sim 9\). Now, if we plot \(\exp(-9x) x^2\), it becomes small both at the surface and at the center, and has a maximum around \(x \sim \num{.2}\).
So, the reasoning still makes sense, but you have to consider the very center of the star as a special case.
\end{bluebox}

On the surface, the imaginary term is negligible. In the interior, it is relevant.

So, near the surface \(\pdv*{x}{\phi }\) is very large: therefore the term multiplying it, \(\pdv*{}{x} \qty(\delta L / L)\), must become small in order for the temperature perturbation not to diverge. This means that the luminosity perturbation cannot change much: it is ``frozen in''. 
Anyway, the imaginary part of the RHS of equation \eqref{eq:temperature-evolution-transition-region} is much larger than the real part.
\todo[inline]{Right?}

On the other hand, near the center the term \(\pdv*{x}{\phi }\) becomes very small, while the spatial derivative of the perturbation does not become very large, so on balance the term remains small: the oscillations are then quasi-adiabatic.

Between these two regimes, we have a \emph{transition region} in which the two terms in equation \eqref{eq:temperature-evolution-transition-region} are similar in terms of order of magnitude: depending on the temperature of the star, this can can occur before or after the ionization region, when going radially outward.

The classical instability strip is the region in the Hertzsprung-Russell diagram in which the partial ionization region roughly coincides with the transition region. 

\todo[inline]{Why is the perturbation ``frozen in'' if its spatial derivative is small? we could have \(\delta L / L\)  constant with respect to \(x\) and it would still evolve in time!}

\todo[inline]{Why are the boundaries of the classical instability strip tilted? It seems like the region is defined by some relation like \(\log L = - (\text{large \#}) \log T\), why is this so?}

\subsection{Star botany}

\subsubsection{RR Lyr\ae}

These are stars with periods between \(\num{.2} \divisionsymbol \SI{1}{d}\), absolute visual magnitude around \num{+.6}, and mean effective temperature around \( \num{6000} \divisionsymbol\SI{7250}{K}\).

It is a stage, which lasts no more than \SI{e8}{yr}, and it is observed only in clusters older than \SI{e10}{yr}.

They are classified by a, b, or c according to the shape of the light curve, its amplitude, its period:

\begin{enumerate}
  \item RRa: sharp rise, large amplitude: the fundamental;
  \item RRb: similar to RRa with smaller amplitude, longer period: the fundamental;
  \item RRc: more symmetric light curve, short periods, low amplitudes: they pulsate in the first overtone.
\end{enumerate}

We have also RRd: bimodal, RRe: second overtone.

These characteristics are seen well in a Bailey diagram, in which we scatter plot the stars with magnitude on the \(y\) axis and period on the \(x\) axis.

[Argument for the different amplitudes at different wavelengths: to understand]

We can do spectral analysis on a set of several different stars: we find that, when applying a spectral filter to the star and doing Fourier analysis on that light curve, we can identify a relation in the form 
%
\begin{align}
M_\nu = k_\nu \log \Pi 
\,,
\end{align}
%
where \(M_{\nu }\) is the magnitude measured in that particular spectral band, while \(k_{\nu }\) is a constant depending on the band. \(k_{\nu }\) can be either positive or negative, and it is increasing with \(\nu \). 

[Qualitative part of the lecture.]

\end{document}
