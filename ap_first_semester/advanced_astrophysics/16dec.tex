\documentclass[main.tex]{subfiles}
\begin{document}

\section*{Mon Dec 16 2019}

\subsubsection{The main uncertainties in SN theory}

There are still many uncertain areas, both in the presupernova phase and about the explosion:

\paragraph{Presupernova phase}

The mass loss of stars in their Red and Blue Supergiant phases is unclear.

Convection is not well understood, especially when it is heavily coupled to nuclear burning. 

The cross section of the reaction \(\ce{^{12}C} (\alpha , \gamma ) \ce{^{16}O}\)\footnote{This is a shorthand notation used in nuclear physics: \(a(b,c)d\) means \(a + b \rightarrow c+d\).} is unclear.

The effects of rotation are not well understood: it can cause convective overshooting, which means that bits of a convective layer have enough momentum to be ``shot into'' another layer.

\paragraph{Explosion}

In a binary system things are more complicated. 

We do not have a self-consistent hydrodinamical model which includes neutrino transport.

% It seems that stars more massive than \(35 M_{\odot}\) have large fallback: they do not contribute to the heavy element population. 

% We do not have a self-consistent hydrodinamical model which includes neutrino transport. 

\subsubsection{Impact of the chemistry of a WR star on the fallback}

The models from Langer in 1989 differ from those of Nugis and Lamers (NL00): Langer predicts a higher mass loss rate, which as was discussed in the treatment of WR stars entails a higher \ce{^{12}C} fraction and a lower overall mass of the CO core.

So, the core is less compact and has a lower binding energy in the models by Langer: the binding energy of the core does \emph{not} increase monotonically with the ZAMS mass of the star, it has a peak of around \(B \sim \SI{6}{foe}\) around \(M _{\text{ZAMS}} \sim 30M_{\odot}\) and then slightly decreases, while in the NL00 models \(B\) increases steadily up to \SI{20}{foe} and more as \(M _{\text{ZAMS}}\) increases.

So, the nuclei in the models by Langer are less compact, therefore we predict less efficient fallback and formation of Neutron Stars as opposed to Black Holes.

These high-ZAMS-mass supernovae match well the observed type Ib/c supernovae.

As it turns out, both the models by Langer (LA89) and by Nugis-Lamers (NL00) are useful: LA89 describes stars with heavy mass loss in the pre-SN stages, while NL00 describe stars with less mass loss in the pre-SN stages, which are more likely to collapse into BHs for high initial mass.

LA89-type massive \(M > 40 M_{\odot}\) stars generally have higher yields of heavy elements than NL00-type stars.

% We can plot the mass of the various mass components of the star with respect to \(\log \qty(t _{\text{collapse}} - t)\). 

% We can also plot the mass fractions of \ce{^{12}C} and \ce{^{16}O} with respect to the fraction of \(X(\ce{^{4}He})\), which decreases with time.

% At some point the helium burning phase stops, and then we are left with a carbon/oxygen fraction which can be either higher or lower than one. 

% Depending on whether we are using a LA89 or an LN00 model we can have different behaviours. 

% The final kinetic energy after the fallback is a ``foe'': \SI{e51}{\erg}. 

% If we have a higher mass-loss rate then we will have a lower mass of the CO core, which will then be less compact. The fallback will be less efficient.

% The state of the art is that at higher initial masses BHs are more favoured. 
% It seems that BHs are more often produced by WR stars.

% According to a certain model, BHs are mainly produced by \emph{failed} supernovae: there is a discontinuity around \(M \sim 35 M_{\odot}\), above which we get a collapse and a failed supernova. 

As the metallicity increases, the mass range resulting in failed supernovae (so, the resulting BHs) decreases. 

\section{Very massive stars}

\subsection{Generalities}

Now, we discuss the evolution and final fate of very massive stars. 
They have masses in the range \(100 M_{\odot} < M _{\text{in}} < \num{5e4}M_{\odot}\). 
For these, we do not have a core collapse, but a thermonuclear explosion instead. 

There is some evidence for the existence of these stars, for example in the cluster R136: the star R136a1 has an estimated mass of \(315M_{\odot}\), the highest known of any star. 

The cluster R136 seems to be very young, with an age (\(\lesssim \SI{2}{Myr}\)). 

In general, the fate of a (not necessarily massive) star is determined by its initial mass, chemical composition, and rotation rate.
For very massive stars, the most relevant quantity is the mass of the core of Helium and other heavy elements, commonly denoted as ``the Helium core'', whose mass is determined by the mass loss rate throughout the star's evolution.

\paragraph{Astrophysical relevance of PCSNe}

In 2007 a Super Luminous supernova (SLSN) was observed: SLSNe can have luminosities brighter by a factor of 10 than the regular supernovae, reaching an absolute magnitude of more than \SI{-21}{mag} (the Sun's is \SI{4}{mag}!). 

Spectral analyses show that in the explosions of these stars we have a hefty production of Nickel.
Beyond the increased luminosity, the light curve of these SLSNe also decreases slower than that of regular Core Collapse Supernovae --- CCSNe.

\paragraph{Metallicity dependence}

We will need to understand under which metallicity regime we can have these SLSNe: there seems to be a metallicity threshold, since PCSNe evolve from very massive stars, as long as they avoid heavy mass loss. 
We have \(\dot{M} \propto Z\), so if the metallicity is too high the star loses too much mass to produce a PCSN. The threshold is denoted as \(Z _{\text{PCSN}}\).
It seems like the threshold is around \(Z _{\text{PCSN}}\sim \num{.006} = Z_{\odot} / 2\).

Most of the observed PCSN are in regions with positive metallicity, for example SN 2007bi exploded in a galaxy with \(Z = Z_{\odot} /3\).

These might be the first contributors to the presence of metal in the universe,\footnote{The Black Sabbath of the cosmos}, but measured abundances of elements in the early universe do not seem to match the simulated yields of PCSNe.

\subsection{The engine of PCSNe}

% These are ``pair creation supernovae'' because the radiation they emit is more energetic than \SI{1}{MeV}? 

Pair Creation SNe occur when the temperature of the core is such that a large fraction of the energy goes into producin electron-positron pairs. This is called the Pair Instability.

This significantly decreases the pressure of the core, causing a contraction, followed by a thermonuclear explosion: formally, the pair production makes \(\gamma _{\text{ad}} < 4/3\), which causes a gravitational collapse.

To see why this is the case, one can look at the LAWE: \eqref{eq:LAWE}, if \(\Gamma_1 = \gamma_{\text{ad}}\) is \(< 4/3\) the differential equation for the radial perturbation \(\zeta \) becomes a harmonic repulsor.

% The relevant phase in stellar evolution is when a large amount of thermal energy goes into pair production instead of producing pressure. 

An important parameter in this process is the mass of the helium core. 

We'd expect more and more massive stars to lose more and more mass loss through winds. 

In order to have pair instability, we need a Helium core with mass such that \(40 M_{\odot} < M _{\text{He}} < 133 M_{\odot}\).

If \(M _{\text{He}} < 65 M_{\odot}\) then we have Pulsation pair instability supernovae, above this we have Pair creation supernovae. 

The star is able to eject most of the material in a series of pulses even if the nuclear energy is smaller than the binding energy: these are \emph{pulsation} PISNe, while if we have \(E _{\text{nuc}} > E _{\text{bind}}\) right away then it is a regular PCSN.

The pulses are usually separated by time intervals on the order of \SI{e5}{s} to \SI{e11}{s}.

The pair creation triggers the collapse. Then, because of the heating induced by the released gravitational energy, we have a runaway thermonuclear reaction. If this exceeds the binding energy of the star, then we have complete disintegration. 

% The threshold for this turns out to be give by around \(65 M_{\odot}\). 

For \(M _{\text{He}} > 133 M_{\odot}\) the infall produces a BH.

% This occurs before oxygen burning. 

\paragraph{Radiation pressure}

Very massive stars are dominated by radiation pressure: 
we have the relation 
%
\begin{align}
\beta^{1/2} (1+\beta )^{3/2} \propto M 
\,,
\end{align}
%
where \(\beta = P _{\text{rad}} / P _{\text{tot}}\).
This relation for \(\beta \) looks complicated but if we plot it it is roughly linear.
\todo[inline]{Instead of \(P _{\text{rad}}\) in the slides we have \(P _{\text{gas}}\), but from the context this is probabably what was meant.}

Radiation pressure scales like \(P _{\text{rad}} = a T^{4}/3\), where \(a =4 \sigma / c \) is the radiation constant.

The core's temperature \(T_c\) scales like 
%
\begin{align}
T_c \propto M^{k} \rho_{c}^{1/3}
\,,
\end{align}
%
where \(\rho_{c}\) is the core density, \(M\) is the total star mass and \(k\) is a parameter depending on the equation of state: for an ideal gas it is \(2/3\), if we only have radiation pressure it is \(1/6\).

Pair instability occurs in the Oxygen burning phase, when the temperature increases beyond \SI{e9}{K}. 

The energy for a photon to decay into an electron-positron pair is \(\SI{1}{MeV} \sim \SI{e10}{K}\), so when the temperature reaches values of the order \SI{e9}{K} the \(>\SI{1}{MeV}\) tail of the Planckian becomes significant.

At \(T = \SI{e9}{K}\) the tail accounts for just \SI{.2}{\percent} of the flux, but already at \(T = \SI{2e9}{K}\) it is already almost \SI{15}{\percent} of the flux, and at \(T = \SI{3e9}{K}\) it is \SI{41}{\percent} of the flux.

The free photons and the \(e^{+} - e^{-}\) pairs reach equilibrium, decreasing the pressure (and so, the adiabatic index).

If this happens in a large enough region it can trigger an instability.

\end{document}