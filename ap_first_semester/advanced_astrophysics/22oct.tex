\documentclass[main.tex]{subfiles}
\begin{document}

\section*{Tue Oct 22 2019}

For the physics undergraduate students: ``Fundamental Astronomy'' by Karttunen, Kröger, Öja, Poutanen, Donner.

\subsection{Red variable stars}

They include the \(\xi \) Hydrae, SR and Mira stars.
These are low temperature stars.

They are evolved stars, and most all of the evolved stars are at least somewhat variable.

Miras, SRVs and OSARGs have very long periods, on the order of a year.

We need good estimates for the radius: because of the period-mean density relation, the fundamental at a certain radius can correspond to the first overtone at a larger radius.

We have different period-luminosity relations corresponding to which overtone we see.
Using Weisenheit indices, which compensate for self-reddening, these relations are even more evident.

We can simulate galaxies, and observe the same patterns.

The linear models are not appropriate for the fundamental.
Today, we do 3D models.

\section{Summary}

\begin{itemize}
    \item \(\tau _{\text{dyn}} \propto 1/\sqrt{\overline{\rho } } \);
    \item the inequality between the timescales;
    \item variability is due to mechanical, acoustic phenomena!
    \item derivation of the period-mean density relation;
    \item the balance of heat absorption;
    \item meaning of perturbation theory;
    \item the final result: the perturbed structure equations;
    \item the adiabatic approximation: justification, cases in which it does not hold: driving layers, stability;
    \item ideas of how the LAWE is solved (not boundary conditions): the Sturm-Liouville problem;
    \item the meaning of the solutions of the LAWE: nodes, the shape of the eigenfunctions, orthogonality;
    \item stability conditions: inequalities which tell us whether a star pulsates or not, driving and damping layers, phase lag;
    \item NOT the derivation of the LNAWE, but qualitative characterization of its solutions; the expression of the coefficient \(\kappa \);
    \item the \(\epsilon \) and \(\kappa \)-\(\gamma \) mechanisms: orders of magnitude;
    \item the opacity bump mechanism;
    \item the difference between self-excited and stochastically driven oscillation;
    \item the classical instability strip: not important to know the derivation;
    \item RR Lyrae, typical parameters;
    \item Cepheids: derivation of the mass-luminosity relation;
    \item NOT nonradial oscillations and astroseismiology;
    \item NOT red variables.
\end{itemize}


\end{document}