\documentclass[main.tex]{subfiles}
\begin{document}

\section*{Thu Dec 12 2019}

\todo[inline]{Check room availability for the first week of January.}

Tomorrow / next week we will discuss some black holes and neutron stars. 

Last week we discussed the nuclear processes which occur in the center of the Sun when \(T \sim \SI{e7}{K}\) is reached. 

The minimum mass for a star in order to ignite fusion, as we found, is around \(\num{.08} M_{\odot}\). 

The most important difference between the processes outlined last week is in the first equation of each: there are very few free neutrons.

So, the weak interaction process dominates: since it is so slow, it has a very low power density: \(P = \SI{4e26}{W}\), but we need to take its ratio to the volume of the Sun. This gives a density lower than that of a human. 

For each \ce{^{4}He} nucleus we get \SI{26}{MeV}, and we need 4 protons to make it. Then, this gives us the number of protons per second the Sun uses in order to produce the power it does. 
We will use the following relation: 
%
\begin{align}
  \SI{1}{MeV} \approx \SI{1.78e-30}{kg} \approx \SI{1.6e-13}{J}
\,,
\end{align}
%
and then, in SI units: we get 
%
\begin{align}
  \frac{\num{4e26}}{\num{2.6e-13} / 4} \approx \num{4e38} \frac{\text{protons}}{\SI{}{s}}
\,.
\end{align}

For each process we also emit one electron neutrino, and we need to do the first two steps of the process twice for each \ce{^{4}He} nucleus, so we are producing \num{2e38} neutrinos per second. 

A proton's mass is around \(m_p \approx \SI{1}{GeV} \approx \SI{1.78e-27}{kg}\), so in the Sun there are around \num{e56} of them: this means that the typical lifetime of the Sun is around \num{e10} years. 

What is the final state of the Sun? After hydrogen runs out, the Sun should start the next process: helium burning. The core contracts and heats. However we also have the degeneracy pressure. 

There is a boundary, the Chandrasekar mass \(M_C \approx \num{1.4} M_{\odot}\), between the final fate of the star being a white dwarf or a neutron star. 

From the point of view of the state of matter, brown dwarves and white dwarves are very similar. 

So the core becomes denser and hotter and starts burning helium. The external parts, by the virial theorem, must then expand. 

So the possibilities, in order of mass, are white dwarf, neutron star, black hole. 

The matter which is expelled can form a \emph{planetary nebula}. 


\begin{figure}[ht]
    \centering
    \begin{tabular}{ccccc}
     Process & Fuel & Products & \(T_{\text{min}}\) & \(M _{\text{min}}\) \\
     \hline
     Hydrogen burning & Hydrogen & Helium & \SI{e7}{K} & \(\num{.08} M_{\odot}\) \\
     Helium burning & Helium & Carbon, Oxygen & \SI{e8}{K} & \(\num{.5} M_{\odot}\) \\
     Carbon burning & Carbon & Oxygen, Neon, Sodium & \SI{5e8}{K} & \(8 M_{\odot}\) \\
     Neon burning & Neon & Magnesium, Oxygen & \SI{e9}{K} & \(9 M_{\odot}\) \\
     Oxygen burning & Oxygen & Magnesium to Sulphur & \SI{2e9}{K} & \(10 M_{\odot}\) \\
     Silicon burning & Silicon  & Iron and nearby elements & \SI{3e9}{K} & \(11 M_{\odot}\) 
    \end{tabular}
    \caption{Solar processes.}
    \label{tab:solar-processes}
\end{figure}    

A plot of the binding energy per nucleon \(B = E - M _{\text{nucleons}}\) shows that it is maximum for iron. 

Something about the possibility to have \ce{^{8}Be} be stable in order for potassium to be formed. 

When the nucleus cannot reach the temperature needed for the next process, the collapse is stopped by electron degeneracy. 

The final radius of the red giant phase of the Sun is around \(70\) times the radius of the Sun, while the white dwarf phase is \(70\) times smaller. \todo[inline]{Is this correct? I couldn't quite hear.}

Our equation of hydrostatic equilibrium is:
%
\begin{align}
  \dv{P}{r} = - \frac{G m(r) \rho (r)}{r^2}
\,,
\end{align}
%
and the equation giving the variation of the mass is \(\dv{m}{r} = 4 \pi r^2 \rho (r)\).
So we get 
%
\begin{align}
  \frac{r^2}{\rho (r)} \dv{P}{r} = - G m(r)
\,,
\end{align}
%
which can be restated as 
%
\begin{align}
  \dv{}{r} \qty(\frac{r^2}{\rho (r)} \dv{P}{r}) = - G \dv{m}{r} = - 4 \pi G \rho (r) r^2
\,,
\end{align}
%
more commonly stated as 
%
\begin{align}
  \frac{1}{r^2} \dv{}{r} \qty(\frac{r^2}{\rho (r)} \dv{P}{r}) = -4 \pi G \rho (r)
\,,
\end{align}
%
which holds if we have hydrostatic equilibrium. 

Commonly the equation of state used for this is called \emph{polytropic}: 
%
\begin{align}
  P = k \rho^{\frac{n+1}{n}}
\,,
\end{align}
%
where \(k = \const\) and \(n = 1/ (\gamma -1)\): so if \(\gamma = 5/3\), which holds for a monoatomic gas, then \(n = 3/2\) while if \(\gamma = 4/3\), which holds for an ultrarelativistic gas, then \(n = 3\). 

\end{document}