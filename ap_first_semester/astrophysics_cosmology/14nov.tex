\documentclass[main.tex]{subfiles}
\begin{document}

\marginpar{Thursday \\ 2019-11-14, \\ compiled \\ \today}
% \section*{Thu Nov 14 2019}

% During inflation, the comoving Hubble radius (\(r_H = 1/\dot{a}\)) decreases.
Let us consider a region of radius approximately \(1 / H(t_b)\), where \(t_b\) is a certain moment before the start of the inflationary phase.
This region is expanded by many \(e\)-foldings through inflation, such that any inhomogeneities are smoothed out: this is the \textbf{cosmic no-hair theorem} \cite[pag.\ 159]{LucchinColes:2002}.

% \(t_i\) is the beginning time of inflation, \(t_f\) is its end, \(t_\Lambda \) is the time when the cosmological constant became dominant, and then we get to now: \(t_0 \).

% What happened before inflation?

% The cosmic no-hair conjecture is what allows inflation to delete inhomogeneties.
So, there might have been perturbations before inflation: we cannot know.
Perturbations on scales larger than the cosmological horizon are not perceivable as perturbations: we only perceive our local mean value.

% Below the largest inflation scale, the perturbations are erased by inflation: we see perturbations on these scales which are produced during inflation.

\paragraph{Reheating}

The energy density of radiation through the inflationary period scales as \(\rho _r \propto a^{-4} \propto e^{-4 Ht}\), while the one of matter instead it scales as \(\rho _m \propto a^{-3} \propto e^{-3Ht}\) since \(a \propto e^{Ht}\).
Qualitatively speaking, as the universe rapidly expands it becomes basically \emph{devoid of particles}, and its temperature drops substantially.

At the end of inflation, \emph{reheating} takes place, substantially raising the temperature and allowing for the formation of most of the SM particles we observe today.
This is due to the latent energy released by our scalar field due to its coupling to the rest of the universe, which acts as a sort of viscous force.
Intuitively speaking, the field ``falls'' into its ground state and then oscillates ever slower, dissipating its energy in the process \cite[fig.\ 7.6]{LucchinColes:2002}. 

% So we get that this latent energy \(\Delta V\) is of the order of \(T^{4}_{\text{rad}}\).
% \todo[inline]{This seems like a rather technical point to end on\dots Is there some larger meaning to be drawn from these last paragraphs?}

\section{Baryogenesis}

This is the process which led to the formation of baryons. 

% What is the typical temperature needed to produce baryon symmetry? 
% \todo[inline]{What is baryon symmetry?}
The main issue we seek to address is that of \textbf{baryon asymmetry}. In the Standard Model each fermion has a corresponding antifermion, and the same holds for the composite particles of quarks, such as mesons and baryons. 

We usually talk of baryons only; they are characterized by the conserved baryon number \(B\), which is the number of baryons minus the number of antibaryons.\footnote{This number can actually be computed from the number of quarks already, so it makes sense to discuss it even before the formation of proper baryons.}

Matter and antimatter annihilate if they meet, producing radiation (\(\gamma \) photons and/or other high-energy particles). 
This allows us to investigate the presence of antimatter in the universe: if there were patches of antimatter as well as patches of matter their boundaries would be regions of great \(\gamma \)-ray emission, which we would be able to detect as a background.
Observations then allow us to rule out the presence of antimatter regions large enough to have \(B = 0\) in the observable universe --- small patches of antimatter may exist, but there are not enough of them to balance out all the matter \cite[]{cohenMatterAntimatterUniverse1998}. 

This constitutes the problem of baryon asymmetry: it seems unnatural for the universe to start off with more baryons than antibaryons, all the known Standard Model interactions conserve \(B\), yet we observe more baryons than antibaryons.
% These questions are hard to answer without a grand unification theory, which.

% How much antimatter is there in the universe?
% A long time ago, it was thought that there might have been regions in the universe which were filled with antimatter by looking for a \(\gamma \)-ray background, but this was not found.
% We did not find them.

In particle physics the absolute number \(B\) is used, but in cosmology absolute numbers as usual are not convenient, so we work in terms of densities. All number densities scale like \(a^{-3}\), so we need to normalize the difference \(n_b - n_a\) with another number density: we define: \(\eta/2 = (n_b - n_a) / (n_b + n_a)\). Here \(b\) means baryons, while \(a\) means antibaryons. 
% This is actually computed with quark numbers, so it can be computed even when protons and neutrons have not yet formed.

The reason for the factor 2 is given by how we can actually estimate this parameter: in the early universe, when the baryons were still relativistic, processes like \(b + \overline{b} \leftrightarrow 2 \gamma \) occurred at thermal equilibrium, so roughly speaking we should have had \(n_b \approx n_a \approx 2 n_\gamma \).

\todo[inline]{What would be the proper chemical-equilibrium considerations?}

Now, all of these densities scaled like \(a^{-3}\) from the moment the baryons became nonrelativistic to now. So, we have 
%
\begin{align}
2 \frac{n_b - n_a}{\underbrace{n_b + n_a}_{2 n_\gamma }} \approx \frac{n_{0b} - \overbrace{n_{0a}}^{\approx 0}}{n_{0\gamma}} \approx \frac{n_{0b}}{n_{0 \gamma }}  = \eta_0 
\,.
\end{align}
 
This allows us to estimate the asymmetry constant \(\eta \) with parameters measured today: the baryon number density and the photon number density. 

% It seems like the only antimatter known is the one which was formed by us, or cosmic rays.

For the baryon number density we can estimate \cprotect\footnote{The result comes about from the following code: \begin{lstlisting}[language=Python]
from astropy.cosmology import Planck15 as cosmo
import numpy as np
import astropy.units as u
from astropy.constants import codata2018 as ac
H0 = u.littleh *100 * u.km/u.s / u.Mpc
rhoC = 3 * H0**2 / (8 * np.pi * ac.G)
(rhoC * cosmo.Ob0 / ac.m_p).to(u.cm**-3 * u.littleh**2)
\end{lstlisting}}
%
\begin{align}
n_{0b} \approx \frac{\rho_{0b}}{m_p} = \frac{\Omega_{0b} \rho_C}{m_p} \approx \SI{5.46e-7}{cm^{-3} \littleh^2} \marginnote{\(\Omega_{0b}\approx 0.0486\).}
\,,
\end{align}
%
while for the photons we can find the number density by integrating the CMB Planckian: the result is \cprotect\footnote{The result comes about from the following code: \begin{lstlisting}[language=Python]
from astropy.cosmology import Planck15 as cosmo
import numpy as np
import astropy.units as u
from astropy.constants import codata2018 as ac
z3 = np.sum([n**-3 for n in range(1, 1000000)])
(2 * z3 / np.pi**2 * cosmo.Tcmb0**3 * (ac.k_B / ac.hbar / ac.c)**3).cgs
\end{lstlisting}}
%
\begin{align}
n_{0 \gamma } \approx \frac{2 \zeta (3)}{\pi^2} T_{0 \gamma }^3 \qty(\frac{k_B }{\hbar c})^3 \approx \SI{410}{cm^{-3}}
\,.
\end{align}

Combining these results yields 
%
\begin{align}
\eta_0 \approx \num{3e-8} \Omega_{0b} h^2 \approx \num{1.3e-9} h^2 \approx \num{6.1e-10}
\,.
\end{align}

How do we interpret this result? It means that in the radiation-dominated epoch the asymmetry between baryons and antibaryons was very slight, about one part in a billion; most of the pairs annihilated but a bit of matter was leftover, and that is the matter we currently have. 

% The value \(\Omega_{0b} \approx 0.04\) is defined as 
% %
% \begin{align}
%   \rho_{0b} = \Omega_{0b} \rho _{c} 
% \,,
% \end{align}
% %
% where \(\rho_c = 3 H_0^2 / (8 \pi G)\).

% When we define the number of protons in the universe we also fix the number of neutrons, since the universe is globally neutral.
% So we can estimate: 
% %
% \begin{align}
%   n_{0b} = \frac{\rho_{0b}}{m_p} \approx \SI{1.12e-5}{cm^{-3}} \times h^2 \Omega_{0b}
% \,,
% \end{align}
% %
% where \(h \sim 0.7\),
% and similarly we can compute 
% %
% \begin{align}
%   n_{0 \gamma } \approx \SI{420}{cm^{-3}}
% \,,
% \end{align}
% %
% so we can 
% look at the baryon to photon number ratio: 
% %
% \begin{align}
%   \eta_0  = \frac{n_{0b}}{n_{0\gamma}}
%   \approx \num{3e-8} \Omega_{0b} h^2
% \,,
% \end{align}
% %
% why does this number have this value? 

% The denominator in \((n_b - n_a) / (n_b + n_a)\) is approximately \(2 n_{\gamma }\) then, and both the difference in the numerator and the numerator scale like \(a^{-3}\), so this value is a constant.
% We do not see antimatter, therefore \(n_a = 0\). So, we get 
%
% \begin{align}
%     \frac{n_b - n_a}{n_b + n_a} \approx \frac{n_b}{2 n_\gamma } = \frac{1}{2} \eta_0 
% \,.
% \end{align}
%
% This must then have been the case also in the matter dominated epoch, during which there was a very slight imbalance in baryons vs antibaryons.

In order to generate a baryon-antibaryon asymmetry we need \begin{enumerate}
    \item ?
    \item \(C\) and \(CP\) violation (while \(CPT\) symmetry must hold for any well-behaved QFT);
    \item out-of-equilibrium processes.
\end{enumerate}

These were proposed by a famous Soviet scientist.

We define the interaction rate \(\Gamma \): it is the number of interactions per unit time.

The Hubble rate \(H = \dot{a} / a \) is also an inverse time: baryons and antibaryons are practically speaking \emph{decoupled} if \(\Gamma \leq H\): this is equivalent to saying that the time of interaction is larger than the age of the universe.

When are particles actually coupled or decoupled?
\(\Gamma \) can be calculated as \(\Gamma = n \expval{\sigma v}\), where \(n\) is the number density, \(v\) is the velocity of the particles, and \(\sigma \) is the cross section of the interaction.

We need to distinguish the types of interactions we are dealing with.
In general interactions are carried by gauge bosons. 
Either they are massless (like the photon) or they are massive (like the the \(W^{\pm}\) and \(Z\) bosons, as long we are below the scale of electroweak symmetry breaking \(\sim \SI{e2}{GeV}\)).

At larger energies than those, the weak interaction also becomes long-range.

In the massless case, the cross-section is \(\sigma \sim \alpha^2 / T^2\), where \(g = \sqrt{4 \pi \alpha }\).

In the massive case, for temperatures \(T \leq m_{x}\), the cross section is of the order \(\sigma \sim G_x^2 T^2\).

There is an inversion in the \(T\) dependence between long and short range interactions.

Typically \(G_x = \alpha / m_x^2\).  Then, \(\sigma \sim \alpha^2 T^2 / m_x^{4}\).

This difference is because in general the formula is like 
%
\begin{align}
  \sigma \sim \frac{T^{2}}{E^{4}}
\,,
\end{align}
%
and we either have \(E \sim m\) in the low-speed case, or \(E \sim T\) in the high-speed case.

So, we know that \(\Gamma = n \expval{\sigma v}\). In the massless case this is something like \(\Gamma \sim T^3 \sigma \sim T^{3} \sim \alpha^2 T\) since \(T \sim 1/ a\).

Then, we get 
%
\begin{align}
  H = \sqrt{\frac{8 \pi G }{3}} g_{*}^{1/2} \qty(\frac{\pi^2}{30})^{1/2} T^2 \sim \frac{T^2}{m _{\text{pl}}}
\,,
\end{align}
%
since \(G \sim 1/m _{\text{pl}}^2\). Then, 
%
\begin{align}
  \frac{\Gamma}{H} \sim \frac{\alpha^2 T m _{\text{pl}}}{T^2} \sim \alpha^2 m _{\text{pl}} \frac{1}{T}
\,,
\end{align}
%
so we have decoupling when \(T > \alpha^2 m _{\text{pl}}\).
Essentially, at temperatures larger than \(T = \SI{e16}{GeV}\) this massless photon is decoupled.

``Above the Planck epoch, even gravitational interactions are decoupled''.

What about the massive interactions? We have \(\Gamma \sim T^2 G_x^2 T^2\), to compare with \(H \sim T^2/ m _{\text{pl}}\): we get 
%
\begin{align}
  \frac{\Gamma}{H} \sim \frac{T^3 G_x^2}{T^2 / m _{\text{pl}}} \sim G_x^2 m _{\text{pl}} T^3 \leq 1
\,,
\end{align}
%
equivalently, \(T < m _{\text{pl}}^{-1/3} G_x^{-2/3}\).

Suppose that we are considering the gravitational interaction: in that case, we get \(T < m _{\text{pl}}\) since \(G_x\) is related to \( G_N\) 

\todo[inline]{What is the relation?}

For the weak interaction, we have 
%
\begin{align}
    T < \qty(\frac{m_x}{\SI{100}{GeV}})^{4/3} \SI{}{MeV}
\,,
\end{align}
%
which is why below \(\SI{1}{MeV}\) neutrinos are decoupled.

Let us now consider the consequences of these decoupling conditions.
First of all we look at the recombination of hydrogen.
At very high temperatures, there are free electrons and free protons.
Protons first appeared in the universe as non-relativistic, at \(T \sim \SI{1}{MeV}\) while \(m_p \sim \SI{1}{GeV}\).

At a certain point, it becomes possible to create neutral H atoms from these free particles.
We will use a special case of the Boltzmann formula, which governs this and many other phenomena: the Saha equation.

The reaction is \(e + p \leftrightarrow H + \gamma \). We want to look at a density in phase space.
We'd need all the scattering matrices, and all the phase space densities of the particles.
The Saha equation is basically an ansatz at thermal and chemical equilibrium: \(\mu _e + \mu_p = \mu_H + \mu_\gamma \). The chemical potential \(\mu \) entered in the exponent of the FD and BE expressions.

We know that \(\mu_{\gamma }=0\).
At thermal equilibrium the number density of the electrons is: 
%
\begin{align}
  n_e = g_e \qty(\frac{m_e T}{2 \pi })^{3/2} \exp(\frac{\mu _e - m_e }{T}) 
\,,
\end{align}
%
and an exactly analogous formula holds for \(n_p\) and \(n_H\): for protons we have 
%
\begin{align}
    n_p = g_p \qty(\frac{m_p T}{2 \pi })^{3/2} \exp(\frac{\mu _p - m_p }{T}) 
  \,.
\end{align}
Degeneracy is not an issue, since we are talking about cosmology.

The number density of photons is given by 
%
\begin{align} \label{eq:n-gamma}
  n_{\gamma } = \frac{2 \zeta (3) T^3}{\pi^2}
\,,
\end{align}
%

The binding energy of the hydrogen is \(B=m_p+m_e-m_H=\SI{13.6}{eV}\). Instead of \(m_H\) we write \(+m_p+m_e-B\). So the number density of hydrogen atoms is given by: 
%
\begin{align}
  n_H = g_H \qty(\frac{m_H T}{2 \pi })^{3/2} \exp(\frac{\mu_p - \mu_e - m_p - m_e + B}{T})
\,,
\end{align}
%
and we can simplify things since \(m_p, m_H \gg m_e, B\), and substitute in the number densities for electrons and protons.
We get approximately
%
\begin{align}
  n_H = \frac{g_H}{g_e g_p} n_e n_p \qty(\frac{m_e T}{2 \pi })^{-3/2}\exp(B/T)
\,,
\end{align}
%
but the universe is locally and globally neutral: \(n_e = n_p\), and \(n_b = n_p + n_H\). 

We will see that most of the hydrogen we produce will not be when the temperature is of the order of the binding energy, but much later.

\end{document}
