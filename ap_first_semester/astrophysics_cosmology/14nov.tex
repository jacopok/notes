\documentclass[main.tex]{subfiles}
\begin{document}

\section*{Thu Nov 14 2019}

During inflation, the comoving Hubble radius (\(r_H = 1/\dot{a}\)) decreases.

\(t_i\) is the beginning time of inflation, \(t_f\) is its end, \(t_\Lambda \) is the time when the cosmological constant became dominant, and then we get to now: \(t_0 \).

What happened before inflation?

The cosmic no-hair conjecture is what allows inflation to delete inhomogeneties.
So, there might have been perturbations before inflation: we cannot know.
Up to which scale? Perturbations on scales larger than the cosmological horizon are not perceivable as perturbations: we only perceive our local mean value.

Below the largest inflation scale, the perturbations are erased by inflation: we see perturbations on these scales which are produced during inflation.

The energy density of radiation scales as \(\rho _r \propto a^{-4}\), while the one of matter instead it scales as \(\rho _m \propto a^{-3} \propto e^{-3Ht}\) since \(a \propto e^{Ht}\).

The maximum temperature of radiation after inflation is given by the one which corresponds to the maximum latent energy released by our scalar field due to its coupling to the rest of the universe, which acts as a sort of viscous force.

So we get that this latent energy \(\Delta V\) is of the order of \(T^{4}_{\text{rad}}\).

What is the typical temperature needed to produce baryon symmetry? 

\todo[inline]{What is baryon symmetry?}

These questions are hard to answer without a grand unification theory.

How much antimatter is there in the universe?
A long time ago, it was thought that there might have been regions in the universe which were filled with antimatter by looking for a \(\gamma \)-ray background, but this was not found.
We did not find them.

The baryon number is the difference between the number of baryons and antibaryons: \((n_b - n_a) / (n_b + n_a)\). This is actually computed with quark numbers, so it can be computed even when protons and neutrons have not yet formed.

It seems like the only antimatter known is the one which was formed by us, or cosmic rays.

The value \(\Omega_{0b} \approx 0.04\) is defined as 
%
\begin{align}
  \rho_{0b} = \Omega_{0b} \rho _{c} 
\,,
\end{align}
%
where \(\rho_c = 3 H_0^2 / (8 \pi G)\).

When we define the number of protons in the universe we also fix the number of neutrons, since the universe is globally neutral.
So we can estimate: 
%
\begin{align}
  n_{0b} = \frac{\rho_{0b}}{m_p} \approx \SI{1.12e-5}{cm^{-3}} \times h^2 \Omega_{0b}
\,,
\end{align}
%
where \(h \sim 0.7\),
and similarly we can compute 
%
\begin{align}
  n_{0 \gamma } \approx \SI{420}{cm^{-3}}
\,,
\end{align}
%
so we can 
look at the baryon to photon number ratio: 
%
\begin{align}
  \eta_0  = \frac{n_{0b}}{n_{0\gamma}}
  \approx \num{3e-8} \Omega_{0b} h^2
\,,
\end{align}
%
why does this number have this value? 

The denominator in \((n_b - n_a) / (n_b + n_a)\) is approximately \(2 n_{\gamma }\) then, and both the difference in the numerator and the numerator scale like \(a^{-3}\), so this value is a constant.
We do not see antimatter, therefore \(n_a = 0\). So, we get 
%
\begin{align}
    \frac{n_b - n_a}{n_b + n_a} \approx \frac{n_b}{2 n_\gamma } = \frac{1}{2} \eta_0 
\,.
\end{align}
%
This must then have been the case also in the matter dominated epoch, during which there was a very slight imbalance in baryons vs antibaryons.

In order to generate a baryon-antibaryon asymmetry we need \begin{enumerate}
    \item ?
    \item \(C\) and \(CP\) violation;
    \item out-of-equilibrium processes.
\end{enumerate}

These were proposed by a famous Soviet scientist.

We define the interaction rate \(\Gamma \): it is the number of interactions per unit time.

The Hubble rate \(H = \dot{a} / a \) is also an inverse time: baryons and antibaryons are practically speaking \emph{decoupled} if \(\Gamma \leq H\): this is equivalent to saying that the time of interaction is larger than the age of the universe.

When are particles actually coupled or decoupled?
\(\Gamma \) can be calculated as \(\Gamma = n \expval{\sigma v}\), where \(n\) is the number density, \(v\) is the velocity of the particles, and \(\sigma \) is the cross section of the interaction.

\end{document}