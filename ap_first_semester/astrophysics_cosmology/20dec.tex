\documentclass[main.tex]{subfiles}
\begin{document}

% \section*{Fri Dec 20 2019}

% The lectures in january will be about \emph{cosmological} gravitational waves mainly. 

% Write to him if you do not have access to the dropbox. 

% On the exam: a traditional oral exam, with questions on the main topics dealt with in class. 
% The days on the calendar do not mean anything. 
% The exams should be agreed upon by email. 

% Now we talk about the formation of dark matter halos. 
% This is in Sabino's notes in the dropbox. 

We say we have a spherical object in the universe, focus on it and apply Birkoff's theorem: we can study it independently of the surroundings. 

We can consider halos of different densities to account for over and under densities. 
In the real world, we will not have spheres, but three-axial ellipsoids: it is known that if we have over-densities the non-sphericity will increase, if we have under-densities it will decrease. 

However we treat spherical models because it is simple.
Historically the Americans supported the spherical models, the Russians supported the ``pancake model''. 
Now we know that we have pancakes with spheres inside (?). 

We consider a sphere. We define: 
%
\begin{align}
\delta (\vec{x}, t) = \frac{\rho (\vec{x}, t) - \overline{\rho}(t)}{\overline{\rho}(t)} 
\,,
\end{align}
%
and we assume \(0<\delta \ll 1\) (a \emph{small over-density}), although in general we could have \(-1 < \delta < + \infty \). 

We found in previous lessons that we have a growing mode \(\delta \propto t^{2/3}\) and a decaying mode \(\delta \propto t^{-1}\), for which we had \(v \propto t^{1/3} \) and \(v \propto t^{-4/3}\) respectively. 

So we choose a time \(t_i\) such that 
%
\begin{align}
\delta (t) = \delta_+ (t_i) \qty(\frac{t}{t_i})^{2/3}
+ \delta_- (t_i) \qty(\frac{t}{t_i})^{-1}
\,.
\end{align}

The linearized continuity equation was: 
%
\begin{align}
v = i \frac{ \dot{\delta}}{k} a 
\propto \qty(\frac{2}{3} \delta_{+} (t_i) \qty(\frac{t}{t_i})^{1/3} - \delta_{-} (t_i) \qty(\frac{t}{t_i})^{-4/3})
\,,
\end{align}
%
since \(a \propto t^{2/3}\). This is in order to write explicitly the fact that we need two initial conditions. 

We suppose that at \(t=t_i\) we have \emph{unperturbed Hubble flow}: \(v(t_i) = 0\), which means 
%
\begin{align}
\delta_{-} (t_i) = \frac{2}{3} \delta_{+}(t_i)
\,,
\end{align}
%
and we can have a generic initial density, which we can express as: 
%
\begin{align}
\delta (t_i) = \delta_{i} = \delta_{+} + \delta_{-} = \frac{5}{3} \delta_{+} 
\,.
\end{align}

Our perturbation can be dealt with cosmologically as being a \emph{local FRLW closed universe}. 
Say our sphere has a radius \(R\): then if we have \( \delta = \delta_i\) then 
%
\begin{align}
\Omega (t_i) = 1 + \delta_{i}
\,.
\end{align}

We have then, from the Friedmann equations: 
%
\begin{align}
\dot{a}^2 = \frac{8 \pi G}{3} \rho a^2 - k 
\,,
\end{align}
%
where \(k = +1\) if \(\delta_{i} > 0\), because of the fact that our sphere is locally a closed universe. 
This then means: 
%
\begin{align}
- k = (1 - \Omega ) a^2 H^2
\,,
\end{align}
%
so 
%
\begin{align}
\frac{\dot{a}^2 }{a_i^2} = H_i^2 \qty(
    \Omega_{p} (t_i) \frac{a_i}{a} + (1 - \Omega_{p}(t_i))
)
\,,
\end{align}
%
where the index \(p\) denotes the fact that we are talking about the perturbation. We have: 
%
\begin{subequations}
\begin{align}
\rho_{p} (t) &= \rho_{p}(t_i) \qty(\frac{a_{pi}}{a_p})^3  \\
&= \rho_{c} (t_i) \Omega_{p} (t_i) \qty(\frac{a_{pi}}{a})^3
\,.
\end{align}
\end{subequations}

We want to derive a time for the \emph{turnaround time} \(t_m\): 
%
\begin{subequations}
\begin{align}
\rho_{p} (t_m) &= \rho_{c} (t_i) \Omega_{p}(t_i) \qty(\frac{\Omega_{p}(t_i) - 1}{\Omega_{p} (t_i)})^3  \\
&= \rho_{c}(t_i) \frac{\qty(\Omega_{p}(t_i) - 1)^3}{\Omega_{p}(t_i)^2}
\,,
\end{align}
\end{subequations}
%
since at \(t = t_m\) we have \(\frac{\dot{a}^2}{a_{i}^2} = 0\). 
\todo[inline]{since \(\dot{a}(t_m) = 0\)? }

Some time ago we had found: 
%
\begin{align}
t (\theta_{m} = \pi ) = t_m = \frac{\pi }{2 H_0 } \frac{\Omega_0 }{(\Omega_0 - 1)^{3/2}}
\,,
\end{align}
%
but calculating this \emph{now} is arbitrary: the same formula holds replacing \(0\) with \(i\). 

Then we have 
%
\begin{align}
t_m = \frac{\pi }{2 H_i} \qty(\frac{\rho_{c}(t_i)}{\rho_{p}(t_m)})^{3/2}
\,,
\end{align}
%
since we have found that in the relation from a few lessons ago we have exactly the inverse of the relation we just derived connecting \(\Omega \) and \(\rho \). 

It is convenient to consider unperturbed Hubble flow so that the Hubble parameter inside and outside is the same. 

Then we have 
%
\begin{align}
H^2(t_i) = \frac{8 \pi G}{3} \rho_{c}(t_i)
\,,
\end{align}
%
so we have a cancellation: we find 
%
\begin{align}
t_m = \frac{\pi }{2 H_i} \qty(\frac{\rho_{c}(t_i)}{\rho_{p}(t_i)})^{1/2} = \qty(\frac{3 \pi }{32 G \rho_{p}(t_m)})^{1/2}
\,,
\end{align}
%
so we get 
%
\begin{align}
\rho_{p} (t_m) = \frac{3 \pi }{32 G t_m^2}
\,,
\end{align}
%
which holds inside the sphere, while the critical density \emph{outside} is 
%
\begin{align}
\rho_{c}(t_m) = \frac{1}{6 \pi G t_m^2}
\,.
\end{align}

We are implicitly using the \emph{synchronous gauge}, and we are finding an exact solution to the EFE: The Lemaître-Tolman-Bondi solution. 

We can ask: at a certain given time, how much is the interior density larger than the exterior one? this is given by 
%
\begin{align}
1 + \delta_{p}(t_m) = \chi (t_m) = \frac{\rho_{p}(t_m)}{\rho_{c}(t_m)} = \frac{3 \pi }{32 G } 6 \pi G = \qty(\frac{3 \pi }{4})^{2} \approx 5.6 
\,.
\end{align}

Then we have \(\delta_{p}(t_m) \approx 4.6\). 
How does this compare to linear theory? 
%
\begin{align}
\delta_p (t_m) = \delta_{t}(t_i) \qty(\frac{t_m }{t_i})^{2/3} + \delta_{-}(t_i) \qty(\frac{t_m}{t_i})^{-1}
\approx \frac{3}{5} \delta_{p}(t_i) \qty(\frac{t_m}{t_i})^{2/3} 
\,,
\end{align}
%
since the second term vanishes. 

We know that \(H_i = 2/ (3 t_i)\), so 
%
\begin{align}
t_m = \frac{3 \pi t_i}{4} \times \qty(\frac{1 + \delta_{i}}{ \delta_{i}^{3/2}}) \approx \frac{3 \pi t_i}{4} \delta_{i}^{-3/2}
\,,
\end{align}
%
which comes from the equation from the perturbation density at \(t_m\), in which we substitute \(\Omega_{p} - 1 = (\delta_{p} + 1)  - 1 = \delta_{p}\), and we approximate \(1 + \delta \approx 1\) since \(\delta \) is small. 

So we have 
%
\begin{align}
\frac{t_m}{t_i} 
=\frac{3 \pi }{4} \delta_{i}^{-3/2}
\,,
\end{align}
%
which gives us a correction 
%
\begin{align}
\delta_{p}(t_m) =
\frac{3}{5} \delta_{i} \qty( \frac{3 \pi }{4} \delta_{i}^{-3/2})^{2/3}
=\frac{3}{5} \qty(\frac{3 \pi }{4})^{2/3} \approx 1.87
\,.
\end{align}

So linear theory would have given us a wrong answer, as we should expect: we are very far from the \(\delta \ll 1\) regime in which we expect linearization to hold. 

We speak of the virialization time (the time after which the VT starts applying?). It is hard to calculate, different books give different values. 
%
\begin{align}
t _{\text{vir}} \approx t_c = 2 t_m
\,.
\end{align}

Then we have 
%
\begin{align}
E _{\text{tot}} = T + E _{\text{gr}} = \frac{1}{2} E _{\text{gr}} = - T  
\,,
\end{align}
%
since \(2 T + E _{\text{gr}} = 0\).

The total energy at virialization is given by 
%
\begin{align}
E _{\text{eq}} = - \frac{1}{2} \frac{3}{5} \frac{GM^2}{R _{\text{eq}}}
\,,
\end{align}
%
where the factor \(3/5\) is given by the geometry of the system, we are assuming constant density for our sphere, while the \(1/2\) comes from the VT.  
At the time of collapse instead the energy is given by 
%
\begin{align}
E_m =- \frac{3}{5} \frac{GM^2}{R_m}
\,,
\end{align}
%
This then means that the radius at virialization, \(R _{\text{eq}}\), is twice the radius at the start of the collapse, \(R_m\): 
%
\begin{align}
2 R _{\text{eq}} = R_{m}
\,.
\end{align}

This also means, by mass conservation, that \(\rho _{\text{vir}} = 8 \rho_{m}\) since the volume shrinks by \(2^{3}\). 

We want to know the density at collapse:: 
%
\begin{align}
\frac{\rho_{p}(t_c)}{\rho_{c}(t_c)} = \underbrace{\frac{\rho_{p}(t_c)}{\rho_{p}(t_m)}}_{8} \underbrace{\frac{\rho_{p}(t_m)}{\rho_{c}(t_m)}}_{\chi \approx 5.6} \underbrace{\frac{\rho_{c}(t_m)}{\rho_{c} (t_c)}}_{2^2}
\approx 180
\,.
\end{align}

This object is called \(1 + \delta(t_c) \), so \(\delta (t_c) \approx 179\). 

What would happen if we were to use linear theory at this time? Then we would find 
%
\begin{align}
\delta_{+} (t_c) = \delta_{+}(t_m) \qty(\frac{t_c}{t_m})^{2/3}
\,,
\end{align}
%
which gives us
%
\begin{align}
\delta_+ (t_c) \approx \frac{3}{5} \qty(\frac{3 \pi }{4})^{2/3} 2^{2/3} \approx 1.686 
\,,
\end{align}
%
which is a kind of ``clock'': how long can we use linear theory for? 

We deal with *** theory: we introduce 
%
\begin{align}
n(M) = \dv{N}{M} = \text{\# of objects per unit volume with mass in } [M, M + \dd{M}]
\,.
\end{align}

Let us consider linear perturbations \(\delta(\vec{x}, t)\): perturbations in the matter density dealt with using linear theory only. 

These tend to fluctuate a lot. We need a \emph{filter}: we use a \emph{low-pass} filter, ignoring the high-frequency modes. It is \(W_R(\vec{x})\): \(R\) is a spatial radius, we ignore modes with spatial frequency larger than \(1/R\). 

We use a filter labelled by a mass \(M\), which we find by assuming a certain density, and then using \(M \propto R^3\). 

It is generally a good idea to assume gaussianity. In the Planck data it was found by Sabino's team that the bounds on non-gaussianity are very low, the data are almost gaussian. So, we have 
%
\begin{align}
p(\delta_{M}) \dd{ \delta_{M}} = \frac{1}{\sqrt{2 \pi \sigma_{M}^2}} \exp( - \frac{ \delta_{M}^2}{2 \sigma_{M}^2}) \dd{ \delta_{M}}
\,,
\end{align}
%
where the variance is typically diverging if we do not apply the filter: we have 
%
\begin{align}
\sigma_{M}^2    = \expval{ \delta_{M}^2 } \propto M^{-2 \alpha }
\,,
\end{align}
%
and typically \(\alpha \sim 1/2\). 

We define a threshold for the value of \(\delta_{M}\), and we want to compute the probability of the value becoming larger than it. 
We use for the critical value \(\rho_{c} = 1.686   \) from before. 
We have 
%
\begin{align}
\mathbb{P}_{> \delta_{c}}(M) = \int_{ \delta_{c}}^{ \infty } \dd{ \delta_{M}} p( \delta_{M})
\,.
\end{align}
%
so we have 
%
\begin{subequations}
\begin{align}
n(M) M \dd{M} 
&= \rho_{m} \qty(\mathbb{P}_{ > \delta_{c}} (M) - \mathbb{P}_{ > \delta_{c}} (M + \dd{M})) \\
&= \rho_{m} \abs{ \dv{\mathbb{P}_{> \delta_{c}}}{M}} \dd{M}  \\
&= \rho_{m} \abs{ \dv{\mathbb{P}_{> \delta_{c}}}{ \sigma_{M}}} \abs{ \dv{\sigma_{M}}{M}} \dd{M}
\,.
\end{align}
\end{subequations}

Integrating \( \dv{\mathbb{P}_{> \delta_{c}}}{M}\) we expect to find the matter density again, but we find \(1/2\) of it. 

This comes from a miscount: as the mass we are considering shrinks, we might me already including smaller objects inside the gravitational influence of larger ones. Properly accounting for this one gets precisely a factor 2. 

Integrating we get 
%
\begin{align}
n(M) = \frac{2}{\sqrt{\pi }} \frac{\rho_{m}}{M_{*}^2}
\alpha \qty(\frac{M}{M_{*}})^{\alpha }
\exp(- \qty(\frac{M}{M_{*}})^{2 \alpha })
\,,
\end{align}
%
where \(M_{*} = \qty(2 / \delta_{c})^{ 1/ 2 \alpha } M_0 \). 

Accounting for non spherical collapse we get much better estimates.

\end{document}