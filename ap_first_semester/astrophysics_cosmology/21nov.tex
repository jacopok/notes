\documentclass[main.tex]{subfiles}
\begin{document}

\section*{Thu Nov 21 2019}

We exponentiate the equation from before: 
we get 
%
\begin{align}
  X_d = X_n X_p
  \exp(-29.33 + \frac{25.82}{T_\rho  } - \frac{3}{2} \log T_\rho + \log ( \Omega_0 h ))
\,,
\end{align}
%
where \(T_\rho  = \) 
\todo[inline]{What is going on? What is \(T_{\rho }\)? }

We want to understand why there is so much He-4 in the universe, since it is destroyed in stars! 

This model fits observation as long as \(\num{.011} h^{-2}\leq \Omega_0 \leq \num{.25} h^{-2}\). 
Most people agree that we are around the upper bound. 

\todo[inline]{This is indirect evidence for dark energy: why?}

The lifetime of the neutron, \(\tau_{1/2}\) is something that is also relevant, since it affects the baryon ratios. 

The Gamow factor \(\Gamma \) is proportional to the Fermi coupling constant \(G_F^2\), which is connected to \(\tau_{1/2}\). 

Let us suppose we increase the lifetime of neutrons, \(\tau_{1/2}\). 
This changes the moment at which we reach equation \(\Gamma \sim H\). 

Increasing the lifetime of neutrons decreases the amount of He-4 in the universe: less is produced. 

We know that 
%
\begin{align}
  H^2 = \frac{8 \pi G}{3} \rho_r
\,,
\end{align}
%
where \(G = 1 / m_p^2\) and \(\rho = \frac{\pi^2}{30} g_{*}(T) T^4\).

If we fix the temperature, and change \(g_{*}\) (by adding degrees of freedom), then we get more He-4. 

This gives us observational constraints on the additions of exotic particles to our theory, since that would change \(g_{*}\). 
This bounds the number of neutrino families by \num{3.} something, so there cannot be more than \num{3} families of light neutrinos. 

If gravitons are thermal, then they also contribute to radiation. 

There is also another parameter in the Friedmann equation; it is \(m_P\): modified gravity theories often predict variations of the gravitational constant with time. 

Dark matter has no relevant electromagnetic interactions: it only interacts gravitationally, and is able to cluster; dark energy, instead, is uniformly distributed. 

We divide it into Hot and Cold dark matter: HDM and CDM. 
There is also something called \emph{Warm} dark matter, which has intermediate properties.

In HDM, particles have very high thermal motion. They move fast, and tend to destroy gravitational potential wells in which they might settle by moving out of them, and thus decreasing the quantity of matter there. 

They do this on scales comparable to the maximum distance travelled by them: this is calculated as \(vt\), where \(v\) is their average thermal velocity, (and \(t\) is the age of the universe?). 

The structures formed by these are of scales similar  to or larger than \(\num{e15} M_{\odot}\), but we observe smaller structures also! They were formed later, by fragmentation: this is the top-down approach. 

We also have a bottom-up approach, which is compatible with CDM. 

Neutrinos were thought to be Dark Matter, and would have been hot. 

The top-down approach, however, is falsified by the observation of high-redshift quasars combined with the scale of the anisotropies of the CMB: in order to account fo high-redshift small-scale structures (we have seen stars at \(z \sim 20\)!) we would have to increase the amplitude of the anisotropies to a scale which is not compatible to the anisotropies we see in the CMB. 

Now, neutrinos are not useful for cosmology. 

We have \(\Gamma \sim T^{5}\), and \(\Gamma = H\): 
\(T^{5} / \tau_{1/2} \sim T^2\) implies that \(\tau_{1/2} \sim T^{3}\). 

The decouplig temperature for HDM is larger than the temperature at which they become nonrelativistic, which is of the order of the mass. 

For CDM, instead, the decoupling temperature is \emph{smaller} than the temperature at which they become relativistic. 

\end{document}