\documentclass[main.tex]{subfiles}
\begin{document}

\section*{Thu Nov 28 2019}

We derived 
%
\begin{align}
  \expval{T} = -\frac{1}{3} \frac{E _{\text{grav}}}{V}
\,,
\end{align}
%
so now we proceed: we want a relation between kinetic and gravitational energy densities. 

We use a statistical mechanics approach. 

The rate of momentum transfer in the direction \(x\) is given by 
%
\begin{align}
  \frac{N}{L^3} p_x v_x
\,,
\end{align}
%
and this will be true for either direction by isotropicity: so 
%
\begin{align}
  P = \frac{n}{3} \expval{\vec{p} \cdot \vec{v}}
\,,
\end{align}
%
in full generality. 

Let us consider the two cases of nonrelativistic and fully relativistic. In the nonrelativistic case we have 
%
\begin{align}
  \epsilon_{p} = mc^2 + \frac{p^2}{2m}
\,,
\end{align}
%
where \(p = mv\). In the ultrarelativistic case we have 
%
\begin{align}
  \epsilon_{p} = pc
\,,
\end{align}
%
and the velocity is approximately the speed of light. 

Then, for a gas of nonrelativistic particles, we have 
%
\begin{align}
  P = \frac{1}{3} n mv^2 = \frac{2}{3} n \expval{\frac{1}{2}m v^2} = \frac{2}{3} \times \text{translational KE density}
\,,
\end{align}
%
while in the relativistic case we have: 
%
\begin{align}
  P = \frac{1}{2} n \expval{pc} = \frac{1}{3} \times \text{translational KE density}
\,.
\end{align}

We will show that, if a star is made of a gas of classical nonrelativistic particles it tends to be stable, if the particles are relativistic then it tends not to be stable.

The virial theorem tells us that 
%
\begin{align}
  2 E _{\text{K}} + E _{\text{grav}} = 0
\,,
\end{align}
%
in the nonrelativistic approximation. 

We define: \(\Delta E _{\text{tot}} = - \Delta E _{\text{K}} = \frac{1}{2} \Delta E _{\text{grav}}\). 

We know that 
%
\begin{align}
  \expval{P} = \frac{1}{3} \frac{E _{\text{K}}}{V} = -\frac{1}{3} \frac{E _{\text{grav}}}{V}
\,
\end{align}
%
by the virial theorem: so the total binding energy is equal to zero, since this gives us 
%
\begin{align}
  E _{\text{grav}} + E _{\text{K}} = E _{\text{tot}} = 0
\,.
\end{align}
%

Now we differentiate the law \(\dd \qty(P V^{\gamma }) = 0\), since it is a constant for an adiabatic transformation: it gives us, using logarithmic derivatives, 
%
\begin{align}
  \gamma \frac{ \dd{V}}{V} + \frac{ \dd{P}}{P} = 0
\,,
\end{align}
%
so 
%
\begin{align}
  \dd{(PV)} = - (\gamma -1 ) P \dd{V}
\,,
\end{align}
%
and we know that for an adiabatic transformation 
%
\begin{align}
  \dd{E _{\text{in}}} + P \dd{V} = 0
\,,
\end{align}
%
which implies 
%
\begin{align}
  \dd{E _{\text{in}}} = \frac{1}{\gamma -1} \dd{(PV)}
\,,
\end{align}
%
and let us assume that \(\gamma \) is approximately constant in the transformation: this means 
%
\begin{align}
  E _{\text{in}} = \frac{PV}{\gamma -1}
\,,
\end{align}
%
so 
%
\begin{align}
  P = (\gamma -1 ) \frac{E _{\text{in}}}{V}
\,,
\end{align}
%
which justifies the relations we used in cosmology, \(P = w \rho \) with \(w = \gamma -1\). 

We can rewrite the equation from before as 
%
\begin{align}
  3(\gamma -1 ) E _{\text{in}} + E _{\text{gr}} = 0
\,,
\end{align}
%
which, together with \(E _{\text{tot}} = E _{\text{in}} + E _{\text{gr}}\), give us 
%
\begin{align}
  E _{\text{tot}} = - (3 \gamma - 4) E _{\text{in}}
\,,
\end{align}
%
which means that \(\gamma > 4/3\) characterizes a bound system, while \(\gamma < 4/3\) characterizes a free system. 

There are two dangers: one is the fight against the pressure forces, one is the fight against the quantum forces (the Pauli exclusion principle) which do not allow the compression to happen further. 

Now we discuss Jeans instability: 
%
\begin{align}
  E _{\text{grav}} = - f \frac{GM^2}{R}
\,,
\end{align}
%
where \(f\) is a numerical factor of the order \(1\), depending on the mass distribution. If the distribution is uniform, it is \(3/2\).
\todo[inline]{3/2?} 

The kinetic energy is 
%
\begin{align}
  E _{\text{K}} = \frac{3}{2} N k_B T
\,,
\end{align}
%
and we want to impose the condition 
%
\begin{align}
  f \frac{GM^2}{R} > \frac{3}{2} N k_B T 
\,,
\end{align}
%
and the Jeans criterion is this boundary of the stability region: 
%
\begin{align}
  f \frac{gM_J^2}{R} = \frac{3}{2} \frac{M_J}{\bar{m}} k_B T
\,,
\end{align}
%
where \(\bar{m} = M / N\). 
The \(J\) denotes the fact that we are considering the specific boundary mass on both sides. Simplifying the formula we find: 
%
\begin{align}
  M_J = \frac{3}{2} \frac{k_B T }{G \bar{m}} R
\,,
\end{align}
%
and we can reframe this in terms of the density, which is defined by 
%
\begin{align}
  M_J = \frac{4 \pi }{3} \rho _J R^3
\,.
\end{align}
%
We cube and multiply on both sides: 
%
\begin{align}
  \rho_J M_J^3 = \qty(\frac{3 k_B T}{2 G \bar{m}})^3R^3 \rho_J
\,,
\end{align}
%
so we get 
%
\begin{align}
  \rho_J = \frac{1}{M_J^3} \qty(\frac{3 k_B T}{G \bar{m}})^3 \qty( \frac{4 \pi }{3} \rho_J R^3) \frac{1}{4 \pi }
\,,
\end{align}
%
so 
%
\begin{align}
  \rho _J = \frac{3}{4 \pi M_J^2} \qty(\frac{3 k_B T}{2 G \bar{m}})^3
\,,
\end{align}
%
and we will habe stability if the density is larger than this. So, if we want a collapse, we must decrease the mass\dots

When the last scattering happens, the pions are decoupled from the photons. Dark matter behaves differently from conventional matter. 

We have 
%
\begin{align}
  \dot{\rho}_r = - 3H \qty(\rho _r + P_r)
\,,
\end{align}
%
and 
%
\begin{align}
  \dot{\rho}_m = -3H \qty(\rho _m + P_m )
\,,
\end{align}
%
and \(P_r = \rho_r / 3\), which scale like \(a^{-4} \) and also as \(T^{4}\), which means \(T \sim 1/a\).  
%
\begin{align}
  \dd \qty(\rho _m c^2 a^3) + P_m d a^3 = 0
\,,
\end{align}
%
where we usually approximate \(\rho _m c^2 = m_p n_b c^2\), but we can include more terms: 
%
\begin{align}
  \rho _m c^2 = m_p n_b c^2 \qty(1 + (\gamma -1 )^{-1} \frac{k_B T}{m_p c^2})
\,,
\end{align}
%
while the pressure is given by \(P = n_b k_B T\): so in the end we find 
%
\begin{align}
  \dd \qty(\qty(m_p n_b c^2 + \frac{3}{2} m_p n_b \frac{k_BT}{m_p})a^3) = - n_b k_B T \dd{a^3}
\,,
\end{align}
%
which after some computation gives us 
%
\begin{align}
  \frac{1}{2} \dd{T } = - T \frac{ \dd{a}}{a}
\,,
\end{align}
%
which implies \(T_m \propto a^{-2}\) after baryogenesis. 

\todo[inline]{This is for monoatomic baryonic matter, right?}

Let us start writing equations for the stellar interior. The continuity equation is 
%
\begin{align}
  \partial_{t} \rho + \nabla \cdot \qty(\rho \vec{v}) = 0
\,,
\end{align}
%
and also we have the Euler equation 
%
\begin{align}
  \partial_{t} \vec{v} + \qty(\vec{v} \cdot \vec{\nabla}) \vec{v}
  = - \frac{1}{\rho } \vec{\nabla} P - \vec{\nabla} \Phi 
\,,
\end{align}
%
which can be written using the convective time derivative: 
%
\begin{align}
  \frac{ \mathrm{D} }{\mathrm{D}t} = \partial_{t} + \vec{v} \cdot \nabla_{x} = u^{\mu } \partial_{\mu }
\,,
\end{align}
%
which allows us to write 
%
\begin{align}
  \frac{ \mathrm{D} }{\mathrm{D}t} \rho + \rho \nabla \cdot \vec{v} = 0
\,,
\end{align}
%
and 
%
\begin{align}
  \frac{ \mathrm{D} }{\mathrm{D}t} \vec{v} 
  = - \frac{1}{\rho } - \nabla \Phi 
\,.
\end{align}

What happens to the entropy? we define the entropy density \(s \) by \(S  = s \rho \). 

An isentropic process is one in which 
%
\begin{align}
  \frac{ \mathrm{D} S}{\mathrm{D}t} = 0
\,,
\end{align}
%
which in terms of the entropy density is 
%
\begin{align}
  \partial_{t} s + \vec{v} \cdot \vec{\nabla} s = 0 
\,.
\end{align}
%

The external force is provided by the gravitational field: 
%
\begin{align}
  \nabla^2 \Phi = 4 \pi G \rho 
\,.
\end{align}

Jeans looked for a simple solution, an ansatz, called the background solution and then tried to perturb it: if it is stable than it was a good solution. 

He started with \(\rho = \const\), \(\vec{v} = 0\), \(s = \const\), \(\Phi = \const\). 

It is obviously wrong! It cannot satisfy the Poisson equation. 

However, we start from it and add some \(\delta \rho \), \(\delta \vec{v}\) (which we just call \(\vec{v}\)), \(\delta s\) and \(\delta \Phi \); then we only keep the linear terms in these perturbations. 

We find: 
%
\begin{align}
  \partial_{t} \delta \rho  +
  \rho_0 \vec{\nabla} \cdot \vec{v} = 0 
\,,
\end{align}
%
%
\begin{align}
  \partial_{t} \vec{v} = - \frac{1}{\rho_0 } \nabla \delta P - \nabla \delta \Phi 
\,,
\end{align}
%
and 
%
\begin{align}
  \nabla^2 \delta \Phi = 4 \pi G \delta \rho 
\,,
\end{align}
%
and finally 
%
\begin{align}
  \partial_{t} \delta s = 0
\,.
\end{align}

We can expand
%
\begin{align}
  \delta P = \pdv{P}{\rho } \delta \rho + \pdv{P}{s} \delta s
\,,
\end{align}
%
and here we define 
%
\begin{align}
  c_s^2 = \pdv{P}{\rho }
\,. 
\end{align}

We will then consider an exponential solution: 
%
\begin{align}
  \delta \rho = \delta \rho_0 \exp(i \qty(\vec{k} \cdot \vec{x} - \omega r))
\,,
\end{align}
%
and similarly for \(\vec{v}\), \(s\), \(\Phi \). 

We will see that we will need to stick to \(\delta s =0\), and find a dispersion relation with \(\omega \) and \(\vec{k}\): it will be 
%
\begin{align}
  \omega^2 = c_s^2 \vec{k}^2 - 4 \pi G \rho 
\,,
\end{align}
%
so if the wavenumber is small enough we will have an imaginary \(\omega \). 

\end{document}
