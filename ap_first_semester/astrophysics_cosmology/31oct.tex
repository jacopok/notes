\documentclass[main.tex]{subfiles}
\begin{document}

\section*{Thu Oct 31 2019}

Neutrinos have very low mass: therefore they become relativistic very quickly.

We saw last time the case of either bosons' or fermions' number density, energy density and pressure. 

In the nonrelativistic case instead we must consider the second order in \(q / \sqrt{m} \), and we get the Boltzmann suppression factor out of the integral: \(\exp(-m/T) \).

In the nonrelativistic case there is no distinction between bosons and fermions.

What happens in the part of the history of the universe where all the particles were in thermal equilibrium?

If the timescales of the interactions are much larger than the cosmological events timescales (such as the energy of the universe), then those interactions do not happen.
These are called \emph{decoupled}.

Let us consider ultrarelativistic particles which are not \emph{decoupled}, in the early universe which is radiation dominated (here ``radiation'' refers to all kinds of ultrarelativistic particles). 

In this case, from the ``conservation of the stress-energy tensor'' we know that \(\rho \propto a^{-4}\).

Our equation is 
%
\begin{equation}
  H^2= \frac{8 \pi G}{3} \rho - \frac{k}{a^2}
\,,
\end{equation}
%
and we want to neglect the last term. The equation becomes 
%
\begin{subequations}
\begin{align}
  \qty(\frac{\dot{a} }{a})^2 &= \frac{8 \pi G }{3} \rho_{*} \qty(\frac{a}{a_{*}})^{-4} - \frac{k}{a^2} 
\,,
\end{align}
\end{subequations}
%
but it is not enough to look at the slopes: we can, however, have information about the normalization as well from present-day observations.
If the second term is much smaller than the first today, then it was even more so in the far past. We can rewrite the equation as
%
\begin{equation}
  1 = \Omega_{\text{tot}} - \frac{k}{a^2 H^2} \equiv \Omega _{\text{tot}} + \Omega_{\text{curvature}}
\,,
\end{equation}
%
where \(\Omega _{\text{curvature}} \equiv -k / (a^2 H^2)\).

So we neglect the second term: approximately we then have 
%
\begin{equation}
  \qty(\frac{\dot{a } }{a})^2 = \frac{8 \pi  G }{3} \rho _{\text{rad}}
\,.
\end{equation}
%
We know that \(a(t) \propto t^{1/2}\), therefore \(\dot{a} / a = H = 1/2t\). 

\todo[inline]{Wait, how does this work?}

So, we have 
%
\begin{equation}
  \frac{1}{4 t^2} = \frac{8 \pi G}{3} g_{*} \frac{\pi^2}{30} T^{4}
\,,
\end{equation}
%
where 
%
\begin{equation}
  g_{*} = g_{*}(T) = \sum _{i \in BE } g_i \qty(\frac{T_{i}}{T})^{4} + \frac{7}{8} \sum _{i \in FD } g_i \qty(\frac{T_{i}}{T})^{4}
\,,
\end{equation}
%
where we insert a correction factor in order to not consider particles which are nonrelativistic (?).
\(BE\) and \(FD\) denote the bosonic and fermionic degrees of freedom respectively.

The \(i\) spans the different types of particles in the SM.
The \(T_i\) are the thermalization temperatures of the various species.

The plot of \(g_{*}\) in terms of the temperature goes ``down in steps'', as more and more species become nonrelativistic.

Then we have a formula for temperature in terms of time: 
%
\begin{equation}
  \frac{1}{2t} = \qty(\frac{8 \pi G}{3})^{1/2} g_{*}^{1/2} \qty(\frac{\pi^2}{30})^{1/2} T^2
\,,
\end{equation}
%
therefore we get: 
%
\begin{equation}
  t \approx 0.301 g_{*}^{-1/2} \frac{m_P}{T^2} \approx \qty(\frac{T}{\SI{}{MeV}})^{-2}s
\,,
\end{equation}
%
where \(m_P  = G^{-1/2}\) is the Planck mass. Beware: there are different conventions for this!
The factor \(g_{*}^{-1/2}\) is of order 1 around \(T \approx \SI{1}{MeV}\), so we ignore it.

The number is found by:
%
\begin{equation}
  0.301 \approx \frac{1}{2 \sqrt{\frac{8 \pi \pi^2}{3 \times 30}}}
\,.
\end{equation}
%

The Planck mass is approximately \SI{1.2e19}{GeV}.
It is the scale at which we need to account for quantum gravitational effects.

When the universe is 1 second old, weak interactions decouple.

Why are we allowing different thermalization temperatures?

A hypothesis is that the particles exit the Planck epoch, \(t \approx m_P\), thermalized.
When decoupling occurs, they can stop being thermalized.

Another hypothesis is that they stop being thermalized at a certain temperature during inflation.

Some time ago we discussed entropy conservation.

The entropy density is given by 
%
\begin{equation}
    s \equiv S/V = (P + \rho ) / T = \frac{4}{3} \frac{\rho}{T} = (2 \pi^2 / 45) g_{* s} T^{3} 
\,;
\end{equation}
%
since \(V \propto a^{3}\) we have \(s a^3 = \const\).

We defined a new \(g_{*}\) to account for the different species: 
%
\begin{equation}
  g_{* s} \equiv \sum _{i \in BE} g_i \qty(\frac{T_i}{T})^3
  + \frac{7}{8} \sum _{i \in FD} g_i \qty(\frac{T_i}{T})^3
\,,
\end{equation}
%
so our new Tolman's law is \(T a g_{* s }^{1/3} = \const\).

When the neutrinos decouple, their temperature keeps scaling like \(1/a\): they keep being thermalized with the photons even though they do not react.

When we reach \(T \approx \SI{0.5}{MeV}\) electrons and positrons become thermalized: they decay into photons: the photon temperature \(T\) increases with respect to what would happen without this process (it actually just decreases slower).

The neutrinos are not updated when this process happens: their temperature then becomes consistently less than the one of the photons, but they keep scaling the same: one of their temperatures is a constant multiple of the other after this event.

Now we will calculate this multiple.

First, at \SI{1}{MeV}, neutrinos decouple. Then, at \SI{0.5}{MeV}, electrons and positrons decouple.

We require continuity at this transition: \(T_{>}\) and \(T_{<}\) must both be \(T_{e}\).  
%
\begin{equation}
  s a^3 = 
  \frac{2 \pi^2}{45} g_{*s>}T^{3}_{>} a^{3}_{>} = 
  \frac{2 \pi^2}{45} g_{*s<}T^{3}_{<} a^{3}_{<}
\,,
\end{equation}
%
but we can drop all the factors we know to be continuous on the boundary:
\todo[inline]{isn't \(a\) different on either side of the boundary, if it is not instant?}
we are left with 
%
\begin{equation}
  T_{<} = \qty(\frac{g_{*s>}}{g_{*s<}})^{1/3} T_{<}
\,,
\end{equation}
%
where we can compute 
%
\begin{equation}
  g_{* s>} = 2 + \frac{7}{8}\qty(4) = \frac{11}{2}
\,,
\end{equation}
%
where we considered photons, electrons but not neutrinos (since they are separated). On the other hand, 
%
\begin{equation}
  g_{*s<} = 2
\,,
\end{equation}
%
since electrons are not thermalized anymore. Therefore 
%
\begin{equation}
  T_{<} = \qty(\frac{11}{4})^{1/3} T_{>}
\,,
\end{equation}
%
which allows us to compute the neutrino temperature at any time: 
%
\begin{equation}
  T_{\nu } = T_{\gamma } \qty(\frac{4}{11})^{1/3}
\,,
\end{equation}
%
since they scale the same.

Now let us compute \(\rho_{r} (T = \SI{0.1}{MeV})\).
Let us assume that the global temperature is the one of the photons: we get 
%
\begin{subequations}
\begin{align}
  g_{*} &=  \sum _{i \in BE} g_i \qty(\frac{T_i}{T})^{4} + 
  \frac{7}{8}\sum _{i \in FD} g_i \qty(\frac{T_i}{T})^{4}  \\
  &= 2 + \frac{7}{8} \qty(3 \times 2 \qty(\frac{T_{\nu }}{T_{\gamma }})^{4})  \\
  &= 2 + \frac{21}{4} \qty(\frac{4}{11})^{4/3} \approx 3.36
\,,
\end{align}
\end{subequations}
%
since we need to consider neutrinos, which contribute to the total energy density, but not electrons which are not relativistic.

\end{document}