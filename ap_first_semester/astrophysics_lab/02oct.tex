\documentclass[main.tex]{subfiles}
\begin{document}

\section*{2 October}

What are the sources for X-ray astronomy, beyond the Sun? Many binary systems in our galaxy, there is variability on the scales of milliseconds.

Also, galactic nuclei emit X-rays; also the gas trapped in the middle of clusters of galaxies is very massive.

Diffuse radiation also exists. We do not really know where it comes from.

On top of our atmosphere there is an \emph{ionosphere}, which reflects our radio signals: we can communicate even when we are so far that the curvature of the Earth would prohibit it.

In the sixties, some people put Geiger counters on WWII V2 rockets, and someone actually put a directional sensor: this allowed people to discover the fact that the ionosphere was ionized by X-rays from the Sun.

In the 1960s Giacconi (the fast-moving one) and Rossi (the clever one) thought of a way to build a ``telescope'' for X-ray astronomy: with gold foil shaped into a section of a parabola we can focus the X-rays.
This would allow people to see something but the Sun, like the moon or the Crab Nebula.

Giacconi founded his own company to build the thing.

They used an \emph{anti-coincidence} shield: this allows us to distinguish (charged) cosmic rays from X-rays.
This allowed Giacconi to find evidence for X-rays from beyond the solar system: PRL, 1962.

We can use lunar occultation in order to specifically figure out which source the X-rays come from. This was used with the Crab nebula.

In 1965 people discover the fact that X-ray sources can  be variabile.

We can use collimators in order to select only a specific direction.

An interesting quantity to look at is the ratio of luminosities \(L_X / L_{\text{optical}}\). For the Sun, this is of the order \num{1e-6}. For Sco X-1 this is \num{1e3} (!)

If something is moving back and forth, we can use red and blueshift periodicity to figure out the periodicity of the system.
With Kepler's law we can get a lower bound for the companion mass.
This is done in the optical.

Compared to 1962 we have \num{1e9} better sensitivity, \num{1e5} better angular resolution, \num{1e4} better spectral resolution (now \(E / (\Delta E ) \sim \num{1e3} \)). Now we use X-ray CCDs, not Geiger counters.

\end{document}
