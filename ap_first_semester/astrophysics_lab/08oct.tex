\documentclass[main.tex]{subfiles}
\begin{document}

\section*{8 October 2019}

We visit \url{https://heasarc.gsfc.nasa.gov}

This is the archive where we will spend most of our time: one can find the data from all the current missions.

At the link \url{http://adsabs.harvard.edu/abstract_service.html} one can find Astrophysics papers, with full text sources.

There is the extragalactic database \url{https://ned.ipac.caltech.edu/}, and a galactic database at \url{http://simbad.u-strasbg.fr/simbad/}.

In \url{https://heasarc.gsfc.nasa.gov} one can go to Tools \(\rightarrow\) Coordinate converter to find coordinates.
If you look at an object and then at another it tells you what the angular distance between them is.

In Tools \(\rightarrow\) energy converter you can convert energies.

Now, let us talk about \emph{specific intensity}:
it is defined as
%
\begin{equation}
  I_\nu = \frac{\dd{E_\nu}}{\dd{A} \dd{\Omega} \dd{\nu} \dd{t} \cos \theta}
  = \frac{h\nu\dd{N_\nu}}{\dd{A} \dd{\Omega} \dd{\nu} \dd{t} \cos \theta}
\end{equation}

We also define \emph{flux density}, measured in \SI{}{\erg\per\square\centi\meter\per\hertz\per\second}:
%
\begin{equation}
  F_\nu = \int_{\text{source}} I_\nu \cos \theta \dd{\Omega}
\end{equation}
%
and \emph{flux}, measured in \SI{}{\erg\per\square\centi\meter\per\second}:
%
\begin{equation}
  F = \int_{\nu_1}^{\nu_2} F_\nu
\end{equation}



\end{document}
