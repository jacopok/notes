\documentclass[main.tex]{subfiles}
\begin{document}

\section*{9 October 2019}

Clarification from last lecture.
Dithering is caused by the fact that any single pixel in an X-ray CCD may be arbitrarily noisy, and if the telescope only looks at an astronomical object from a certain viewpoint without rotation a certain region of the object will always be imaged by the same pixel, therefore the errors between images will be correlated.
So, we move our telescope around in order to have a certain region of the astronomical object be imaged by several uncorrelated pixels.

Normal distribution: the pdf is
%
\begin{equation}
  \mathcal N (x; \mu, \sigma^2) = \frac{1}{\sqrt{2 \pi \sigma^2}} \exp(\frac{-(x-\mu)^2}{2\sigma^2} ) \,.
\end{equation}

Poisson distribution: the probability of getting \(m\) events if they happen independently at a rate \(\lambda\) is
%
\begin{equation}
\P (m ; \lambda) = \frac{e^{-\lambda} \lambda^m}{m!} \,,
\end{equation}
%
and it is easy to prove that for a Poisson distribution \(\mu = \sigma^2 = \lambda\).

If the rate is, say, a temporal rate like \([\alpha] = \SI{}{\hertz}\) then we substitute \(\lambda = \alpha t_{\text{obs}}\).

We discuss detectors: semiconductors, NP junctions.

\end{document}
