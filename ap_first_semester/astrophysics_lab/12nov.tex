\documentclass[main.tex]{subfiles}
\begin{document}

\section*{Tue Nov 12 2019}

Today we will start with 20' of lectures, then research.
This will be the same for a few lectures onwards.

Chandra data can be retrieved at \url{https://cda.harvard.edu/chaser}.

Both CHANDRA and XMM-Newton are \emph{huge}.
CHANDRA is optimized to do images (good angular resolution), while XMM-Netwon has a much better spectral resolution.

CHANDRA has 4 nested shells, which limit the background noise \(B\) in 
%
\begin{align}
  SNR = \frac{S}{\sqrt{S + B}}  
\,,
\end{align}
%
on the other hand, we can increase \(S\): we make the effective area larger. XMM has three telescopes, each of which 58 shells: a total area of \SI{120}{m^2}.

The PSF FWHM is \SI{0.5}{\arcsec}.
PSF means ``Point Spread Function'': the response of the detector to a Dirac delta.

For each pixel, we have a possibility of the equal photons getting in our detector at the same time: they are then detected as having double the energy. This makes the spectrum seem harder.
This can be fixed by looking at a source off-axis if it is very bright.

Usually we look at a circle around the source which encloses 80\% of the power emitted.

All of these effects are encoded in the \texttt{arf}: the Ancillary Response File.

The information of the channel-to-energy
conversion and the spectral resolution will be
taken care of by the:
Redistribution Matrix File (rmf).

Evaluation criteria: content, presentation, language, knowledge.

We are given an envelope, which gives us the observation ID from CHANDRA: that gives us the group number also.

Our ID is 2121, the group number is 3.

The object is \url{http://simbad.u-strasbg.fr/simbad/sim-id?Ident=MCG-5-23-16}

Different wavelengths \url{http://cdsportal.u-strasbg.fr/?target=ESO%20434-40}

Title	The reprocessing features in the X-ray spectrum of the NELG MCG -5-23-16
Authors	Balestra, I.; Bianchi, S.; Matt, G.
Bibcode	2004A\&A...415..437B
Abstract	We present results from the spectral analysis of the Seyfert 1.9 galaxy MCG -5-23-16, based on ASCA, BeppoSAX, Chandra and XMM-Newton observations. The spectrum of this object shows a complex iron K$\alpha$ emission line, which is best modeled by a superposition of a narrow and a broad (possibly relativistic) iron line, together with a Compton reflection component. Comparing results from all (six) available observations, we do not find any significant variation in the flux of both line components. The moderate flux continuum variability (about 25\% difference between the brightest and faintest states), however, does not permit us to infer much about the location of the line-emitting material. The amount of Compton reflection is lower than expected from the total iron line EW, implying either an iron overabundance or that one of the two line components (most likely the narrow one) originates in Compton-thin matter.

Title	The Cores of the Fe K$\alpha$ Lines in Active Galactic Nuclei: An Extended Chandra High Energy Grating Sample
Authors	Shu, X. W.; Yaqoob, T.; Wang, J. X.
Bibcode	2010ApJS..187..581S
Abstract	We extend the study of the core of the Fe K$\alpha$ emission line at ~6.4 keV in Seyfert galaxies reported by Yaqoob \& Padmanabhan using a larger sample observed by the Chandra high-energy grating (HEG). The sample consists of 82 observations of 36 unique sources with z < 0.3. Whilst heavily obscured active galactic nuclei are excluded from the sample, these data offer some of the highest precision measurements of the peak energy of the Fe K$\alpha$ line, and the highest spectral resolution measurements of the width of the core of the line in unobscured and moderately obscured (N H < 1023 cm-2) Seyfert galaxies to date. From an empirical and uniform analysis, we present measurements of the Fe K$\alpha$ line centroid energy, flux, equivalent width (EW), and intrinsic width (FWHM). The Fe K$\alpha$ line is detected in 33 sources, and its centroid energy is constrained in 32 sources. In 27 sources, the statistical quality of the data is good enough to yield measurements of the FWHM. We find that the distribution in the line centroid energy is strongly peaked around the value for neutral Fe, with over 80\% of the observations giving values in the range 6.38-6.43 keV. Including statistical errors, 30 out of 32 sources (~94\%) have a line centroid energy in the range 6.35-6.47 keV. The mean EW, among the observations in which a non-zero lower limit could be measured, was 53 ± 3 eV. The mean FWHM from the subsample of 27 sources was 2060 ± 230 km s-1. The mean EW and FWHM are somewhat higher when multiple observations for a given source are averaged. From a comparison with the H$\beta$ optical emission-line widths (or, for one source, Br$\alpha$), we find that there is no universal location of the Fe K$\alpha$ line-emitting region relative to the optical broad-line region (BLR). In general, a given source may have contributions to the Fe K$\alpha$ line flux from parsec-scale distances from the putative black hole, down to matter a factor ~2 closer to the black hole than the BLR. We confirm the presence of the X-ray Baldwin effect, an anti-correlation between the Fe K$\alpha$ line EW and X-ray continuum luminosity. The HEG data have enabled isolation of this effect to the narrow core of the Fe K$\alpha$ line.

\end{document}