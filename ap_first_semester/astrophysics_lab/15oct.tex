\documentclass[main.tex]{subfiles}
\begin{document}

\section*{Tue Oct 15 2019}

We come back to the signal-to-noise ratio. 
The SNR is dimensionless.

What happens to the SNR if we increase pixel size?
It improves. However, if we normalize for the area the thing is different: \(D\) and \(R_N^2\) go up with pixel size. Overall there is more noise with bigger pixels.

Background-limited instruments: if \(B\) is negligible then \(\text{SNR} \propto \sqrt{F _{\text{src}} t} \), if it is dominant then \(\text{SNR} \sim F _{\text{src}}/ F _{\text{bkg}} \sqrt{t} \).

The \emph{Quantum efficiency} is the fraction of photons we can detect. The human eye is around 1\%, CCDs can get up to 90\%.

A bit of lesson on telescopes, it can be found on the slides.

CHANDRA has good angular resolution while XMM has good spectral resolution.

INTEGRAL stops at \SI{100}{keV}, Fermi starts at \SI{100}{MeV}.

Now, \emph{inverse Compton emission}.
In the rest frame of the electron, which we will use throughout, it loses energy like 
%
\begin{equation}
  - \qty(\dv{E}{t} )' = \sigma _T c U' _{\text{rad}}
\,,
\end{equation}
%
where \(U' _{\text{rad}}\) is the energy density of radiation,  while \(\sigma _T\) is the Thomson scattering cross section.

Depending on the temperature of the electrons, we can have regions with mostly Compton scattering or inverse Compton scattering: 
%
\begin{equation}
  \frac{\Delta E}{E_0} = \frac{4 k_B T_e - E_0 }{m_e c^2}
\,,
\end{equation}
%
depending on the sign of the numerator on the RHS we can see whether energy is transferred to or from electrons.

\begin{greenbox}
  What is \(E_0 \)?
\end{greenbox}

In the Kompaneets equation we impose \(\pdv*{n}{t} = 0\), since \(n\) is frequency integrated and Compton scattering conserves photon number.

Now, follow the instructions in the slides. Open \texttt{ISGR-EBDS-MOD}: ``all'', you can then see the conversion of energy to channel.

The quantity measured in 
%
\begin{equation}
  \SI{}{keV}^2 \qty(\SI{}{photon / cm^2 / s / keV})
\,,
\end{equation}
%
describes the energy output of the source at each energy.

\end{document}
