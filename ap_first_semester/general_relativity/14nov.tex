\documentclass[main.tex]{subfiles}
\begin{document}

\section*{Thu Nov 14 2019}

\subsection{Riemann normal coordiates}

We want to actually build a LIF, in which \(g_{\mu \nu } (P ) = \eta_{\mu \nu }\) and \(g_{\mu \nu , \rho } (P)= 0\).

We can choose a set of vectors \(e_{(\mu )} (P)\) which are orthonormal: \(e_{(\mu )} \cdot e_{(\nu )} = \eta_{(\mu )(\nu )}\). 

If we choose this \emph{tetrad} as a basis, then the metric becomes the Minkowski one at that point.

Take a vector \(n^{\mu }\) at \(P\), and consider all possible geodesics which start from \(P\) with initial tangent vector \(n^{\mu }\).

Then, the coordinates of a point \(Q\) we get by moving for a time \(\tau \) along this geodesic are \(x^{\alpha } = \tau n^{\alpha }\) if \(n^{\alpha }\) is timelike, \(x^{\alpha } = s n^{\alpha }\) if it is spacelike, where 
%
\begin{align}
  \tau = \int _{P}^{Q} \dd{\tau } 
\,.
\end{align}

Since these lines are geodesics, they satisfy the geodesic equation: 
%
\begin{align}
  \dv[2]{x^{\alpha }}{\tau } + \Gamma^{\alpha }_{\beta \gamma } \dv{x^{\beta }}{\tau } \dv{x^{\gamma }}{\tau } = 0
\,,
\end{align}
%
but we can insert the relation \(x^{\alpha } = \tau n^{\alpha }\) here, to get \(\Gamma^{\alpha }_{\beta \gamma } n^{\beta } n^{\gamma }=0\): but this can be written for any \(n^{\alpha }\), therefore we immediately get that all the Christoffel symbols vanish: \(\Gamma^{ \alpha }_{\beta \gamma } \equiv 0\). 

Therefore, we have \(4^{3}\) equations of sums of derivatives of the metric which vanish: but the gradient of the metric only has \(4^{3}\) independent components, therefore the solution \(g_{\mu \nu , \alpha } \equiv 0\) is the only one.

Can the geodesics cross each other far from the starting point? \emph{Yes}, so the Riemann normal coordinates are only defined in a neighbourhood.

Let us consider an example: Riemann normal coordinates around the north pole of a sphere.

We can map every point on the sphere but the south pole by specifying the meridian and the distance to travel along the meridian.

So in our case, if \(R\) is the radius of the sphere, we get \(x^{\alpha } = (R \theta \cos(\phi ), R \theta \sin(\phi ))\), where then \(R \theta = \tau \) and \(n^{\alpha } = (\cos(\theta ), \sin(\theta ))\).
The angles \(\phi \) and \(\theta \) are the usual spherical coordinate angles: \(\phi \) specifies the meridian, while \(R\theta \) gives us the distance travelled away from the north pole (since \(\theta \) is in radians).

The second order expansion of the metric is 
%
\begin{align}
  g_{ij} = \left[\begin{array}{cc}
  1 - \frac{2y^2}{3R^2} & \frac{2xy}{3R^2} \\ 
  \frac{2xy}{3R^2} & 1-\frac{2x^2}{3R^2}
  \end{array}\right]
\,,
\end{align}
%
where \(x^{\alpha } = (x, y)\).

We can see that the metric is \(\delta_{ij}\) at the north pole, and its derivatives are \(g_{ij,k}= 0\) there.

\section{The Schwarzschild solution}

It descrives the geometry outside a stationary, spherically symmetric object which is not rotating and not electrically charged, such as a star, planet or BH.

In general \(\dd{s^2} = -A(r) \dd{t^2} + B(r) \dd{r^2} + C(r)^2 \qty(\dd{\theta^2} + \sin^2 \theta \dd{\phi })\) is our line element.

We can define \(\widetilde{r} = C(r)\), and then express \(A\), \(B\) with respect to to this, and recalling 
%
\begin{align}
    \dd{r^2} = 
  \frac{\dd{\widetilde{r}^2} }{(\dv*{C}{r})^2}
\,,
\end{align}
%
which is what multiplies \(B\), so we redefine \(B(\widetilde{r}) \) as \(B(\widetilde{r}) / (\dv*{C}{r})^2\).

So, we can just relabel \(B\) as this: the expression 
%
\begin{align}
    \dd{s^2} = -A(r) \dd{t^2} + B(r) \dd{r^2} + r^2\qty(\dd{\theta^2} + \sin^2 \theta \dd{\phi })
\,
\end{align}
%
is fully general. So far we have used the hypothesis of stationarity by writing everything only as a function of \(r\).

Recall the inverted Einstein equations: 
%
\begin{align}
  \frac{1}{8 \pi G} \qty(T_{\mu \nu } - \frac{T}{2} g_{\mu \nu }) = R_{\mu \nu }
\,.
\end{align}
%

We want to solve these outside of our source: we look for \emph{vacuum solutions}. Then the equations are just \(R_{\mu \nu } =0\).
A trivial solution to these is \(g_{\mu \nu } = \eta_{\mu \nu }\), but we will show that it is not the only one!
As a matter of fact, we know that the solution to a differential equation is determined by the boundary conditions: in our case, the mass of the object which sits at the origin.
The Minkowski metric respects these boundary conditions if \(M=0\). In general, it does not.



\end{document}