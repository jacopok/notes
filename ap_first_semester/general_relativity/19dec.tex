\documentclass[main.tex]{subfiles}
\begin{document}

\section*{Thu Dec 19 2019}

We found the equation 
%
\begin{align}
\dot{\rho}  + 3 \frac{\dot{a}}{a} \rho (1 + w)=0
\,,
\end{align}
%
where \(w = p / \rho \), and we found that for
vacuum energy, with constant \(\rho \), we get \(w = -1\). 

Now we will derive the fact that \(w = 1/3\) for radiation.
This can be derived from Maxwell's equations, but it also can come from a more illustrative argument: we consider photons in a box. 

If they travel along the \(x\) axis, they have wavevectors \(k^{\mu } = (\omega ,  \omega , 0, 0)\) and momenta \(p^{\mu } = \omega \hbar (1, 1, 0, 0,)\).

In general they will travel in a different direction, of course, but we treat this case. 
Each photon hits a wall every \(\Delta t = 2L\), and transfers momentum equal to \(2 \hbar \omega \). So, the average force exerted is \(\Delta p / \Delta t = \hbar \omega /L\). 

In the cavity there is an energy \(\rho V = \rho AL\), where \(\rho \) is the photons' energy density, \(L\) is the length along \(x\) while \(A\) is the area normal to the \(x\) direction. 

Each photon has energy \(\hbar \omega \), so there are \(\rho A L / \hbar \omega \). We assume for simplicity (but it gives the exact correct answer) that \(1/3\) of the photons travel along each spatial direction. 

So there are \(\rho A L / 3 \hbar \omega \) photons travelling in the \(x\) direction. So the average force exerted on the wall is given by 
%
\begin{align}
F_{x} = \frac{\hbar \omega }{L} \times \frac{\rho A L}{3 \hbar \omega } = \frac{\rho A}{3}
\,,
\end{align}
%
so the average pressure is \(\rho /3\), so \(w = 1/3\). 

Now let us consider nonrelativistic particles: if \(v \ll 1\) then the 4-momentum looks like 
%
\begin{align}
p^{\mu } = (m, mv, 0, 0)
\,,
\end{align}
%
so the force along the \(x\) axis is given by 
%
\begin{align}
F_{\text{one particle}} = \frac{\Delta p_{x}}{\Delta t} = \frac{2mv}{2L / v} = \frac{mv^2}{L}
\,,
\end{align}
%
and as before the total energy is \(\rho L A\), so the number of particles travelling along the \(x\) direction is \(\rho L A /3 m\), since the momentum contribution to the particle's energy is negligible: therefore 
%
\begin{align}
F_{x, \text{tot}} = \frac{mv^2}{L} \times \frac{\rho LA}{3m} 
= \frac{\rho A v^2}{3} \approx 0
\,,
\end{align}
%
which we can neglect since it is quadratic in \(v\) which is very small. So we say that for relativistic matter \(w = 0\). 

We have the differential equation 
%
\begin{align}
\dv{\rho }{t} + 3 \frac{1}{a} \dv{a}{t} (1+w)\rho =0
\,,
\end{align}
%
so we can integrate: 
%
\begin{align}
\int \frac{ \dd{\rho }}{\rho } = - 3 (1+w) \int \frac{ \dd{a}}{a}
\,,
\end{align}
%
wihch is readily solved by integrating from \(\rho =\rho_0 \) and \(a=a_0 \):
%
\begin{align}
\rho = \rho_0 \qty(\frac{a_0 }{a})^{3 (1+w)}
\,.
\end{align}
%

From the square root of the 00 EFE we get: 
%
\begin{align}
\frac{\dot{a}}{a} = \frac{1}{\sqrt{3}M_P} \rho^{1/2}
\,,
\end{align}
%
into which we can put the solution we found: 
%
\begin{align}
\frac{\dot{a}}{a} = \frac{1}{\sqrt{3}M_P} \rho_0^{1/2} \qty(\frac{a_0 }{a})^{\frac{3(1+w)}{2}}
\,,
\end{align}
%
and if we define \(x = a / a_0 \) this becomes: 
%
\begin{align}
\dv{x}{t} x^{\frac{3(1+w)}{2} - 1} = \frac{\rho_0 ^{1/2}}{\sqrt{3}M_P}
\,,
\end{align}
%
so we integrate: 
%
\begin{align}
\int_{1}^{a / a_0 } \dd{x} x^{\frac{3(1+w)}{2}} = 
\frac{\rho_0^{1/2}}{\sqrt{3}M_P} (t-t_0 )
\,,
\end{align}
%
so we need to integrate \(\int x^{\alpha } \dd{x}\): this can be \(\log x\) if \(\alpha = -1\) or \(x^{\alpha +1} / (\alpha +1)\) if \(\alpha \neq -1\). 

We get the logarithm when \(w = -1\): then 
%
\begin{align}
\log \frac{a}{a_0 }  = \frac{\rho_0}{\sqrt{3}M_P} \qty(t - t_0 )
\,,
\end{align}
%
which means 
%
\begin{align}
a = a_0 \exp(\frac{\rho_0^{1/2}}{\sqrt{3}M_P} \qty(t - t_0 ))
\,,
\end{align}
%
which is called \emph{De Sitter spacetime}. Notice that in this case \(\rho \equiv \rho_0 \) since the density of vacuum energy is constant. In all other cases 
%
\begin{align}
\frac{2}{3 (1+w)} \qty(\qty(\frac{a}{a_0 })^{3\frac{1+w}{2}} -1) = \frac{r_0^{1/2}}{\sqrt{3}M_P} \qty(t - t_0 )
\,,
\end{align}
%
and notice that \(t_0 \) is an arbitrary choice: we usually choose 
%
\begin{align}
t_0 = \frac{2}{3 (1+w)} \frac{\sqrt{3}M_P}{\rho_0^{1/2}}
\,,
\end{align}
%
which then gives: 
%
\begin{align}
\qty(\frac{a}{a_0 })^{\frac{3 (1+w)}{2}} = \frac{3 (1+w)}{2} \frac{\rho_0^2 }{\sqrt{3}M_P} t = \frac{t}{t_0 }
\,.
\end{align}
%
Then we get an ``easy'' answer: 
%
\begin{align}
\frac{a}{a_0 } = \qty(\frac{t}{t_0 })^{\frac{2}{3 (1+w)}} 
\,.
\end{align}

Some cases are: 
\begin{enumerate}
    \item Matter: \(w=0\) implies \(\rho \sim a^{-3}\), \(a \sim t^{2/3}\);
    \item Radiation: \(w=1/3\) implies \(\rho \sim a^{-4}\) and \(a \sim t^{1/2}\);
    \item Vacuum energy: \(w = -1\) implies \(\rho = \const\) and \(a \sim \exp t\).
\end{enumerate}

If \(a = \overline{a} \equiv \const\) then we can rescale the spatial coordinated \(\vec{x} \rightarrow \overline{a} \vec{x}\), so we recover Minkowski spacetime. 

This implies that, in a flat Friedmann - Robertson - Lemaître - Walker universe, the normalization of the scale factor is unphysical. 

This can be noticed from the equations: only the \emph{logarithmic} derivative of the scale factor enters them. 

If we have a universe which is \emph{not} flat, we get an additional factor \(k / a^2\), with \(k = \pm 1\) in the Friedmann equations: then the normalization of the scale factor becomes physical. 

Let us characterize expansion: let us consider two galaxies, one at the spatial coordinates \(\vec{x}_{1} = \vec{0}\) and the other at \(\vec{x}_{1} = (\overline{x}, 0,0)\). 

The distance between them is given  by 
%
\begin{align}
d_{P} (t) = \int_{0}^{\overline{x}} \dd{x} \sqrt{g_{11} } = a(t) \overline{x}
\,.
\end{align}

How much does the distance change between two times \(t_1 \) and \(t_2 \)? Their ratio is 
%
\begin{align}
\frac{  d_{P}(t_1 )}{d_{P}(t_2 )} = \frac{\overline{x} a(t_1 )}{\overline{x} a(t_2 )} = \frac{a(t_1 )}{a(t_2 )}
\,.
\end{align}

These are called \emph{comoving coordinates}: the coordinates on a grid which is expanding. 

So the distance can change both because of this comoving expansion, and because of \emph{proper motion}: things actually moving through the grid. 

We can write the FRLW metric with respect to spherical coordinates: 
%
\begin{align}
\dd{s^2} = - \dd{t^2} + a^2(t) \qty(\dd{r^2} + r^2 \dd{\Omega^2})
\,.
\end{align}

A photon moves with \(\dd{s^2} =0\), which implies 
%
\begin{align}
\dd{r} = \frac{ \dd{t}}{a(t)}
\,.
\end{align}

Say we have an emitter Alice and an observer Bob, both at fixed radial coordinates. Alice sends two photons (or wavecrests, whatever) at a time difference  \(\Delta t_A\) apart, and Bob receives them at a time difference \(\Delta t_B\) apart. 

In general \(\Delta t_A \neq \Delta t_B\). We have 
%
\begin{align}
\int_{r_A}^{r_B} \dd{r} = \int_{t_A}^{t_B}  \frac{ \dd{t}}{a(t)} 
= \int_{t_A + \Delta t_A}^{t_B + \Delta t_B}  \frac{ \dd{t}}{a(t)} 
\,.
\end{align}

This then implies that 
%
\begin{align}
\int_{ t_A }^{ t_A + \Delta t_A} \frac{ \dd[]{t}}{a(t)} = 
\int_{ t_B }^{ t_B  + \Delta t_B} \frac{ \dd[]{t}}{a(t)} 
\,.
\end{align}

The \(\Delta t\) are \emph{very} small with respect to the total times: then this means 
%
\begin{align}
\frac{\Delta t_A}{a(t_A)} \approx \frac{ \Delta t_B}{ a(t_B)}
\,,
\end{align}
%
which can be written as 
%
\begin{align}
\Delta t_B = \Delta t_A \frac{a(t_B)}{a(t_A)}
\,,
\end{align}
%
or, equivalently, 
%
\begin{align}
\omega_{B} = \omega_{A} \frac{a(A)}{a(B)}
\,,
\end{align}
%
or 
%
\begin{align}
\lambda_{B} = \lambda_{A} \frac{a(B)}{a(A)}
\,,
\end{align}
%
since \(\omega \lambda = 2 \pi \). Since \(a(B) > a(A)\) we get \(\lambda_{B} > \lambda_{A}\): this is a \emph{redshift}. 

Since \(E_{\gamma } = \hbar \omega \propto a^{-1}\), while the volume density \(N_{\gamma } \propto a^{-3}\), we get the global effect of \(\rho_{\gamma } \propto a^{-4}\). 

We define 
%
\begin{align}
z = \frac{\lambda_{B}}{\lambda_{A}} - 1 = \frac{a(t_{B})}{a(t_{A})} - 1
\,,
\end{align}
%
fixing \(t_B = \text{now} = t_0 \). 

\emph{Hubble's law} is a relation between redshift and distance of nearby objects. 

Since we are nearby, we Taylor expand the scale factor: we say 
%
\begin{align}
a(t) = a_0 + \dot{a}_{0} (t - t_0 ) + \mathcal{O}(t^2)
\,,
\end{align}
%
where by \(\dot{a}_{0}\) we mean the derivative of the scale factor computed at the present time.

We have 
%
\begin{align}
d_{P}(t) = a(t) \int_{t_{A}}^{t_B} \dd{t} = a(t) \int_{t_A}^{t_B} \frac{ \dd{\widetilde{t}}}{a (\widetilde{t})}
\,,
\end{align}
%
so this, at the time \(t_0\) gives us 
%
\begin{align}
d_P (t_0 ) = a_0 \int \frac{\dd{t}}{a_0 } = t_B - t_A + \mathcal{O}(t^2)
\,,
\end{align}
%
if we consider the lowest possible order (at which the universe is not expanding).
We can identify \(B = 0\), that is, we observe now.

We can write the first order expansion of the scale factor as 
%
\begin{align}
a_0 - a(t_A) = \dot{a}_{0} (t_0 - t_A) \implies
t_{0} - t_A = \frac{a_0 - a(t_A)}{\dot{a}_{0}}
\,,
\end{align}
%
so we get 
%
\begin{align}
d_P = t_0 - t_A = \frac{a_0 }{\dot{a}_{0}} \qty(1 - \frac{a(t_A)}{a_0 }) 
= \frac{a_0 }{\dot{a}_{0}} \qty(1 - \frac{1}{1+z})
\approx \frac{a_0 }{ \dot{a}_{0}} z + \mathcal{O} (z^2)
\,,
\end{align}
%
since \(z / (1+z) \sim z \) if \(z \sim 0\). 

The (inverse of the) constant multiplying \(z\) is called the \emph{Hubble Constant}: 
%
\begin{align}
H_0 = \frac{\dot{a}_{0}}{a_0 }
\,.
\end{align}

There is also the fact we neglected: what is actually measurable is not the comoving distance but the luminosity distance. 
The value of this constant is around 
%
\begin{align}
H_0 \approx \SI{70}{km s^{-1} Mpc^{-1}}
\,.
\end{align}

There is a disagreement between the measurement of \(H_0 \) from the CMB and the one using standard candels. 

We finish off the topic by discussing the relationship between time and energy. We discuss the present age of the universe and the time of the Big Bang nucleosynthesis. 

As long as \(w \neq 1\) we can immediately derive a relation for \(\dot{a} / a \) by differentiating: 
%
\begin{align}
\frac{\dot{a}}{a} = \frac{2 }{3 (1+w) t}
\,,
\end{align}
%
so the ratio does depend on \(w\), but the \(t\)-dependence is always \(H \sim 1/t\). To get an order of magnitude, we can use a matter-dominated universe, so the prefactor becomes \(2/3\): 
%
\begin{align}
t_0 \sim \frac{1}{H_0} \approx  \SI{4.4e17}{s} \sim \SI{1.4e10}{yr}
\,.
\end{align}
%

At which time did the universe have a temperature of \SI{1 }{MeV}? We know that the energy density looks like 
%
\begin{align}
\rho_{\gamma } = c (k_B T)^{4}
\,,
\end{align}
%
which follows from integrating the Planck distribution. 

In natural units (\(c = \hbar = k_B =1\)), an energy density has dimensions of an energy to the fourth power, since it is energy divided by length cubed, but lengths are inverse energies. Then, 
%
\begin{align}
\rho_{\gamma } \propto T^{4}
\,,
\end{align}
and the proportionality constant is approximately equal to 1. 
We know that 
%
\begin{align}
H^2 = \frac{1}{3 M_P^2} \rho 
\,,
\end{align}
%
which  means that, neglecting the order-1 factor of 3:
%
\begin{align}
H = \frac{T^2}{M_P}
\,,
\end{align}
%
so 
%
\begin{align}
t \sim \frac{M_P}{T^2}
\,,
\end{align}
%
which is the relation between time and temperature in a radiation-dominated universe. In natural units, \(M_P \sim \SI{2e18 }{GeV}\), so 
%
\begin{align}
t \sim \frac{\SI{2e18}{GeV}}{\qty(\SI{1}{MeV})^2} = \SI{2e24}{GeV^{-1}}
\,,
\end{align}
%
so this can be transformed in seconds by multiplying by 
%
\begin{align}
\hbar = \SI{6.6e25}{GeV s}
\,,
\end{align}
%
so \(\SI{}{GeV^{-1}} = \SI{6.6e-25}{s}\): then 
%
\begin{align}
t  = \SI{2e24}{GeV^{-1}} \times \SI{6.6e-25}{GeV s} \approx \SI{1.2}{s}
\,.
\end{align}

This holds for a radiation-dominated universe. 

\end{document}