\documentclass[main.tex]{subfiles}
\begin{document}

\section{Sheet 11}

\subsection{Wave equation}

\subsubsection{General travelling impulse}

We want to show that any twice-differentiable function \(F(z \pm v t) = f(z, t)\) solves the wave equation: 
%
\boxalign{
\begin{align} \label{eq:wave-equation}
\qty(-\partial_{t}^2 + v^2 \partial_{z}^2) f(t, z) = 0
\,. 
\end{align}}

We denote \(\alpha (t,z) = z \pm vt\), so that \(f = F \circ \alpha \) and: 
%
\begin{subequations}
\begin{align}
\pdv[2]{f}{t} = \pdv{}{t }\qty( \dv{F}{\alpha } \pdv{\alpha }{t}) &= \dv{F}{\alpha } \pdv[2]{\alpha }{t} +  \pdv{\alpha }{t} \pdv{}{t} \qty(\dv{F}{\alpha })  \\
&= \dv{F}{\alpha } \pdv[2]{\alpha }{t}
+ \pdv{\alpha }{t}\dv{}{\alpha } \qty(\dv{F}{\alpha } \pdv[]{\alpha }{t}) \marginnote{Commuted the derivatives}\\
&= \dv{F}{\alpha } \pdv[2]{\alpha }{t}
+ \pdv{\alpha }{t} \qty( \dv[2]{F}{\alpha } \pdv[]{\alpha }{t} 
+ \dv{F}{\alpha } \cancelto{}{\pdv{}{t} \dv{\alpha }{\alpha }}) \marginnote{Derivative of a constant}  \\
&= \dv{F}{\alpha } \pdv[2]{\alpha }{t}
+ \dv[2]{F}{\alpha } \qty(\pdv[]{\alpha }{t})^2  \label{eq:faa-di-bruno}\\
&= v^2 \dv[2]{F}{\alpha }
\,,
\end{align}
\end{subequations}
%
where the expression we derived is fully general up to equation \eqref{eq:faa-di-bruno}, while the last passage comes from the fact that \(\pdv*{\alpha }{t} = \pm v\), so its square is \(v^2\) while the second derivative is zero since \(v\) is a constant. By the same reasoning, 
%
\begin{subequations}
\begin{align}
\pdv[2]{f}{z} &= \dv{F}{\alpha } \pdv[2]{\alpha }{z} 
+ \dv[2]{F}{\alpha } \qty(\pdv{\alpha }{z})^2  \\
&= \dv[2]{F}{\alpha }
\,.
\end{align}
\end{subequations}

Therefore, 
%
\begin{align}
\qty(-\partial_{t}^2 + v^2 \partial_{z}^2) f(t, z) 
= (-v^2 + v^2) \dv[2]{F}{\alpha } \equiv 0 
\,.
\end{align}

\subsubsection{Gaussian pulse}

At any fixed time \(t\), the function 
%
\begin{align}
F(\alpha ) = \exp(- \alpha^2) = \exp(- \qty(z - v t)^2)
\,
\end{align}
%
looks like a rescaled Gaussian, centered around \(z = vt\). As \(t\) increases it travels rightward with speed \(v\). A plot of this, with \(v = 6\) and \(t = 0, 1, 2\) is shown in figure \ref{fig:moving-gaussian-pulse}.

\begin{figure}[H]
\centering
\includegraphics[width=\textwidth]{figures/gauss.pdf}
\caption{Pulse for \(t = 0, 1, 2\).}
\label{fig:moving-gaussian-pulse}
\end{figure}

\subsubsection{Harmonic wave solution}

We consider the harmonic solution 
%
\begin{align}
f(t,z) 
= A \cos(2 \pi \qty(\frac{z}{\lambda } - \frac{t}{T})) 
= A \cos(\frac{2\pi}{\lambda } \alpha (t,z)) 
\,,
\end{align}
%
where \(\alpha  = z - \lambda t /T\). Writing the harmonic solution this way allows us to see the properties easily: we have \(v  = \lambda / T\) by comparison with the formula from before (\(\alpha = z - vt\)), while the periodicity follows from that of the cosine: we start from the fact that \(\cos(x) = \cos(x \pm 2 \pi )\).

If we map \(z \rightarrow z  + \lambda \) we get \(\alpha \rightarrow \alpha +\lambda \), and 
%
\begin{align}
\cos(\frac{2 \pi }{\lambda } \alpha ) \rightarrow \cos(\frac{2 \pi }{\lambda }\qty(\alpha + \lambda )) = \cos(\frac{2 \pi }{\lambda } \alpha  + 2 \pi ) = \cos(\frac{2 \pi }{\lambda } \alpha )
\,.
\end{align}

Similarly, if we map \(t \rightarrow t + T\) we get \(\alpha \rightarrow \alpha - \lambda T/T = \alpha - \lambda  \), so we can apply the same reasoning. This means that the wave has spatial period \(\lambda \) and temporal period \(T\). 

\subsubsection{Lightspeed waves}

The equation \(\square f = 0\) is analogous to \eqref{eq:wave-equation} for \(v = 1 (= c)\). The difference is the three-dimensionality of the spatial derivatives, however this does not present an issue. 
We still consider solutions in the form \(f (x^{\mu }) = F(\alpha (x^{\mu }) )\), with \(\alpha (x^{\mu }) = k_{\mu } x^{ \mu }\) for a constant covector \(k_{\mu }\). We will have the following derivatives 
%
\begin{align}
\pdv[2]{f}{(x^{\mu })} 
= \dv[2]{F}{\alpha }
\qty( \pdv{\alpha }{(x^{\mu})} )^2
= \dv[2]{F}{\alpha }
\qty(k_{\mu })^2
\,,
\end{align}
%
where the index \(\mu \) can take the conventional values from \(0\) to \(3\). 
So, the equation \(\square f = 0\) reads 
%
\begin{align}
\dv[2]{F}{\alpha } \qty(-\qty(k_{0})^2 + \sum_{i=1 }^{3} \qty(k_{i })^2) = 0
\,,
\end{align}
%
which is always satisfied if we take for  \(k_{\mu }\) a null vector with respect to the Minkowski metric, \(k_{\mu } \eta^{\mu \nu } k_{\nu } = 0 \). 
In order for the wave to be propagating forwards in time we need to set \(k_{0} < 0\), or equivalently \(k^{0}>0\). 

So, the wave equation does indeed have travelling waves as solutions, which propagate with speed 1 (which can be gathered from the factor multiplying the Laplacian). 

\subsection{Gravitational waves}

\subsubsection{Weak-field wave equation}

We consider a perturbed metric \(g_{\mu \nu } = \eta_{\mu \nu } + h_{\mu \nu }\), to first order in \(h_{\mu \nu }\). 

First of all we need to compute the Christoffel symbols: they are given by 
%
\begin{subequations}
\begin{align}
\Gamma^{\mu }_{\nu \rho } 
&= \frac{1}{2} g^{\mu \alpha } \qty(g_{\alpha \nu , \rho } + g_{\alpha \rho , \nu } - g_{\nu \rho , \alpha })  \\
&= \frac{1}{2} g^{\mu \alpha } \qty(h_{\alpha \nu , \rho } + h_{\alpha \rho , \nu } - h_{\nu \rho , \alpha })  \marginnote{\(\eta_{\mu \nu }  \) is constant}  \\
&= \frac{1}{2} \eta^{\mu \alpha } \qty(h_{\alpha \nu , \rho } + h_{\alpha \rho , \nu } - h_{\nu \rho , \alpha })   + \mathcal{O}(h^2) \marginnote{We only consider the first order in \(h\)}  \\
&= \frac{1}{2} \qty(h^{\mu }_{\nu , \rho } + h^{\mu }_{\rho , \nu } - \tensor{h}{_{\nu \rho , }^{\alpha }}) + \mathcal{O}(h^2) 
\,.
\end{align}
\end{subequations}

Now, let us look at the Ricci tensor: it is given 
%
\begin{subequations}
\begin{align}
R_{\mu \nu } = R^{\alpha }_{\mu \alpha \nu } 
&= -2 \qty(\Gamma^{\alpha }_{\mu [\alpha , \nu ]}
+ \Gamma^{\beta }_{\mu [\alpha } \Gamma^{\alpha }_{\nu ], \beta })  \\
&= \Gamma^{\alpha }_{\mu \nu , \alpha } - \Gamma^{\alpha }_{\mu \alpha , \nu } + \mathcal{O}(h^2)  \\
&= \frac{1}{2} \qty(\partial_{\alpha } \qty(h^{\alpha }_{\mu , \nu } + h^{\alpha }_{\nu , \mu } - \tensor{h}{_{\mu \nu , }^{\alpha }}) - \partial_{\nu } \qty(h^{\alpha }_{\mu , \alpha } + h^{\alpha }_{\alpha , \mu } - \tensor{h}{_{\mu \alpha ,}^{\alpha }}))  \\
&= \frac{1}{2} \qty(\cancelto{}{h^{\alpha }_{\mu , \nu \alpha }} + h^{\alpha }_{\nu , \mu \alpha } - \tensor{h}{_{\mu \nu , }^{\alpha }_{\alpha }} - \cancelto{}{h^{\alpha }_{\mu , \alpha \nu }} - h^{\alpha }_{\alpha , \mu \nu } + \tensor{h}{_{\mu \alpha ,}^{\alpha }_{\nu }}) \marginnote{Commuting derivatives}  \\
&= - \frac{1}{2} \square h_{\mu \nu } + \frac{1}{2} \qty(h^{\alpha }_{\nu , \mu \alpha } + \tensor{h}{_{\mu \alpha, }^{\alpha }_{\nu }} - h_{,\mu \nu } ) \marginnote{Introduced \(\square = \partial^{\alpha } \partial_{\alpha }\) and  \(h = h^{\alpha }_{\alpha }\)}  \\
&= - \frac{1}{2} \square h_{\mu \nu } + \frac{1}{2} 
\partial_{\mu } \qty(h_{\nu, \alpha  }^{\alpha } - \frac{1}{2} h_{, \nu } ) 
+ \frac{1}{2 } \partial_{\nu } \qty(h_{\mu, \alpha  }^{\alpha } - \frac{1}{2} h_{, \mu }) \marginnote{Split \(h_{, \mu \nu } \) in two, swapped the positions of two contracted indices}
\,,
\end{align}
\end{subequations}
%
which is our final expression. 

For brevity\footnote{Which is the formal way to say that I do not feel like writing it all out.} I omit the proof of the fact that setting \(2 h^{\alpha }_{\nu , \alpha } - h_{, \nu } = 0\) (the harmonic gauge condition) is actually a valid gauge choice, it can be found in the lecture notes. Making this gauge choice, the two terms in parentheses cancel so we find 
%
\boxalign{
\begin{align}
R_{\mu \nu } = - \frac{1}{2} \square h_{\mu \nu }
\,,
\end{align}}
%
which means that, if \(T_{\mu \nu } = 0\), the Einstein Field Equations read \(\square h_{\mu \nu } = 0\), which is precisely the wave equation for a wave propagating with speed \(c\). 

\subsubsection{Gravitational waves polarizations}

We are considering a wave in the form 
%
\begin{align}
h_{\mu \nu } = \exp(i k_{\beta } x^{\beta }) \epsilon_{\mu \nu } = \exp(i \qty(- k t + kz)) \epsilon_{\mu \nu }
\,,
\end{align}
%
for a constant matrix \(\epsilon_{\mu \nu }\) and for \(k_{\mu } = (-k, 0,0,k)\). We have already shown in the exercise before that this satisfies the wave equation. We want to impose the conditions: 
\begin{enumerate}
    \item \(\epsilon_{0 \nu } = 0\);
    \item \(2\partial_{\alpha } h^{\alpha }_{\nu } - \partial_{\nu } h = 0\). 
\end{enumerate}

The first of these also implies \(\epsilon_{\nu 0} = 0\) by symmetry (the symmetry of \(g_{\mu \nu } \) implies the symmetry of \(h_{\mu \nu }\), which in turn implies the symmetry of \(\epsilon_{\mu \nu }\), since the exponential is a scalar).

So, we can write the trace of \(h_{\mu \nu }\) as 
%
\begin{align}
h= \eta^{\mu \nu } h_{\mu \nu } &= \exp(i k_{\beta } x^{\beta } ) \epsilon_{\mu \nu } \eta^{\mu \nu }  \\
&= \exp(i k_{\beta } x^{\beta } ) \epsilon_{ij} \delta^{ij} \marginnote{The components with a 0 vanish, and \(\eta_{ij} = \delta_{ij}\)}
\,,
\end{align}
%
so \(h\) depends on the sum of the diagonal entries of \(\epsilon_{ij}\), which is a constant, we will denote it as \(\epsilon = \epsilon_{ij} \delta^{ij}\). 
Let us now think through the possible values of \(\nu \) in the second condition. If \(\nu =0\) we find: 
%
\begin{align}
2 \partial_{\alpha } h^{\alpha }_{0} - \partial_{0} h = 0
\,,
\end{align}
%
but the first term vanishes since all the components of \(\epsilon_{\mu \nu }\) with an index \(0\) vanish, so the same holds for \(h_{\mu \nu }\). Therefore we get \(\partial_{0} h = 0\), which can be expanded into 
%
\begin{align}
- i k \exp(ik_{\beta } x^{\beta }  ) \epsilon = 0
\,,
\end{align}
%
but \(k \) cannot be 0 for a nontrivial solution, and the exponential always has norm 1, so we must conclude that \(\epsilon = 0\); this implies \(h = 0\).  

Moving on to \(\nu = j \): we get 
%
\begin{align}
0 = \partial_{\alpha } h_{\alpha j} = \partial_{i } h_{ij} = \partial_{3} h_{3 j} = ik \exp(ik_{\beta } x^{\beta } ) \epsilon_{3j}
\,,
\end{align}
%
but, similarly to before, nor \(k\) nor the exponential can be zero, therefore \(h_{3j}=0\), and by symmetry also \(h_{j3}=0\). This means that the matrix \(\epsilon_{\mu \nu }\) looks like 
%
\begin{align}
\epsilon_{\mu \nu } = \left[\begin{array}{cccc}
0 & 0 & 0 & 0 \\ 
0 & \epsilon_{11} & \epsilon_{12} & 0 \\ 
0 & \epsilon_{21} & \epsilon_{22} & 0 \\ 
0 & 0 & 0 & 0
\end{array}\right]
\,,
\end{align}
%
so at this point we are left with two possible degrees of freedom: \(\epsilon_{12} = \epsilon_{21}\), and we found before that \(\epsilon_{\mu \nu }\) must be traceless, which means \(\epsilon_{11} = - \epsilon_{22}\). 

For simplicity I will now write only the smaller \(2 \times 2\) matrix \(\widetilde{\epsilon}_{mn}\), where \(m\) and \(n\) range from 1 to 2. Two possible real matrices which satisfy our conditions are the Pauli matrices \(\widetilde{\epsilon}=   \sigma_{x} \) and \( \widetilde{\epsilon} = \sigma_{z}\): 
%
\begin{align}
\sigma_{x} = \left[\begin{array}{cc}
0 & 1 \\ 
1 & 0
\end{array}\right] \qquad \text{and} \qquad
\sigma_{z} = \left[\begin{array}{cc}
1 & 0 \\ 
0 & -1
\end{array}\right]
\,.
\end{align}

These two basis matrices represent the two basis \emph{linear} polarizations of a gravitational wave and we can get any linearly polarized GW along the \(z\) direction by considering linear combinations of these with real coefficients. 

An interesting thing to note, however, is the fact that we can consider \emph{complex} combinations of these basis matrices: the matrices \((\sigma_{z} \pm i \sigma_{x} ) / \sqrt{2}\) represent circularly polarized waves, and they form a basis over \(\mathbb{C}\) for all of the polarizations. 

\end{document}
