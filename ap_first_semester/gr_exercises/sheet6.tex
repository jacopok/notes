\documentclass[main.tex]{subfiles}
\begin{document}

\section{Sheet 6}

\subsection{Sphere pole Riemann coordinates}

\subsection{Schwarzschild metric curvature}

\subsection{Schwarzschild geometry orbits}

The derivation up to the equation for the perturbed orbit equation is documented in the lecture notes, I might copy it here later, but for now one can find it there.

During the lecture we got up to the first order equation for the perturbation \(w\) for the orbit \(u\), written in the form \(u(\varphi ) = u_{c} \qty(1+w(\varphi ))\): 
%
\begin{align}
    \dv[2]{w}{\varphi }  = (6GMu_c-1) w
\,,
\end{align}
%
which is in the form \(\ddot{w} + \omega^2 w = 0\), for \(\omega^2 = 1- 6GMu_c\). Now, we know that the first order equation must be complemented by the zeroth order one: 
%
\begin{align}
    u_c = \frac{GM}{l^2} + 3GM u_c^{2}
\,,
\end{align}
%
which can be solved for \(u_c\) to yield: 
%
\begin{align}
  u_c = \frac{1 \pm \sqrt{1 - 3 \times 4 \frac{G^2 M^2}{l^2}}}{6GM}
\,,
\end{align}
%
therefore the square angular velocity of the perturbation's evolution is: 
%
\begin{align}
  \omega^2 = 1 - \cancelto{}{6GM} \qty(\frac{1 \pm \sqrt{1 - 12 \frac{G^2 M^2}{l^2}}}{ \cancelto{}{6GM}})
  = \pm \sqrt{1 - 12 \frac{G^2 M^2}{l^2}}
\,.
\end{align}

The solution with the minus sign has no meaning for us, since the solution we want to consider must be stable, with positive \(\omega^2\). So, the angular velocity is 
%
\begin{align}
  \omega = \qty(1 - 12 \frac{G^2M^2}{l^2})^{1/4}
\,,
\end{align}
%
and we know that angular velocity and period are related by \(T = 2 \pi / \omega \): therefore we get 
%
\begin{align}
  T = 2 \pi \qty(1 - 12 \frac{G^2M^2}{l^2})^{-1/4}
\,,
\end{align}
%
which we can Taylor expand: at \(x=0\) we have 
%
\begin{align}
    (1-12x)^{-1/4} = 1 - \frac[]{1}{4} (1-12 \times 0)^{-5/4} (-12x) + O(x^2) = 1 + 3x + O(x^2 )
\,.
\end{align}
%

Therefore: 
%
\begin{align}
  T = 2 \pi \qty(1 + 3 \qty(\frac{GM}{l})^2) + O\qty(\qty(\frac{GM}{l})^{4})
\,,
\end{align}
%
which is approximately \(2 \pi \) as we should expect: the Newtonian approximation is \(l \gg GM\), and Newtonian orbits have a period of exactly \(2 \pi \). Then we can read off the first-order correction directly from the first term in the expansion: it is 
%
\begin{align}
  \delta \varphi = 6 \pi \qty(\frac{GM}{l})^2
\,.
\end{align}
%


\end{document}