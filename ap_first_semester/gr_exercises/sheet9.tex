\documentclass[main.tex]{subfiles}
\begin{document}

\section{Sheet 9}

\subsection{Censorship principle justification}

\subsection{Kerr geometry null surfaces}

\subsection{Earth's implausible Kerr description}

We need to estimate the angular momentum of the Earth: for our order-of-magnitude calculation we will model it as a constant-density sphere, with \(M = \SI{6e+24}{kg}\) and \(r = \SI{6e6}{m}\). The moment of inertia is then given by 
%
\begin{align}
  I = \frac{2}{5} M r^{2} \approx \SI{1e38}{kg m^2}
\,,
\end{align}
%
and then, since the Earth rotates at a revolution per day so with \(\omega = 2\pi / \SI{1}{d}\), we have our estimate for the angular momentum: 
%
\begin{align}
  J = I \omega 
  \approx \SI{7e+33}{kg m^2 s^{-1}}
\,.
\end{align}

This is a slight over-estimate: more of the Earth's mass is concentrated around the center, so the actual moment of inertia is about \(20\%\) lower. This is close enough for our purposes. 

In order to find the length corresponding to the natural-units formula \(a = J/M\) we need to divide by \(c\): so we get 
%
\begin{align}
  a = \frac{J}{Mc} \approx \SI{4}{m}
\,,
\end{align}
%
while we have \(GM/c^2 \approx \SI{4e-3}{m}\): their ratio is of the order \num{e3}.

This of course does not mean that there is a naked singularity deep inside the Earth: we'd need all the mass to be concentrated around \(r=0\) for that to be the case, and instead the radius of the Earth is quite larger than \SI{4}{m}.

So we have the quite interesting result that the general-relativistic description of the Earth corresponds to a Kerr metric with \(a \gg GM\). It is not really clear to me what meaning should be drawn from this but it seems to be the case. 

\end{document}