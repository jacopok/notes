\documentclass[main.tex]{subfiles}
\begin{document}

\section*{Fri Nov 29 2019}

The Plummer sphere is defined by 
%
\begin{subequations}
\begin{align}
  \rho (r) \dd[3]{r} &= \frac{3M}{4 \pi a^3} \qty(1 + \frac{r^2}{a^2})^{-5/2} \dd[3]{r}  \\
  &= \frac{3M}{4 \pi a^3} \qty(1 + \frac{r^2}{a^2})^{-5/2} r^2 \dd{r} \dd{\Omega }
\,,
\end{align}
\end{subequations}
%
and it models the mass density in a star cluster.
We can marginalize over the angles to find the pdf of the radius alone: this just amounts to multiplying by \(4 \pi \) by isotropy,
%
\begin{align}
  \rho (r) \dd{r} = \frac{3M}{a^3} \qty(1 + \frac{r^2}{a^2})^{-5/2} r^2 \dd{r}
\,,
\end{align}
%
and if we rescale the radius as \(R = r/a\) we find the PDF 
%
\begin{align}
  \rho (R) \dd{R} = 3M \qty(1 + R^2)^{-5/2} R^2 \dd{R}
\,,
\end{align}
%
which can be integrated analytically: we get 
%
\begin{align}
  \int_{0}^{R_{\text{max}}} \rho(R) \dd{R} = \frac{ M R^3 _{\text{max}}}{\qty(R^2 _{\text{max}} + 1)^{3/2}}
\,.
\end{align}
%

The velocities are isotropically distributed, with the probability of their moduli being described by a Maxwellian density: 
%
\begin{align}
  p(v) \dd{v} = \sqrt{\frac{2}{\pi }} \frac{v^2}{\sigma^3}
  \exp( -\frac{v^2}{2 \sigma^2}) \dd{v}
\,.
\end{align}

Do note that this is \emph{not} normalized on \(\mathbb{R}\) as given: its integral is \(2\), it is normalized on \(\mathbb{R}^{+}\).

The parameters are given by: \(M = \num{e4} M_{\odot}\), \(a = \SI{5}{\parsec}\), \(\sigma = \SI{5}{km/s}\).

Here is my approach to the problem without reading the suggestions, it might not be the most efficient way to do it. 

We need to be able to draw samples from three distributions: the Plummer sphere, the Maxwell distribution and the angular distribution; since both the star's positions and their velocities are isotropically distributed on the 2-sphere. 
The volume element on the sphere \(S^{2}\) is given by \(\dd{A} = \sin \theta \dd{\theta } \dd{\varphi }\); so we can draw our \(\varphi \) from a uniform distribution on \([0, 2 \pi ]\), while our \(\theta \) will need to be distributed according to \(p(\theta ) \dd{\theta }= \sin \theta \dd{\theta }  \). 

Since we are computing angles spanning the whole \(2\)-sphere, for both the radius \(r\) and the velocity \(v\) we do not need to simulate negative values.

This was my first approach, but then I noticed that it is really hard to sample from a distribution \(f(x) \sim x^2 \exp(-x^2)\). 

For the Plummer distribution it makes sense to sample from the radial and angular distribution separately, while for the Maxwell distribution it is much easier to sample the three components of the velocity vector in cartesian coordinates. Let us see this: first, we rescale \(V = v/\sigma \). We get: 
%
\begin{align}
  p(V) \dd{V} = \sqrt{\frac{2}{\pi }} \exp(- V^2/2)V^2 \dd{V}
\,,
\end{align}
%
and we can add in the uniform distribution \(1/ 4\pi \) of the angular part: we find 
%
\begin{subequations}
\begin{align}
p(V) \dd{V} \dd{\Omega } &= \frac{1}{4 \pi } \sqrt{\frac{2}{\pi }} \exp(-V^2/2) V^2 \dd{V} \dd{\Omega }  \\
&= \frac{1}{(2 \pi )^{3/2}} \exp(-V^2/2) \dd[3]{V}  \\
&= \prod_{i=1}^{3} \qty(\frac{1}{\sqrt{2 \pi }} \exp(-V_i^2/2) \dd{V_i})
\,,
\end{align}
\end{subequations}
%
so we can see that in cartesian coordinates each component is just distributed according to a Gaussian. 

\end{document}