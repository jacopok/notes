\documentclass[main.tex]{subfiles}
\begin{document}

\section*{Fri Nov 29 2019}

The Plummer sphere is defined by 
%
\begin{align}
  \rho (r) \dd{r} = \frac{3M}{4 \pi a^3} \qty(1 + \frac{r^2}{a^2})^{-5/2} \dd{r}
\,,
\end{align}
%
and it models the mass density in a star cluster; the velocities are isotropically distributed, with the probability of their moduli being described by a Maxwellian density: 
%
\begin{align}
  p(v) \dd{v} = \sqrt{\frac{2}{\pi }} \frac{v^2}{\sigma^3}
  \exp( -\frac{v^2}{2 \sigma^2})
\,.
\end{align}

The parameters are given by: \(M = \num{e4} M_{\odot}\), \(a = \SI{5}{\parsec}\), \(\sigma = \SI{5}{km/s}\).

Here is my approach to the problem without reading the suggestions, it might not be the most efficient way to do it. 

We need to be able to draw samples from three distributions: the Plummer sphere, the Maxwell distribution and the angular distribution; since both the star's positions and their velocities are isotropically distributed on the 2-sphere. 
The volume element on the sphere \(S^{2}\) is given by \(\dd{A} = \sin \theta \dd{\theta } \dd{\varphi }\); so we can draw our \(\varphi \) from a uniform distribution on \([0, 2 \pi ]\), while our \(\theta \) will need to be distributed according to \(p(\theta ) \dd{\theta }= \sin \theta \dd{\theta }  \). 

Since we are computing angles spanning the whole \(2\)-sphere, for both the radius \(r\) and the velocity \(v\) we do not need to simulate negative values.


 
\end{document}