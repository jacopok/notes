\documentclass[main.tex]{subfiles}
\begin{document}

\subsection{A decay example}

\marginpar{Wednesday\\ 2020-4-1, \\ compiled \\ \today}

This and next week we will finish the introduction to particle physics, then we will start discussing the open problems in cosmology and astroparticle physics. 

We consider the following process: 
%
\begin{align}
e^{+} e^{-} \to \mu^- \mu^+
\,,
\end{align}
%
where the mass of the electron is around \(m_e \sim \SI{.5}{MeV}\), the mass of the muon is around \(m_{\mu } \sim \SI{100}{MeV}\). 

Digression: there are different families of fermions (leptons and quarks), the first encompasses \(e, \nu_{e}, u, d\); the second encompasses \(\mu, \nu_{\mu }, c, s\) and the third encompasses \(\tau, \nu_{\tau }, t, b\). 
The characteristics of the four members of the family are well-known, and between families the characteristics are the same: the only thing which varies between the families is the mass. 

So, Rabi famously asked ``who ordered the fermions''? 

Coming back to our problem: the state \(\ket{e^{+}e^{-}}\) must be annihilated by the EM current \(j^{\mu }_{EM} = \overline{\psi}_e \gamma^{\mu } \psi_e \); it is then converted to a photon, which however is not on mass shell --- it cannot be, since its momentum must be that of the electron-positron pair, so a timelike vector. 
It is then called a \emph{virtual photon}: it can exist, as long as it does so for a short time. 

Then, this photon decays to a muon-antimuon pair: then we will have a term \(\bra{\mu^- \mu^+} j^{\mu }_(\mu )\). The index between parenthesis is not a Lorentz one, it just means that this is a muonic current, different from the electronic one.

Let us call \(p_{-}\) and \(p_{+}\) the momenta of the electron and positron, and \(p^{\prime }_{-}\) and \(p^{\prime }_{+}\) those of the muon and antimuon. 

The momentum \(q\) of the photon cannot have \(q^2  =0 \), but this is fine: it is just an excitation. 

The physics of the process is all contained in the matrix element \(\mathcal{M}(e^{+}e^{-} \to \mu^+ \mu^-)\). How do we calculate it? We will not go into details here, but it can be directly derived from the Feynman diagram of the interaction \cite[sec.\ 4.1.2]{tissinoTheoreticalPhysicsNotes2020}: we have 
%
\begin{align}
\mathcal{M}(e^{+}e^{-} \to \mu^+ \mu^-) = 
(-e)
\bra{\mu^- \mu^+} j^{\mu } \ket{0}
\frac{1}{q^2}
(-e)
\bra{0} j_{\mu } \ket{e^{+} e^{-}}
\,,
\end{align}
%
where the \(-e\) factor is because of the EM coupling to the photon. 
The Breit-Wigner factor looks like 
%
\begin{align}
\frac{1}{P^2-M_R^2}
\,,
\end{align}
%
but for the photon we have no mass, therefore we only get a factor \(1/q^2\). 

The Feynman diagrams are just a way to collect the Feynman rules needed to compute the process, they are not meant to represent how the process ``looks like''. 

Let us take the ultrarelativistic limit, in which the energy of the process is much larger than the muon's mass. 
If this is the case, then we can set \(m_e = m_\mu  = 0\). 

Let us consider the Dirac equation, in the case in which the mass \(m\) is equal to zero: then we get 
%
\begin{align}
i \slashed{\partial} \psi = 0
\,.
\end{align}

Let us use the chiral representation for the \(\gamma \) matrices: 
%
\begin{subequations}
\begin{align}
\gamma^{\mu } = \left[\begin{array}{cc}
0 & \sigma^{\mu } \\ 
\overline{\sigma}^{\mu } & 0
\end{array}\right] 
\,,
\end{align}
\end{subequations}
%
where we mean by \(\sigma^{0} = \mathbb{1}\), \(\sigma^{\mu } = (\sigma^{0}, \sigma^{i})\) and \(\overline{\sigma}^{\mu } = (\sigma^{0}, - \sigma^{i})\).

Let us then split the spinor \(\psi \) into 
%
\begin{subequations}
\begin{align}
\psi = \left[\begin{array}{c}
\psi_{L} \\ 
\psi_{R}
\end{array}\right]
\,,
\end{align}
\end{subequations}
%
where \(\psi_{L, R}\) are two-component spinors. This allows us to write two two-dimensional equations: 
%
\begin{subequations}
\begin{align}
i \overline{\sigma}^{\mu } \partial_{\mu } \psi_{L} &= 0 \\
i \sigma^{\mu } \partial_{\mu } \psi_{R} &= 0
\,,
\end{align}
\end{subequations}
%
where we get no interaction terms between the two: if we have no mass the equations decouple. 

We can do the same thing if the Dirac equation is coupled to the EM field, since the issue is with the structure of the \(\gamma^{\mu }\), it does not matter if we have \(\gamma^{\mu } \DD_{\mu }\) or \(\gamma^{\mu } \partial_{\mu }\); however we will write the decoupled solution for now.

So, for the right-handed spinor we have:
%
\begin{align}
\qty(i \partial_{t} + i \vec{\sigma} \cdot \vec{\partial}) \psi_{R}
\,,
\end{align}
%
which is solved by a plane wave: 
%
\begin{align}
\psi_{R} = u_R (p) e^{-iEt + i \vec{p} \cdot \vec{x}}
\,.
\end{align}

Let us suppose the equation reads 
%
\begin{subequations}
\begin{align}
\qty(E - p \sigma^{3}) u_R = \left[\begin{array}{cc}
E-p & 0 \\ 
0 & E+p
\end{array}\right] u_{R} = 0
\,,
\end{align}
\end{subequations}
%
so we must have two solutions: they look like 
%
\begin{subequations}
\begin{align}
\psi_{R} = \left[\begin{array}{c}
1 \\ 
0
\end{array}\right] e^{-i Et + i E x_3 }
\qquad \text{and} \qquad
\psi_{R} = \left[\begin{array}{c}
0 \\ 
1
\end{array}\right] e^{+i Et + i E x_3 }  
\,.
\end{align}
\end{subequations}

The solution \(\psi_{R} \sim \exp(-iEt + i E x_3)\) describes a right-handed electron with spin eigenvalue \(+1/2\) along the direction of motion (this is called the \emph{helicity} \cite[sec.\ 1.4.9]{tissinoTheoreticalPhysicsNotes2020})

On the other hand, the solution \(\psi_{R} \sim \exp(iEt + iE x_3)\) describes a right-handed electron with spin eigenvalue \(s = - 1/2\). 

Our quantum field operator \(\psi_{R}\) acts as on the right by destroying a right-handed electron:
%
\begin{align}
\bra{0}\psi_{R} \ket{e^{-}_{R} (p)} = u_R(p) e^{-ipx}
\,.
\end{align}

Then, we can have it acting on the left by destroying a left-handed positron:
%
\begin{align}
\bra{e^{+}_{L}(p)} \psi_{R} \ket{0} = v_L (p) e^{+ipx}
\,.
\end{align}

If we were to repeat the analysis for the other spinor, we would get the specular result. 

The Lagrangian can be written as 
%
\begin{align}
\mathscr{L} = 
\psi_R ^\dag \qty(i \sigma \cdot \partial) \psi_{R}+
\psi_L ^\dag \qty(i \overline{\sigma} \cdot \partial) \psi_{L}
- m \qty( \psi ^\dag _R \psi_{L}
+ \psi ^\dag_L \psi_{R})
\,,
\end{align}
%
so the coupling between the left and right handed fermions depends on the mass, if we are in a situation in which \(T \gg m\) they effectively decouple.


In order to describe this spin, we introduce the helicity quantum number, which is defined as 
%
\begin{align}
h = \hat{p} \cdot \vec{s}
\,,
\end{align}
%
the projection of the spin along the direction of motion. 
For \(\psi_{R}\), we have the \((1,0)\) state with helicity \(h = 1/2\), while the state \((0,1)\) is a positron with helicity \(h = - 1/2\). 

If \(m=0\), then helicity is exactly conserved. At high energies, it is suppressed by a factor \(m / E\). 

Let us compute the cross section: we must calculate  
%
\begin{align}
\bra{0} j^{\mu } \ket{e^{-}_{R} (p_{-}) e^{+}_{L} (p_+)} 
\,,
\end{align}
%
where we have a term \(j^{\mu } = \overline{\psi} \gamma^{\mu } \psi = \psi ^\dag \gamma^{0} \gamma^{\mu } \psi \); the term \(\gamma^{0} \gamma^{\mu }\) reads 
%
\begin{align}
\gamma^{0} \gamma^{\mu } = \left[\begin{array}{cc}
0 & \mathbb{1} \\ 
\mathbb{1} & 0
\end{array}\right]
\left[\begin{array}{cc}
0 & \sigma^{\mu } \\ 
\overline{\sigma}^{\mu } & 0
\end{array}\right] 
= \left[\begin{array}{cc}
\overline{\sigma}^{\mu } & 0 \\ 
0 & \sigma^{\mu }
\end{array}\right]
\,,
\end{align}
%
so we get 
%
\begin{align}
j^{\mu } = \psi_{L} ^\dag \overline{\sigma}^{\mu } \psi_{L} + \psi_{R} ^\dag \sigma^{\mu } \psi_{R}
\,.
\end{align}

We describe the process in the center-of-mass frame, and we choose to align the axes so that the electron and positron have momenta \(p^{\mu } = (E, 0, 0, \pm E)\) respectively (minus for the positron).

The wavefunction \(\psi ^\dag_{R}\) annihilates the positron \(e^{+}_{L}\) yielding a term \(v ^\dag _L (p_+)\), the wavefunction \(\psi_{R}\) annihilates the electron \(e^{-}_{R}\) yielding a term \(u_R (p_-)\). 

Then, we are left with  
%
\begin{align}
v_L ^\dag (p_+) \sigma^{\mu } u_R (p_-) &= \sqrt{2E} 
\left[\begin{array}{cc}
0 & 1
\end{array}\right]
(\mathbb{1}, \vec{\sigma}) \sqrt{2E}
\left[\begin{array}{c}
1 \\ 
0
\end{array}\right]  \\
&= 2E (0, 1, i, 0)^{\mu }
\,,
\end{align}
%
so if we define the vector \(\vec{\epsilon}_{+} = (\hat{1} + i \hat{2}) / \sqrt{2}\) we can write 
%
\begin{align}
\bra{0}
j^{\mu } 
\ket{e^{-}_R (p_-) e^{+}_L (p_+)}
= 2 E \sqrt{2} \qty(0, \vec{\epsilon}_{+})^{\mu }
\,,
\end{align}
%
while we would have 
%
\begin{align}
\bra{0}
j^{\mu } 
\ket{e^{-}_L (p_-) e^{+}_R (p_+)}
= - 2 E \sqrt{2} \qty(0, \vec{\epsilon}_{-})^{\mu }
\,,
\end{align}
%
with \(\vec{\epsilon}_{-} =(\hat{1} + i \hat{2}) / \sqrt{2} \). 

On the other hand, the terms \(e_{R}^{-} e_{R}^{+}\) and \(e_{L}^{-} e_{L}^{+}\) do not contribute (in our \(m = 0\) approximation).

The muons are massless fermions as well in our treatment, so we get analogous terms: 
%
\begin{align}
\bra{\mu^-_R (p'_-) \mu^+_L (p'_+)} j^{\mu } \ket{0} &= 2E \sqrt{2} \qty(0, \vec{\epsilon}^{\prime *}_+)^{\mu } \\
\bra{\mu^-_L (p'_-) \mu^+_R (p'_+)} j^{\mu } \ket{0} &= - 2E \sqrt{2} \qty(0, \vec{\epsilon}^{\prime *}_-)^{\mu }
\,,
\end{align}
%
so in the end we find 
%
\begin{align}
\mathcal{M} (e^{-}_{L} e^{+}_{R} \to \mu^-_R \mu^+_L) 
&= - \frac{e^2}{q^2} 2 (2E)^2 
\vec{\epsilon}^{\prime *}_{+} \cdot \vec{\epsilon}_{+}  \\
&= - 2 e^2 \vec{\epsilon}^{\prime *}_{+} \cdot \vec{\epsilon}_{+}
\,,
\end{align}
%
since \(q  =2 E\). 
The scalar product here depends on the direction of emission of the muons in the center of mass frame, \(\theta \); we find the absolute value \(\abs{\mathcal{M}}^2 = e^{4} (1 \pm \cos \theta )^2\), depending on whether we are looking at an \(LR \to LR\) process or \(LR \to RL\) process. The unpolarized (spin-averaged) differential cross section comes out to be: 
%
\begin{align}
\dv{\sigma }{\cos \theta } = \frac{1}{2} \frac{\pi \alpha^2}{2 E_{CM}^2} \qty(1 + \cos^2\theta )
\,,
\end{align}
%
which can be integrated across the sphere to get the total cross section for the process \(e^{-} e^{+} \to \mu^- \mu^+\): 
%
\begin{align}
\sigma = \frac{4 \pi }{3} \frac{\alpha^2}{E^2_{CM}}
\,.
\end{align}

We could have guessed the dependence on \(\alpha^2  / E^2_{CM}\), but for the numerical factor we needed to do the full computation. 
This is because the cross section is a length square, so it must depend on the inverse square of our only energy parameter, \(E_{CM}\). 

Also, the coupling was fixed: we are working in QED, so we only have a coupling constant: \(e\), so we will have terms \(e^2 / 4 \pi = \alpha \) inside of \(\mathcal{M}\), so we will get \(\alpha^2\) inside of \(\abs{\mathcal{M}}^2\). 
The \(4 \pi \)s will cancel because of the phase-space angular integrals.

Once we have done this, can we generalize it? suppose we want to compute the cross section \(\sigma (e^{+}e^{-} \to \text{hadrons})\). How could we do it?

We can make a similar kind of reasoning: the process will look like \(e^{-} e^{+} \to q \overline{q}\), and while the coupling for the first vertex in the Feynman diagram will be \(-e\) the one for the second vertex will look like \(Q_q\), the charge of these quarks.
We calculate the unpolarized cross section, counting all the quarks which can be produced at a fixed COM energy: we assume \(E \sim \SI{100}{GeV}\), so all the quarks except for the top are candidates; so the computation goes: 
%
\begin{align}
\frac{\sigma (e^{+} e^{-} \to q \overline{q})}{\sigma (e^{+ }e^{-} \to \mu^+ \mu^-)} \approx \sum _{q} Q_q^2 = \underbrace{2 \frac{4}{9} }_{u, c} + \underbrace{3 \frac{1}{9}}_{d, s, b}
= \frac{11}{9}
\,.
\end{align}

In experiments, however, we get a cross section ratio which is \(11/3\), 3 times larger than expected: this indicates that we have a different type of charge, color charge, which means we have a multiplicity of \(3\) for each quark. 

We shall see that this is related to certain kinds of internal symmetries of our field theory. 

\end{document}
