\documentclass[main.tex]{subfiles}
\begin{document}

\marginpar{Tuesday\\ 2020-4-21, \\ compiled \\ \today}

Now we have a dilemma: beauty versus pragmatism.

How do we provide a mass to the \(W\) vector boson?

% We could introduce an \(SU(2)\) symmetry. 

As we were discussing, we could add a mass term and write 
%
\begin{align}
\mathscr{L} = \mathscr{L} _{\text{Yang-Mills}} + M^2 W^{+}_{\mu } W^{\mu, -}
\,,
\end{align}
%
but this term \emph{brutally} breaks the symmetry.
This is explicit breaking of the symmetry. 

However, if we do not break the symmetry we cannot reproduce the data\dots or can we?

\subsection{Spontaneous Symmetry Breaking}

We introduce \textbf{Spontaneous Symmetry Breaking}. 
This is a phenomenon which is not exclusive to HEP: it happens in ferromagnets, for example.
They start off (above the Curie temperature) with the spins pointing in uniformly distributed directions: the situation is symmetric.
As we cool them, at a certain stage all the spins align.
We cannot predict the direction along which they will align (since the initial state is symmetric), but once they do the symmetry is \emph{spontaneously broken}.

Let us take the standard approach (also done in the Theoretical Physics course \cite[sec.\ 5.2]{tissinoTheoreticalPhysicsNotes2020}) introduce the Lagrangian of a complex scalar field \(\phi \): 
%
\begin{align}
\mathscr{L} = \qty(\partial^{\mu  } \phi^{*}(x)) \qty(\partial_{\mu } \phi (x)) 
\underbrace{- \mu^2 \abs{\phi (x)}^2
- \lambda^{4} \abs{\phi (x)}^{4}}_{- V(\phi )}
\,,
\end{align}
%
and to find the ground state we can move to the Hamiltonian: 
%
\begin{align}
\mathscr{H} = \pi(x) \phi (x) - \mathscr{L}
\,.
\end{align}

The parameter \(\lambda \) must be \(>0\), while \(\mu^2\) has no constraints.

It is a critical parameter: if \(\mu^2\) is positive we are in the symmetric phase of the system, and the vacuum is only \(\phi = 0\): the VEV is \(\bra{0} \phi \ket{0} = 0\). 

If, on the other hand, \(\mu^2\) is negative we transition to a new phase which is not symmetric: the vacuum becomes a whole circle, whose VEV is \(v = \sqrt{- \mu^2 / 2 \lambda }\). The whole region parametrized as \(v e^{i \theta }\), with \(\theta \in \mathbb{R}\), provides equivalent vacua:
a specific ground state is not symmetric under the whole symmetry of the Lagrangian.

Why do we only have terms in \(\phi^2\) and \(\phi^{4}\)? we basically constructed the simplest potential which has the properties we want.

If \(\mu^2>0\) we can perturb around the state without the quartic term, which is described by the KG equation.

If, instead, we want to perturb around a vacuum in the \(\mu^2<0\) case we need to distinguish two directions, since the Hessian of the potential has two different eigenvalues, one positive and one equal to zero. We parametrize 
%
\begin{align}
\phi = \frac{1}{\sqrt{2}} \qty(v + \sigma (x) + i\eta (x))
\,,
\end{align}
%
where \(\sigma \) and \(\eta \) are both real. 

The Lagrangian can then be rewritten as 
%
\begin{align}
\mathscr{L} = \frac{1}{2} \partial^{\mu } \sigma \partial_{\mu } \sigma 
+ \frac{1}{2} \partial^{\mu } \eta \partial_{\mu } \eta 
+ 2 \lambda v^2 \frac{1}{2} \sigma^2 
- \frac{\lambda}{4} \qty(\sigma^2 + \eta^2)^2
- \lambda v \sigma \qty(\sigma^2 + \eta^2)
\,,
\end{align}
%
so we have two real fields, one massive  (\(\sigma \)) and one massless (\(\eta \)). The massless field is called a \textbf{Goldstone boson}.

If we compute the VEV of \(\phi \) in this configuration we get     \(v / \sqrt{2 }\). 

In general, the Golstone theorem states that for each broken symmetry generator we get a massless boson. 

It seems that instead of solving the problem of our massless gauge bosons we have created another problem, predicting more massless particles; but in fact our two problems solve each other.

We must break a different symmetry, though: a gauge local symmetry, instead of a global one.
If we explicitly break a \(U(1)\) global symmetry, we have a Goldstone boson.
If we break a \(U(1)\) local symmetry, instead, we get a massless vector boson and a massless scalar.
This spontaneous breaking of local gauge symmetry is the \textbf{Higgs mechanism}.

\subsubsection{SQED Higgs mechanism}

We will see this explicitly, with the breaking of the scalar-QED \(U(1)\) symmetry: let us consider a Lagrangian 
%
\begin{align}
\mathscr{L} = \qty(\DD^{\mu } \phi )^{*} \DD_{\mu } \phi 
- \mu^2 \abs{\phi}^2 - \lambda \abs{\phi }^4
- \frac{1}{4} F^{\mu \nu } F_{\mu \nu }
\,,
\end{align}
%
where \(\DD_{\mu } = \partial_{\mu } + iq A_{\mu} \).


% The wonderful thing is that the two dof of the massless vector ad the single degree of freedom of the massless scalar couple to give a massive vector with three degrees of freedom.

% It is a ``transmutation'' of degrees of freedom.

\end{document}
