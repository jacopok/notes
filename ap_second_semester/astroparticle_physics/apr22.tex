\documentclass[main.tex]{subfiles}
\begin{document}

\marginpar{Wednesday\\ 2020-4-22, \\ compiled \\ \today}

% Last time we considered the spontaneous breaking of a \emph{global} symmetry: we saw the appearance, corresponding to the breaking of the \(U(1)\) symmetry, of massless Goldstone bosons.
% Each broken generator has a corresponding Goldstone boson.

% We had a quadratic term in the Lagrangian, and a quartic term. 

What would \(\mu^2<0\) physically mean? It would be a tachyonic particle.

The field \(\sigma (x)\) is called the \textbf{Higgs field}; the would-be Goldstone boson disappears yielding the longitudinal polarization of \(A_{\mu } (x)\). 
% The Vacuum Expectation Value goes from 0 to \(v\): in this case we have broken the \(U(1)\) symmetry.

% The imaginary part of the field \(\phi \) corresponds to a massless scalar field.

% Every time we have a certain global symmetry described by a group \(G\) with generators \(t^{a}\), which is broken to a subgroup \(G'\) (which can also be just the identity) with generators \(t^{i}\), we can identify the broken generators with Goldstone bosons.

% We have the Dirac equation \(\qty(i \slashed{\partial} - m) \psi =0\), where \(\psi \) has a global \(U(1)\) symmetry. 
% This can\emph{not} be generalized to a local \(U(1)\) symmetry, unless we introduce a compensating gauge field.

% Let us consider the Lagrangian from yesterday for the scalar field with a kinetic, quadratic and quartic term. Can we do \(\partial_{\mu } \to \DD_{\mu }\) as we did with QED, to generalize the \(U(1)\) symmetry of \(\phi \)?

% We can do it and encounter no issues. 

% What if \(\mu^2<0\)? The result is surprising: we do as before, perturbing around the vacuum \(\phi = v\), writing \(\phi = v + \sigma (x) + i \eta (x)\). 
% This is around pages 92--93 of the notes.

% We start from \(\phi \), which has 2 dof since it is a complex scalar, and \(A_{\mu }\), which also has 2 dof since it is a massless vector. 

% We would expect to get the fields \(\sigma \) and \(\eta \): however, the field \(\eta \) does not appear, it is ``eaten up'' by the vector field \(A_{\mu }\).
% We only find a massive real scalar \(\sigma (x)\), which has 1 dof, and a massive vector boson \(A_{\mu }\), which has 3 dof.

% This is the \emph{transmutation}, the scalar degree of freedom is absorbed to a degree of freedom in the vector field.

We are curing two different problems: we have found massive gauge bosons, and removed the unphysical Goldstone bosons.

\subsubsection{A more general formulation: \(SU(2)\) symmetry breaking}

This section follows Peskin pretty closely \cite[sec.\ 16.2]{peskinConceptsElementaryParticle2019}.

We can repeat this in the general case, with the group \(G\) being broken to \(G'\). Suppose, for clarity, that this is \(SU(2)\) being broken to \(U(1)\).

The three vectors \(A^{a}_{\mu }\) generate rotations around the axes \(x^{a}\) respectively, as \(a = 1, 2, 3\). 

The adjoint representation of \(SU(2)\) is given by three real scalar fields \(\phi^{a}\); their covariant derivative is given by 
%
\begin{align}
\DD_{\mu } \phi^{a} 
= \partial_{\mu } \phi^{a}
+ g \epsilon^{abc} A^{b}_{\mu } \phi^{c}
\,,
\end{align}
%
since the representation matrices in the adjoint representation are \((t^{b}_{G})_{ac} = i f^{abc}\). 

We want to choose a potential \(V(\phi )\) which is minimized by a configuration with \(\expval{\abs{\phi^{a}}} = v\). 
A vacuum for this potential is, for example, given by \(\phi^{a }= v \delta^{a3}\): it retains some rotational invariance (around the \(\hat{3}\) axis), but invariance for rotations around the other two axis is broken. 

A perturbative expansion around such a vacuum will then be given by 
%
\begin{align}
\phi (x) = \qty(\pi^{1} (x), \pi^{2}(x), v + h(x))
\,,
\end{align}
%
where \(\pi^{1, 2}\) are the would-be goldstone bosons, which will contribute to the longitudinal components of \(A^{1,2}_{\mu }\), the bosons which become massive. 

On the other hand, \(A^{3}_{\mu } \) remains massless. 
As we go to the unitary gauge we can eliminate \(\pi^{1, 2}\): so we get 
%
\begin{align}
\phi (x) = (0, 0, v + h(x))
\,.
\end{align}

We know that 
%
\begin{align}
\DD_{\mu } \phi^{a} 
&= \partial_{\mu } h \delta^{3a} 
+ g \epsilon^{abc} A^{b}_{\mu } \phi^{c}  \\
&= \partial_{\mu } h \delta^{3a}
+ g \epsilon^{ab3} A^{b}_{\mu } (v + h(x))
\,.
\end{align}

The kinetic term of \(\phi^{a}\) will then become: 
%
\begin{align}
\frac{1}{2} \qty(\DD_{\mu } \phi^{a})^2
&= \frac{1}{2} \DD_{\mu } \left[\begin{array}{ccc}
0 & 0 & v+h
\end{array}\right]
\DD^{\mu } \left[\begin{array}{c}
0 \\ 
0 \\ 
v+h
\end{array}\right]  \\
&= \frac{1}{2} \qty[
    g \epsilon^{ab3} A^{b}_{\mu } (v+h) 
    + \partial_{\mu } h \delta^{a3}
]^2  \marginnote{Square entails contraction of both \(a\) and \(\mu \).}\\
&= \frac{g^2 v^2}{2} \epsilon^{ab3} A^{b}_{\mu } \epsilon^{ad3} A^{d, \mu } + \mathcal{O}(h) \\
&= \frac{g^2 v^2}{2} \qty(A^{1}_{\mu } A^{\mu 1} + A^{2}_{\mu } A^{\mu 2 } ) + \mathcal{O}(h)
\,.
\end{align}

So, \(A^{1, 2}\) became massive, with mass \(M_W^2 = g^2 v^2\), while \(A^{3}\) remained massless.

As \(SU(2)\) is broken to \(U(1)\), the 3 generators \(A_{\mu }^{a}\), with \(a = 1, 2, 3\) are broken to give two massive vector bosons (\(W^{\pm}\)) and one massless vector boson (the photon), while the real field \(\sigma \) is the Higgs field.

The two \(W\) bosons are not actually the components \(A^{1, 2}\): instead, we choose them to be the eigenvalues of rotations around the \(z\) axis: 
%
\begin{align}
W^{\pm}_{\mu } = \frac{1}{\sqrt{2}} \qty(A^{1}_{\mu } \mp i A^{2}_{\mu }) 
\,,
\end{align}
%
since the rotation matrix around the \(z\) axis in the spin-1 representation is given by 
%
\begin{align}
J^{3} = \left[\begin{array}{ccc}
0 & -i & 0 \\ 
i & 0 & 0 \\ 
0 & 0 & 0
\end{array}\right]
\,,
\end{align}
%
whose eigenvectors are indeed \((1, \mp i, 0)\) with eigenvalues \(\pm 1\). 

This is called the \textbf{Georgi-Glashow model}. It predicts two massive bosons of the \(V-A\) interaction. 
However, it does not work well phenomenologically. 

\subsubsection{Hypercharge: why the GG model does not work}

This was a computationally easy example to clarify what this mechanism looks like, not the one we actually will use: we will have to choose another symmetry group.

The issue lies with the fact that we are associating electric charge with the generator of rotations about \(\hat{3}\).
We know that an electron current can be turned into an electronic neutrino current, and that the neutrino is electrically neutral.
They must belong to the same isospin multiplet in order for this to happen.

\todo[inline]{Isospin means the quantum number associated with \(SO(3)\), right?}

We cannot have a doublet with only \(\nu \) and \(e^{-}\) because of the fact that the neutrino is neutral: the eigenvalues of the rotation about the \(3\) axis would have to be \(0\) and \(1\).

\todo[inline]{and why does this not work? should the sum of the eigenvalues of the same multiplet always be zero?} 

We then predict the existence of a heavy electron \(E^{+}\), such that the triplet looks like 
%
\begin{align}
\left[\begin{array}{c}
E^{+} \\ 
\nu  \\ 
e^{-}
\end{array}\right]
\,,
\end{align}
%
where the eigenvalues of this rotation around the 
\todo[inline]{why?}
but this heavy electron was not observed; the experimental bound is at \SI{400}{GeV} currently. 

This will be a way to unify electromagnetic and weak interactions, and we will get to use a single coupling constant for our new electroweak theory.

\subsection{Electroweak symmetry breaking}

The correct symmetry group for the electroweak theory is 
%
\begin{align}
SU(2)_{L} \otimes U(1) _{\text{hypercharge}}
\,,
\end{align}
%
where the hypercharge is usually denoted as \(Y\).

We have a doublet under \(SU(2)\), we will see that one component of this doublet has charge \(+\) while the other has charge 0.

We have three vector fields for \(SU(2)\), which we call \(A_{\mu}^{i}\), and also a vector field for \(U(1)\): \(B_{\mu }\).

The VEV of our field will be 
%
\begin{subequations}
\begin{align}
\expval{\phi }= \left[\begin{array}{c}
0 \\ 
v
\end{array}\right]
\,.
\end{align}
\end{subequations}

The residual gauge symmetry can be identified with \(U(1)_{\text{em}}\). Do note that \(U(1)\)-hypercharge \emph{is} broken. However, a certain combination of the generators gives us the \(U(1)\) electromagnetic.
The combination is 
%
\begin{align}
T^{3} + Y
\,,
\end{align}
%
where \(T^{3}\) is the third generator of \(SU(2)\), while \(Y\) is the generator of hypercharge. 
A combination which is broken is 
%
\begin{align}
Q_{z} = T^{3} - \sin^2 (\theta_{w}) Q
\,,
\end{align}
%
where \(\theta_{w}\) is the Weak, or Wonder angle, defined by 
%
\begin{align}
\tan(\theta_{w}) = \frac{g'}{g}
\,,
\end{align}
%
where \(g'\) and \(g\) are the coupling constants relative to \(U(1)\) and \(SU(2)\) respectively.

The model is called the Glashow-Weinberg-Salam model. This is the Standard Model of the electroweak interaction.

How is this unification since we have two coupling constants? The constants are not the coupling constants of weak and electromagnetic interaction. The photon is given by
%
\begin{align}
A_{\mu } = \sin(\theta_{w}) A^{3}_{\mu } + \cos(\theta_{w}) B_{\mu }
\,,
\end{align}
%
while the orthogonal combination is 
%
\begin{align}
Z^{0}_{\mu } = \cos(\theta_{w}) A^{3}_{\mu } - \sin(\theta_{w}) B_{\mu }
\,,
\end{align}
%
so we cannot decouple the weak and electromagnetic bosons.
This is why the symmetry is called the electroweak symmetry.

When the symmetry is broken, we are left with a long-range interaction and a short range one.

If \(\mu^2\) is a function of \(T\), then we can get a phase transition.
This is called the electroweak phase transition.

Standard Model Lagrangian: 
%
\boxalign{
\begin{align}
\mathscr{L} = - \frac{1}{4} F_{\mu \nu } F^{\mu \nu }
+ i \overline{\psi} \slashed{\DD} \psi  
+ \psi_{i} y_{ij} \psi_{j} \phi 
+ \text{h. c.} 
+\abs{\DD_{\mu } \phi }^2 - V(\phi )
\,.
\end{align}}

We have not seen the \(y_{ij}\) bit yet, it is responsible for the fermion masses.

\end{document}
