\documentclass[main.tex]{subfiles}
\begin{document}

\marginpar{Wednesday\\ 2020-4-22, \\ compiled \\ \today}

% Last time we considered the spontaneous breaking of a \emph{global} symmetry: we saw the appearance, corresponding to the breaking of the \(U(1)\) symmetry, of massless Goldstone bosons.
% Each broken generator has a corresponding Goldstone boson.

% We had a quadratic term in the Lagrangian, and a quartic term. 

What would \(\mu^2<0\) physically mean? It would be a tachyonic particle. The Vacuum Expectation Value goes from 0 to \(v\): in this case we have broken the \(U(1)\) symmetry.

The imaginary part of the field \(\phi \) corresponds to a massless scalar field.

Every time we have a certain global symmetry described by a group \(G\) with generators \(t^{a}\), which is broken to a subgroup \(G'\) (which can also be just the identity) with generators \(t^{i}\), we can identify the broken generators with Goldstone bosons.

We have the Dirac equation \(\qty(i \slashed{\partial} - m) \psi =0\), where \(\psi \) has a global \(U(1)\) symmetry. 
This can\emph{not} be generalized to a local \(U(1)\) symmetry, unless we introduce a compensating gauge field.

Let us consider the Lagrangian from yesterday for the scalar field with a kinetic, quadratic and quartic term. Can we do \(\partial_{\mu } \to \DD_{\mu }\) as we did with QED, to generalize the \(U(1)\) symmetry of \(\phi \)?

We can do it and encounter no issues. 

What if \(\mu^2<0\)? The result is surprising: we do as before, perturbing around the vacuum \(\phi = v\), writing \(\phi = v + \sigma (x) + i \eta (x)\). 
This is around pages 92--93 of the notes.

We start from \(\phi \), which has 2 dof since it is a complex scalar, and \(A_{\mu }\), which also has 2 dof since it is a massless vector. 

We would expect to get the fields \(\sigma \) and \(\eta \): however, the field \(\eta \) does not appear, it is ``eaten up'' by the vector field \(A_{\mu }\).
We only find a massive real scalar \(\sigma (x)\), which has 1 dof, and a massive vector boson \(A_{\mu }\), which has 3 dof.

This is the \emph{transmutation}, the scalar degree of freedom is absorbed to a degree of freedom in the vector field.

We can repeat this in the general case, with the group \(G\) being broken to \(G'\). Suppose, for clarity, that this is \(SU(2)\) being broken to \(U(1)\). 

We are curing two different problems: we have found massive gauge bosons, and removed the unphysical Goldstone bosons.

As \(SU(2)\) is broken to \(U(1)\), the 3 generators \(A_{\mu }^{a}\), with \(a = 1, 2, 3\) are broken to give two massive vector bosons (\(Z^{\pm}\)) and one massless vector boson (the photon), while the real field \(\sigma \) is the Higgs field.

We are missing the \(Z\) boson: this is just an example to clarify what this mechanism looks like, it is not what we actually will use: we will have to choose another symmetry group.

This will be a way to unify electromagnetic and weak interactions, and we will get to use a single coupling constant for our new electroweak theory.

The correct symmetry group for the electroweak theory is 
%
\begin{align}
SU(2)_{L} \otimes U(1) _{\text{hypercharge}}
\,,
\end{align}
%
where the hypercharge is usually denoted as \(Y\).

We have a doublet under \(SU(2)\), we will see that one component of this doublet has charge \(+\) while the other has charge 0.

We have three vector fields for \(SU(2)\), which we call \(A_{\mu}^{i}\), and also a vector field for \(U(1)\): \(B_{\mu }\).

The VEV of our field will be 
%
\begin{subequations}
\begin{align}
\expval{\phi }= \left[\begin{array}{c}
0 \\ 
v
\end{array}\right]
\,.
\end{align}
\end{subequations}

The residual gauge symmetry can be identified with \(U(1)_{\text{em}}\). Do note that \(U(1)\)-hypercharge \emph{is} broken. However, a certain combination of the generators gives us the \(U(1)\) electromagnetic.
The combination is 
%
\begin{align}
T^{3} + Y
\,,
\end{align}
%
where \(T^{3}\) is the third generator of \(SU(2)\), while \(Y\) is the generator of hypercharge. 
A combination which is broken is 
%
\begin{align}
Q_{z} = T^{3} - \sin^2 (\theta_{w}) Q
\,,
\end{align}
%
where \(\theta_{w}\) is the Weak, or Wonder angle, defined by 
%
\begin{align}
\tan(\theta_{w}) = \frac{g'}{g}
\,,
\end{align}
%
where \(g'\) and \(g\) are the coupling constants relative to \(U(1)\) and \(SU(2)\) respectively.

The model is called the Glashow-Weinberg-Salam model. This is the Standard Model of the electroweak interaction.

How is this unification since we have two coupling constants? The constants are not the coupling constants of weak and electromagnetic interaction. The photon is given by
%
\begin{align}
A_{\mu } = \sin(\theta_{w}) A^{3}_{\mu } + \cos(\theta_{w}) B_{\mu }
\,,
\end{align}
%
while the orthogonal combination is 
%
\begin{align}
Z^{0}_{\mu } = \cos(\theta_{w}) A^{3}_{\mu } - \sin(\theta_{w}) B_{\mu }
\,,
\end{align}
%
so we cannot decouple the weak and electromagnetic bosons.
This is why the symmetry is called the electroweak symmetry.

When the symmetry is broken, we are left with a long-range interaction and a short range one.

If \(\mu^2\) is a function of \(T\), then we can get a phase transition.
This is called the electroweak phase transition.

Standard Model Lagrangian: 
%
\boxalign{
\begin{align}
\mathscr{L} = - \frac{1}{4} F_{\mu \nu } F^{\mu \nu }
+ i \overline{\psi} \slashed{\DD} \psi  
+ \psi_{i} y_{ij} \psi_{j} \phi 
+ \text{h. c.} 
+\abs{\DD_{\mu } \phi }^2 - V(\phi )
\,.
\end{align}}

We have not seen the \(y_{ij}\) bit yet, it is responsible for the fermion masses.

\end{document}
