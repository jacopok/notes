\documentclass[main.tex]{subfiles}
\begin{document}

\subsection{Symmetries of the SM}

\marginpar{Wednesday\\ 2020-4-29, \\ compiled \\ \today}

Let us write the Standard Model Lagrangian \emph{after} spontaneous symmetry breaking (so, at low energies): 
%
\begin{align}
\mathscr{L} = - \frac{1}{4} \sum _{a} \qty(F_{\mu \nu }^{a})^2
+ m_W^2 W^{+}_{\mu } W^{-, \mu }
+ \frac{1}{2} m_Z^2 Z_{\mu } Z^{\mu }
+ \sum _{f} \overline{\psi}_{f} \qty(i \slashed{\DD} - m_f) \psi_{f} 
+ \frac{1}{2} \qty(\partial_{\mu } h)^2 - V(h)
\,,
\end{align}
%
where \(a\) runs over all our \(8 + 3 + 1 = 12\) gauge bosons (8 gluons, the \(W^{\pm }\) and \(Z^{0}\), the photon), and the covariant derivative is given by: 
%
\begin{align}
\DD_{\mu f} = \partial_{\mu } - i e Q_f A_{\mu } 
- i \frac{g}{\cos \theta_{w}} Q_{Zf} Z_{\mu } 
- i g_s A^{a}_{\mu } t^{a}
\,.
\end{align}

\todo[inline]{Something that should have come up before: what's up with the \(W\) mass term? It does not look like a usual mass term\dots }

The weak force interactions are diagonal for the leptons, proportional to \(V_{CKM}\) for the quarks.

Let us discuss the remaining symmetries of the model. 
These are symmetries which are not imposed on the model by hand, instead they follow automatically from the terms which are allowed by the chosen gauge symmetries.

\subsubsection{Baryon and Lepton number conservation}

We define 
%
\begin{align}
B(\text{Quark}) = \frac{1}{3} \qquad \text{and} \qquad
B(\text{antiQuark}) = - \frac{1}{3}
\,,
\end{align}
%
while for any other particle we assign 0.
This is conserved, but we need not impose it: the most general Lagrangian we write with our symmetries has it.
This is a global symmetry: \(U(1)_{B}\). 

The way to find it is to construct a baryonic current \(j^{\mu }_{B}\), in such a way that it is conserved in our Lagrangian, then the integral of \(j^{0}_{B}\) will give the conserved charge.

This has heavy consequences: the lightest baryon is the proton. If it were to decay, this would not conserve baryon number. 

If we are allowed to violate baryon number conservation we can have a decay like \(p \to e^{+ } + \gamma \), which would occur and destabilize atoms.

We get a bound on the lifetime of the proton of 
%
\begin{align}
\tau _{\text{proton}} > \SI{e32}{yr}
\,.
\end{align}

We also have lepton number conservation.
This is more interesting since we have neutrinos, which are hard to detect.

We will see a method to give mass to neutrinos, which will however violate lepton number conservation.

\subsubsection{\(C\), \(P\) and \(T\) symmetries}

Strong and EM interaction conserve these three symmetries, while the coupling of the weak bosons to the fermions violate \(C\) and \(P\) maximally.
If the couplings were real-valued, \(CP\) symmetry would be preserved, but the \(V_{CKM}\) matrix has a physical phase, so this is not the case. 

By the \(CPT\) theorem, \(T\) is also violated.

\subsubsection{Flavour number conservation}

The only thing which couples different families of fermions together is the \(CKM\) matrix, which gives mixing terms between the quarks, as 
%
\begin{align}
\overline{u}_{i} \gamma^{\mu } (V_{CKM})_{ij} \phi_{j}
\,,
\end{align}
%
and the Cabibbo angle measures how much this happens; we then expect to see decays like \(u \to s + W^{+}\). 

For leptons, instead, we expect perfect conservation of lepton number since there is no mixing. Experimentally this seems to be quite well verified: if process like \(\mu \to e \gamma \) or \(\tau \to e \gamma \) happen they must do so extremely rarely. 

\subsection{Neutrinos}

The Yukawa Lagrangian is written as 
%
\begin{align}
\mathscr{L}_{\text{Yukawa}}
= - y_{e} L ^\dag_{a} \phi_{a} e_{R}
-y_{d} O ^\dag_{a} \phi_{a} d_{r}
-y_{a} O ^\dag_{a} \epsilon_{ab} \phi ^\dag_{b} u_{R} 
+ \text{h.c.}
\,.
\end{align}

Commonly this is written as \(\overline{\psi}_{L} \phi \psi_{R}\), this however does not mean \(\overline{\psi} = \psi ^\dag \gamma^{0}\), since it is applied to a 2-component spinor. 

The index \(a\) is an SU(2) index, and the Higgs field looks like 
%
\begin{subequations}
\begin{align}
\phi = \left[\begin{array}{c}
\phi^{+} \\ 
\phi^{0}
\end{array}\right]
\,;
\end{align}
\end{subequations}
%
we have two terms like 
%
\begin{subequations}
\begin{align}
\left[\begin{array}{cc}
\nu_{L} & e^{-}_{L}
\end{array}\right]
\left[\begin{array}{c}
\phi^{+} \\ 
\phi^{0}
\end{array}\right]
e_{R}
\,,
\end{align}
\end{subequations}
%
and also 
%
\begin{subequations}
\begin{align}
\left[\begin{array}{cc}
\nu_{R}^{c} & e^{+}_{R}
\end{array}\right]
\left[\begin{array}{c}
\phi^{+} \\ 
\phi^{0}
\end{array}\right]
e_{R}^{-}
\,.
\end{align}
\end{subequations}

This means that the neutrino is massless: the VEV of \(\phi_{a}\) is 0 for \(\phi^{+}\), and \(v\) for \(\phi^{0}\). 
So, we get Dirac mass terms for the electron but not for the neutrino.

Ettore Majorana proposed a way to give a mass for the neutrino (or any neutral fermion, really).

% The symmetries of the SM are, generally, \(\text{Lorentz} \otimes \text{Gauge}\).
% We then write all the terms we can which are invariant under these. 

% From these we can derive conservation laws, such as that for the Baryon number. 

Parity and charge are massively violated.

CP symmetry moves us from \(e^{-}_{L}\) to \(e^{+}_{L}\) to \((e^{+})_{R}\).

If we have CP symmetry, then if in our model we have a \(\nu_{L}\) we will also need to have a \(\nu^{c}_{R}\).
That is, there are only two degrees of freedom between them.

In the SM we introduce only left-handed neutrinos. However, we can also introduce 
%
\begin{align}
\nu_{R} \iff (\nu^{c})_{L}
\,.
\end{align}

The beta decay is always 
%
\begin{align}
n \to p + e^{-} + \overline{\nu}_{e}
\,,
\end{align}
%
and we look at the spectrum of the energies of the electron. 

If the neutrinos are massless the curve gets to a certain point, if they are massive the curve stops a bit earlier. 
People have done this to a very high degree of precision.

What we have found up to now is \(m_{\nu_{e}} < \SI{2}{eV}\) in these beta decay experiments. 
The result that the neutrinos are massive did not come from here. 

From cosmic ray interactions we have charged pions, which decay into 
%
\begin{align}
\pi^{+} \to \mu^+ + \nu_{\mu }
\,.
\end{align}

Then, the muon decays into 
%
\begin{align}
\mu^+ \to e^{+} + \nu_{e} + \overline{\nu}_{\mu }
\,.
\end{align}

So, we expect twice as many \(\nu_{\mu }\) as \(\nu_{e}\).

We can build detectors in order to see these. We expect a muonic neutrino and antineutrino. 

The state \(\nu_{\mu }\) is a charge eigenstate, not a mass eigenstate: in general it will be written as \(a \nu_1 + b n_2 +c \nu_3 \).

Its time-evolution will then exhibit an oscillation in the probability to still find a \(\nu_{\mu }\), since the mass components oscillate at different frequencies.

The experiment OPERA found that some of the neutrinos from CERN going to Gran Sasso were converted to \(\nu_{\tau }\) and such. 

Also, Kamiokande found a ratio different from 2 of \(\nu_{\mu }\) to \(\nu_{e}\).

So, if there is an oscillation it means that the neutrinos must indeed be massive, or at least some of them must be.

We put a Dirac mass term for the \(\nu_{R}\): the mass will look like 
%
\begin{align}
m_{\nu } =  y_{\nu } v
\,.
\end{align}

But we know that the masses of the neutrinos are very small, from beta decay but also from structure formation in the early universe.

So, since \(v \sim \SI{100}{GeV}\) and \(m_{\nu } \lesssim \SI{1}{eV}\) we must have \(y_{\nu } \sim \num{e-11}\).

The Dirac mass term is 2+2 degrees of freedom. 

If \(L\) is a lepton doublet, we can write a term \(L \phi \).

However, \(L\) has an index: but we can consider a term 
%
\begin{align}
L^{i} \phi L^{j} \phi \epsilon_{ij}
\,.
\end{align}

We can do this for the neutrinos but not for the other particles since they are uncharged. 

So the term looks like 
%
\begin{align}
\nu_{L} \phi^{0} \nu_{L} \phi^{0}
\,.
\end{align}

When we compute the VEV we get a mass for the neutrino. 

This is for the left-handed neutrino. The dimension of this term is 5, since the fermions have dimension \(3/2\), while the Higgs has dimension 1: we should put a mass term in the denominator.
We will then have a term which looks like 
%
\begin{align}
y^2_{\nu } \frac{L L \phi^2}{M}
\,,
\end{align}
%
so that in the end \(m_{\nu } = y^2_{\nu } v^2 / M\).

This \(M\) must be very large in order to account for the small neutrino mass.

This is a sort of seesaw model: the mass matrix looks something like 
%
\begin{subequations}
\begin{align}
\left[\begin{array}{cc}
0 & m _{\text{Dirac}} \\ 
m _{\text{Dirac}} & M
\end{array}\right]
\,.
\end{align}
\end{subequations}

The thing is: \(\nu_{R}\) has no gauge numbers. 
The mass of the right-handed neutrino is not ``protected'' by the gauge symmetry, it can be as large as it likes. 

% If neutrinos possess a Majorana mass, 

It is important to find out whether the mass of the neutrino is due to a Majorana or Dirac mass term.

The experiments which can determine this are called \emph{neutrinoless double beta decay}.

If this were found, it would mean that neutrinos have a Majorana mass.

This concludes the discussion of the particle physics standard model. Next week we are going to start looking at the standard model of cosmology.

\end{document}
