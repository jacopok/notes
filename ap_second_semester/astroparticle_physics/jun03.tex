\documentclass[main.tex]{subfiles}
\begin{document}

\section{Baryogenesis}

\subsection{Electroweak baryogenesis}

\marginpar{Wednesday\\ 2020-6-3, \\ compiled \\ \today}

The classical Lagrangian is invariant under \(q \to e^{i \alpha } q\), which gives us the classical \(B\)-number conservation; and it is also invariant under \(\ell_i \to e^{i \beta_{i}} \ell_i\), where \(i\) labels the leptons' generations. 

This gives us conservation of the three lepton numbers. 
This holds as long as the neutrinos are massless, if they are not we get a single symmetry \(U(1)_L\), where \(L = L_e + L_\mu + L_c\). 

However, at the quantum level the currents associated with these symmetries are anomalous: we have 
%
\begin{align}
\partial_{\mu } j^{\mu B} = 3 \frac{g_s^2}{32 \pi^2} W^{\mu \nu , a} \widetilde{W}_{\mu \nu }^{a} 
\qquad \text{and} \qquad
\partial_{\mu } j^{\mu L} = \frac{g_s^2}{32 \pi^2} W^{\mu \nu , a} \widetilde{W}_{\mu \nu }^{a} 
\,,
\end{align}
%
where 
%
\begin{align}
W^{a}_{\mu \nu } = 2 \partial_{[\mu } W^{a}_{\nu ]} + g \epsilon^{abc} W^{b}_{\mu } W^{c}_{\nu }
\,
\end{align}
%
is the field strength tensor of the \(SU(2)_L\) field and \(\widetilde{W}\) is the Hodge dual: 
%
\begin{align}
\widetilde{W}^{a}_{\mu \nu } = \frac{1}{2} \epsilon_{\mu \nu \rho \sigma} W^{\rho \sigma , a} 
\,.
\end{align}

This is a result which holds for any axial current \(j^{\mu }_{A} \sim \overline{\psi} \gamma^{\mu } \gamma^{5} \psi \): we can write its divergence as proportional to the norm of the field strength. 

Now, for the strong interaction gluons interact equally with left-handed and right-handed charges. 
On the contrary, for the weak interaction only the left-handed charges interact with the bosons, so neither \(B\) nor \(L\) are conserved. 

Geometrically, this is connected to the fact that 
%
\begin{align}
\int W \widetilde{W} \dd[4]{x} = \int \dd{S_{\mu }} K^{\mu } \neq 0
\,,
\end{align}
%
where \(W_{\mu \nu } \widetilde{W}^{\mu \nu } = \partial_{\mu } k^{\mu }\).

For an abelian symmetry, the integral always vanishes, while for a nonabelian one it can be nonzero: there is a ``winding number'' 
%
\begin{align}
\nu = - \frac{i g_2^3}{24 \pi^2} \int \dd{S_\mu } \epsilon_{\mu \nu \rho \sigma } \Tr[A_{\nu }A_{\rho }A_{\sigma }]
\,.
\end{align}

We have topologically distinct vacua: \emph{sphalerons}. \emph{Instantons} (nonpertubative ``tunneling'' solutions) can go between them. 

Even in a process with \(\Delta \nu \neq 0\), though, we have \(\Delta L = \Delta B\), so \(L-B\) is always conserved. 

The height of the potential barrier is of the order \(E _{\text{sph}} \sim M_W / g_2^2\). 

At zero temperature, transmission is only possible through tunneling, in which case we have \(\Gamma \sim e^{-4 \pi / \alpha_{w}}\), so \(\Gamma \sim \num{e-165}\): practically zero. 

However, at nonzero temperature we have 
%
\begin{align}
\Gamma _{\text{sph}} \sim T^{4} \exp( - \frac{M_W(T)}{\alpha_{w} T})
\,,
\end{align}
%
but if \(T > v\) we go into the unbroken electroweak regime, when \(M_W(T ) = 0\). So, at high energies \(\Gamma _{\text{sph}} \sim \alpha_{w} T^{4}\). 

\todo[inline]{Where did that \(\alpha_{w}\) come from? }

The electroweak \(B\)-breaking processes are in thermal equilibrium if the rate per particle \(\Gamma _{\text{sph}} / n \sim \Gamma _{\text{sph}} / T^3 \sim \alpha_w^{4} T\) is larger than \(H(T) = T^2 / M_P^{*}\). 


So, sphalerons are in equilibrium as long as 
%
\begin{align}
M_W \sim \SI{100}{GeV} < T < \SI{e12}{GeV} \sim \alpha_w^{4}M_P^{*}
\,.
\end{align}



\subsection{Leptogenesis}

\end{document}
