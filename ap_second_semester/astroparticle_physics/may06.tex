\documentclass[main.tex]{subfiles}
\begin{document}

\section{Thermodynamics of the early universe}

\marginpar{Wednesday\\ 2020-5-6, \\ compiled \\ \today}

We start off by discussing particles in thermal equilibrium, but the interesting thing is the transition between this equilibrium and non-equilibrium.

We use the usual approximation of a dilute gas with \(g\) degrees of freedom, we take a Boltzmann distribution function in momentum space \(f(p)\), from which we can compute \(n\), \(\rho \) and \(P\): 
%
\begin{align}
n &= \frac{g}{(2\pi )^3} \int f(\vec{p}) \dd[3]{p} \\
\rho  &= \frac{g}{(2\pi )^3} \int E(\vec{p}) f(\vec{p}) \dd[3]{p} \\
P  &= \frac{g}{(2\pi )^3} \int \frac{\abs{p}^2}{3E} f(\vec{p}) \dd[3]{p} 
\,.
\end{align}

In thermal equilibrium the distribution function in momentum space is 
%
\begin{align}
f(\vec{p}) = \frac{1}{\exp(\frac{E-\mu }{T}) \pm 1}
\,,
\end{align}
%
where we have \(+\) for fermions and \(-\) for bosons.
The energy as a function of momentum is given by 
%
\begin{align}
E(\vec{p}) = \sqrt{\abs{\vec{p}}^2 + m^2}
\,.
\end{align}

We can make some useful approximations in the relativistic limit \(T \gg m\) and in the nonrelativistic limit \(T \ll m\): 
in the \textbf{relativistic} limit we find 
%
\begin{align}
\rho = 
\begin{cases}
  \displaystyle \frac{\pi^2}{30 } g T^{4} &  \text{bosons} \\
  \displaystyle \frac{7}{8} \frac{\pi^2}{30 } g T^{4} &  \text{fermions}
\end{cases}
\qquad \text{and} \qquad
n = 
\begin{cases}
  \displaystyle \frac{\zeta (3) g T^{3}}{\pi^2} &  \text{bosons} \\
  \displaystyle \frac{3}{4}\frac{\zeta (3) g T^{3}}{\pi^2} &  \text{fermions},
\end{cases}
\end{align}

where \(\zeta \) is the Riemann zeta function, so that \(\zeta (3) = \sum _{n \in \mathbb{N}} n^{-3} \approx \num{1.2}\). 
Since we are in the ultrarelativistic approximation, we will have \(P  = \rho /3\).

In the \textbf{nonrelativistic} limit, instead, we get 
%
\begin{align}
n = g \qty(\frac{mT}{2 \pi })^{3/2} e^{-(m - \mu ) / T}
\,,
\end{align}
%
while \(\rho = m n\) and \(P = n T \ll \rho \).

\todo[inline]{Add reference to Fundamentals notes, chapter 3, when it will be done.}

A very useful parameter to define in this context is \(g_{*}\): it is the \emph{effective} number of degrees of freedom for different particle species, computed as 
%
\begin{align}
g_{*} = \sum _{i \in \text{bosons}} g_{i} \qty(\frac{T_{i}}{T})^{4} + \frac{7}{8} \sum _{i \in \text{fermions}} g_{i} \qty( \frac{T_i}{T})^{4}
\,,
\end{align}
%
since we need to weigh the contributions to the degrees of freedom according to whether the particles are in equilibrium or not, and when the particle decoupled.
With this definition, in the relativistic limit we can simply write 
%
\begin{align}
\rho_{R} = \sum _{i} \rho_{i} =  \frac{\pi^2}{30 } g_{*} T^{4}
\qquad \text{and} \qquad
P_R = \frac{\pi^2}{90 } g_{*} T^{4}
\,.
\end{align}

\subsubsection{Effective number of degrees of freedom examples}

Let us compute \(g_{*}\) in a couple of examples. First, let us consider \(T \ll \SI{}{MeV}\) --- but at a time in which the photons are still coupled, that is, \emph{before} the emission of the CMB. 

So, in the standard model the only decoupled particles are photons and the three neutrino species.
The temperature of the neutrinos, as we will see later, is given in this time period by 
%
\begin{align}
T_{\nu } = \sqrt[3]{ \frac{4}{11}} T_{\gamma }
\,.
\end{align}

So, since: photons have two physical polarizations and so do (left-handed) neutrinos, there are three neutrino species and neutrinos are fermions we get
%
\begin{align}
g_{*} (T\ll\SI{}{MeV}) = 2 + \frac{7}{8} \times 3 \times 2 \times \qty( \frac{4}{11})^{4/3} \approx 3.37
\,.
\end{align}

If we consider a temperature around \(\SI{1}{MeV} < T < \SI{100}{MeV}\), we also will have to account for electrons and positrons. Now, the temperature of all these species can be taken to be equal: so, we get 
%
\begin{align}
g_{*} (\SI{1}{MeV} < T < \SI{100}{MeV}) \approx 2 + \frac{7}{8} \qty(3 \times 2 + 1 \times 4) \approx \num{10.75}
\,.
\end{align}

For \(T > \SI{200}{GeV}\), all the SM particles will be relativistic, so we will find \(g_* \approx \num{106.75}\). 
For a nice figure summarizing the evolution of \(g_*\) as the temperature decreases see figure 1 in a recent paper by Lars Husdal \cite[]{husdalEffectiveDegreesFreedom2016}.

\subsubsection{The radiation domination epoch}

This was the period in the early universe in which \(\rho_{R} \) was larger than \(\rho_{M}\); we will consider the early phase in which this difference was large (in order to approximate \(\rho_{M} \approx 0\)), but the inequality stayed on the radiation side for about a millennium.

In the era of radiation domination, we get 
%
\begin{align}
a(t) \propto \sqrt{t}
\,,
\end{align}
%
and the Hubble parameter scales like 
%
\begin{align}
H \approx 166 \sqrt{g_{*}} \frac{T^2}{M_{\text{Pl}}} = \frac{T^2}{M _{\text{Pl}}}
\,.
\end{align}

So, we get a time of the order of 
%
\begin{align}
t \approx \qty( \frac{T}{\SI{}{MeV}})^2 \SI{}{s}
\,.
\end{align}

How do we quantitatively describe coupling? We take the rate \(\Gamma_{i}^{\nu }\) from our particle to another, considering all possible processes.

Let us suppose that \(T > M_W\), so roughly \(T > \SI{100}{GeV}\).

The rate is given by 
%
\begin{align}
\Gamma_{\nu } = \sigma n v 
\,,
\end{align}
%
and the physics is really given by the cross section \(\sigma \): the other factors are always roughly the same, \(n \sim T^{3}\) and \(v \sim 1\).

We have the coupling constant 
%
\begin{align}
\frac{e^2}{4 \pi } = \alpha _{\text{em}}
\,,
\end{align}
%
and the charge is \(e = g \sin(\theta_{w})\), where we know experimentally that \(\sin^2\theta_{w} \approx \num{.24}\).

So, we know that the scaling is 
%
\begin{align}
\sigma \sim \qty( \frac{g^2_{w}}{4 \pi })^{2}
\sim \qty(\alpha^2_{w})
\,,
\end{align}
%
and on the denominator we need a mass square: but the only energy scale we know of here is \(T\), so we finally get \(\sigma \sim \alpha^2_{w} / T^2\). So, in the end we find 
%
\begin{align}
\sigma \sim \frac{\alpha^2_{w}}{T^2} T^3 \times 1 \sim \alpha^2_{w} T
\,.
\end{align}

The result we get is: \(T > H\) when 
%
\begin{align}
T < \frac{\alpha^2_{w }M _{\text{Pl}}}{\sqrt{g_{*}}}
\,.
\end{align}

In another case (which?) we have \(\sigma \sim T^{2}\), so \(\Gamma \sim T^{5}\). 

We must compare to the expansion of the universe. 

The end result is \(\Gamma_{\nu } > H\) as long as \(T > \SI{1}{MeV}\), roughly. So, for larger temperatures the neutrinos are coupled, for smaller temperatures they are decoupled. Below \SI{1}{MeV} they completely decouple. 

The ``relic neutrinos'' are colder than the relic photons. Moreover, their number density is smaller. 
The entropy density is given by 
%
\begin{align}
s = \frac{\rho + P}{T} = \frac{4}{3} \frac{\rho}{T}
\,.
\end{align}

The temperature of neutrinos today is around \(T^{0}_{\nu } \approx \SI{1.96}{K}\). 

Next week we will consider nucleosynthesis. 

\end{document}
