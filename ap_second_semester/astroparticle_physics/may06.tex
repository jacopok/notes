\documentclass[main.tex]{subfiles}
\begin{document}

\section{Thermodynamics of the early universe}

\marginpar{Wednesday\\ 2020-5-6, \\ compiled \\ \today}

We start off by discussing particles in thermal equilibrium, but the interesting thing is the transition between this equilibrium and non-equilibrium.

We use the usual approximation of a dilute gas with \(g\) degrees of freedom, we take a Boltzmann distribution function in momentum space \(f(p)\), from which we can compute \(n\), \(\rho \) and \(P\): 
%
\begin{align}
n &= \frac{g}{(2\pi )^3} \int f(\vec{p}) \dd[3]{p} \\
\rho  &= \frac{g}{(2\pi )^3} \int E(\vec{p}) f(\vec{p}) \dd[3]{p} \\
P  &= \frac{g}{(2\pi )^3} \int \frac{\abs{p}^2}{3E} f(\vec{p}) \dd[3]{p} 
\,.
\end{align}

In thermal equilibrium the distribution function in momentum space is 
%
\begin{align}
f(\vec{p}) = \frac{1}{\exp(\frac{E-\mu }{T}) \pm 1}
\,,
\end{align}
%
where we have \(+\) for fermions and \(-\) for bosons.
The energy as a function of momentum is given by 
%
\begin{align}
E(\vec{p}) = \sqrt{\abs{\vec{p}}^2 + m^2}
\,.
\end{align}

We can make some useful approximations in the relativistic limit \(T \gg m\) and in the nonrelativistic limit \(T \ll m\): 
in the \textbf{relativistic} limit we find 
%
\begin{align} \label{eq:density-number-density-relativistic-limit}
\rho = 
\begin{cases}
  \displaystyle \frac{\pi^2}{30 } g T^{4} &  \text{bosons} \\
  \displaystyle \frac{7}{8} \frac{\pi^2}{30 } g T^{4} &  \text{fermions}
\end{cases}
\qquad \text{and} \qquad
n = 
\begin{cases}
  \displaystyle \frac{\zeta (3) g T^{3}}{\pi^2} &  \text{bosons} \\
  \displaystyle \frac{3}{4}\frac{\zeta (3) g T^{3}}{\pi^2} &  \text{fermions},
\end{cases}
\end{align}

where \(\zeta \) is the Riemann zeta function, so that \(\zeta (3) = \sum _{n \in \mathbb{N}} n^{-3} \approx \num{1.2}\). 
Since we are in the ultrarelativistic approximation, we will have \(P  = \rho /3\).

In the \textbf{nonrelativistic} limit, instead, we get 
%
\begin{align}
n = g \qty(\frac{mT}{2 \pi })^{3/2} e^{-(m - \mu ) / T}
\,,
\end{align}
%
while \(\rho = m n\) and \(P = n T \ll \rho \).

\todo[inline]{Add reference to Fundamentals notes, chapter 3, when it will be done.}

A very useful parameter to define in this context is \(g_{*}\): it is the \emph{effective} number of degrees of freedom for different particle species, computed as 
%
\begin{align}
g_{*} = \sum _{i \in \text{bosons}} g_{i} \qty(\frac{T_{i}}{T})^{4} + \frac{7}{8} \sum _{i \in \text{fermions}} g_{i} \qty( \frac{T_i}{T})^{4}
\,,
\end{align}
%
since we need to weigh the contributions to the degrees of freedom according to whether the particles are in equilibrium or not, and when the particle decoupled.
With this definition, in the relativistic limit we can simply write 
%
\begin{align}
\rho_{R} = \sum _{i} \rho_{i} =  \frac{\pi^2}{30 } g_{*} T^{4}
\qquad \text{and} \qquad
P_R = \frac{\pi^2}{90 } g_{*} T^{4}
\,.
\end{align}

\subsubsection{Effective number of degrees of freedom examples}

Let us compute \(g_{*}\) in a couple of examples. First, let us consider \(T \ll \SI{}{MeV}\) --- but at a time in which the photons are still coupled, that is, \emph{before} the emission of the CMB. 

So, in the standard model the only decoupled particles are photons and the three neutrino species.
The temperature of the neutrinos, as we will see later, is given in this time period by 
%
\begin{align}
T_{\nu } = \sqrt[3]{ \frac{4}{11}} T_{\gamma }
\,.
\end{align}

So, since: photons have two physical polarizations and so do (left-handed) neutrinos, there are three neutrino species and neutrinos are fermions we get
%
\begin{align}
g_{*} (T\ll\SI{}{MeV}) = 2 + \frac{7}{8} \times 3 \times 2 \times \qty( \frac{4}{11})^{4/3} \approx 3.37
\,.
\end{align}

If we consider a temperature around \(\SI{1}{MeV} < T < \SI{100}{MeV}\), we also will have to account for electrons and positrons. Now, the temperature of all these species can be taken to be equal: so, we get 
%
\begin{align}
g_{*} (\SI{1}{MeV} < T < \SI{100}{MeV}) \approx 2 + \frac{7}{8} \qty(3 \times 2 + 1 \times 4) \approx \num{10.75}
\,.
\end{align}

For \(T > \SI{200}{GeV}\), all the SM particles will be relativistic, so we will find \(g_* \approx \num{106.75}\). 
For a nice figure summarizing the evolution of \(g_*\) as the temperature decreases see figure 1 in a recent paper by Lars Husdal \cite[]{husdalEffectiveDegreesFreedom2016}.
The take-away is that, going from the \SI{}{TeV} to the \SI{}{keV}, if we plot \(g_*\) on a log scale, it has sharp drop-offs around \SI{200}{MeV} (a QCD transition) and around \SI{500}{keV} (electrons and positrons), while it is mostly flat elsewhere.

\subsubsection{The radiation domination epoch}

This was the period in the early universe in which \(\rho_{R} \) was larger than \(\rho_{M}\); we will consider the early phase in which this difference was large (in order to approximate \(\rho_{M} \approx 0\)), but the inequality stayed on the radiation side for about a millennium.

In the era of radiation domination the scale factor varies like 
\(a(t) \propto \sqrt{t} \)
and the Hubble parameter scales like 
\begin{align}
H \approx \num{1.66} \sqrt{g_{*}} \frac{T^2}{M_{\text{Pl}}}
\,.
\end{align}

So, we get a relation between time and temperature: approximately it is 
%
\begin{align}
t \approx \qty( \frac{T}{\SI{}{MeV}})^{-2} \SI{}{s}
\,,
\end{align}
%
and more precisely it is 
%
\begin{align}
t T^2 = \sqrt{ \frac{90}{32 \pi^2} \frac{1}{G_N g_{*}}}
\,.
\end{align}

\subsection{Thermal equilibrium}

If the universe were not expanding, after a certain amount of time we would reach thermal equilibrium between the particle species. 
But this is not the case: the expansion rate of the universe is \(H\).

So, we take the decay rate of a certain particle species \(i\), which is denoted by \(\Gamma_{i}\).
A particle is said to reach equilibrium if its decay rate is \(\Gamma_{i} > H\).
This is a rule of thumb, an order of magnitude estimate: if \(\Gamma_{i} \gg H\) then the particle has time to decay many times within the age of the universe, while if \(\Gamma_{i} \ll H\) then its decay curve has barely started since the beginning of the universe.
We are doing cosmology so the precise details of what happens when \(\Gamma_{i} \approx H\) are not very important, that is the region in which a macroscopic fraction of our particle species has has time to decay but not all of it.

Note that this is a time- (or temperature-) dependent condition: a particle may be in equilibrium at a specific time and go out of equilibrium later. 

\subsubsection{Neutrino decoupling: the \emph{freeze-out}}

We want to discuss processes involving the neutrino, which are things like \(e^{+} e^{-} \to \nu \overline{\nu}\) or \(e \nu \to e \nu \).

This are processes mediated by the weak interaction, so it is relevant to ask whether the temperature is larger or smaller than the mediator's mass.
Let us first suppose that \(T > M_W\), so roughly \(T > \SI{100}{GeV}\).
This means that the Breit-Wigner term in the cross section can be approximated as 
%
\begin{align}
\frac{1}{T^2 - M_W^2} \sim \frac{1}{T^2}
\,.
\end{align}

The rate of these processes is given by 
%
\begin{align}
\Gamma_{\nu } = \sigma n v 
\,,
\end{align}
%
where the physics is really given by the cross section \(\sigma \): the other factors are always roughly the same, \(n \sim T^{3}\) and \(v \sim 1\).
The cross section depends on the coupling and the temperature: the coupling constant for weak processes is \(\alpha_{w} = g_w^2 / 4 \pi \), and the cross section is given by \(\sigma \sim \alpha_{w}^2 / T^2\),\footnote{The factor which appears in the cross section is precisely the Breit-Wigner one.} so the decay rate is roughly
%
\begin{align}
\Gamma \sim \frac{\alpha^2_{w}}{T^2} \times  T^3 \times 1 = \alpha^2_{w} T
\,.
\end{align}

On the other hand, the Hubble rate is roughly proportional to 
%
\begin{align} \label{eq:hubble-rate-temperature-scaling}
H \sim \sqrt{g_{*}} \frac{T^2}{M_P}
\,,
\end{align}
%
so if we impose \(\Gamma_{\nu } > H\) we get  
%
\begin{align}
T < \frac{\alpha^2_{w} M_{P}}{\sqrt{g_{*}}}
\,.
\end{align}

So, our result is that for 
%
\begin{align}
M_W < T < \frac{\alpha^2_{w} M_P}{\sqrt{g_{*}}}
\,
\end{align}
%
neutrinos are in thermal equilibrium.
In terms of actual energies, this means \(\SI{e2}{GeV} < T < \SI{e16}{GeV}\). The higher end of this is almost at the Planck mass, so we will not worry about it.

Now let us consider the later time at which \(T < M_W\). In this case, in the Breit-Wigner the mass of the weak interaction boson will dominate, and we will have 
%
\begin{align}
\sigma \sim \mathcal{O}(\num{e-2}) \frac{T^2}{M_W^4} \approx \num{e-10} \frac{T^2}{\SI{}{GeV}^{4}}
\,.
\end{align}
\todo[inline]{What is the factor which is of the order 0.01?}

Now, since we have \(\sigma \sim T^{2}\) the rate will scale like \(\Gamma \sim T^{5}\). Specifically, 
%
\begin{align}
\Gamma \sim \num{e-10} \frac{T^{5}}{\SI{}{GeV^{4}}}
\,.
\end{align}

We must compare to the expansion rate of the universe, which as before scales according to \ref{eq:hubble-rate-temperature-scaling}. Then, we find 
%
\begin{align}
\num{e-10} \frac{T^{5}}{\SI{}{GeV^{4}}} &> \sqrt{g_{*}} \frac{T^2}{\SI{e19}{GeV}}  \\
T &> \sqrt[3]{\frac{\sqrt{g_{*}} \num{e10}}{\num{e19}}} \SI{}{GeV} 
= \sqrt[6]{g_{*}} \SI{}{MeV}
\,.
\end{align}

Now, \(g_{*}^{1/6} \sim 1\), so we find that for \(\SI{1}{MeV} < T < M_W\) neutrinos are coupled. 

The end result is \(\Gamma_{\nu } > H\) as long as \(T > \SI{1}{MeV}\), roughly.
So, for larger temperatures the neutrinos are coupled, for smaller temperatures they are decoupled. Below \(T_D \approx\SI{1}{MeV}\) they completely decouple. 
At this temperature the universe becomes transparent for them, they no longer significantly interact with matter.
The decoupling temperature is defined by the relation given
%
\begin{align}
\Gamma_{\nu } \qty(T^{\nu }_{D}) = H(T^{\nu }_{D})
\,.
\end{align}

Now, we can define the current temperature and number density of the leftover neutrinos: they are respectively \(T^{\nu }_{0}\) and \(n_\nu \).
When the temperature was \(T > \SI{1}{MeV}\) the neutrinos and photons were in thermal equilibrium: we had \(T_\nu \sim T _{\text{plasma}} \sim T_\gamma \). In this plasma there were neutrinos, photons, electrons and positrons.

At \(T < \SI{500}{keV}\) electron-positron annihilation was still occurring, but pair creation slowed: there was not enough energy for a process like \(\nu \overline{\nu} \to e^{+}e^{-}\).
The pair annihilation poured energy into photons only, since neutrinos were decoupled by that point. 

So, the ``relic neutrinos'' are colder than the relic photons. Moreover, their number density is smaller. 

The entropy density for relativistic matter with \(P = \rho / 3\) is given\footnote{To see this, start from the thermodynamic relation \(T \dd{S} = \dd{E } + P \dd{V} \), divide through by \(T \dd{V}\) to get 
%
\begin{align}
\dv{S}{V} = \frac{1}{T}\dv{E}{V} + \frac{P}{T} = \frac{\rho + P}{T}
\,.
\end{align}
%
} by 
%
\begin{align}
s = \frac{\rho + P}{T} = \frac{4}{3} \frac{\rho}{T}
\,.
\end{align}

So, for the \(i\)-th particle species we can write 
%
\begin{align}
s_i = \frac{4}{3} \frac{\rho_{i}}{T}
= g_i \frac{2 \pi^2}{45} T^3
\,
\end{align}
%
for bosons, and the same result times \(7/8\) for fermions. 

Now, let us define \(T_{>}\) as the temperature of the photons, \(T_\gamma \), before the disappearance of the \(e^{+}e^{-}\) pairs. This temperature will be in the range \(\SI{500}{keV} < T_> < \SI{1}{MeV}\). 

Then, we can define a corresponding entropy \(s_>\): it will be given by 
%
\begin{align}
s_> = \frac{4}{3} \frac{\rho_{>}}{T_>} = \frac{2 \pi^2}{45} g_* (T_>) T^3_> 
\,,
\end{align}
%
where the effective number of degrees of freedom is given by:
%
\begin{align}
g_* (T_>) = 2 + \frac{7}{8} \qty(2 + 2) = \frac{11}{2}
\,,
\end{align}
%
since we need to account only for photons, electrons and positrons --- neutrinos have already decoupled. 

Also, let us define another temperature, \(T_<\), such that \(T_< < \SI{500}{keV}\). The corresponding \(g_* (T_<)\) will be of 2, since electrons and positrons will have decoupled. 
Then, the corresponding entropy density will read: 
%
\begin{align}
s_< = \frac{4}{3} \frac{\rho_{<}}{T_<} = \frac{4}{3} \frac{\pi^2}{30} g_* T_<^3 = \frac{2 \pi^2}{45} g_*(T_<) T_<^3
\,.
\end{align}

Now, we impose \(s_< = s_>\): in the phase transition the temperature can be discontinuous by the entropy is conserved. This yields 
%
\begin{align}
\frac{2 \pi^2}{45} g_*(T_<) T_<^3 &= \frac{2 \pi^2}{45} g_* (T_>) T^3_> \\
g_*(T_<) T_<^3 &= g_* (T_>) T^3_> \\
2 T_<^3 &= \frac{11}{2} T^3_> \\
T_< &= \sqrt[3]{\frac{11}{4}} T^3 
\,.
\end{align}

The next step is to state that \(T_> / T_< = T_\gamma / T_\nu \). This is because photons can still thermalize at \(T_<\), while neutrinos are ``stuck'' at the temperature \(T_>\).

So, the temperature of relic neutrinos today can be calculated from that of the CMB to be around \(T^{0}_{\nu } = \sqrt[3]{4/11} T^{0}_{\gamma } \approx \SI{1.96}{K}\). 

We also know that the number density of relic photons today is given by 
%
\begin{align}
n^{0}_{\gamma } = \frac{\zeta (3)}{\pi^2} 2 T_\gamma^3 \approx \SI{422}{cm^{-3}}
\,,
\end{align}
%
so we can calculate that of neutrinos using what we know from \eqref{eq:density-number-density-relativistic-limit}: 
%
\begin{align}
\frac{n^{0}_{\nu }}{n^{0}_{\gamma }} = \frac{3}{4} \qty( \frac{T_\nu}{T_\gamma } )^3 = \frac{3}{11}
\,.
\end{align}

So, \(n^{0}_{\nu } \approx \SI{100}{cm^{-3}}\).

Another example of a particle decoupling is given by the graviton: the rate of interactions which are only gravitational is given by 
%
\begin{align}
\Gamma = n \sigma v \approx G_N^2 T^{5} \sim \frac{T^{5}}{M_P^{4}}
\,,
\end{align}
%
so if we impose \(\Gamma < H \) we get the expected result: the decoupling temperature is just the Planck mass, 
%
\begin{align}
\frac{T^{5}}{M_P^{4}} < \frac{T^2}{M_P} \implies T^{3} < M_P^{3} \implies T < M_P
\,.
\end{align}

The \(g_*\) at that early time was around 100, so we find that the temperature of relic gravitational radiation should be around \(T _{\text{grav}} \approx \SI{0.8}{K}\) and its number density should be \(n _{\text{grav}} \approx \SI{15}{cm^{-3}}\)

Next week we will consider nucleosynthesis. 

\end{document}
