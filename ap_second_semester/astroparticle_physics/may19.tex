\documentclass[main.tex]{subfiles}
\begin{document}

\subsection{WIMP candidates for Dark Matter}

\marginpar{Tuesday\\ 2020-5-19, \\ compiled \\ \today}

WIMP means Weakly Interacting Massive Particle, this is a so-called heavy stable relic of the early universe. 

So, we require stability; this could also be a very long-lived particle, the only requirement is that it be still abundant now. 

The cross section must also be small. 

Let us consider a typical WIMP, with mass \(M_x \sim \SI{100}{GeV}\).

Let us consider a scenario in which the particle remains in equilibrium even when the temperature drops below the mass of the particle. 

We want to request that \(T_f< M_X\), where \(T_f\) is the temperature of the freezeout.

The value of \(\rho_{X} / \rho_{C} = \Omega_{X}\) is of the order \(1/4\) today. If we had \(n_X \sim n_\gamma \) before the freezeout, because then DM was radiation-like. 

Today we have around a baryon every billion photons, and \(\Omega_{B} \sim \SI{5}{\percent}\). 
The statistics of the DM distribution are given by 
%
\begin{align}
(M_X T)^{3/2} \exp(- \frac{M_X}{T})
\,,
\end{align}
%
so the exponential suppression plays a very important role as the temperature decreases.
We cannot over-close the universe, if we get \(\Omega_{X}> 1\) the universe would become strongly closed. 

Now, let us compute the freezeout temperature, assuming that the particle's mass is \SI{100}{GeV}. We want to show that \(T_f < M_X\).

What do we mean by ``equilibrium'' precisely? There are two possible meanings; one is the chemical equilibrium, and the other is the kinetic equilibrium. 

The annihilation of \(X\) with its antiparticle changes the total number of particles. 
In discussing the freezeout temperature we are interested in the \emph{chemical} equilibrium. 
So, we must ask: ``what is the temperature at which the annihilation \(\Gamma_{X}\) is equal to the expansion rate of the universe \(H\)?''

The first is given by 
%
\begin{align}
\Gamma_{X} = \expval{\sigma v} n_X 
\,,
\end{align}
%
where 
%
\begin{align}
n_X = \qty(M_X T )^{3/2} e^{-M_X / T}
\,,
\end{align}
%
while \(\expval{\sigma v}\) is model-dependent. Therein lies the physics of the problem.
Without a specific model, we take something like 
%
\begin{align}
\sigma \sim \frac{\alpha_{W}^2}{M_w^{4}} 
\,.
\end{align}

We plug everything into the expression, using a standard cosmology for \(H\). 
We can get an order-of-magnitude estimate. 
If we did the calculation more properly, we would be able to get the Boltzmann equation: 
%
\begin{align}
\dv{n_x}{t} + 3 H n_x = - \expval{\sigma v} \qty(n_x^2 -)
\,,
\end{align}
%
\todo[inline]{complete formula}
which can be found in the notes. 
We find 
%
\begin{align}
\Omega_{X} h^2 \approx \num{e-10} \qty(\frac{\SI{}{GeV^{-2}}}{\expval{\sigma v}}) \frac{1}{\sqrt{g(T_f)}} L
\,,
\end{align}
%
where 
%
\begin{align}
L = \log \qty(\frac{g_X M_X }{}) \dots
\,,
\end{align}
%
\todo[inline]{complete formula}
which is generally of the order of \num{20} to \num{30}.

So in the end the temperature is given by 
%
\begin{align}
T_f \approx M_X L^{-1}
\,.
\end{align}

We are, however, assuming that the annihilation goes like the weak interaction --- is this accurate?
If the value \(\alpha_{W} \) were different, we could get the correct number for \(\Omega_{X}\). 
There are good reasons to think that there might be new physics in the electroweak scale. 

\todo[inline]{How is the estimate for \(\Omega_{X}\) derived?}

Suppose we asked for \(\Omega_{X} < \num{.3}\). Then, we need to reduce the number of \(X\), \(n_X\). 

% A WIMP candidate is a baryon: if the number of baryons is equal to teh number of antibaryons.

\end{document}
