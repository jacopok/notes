\documentclass[main.tex]{subfiles}
\begin{document}

\subsection{Other candidates for Dark Matter}

\subsubsection{Baryons}

\marginpar{Tuesday\\ 2020-5-19, \\ compiled \\ \today}

% WIMP means Weakly Interacting Massive Particle, this is a so-called heavy stable relic of the early universe. 
We look at other kinds of heavy relics: first, we consider baryons, assuming that the universe is \(B\)-symmetric (so that \(N_B = N_{\overline{B}}\)).

The residual density of these (which we will still call \(X\)), as before, will be given by 
%
\begin{align} \label{eq:estimate-X-particle-density-fraction}
\Omega_{X} h^2 = \num{1.8e-10} \qty(\frac{\SI{}{GeV^{-2}}}{\sigma_0 }) 
\frac{1}{\sqrt{g_* (T_f)}} L 
\,,
\end{align}
%
where 
%
\begin{align}
L = \log \qty(\frac{g_X m_X M_P \sigma_0 }{(2 \pi)^{3/2} \sqrt{g_*}})
\,,
\end{align}
%
and the temperature \(T_f\) at which chemical equilibrium is broken can be calculated from \(T_f / m_B = x_f \approx L^{-1}\). 

If \(X\) is a baryon, we will have \(\sigma_0 \approx \qty( \SI{100}{MeV})^{-2}\).
Then, we can compute \(T_f\) by assuming \(m_B \sim \SI{1}{GeV}\):  
%
\begin{align}
T_f = \frac{m_B}{\log \qty(\frac{g_B m_B M_P \qty(\SI{100}{MeV})^{-2}}{(2\pi )^{3/2} \sqrt{g_*}})} \approx \SI{20}{MeV}
\,,
\end{align}
%
since the argument of the logarithm is about \(\num{e22}\), so \(L^{-1} \approx 1 / (22 \times \log 10) \approx 1/50\).
The main contribution is the Planck mass \(M_P \sim \SI{e19}{GeV}\), the rest of the factors are corrections.

So, we have \(m_B \sim \SI{1}{GeV}\) and \(T_f \sim \SI{20}{MeV}\): 
between these temperatures, we have \(B \overline{B}\) annihilation, which means the number density must be multiplied by a suppression factor \(\exp( - m_B / T_f)\). 

Our estimate for the fraction of the critical density given by baryons, according to \eqref{eq:estimate-X-particle-density-fraction}, is
%
\begin{align}
\Omega_{B} \approx \num{4e-10} \num{e-2} \frac{1}{\sqrt{10}} 50 \approx \num{5e-11}
\,,
\end{align}
%
while today we have around a baryon every billion photons, and \(\Omega_{B} \sim \SI{5}{\percent}\). 
There is a mismatch of 9 orders of magnitude!

\todo[inline]{So, is this not a problem? Does the line of reasoning not hold for regular baryons?}

\subsubsection{Heavy neutrinos}

Another possibility for a heavy relic is a heavy neutrino of a fourth generation.   
Neutrinos of the known generations are light, with \(m_\nu \ll T_f\); let us call this new one \(\nu_4 \), and assume \(M_{\nu_4 }> T_f^{(4)}\). 
Let us also assume that the mass of this heavy neutrino is well below the electroweak scale of \(\sim \SI{100}{GeV}\): something like \(M_{\nu_4 } \sim \SI{1}{GeV}\). 

Applying our formula again, we get: 
%
\begin{align}
\Omega_{\nu_4 } h^2 &\approx \frac{\num{1.8e-10}}{G_F^2M^2_{\nu_4 }}
\frac{1}{\sqrt{g_*(t_f)}} 
\log \qty(\frac{2 M_{\nu_4} M_P G_F^2 M^2_{\nu_4 }}{(2 \pi )^{3/2} \sqrt{g_*}})  \\
&\approx 
\frac{\num{2e-10} \times \SI{e10}{GeV^{4}}}{M^2_{\nu_4 }}
\frac{1}{\sqrt{g_*} (t_f)}
\log \qty(
    \frac{2}{(2 \pi )^{3/2}}
    \frac{M^3_{\nu_4 } M_P}{\sqrt{g_*}}
    \SI{e-10}{GeV^{-4}}
)
\,,
\end{align}
%
where we have two things to be determined: \(M_{\nu_4 }\) and \(g_*\); for the first we assume \(M_{\nu_4 } \approx \SI{3}{GeV}\).
We will not fully fix the  value of \(M_{\nu_4 }\), but we assume the order of magnitude, which we can then substitute into the slowly-varying parts of the equation, while we leave its appearances in the fast-varying part of the equation free.

For the second we need to temperature to calculate at: we can use our formula \(T = M L^{-1}\), so that we get 
%
\begin{align}
T_f^{\nu_4 } = \frac{\SI{3}{GeV}}{\log \qty(
    \frac{2}{(2 \pi )^{3/2}}
    \frac{3^{3} \num{e19}}{\sqrt{g_*}}
    \num{e-10}
)}
\approx \frac{\SI{}{GeV}}{3 \divisionsymbol 4}
\,.
\end{align}

One may be worried by the circularity of estimating \(g_*\) inside this calculation: properly speaking we should solve the equation using a model for \(g_* (T_f)\), but in this kind of order of magnitude reasoning whether \(g_*\) is 10 or 100 does not really affect the result significantly, so we do not bother. 

At this temperature, we have photons, gluons, quarks and regular neutrinos, plus our fourth neutrino --- we are at the very limit of the time at which it is in chemical equilibrium, so we still need to account for it.
\todo[inline]{Since the formula is found by assuming that\dots right?} 
So, we get 
%
\begin{align}
\sqrt{g_*} = 2 + 8 \times 2 + \frac{7}{8} \qty(
    2 \times 4
    + 2 \times 3
    + 3 \times 4 \times 3
) = \sqrt{ \frac{183}{4}}
\,,
\end{align}
%
so plugging everything into the formula we get 
%
\begin{align}
\Omega_{\nu_4 } \approx \num{0.55} \qty(\frac{ \SI{3}{GeV}}{M_{\nu_4 }})^2
\,.
\end{align}

So, we must have \(M_{\nu_4 } \gtrsim \SI{3}{GeV}\) in order to not over-produce dark matter. 

% So, we require stability; this could also be a very long-lived particle, the only requirement is that it be still abundant now. 

% The cross section must also be small. 

% Let us consider a typical WIMP, with mass \(M_x \sim \SI{100}{GeV}\).

% Let us consider a scenario in which the particle remains in equilibrium even when the temperature drops below the mass of the particle. 

% We want to request that \(T_f< M_X\), where \(T_f\) is the temperature of the freezeout.

% The value of \(\rho_{X} / \rho_{C} = \Omega_{X}\) is of the order \(1/4\) today. If we had \(n_X \sim n_\gamma \) before the freezeout, because then DM was radiation-like. 

% The statistics of the DM distribution are given by 
% %
% \begin{align}
% (M_X T)^{3/2} \exp(- \frac{M_X}{T})
% \,,
% \end{align}
% %
% so the exponential suppression plays a very important role as the temperature decreases.
% We cannot over-close the universe, if we get \(\Omega_{X}> 1\) the universe would become strongly closed. 

% Now, let us compute the freezeout temperature, assuming that the particle's mass is \SI{100}{GeV}. We want to show that \(T_f < M_X\).

% What do we mean by ``equilibrium'' precisely? There are two possible meanings; one is the chemical equilibrium, and the other is the kinetic equilibrium. 

% The annihilation of \(X\) with its antiparticle changes the total number of particles. 
% In discussing the freezeout temperature we are interested in the \emph{chemical} equilibrium. 
% So, we must ask: ``what is the temperature at which the annihilation \(\Gamma_{X}\) is equal to the expansion rate of the universe \(H\)?''

% The first is given by 
% %
% \begin{align}
% \Gamma_{X} = \expval{\sigma v} n_X 
% \,,
% \end{align}
% %
% where 
% %
% \begin{align}
% n_X = \qty(M_X T )^{3/2} e^{-M_X / T}
% \,,
% \end{align}
% %
% while \(\expval{\sigma v}\) is model-dependent. Therein lies the physics of the problem.
% Without a specific model, we take something like 
% %
% \begin{align}
% \sigma \sim \frac{\alpha_{W}^2}{M_w^{4}} 
% \,.
% \end{align}

% We plug everything into the expression, using a standard cosmology for \(H\). 
% We can get an order-of-magnitude estimate. 
% If we did the calculation more properly, we would be able to get the Boltzmann equation: 
% %
% \begin{align}
% \dv{n_x}{t} + 3 H n_x = - \expval{\sigma v} \qty(n_x^2 -)
% \,,
% \end{align}
% %
% \todo[inline]{complete formula}
% which can be found in the notes. 
% We find 
% %
% \begin{align}
% \Omega_{X} h^2 \approx \num{e-10} \qty(\frac{\SI{}{GeV^{-2}}}{\expval{\sigma v}}) \frac{1}{\sqrt{g(T_f)}} L
% \,,
% \end{align}
% %
% where 
% %
% \begin{align}
% L = \log \qty(\frac{g_X M_X }{}) \dots
% \,,
% \end{align}
% %
% \todo[inline]{complete formula}
% which is generally of the order of \num{20} to \num{30}.

% So in the end the temperature is given by 
% %
% \begin{align}
% T_f \approx M_X L^{-1}
% \,.
% \end{align}

% We are, however, assuming that the annihilation goes like the weak interaction --- is this accurate?
% If the value \(\alpha_{W} \) were different, we could get the correct number for \(\Omega_{X}\). 
% There are good reasons to think that there might be new physics in the electroweak scale. 

% \todo[inline]{How is the estimate for \(\Omega_{X}\) derived?}

% Suppose we asked for \(\Omega_{X} < \num{.3}\). Then, we need to reduce the number of \(X\), \(n_X\). 

% A WIMP candidate is a baryon: if the number of baryons is equal to teh number of antibaryons.

\end{document}
