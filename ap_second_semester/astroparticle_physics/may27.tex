\documentclass[main.tex]{subfiles}
\begin{document}

\section{Symmetry breaking and phase transitions}

\marginpar{Sunday\\ 2020-7-5, \\ compiled \\ \today}

Let us consider the time, in the early universe, before BBN. The times are 
\begin{figure}
\centering 
\begin{tabular}{ccc}
Temperature & Symmetry group & time \\
\hline
\SI{e19}{GeV}  & \(G _{\text{GUT}}\) & \SI{e-43}{s} \\
\num{e15} \(\divisionsymbol\) \SI{e16}{GeV} & \(G _{\text{GUT}}\), barely & \SI{e-38}{s} \\
% \num{e2} \(\divisionsymbol\) \SI{e3}{GeV} & \(SU(3)_c \times SU(2)_L \times U(1)_Y\) & \num{e-10} \(\divisionsymbol\) \SI{e-12}{s} \\
\num{e2} \(\divisionsymbol\) \SI{e3}{GeV} & \(SU(3)_c \times SU(2)_L \times U(1)_Y\), barely & \num{e-10} \(\divisionsymbol\) \SI{e-12}{s} \\
\num{200} \(\divisionsymbol\) \SI{300}{MeV} & \(SU(3)_c \times U(1) _{\text{em}}\): quark-gluon confinement & \num{e-4} \(\divisionsymbol\) \SI{e-5}{s} \\
\SI{1}{MeV} & \(SU(3)_c \times U(1) _{\text{em}}\): nucleosynthesis & \SI{1}{s} 
\end{tabular}
\label{tab:phase-transitions}
\caption{Phase transitions in the early universe.}
\end{figure}

The time shortly thereafter, from 1 to \SI{100}{s}, is nucleosynthesis, the earliest time we can see; we do so by exploring the abundances of light nuclei. 
Everything before comes out of our model for particle physics. 

After BBN, we have an epoch of radiation dominance, followed by matter dominance, then recombination, accelerated expansion. 

We will not consider these, but instead keep to the first second. 

The properties of the cosmic medium can either change in a \textbf{smooth crossover}, or abruptly in a \textbf{phase transition}.
Formally, the latter corresponds to an order parameter being zero in a phase and nonzero in another. 

The order phase transitions describes the first derivative order which is discontinuous at them: so, first-order transitions are the sharpest, they have latent heat; while second-order transitions are smoother. 

We get phase transitions when there is a mismatch between the ground state at zero temperature and at some finite temperature. Let us denote by \(\expval{\phi }_{T}\) the absolute minimum of the effective potential \(V  _{\text{eff}} (T, \phi )\). 

Sometimes, the ground state being \(\phi = 0\) is restored at a finite temperature. 

First-order phase transitions have a discontinuity in \(\expval{\phi }_{T}\) at a certain \(T = T_C\); for second-order phase transitions it is continuous but not differentiable. 

An example of a first-order transitions is the boiling of a liquid, second-order ones are the transition of a ferromagnet, of a superconductor, of a superfluid.

For second-order and smooth-crossover transitions the medium may always be at equilibrium, on the other hand for first-order transitions switching to the other minimum becomes thermodynamically favorable if the temperature crosses the critical point: so, we can get \textbf{bubble nucleation}. 

The expectation value of the field becomes inhomogeneous in space. We get bubbles with the new vacuum, for which it is energetically favorable to expand. 

The transition being energetically favorable is not enough: the nucleation rate must be higher than the expansion of the universe, otherwise the transition does not complete. 

\subsection{The electroweak phase transition}

The potential for the Higgs field is given by 
%
\begin{align}
V(\phi ) = \mu^2 \abs{\phi }^2 + \lambda \abs{\phi }^{4}
\,,
\end{align}
%
with \(\mu^2<0\) and \(\lambda >0\). Then, the VEV is 
%
\begin{align}
\expval{\phi }_0 = \sqrt{\frac{- \mu^2}{2 \lambda }} = \frac{v}{\sqrt{2}}
\,,
\end{align}
%
so at zero temperature the potential can be written as 
%
\begin{align}
V(\phi) = \lambda \qty(\phi^2 - \frac{v^2}{2})^2
\,.
\end{align}

What do we write for a nonzero temperature? The term to add is: 
%
\begin{align}
V &= V(T = 0) + \frac{T^2}{24} \qty[ \sum _{\text{bosons}} g_i m_i^2 (\phi ) + \frac{1}{2} \sum _{\text{fermions}} g_i m_i^2 (\phi )]  \\
&= \qty(- \lambda v^2 + \frac{\alpha}{24} T^2) \phi  + \lambda \phi^{4}
\,,
\end{align}
%
where \(m_i (\phi ) = h_i \phi \), \(h_i\) being the coupling constant of that specific fermion or boson to \(\phi \). 
The sum over the bosons runs over the Higgs itself as well: so, we get a term \(g_h h_h^2 H^2\) as well. 
Finally, we defined 
%
\begin{align}
\alpha = \sum _{\text{bosons}} g_i h_i^2 (\phi ) + \frac{1}{2} \sum _{\text{fermions}} g_i h_i^2 (\phi )
\,.
\end{align}

So, we can see that when the coefficient of \(\phi^2\) becomes larger than \(0\) the transition takes place: this will happen at \(\alpha T^2 / 24 = \lambda v^2\), which means 
%
\begin{align}
T_C = 2 v \sqrt{ \frac{6\lambda}{\alpha }} 
= 2 \sqrt{ \frac{M_H^2}{4 M_W^2 + 2 M_Z^2 + 4 m_t^2 + M_H^2}} v 
\approx \SI{146}{GeV}
\,.
\end{align}

This is a tree-level perturbative result: if we account for nonperturbative effects we find \(T_C \approx \SI{160}{GeV}\). 

If we were to use the perturbative approach (at one loop) we would find that the electroweak phase transition is of first order, however using a nonperturbative one we get that it is smoother. 
Perturbing would work for small \( M_H / M_W \), but actually this is larger than 1. 

Working nonperturbatively, one finds that the transition is of second order or even smoother. 

\subsubsection{Phase transitions and inflation}

\todo[inline]{Graph of \(V(\phi )\) with a bump, a slow decrease region, and then an absolute minimum: where does this shape come from?}

At the GUT temperature, the average energy of the Higgs field is around the GUT energy scale.
After the temperature goes below this, the GUT symmetry is broken. 

The scale factor scales like \(H \sim T^2 / M_P^{*}\), where \(M_P^{*} \approx M_P / \sqrt{g_{*}} \approx \SI{e18}{GeV}\). 
Then, if we take \(M _{\text{GUT}} \sim \SI{e16}{GeV}\) we get 
%
\begin{align}
t = H^{-1} = \frac{\num{e2}}{M _{\text{GUT}}}
\,.
\end{align}

The vacuum energy density will be given by the potential at \(\phi  = 0\). In this dark energy dominated period, the scale factor will scale like \(a \sim e^{Ht}\).

The energy of this false vacuum is much larger than the one of the true vacuum. 

If the time it takes for \(\phi \) to evolve from 0 to \(v\) is of the order \(t \sim 100 H^{-1}\), then initial smooth patches whose size is comparable to \(H^{-1} \sim \SI{e-28}{cm}\) will be blown out by \(e^{H t} = e^{100}\): they will become of a size \SI{3e15}{cm}, or about \SI{200}{AU}.

The entropy increase is enormous: we can compute it like \(S = T_C^3 / H^3\). 
This is because the entropy density \(s\) scales like \(T^3\) \cite[eq.\ 8.46]{bergstromCosmologyParticleAstrophysics2003}, and we retrieve the total entropy by multiplying by \(V \sim a^{3}\). 

\todo[inline]{Hold on, the volume is \(a^3\), not \(H^{-3}\)! In a steady-state solution \(H\) is constant. The reasoning works out, since it is \(t\) which scales by 100, but what he wrote is incorrect\dots}

So, even if the transition does not correspond to a temperature change, the variation in temperature is \(\qty( e^{Ht})^3 \sim \num{e130}\), since \(t\) is multiplied by 100.

Some early models of slow-roll inflation are based on a transition between \(SU(5)\) to the SM symmetries.

The idea is to get a GUT potential which is very close to flat near the origin, so that the quadratic term \(m^2\phi^2\) vanishes. 

This way, we have no SSB in the tree-level GUT Higgs potential (since the self-interaction terms are only relevant at higher powers of \(\phi  \)).
The only corrections are \emph{loop} corrections, possibly with loops constituted by \(SU(5) \) gauge bosons. 

\subsubsection{Coleman-Weinberg mechanism}

We take 
%
\begin{align}
V(\phi ) = \const \times \phi^{4} \qty[\log \qty( \frac{\phi^2}{v^2}) - \frac{1}{2}] + \const \times v^4
\,,
\end{align}
%
where \(v = \expval{\phi } \sim \SI{e16}{GeV}\). 

If \(\phi \ll v\), the potential looks like \(V \sim v^{4} - \lambda \phi^{4} / 4\), where \(\lambda = \const \log (\phi^2 / v^2) - 1/2 \sim \num{e-1} \divisionsymbol \num{e-2}\). 

If \(\phi \) is close to 0, then, the potential is approximated up to third order as 
%
\begin{align}
V(\phi ) \sim \alpha^2 _{\text{GUT}} \frac{v^{4}}{2}
\,,
\end{align}
%
where \(\alpha^2 _{\text{GUT}} \sim \num{e-3}\). 
\todo[inline]{Should \(\alpha^2 \) not be related to \(\lambda \)? Also, I do not understand the characterization of the situations: how is \(\phi \ll v\) different from \(\phi \sim 0\)? }

In this model, the expansion rate is given by 
%
\begin{align}
H^2 \approx \frac{4 \pi }{3} \frac{\const v^{4}}{M_P^2} \approx \qty( \SI{e11}{GeV})^2
\,.
\end{align}

With this model we get a phase transition which is first order at \(T_C \sim \SI{e16}{GeV}\).
There are, however, issues with this model. 

\todo[inline]{What can we say about this model?}

\section{Cosmic matter-antimatter asymmetry}

The matter-antimatter asymmetry problem is two-fold: 
\begin{enumerate}
    \item we want to find a way to generate a nonzero baryon number \(\Delta B = n_B - n_{\overline{B}}\) starting from a symmetric \(\Delta B = 0\) universe;
    \item we want to account for the smallness of the baryon-to-photon ratio: 
    %
    \begin{align}
    \eta_{B} = \frac{n_{B, 0}}{n_{\gamma , 0}} \approx \num{6e-10}
    \,.
    \end{align}
\end{enumerate}

We can express the latter using 
%
\begin{align}
\Delta B = \frac{n_B - n_{\overline{B}}}{s} \approx \num{9e-11}
\,,
\end{align}
%
the baryon-to-entropy ratio. 

If we had \(\Delta B = 0\), the fraction of the matter density in the universe which is represented by baryons would be very much smaller than \SI{5}{\percent}. 

So far, there is no experimental evidence for large antimatter regions.
If it existed, it would have to be separated from regular matter by extremely large distances, since otherwise we would detect a gamma-ray background from the common annihilation. 

\subsection{Sakharov's conditions}

They are necessary conditions for the creation of baryon asymmetry starting from a baryon-symmetric situation. 
They are: 
\begin{enumerate}
    \item Baryon number non-conservation;
    \item \(C\) and \(CP\) violation;
    \item out-of-equilibrium processes.
\end{enumerate}

The first condition arises because if all processes conserved baryon number then there could be no formation of asymmetry. 
The second condition arises because if there was charge-conjugation symmetry then processes generating matter and antimatter would have the exact same properties, and would therefore statistically occur in the same amounts.
Out-of-equilibrium processes are needed since otherwise, because of \(CPT\) symmetry, \(\Delta B\) increasing and decreasing processes would occur in the same amounts. 

\subsubsection{Baryogenesis in a GUT}

Let us suppose that there is a GUT scalar particle \(X\) which has two decay channels, either \(qq\) or \(\overline{q} \overline{\ell}\), where \(q\) is a quark and \(\ell\) is a lepton. 

These are both \(B\)-violating, the first introduces \(B = +2/3\) and the second introduces \(B = - 1/3\).
Let us suppose that the branching ratios are \(\tau \) and \(1 - \tau \) respectively. 

The antiparticle \(\overline{X}\) can decay into \(\overline{q} \overline{q}\) or \(q \ell\), which have \(B = - 2/3\) and \(B = + 1/3\) and branching ratios \(\overline{\tau}\) and \(1 - \overline{\tau}\) respectively.

If there is \(C\) violation, we can have \(\tau \neq \overline{\tau}\). 
So, if we have a \(X \overline{X}\) pair, and both of the particles decay, then the changes of baryon number are \( \tau - 1/3\) and \(- \overline{\tau} + 1/3\) respectively, so in total we find \(\Delta B_{X \overline{X}} = \tau - \overline{\tau}\).

This yields baryon number violation by violating the first two Sakharov conditions: where does the third one come in? 

Let us again consider the boson \(X\), a ``leptoquark'' gauge boson which is coupled to all fermions. Its coupling constant will be \(\alpha = \alpha _{\text{GUT}}\). 
Its SM charges can either be \((3, 2, +4/3)\) or \((3,2, + 1/3)\). 

The decay rate of \(X\) into fermions will be given by \(\Gamma^{X}_{D} \approx \alpha M_X\). 
The decay of \(X\) can occur as long as \(\Gamma^{X}_{D} \geq H \sim
 T^2 / M_P^{*}\), so that on average a particle has good odds to have actually managed to decay since the beginning of the universe. 
 
This condition can be written as \(T^2< \alpha M_X M^{*}_{P}\), and if we consider the time at which \(T = M_X\) it becomes \(M_X < \alpha M_P^{*}\).
The condition of this \(X\) being able to decay means that it can reach equilibrium: so, at \(T = M_X\) equilibrium can be achieved as long as \(X\) is less massive than \(\alpha \times \SI{e18}{GeV}\). 

If this condition is fulfilled, then as \(T\) drops below \(M_X\) the particle \(X\) is still in equilibrium and the \(B\) violating processes are still happening. So, as this happens, all the \(B\) violation is erased. 
\todo[inline]{Wait, how?}

If, on the other hand, \(M_X \geq \alpha \SI{e18}{GeV}\), the lifetime of \(X\) is longer than the age of the universe when the temperature reaches its mass. 
Therefore, it only starts decaying \emph{later}, when \(T < M_X\), and these decays are occurring out of equilibrium. 

The ratio \(n_X / n_\gamma \) is approximately 1, but the density of \(X\) is far from the equilibrium one, which would satisfy 
%
\begin{align}
n_X^{\text{eq}} \sim (M_X T)^{3/2} e^{-M_X / T} 
\,,
\end{align}
%
and since \(M_X > T\) this would be suppressed, and we'd have \(n_X^{\text{eq}} \ll n_\gamma \).
Intuitively, if we \emph{did} have equilibrium almost all the \(X\)s would have decayed earlier, many more of them are left than this. 

So, the net \(\Delta B \neq 0\) cannot be erased. If we define \(\Delta B = \tau - \overline{\tau}\), then we will be left with a number density of baryons 
%
\begin{align}
n_B = \Delta B n_X
\,
\end{align}
%
at the time when \(T = M_X\). Since \(n_X \sim n_\gamma \), we also have \(n_B \approx \Delta B n_\gamma \).

At this point, \(\Delta B\) is frozen
\todo[inline]{Is it? Many \(X\)s still have to decay, and when they do they violate \(B\)!}
however \(n_\gamma \) is not conserved in later stages of the evolution --- for example, when electron-positron annihilation happens ---, so we cannot rely on the ratio \(n_B / n_\gamma \) to be constant. 

Instead, we can use the specific entropy ratio \(n_B / s\), where \cite[eqs.\ 8.46 and 8.49]{bergstromCosmologyParticleAstrophysics2003}
%
\begin{align}
s = \frac{2 \pi^2}{45} g_s (T) T^3 
= \frac{\pi^{4}}{45 \zeta (3)} g_s (T) n_\gamma 
\,,
\end{align}
%
where, at \(T \sim M_X\), we have \(g_s (T) \sim 100\), so this means \(s \sim 100 n_\gamma \).

Note that this \(g_s(T)\) is not the same as the \(g(T)\) we get when computing the density, although they are close in value \cite[fig.\ 1]{husdalEffectiveDegreesFreedom2016}.
For an in-depth review of how they are defined, refer to Husdal \cite[sec.\ 4.5]{husdalEffectiveDegreesFreedom2016}; the gist is that thermodynamic quantities have different dependencies on the phase space density and the temperature, so when we weigh the various particles we must account for these differences. The \(g_s\) describes the effective degrees of freedom associated with the computation of the entropy, while \(g\) refers to the degrees of freedom associated with the energy density. 

Therefore, 
%
\begin{align}
\frac{n_B}{s} \sim \frac{\Delta B}{100}
\,.
\end{align}

From Big Bang Nucleosynthesis we can place a bound on \(\eta  = n_B / n_\gamma \), namely \(\num{5e-10} < \eta < \num{7e-10}\).

The same reasoning can be applied today: now we have \(g_s(T) \approx 4\), so we find \(s \approx 7 n_\gamma \).
Then, using the same bound from BBN we get 
%
\begin{align}
\num{7e-11} < 
\frac{n_B}{s}
< \num{e-10}
\,.
\end{align}

\todo[inline]{And this is incompatible with the earlier one\dots which should we use?}

Now, since \(n_B / s \approx \num{e-2} \Delta B\) we get that we can have \(\tau - \overline{\tau} \sim \num{e-8}\) to explain observations. 

In order to have \(\tau \neq \overline{\tau}\) we need \(B\), \(C\) and \(CP\) violation. 
However, even if these symmetries are violated, at tree level the decay rates for \(\Gamma (X \to qq)\) and \(\Gamma (\overline{X} \to \overline{q} \overline{q})\) are equal: all the kinematics is the same, and we will get 
%
\begin{align}
\Gamma (X \to qq) \propto \abs{g}^2 M_X
\qquad \text{and} \qquad
\Gamma (\overline{X} \to \overline{q} \overline{q}) \propto \abs{g^{*}}^2 M_X
\,,
\end{align}
%
so they will be equal! 
This can change if we consider loops. 

\begin{claim}
Say \(I\) is the complex-valued loop integral, and we assume that a particle \(Y\) propagates between the quarks with a coupling \(g'\).
Then, the result reads:
%
\begin{align}
\tau - \overline{\tau} = \frac{2 \Im I \Im \qty(g_2 g_1^{*} g_1' g^{\prime *}_2)}{\abs{g_1 }^2 + \abs{g_2 }^2}
\approx \frac{g^2}{4 \pi } f \qty( \frac{m_X}{m_Y})
\,,
\end{align}
%
where \(f\) is a function such that \(f(1) = 0\), and which is of order 1 for large arguments. 
\end{claim}

Since we want \(\tau - \overline{\tau} \lesssim g^2 / 4 \pi \), we must have \(g^2 / 4 \pi \gtrsim \num{e-8}\). 

This allows us to specify our out-of-equilibrium condition better: it was \(M_X \gtrsim \alpha M_P^{*}\), but with the new bound on \(\alpha \) we can write it as 
%
\begin{align}
M_X > \SI{e10}{GeV}
\,.
\end{align}

So, \(X\) must be a superheavy particle.
In order to implement this baryogenesis process the reheating temperature must be quite high, \(T _{\text{RH}} > \SI{e10}{GeV}\).

Is \(\Delta B\) preserved after the breaking of \(G _{\text{GUT}}\)? If, under the SM symmetries, \(B\) is conserved we should have no issue. 
This, as we shall see, might not be the case. 

\todo[inline]{Hold up: if the decay channel of the superheavy \(X\) are \(B\)-breaking, what does the temperature have to do with it? If there are \(X\)s leftover they still need to decay\dots}

\end{document}