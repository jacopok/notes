\documentclass[main.tex]{subfiles}
\begin{document}

\section{Symmetry breaking and phase transitions}

\marginpar{Sunday\\ 2020-7-5, \\ compiled \\ \today}

Let us consider the time, in the early universe, before BBN. The times are 
\begin{figure}
\centering 
\begin{tabular}{ccc}
Temperature & Symmetry group & time \\
\hline
\SI{e19}{GeV}  & \(G _{\text{GUT}}\) & \SI{e-43}{s} \\
\num{e15} \(\divisionsymbol\) \SI{e16}{GeV} & \(G _{\text{GUT}}\), barely & \SI{e-38}{s} \\
% \num{e2} \(\divisionsymbol\) \SI{e3}{GeV} & \(SU(3)_c \times SU(2)_L \times U(1)_Y\) & \num{e-10} \(\divisionsymbol\) \SI{e-12}{s} \\
\num{e2} \(\divisionsymbol\) \SI{e3}{GeV} & \(SU(3)_c \times SU(2)_L \times U(1)_Y\), barely & \num{e-10} \(\divisionsymbol\) \SI{e-12}{s} \\
\num{200} \(\divisionsymbol\) \SI{300}{MeV} & \(SU(3)_c \times U(1) _{\text{em}}\): quark-gluon confinement & \num{e-4} \(\divisionsymbol\) \SI{e-5}{s} \\
\SI{1}{MeV} & \(SU(3)_c \times U(1) _{\text{em}}\): nucleosynthesis & \SI{1}{s} 
\end{tabular}
\label{tab:phase-transitions}
\caption{Phase transitions in the early universe.}
\end{figure}

The time shortly thereafter, from 1 to \SI{100}{s}, is nucleosynthesis, the earliest time we can see; we do so by exploring the abundances of light nuclei. 
Everything before comes out of our model for particle physics. 

After BBN, we have an epoch of radiation dominance, followed by matter dominance, then recombination, accelerated expansion. 

We will not consider these, but instead keep to the first second. 

The properties of the cosmic medium can either change in a \textbf{smooth crossover}, or abruptly in a \textbf{phase transition}.
Formally, the latter corresponds to an order parameter being zero in a phase and nonzero in another. 

The order phase transitions describes the first derivative order which is discontinuous at them: so, first-order transitions are the sharpest, they have latent heat; while second-order transitions are smoother. 

We get phase transitions when there is a mismatch between the ground state at zero temperature and at some finite temperature. Let us denote by \(\expval{\phi }_{T}\) the absolute minimum of the effective potential \(V  _{\text{eff}} (T, \phi )\). 

Sometimes, the ground state being \(\phi = 0\) is restored at a finite temperature. 

First-order phase transitions have a discontinuity in \(\expval{\phi }_{T}\) at a certain \(T = T_C\); for second-order phase transitions it is continuous but not differentiable. 

An example of a first-order transitions is the boiling of a liquid, second-order ones are the transition of a ferromagnet, of a superconductor, of a superfluid.

For second-order and smooth-crossover transitions the medium may always be at equilibrium, on the other hand for first-order transitions switching to the other minimum becomes thermodynamically favorable if the temperature crosses the critical point: so, we can get \textbf{bubble nucleation}. 

The expectation value of the field becomes inhomogeneous in space. We get bubbles with the new vacuum, for which it is energetically favorable to expand. 

The transition being energetically favorable is not enough: the nucleation rate must be higher than the expansion of the universe, otherwise the transition does not complete. 

\subsection{The electroweak phase transition}

The potential for the Higgs field is given by 
%
\begin{align}
V(\phi ) = \mu^2 \abs{\phi }^2 + \lambda \abs{\phi }^{4}
\,,
\end{align}
%
with \(\mu^2<0\) and \(\lambda >0\). Then, the VEV is 
%
\begin{align}
\expval{\phi }_0 = \sqrt{\frac{- \mu^2}{2 \lambda }} = \frac{v}{\sqrt{2}}
\,,
\end{align}
%
so at zero temperature the potential can be written as 
%
\begin{align}
V(\phi) = \lambda \qty(\phi^2 - \frac{v^2}{2})^2
\,.
\end{align}

What do we write for a nonzero temperature? The term to add is: 
%
\begin{align}
V &= V(T = 0) + \frac{T^2}{24} \qty[ \sum _{\text{bosons}} g_i m_i^2 (\phi ) + \frac{1}{2} \sum _{\text{fermions}} g_i m_i^2 (\phi )]  \\
&= \qty(- \lambda v^2 + \frac{\alpha}{24} T^2) \phi  + \lambda \phi^{4}
\,,
\end{align}
%
where \(m_i (\phi ) = h_i \phi \), \(h_i\) being the coupling constant of that specific fermion or boson to \(\phi \). 
The sum over the bosons runs over the Higgs itself as well: so, we get a term \(g_h h_h^2 H^2\) as well. 
Finally, we defined 
%
\begin{align}
\alpha = \sum _{\text{bosons}} g_i h_i^2 (\phi ) + \frac{1}{2} \sum _{\text{fermions}} g_i h_i^2 (\phi )
\,.
\end{align}

So, we can see that when the coefficient of \(\phi^2\) becomes larger than \(0\) the transition takes place: this will happen at \(\alpha T^2 / 24 = \lambda v^2\), which means 
%
\begin{align}
T_C = 2 v \sqrt{ \frac{6\lambda}{\alpha }} 
= 2 \sqrt{ \frac{M_H^2}{4 M_W^2 + 2 M_Z^2 + 4 m_t^2 + M_H^2}} v 
\approx \SI{146}{GeV}
\,.
\end{align}

This is a tree-level perturbative result: if we account for nonperturbative effects we find \(T_C \approx \SI{160}{GeV}\). 

If we were to use the perturbative approach (at one loop) we would find that the electroweak phase transition is of first order, however using a nonperturbative one we get that it is smoother. 
Perturbing would work for small \( M_H / M_W \), but actually this is larger than 1. 

Working nonperturbatively, one finds that the transition is of second order or even smoother. 

\subsubsection{Phase transitions and inflation}

\todo[inline]{Graph of \(V(\phi )\) with a bump, a slow decrease region, and then an absolute minimum: where does this shape come from?}

At the GUT temperature, the average energy of the Higgs field is around the GUT energy scale.
After the temperature goes below this, the GUT symmetry is broken. 

The scale factor scales like \(H \sim T^2 / M_P^{*}\), where \(M_P^{*} \approx M_P / \sqrt{g_{*}} \approx \SI{e18}{GeV}\). 
Then, if we take \(M _{\text{GUT}} \sim \SI{e16}{GeV}\) we get 
%
\begin{align}
t = H^{-1} = \frac{\num{e2}}{M _{\text{GUT}}}
\,.
\end{align}

The vacuum energy density will be given by the potential at \(\phi  = 0\). In this dark energy dominated period, the scale factor will scale like \(a \sim e^{Ht}\).

The energy of this false vacuum is much larger than the one of the true vacuum. 

If the time it takes for \(\phi \) to evolve from 0 to \(v\) is of the order \(t \sim 100 H^{-1}\), then initial smooth patches whose size is comparable to \(H^{-1} \sim \SI{e-28}{cm}\) will be blown out by \(e^{H t} = e^{100}\): they will become of a size \SI{3e15}{cm}, or about \SI{200}{AU}.

The entropy increase is enormous: we can compute it like \(S = T_C^3 / H^3\). 
This is because the entropy density \(s\) scales like \(T^3\) \cite[eq.\ 8.46]{bergstromCosmologyParticleAstrophysics2003}, and we retrieve the total entropy by multiplying by \(V \sim a^{3}\). 

\todo[inline]{Hold on, the volume is \(a^3\), not \(H^{-3}\)! In a steady-state solution \(H\) is constant. The reasoning works out, since it is \(t\) which scales by 100, but what he wrote is incorrect\dots}

So, even if the transition does not correspond to a temperature change, the variation in temperature is \(\qty( e^{Ht})^3 \sim \num{e130}\), since \(t\) is multiplied by 100.

Some early models of slow-roll inflation are based on a transition between \(SU(5)\) to the SM symmetries.

The idea is to get a GUT potential which is very close to flat near the origin, so that the quadratic term \(m^2\phi^2\) vanishes. 

This way, we have no SSB in the tree-level GUT Higgs potential (since the self-interaction terms are only relevant at higher powers of \(\phi  \)).
The only corrections are \emph{loop} corrections, possibly with loops constituted by \(SU(5) \) gauge bosons. 

\subsubsection{Coleman-Weinberg mechanism}

We take 
%
\begin{align}
V(\phi ) = \const \times \phi^{4} \qty[\log \qty( \frac{\phi^2}{v^2}) - \frac{1}{2}] + \const \times v^4
\,,
\end{align}
%
where \(v = \expval{\phi } \sim \SI{e16}{GeV}\). 

If \(\phi \ll v\), the potential looks like \(V \sim v^{4} - \lambda \phi^{4} / 4\), where \(\lambda = \const \log (\phi^2 / v^2) - 1/2 \sim \num{e-1} \divisionsymbol \num{e-2}\). 

If \(\phi \) is close to 0, then, the potential is approximated up to third order as 
%
\begin{align}
V(\phi ) \sim \alpha^2 _{\text{GUT}} \frac{v^{4}}{2}
\,,
\end{align}
%
where \(\alpha^2 _{\text{GUT}} \sim \num{e-3}\). 
\todo[inline]{Should \(\alpha^2 \) not be related to \(\lambda \)? Also, I do not understand the characterization of the situations: how is \(\phi \ll v\) different from \(\phi \sim 0\)? }

In this model, the expansion rate is given by 
%
\begin{align}
H^2 \approx \frac{4 \pi }{3} \frac{\const v^{4}}{M_P^2} \approx \qty( \SI{e11}{GeV})^2
\,.
\end{align}

With this model we get a phase transition which is first order at \(T_C \sim \SI{e16}{GeV}\).
There are, however, issues with this model. 

\todo[inline]{What can we say about this model?}

\section{Cosmic matter-antimatter asymmetry}

The matter-antimatter asymmetry problem is two-fold: 
\begin{enumerate}
    \item we want to find a way to generate a nonzero baryon number \(\Delta B = n_B - n_{\overline{B}}\) starting from a symmetric \(\Delta B = 0\) universe;
    \item we want to account for the smallness of the baryon-to-photon ratio: 
    %
    \begin{align}
    \eta_{B} = \frac{n_{B, 0}}{n_{\gamma , 0}} \approx \num{6e-10}
    \,.
    \end{align}
\end{enumerate}

We can express the latter using 
%
\begin{align}
\Delta B = \frac{n_B - n_{\overline{B}}}{s} \approx \num{9e-11}
\,,
\end{align}
%
the baryon-to-entropy ratio. 

If we had \(\Delta B = 0\), the fraction of the matter density in the universe which is represented by baryons would be very much smaller than \SI{5}{\percent}. 

So far, there is no experimental evidence for large antimatter regions.
If it existed, it would have to be separated from regular matter by extremely large distances, since otherwise we would detect a gamma-ray background from the common annihilation. 

\subsection{Sakharov's conditions}

They are necessary conditions for the creation of baryon asymmetry starting from a baryon-symmetric situation. 
They are: 
\begin{enumerate}
    \item Baryon number non-conservation;
    \item \(C\) and \(CP\) violation;
    \item out-of-equilibrium processes.
\end{enumerate}

The first condition arises because if all processes conserved baryon number then there could be no formation of asymmetry. 
The second condition arises because if there was charge-conjugation symmetry then processes generating matter and antimatter would have the exact same properties, and would therefore statistically occur in the same amounts.
Out-of-equilibrium processes are needed since otherwise, because of \(CPT\) symmetry, \(\Delta B\) increasing and decreasing processes would occur in the same amounts. 

\subsection{Baryogenesis in a GUT}

Let us suppose that there is a GUT scalar particle \(X\) which has two decay channels, either \(qq\) or \(\overline{q} \overline{\ell}\), where \(q\) is a quark and \(\ell\) is a lepton. 

These are both \(B\)-violating, the first introduces \(B = +2/3\) and the second introduces \(B = - 1/3\).
Let us suppose that the branching ratios are \(\tau \) and \(1 - \tau \) respectively. 

The antiparticle \(\overline{X}\) can decay into \(\overline{q} \overline{q}\) or \(q \ell\), which have \(B = - 2/3\) and \(B = + 1/3\) and branching ratios \(\overline{\tau}\) and \(1 - \overline{\tau}\).

If we have \(C\) violation we can see \(\tau \neq \overline{\tau}\). 

\end{document}