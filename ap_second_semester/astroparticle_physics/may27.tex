\documentclass[main.tex]{subfiles}
\begin{document}

\section{Symmetry breaking and phase transitions}

\marginpar{Sunday\\ 2020-7-5, \\ compiled \\ \today}

Let us consider the time, in the early universe, before BBN. The times are 
\begin{figure}
\centering 
\begin{tabular}{ccc}
Temperature & Symmetry group & time \\
\hline
\SI{e19}{GeV}  & \(G _{\text{GUT}}\) & \SI{e-43}{s} \\
\num{e15} \(\divisionsymbol\) \SI{e16}{GeV} & \(G _{\text{GUT}}\), barely & \SI{e-38}{s} \\
% \num{e2} \(\divisionsymbol\) \SI{e3}{GeV} & \(SU(3)_c \times SU(2)_L \times U(1)_Y\) & \num{e-10} \(\divisionsymbol\) \SI{e-12}{s} \\
\num{e2} \(\divisionsymbol\) \SI{e3}{GeV} & \(SU(3)_c \times SU(2)_L \times U(1)_Y\), barely & \num{e-10} \(\divisionsymbol\) \SI{e-12}{s} \\
\num{200} \(\divisionsymbol\) \SI{300}{MeV} & \(SU(3)_c \times U(1) _{\text{em}}\): quark-gluon confinement & \num{e-4} \(\divisionsymbol\) \SI{e-5}{s} \\
\SI{1}{MeV} & \(SU(3)_c \times U(1) _{\text{em}}\): nucleosynthesis & \SI{1}{s} 
\end{tabular}
\label{tab:phase-transitions}
\caption{Phase transitions in the early universe.}
\end{figure}

The time shortly thereafter, from 1 to \SI{100}{s}, is nucleosynthesis, the earliest time we can see; we do so by exploring the abundances of light nuclei. 
Everything before comes out of our model for particle physics. 

After BBN, we have an epoch of radiation dominance, followed by matter dominance, then recombination, accelerated expansion. 

We will not consider these, but instead keep to the first second. 

The properties of the cosmic medium can either change in a \textbf{smooth crossover}, or abruptly in a \textbf{phase transition}.
Formally, the latter corresponds to an order parameter being zero in a phase and nonzero in another. 

The order phase transitions describes the first derivative order which is discontinuous at them: so, first-order transitions are the sharpest, they have latent heat; while second-order transitions are smoother. 

We get phase transitions when there is a mismatch between the ground state at zero temperature and at some finite temperature. Let us denote by \(\expval{\phi }_{T}\) the absolute minimum of the effective potential \(V  _{\text{eff}} (T, \phi )\). 

Sometimes, the ground state being \(\phi = 0\) is restored at a finite temperature. 

First-order phase transitions have a discontinuity in \(\expval{\phi }_{T}\) at a certain \(T = T_C\); for second-order phase transitions it is continuous but not differentiable. 

An example of a first-order transitions is the boiling of a liquid, second-order ones are the transition of a ferromagnet, of a superconductor, of a superfluid.

For second-order and smooth-crossover transitions the medium may always be at equilibrium, on the other hand for first-order transitions switching to the other minimum becomes thermodynamically favorable if the temperature crosses the critical point: so, we can get \textbf{bubble nucleation}. 

The expectation value of the field becomes inhomogeneous in space. We get bubbles with the new vacuum, for which it is energetically favorable to expand. 

The transition being energetically favorable is not enough: the nucleation rate must be higher than the expansion of the universe, otherwise the transition does not complete. 

\subsection{The electroweak phase transition}

The potential for the Higgs field is given by 
%
\begin{align}
V(\phi ) = \mu^2 \abs{\phi }^2 + \lambda \abs{\phi }^{4}
\,,
\end{align}
%
with \(\mu^2<0\) and \(\lambda >0\). Then, the VEV is 
%
\begin{align}
\expval{\phi }_0 = \sqrt{\frac{- \mu^2}{2 \lambda }} = \frac{v}{\sqrt{2}}
\,,
\end{align}
%
so at zero temperature the potential can be written as 
%
\begin{align}
V(\phi) = \lambda \qty(\phi^2 - \frac{v^2}{2})^2
\,.
\end{align}

What do we write for a nonzero temperature? The term to add is: 
%
\begin{align}
V &= V(T = 0) + \frac{T^2}{24} \qty[ \sum _{\text{bosons}} g_i m_i^2 (\phi ) + \frac{1}{2} \sum _{\text{fermions}} g_i m_i^2 (\phi )]  \\
&= \qty(- \lambda v^2 + \frac{\alpha}{24} T^2) \phi  + \lambda \phi^{4}
\,,
\end{align}
%
where \(m_i (\phi ) = h_i \phi \), \(h_i\) being the coupling constant of that specific fermion or boson to \(\phi \). 
The sum over the bosons runs over the Higgs itself as well: so, we get a term \(g_h h_h^2 H^2\) as well. 
Finally, we defined 
%
\begin{align}
\alpha = \sum _{\text{bosons}} g_i h_i^2 (\phi ) + \frac{1}{2} \sum _{\text{fermions}} g_i h_i^2 (\phi )
\,.
\end{align}

So, we can see that when the coefficient of \(\phi^2\) becomes larger than \(0\) the transition takes place: this will happen at \(\alpha T^2 / 24 = \lambda v^2\), which means 
%
\begin{align}
T_C = 2 v \sqrt{ \frac{6\lambda}{\alpha }} 
= 2 \sqrt{ \frac{M_H^2}{4 M_W^2 + 2 M_Z^2 + 4 m_t^2 + M_H^2}} v 
\approx \SI{146}{GeV}
\,.
\end{align}

This is a tree-level perturbative result: if we account for nonperturbative effects we find \(T_C \approx \SI{160}{GeV}\). 

If we were to use the perturbative approach (at one loop) we would find that the electroweak phase transition is of first order, however using a nonperturbative one we get that it is smoother. 
Perturbing would work for small \( M_H / M_W \), but actually this is larger than 1. 

Working nonperturbatively, one finds that the transition is of second order or even smoother. 

\subsubsection{Phase transitions and inflation}



\section{Cosmic matter-antimatter asymmetry}


\end{document}