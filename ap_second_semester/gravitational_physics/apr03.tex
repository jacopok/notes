\documentclass[main.tex]{subfiles}
\begin{document}

\subsection{Frequency evolution and time to coalesce}

\marginpar{Friday\\ 2020-4-3, \\ compiled \\ \today}

% As we were discussing yesterday, we can approximate every orbit as a circular one. 
From the equation for the evolution of the binary's radius \eqref{eq:radius-evolution-binary}, its explicit expression in terms of \(\omega_{s}\) \eqref{eq:derivative-radius-evolution-binary} and the expression for the gravitational wave emitted power \eqref{eq:gravitational-wave-total-power-emitted} we can write 
%
\begin{align}
- \frac{2}{3} \frac{\dot{\omega}_{s}}{\omega_{s}} R &= - \frac{R^2}{GM \mu } \dot{E}_{GW} \\
\dot{\omega}_{s} &= \frac{3}{\mu } \qty(GM)^{-2/3} \omega_{s}^{1/3} 
\frac{32}{5} \frac{c^{5}}{G} 
\qty( \frac{G M_c \omega_{GW}}{2 c^{3}})^{10/3} \marginnote{Substituted in \(\dot{E}\), used \(R = (GM)^{1/3} \omega_{s}^{-2/3} \).}  \\
&= \frac{6}{5} \sqrt[3]{2} \qty(\frac{M_c G}{c^{3}})^{5/3} \omega_{GW}^{11/3}
\,.
\end{align}

If we substitute in \(\omega_{GW}\) from \(\omega_{s}\)
we finally get the relation 
%
\begin{align}
\dot{\omega}_{GW} = \frac{12}{5} \sqrt[3]{2}
\qty(\frac{M_c G}{c^{3}})^{5/3} \omega_{GW}^{11/3}
\qquad \text{or} \qquad
\dot{f}_{GW} = \underbrace{\frac{96}{5} \pi^{8/3} \qty(\frac{M_c G}{c^{3}})^{5/3}}_{k}
f_{GW}^{11/3} 
\,.
\end{align}

This can be integrated directly, by separating the variables as \(f^{-11/3} \dd{f} = \dd{t}\): the result is 
%
\begin{align}
- \frac{3}{8k} f^{-8/3} = t - t _{\text{coal}} 
\overset{\text{def}}{=} -\tau 
\,,
\end{align}
%
where \(t _{\text{coal}}\) is an integration constant, chosen so that \(t = t _{\text{coal}}\) when the frequency diverges.
We can get the frequency as a function of the time until coalescence \(\tau \):
%
\begin{align}
f_{GW} = \qty( \frac{3}{8 k \tau })^{3/8} 
= \frac{1}{\pi } \qty( \frac{5}{256 \tau })^{3/8} \qty(\frac{c^3}{G M_c})^{5/8}
\,.
\end{align}

\todo[inline]{Wrong sign in the slides at this point.}

Inverting this expression we can get the time until coalescence from the frequency: 
%
\begin{align}
\tau = \frac{5}{256} \qty( \frac{1}{\pi f_{GW}})^{8/3} \qty( \frac{c^3}{G M_c})^{5/3}
\,.
\end{align}

Now, we want to get an expression for the \textbf{evolution of the radius}. We know that \(f_{GW} \propto \tau^{-3/8}\), 

We can also get an expression for the radius at a given time from coalescence: 
%
\begin{align}
R(\tau ) = R_0 \qty(\frac{\tau }{\tau_0 })^{1/4} = R_0 \qty(\frac{t _{\text{coal}} - t}{t _{\text{coal}} - t_0 })^{1/4}
\,,
\end{align}
%
where \(\tau_0 \) is the time to coalescence at \(t_0 \).

If we plot this, it has a ``plunge'' phase, up to which we can trust our plot. 

\subsection{Chirping waveform}

The \emph{phase} is the argument of the cosine, so we write \(\cos(\phi (t))\). 
The angular frequency is given by \(\omega_{GW} = \phi'\).

The chirping waveform cannot be trusted near the end. 
We know that if 
%
\begin{align}
R < R _{\text{ISCO}} = \frac{6GM}{c^2}
\,
\end{align}
%
then orbits are not stable. This formula only holds for extreme mass ratios (we actually could have these SMBHs merging with solar mass ones!). 
Anyway, we use it as a guideline to see when our approximations break down. 

The shape of the chirping waveform is basically correct; it goes out of phase with the numerical relativity calculation, but it works somewhat. 

\todo[inline]{Numerical relativity has a \emph{lower} frequency than the quadrupole approx!}

What about eccentric binaries? 
We can also analyze them. The formula is 
%
\begin{align}
\dv{E}{t} = \frac{32}{5} \frac{G \mu^2}{c^{5}} a^{4} \omega_0^{6} f(e)
\,,
\end{align}
%
where 
%
\begin{align}
f(e) = \frac{1}{(1 - e^2)^{7/2}} \qty(1 + \frac{73}{24} e^2 + \frac{37}{96} e^{4}) \geq 1 
\,.
\end{align}

We have emission at frequencies other than the orbital one; also, the GW emission has the effect of circularizing the orbit. 
So, we usually observe circular systems. 

\section{Hulse-Taylor binaries}

It is debatable whether the observation of this was the first observation of gravitational waves. 

This is a binary system in which one star is a pulsar. 

What is a pulsar? It's a kind of neutron star.
Not a moral judgement, but you are completely empty.

A pulsar has a large magnetic field; at a distance \(r_c = c/\omega \) the field lines cannot close so a radio beam escapes. 
This provides a clock!

``Taking the pulse'' of a pulsar: they usually have a certain well-defined shape, if we average over a few periods. 

The procedure is: take signal, FFT to get the fundamental, average over periods. 

The period can then be measured precisely, and we can observe its variations. 

Some relevant frequencies: the radio waves are on the order of \SI{e8}{Hz}, the pulsar's frequency is of the order \SI{10}{Hz}, the frequency of the binary period is \SI{e-5}{Hz}, the motion of the Earth around the Sun at \SI{e-8}{Hz} is also relevant. 

Since the pulsar frequency is very small, we can still average many pulses and still be measuring at what is basically ``a single point''. 



\end{document}
