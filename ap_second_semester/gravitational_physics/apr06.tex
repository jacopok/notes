\documentclass[main.tex]{subfiles}
\begin{document}

\subsubsection{Rømer delay (observer)}

\marginpar{Monday\\ 2020-4-6, \\ compiled \\ \today}

% We are discussing the Hulse-Taylor pulsar, which is a very precise clock.
% There are many effects by which the time-of-arrival is shifted, if we take them all out we can get the effect from the source. 

This effect is denoted as \(\Delta_{R, {\odot}}\), it is due to the fact that if Earth is on the far side of the Sun the radio signal takes longer to get to it than it would if it were on the near side to the pulsar. The simplest way to model it is  
%
\begin{align}
\Delta_{R, {\odot}} = t_0 \cos(\Omega t - \lambda ) \cos(\beta )
\,,
\end{align}
%
where \(t_0 = R / c\), \(R\) being the radius of the Earth's orbit (here taken to be circular), while \(\lambda \) and \(\beta \) define the position of the source in the orbital plane. 

Proper modelling would require us to also take into account the ellipticity of the orbit and the Earth's rotation. 
In general, the effect can be written as 
%
\begin{align}
\Delta_{R, {\odot}} = - \frac{\vec{r} _{\text{ob}} \cdot \hat{n}}{c}
\,,
\end{align}
%
where \(\hat{n}\) denotes the direction of the source in barycentric coordinates for the solar system, while \(\vec{r} _{\text{ob}}\) is the vector connecting the barycenter of the Solar System to the observer, which can be decomposed as a sum of vectors to the Sun's center, Earth's center and finally to the observer:
%
\begin{align}
\vec{r} _{\text{ob}} = \vec{r}_{\text{ob}, \oplus} + \vec{r}_{\oplus, \odot} + \vec{r}_{\odot, \text{barycenter}}
\,.
\end{align}

Modeling this allows us get information about \(\lambda \) and \(\beta \), which define \(\hat{n}\). 
The \textbf{effect size} is dominated by the Earth's motion around the Sun, and it has a magnitude of about \(\pm \SI{e3}{s}\) in the span of six months.

\subsubsection{Shapiro delay (observer)}

% We will have the Shapiro delay from the Sun, the Einstein delay at the receiver from the curvature of the spacetime around the Earth, which is given by 

This effect is due to the curvature introduced by the gravitational potential of the Sun. 
The Schwarzschild metric in Cartesian coordinates reads: 
%
\begin{align}
\dd{s^2} &= - \qty(1 + 2 \phi (x)) c^2 \dd{t^2} + \frac{1}{\qty(1 + 2 \phi (x))} \dd{\vec{x}}^2  \\
&\approx - \qty(1 + 2 \phi (x)) c^2 \dd{t^2} + \qty(1 - 2 \phi (x)) \dd{\vec{x}}^2
\,,
\end{align}
%
where \(\phi (x) = -GM / \abs{x}c^2 \ll 1\) is the gravitational potential in the weak-field approximation.
Photons travel along null geodesics, so they have \(\dd{s}^2 = 0\): this means that 
%
\begin{align}
c\dd{t} = \pm\sqrt{\frac{1 - 2 \phi }{1 + 2 \phi }} \abs{\dd{x}}
\approx \pm \qty(1 - 2 \phi ) \abs{ \dd{x}} 
\,,
\end{align}
%
so we can compute the Shapiro delay as the correction to the total travel time:
%
\begin{align}
c \qty(t _{\text{obs}} - t _{\text{emit}}) 
&= \int_{r _{\text{obs}}}^{t _{\text{emit}}} \abs{ \dd{x}} \abs{1 - 2 \phi (r)} 
\,,
\end{align}
%
which automatically incorporates the Rømer delay, since we find 
%
\begin{align}
t _{\text{obs}} - t _{\text{emit}} =  \underbrace{\frac{\abs{\vec{r} _{\text{emit}} - \vec{r} _{\text{bary}}}}{c}}_{\text{simple propagation}} + \underbrace{\frac{\vec{r} _{\text{obs}} \cdot \hat{n}}{c}}_{\text{Rømer delay}}
\underbrace{- \frac{2}{c} \int_{r _{\text{obs}}}^{r _{\text{emit}}} \abs{ \dd{x}} \phi (x)}_{\text{Shapiro delay: } \Delta_{S, \odot}}
\,.
\end{align}
%
\todo[inline]{The sign of the Rømer effect is different here!}

Let us explicitly compute \(\Delta_{S, \odot}\): we consider a reference frame in which the Sun is at the center, the Earth is on the \(x\) axis at a distance \(r_{\oplus \odot}\), while the pulsar is at a distance \(\rho \) from the Earth and at an angle \(\theta \) with respect to the Sun-Earth axis: so, the Sun-pulsar distance \(r\) is given by 
%
\begin{align}
r^2 = \qty(r_{\oplus \odot} + \rho \cos \theta )^2 + \qty(\rho \sin \theta )^2
= r_{\oplus \odot }^2 \qty(u^2 + 2 u \cos \theta + 1)
\marginnote{Used \(\cos^2\theta+ \sin^2\theta = 1\).}
\,,
\end{align}
%
where \(u = \rho / r_{\oplus \odot}\).
The delay is given by
%
\begin{align}
\Delta_{S, \odot} &= - \frac{2}{c} \int_{r _{\text{obs}}}^{r _{\text{emit}}} \abs{ \dd{x}} \phi (x)
= \frac{2}{c} \int_{0}^{d} \dd{\rho } \frac{G M_{\odot}}{ c^2 r}  \\
&= \frac{2 G M_{\odot}}{c^3} \int_{0}^{d / r_{\oplus \odot} }
\dd{u} \frac{r_{\oplus \odot}}{r}   \\
&= \frac{R_{S, \odot}}{c} \int_{0}^{d / r_{\oplus \odot} }
\dd{u} \frac{1}{\sqrt{u^2 + 2 u \cos \theta  +1}}
\,,
\end{align}
%
where \(R_{S, \odot}\) is the Sun's Schwarzschild radius, while \(d\) is the distance from the Earth to the pulsar.
We used the fact that \(\abs{ \dd{x}} = \dd{ \rho }\): the differential \(\abs{\dd{x}}\) represents the \emph{coordinate distance} along the path from the Earth to the pulsar, which is linearly parametrized by \(\rho \); \(r\) on the other hand can be used to parametrize the curve but that would be a \emph{nonlinear} parametrization, not useful for our purposes.

We need to estimate this integral; it is a divergent one but the divergence is only logarithmic. It is reasonable to estimate it assuming \(d/r_{\oplus \odot} \gg 1\) --- realistically this is of the order of \num{e9}, since the distance to the pulsar is of the order \SI{5}{kpc} \cite[pag.\ L53]{hulseDiscoveryPulsarBinary1975}.


So, we add and subtract \(1/ \sqrt{u^2 + 1}\) in the integrand: then we find 
%
\begin{align}
\Delta_{S, \odot} = \frac{R_{S, \odot}}{c} \qty[ 
    \int_{0}^{d/r_{\oplus \odot}} \dd{u} \frac{1}{\sqrt{u^2+ 1}}
    +
    \int_{0}^{d/r_{\oplus \odot}} \dd{u} 
    \frac{1}{\sqrt{u^2 + 2 u \cos \theta +1}}
    -\frac{1}{\sqrt{u^2+ 1}}
]
\,,
\end{align}
%
where the first integral is a hyperbolic arcsine, which can be estimated by \(\log (2 d/ r_{\oplus \odot})\); the second term can be estimated with the following 
\begin{claim}
\begin{align}
\int_{0}^{ \infty } \dd{u} \qty[
\frac{1}{\sqrt{u^2 + 2 u \cos \theta +1}}
-\frac{1}{\sqrt{u^2+ 1}}]
= - \log \qty(1 + \cos \theta )
\,.
\end{align}
\end{claim}

\begin{proof}
Mathematica says so.
\end{proof}

Then finally our estimate is 
%
\boxalign{
\begin{align}
\Delta_{S, \odot} = \frac{R_{S, \odot}}{c} \qty(\log ( \frac{2d}{r_{\oplus \odot}}) - \log (1 + \cos \theta ) )
\,,
\end{align}}
%
where \(\theta \) varies seasonally, as the Earth moves around the Sun, while \(d\) is basically constant. 

It might seem that this can diverge as \(\theta \to \pi \), but this is not actually the case: that would correspond to the radiation coming through the center of the Sun, which it cannot. Actually, there is a range of values of \(\theta \) around \(\pi \) for which the radiation cannot reach us since the Sun is in the way. The maximum value of \(\theta \) that can be reached, as the radiation is tangent to the surface of the Sun, is called the grazing angle \(\theta_{g}\). 
This angle can be estimated by \(\theta_{g} \approx \pi - R_{\odot} / r_{\oplus \odot}\). 

The delay is maximal when the radiation grazes the Sun, and minimal when it is coming from the opposite direction as the Sun: \(\theta = 0\).
So, the \textbf{maximum size of the effect} is given by the difference 
%
\begin{align}
\Delta_{S, \odot} (\theta = \theta_{g}) - 
\Delta_{S, \odot} (\theta = 0) 
&= \frac{R_{S, \odot}}{c} \qty[
    \log( \frac{2d}{r_{\oplus \odot}} )
    - \log (1 + \theta_{g})
    -\log( \frac{2d - 2 r_{\oplus \odot}}{r_{\oplus \odot}})
] \marginnote{\(\log 1 = 0\).}  \\
&\approx - \frac{R_{S, \odot}}{c}
\log (1 + \cos(\pi - \frac{R_{\odot}}{r_{\oplus \odot}}))  \\
&\approx - \frac{R_{S, \odot}}{c}
\log (1 - 1 + \frac{1}{2}\qty(\frac{R_{\odot}}{r_{\oplus \odot}})^2)  
\marginnote{Used \(\cos(\pi - x) = - \cos(x)\) and expanded.}
\\
&\approx - \frac{R_{S, \odot}}{c}
\log (\frac{1}{2}\qty(\frac{R_{\odot}}{r_{\oplus \odot}})^2)
\approx + \SI{56}{\micro s}
\,,
\end{align}
%
where we discarded the difference of the two logarithms with \(d\) since the relative difference between their arguments is on the order \num{e-9}.
\todo[inline]{In the slides this is reported as being on the scale of \SI{100}{\micro s} with the logarithm's argument being \(2 r_{\oplus \odot} / d\): I do not see how that could come about, and a dependence on \(d\) seems not to make sense physically.}  

\todo[inline]{Also, the slide is repeated twice.}

\subsubsection{Einstein delay (observer)}

This effect is about the gravitational shift and \emph{transverse} Doppler shift due to gravitational field of the Earth. We start from the proper time at the observer:
%
\begin{align}
c^2 \dd{\tau^2} &\approx \qty(1 + 2 \phi (x _{\text{obs}})) \dd{t^2}
 - \qty(1 - 2 \phi (x _{\text{obs}})) \dd{\vec{x}}^2 _{\text{obs}}  \\
\frac{ \dd{\tau^2}}{ \dd{t^2}} &\approx 1 + 2 \phi (x) - \frac{1 - 2 \phi (x)}{c^2} \frac{ \dd{x^2}}{ \dd{t^2}}  \\
&= 1 + 2 \phi (x) - \frac{v^2}{c^2} + \mathcal{O}(\phi v^2 / c^2)
\,,
\end{align}
%
and if we discard the higher order terms (both \(\phi \) and \(v/c\) are small for an Earth-bound observer, so a term containing their product is negligible) we can expand the square root as
%
\begin{align}
\dv{\tau }{t} \approx 1 + \phi \qty(x _{\text{obs}}) - \frac{v^2 _{\text{obs}}}{2c^2}
\,,
\end{align}
%
so 
%
\begin{align}
\tau \approx t + \int^{t} \dd{\widetilde{t}} \qty(\phi \qty(x _{\text{obs}} (\widetilde{t})) - \frac{v^2 _{\text{obs}} (\widetilde{t})}{2c^2})
= t - \Delta_{\oplus, \odot}
\,,
\end{align}
%
where the lower bound of the integration is arbitrary, amounting only to a shift in the origin from which we measure times. 

If most of the velocity is due to the motion of the Earth in its elliptic orbit, from Keplerian dynamics we know that the energy is given by the \emph{vis viva} equation:
%
\begin{align}
E = - \frac{G M \mu }{2 a} = \frac{1}{2} \mu v _{\text{obs}}^2- \frac{GM \mu }{r}
\,,
\end{align}
%
which means that the square velocity can be written as
%
\begin{align}
\frac{v^2 _{\text{obs}}}{2} = \frac{GM}{r} - \frac{GM}{2a}
\,,
\end{align}
%
so that we find 
%
\begin{align}
\dv{ \Delta_{E, \odot}}{t} \approx \frac{v^2}{2c^2} - \phi = \frac{GM_{\odot}}{c^2} \qty( \frac{1}{r} - \frac{1}{2a} + \frac{1}{r})
= R_{S, \odot} \qty( \frac{1}{r} - \frac{1}{4a})
\,,
\end{align}
%
where \(R_{S, \odot} = 2 GM_{\odot} / c^2\) is the Schwarzschild radius of the Sun. 
\todo[inline]{Wrong sign in the slides!}

The term depending on \(r\) is relevant to the modulation of our signal; the other one is a correction to apply to our estimates of the parameters of the binary and pulsar but it is not relevant to the modulation as it is a constant. 

In order to estimate the magnitude of this modulation we need to ask by how much the radius of the Earth's orbit changes: the average orbital distance is around \SI{150e9}{m}, with annual variations of the order of \SI{2e9}{m}. So, the magnitude of the first term is around \num{e-8} seconds per second, with annual variations on the order of \num{e-10} seconds per second.

\todo[inline]{Not sure about the \(\pm \SI{4}{ns} / \SI{}{s}\): that would mean \num{4e-9} seconds per second\dots but the number we get is not the \emph{amplitude} of the oscillation, it is the value of the correction: the oscillation is much smaller, corresponding to the variation of the orbital radius of the Earth!}

\subsubsection{Interstellar medium propagation}

% We must also consider the group velocity of a signal of frequency which travels through the ISM, which is ionized gas.
The ISM is made of ionized gas mostly, and it causes the group velocity of the radio signal to not be precisely \(c\): there is a correction depending on the frequency \(\nu \) of the signal, given by 
%
\begin{align}
v_g \approx c \qty(1 - \frac{n_e e^2}{2 \pi m_e} \frac{1}{\nu^2})
\,,
\end{align}
%
which causes a total time lag in the arrival of a pulse of fixed frequency:
%
\begin{align}
t_d &= \int_{0}^{d} \frac{ \dd{l}}{v_g}
\approx \frac{d}{c} + \frac{e^2}{2 \pi m_e} \frac{1}{\nu^2} \int_{0}^{d} n_e \dd{l}
\marginnote{First order approximation of \(1/(1-x) \sim 1+x\).} \\
&= \frac{d}{c} + \frac{e^2}{2 \pi m_e c} \frac{1}{\nu^2} DM
\,,
\end{align}
%
where \(DM\) is the Dispersion Measurement: it corresponds to the column number density of electrons between us and the pulsar, and it has the units of an inverse area.

For the HT pulsar, the bandwidth of the interesting signal is around \SI{4}{MHz}, and from the top to the bottom of the band the time delay is of the order of \SI{70}{ms}!

Fortunately we can measure precisely in the spectral domain: the arrival times for spectral bands which are near to each other will be very close, and they will form a continuous line:
so, we can ``connect the dots'' to find what corresponds to a single pulse.
The process of doing this and correcting for the DM is called \textbf{de-dispersion}; after doing it we can sum all the components in order to get the original pulse. 

\subsubsection{Combination of local effects}

Taking all the effects pertaining to the source into account, we get 
%
\begin{align}
t _{\text{ssb}} = \tau - \frac{D}{\nu^2} + \Delta_{E, {\odot}} - \Delta_{S, {\odot}} + \Delta_{R, {\odot}}
\,,
\end{align}
%
where \(t _{\text{ssb}}\) means ``time in solar-system barycenter'', and
%
\begin{align}
D = \frac{e^2}{2 \pi m_e c} DM
\,.
\end{align}

\subsubsection{Einstein delay at the source}

We need to look at the gravitational time delay at the source: the gravitational field will be able to be written as 
%
\begin{align}
\phi (x) = 
- \frac{G m_p}{c^2 \abs{\vec{x} - \vec{x}_p}}
- \frac{G m_c}{c^2 \abs{\vec{x} - \vec{x}_c}}
\,,
\end{align}
%
% there is a contribution from the gravitational field of the NS itself, which is hard to calculate but constant, so we do not worry about it. 
where \(p\) and \(c\) denote ``pulsar'' and ``companion'' respectively. 
When we compute the delay due to the source we cannot apply the weak-field approximation, which makes the calculation hard; however, this effect gives a constant frequency shift, which we cannot observe.

The potential from the companion, on the other hand, modulates the signal and is therefore observable. So, like we did for the Earth-Sun system we can write the proper-to-coordinate time ratio as
%
\begin{subequations}
\begin{align}
\dv{T}{t} &= 1 - \frac{G m_c}{c^2 \abs{x_p - x_c}} - \frac{v_p^2}{2 c^2}  \\
\dv{T}{u} &\approx \frac{P_b}{2 \pi } \qty(1 - \frac{G}{c^2} \frac{2 m_c m_p + 3 m_c^2}{2 a (m_p + m_c)})\qty(1 - e \cos u \qty(1 + \frac{G}{c^2} \frac{m_c (m_p + 2 m_c)}{a (m_p + m_c)}))  \\
&= \frac{P_b}{2\pi } \qty(1 - e \cos u) - \gamma \cos u
\,,
\end{align}
\end{subequations}
%
where we plugged in the equation for relativistic eccentric motion; \(u\) is the adimensional angular parameter describing the orbit, which is related to the coordinate time by \(u - e \sin u = (2 \pi / P_b) (t - t_0 )\), while \(e\) is the eccentricity.
This expression is post-Newtonian: it is valid up to first order in \(G\).
Its simpler formulation is given in terms of the \textbf{Einstein parameter} 
%
\begin{align}
\gamma = e \frac{P_b}{2 \pi } \frac{G}{c^2} \frac{m_c (m_p + 2 m_c)}{a (m_p + m_c)}
= e \qty(\frac{P_b}{2 \pi })^{1/3} \frac{G^{2/3}}{c^2}  \frac{m_c (m_p + 2 m_c)}{(m_p + m_c)^{4/3}}
\,.
\end{align}

What is the magnitude of this effect? Well, we define the Einstein time difference \(\Delta_{E}\) by \(T = t - \Delta_{E}\); we can differentiate this equality with respect to \(u\) to find 
%
\begin{align}
\dv{ \Delta_{E}}{u} = \dv{t}{u} - \dv{T}{u}
= \frac{P_b}{2 \pi } \qty(1 - e \cos u) - 
\frac{P_b}{2\pi } \qty(1 - e \cos u) + \gamma \cos u = \gamma \cos u
\,,
\end{align}
%
where we computed \(\dv*{t}{u}\) starting from \(t = (P_b / 2\pi ) (u - e \sin u) + \const\), which comes from the definition. 
So, the amplitude of the modulation comes from the parameter \(\gamma \), which has the units of a time, roughly on the order of magnitude of the period of the binary multiplied by the ratio between the Schwarzschild radius of the companion and the semimajor axis of the binary: for the Hulse-Taylor pulsar it comes out to be of the order \SI{4.292}{ms}.

\subsubsection{Rømer delay (source)}

Like in the Solar System, we have the Rømer delay, which is given in general by 
%
\begin{align}
\Delta_{R} = \frac{\hat{z} \cdot x_{pb}}{c}
\,,
\end{align}
%
where \(\hat{z}\) is the direction of the Earth, while \(x_{pb}\) is the vector connecting the binary barycenter to the pulsar. 

The polar coordinates \(r_1 \) and \(\psi \) for a Keplerian orbit are given by
%
\begin{align}
r_{pb} = r_1 = a_1 \qty(1 - e \cos u )
\qquad \text{and} \qquad
\cos \psi = \frac{\cos u - e}{1- e \cos u}
\,,
\end{align}
%
where \(u\) is the angular parameter of the orbit, while \(a_1 \) is its semimajor axis.

The Rømer delay can then be written in terms of the \emph{observation angle} \(\iota \) (the angle between the observation direction and the normal to the orbital plane) and the \emph{argument of periapsis} \(\omega\) (the angle of the orbit at the point in which the body is closest to us --- this is simply \(\pi /2\) for circular orbits, it can differ for elliptic ones):
%
\begin{align}
\Delta_{R} = r_1 \sin \iota \sin(\omega+\psi ) 
&= r_1 \sin \iota \qty(\cos \psi \sin \omega + \cos \omega \sin \psi )  \\
& =r_1 \sin \iota \qty( (\cos u - e) \sin \omega + \sqrt{1 - e^2} \sin u \cos \omega )
\,.
\end{align}
%
% where \(\iota \) is the observation angle, while \(\psi \) is the angle from the line of notes (see drawing). 

Note that if \(\psi \) is measured from the \emph{line of nodes}, the line where the orbital plane crosses the plane drawn from the barycenter and normal to the observation direction, \(\omega + \psi \) gives the angle of the pulsar from the line of nodes.

This is all classical --- the relativistic corrections, however, are large. 
We will not do the calculation, instead we show the result: it is written similarly,
%
\begin{align}
\Delta_{R} = a_1 \sin \iota 
\qty((\cos u - e_r ) \sin \omega + \sqrt{1 - e^2_{\theta }} \sin u \cos \omega )
\,, 
\end{align}
%
where 
%
\begin{subequations}
\begin{align}
e_{r, \theta } &= e (1 + \delta_{r, \theta })  \\
\delta_{r} &= \frac{G}{c^2} \frac{3 m_p^2 + 6 m_p m_c + 2 m_c^2}{a (m_p + m_c)}  \\
\delta_{\theta } &= \frac{G}{c^2} 
\frac{(7/2) m_p^2 + 6 m_p m_c + 2 m_c^2}{a (m_p + m_c)}
\,.
\end{align}
\end{subequations}

To get a feeling for the numbers: if the masses of the objects are comparable and similar to \(m\), the order of magnitude of the corrections is given by the ratio \(G m / a c^2\), the ratio of the Schwarzschild radius corresponding to the mass to the binary's radius.

Also, note that here the advance of the periastron is much more significant than it is for Mercury, it is given by 
%
\begin{align}
\expval{\dot{\omega}}
= \frac{3G }{c^2} \qty(m_p +m_c)^{2/3} \qty( \frac{2 \pi }{P_b})^{5/3}
\frac{1}{1 - e^2} \sim \SI{4.2}{\degree / yr}
\,.
\end{align}

\subsubsection{Shapiro delay (source)}

The computation to get the Shapiro delay at the source is similar to that of the Solar System.  
We only need to account for the companion's gravitational field, the pulsar's is included in the Einstein delay.

The expression comes out to be 
%
\begin{align}
\Delta_{S} = - 2 \frac{G m_c}{c^3} \log
((1 - e \cos u) - \sin \iota \qty(\sin \omega \qty(\cos u - e) + \sqrt{1 - e^2} \cos \omega \sin u)) 
\,.
\end{align}

To get all the corrections we need to account for the \emph{relativistic aberration} due to the motion of the source, as well as the \emph{Doppler shift} due to the relative motion of the barycenters of the Solar System and the binary system.

% If we get all the Keplerian parameters and two of the post-Newtonian ones then we should know everything.

\subsection{Putting all the parameters together}

The parameters of the system can be categorized into: 
\begin{enumerate}
    \item the pulsar's parameters: position, frequency, phase, spin-down and others;
    \item the Keplerian parameters of the orbit: \(P_b, T_0, x= (a/c) \sin \iota , e, \omega \);
    \item the relativistic corrections to the orbit: \(\dot{\omega}
    , \gamma, P_b, r, s, \delta_{\theta }, \dot{e}, \dot{x}\).
\end{enumerate}

The system is fully characterized by all the classical parameters, plus two of the relativistic ones. 
If we can measure three relativistic parameters, we can use the third as a GR test.

We extract from the data the classical parameters, the precession of the periastron \(\expval{\dot{\omega}}\) and the Einstein parameter \(\gamma \). Then,  we use the derivative of the orbital period \(\dot{P}\) to test GR.
% We measure \(P_b, T_0, x= a \sin \iota /c, e, \omega \) and the post-Newtonian \(\dot{\omega}, \gamma \) and finally we make a prediction for \(\dot{P}\). This matches the data very well.

The prediction from the calculations is 
%
\begin{align}
\dot{P}_{b, GR} = - \frac{192 \pi G^{5/3}}{5 c^{5}}
\frac{m_p m_c}{(m_p + m_c)^{1/3}} \qty( \frac{P_b}{2 \pi })^{- 5/3}
\frac{1}{(1 - e^2)^{7/2}} \qty(1 + \frac{73}{24} e^2 + \frac{37}{96}e^{4}) \approx -\num{2.40242(2)e-12}
\,,
\end{align}
%
and the measured value is \(- \num{2.4056(51)e-12}\).

\section{GW from a rotating rigid body}

The moment of inertia tensor can be defined as 
%
\begin{align}
I^{ij} = \int \dd[3]{x} \rho (x) \qty(r^2 \delta^{ij} - x^{i} x^{j})
\,.
\end{align}

There exists a frame in which this tensor is diagonal, its eigenvalues are the moments of inertia, its eigenvectors are the axes of inertia.
They are then defined by equations like 
%
\begin{align}
I_1 = \int \dd[3]{x} \rho \qty(x_2^2 + x_3^2) 
\,.
\end{align}
 
For an ellipsoid with axes \(a, b, c\) and mass \(m\) we have 
%
\begin{align}
I_1 = \frac{m}{5} \qty(b^2 +c^2)
\,.
\end{align}

The rotational kinetic energy is given by 
%
\begin{subequations}
\begin{align}
E _{\text{rot}} &= \frac{1}{2} I_{ij} \omega_{i} \omega_{j} \\ 
&= \frac{1}{2} I_i \omega_{i}^2
\,,
\end{align}
\end{subequations}
%
where the last equality holds in the body frame. 

Suppose we have a body spinning around an axis, such that the position of any point shifts by a rotation matrix \(R_{ij}\). 

The inertia tensor shifts by \(I \rightarrow R^{\top} I R\).

The tensor we defined before, 
%
\begin{align}
M^{ij} = \frac{1}{c^2} \int \dd[3]{x} T^{00} (x) x^{i} x^{j} 
\approx - I^{ij} + \int \dd[3]{x} \rho (x) r^2 \delta^{ij}
\,,
\end{align}
%
which we can substitute in. This is the trace of the inertia tensor. 

Suppose we had a body whose angular momentum is not aligned with the moment of inertia: we can use Euler angles to express the rotation matrix.


\end{document}
