\documentclass[main.tex]{subfiles}
\begin{document}

\marginpar{Friday\\ 2020-4-10, \\ compiled \\ \today}

Now we are considering a body which is axisymmetric, and rotating along an axis which is not aligned with its axes of inertia. 

An astrophysical example of this will usually look like an ellipsoid.
We should ``clean our minds'' from the idea of a spinning spintop precessing.

Here, the axis going around is the faster motion, the rotation of the body around its axis is slower. 

This wobbling motion is similar to the one of a coin thrown on a table.

If we compute the evolution of the inertial tensor, we get terms both at \(\omega \) and at \(2 \omega \). 

We are interested in the projection of this variation onto the plane orthogonal to the direction of a propagation. 

If we have a distribution which looks like a coin (\(I_1 \sim I_2 \ll I_3 \)) then it looks to us like a binary if we look at it from the top (in terms of periodicity at least), so we expect \(2 \omega \) emission, since the system looks the same to us, since it repeats after a rotation of \(\pi \). 

If, instead, we look at it from the side, the periodicity is the full period: after half a rotation the coin is edge-on, but it appears at two different angles with respect to the vertical direction.

Therefore, we both have \(\omega \) and \(2 \omega \) emission. 

If we were able to determine the amplitude at different inclinations, we would be able to determine the inclination \(\iota \). 

[formula for back reaction is wrong!]

In order to calculate the backreaction we assume that the motion is approximately constant during a single period. 

We find differential equations telling us that \(\dot{\beta}
\) and \(\alpha \) both decrease: the first means that the motion is slowing down; the second means that the wobbling is decreasing, as the rotation is aligning with the angular momentum.

We can define the parameter \(u(t) = \dot{\beta} / \dot{\beta}_{0}\), and a characteristic time \(\tau_0 \): 
%
\begin{align}
\tau_0  = \qty( \frac{2G}{5 c^{5}} \frac{(I_1 - I_3 )^2}{I_1  } \dot{\beta}_{0}^{4})^{-1}
\,,
\end{align}
%
which has a typical value of 
%
\begin{align}
\tau_0 = \SI{1.8e6}{yr} \qty(\num{e-7} \frac{I_3 }{I_1 - I_3 })^2 \qty( \frac{ \SI{1}{kHz}}{f_0 })^{4} \qty(\num{e38} \frac{\SI{}{kg m^2}}{I_1 })
\,,
\end{align}
%
and we can write differential equations for \(\dot{u}\) and \(\dot{\alpha}\): but we must have \(\dot{\beta} \cos(\alpha ) = \const\). 
This implies that the boundary conditions must be \(\alpha_{ \infty } = 0\) and \(u_{ \infty } = \cos \alpha_0 \). Also, asymptotically, 
%
\begin{align}
\dot{\alpha}_{t \to \infty } = \dots
\,.
\end{align}

These conditions do not apply in general, neutron stars are not truly rigid bodies since they have an internal structure. 
In a generic case we will have emission at different frequencies. 

We have not seen pulsars yet in GW, but we can put upper bounds to the amplitude of their emission. 

``Beating the spin-down limit'' means that we know that we would be able to see the GW emission in a certain case if the spin-down was only due to GW.

Could we differentiate a pulsar rotating and seen head-on and a binary system? 
Surely they are phenomena which happen in different frequency ranges, and last very much different times. 
If the binary is spinning at those frequencies it's evolving very rapidly, instead a pulsar can give out a stable signal. 

Also, in full numerical relativity the waveform looks different. 


\end{document}
