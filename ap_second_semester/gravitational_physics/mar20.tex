\documentclass[main.tex]{subfiles}
\begin{document}

\marginpar{Friday\\ 2020-3-20, \\ compiled \\ \today}

\section{The physical effects of gravitational waves}

We want to discuss how we can build instruments which can detect gravitational waves. 

An open question for decades was to see whether the effects of gravitational waves could be removed using a proper gauge choice. 
There was a conference in Chapel Hill (?) which showed examples of non-removable gravitational wave effects, such as the ``beads on a stick'', which move and dissipate energy if a GW passes through them: a non removable effects. 

What happens to free particles in the TT gauge? The geodesic equation for the spatial indices reads 
%
\begin{align}
\dv[2]{x^{i}}{\tau} = - \Gamma^{i}_{\mu \nu } \dv{x^{\mu }}{\tau } \dv{x^{\nu }}{\tau }
\,,
\end{align}
%
where the parameter \(\tau \) parametrize our curve. 
We assume that the particle starts out at rest: then its four-velocity is \(\dv*{x^{\mu }}{\tau } = (\dv*{x^{0}}{\tau }, \vec{0})\). So, we get the simplification 
%
\begin{align}
  \dv[2]{x^{i}}{\tau} =
  - \Gamma^{i}_{00} \qty(\dv{x^{0}}{\tau })^2
\,,
\end{align}
%
and in linearized gravity  
%
\begin{align}
\Gamma^{i}_{00} \approx \frac{1}{2} \qty(2 \partial_0 h^{i}_{0} - \partial^{i} h_{00} ) =0
\,
\end{align}
%
if we use the TT gauge. 
This means that the derivative of the velocity is zero: so, the velocity of a stationary particle remains zero indefinitely. 
Let us consider geodesic deviation between two particles instead: say that the first particle has the geodesic \(x(\tau )\) and the second is \(x(\tau ) + \xi (\tau )\). Their geodesic equations will read 
%
\begin{subequations}
\begin{align}
\dv[2]{x^{\sigma }}{\tau } + \Gamma^{\sigma }_{\mu \nu } (x) \dv{x^{\mu  }}{\tau } \dv{x^{\nu }}{\tau } &= 0 \\
\dv[2]{(x^{\sigma } + \xi (\sigma )) }{\tau } + \Gamma^{\sigma }_{\mu \nu } (x + \xi ) \dv{x^{\mu  } + \xi^{\mu }}{\tau } \dv{x^{\nu } + x^{\nu }}{\tau } &= 0 
\,,
\end{align}
\end{subequations}
%
 which we can expand to first order using the perturbation: \(\Gamma^{\sigma }_{\mu \nu } (x + \xi ) = \Gamma^{\sigma }_{\mu \nu } (x) + \partial_{\gamma } \Gamma^{\sigma }_{\mu \nu } \xi^{\gamma } \), using which and expanding we finally get 
 %
 \begin{align}
 \dv[2]{\xi^{\sigma }}{\tau } + 2 \Gamma^{\sigma}_{\mu \nu } \dv{x^{\mu }}{\tau } \dv{\xi^{\nu }}{\tau } + \xi^{\gamma } \partial_{\gamma } \Gamma^{\sigma}_{\mu \nu } \dv{x^{\mu }}{\tau } \dv{x^{\nu }}{\tau }
 \,,
 \end{align}
 %
 so if we restrict ourselves to only spatial components, and assume that the particles start out stationary we get 
 %
 \begin{align}
 \dv[2]{\xi ^{i}}{\tau } = - 2 \Gamma^{i}_{0 \nu } \dv{x^{0}}{\tau } \dv{\xi^{\nu }}{\tau } + \xi^{\gamma  } \partial_{\gamma } \Gamma^{i}_{00} \dv{x^{0}}{\tau }  \dv{x^{0}}{\tau }
 \,,
 \end{align}
 %
so, using the expressions for the Christoffel symbols in the TT gauge we get 
%
\begin{align}
\dv[2]{\xi^{i}}{\tau } = - 2 c \Gamma^{i}_{0j} \dv{\xi^{j}}{\tau } = -c \partial_0  h^{ij} \dv{\xi^{j}}{\tau }
\,,
\end{align}
%
so, parallel geodesics remain parallel: if the separation initially is stationary, it will remain so. 

The issue is that in the TT gauge we are using a special set of coordinates which ``follow'' the gravitational wave. 
We see no change in coordinate distance since the coordinates are moving around with the gravitational wave: we did a coordinate change using \(\xi^{\mu }\) satisfying \(\square \xi^{\mu } = 0\), so the coordinates are harmonically moving, together with the GW. 

It is like we defined wave-like coordinates, ``gauging away'' the wave-like motion. 

We can overcome this issue by calculating \emph{proper distances} instead of coordinate distances: for two space-like separated objects along the \(x\) axis we get
%
\begin{align}
\dd{s^2} = g_{\mu \nu } \dd{x^{\mu }} \dd{x^{\nu }} = \dd{x^2} + h^{TT}_{\mu \nu } \dd{x^{\mu }} \dd{x^{\nu }} 
\,.
\end{align}

Let us apply this to the case of a GW propagating along the \(z\) axis, for particles separated along the \(x\) axis. The full metric perturbation looks like 
%
\begin{subequations}
\begin{align}
h^{\mu \nu }_{TT} = \left[\begin{array}{cccc}
0 & 0 & 0 & 0 \\ 
0 & h_{+} & h_{ \times} & 0 \\ 
0 & h_{ \times } & - h_{+} & 0 \\ 
0 & 0 & 0 & 0
\end{array}\right] e^{i \omega \qty(t - z/c)}
\,,
\end{align}
\end{subequations}
%
then the distance, in the case of an \(h_{+}\) polarized wave, becomes 
%
\begin{align}
s = (x_1 - x_2 ) \sqrt{1 + h_{+} \cos(\omega t)}
\approx (x_1 -x_2 ) \qty(1 + \frac{1}{2} h_{+} \cos(\omega t))
\,.
\end{align}

So, the fractional change is given by \(h_{+} / 2\). 

For two general events separated by the 4-vector \(L\): 
%
\begin{align}
s^2 = \qty(\eta_{\mu \nu } + h_{\mu \nu } )L^{\mu } L^{\nu }
\approx L \qty(1 + \frac{1}{2} h_{ij} L^{i} L^{})
\,.
\end{align}

We would, however, like to work in coordinates. 
The useful frame to define is the \emph{free falling frame}, whose coordinates are rigid and not perturbed by the GW. 

In order to build such a frame, a local inertial frame which will be inertial, we will need to define 4 orthogonal vectors on the point \(P\): 
%
\begin{align}
\eta_{\mu \nu } e^{\mu }_{\alpha } e^{\nu }_{\beta } = \eta_{\alpha \beta }
\,.
\end{align}

Consider a geodesic through point \(P\) whose tangent vector at \(P\) is a unit vector \(\hat{n}\). If it is spacelike, we parametrize it by \(s\) (defined with \(\dd{s^2}\) from the metric), if it is timelike we parametrize it with \(\tau\) (defined by \(\dd{\tau^2 } = - \dd{s^2}\)).  We call \(\lambda \) a generic one of those. 

Now, the coordinates of point \(Q\) are generically \(\lambda \hat{n}\), if the geodesic starting with unit vector \(\hat{n}\) reaches \(Q\) when its parameter is \(\lambda \). 

We can reach almost every point this way, the points which are only connected through null geodesics can be reached by continuity, and in a small enough region the coordinates of a point \(Q\) are unique --- that is, the geodesics do not cross. 

In this frame, then, \(g_{\mu \nu } (P) = \eta_{\mu \nu } (P)\); also, in the geodesic equation 
%
\begin{align}
\dv[2]{x^{\mu }}{\lambda } + \Gamma^{\mu }_{ \nu \rho } \dv{x^{\nu }}{\lambda } \dv{x^{\rho }}{\lambda }
\,,
\end{align}
%
we have that the second derivatives are zero since \(x^{\mu}(\lambda )\) is linear, so we must have \(\Gamma^{\mu }_{\nu \rho } n^{\nu } n^{\rho } = 0\). This must be true for any unit vector, therefore we have \(\Gamma^{\mu }_{\nu \rho } =0 \). The system giving \(g_{\mu \nu , \rho } (P) \) from \(\Gamma^{\mu }_{\nu \rho }\) is nondegenerate, so we also have \(g_{\mu \nu , \rho }(P) =0\).\footnote{Do note that this reasoning works only at the point, since if we moved along a geodesic we do not have access to the other unit vectors anymore (in these coordinates).}
These are called \textbf{Riemann normal coordinates}. 

The conditions on the metric and its derivatives only hold at the point. We can do slightly better with \emph{Fermi normal coordinates}, where we require a gyroscope's angular momentum to be parallel-transported along the geodesics. 

How do we make such a frame? we use free falling particles: we could put them in orbit. 
This is not actually that simple: consider \emph{drag-free satellites}. 

Consider a particle orbiting the Sun. 
The Sun's radiation pressure pushes the particle away from a geodesic. 

Maybe we could put a thrusted spacecraft around our test mass: it balances the sun's radiation pressure; we measure the distance to the test mass without touching it, and then balance the thrusters by keeping at a constant distance from it.

This is the idea behind LISA. 
The distances between the spacecrafts should be about \SI{5}{Gm} apart.
We do not measure the distances between the spacecrafts, but instead the distances between the test masses inside them, which are \SI{2}{kg}, \SI{4.6}{cm} side, gold-platinum shielded cubes: this is known as \emph{drag-free} navigation. 

The interferometric measurements have pico-meter (\SI{e-12}{m}) sensitivity. 
It takes about \SI{30}{s} for light to move between the mirrors: this time-delayed interferometry needs special consideration. 

We have a \emph{gravitational reference sensor}, a cubic shell around the cube: we keep measuring the distances between the two. 
We also need to discharge the masses, otherwise electrostatic forces are too strong.  
Also, the thrusters need to be very weak, on the order of the \(\SI{}{\micro N }\). 

The LISA Pathfinder mission checked all of these boxes, except for time-delayed interferometry. 
It only used one spacecraft, and measured how well the drag-free navigation worked. 

Using two masses, we need to account for the gravitational pull between them. 

Even accounting for this by relaxing the acceleration precision requirement 10-fold, the results were exceptional. 

Let us discuss the sources of noise: at high frequencies, inertia prevents a force from creating significant displacement. 
This applies to external forces, not to gravitational forces, since the latter are proportional to the mass. 
So, there is a limit at high frequencies because of our inability to measure that fast. 

At low frequencies, it is easy to measure, but it is hard to verify whether the mass is indeed in free fall. 

In the end, it was verified that we can do 
%
\begin{align}
s^{1/2}_{a} \leq \SI{3e-14}{m / s^2 / \sqrt{Hz}}
\,
\end{align}
%
at \SI{1}{mHz}. Solar radiation pressure is two orders of magnitude higher. 

We can have issues with the parasitic coupling of the test mass to the spacecraft. 

Let us now come back to theory by discussing the \emph{proper detector frame}: coordinates defined by a rigid ruler. 
Rigid rulers do not really exist, but we can approximate it well enough. If the gravitational pull is small compared to the restoring forces in the ruler than its length will approximately not change. 

Let us put ourselves in a free falling frame in Fermi local coordinates; in the origin, the metric is always flat. Let us expand to second order in the spatial coordinates 
%
\begin{subequations}
\begin{align}
g_{\mu \nu } (x) &\approx g_{\mu \nu } (0)
+ x^{i} \partial_{i} \eval{g_{\mu \nu }}_{x=0} 
+ \frac{1}{2} x^{i} x^{j} \eval{\partial_{i} \partial_{j} g_{\mu \nu }}_{x=0}  \\
&= \eta_{\mu \nu } + \frac{1}{2} x^{i} x^{j} \eval{\partial_{i} \partial_{j} g_{\mu \nu }}_{x=0}
\,,
\end{align}
\end{subequations}
%
which we can rewrite in terms of the Riemann tensor by making use of the expression of the Riemann tensor in the LIF: we get 
%
\begin{align}
\dd{s^2} \approx - c^2 \dd{t^2} \qty(1 + R_{0i0j} x^{i}x^{j})
-2 c \dd{t} \dd{x^{i}} \qty(\frac{2}{3}R_{0ijk} x^{j}x^{k})
+ \dd{x^{i}} \dd{x^{j}} \qty(\delta_{ij} - \frac{1}{3} R_{ijkl} x^{k} x^{l})
\,,
\end{align}
%
so the corrections are of the order \(\order{r^2 / L_B^2}\), where \(r^2\) is the square distance from the origin, while \(L_B\) is the typical spatial scale of the variation of the metric, such that \(R_{0ijk} = \order{L_B^{-2}}\). 
This \(L_B\) is the wavelength of the GW if we are describing a GW.

So, this coordinate description works as long as the scale of the region we are describing is small compared to the wavelength of the GW. 

\end{document}
