\documentclass[main.tex]{subfiles}
\begin{document}

\subsubsection{Locking and alignment}

So far, we have always assumed that the cavity is at resonance. 

\marginpar{Monday\\ 2020-5-11, \\ compiled \\ \today}

We have an issue if noise is strong enough to move our mirrors out of lock, even if we do not care to observe GW at the frequency of that noise.

Our cavities have a finesse of \(\mathcal{F} \sim 500\), and the free spectral range will be of the order of \(c / 2L \sim \SI{50}{kHz}\). 
This means that the FWHM of the peaks will be of the order of \(\text{FWHM} \approx \text{FSR} / \mathcal{F} \approx \SI{100}{Hz} = \Delta f\).

If the variation of the frequency is due to a variation of the length of the cavity then the relative variations will be equal (at least to the linear level): \(\Delta L / L = \Delta f / f\), which means that, \emph{at the very least}, we will need to control the length of the cavity to the order of 
%
\begin{align}
\Delta L = L \frac{ \Delta f}{f} \approx \SI{e-9}{m}
\,.
\end{align}

In reality, we are able to control arm lengths to within \SI{e-15}{m} (root-mean-square of the position variation). 
There are many sources of noise in this respect: the seismic motion of the ground, the moon's pull, the intrinsic laser noise. 
Fortunately, there is a technique we can use to \textbf{keep} the cavity locked onto the wavelength of the laser. 

The first thing we might try is to measure the transmitted intensity of the laser light from the FP cavity to check whether the length is the correct one.
This has two issues: if the power decreases we cannot tell whether the cavity is slightly too \emph{long} or too \emph{short}; also, we cannot distinguish an intensity fluctuation due to a length imperfection from an intrinsic fluctuation of the laser. 

The solution is the \textbf{Pound-Drever-Hall} technique: an electro-optical modulator is used in order to insert sidebands at \(\omega_{l} \pm \Omega \) by doing phase modulation. 
These can be used as oscillators which detect any departure from resonance: they are \emph{not} at resonance in the cavity, so while a length fluctuation of the cavity affect the carrier frequency a lot, it leaves them basically unchanged. 
So, we see a term oscillating at \(\Omega \) whose amplitude is linear in \(\Delta \phi \), measuring it we can tell the sign of the length variation, and we can distinguish it from a laser power oscillation. 

If we know this, we can then actuate the cavity to follow the laser. 

Also, we need actuators to control the beam position: the angular control we need in order to prevent noise is of the order of \SI{e-9}{rad}.

\subsubsection{Antenna pattern}

The antenna pattern of the interferometer is described by the \textbf{detector tensor} \(D_{ij}\), which transforms the perturbation \(h_{ij}\) into the observed scalar as \(h (t) = D_{ij} h_{ij}(t)\). It is given by 
%
\begin{align}
D_{ij} = \frac{1}{2} \qty(\hat{x}_{i} \hat{x}_{j} - \hat{y}_{i} \hat{y}_{j}) 
\,,
\end{align}
%
so the output of the detector will look like
%
\begin{align}
h(t) = \frac{1}{2} \qty(\ddot{h}_{xx} - \ddot{h}_{yy})
\,,
\end{align}
%
as long as the arms are aligned with the \(\hat{x}\) and \(\hat{y}\) axes. 
The function \(D_{ij}\) describes the sensitivity of our detector in different directions. 

These kinds of detectors return a scalar (a timeseries, yes, but a scalar with respect to 3D space). This scalar will be linear in the tensor \(h_{ij}\), so we can express the observation as 
%
\begin{align}
h(t) = D_{ij} h_{ij} (t)
\,.
\end{align}

We must perform a rotation with two angles \(\phi \) and \(\theta \) to go from the source frame and the detector frame. 

We must make a choice to select what we call \(h_{+}\) and \(h_{ \times }\). In the end, we find 
%
\begin{subequations}
\begin{align}
h(t) &= F_{+} (\theta , \phi ) h_{+}+ F_{\times} (\theta , \phi ) h_{\times}  \\
F_{+} &= \frac{1}{2} \qty(1 + \cos^2\theta ) \cos( 2 \phi )  \\
F_{ \times } &= \cos(\theta ) \sin(2 \phi )
\,.
\end{align}
\end{subequations}

\subsection{The interferometer's noise budget}

The noise is dominated by the quantum noise: quantum fluctuations of the laser light. 
The other source of noise giving us problems in the \SI{100}{Hz} region is the coating Brownian noise. 

At high frequencies, the problem is that it is hard to measure small displacements with small integration time. 
At low frequencies, the problem is that the mirrors move too much. 

\subsubsection{Quantum noise}

The fluctuation in square power --- the \emph{shot noise}, error in the count of photons ---  is given, since it is a Poisson process, by
%
\begin{align}
\Delta P^2 = \frac{\Delta E^2}{T^2} = 
\frac{\Delta N^2 \hbar^2 \omega_{l}^2}{T^2}
= N \frac{\hbar^2   \omega_{l}^2}{T^2}
= \frac{P_0 \hbar \omega_{l}}{T}
= 2\frac{S_P(\omega )}{T}
= \frac{1}{2} \int_{0}^{1/T} S_P (\omega) \dd{\omega }
\,,
\end{align}
%
so we get \(S_P (\omega ) = 2 P_0 \omega_{l}\). 

We also have \emph{radiation pressure}, which scales differently. 
So, we must reach a compromise with both power and finesse. 

The shot noise is flat in frequency, the RP noise decreases with frequency. 
At each frequency, we can define a Standard Quantum Limit, which is the lowest noise we could have at that frequency. 

We can go below this limit using Quantum Vacuum Squeezing. 

\subsubsection{Thermal Noise}

We have contribution from all dissipation sources. 
There is thermo-elastic noise: as a material bends, the side which compresses heats up a little. 

The limit is the internal dissipation: 
%
\begin{align}
S_{F, \text{th}} = 4 k_B T \Re[Z(\omega )]  
\,.
\end{align}

\subsubsection{Seismic noise}

\subsubsection{Newtonian noise}

\end{document}
