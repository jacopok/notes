\documentclass[main.tex]{subfiles}
\begin{document}

\marginpar{Monday\\ 2020-5-11, \\ compiled \\ \today}

We have an issue if noise is strong enough to move our mirrors out of lock, even if we do not care to observe GW at the frequency of that noise.

Today we will discuss the \textbf{antenna pattern}: the detector tensor is given by 
%
\begin{align}
D_{ij} = \frac{1}{2} \qty(\hat{x}_{i} \hat{x}_{j} - \hat{y}_{i} \hat{y}_{j}) 
\,,
\end{align}
%
so that 
%
\begin{align}
h(t) = \frac{1}{2} \qty(\ddot{h}_{xx} - \ddot{h}_{yy})
\,,
\end{align}
%
as long as the arms are aligned with the \(\hat{x}\) and \(\hat{y}\) axes. 
The function \(D_{ij}\) describes the sensitivity of our detector in different directions. 

These kinds of detectors return a scalar (a timeseries, yes, but a scalar with respect to 3D space). This scalar will be linear in the tensor \(h_{ij}\), so we can express the observation as 
%
\begin{align}
h(t) = D_{ij} h_{ij} (t)
\,.
\end{align}

We must perform a rotation with two angles \(\phi \) and \(\theta \) to go from the source frame and the detector frame. 

We must make a choice to select what we call \(h_{+}\) and \(h_{ \times }\). In the end, we find 
%
\begin{subequations}
\begin{align}
h(t) &= F_{+} (\theta , \phi ) h_{+}+ F_{\times} (\theta , \phi ) h_{\times}  \\
F_{+} &= \frac{1}{2} \qty(1 + \cos^2\theta ) \cos( 2 \phi )  \\
F_{ \times } &= \cos(\theta ) \sin(2 \phi )
\,.
\end{align}
\end{subequations}

\subsection{The interferometer's noise budget}

The noise is dominated by the quantum noise: quantum fluctuations of the laser light. 
The other source of noise giving us problems in the \SI{100}{Hz} region is the coating Brownian noise. 

At high frequencies, the problem is that it is hard to measure small displacements with small integration time. 
At low frequencies, the problem is that the mirrors move too much. 

\subsubsection{Quantum noise}

The fluctuation in square power --- the \emph{shot noise}, error in the count of photons ---  is given, since it is a Poisson process, by
%
\begin{align}
\Delta P^2 = \frac{\Delta E^2}{T^2} = 
\frac{\Delta N^2 \hbar^2 \omega_{l}^2}{T^2}
= N \frac{\hbar^2   \omega_{l}^2}{T^2}
= \frac{P_0 \hbar \omega_{l}}{T}
= 2\frac{S_P(\omega )}{T}
= \frac{1}{2} \int_{0}^{1/T} S_P (\omega) \dd{\omega }
\,,
\end{align}
%
so we get \(S_P (\omega ) = 2 P_0 \omega_{l}\). 

We also have \emph{radiation pressure}, which scales differently. 
So, we must reach a compromise with both power and finesse. 

The shot noise is flat in frequency, the RP noise decreases with frequency. 
At each frequency, we can define a Standard Quantum Limit, which is the lowest noise we could have at that frequency. 

We can go below this limit using Quantum Vacuum Squeezing. 

\subsubsection{Thermal Noise}

We have contribution from all dissipation sources. 
There is thermo-elastic noise: as a material bends, the side which compresses heats up a little. 

The limit is the internal dissipation: 
%
\begin{align}
S_{F, \text{th}} = 4 k_B T \Re[Z(\omega )]  
\,.
\end{align}

\subsubsection{Seismic noise}

\subsubsection{Newtonian noise}

\end{document}
