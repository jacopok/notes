\documentclass[main.tex]{subfiles}
\begin{document}

\marginpar{Friday\\ 2020-5-22, \\ compiled \\ \today}

% We are discussing LISA. 

\section{Pulsar Timing Array}

We seek a method to measure SMBH mergers and other extremely long wavelength GW. 
What we do is measure the distance from millisecond pulsars. 

The light travel time depends on how the space is curved by GW. 

The key to the data analysis is that the GW signal is one of the only correlated signals across the sky. The contribution is 
%
\begin{align}
r_{\alpha , GW} (t) = \int_{0}^{t} \dd{t'} \frac{ \delta \nu _\alpha }{\nu _\alpha } (t')
\,.
\end{align}

Here \(\alpha \) is an index representing which pulsar we are interested in. 
We can divide the signal \(r\) into a part at Earth,  \(r^{e}_{\alpha }(t)\), and a part at the pulsar, \(r^{p}_{\alpha }(t)\). 

\todo[inline]{Is there an averaging effect in PTA? If the GW is propagating in the same direction as the pulsar signal the signal is always in the same gravitational field\dots not clear really}

\section{Atom interferometry}

\end{document}
