\documentclass[main.tex]{subfiles}
\begin{document}

\section*{Introduction}

\marginpar{Monday\\ 2020-3-23, \\ compiled \\ \today}

Organization: the instructor for this first part is professor Michela Mapelli. 

Structure of the course:
\begin{enumerate}
  \item today we start with background on GW, theory and data;
  \item tomorrow we will discuss BH formation from single stars;
  \item on wednesday we will discuss BH formation form binary evolution;
  \item next week we will discuss BH dynamics.
\end{enumerate}

In the next part we will discuss neutron stars with professor Thomas Tauris. 

There is no final grade for the exam, only a pass/fail. 
This year, there will be a google form with multiple choice questions. 

\subsection*{What is GWA?}

It is a branch of astrophysics which studies the astrophysical characterization of GW sources. 
Mostly, it is about discussing binary black holes, binary neutron stars and NS-BH binaries, which emit in the frequency range we can observe right now. 

The main open question is: what are the formation channels of the compact objects binaries observed by ground-base interferometers.

\chapter{Gravitational wave summary}

From wikipedia: ``GW are ripples in the curvature of spacetime which propagate as waves, travelling outward from the source, at the speed of light''.
The math describing these is that of GR: the main equations are the Einstein Field Equations 
%
\begin{align}
R_{\mu \nu } - \frac{1}{2} R g_{\mu \nu } = \frac{8 \pi G}{c^{4}} T_{\mu \nu }
\,,
\end{align}
%
where \(R_{\mu \nu }\) is the trace of the Riemann tensor, which is a measure of curvature; on the other hand \(T_{\mu \nu }\) is a measure of the density of \(4\)-momentum in spacetime, which acts as a source. 

The metric \(g_{\mu \nu }\) is used in order to measure distances. 

The EFE are in general 10 second-order nonlinear PDEs. 
In the strong-curvature case, with rapidly moving sources we must use numerical solutions of the full nonlinear EFE. 

In the weak field limit, with almost flat spacetime and slowly moving sources, we have analytic solutions. 

We assume the perturbation looks like \(g_{\mu \nu } = \eta_{\mu \nu } + h_{\mu \nu } \), then we can linearize the EFE. We can rewrite the linearized equations using the trace-reversed perturbation 
%
\begin{align}
\overline{h}_{\mu \nu } = h_{\mu \nu } - \frac{1}{2} \eta_{\mu \nu } h
\,,
\end{align}
%
and then in a special coordinate system (in a special gauge) we get 
%
\begin{align}
\square \overline{h}_{\mu \nu } = \frac{16 \pi G}{c^{4}} T_{\mu \nu } + \mathcal{O}(h^2)
\,,
\end{align}
%
which can be solved in general using the method of Green's functions, just like in electrodynamics, to find 
%
\begin{align}
\overline{h}^{\mu \nu } = \frac{4G}{c^{4}} \int \dd[3]{y} \frac{\eval{T^{\mu \nu} (t, \vec{y})}_{\text{ret}}}{\abs{\vec{x}- \vec{y}}}
\,,
\end{align}
%
where \(t _{\text{ret}} = t - \abs{\vec{x} - \vec{y}}/c\). 

This is general but not easy to visualize. In order to discuss it we first assume we are far from our source:
so we can say that the support of the stress-energy tensor is all at a distance \(r\) from us. So, we get 
%
\begin{align}
\overline{h}^{\mu \nu } (t, \vec{x}) = \frac{4}{r} \frac{G}{c^{4}} \int \dd[3]{y} T^{\mu \nu } (t - r/c, \vec{y})
\,.
\end{align}

If the source is moving slowly, that is, with a nonrelativistic velocity, we get 
%
\begin{align}
\overline{h}^{ij} (t, \vec{x}) = \frac{2}{r} \frac{G}{c^{4}} \ddot{I}^{ij} (t- r/c)
\,,
\end{align}
%
where 
%
\begin{align}
I^{ij} (t- r/c) = \int \dd[3]{y} x^{i} x^{j} \rho (t- r/c, \vec{y})
\,.
\end{align}

For a full derivation see Hartle \cite[]{hartleGravityIntroductionEinstein2003}. 

This means that not all accelerating masses produce GWs: only those with non-zero quadrupole moment do; there is a need for an asymmetry in mass distribution.

Let us consider the simplest case we can have: a binary system, with two stars of equal mass \(M\) orbiting each other in a circular orbit of radius distance \(a\) with angular velocity \(\omega \). We can consider the center of mass and reduced mass coordinates. 

The components of the second mass moment are found from \(x(t) = a \cos(\omega t)\), \(y = a \sin( \omega t)\), \(z = 0\): 
%
\begin{subequations}
\begin{align}
I_{xx} &= Ma^2 \qty[1 + \cos(2 \omega t)] \\
I_{yy} &= Ma^2 \sin(2 \omega t) \\
I_{zz} &= Ma^2 \qty[1 - \cos(2 \omega t)] 
\,,
\end{align}
\end{subequations}
%
so we get 
%
\begin{subequations}
\begin{align}
\overline{h}^{ij} \sim \frac{2}{r} \frac{G}{c^{4}} (2 \omega )^2  M a^2
\left[\begin{array}{ccc}
\cos((2 \omega (t - r/c))) & \sin((2 \omega (t - r/c))) & 0 \\ 
\sin((2 \omega (t - r/c))) & -\cos((2 \omega (t - r/c))) & 0 \\ 
0 & 0 & 0
\end{array}\right]
\,,
\end{align}
\end{subequations}
%
which we can simplify using 
%
\begin{align}
\omega = \sqrt{\frac{2GM}{a^2}}
\,,
\end{align}
%
so we can see that the frequency of the GW is \emph{twice} \(\omega \): \(\omega_{GM} = 2 \omega  \). 

We also get the two polarizations, \(h_{+}\) and \(h_{ \times }\).

The amplitudes are of the order 
%
\begin{align}
h = \frac{1}{2} \qty(h_{+}^2 + h_{ \times }^2)^{1/2} \sim \frac{8 G^2}{c^{4}} \frac{M^2}{ra}
\,,
\end{align}
%
so with a solar mass binary, orbiting at a distance of a solar radius a kiloparsec away we get 
%
\begin{align}
h \sim \num{e-21} \qty(\frac{M}{M_{\odot}})^2 \qty(\frac{\SI{1}{kpc}}{r}) \qty(\frac{R_{\odot}}{r})
\,.
\end{align}

The amplitude is called the \emph{strain}: the bigger it is, the easier the detection. 

The easiest events to detect involve large masses, are close, and involve very close inspirals. 

\medskip

The crucial thing is that GW emission implies energy loss: this can be express iin the Keplerian limit as
%
\begin{align}
E _{\text{orb}} = - G \frac{m_1 m_2 }{2a}
\,.
\end{align}

The variable thing here is the radius \(a\): it decreases, until coalescence. 
As the semimajor axis decreases, the frequency increases. 

In order to quantify this, we can use the expression for the power emitted: 
%
\begin{align}
P_{GW} = \frac{32}{5} \frac{G^{4}}{c^{5}}
\frac{1}{a^{5}} m_1^2 m_2^2 (m_1 +m_2 )
\,,
\end{align}
%
so we can write 
%
\begin{align}
P_{GW} = \dv{E _{\text{orb}}}{t} = \frac{Gm_1 m_2 }{2a^2} \dv{a}{t}
\,,
\end{align}
%
which means 
%
\begin{align}
\dv{a}{t} = \frac{64}{5} \frac{G^3}{c^{5}} \frac{1}{a^3} m_1 m_2 (m_1 + m_2 )
\,,
\end{align}
%
which we can integrate in order to find the necessary time for coalescence: 
%
\begin{align}
t_{GW} = \frac{5}{256} \frac{c^{5}}{G^{3}} \frac{a^{4}}{m_1 m_2 (m_1 + m_2 )}
\,.
\end{align}

Here we neglected the eccentricity, if we account for it we need to multiply by a term \((1- e^2)^{7/2}\). 

This is an extremely long timescale. If \(a = \SI{1}{Au}\) and \(m_1 = m_2 = M_{\odot}\) we get timescales like \(t_{GW} \sim \SI{e17}{yr}\)!

The aforementioned equations hold for the ``inspiral phase''. The phases we usually distinguish are 
\begin{enumerate}
  \item inspiral, when the Keplerian approximation still holds;
  \item merger, when the distance between the objects is closer than the ISCO;
  \item ringdown, when the objects have coalesced.
\end{enumerate}

We can use post-Newtonian techniques for the inspiral, then full numerical relativity for the merger, and ``black hole perturbation methods'' for the ringdown. 
The ISCO has a radius is 
%
\begin{align}
r_{ISCO} = 6 \frac{G (m_1 + m_2 )}{c^2}
\,,
\end{align}
%
and we can get a good estimate of the frequency using the Keplerian formula: 
%
\begin{align}
\omega_{GW, ISCO} = 2 \sqrt{ \frac{G (m_1 + m_2 )}{r_{ISCO}^3}}
\,,
\end{align}
%
which works surprisingly well, better than a factor of 2. 

See Abbott et al \cite[]{ligoscientificcollaborationandvirgocollaborationObservationGravitationalWaves2016} for graphs. 

With the sensitivity of current ground-based detectors we can detect small BH and NS mergers. 

SMBH mergers are too low frequency, but we could maybe see extremely high mass ratio mergers. 

We could see GWs from neutrons stars with crustal asymmetries, with \SI{10}{Hz} to \SI{e-2}{Hz}.

Asymmetric supernova explosions would also be in this frequency range, but with too low amplitude. 

In this course we will focus on binary compact objects. 

\section{Observations of GW}

In the first two observations by LIGO-VIRGO we have seen 10 BBHs and 1 binary neutron star. 

The third observing run is ongoing, it will finish about now. 

From these observations we know that 
\begin{enumerate}
  \item BBHs exist;
  \item BBHs can merge in a Hubble time;
  \item Massive BHs (with mass \(>20  M_{\odot}\)) exist.
\end{enumerate}

As the observation number increases, the range of the observations increases. For O3, we will be able to see up to around \SI{110}{Mpc}: this number is derived discussing a test well-oriented \(1.3 M_{\odot}\) neutron star, which would give a SNR of around 12. 
This is roughly proportional to \(m^{5/3}\): the more massive the object, the larger the horizon: we will be able to see further for more massive objects. 

What are the parameters which are directly observables (neglecting eccentricity): 
\begin{enumerate}
  \item 2 masses;
  \item 6 spin components;
  \item polarization;
  \item inclination of binary with respect to interferometers;
  \item 2 angles (RA, DEC);
  \item redshift;
  \item reference time;
  \item phase at a reference time.
\end{enumerate}

\subsection{Masses and spins}
The ``no hair theorem'' states that BHs are unique up to mass and spin, since charge is negligible. 
Conventionally, we have \(m_1 > m_2 \). 
LIGO-VIRGO measure two masses: 
%
\begin{subequations}
\begin{align}
m _{\text{chirp}} &= \frac{(m_1 m_2 )^{3/5}}{(m_1 + m_2 )^{1/5}}  \\
M &= m_1 + m_2 
\,,
\end{align}
\end{subequations}
%
where \(m _{\text{chirp}}\) defines the change of the frequency in the inspiral phase. 
The total mass is measured by measuring the frequency at merger. 

Generally, we can measure well one of these but not the other. 
For high-mass mergers, we have the merger at the central frequency in the LIGO-VIRGO range, at around \SI{400}{Hz}. 
For low-mass mergers, the merger is at a higher frequency: this means that we can see more of the inspiral, since it stays in the high-sensitivity range longer. 

The detected BH mergers span the range from around 10 to \(60M_{\odot}\). 

The masses of the NS detected through GW are consistent with those measured through X-ray binaries; on the other hand the BHs measured with GWs are heavier than those measured by EM measurements. 

The spins of the BHs are defined by the vector 
%
\begin{align}
\vec{S} = \frac{\vec{J} c}{G m^2}
\,,
\end{align}
%
which goes from \(S = 0\) for Schwarzschild to \(S = \num{.998}\) for tan extremal Kerr. 

LIGO and VIRGO are not sensitive to individual spins. They can measure two combinations: if \(\hat{L}\) is the BBH orbital angular momentum, we define the effective spin as 
%
\begin{align}
\chi _{\text{eff}} \frac{\qty(m_1 \vec{S}_{1} + m_2 \vec{S}_{2})}{m_1 + m_2 } \cdot \hat{L}
\,,
\end{align}
%
which ranges from \(-1\) to \(1\). 
This is measures since it affects the phase of the GWs, while the orthogonal component of the spin affects precession: 
%
\begin{align}
\chi_{p} = \frac{1}{B_1 m_1^2} \max \qty(B_1 S_{1, \perp}, B_2 S_{2, \perp})
\,,
\end{align}
%
where\dots

These are really uncertain. 
There is heavy degeneracy between the value \(q = m_2 / m_1 \) and the effective spin. 

\todo[inline]{What does a 2D pdf plot showing degeneracy look like in general?}

\end{document}