\documentclass[main.tex]{subfiles}
\begin{document}

\chapter{Thomas Tauris' part}

\marginpar{Monday\\ 2020-3-30, \\ compiled \\ \today}

The topics will be:
\begin{enumerate}
  \item X-ray Binaries and Recycling Millisecond Pulsars;
  \item Spin and B-field evolution of Neutron Stars;
  \item Formation of BNSs;
  \item BNS and GW at low and high frequencies.
\end{enumerate}

\section{X-ray Binaries and Recycling Millisecond Pulsars}

For a review: van den Heuvel \& Tauris 2006, and also a textbook by the same authors in 2020-2021. 

Based on the observations in the sixties, we can deduce that most of the X-ray sources are accreting neutron stars. 
Their luminosities were huge, on the orders of \(L_X \sim \SI{e37}{erg /s }\). 

If they were normal stars or white dwarfs, they would need to accrete huge amounts of material, and that much material would obscure the radiation emitted. 
So, they can only be neutron stars or black holes. 

In '67, then, the first radio pulsar was then discovered. 

The accretion luminosity can come from either the release of gravitational binding energy and nuclear burning at the surface of the object. 

\textbf{High-mass X-ray binaries}: something like a \(16 M_{\odot}\) star accreting onto a \(1.3 M_{\odot}\) compact object. 
The dynamics of these is driven by wind accretion, and Roche-lobe overflow. 

\textbf{Be-star X-ray binaries}: a star undergoing de-cretion, orbited by a NS which enters in the decretion disk: regular X-ray emission. 

\textbf{Low mass X-ray binaries}: they are very long-lived, the donor star has a mass of less than a solar mass. Their periods are very short. 
The material accretes on top of the NS, and can undergo runaway nuclear fusion. These bursts typically last a minute or so. 

We can have Roche-lobe overflow at different stages in stellar evolution: if it does while it is hydrogen burning, helium burning or even after we have cases A, B and C. 

It is really hard to account for all the processes in compact binary evolution. 

We define the exponents 
%
\begin{align}
\zeta  = \pdv{\log R}{\log M }
\,,
\end{align}
%
which we can use to define some initial stability criteria. 

If there is mass loss (\(\beta >0\)) then we can have the orbit shrinking even if the period decreases. 

\subsection{Recycling and MilliSecond Pulsars}

How do they form? We have the \(P - \dot{P}\) diagram. \(P\) means period. 
MSPs are in the low \(P\), low \(\dot{P}\) region. 

\end{document}
