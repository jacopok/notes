\documentclass[main.tex]{subfiles}
\begin{document}

\marginpar{Tuesday\\ 2020-3-10}

\section*{Introduction}

% It's Multimessenger Astro\emph{particle} Physics. 

A questionnaire will be delivered in the middle of the course, for constructive criticism. It is optional, but we are asked to do it. 

The professor is Elisa Bernardini, \url{elisa.bernardini@unipd.it}. Office at Paolotti, 3rd floor. 

It is a very new field, there are no comprehensive textbooks. 

Multimessenger Astrophysics is about using: 
\begin{enumerate}
  \item cosmic rays (\ce{p});
  \item neutrinos (\(\nu \));
  \item photons (\(\gamma \)) at all wavelengths;
  \item gravitational waves (GW).
\end{enumerate}

Each of these represents a certain fundamental force: strong, weak, electromagnetic and gravitational.

The first time neutrinos were detected for a certain astronomical event was with SN 1987 A. 
Another one was, recently, TXS 0506+056 and IC-170922A, also with neutrinos. 
Plus, one with GWs.

There will be \emph{exercises}. 
They will not be asked in the exam, but small calculations might.

This course will assume some knowledge of fundamental particles and interactions, if we need support we should ask for it.

There are four suggested books, the professor prefers Spurio's and Perkins'. 
Longair will be used just for a specific topic: the interaction of cosmic rays. 

The professor is a physicist, involved with antarctic experiments.

High Energy Physics is originally HE Astrophysics. 

The flux (per unit area, time, energy, angle) of cosmic rays is roughly a powerlaw. 

Open questions are: what are the sources of these particles? What is their acceleration mechanism?

A useful way to probe this is an astrophysical beam dump. 

We can distinguish \emph{astrophysics}, which uses electromagnetic radiation alone, and \emph{astroparticle} physics which uses cosmic rays, neutrinos, high-energy gamma rays, searches for dark matter. 
A heuristic is that astrophysics measures energies in ergs, while astroparticle physics measures them in electronVolts. 

\subsection*{Syllabus}

\begin{enumerate}
  \item Interactions of astroparticles;
  \item acceleration, propagation, interaction of cosmic rays;
  \item measurements and candidate sources for cosmic rays;
  \item MultiMessenger approach: combining information from different types of particles and waves: 
  \begin{enumerate}
    \item Gamma  ray astrophysics;
    \item multi-wavelength observations of astrophysical sources;
    \item neutrino astrophysics;
    \item GW.
  \end{enumerate}
\end{enumerate}

\chapter{Interactions of astroparticles}

\section{Lorentz transformations}

We start from the assumption that the transformations should be linear and that light propagates at the same velocity in each inertial frame. This directly implies that 
%
\begin{align}
\dd{s^2} = - c^2 \dd{t^2} + \dd{x^2} + \dd{y^2} + \dd{z^2}
\,,
\end{align}
%
the spacetime interval, is an invariant.

Then, if reference \(S'\) is moving with velocity \(v\) along the \(x\) axis with respect to \(S\) the transformation law is given by 
%
\begin{align}
t' &= \gamma \qty(t - vx / c^2)  \\
x' &= \gamma \qty(x - vt)  \\
y' &= y  \\
z' &= z
\,,
\end{align}
%
where \(\gamma = 1 / \sqrt{ 1 -  v^2 / c^2}\).
This implies the presence of time dilation and length contraction with respect to the rest frame. 
A moving observer will measure a \emph{longer} duration than the stationary (comoving) one. 
A moving observer will measure a \emph{shorter} duration than the stationary (comoving) one.

\paragraph{Exercise: muons}

A direct experimental confirmation of this is the observation of muons produced by cosmic rays in the earth's atmosphere. 

Muons are produced around \SI{10}{km} up by interactions of protons, in the Earth's atmosphere. 
After some time they decay into pions. 
The lifetime of muons is around \(\tau_{\mu } \approx \SI{2.2e-6}{s}\). The lifetime of pions is \(\tau_{\pi } \approx \SI{2.6e-8}{s}\).

We can say that, since the muons are relativistic, their velocity is \(v \approx c\). So, the length they travel in their lifetime is \(l_{\mu } = c \tau_{\mu } \approx \SI{600}{m}\) and \(l_{\pi } = c \tau_{\pi } \approx \SI{8}{m}\). 

However, we did not account for time dilation! the typical energy of a \(\mu \) particle is \(E_{\mu } \approx \SI{10}{GeV}\), so 
%
\begin{align}
\gamma = \frac{E}{m_\mu c^2} \approx 100
\,,
\end{align}
%
so the time of decay in our frame is \(t = \gamma \tau_{\mu } = 100 \times \SI{2.2}{\micro s} \approx \SI{60}{km}\), which is enough for them to reach the ground.

For the pions, on the other hand, their path will be something like \(l_{\pi } = \gamma \tau_{\pi } c \approx
\SI{800}{m} \). Indeed, they are rare, and were detected using an emulsion field in high mountains. 

In the muon's perspective, the length is contracted: so the muon sees \(L = L' / \gamma \), and since \(L' \approx \SI{10}{km}\) we have  \(L \approx \SI{100}{m}\).

\end{document}