\documentclass[main.tex]{subfiles}
\begin{document}

\begin{abstract}
    To write
\end{abstract}

\section{Introduction}

The Large Scale Structure of the cosmos formed as a result of gravitational clustering, starting from ``seed'' inhomogeneities in the density field in the early universe. 
The generation of these perturbations is commonly thought to have been the result of quantum fluctuations in the early universe, which have reached galactic scales in the inflationary stage of the expansion. 

These perturbations may be considered as pertaining to the gravitational field \(\Phi \) or to the density field \(\rho \): the former approach has shown itself to be more fruitful in recent years.

A model which so far has not been contradicted by observation is to consider perturbations as being a realization of a Gaussian random field. 
Even if the field were not truly Gaussian, we would expect the Gaussian approximation to be a rather good one as it happens generally for processes which arise from the combined effects of many independent random variables \cite[pag.\ 2]{celoriaPrimordialNonGaussianity2018}. 

However, the study of non-Gaussianities in the primordial universe is crucial. Inflationary models commonly predict 

\end{document}