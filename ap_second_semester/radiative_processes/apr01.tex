\documentclass[main.tex]{subfiles}
\begin{document}

\marginpar{Wednesday\\ 2020-8-19, \\ compiled \\ \today}

Under our assumptions the electric field can be written as 
%
\begin{align}
\vec{E} = E_0 \sin \omega_0 t \vec{\epsilon}
\,,
\end{align}
%
where \(\vec{\epsilon}\) is a unit vector which is perpendicular to the propagation direction: \(\vec{\epsilon} \cdot \vec{k} = 0\); while \(E_0 \) is the amplitude of the electric field and \(\omega_0\) is its frequency. 

The equations of motion of the charge read 
%
\begin{align}
m \ddot{\vec{r}} = e \vec{E} = e E_0 \sin \omega_0 t \vec{\epsilon}
\,,
\end{align}
%
which  we can also express through the dipole moment \(\vec{d} = e \vec{r}\): the equation for its evolution will then read 
%
\begin{align}
\ddot{\vec{d}} = \frac{e^2}{m} E_0 \sin \omega_0 t \vec{e}
\,.
\end{align}

If we integrate in \(\dd{t}\) two times we find 
%
\begin{align}
\vec{d} (t) = - \frac{e^2 E_0 }{m \omega_0^2} \sin \omega_0 t \vec{\epsilon}
\,,
\end{align}
%
so the response to the impinging EM wave is an oscillation of the dipole, with a frequency \(\omega_0\) equal to that of the EM wave and an amplitude equal to 
%
\begin{align}
d_0 = \frac{e^2 E_0 }{m \omega_0^2}
\,.
\end{align}

The electron is accelerating, so it will radiate: the power emitted per unit solid angle will be 
%
\begin{align}
\frac{ \dd{w}}{ \dd{t} \dd{\Omega }} = \frac{1}{c^3} \frac{ \ddot{d}^2}{4 \pi } \sin^2 \Theta 
= \frac{1}{4 \pi c^3} \frac{e^{4}}{m^2} E_0^2 \sin^2 \omega_0 t \sin^2 \Theta 
\,,
\end{align}
%
where \(\Theta \) is the angle between the direction of propagation of the wave and the direction of observation. 
This oscillates in time; we can compute the average over an oscillation: this means that we substitute the square sine with a factor \(1/2\): 
%
\begin{align}
\expval{\frac{ \dd{w}}{ \dd{t} \dd{\Omega }}} = 
\frac{e^{4} E_0^2 \sin^2 \Theta }{8 \pi c^3 m^2}
\,.
\end{align}

Integrating over the solid angle to get the average power amounts to multiplying by \(4 \pi \) times \(2/3\), because of the solid angle in the sphere and because of the integral of \(\sin^2 \Theta\). 
This yields 
%
\begin{align}
\expval{\dv{w}{t}} = \frac{e^{4} E_0^2}{3 m^2 c^3}
\,.
\end{align}

Now, let us compute the flux of energy which is carried away by the incident EM wave: 
%
\begin{align}
S = \frac{ \dd{w}}{ \dd{A} \dd{t}} 
= \frac{c}{4 \pi } E_r^2 
= \frac{c}{4 \pi R^2} E_0^2 \sin^2 \omega_0t 
\,,
\end{align}
%
then the power per unit solid angle is 
%
\begin{align}
\frac{ \dd{w}}{ \dd{\Omega } \dd{t}} = \frac{c}{4 \pi } E_0^2 \sin^2 \omega_0 t 
\,,
\end{align}
%
whose average as before is 
%
\begin{align}
\expval{ \frac{ \dd{w}}{ \dd{\Omega } \dd{t}}} = \frac{c E_0^2}{8 \pi }
\,.
\end{align}

\todo[inline]{Except this, dimensionally, is a power per unit \emph{area}!}

Now, let us define the \textbf{differential scattering cross section} as 
%
\begin{align}
\dv{\sigma }{\Omega } = 
\frac{
    \expval{\frac{ \dd{w}}{ \dd{t} \dd{\Omega }}}_{\text{emitted}}
    }{
    \expval{\frac{ \dd{w}}{ \dd{t} \dd{\Omega }}}_{\text{incoming}}
}
\,,
\end{align}
%
\todo[inline]{is this not dimensionally inconsistent? the differential scattering cross section should have the dimensions of an area / steradian\dots Maybe the incoming power should be considered per unit \emph{area}, since it is a plane wave whose source is at infinity?

This seems indeed to be the case, see \cite[eq.\ 3.36]{rybickiRadiativeProcessesAstrophysics1979}}
and if we compute this for our case we will have 
%
\begin{align}
\dv{\sigma }{\Omega } = \frac{e^{4}E_0^2 \sin^2 \Theta }{8 \pi c^3 m^2}
\frac{8 \pi }{c E_0^2} 
= \frac{e^{4}}{c^{4} m^2} \sin^2 \Theta 
\,.
\end{align}

This expression can also be written through the classical electron radius: 
%
\begin{align}
r_0 = \frac{e^2}{mc^2} \approx \SI{2.82e-13}{cm}
\,,
\end{align}
%
whose expression is found by equating the rest energy of the electron \(m c^2\) with its electromagnetic self-energy \(e^2 / r_0 \). 
Then, the scattering cross section can be expressed as 
%
\begin{align}
\dv{\sigma }{\Omega } = r_0^2 \sin^2 \Theta 
\,.
\end{align}

The total cross section is found from the integral of this over all the solid angle: 
%
\begin{align}
\sigma_T = \int \dv{\sigma }{\Omega } \dd{\Omega } = r_0^2 \underbrace{\int (1 - \mu^2) \dd{\mu } \times 2 \pi }_{= 8 \pi /3} = \frac{8 \pi }{3} r_0^2
\,.
\end{align}

This is the \textbf{Thompson cross section} of the electron \(\sigma_{T} \approx \SI{0.665e-24}{cm^2}\). 

Now, for some observations. This cross section does not change depending on the frequency of the incoming wave: it is ``color blind''.
This formula for the scattering is not always valid, but in its regime of validity (which will be discussed later) we expect no frequency dependence. 

Moreover, this scattering is \textbf{conservative} or coherent: the frequency of the scattered radiation is the same as the frequency of the incoming radiation.

It also is \textbf{not isotropic} because of the factor \(\sin^2 \Theta \). This scattering ``prefers'' for the light to go along the same direction it came from, and the probability density for it goes to zero for  \(\Theta = \pi /2\), meaning for a scattered photon orthogonal to the direction of propagation. 
Also, there is forward-backward symmetry, which will be useful to simplify certain calculations.
The degree of anisotropy is also rather mild, it varies slowly over the solid angle, so in certain situations it will be alright to approximate Thompson scattering as isotropic.
This is the closest we can get to completely isotropic and coherent scattering. 

\end{document}
