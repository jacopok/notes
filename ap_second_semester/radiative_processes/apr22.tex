\documentclass[main.tex]{subfiles}
\begin{document}

\subsection{The physical effects of Comptonization}

\marginpar{Thursday\\ 2020-8-27, \\ compiled \\ \today}

As we have said it is hard to find numerical solutions to the Kompaneets equation, however there is an interesting quantity whose evolution we can study analytically: the photon energy density. 
The energy density can be calculated as 
%
\begin{align}
u = \int \dd[3]{p} h \nu \frac{N}{ \dd[3]{p}\dd[3]{x}}
= \int \dd[3]{p} h \nu   \frac{2 n}{h^3} 
= \frac{8 \pi (k_B T)^{4}}{h^3c^3} \int_0^{\infty } \dd{x} x^3 n
\,,
\end{align}
%
where, as usual, \(x = h \nu / k_B T\), and the last equality comes about by substituting this and \(\dd[3]{p} = p^2 \dd{p} \dd{\Omega }\) and assuming isotropy. 

In order to see how it changes in the first stage of the Comptonization, when the photon energy is low compared to the electron temperature, let us differentiate this energy density with respect to time (rescaled, as in the Kompaneets equation, by the mean time between two scatterings): 
%
\begin{align}
\dv{u}{t_s} 
= \frac{8 \pi (k_B T)^{4}}{h^3c^3} \int_0^{\infty } \dd{x} x^3 \pdv{n}{t_s}
= \frac{8 \pi (k_B T)^{4}}{h^3c^3} \int_0^{\infty } \dd{x} x^3 \frac{\Theta}{x^2} \pdv{}{x} \qty[x^{4 } (n' + n + n^2)]
\marginnote{Used the Kompaneets equation.}
\,.
\end{align}

Now, let us suppose that \(n\) is very small, so that we can neglect \(n\) and \(n^2\) inside the derivative, keeping \(n'\): this yields 
% \todo[inline]{Not super clear why \(n \ll n'\) if \(n\) is small should hold\dots }
%
\begin{align}
\dv{u}{t_s} \approx \frac{8 \pi (k_B T)^{4}}{h^3 c^3} \Theta \int_0^{\infty } \dd{x} x \pdv{[x^{4} n']}{x}
\,.
\end{align}

We integrate by parts twice and assume that \(x^{k} n' \to 0 \) for \(k> 3\): this give us 
%
\begin{align}
\dv{u}{t_s} \approx \frac{8 \pi (k_B T)^{4}}{h^3 c^3} 4 \Theta \int_0^{\infty } \dd{x} x^3 n = 4 \Theta u
\,.
\end{align}

The solution to this is an exponential: 
%
\begin{align}
u(t_s) = u(0 ) \exp( 4 \Theta t_s) = u(0 ) \exp( t / t_c )
\,,
\end{align}
%
where \(t_c = 1/ (4 \Theta n \sigma _T c)\) is called the Compton time. 
This means that the radiative energy density increases \textbf{exponentially fast}, over a characteristic time of the order \(t_c\). 

\todo[inline]{So, to see if I understand correctly: the contributions \(n + n^2\) are relevant as we move closer to equilibrium, while the \(n'\) term dominates the first stage of the evolution; and the exponential describes the first stage of an evolution which is shaped like some kind of sigmoid?}

\subsection{Relativistic Kompaneets equation}

We made the hypotheses that \(\Theta \ll 1\) and \(\epsilon \ll 1\) at the start: the electrons and photons were nonrelativistic.
If we keep all the other hypotheses (isotropy, homogeneity, small fractional energy change and so on) can we generalize the Kompaneets equation to relativistic particles?

We will need to use a relativistic Maxwellian (Jüttner distribution) instead of a Maxwellian, and the Klein-Nishina cross section instead of the Thomson one. This will allow us to deal with generic \(\Theta \) and \(\epsilon \): the result is 
%
\begin{align}
\pdv{n}{t} = \frac{1}{\epsilon^2} \pdv{}{\epsilon } \qty[ \alpha (\epsilon , \Theta ) \qty(\Theta \pdv{n}{\epsilon } + n + n^2)]
\,,
\end{align}
%
where \(\alpha \) is a certain known function, 
%
\begin{align}
\alpha (\epsilon , \Theta ) = \frac{\epsilon^4}{2 K_2 (1 / \Theta )} \int_0^{\infty } \dd{z} z^2 \alpha_0 (\epsilon z) \exp( - \frac{z + 1/z}{2 \Theta })
\,,
\end{align}
%
where \(\alpha_0\) is a known analytical function, whose expression is omitted here. 

The form of the relativistic Kompaneets equation is very similar to the nonrelativistic one. 

The shape of \(\alpha \) is quite similar to \(\epsilon^4\), its nonrelativistic counterpart, for low photon energies; while for high photon energies \(\alpha \) is lower than \(\epsilon^4\). 

The take-away, here, is that if we were to need the relativistic form of the equation we can find it in the literature and apply it to our problem.

\subsection{The emerging spectrum}

In astrophysical (non-cosmological) settings we will not find infinite homogeneous media; instead, we need to treat clouds of finite size. 
However, we cannot simply use the Kompaneets equation for them.

What is the spectrum we expect to observe at infinity? 
Let us suppose we have a source inside a spherical cloud of radius \(R\); this source will emit a pulse of radiation at a time \(t\), so that the density at a time \(t_0 \) is 
%
\begin{align}
n(\nu , t_0 ) = n_0 (\nu ) \delta (t_0  - t )
\,.
\end{align}

Photons will propagate and scatter in the cloud, until finally they are emitted and some of them can be observed. What will be the spectral distribution of these? 

Roughly speaking, we can express the observed number density as 
%
\begin{align}
n _{\text{obs}}(\epsilon ) = \sum _{i} p_i n(\epsilon , t_i)
\,,
\end{align}
%
where \(t_i\) are different times after \(t_0 \), \(n(\epsilon, t_i)\) are solutions of the Kompaneets equation at these times, and \(p_i\) are the probabilities that a photon emitted at \(t\) will exit the cloud at \(t_i\).
Properly speaking this should be an integral, but we are just giving the rough idea. We will turn it into an integral later.

The escape probability will satisfy 
%
\begin{align}
p(t) \dd{t} = \frac{ \dd{n}}{n}
\,,
\end{align}
%
so that it describes the fraction of photons (\(\dd{n} / n\)) which escape the medium in a time \(\dd{t}\). 
The observed intensity will then be 
%
\begin{align}
n _{\text{obs}}(\epsilon ) = \int_0^{\infty } p(\tau) n(\epsilon , \tau ) \dd{\tau }
\,.
\end{align}

The typical timescale for a photon to leave the cloud will be given by the mean number of scatterings times the typical time per scattering: if the medium is assumed to be optically thick we get 
%
\begin{align}
t _{\text{esc}} \sim N_s t_s \sim \max \frac{\qty(\tau,\tau^2) }{n \sigma _T c} \sim \frac{\tau^2 R}{nR \sigma _T c} \sim \frac{\tau R}{c}
\,,
\end{align}
%
since \(n R \sigma _T = \tau \). 
This result makes sense: it is the light travel time \(R / c\) times the optical depth. 

The form of \(p(t)\) depends on the shape of the cloud; for example in our spherical cloud we have 
%
\begin{align}
p(t) = \frac{1}{\tau } \sum _{n=1}^{\infty } \lambda_n \sin \lambda_n \exp( - \frac{\lambda _n^2}{3} \frac{t}{ t _{\text{esc}}}) 
\,,
\end{align}
%
where the coefficients \(\lambda _n\) are the solutions of \(\tan \lambda _n  = \lambda _n / (1 - 3 \tau / 2)\). 

It is important to apply this procedure instead of blindly plugging the distribution inside the Kompaneets equation: we know that in that case the photons will eventually thermalize, which is not of physical interest in most situations. 

\subsubsection{Unsaturated Comptonization and power-law tails}

Let us consider a phenomenologically-motivated modified Kompaneets equation, which accounts for the injection and escape of photons in a finite medium: 
%
\begin{align}
\dot{H} - \dot{S} = \frac{\Theta}{x^2} \pdv{[x^{4} (n' + n + n^2)]}{x}
\,,
\end{align}
%
where \(\dot{S} = Q(x)\) quantifies the photons gained per unit time, while \(\dot{H} = n / \max(\tau , \tau^2)\) quantifies the photons lost per unit time.

\todo[inline]{Are the signs of the \(\dot{S}\) and \(\dot{H}\) not the other way around?}

\todo[inline]{How does this work dimensionally? Are \(\dot{H}\) and \(\dot{S}\) not supposed to be inverse times?}

Suppose that the input is \emph{soft}, that is, low-energy. 
This means that we will have \(Q(x)\) different from zero only for \(x\) smaller than a certain threshold, \(x< x_s \ll 1\). 
We then look at the high-energy tail in emissions, \(x \gg 1\).

We will neglect the \(n^2\) term. The equation  will then read 
%
\begin{align}
0 = \frac{\Theta}{x^2} \pdv{[x^{4}(n' + n)]}{x} + \underbrace{Q(x) - \frac{n}{\max (\tau , \tau^2)}}_{\text{vanish in the high \(x\) regime}}
\,,
\end{align}
%
so we are left with \(n' + n = \const\), meaning that \(n \sim \exp(- x)\). 
This means that at very high energies the decay of the emission is exponential.

\todo[inline]{Why does the term proportional to \(n\) vanish?}

Now, instead, we look for mid-energy solutions, in the range between the threshold for the emission \(x_s\) and 1. We will suppose that \(x_s \ll x \ll 1\). 
We will then neglect the \(n\) and \(n^2\) terms, meaning that we can ignore \(Q(x)\) but not the escape term, so we have 
%
\begin{align}
0 = \frac{\Theta}{x^2} \pdv{[x^{4} (n')]}{x} - \frac{n}{\max (\tau , \tau^2)}
\,.
\end{align}

If we take a powerlaw ansatz, \(n \propto x^{m}\), we find that the equation for \(m\) becomes 
%
\begin{align}
m(m+3) - \frac{4}{4 \Theta \max (\tau, \tau^2)}= 0 
\,,
\end{align}
%
where we recognize the Compton parameter \(4 \Theta \max (\tau , \tau^2) = y\), which quantifies how much the scatterings affect the spectrum. 
The second-order equation for \(m\) then becomes 
%
\begin{align}
m = \frac{-3 \pm \sqrt{9 + 16 / y^2}}{2}
\,.
\end{align}

If we take the positive solution it will be \(m_+>0\), so the spectrum will increase at high energies, which is unphysical. 
So, we consider the negative solution: expanding the square root under the assumption that \(y < 1\)\footnote{As \(a\) and \(b\) get further from one another in magnitude, the approximation \(\sqrt{a + b} = \sqrt{a} + \sqrt{b}\) gets closer to being true. For small \(y\) we have quite a good result, while for \(y\) closer to 1 the exponent is slowly-varying anyway.} we get 
%
\begin{align}
m_- \approx -3 - \frac{2}{y}
\,.
\end{align}

This is then a powerlaw spectrum: for any soft photon input, then, we have found a qualitative result. It is a powerlaw for medium photon energies and an exponential for high photon energies. 
This then gives a typical signature for the observational spectrum we observe in Compton scattering. 

Note that, like in the Planck function, the spectrum of the observed intensity scales like \(I_\nu \sim n \nu^3\), so we expect to see \(I_\nu \sim \nu^{3+m} \sim \nu^{-2/y}\) for medium energies, and \(I_\nu \sim \nu^3 e^{-x}\). 

\subsubsection{Dynamical Comptonization}

We have so far discussed the effects of repeated scattering of photons onto electrons which are moving due to thermal motion. This is however not the only reason electrons may move! 

An interesting situation is one in which the temperature \(T\) of the electrons is low, but the bulk velocity of the plasma \(\vec{v}\) is high, so that interaction with photons is still capable of causing inverse Compton scattering. 

Suppose we have a flow of electrons moving with a uniform \(\vec{v}\), constant for all of them. Then, the Electron Rest Frame is the same for all the electrons. 

So, in order to deal with the first scattering we must boost to the ERF, but then we are done: we can stay in that frame for all subsequent scatterings. 
The energy the photon will emerge with will be \(\epsilon_2 \approx \gamma^2 \epsilon_1 \) regardless of the number of scatterings.

If, on the other hand, the velocity is not uniform we need to insert a relative \(\gamma \) factor for each new electron.
This effect will build up for each scattering. 
So, dynamical Comptonization requires a nonvanishing velocity gradient to work. 

\todo[inline]{He writes \(\nabla \cdot \vec{v} \neq 0\), which is not the same\dots why would the \emph{divergence} not vanishing specifically be the condition? what about the curl, or the other components of \(\partial_{i} v_j\) anyway?}

This is similar to \emph{second-order Fermi acceleration.}
\todo[inline]{what is that?}

\end{document}