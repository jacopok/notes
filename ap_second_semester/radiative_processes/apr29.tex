\documentclass[main.tex]{subfiles}
\begin{document}

\subsubsection{Angular distribution of emitted and received energy and power}

\marginpar{Friday\\ 2020-8-28, \\ compiled \\ \today}

We want to discuss how a relativistic particle emits radiation in the lab frame. 
We start off in the rest frame of the particle: there, it emits an energy \(\dd{w}'\) across a solid angle \(\dd{\Omega '} = \sin \theta ' \dd{\theta '} \dd{\varphi '} = \dd{\mu '} \dd{\phi '}\); in the lab frame we can say the same thing without the primes. 
We choose our reference frame for both lab and rest frame so that the particle is moving along the \(\hat{x}\) axis: \(\vec{v} = (v, 0, 0)\), while our observation direction is positioned at an angle \(\theta \) from the \(x\) axis. 

The energies \(\dd{w}\) and \(\dd{w}'\) are related by a Lorentz transformation, which in general reads 
%
\begin{align}
\dd{w} = \gamma \dd{w'} + \gamma v \dd{p'}_x
= \gamma \qty( \dd{w'} + \frac{v}{c} \cos \theta ' \dd{w'})
= \gamma \dd{w'} \qty(1 + \beta \mu')
\,.
\end{align}

In the second equality we have used the fact that the radiation four-momentum satisfies \(p^{\mu } p_{\mu } = 0\) since photons are massless, and therefore \((\dd{w'}, \dd{\vec{p}}')\) must be a null vector, meaning that \(\dd{w'} = \mu ' \dd{w'}/ c\).

We want to express this in terms of \(\mu \), the cosine of the angle in the lab frame. A geometric result we can derive is the \textbf{angular aberration formula}, 
%
\begin{align}
\mu = \frac{\mu ' + \beta }{1 + \beta \mu '}
\,,
\end{align}
%
which can be differentiated to yield 
%
\begin{align}
\dd{\mu } = \frac{ \dd{\mu '}}{\gamma^2 \qty(1 + \beta \mu ')^2}
\,.
\end{align}

The Lorentz boost does not affect the azimuthal direction since it is orthogonal to it, so \(\dd{\phi } = \dd{\phi '}\).
Therefore, we can transform the solid angle differential as 
%
\begin{align}
\dd{\Omega } = \dd{\mu } \dd{\phi } = \frac{ \dd{\mu '} \dd{\phi }' }{\gamma^2 ( 1 + \beta \mu ')^2} = \frac{ \dd{\Omega '}}{\gamma^2 ( 1 + \beta \mu ')^2}
\,.
\end{align}

The angular distribution of energy in the lab frame then can be expressed as 
%
\begin{align}
\dv{w}{\Omega } = \frac{\gamma (1+ \beta \mu ') \dd{w'}}{ \dd{\Omega '}} \gamma^2 (1 + \beta \mu ')^2
= \gamma^3 (1 + \beta \mu ')^3  \dv{w'}{\Omega '}
\,.
\end{align}

This is the distribution of the \emph{energy} emission: in order to find the \emph{power} emission we need to divide by time. However, we need to choose a time by which to divide: do we choose the one in the lab or rest frame? they are related by 
%
\begin{align}
\dd{t} = \gamma (1 - \beta \mu ) \dd{t}'
\,,
\end{align}
%
and it can be shown that the power emission in the two frames is related by 
%
\begin{align}
\frac{ \dd{w}}{ \dd{t} \dd{\Omega }}
= \frac{\gamma^3(1+ \beta \mu ')^3}{\gamma (1 - \beta \mu ) \dd{t'}} \dv{w'}{\Omega '}
= \frac{1}{\gamma^{4} (1- \beta \mu )^{4}} \frac{ \dd{w'}}{ \dd{\Omega '} \dd{t'}} 
\,.
\end{align}

Now, if radiation emission is isotropic in the particle frame then \(\dd{w'} / \dd{\Omega '}  \dd{t'}\) is independent of the angle. 
Under this assumption, the angular distribution of the received radiation is given by the prefactor. 

Let us analyze this for an ultra-relativistic particle, with \(\gamma \gg 1\).  Then, we can approximate 
%
\begin{align}
\beta = \sqrt{1 - \frac{1}{\gamma^2}} \approx 1 - \frac{1}{2 \gamma^2}
\,.
\end{align}

Inserting this into the prefactor yields 
%
\begin{align}
\frac{1}{\gamma^{4} (1 - \beta \mu )^{4}}
\approx \frac{1}{\gamma^{4} \qty[1 - \qty(1 - \frac{1}{2 \gamma^2}) \mu ]^{4}}
\,,
\end{align}
%
which in general will be small because of the \(\gamma^{-4}\) suppression. However, if \(\mu \sim 1\) then the quantity in square brackets is close to zero. We can expand \(\mu \sim 1 - \theta^2 /2\) since it is the cosine of \(\theta \): with this we can manipulate the expression into 
%
\begin{align}
\frac{1}{\gamma^{4} (1 - \beta \mu )^{4}} \sim \qty(\frac{2 \gamma }{1 + \gamma^2 \theta^2})^{4}
\,,
\end{align}
%
for small \(\theta \) at least. This is sharply peaked around \(\theta = 0\), with the FWHM of the peak of the order of \(1/\gamma \). 
This effect is \textbf{relativistic beaming}. 

\end{document}