\documentclass[main.tex]{subfiles}
\begin{document}

\section{Thermal radiation}

\marginpar{Saturday\\ 2020-8-15, \\ compiled \\ \today}

Consider a blackbody enclosure at equilibrium at a certain temperature \(T\). If we open a small hole in the enclosure and measure the radiation inside it, the intensity \(I_\nu \) we will see will be isotropic, and only depending on the temperature \(T\): let us call the (still unspecified) way this dependence look \(I_\nu = B_\nu (T)\). The function \(B_\nu (T)\) will be called the \textbf{blackbody function}. 

Now, imagine we put a small chunk of material inside the blackbody enclosure. If this material is characterized by a source function \(S_\nu = j_\nu / \alpha_{\nu }\), then the evolution of the intensity inside the cavity will be 
%
\begin{align}
\dv{I_\nu }{\tau_{\nu }} = - I_\nu + S_\nu 
\,.
\end{align}

If we wait for equilibrium, though, we must have \(I_\nu \equiv B_\nu \), meaning that its derivative must be zero: this implies that \(I_\nu = B_\nu = S_\nu \). 

This is called \textbf{Kirkhoff theorem}: a chunk of material which can emit and absorb and which is placed in a blackbody enclosure then it must satisfy 
%
\begin{align}
B_\nu = \frac{j_\nu}{\alpha_{\nu }} = S_\nu 
\,.
\end{align}

\subsubsection{Planck function derivation}

What is the explicit expression of the blackbody function \(B_\nu \)? 
From Bose-Einstein statistics we know that the phase space density of photons in thermal equilibrium is given by 
%
\begin{align}
\frac{ \dd{N}}{ \dd[3]{x} \dd[3]{p}} = \frac{2}{h^3} \frac{1}{\exp( \frac{h \nu }{k_B T}) - 1}
\,,
\end{align}
%
where the 2 comes from the two polarizations of the photons, \(h^3\) is the size of the phase space cell, while the exponential factor is the occupation number. 

Now, we would like to write this as a function of frequency, not of the three components of the momentum. The frequency \(\nu \) and the  momentum \(p = \abs{\vec{p}}\) of a photon are connected by \(h \nu = E =  c p \); and the momentum space element \(\dd[3]{p}\) can be written as \(\dd[3]{p} = \dd{\Omega } p^2 \dd{p}\). 

Bringing this to the other side and integrating over the solid angle (since the frequency of the photon does not depend on its direction we can substitute \(\int \dd{\Omega } = 4\pi \)) we can write 
%
\begin{align}
\frac{ \dd{N}}{ \dd[3]{x}} = \frac{8\pi}{c^3} \frac{\nu^2 \dd{\nu }}{
    \exp( \frac{h \nu }{k_B T}) - 1
} 
\,,
\end{align}
%
the \textbf{number density} of photons. 
From this we can quickly recover the \textbf{energy density} as well, since each photon with frequency \(\nu \) has energy \(h \nu \): therefore 
%
\begin{align}
\frac{ \dd{E}}{ \dd[3]{x}} = \frac{8 \pi }{c^3} \frac{h \nu^3 \dd{\nu }}{\exp( \frac{h \nu }{k_B T}) - 1}
\,.
\end{align}

If we want to recover the energy density per unit frequency and per unit solid angle we can divide by \(\dd{\Omega } \dd{\nu }\): this yields 
%
\begin{align}
u_\nu (\Omega ) = \frac{ \dd{E}}{ \dd[3]{x} \dd{\Omega } \dd{\nu }} = \frac{2h}{c^3} \frac{\nu^3}{\exp( \frac{h \nu }{k_B T}) - 1}
\,,
\end{align}
%
which we have previously shown to be equal to \(u_\nu (\Omega ) = I_\nu / c\): therefore the specific intensity, which corresponds to the Planck function \(B_\nu (T)\), is 
%
\begin{align}
I_\nu = B_\nu (T) =  \frac{2h}{c^2} \frac{\nu^3}{\exp( \frac{h \nu }{k_B T}) - 1}
\,.
\end{align}

\subsubsection{Properties of the Planck function}

Let us consider the limits of the Planck function. If the energy of the photon, \(h \nu \), is much lower than the average photon energies \(k_B T\) then we have 
%
\begin{align}
B_\nu (T) 
= \frac{2h}{c^2} \frac{\nu^3}{\exp( \frac{h \nu }{k_B T}) - 1}
\approx \frac{2h}{c^2} \frac{\nu^3}{1 + \frac{h \nu }{k_B T} - 1}
= \frac{2 k_B T}{c^2} \nu^2 
\,,
\end{align}
%
the \textbf{Rayleigh-Jeans} law. 

On the other hand, if the photon energy \(h \nu \) is much larger than \(k_B T\) we can ignore the \(-1\) in the denominator and just write 
%
\begin{align}
B_\nu (T) \approx \frac{2h \nu^2}{c^2} \exp(- \frac{h \nu }{k_B T})
\,,
\end{align}
%
the \textbf{Wien law}. 
The exponential term dominates, so asymptotically the Planckian goes to  zero. 

Since it is a continuously differentiable function which increases at low frequency and decreases at high frequency, it will have a maximum, which we expect to appear somewhere around \(h \nu \approx k_B T\). This is precisely what is described by the \textbf{Wien displacement law}. 

Let us see where this maximum lies if the Planckian is expressed as a function of frequency, as we are doing now. 
If we define \(x = h \nu / k_B T\), the Planck function is proportional to 
%
\begin{align}
B_\nu (T ) \propto \frac{x^3}{e^{x} - 1}
\,,
\end{align}
%
so we can find its maximum by seeking the stationary point 
%
\begin{align}
\pdv{B_\nu }{\nu } \propto \pdv{B_\nu }{x} = 0
\,. 
\end{align}

The derivative reads 
%
\begin{align}
\pdv{B_\nu }{x} \propto \frac{3 x^2 (e^{x}-1) - x^3 e^{x}}{\qty(e^{x} - 1)^2}
\propto e^{x} (3 - x) - 3 
\overset{!}{=} 0
\,.
\end{align}

This can also be expressed as \(x = 3 (1 - e^{-x})\) as well. 
It is a transcendental equation: there is no closed-form solution, however we can easily approximate it numerically. 
The maximum is found to be approximately \(x = \num{2.82}\), which means that the maximum is attained at the frequency 
%
\begin{align}
\nu \approx \num{2.82} \frac{k_B T}{h}
\,.
\end{align}

The fact that the peak of the curve depends linearly on \(T\) is precisely the statement of the Wien displacement law. 

Another interesting question to ask is about how \(B_\nu \) varies with \(T\): its partial derivative reads 
%
\begin{align}
\pdv{B_\nu }{T} = \frac{2 h^2 \nu^4}{k_B T^2 c^2} \frac{e^{h \nu  / k_B T}}{\qty(\exp(\frac{h \nu }{k_B T}) -1)^2} > 0
\,,
\end{align}
%
so it is positive for any value of \(\nu \). 
It will be clearer with a picture, however we can say that this increase with \(T\) is much faster for \(h \nu < k_B T\) than it is for \(h \nu > k_B T\). 

The specific energy density is found from \(u_\nu = I_\nu / c\) and integrating over the solid angles (so, multiplying by \(4 \pi \)), so we have 
%
\begin{align}
u_\nu (T) = \frac{4 \pi }{c} B_\nu 
\,,
\end{align}
%
and if we want to compute the total energy density we will have 
%
\begin{align}
u &= \int_{0}^{ \infty } u_\nu (T) \dd{\nu } = \frac{4\pi}{c} \frac{2h}{c^2} \int_{0}^{\infty } \frac{\nu^3}{\exp( \frac{h\nu}{k_B T} ) -1 } \dd{\nu }  \\
&= \frac{8 \pi h}{c^3} \frac{k_B^4 T^{4}}{h^{4}} \int_{0}^{\infty } \frac{x^3}{e^{x} -1} \dd{x}
\,,
\end{align}
%
where, as before, \(x = h \nu / k_B T\). This already shows the main result: \(u \propto T^{4}\), however to get a closed form expression we need to solve the integral as well.
This integral is of the type
%
\begin{align}
\int_{0}^{\infty } \frac{x^{n}}{e^{x} -1} \dd{x}   = n! \zeta (n+1)
\,,
\end{align}
%
which we need to evaluate for \(n = 3\). The Riemann zeta function fortunately has a closed form for even integers, in our case \(\zeta (4) = \pi^{4} / 90\), which we multiply by \(3! = 6\). This yields 
%
\begin{align}
u = \frac{8 \pi h}{c^3} \frac{k_B^4 T^{4}}{h^{4}} \frac{\pi^{4}}{15}
= \frac{8 \pi^{5}k_B^{4}}{15 c^3 h^3} T^{4} = aT^{4}
\,,
\end{align}
%
where the constant \(a \approx \SI{7.57e-16}{kg K^{-4} m^{-1} s^{-2}}\) is called the \textbf{blackbody constant}. 

Now, we discuss the \textbf{flux} of the radiation field. If we choose a surface and measure the flux across it, our assumption of isotropy means that we will always find zero; however it is interesting to measure the \emph{outgoing} flux from an oriented surface, counting only the photons going through it in one direction. 

\todo[inline]{I understand that this gives a factor 1/2, but why should it correspond to integrating over half the solid angle, except by isotropy? The half-sphere does not seem to me to have physical meaning\dots}

The flux we can compute is then: 
%
\begin{align}
F &= \frac{1}{2}  \int \dd{\Omega } \dd{\nu } B_\nu \cos \theta 
= \frac{2 \pi }{2} \int_{0}^{\pi } \dd{\theta } \cos \theta \sin \theta \int_{0}^{\infty } \dd{\nu } B_\nu   \\
&= \pi \underbrace{\int_{-1}^{1} \mu \dd{\mu }}_{=1} \underbrace{\int_{0}^{\infty } B_\nu \dd{\nu }}_{= u c / (4 \pi )} = \frac{uc}{4}
 \\
&= \frac{2 \pi^{5} k_B^{4}}{15 c^2h^3} T^{4}
= \sigma T^{4}
\,,
\end{align}
%
where \(\sigma \) is called the \textbf{Stefan-Boltzmann constant}, which is related to \(a\) by \(\sigma = ca / 4\). 

Another useful result to state here is that the integral of the Planck function in \(\dd{\nu }\) is:
%
\begin{align}
\int_{0}^{\infty } B_\nu \dd{\nu } = \frac{uc}{4 \pi } = \frac{\sigma }{\pi } T^{4}
\,.
\end{align}

\section{Scattering}

This is the third process which can affect the radiation field, besides absorption and emission. 
In the astrophysical context, we will only need to concern ourselves with scattering by \emph{free electrons}. 

When modelling absorption, we had the true absorption coefficient \(\alpha_{\nu } = n \sigma_{\nu }\), where \(\sigma_{\nu }\) was the cross section of the individual absorber. 
We can model scattering in the same way: we can introduce a ``scattering absorption coefficient'' \(\alpha_{\nu}^{(s)}\), which differs from the true one in that in absorption photons are not destroyed but merely deflected. 

However, from the point of view of the beam in a specific direction they disappear, as if they were absorbed. 
We will then also have to consider the inverse phenomenon: photons coming into the beam we are considering from another. 
This can be modelled as a ``scattering emission''. 

We express the amount of photons lost from our beam as 
%
\begin{align}
\int \alpha^{(s)}_{\nu } I_\nu \dd{\Omega }
\,.
\end{align}

Let us make a simplifying assumption: that the scattering is \textbf{isotropic}, the scattered photon has an equal probability to go in any direction. Also, let us assume that the frequency \(\nu \) of the photon does not change upon scattering. 
We will see that both of these conditions are satisfied for Thompson scattering but not in the general situation. 

Then, we are left with \(\alpha_{\nu }^{(s)} \int I_\nu \dd{\Omega } = 4 \pi \alpha_{\nu }^{(s)} J_\nu \). 

This is in any direction, while the radiation scattered in the unit solid angle will be 
%
\begin{align}
j_{\nu }^{(s)} = \alpha_{\nu }^{(s)} J_\nu 
\,.
\end{align}

In general, since we do not have either conservation of photon frequency nor isotropy we will need to write a differential scattering cross section like 
%
\begin{align}
\frac{ \dd{\sigma }}{ \dd{\nu } \dd{\Omega } \dd{\nu}' \dd{\Omega }'}
\,,
\end{align}
%
while the full expression for the radiation scattered in a specific direction \(\Omega \) at a frequency \(\nu \) will read 
%
\begin{align}
j_\nu^{(s)} = \int \dd{\Omega}' n \frac{ \dd{\sigma }}{ \dd{\nu } \dd{\Omega } \dd{\nu}' \dd{\Omega }'} I_{\nu'} (\Omega') \dd{\nu }
\,. 
\end{align}

This is called the \textbf{scattering kernel}. 

\end{document}
