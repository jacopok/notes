\documentclass[main.tex]{subfiles}
\begin{document}

\subsubsection{The transfer equation for scattering}

\marginpar{Sunday\\ 2020-8-16, \\ compiled \\ \today}

If we only account for true emission and absorption the transfer equation looks like 
%
\begin{align}
\dv{I_\nu }{s} = - \alpha_{\nu } I_\nu + j_\nu 
\,.
\end{align}

For isotropic and conservative scattering we can modify this equation by adding on the scattering absorption and emission terms: 
%
\begin{align}
\dv{I_\nu }{s} = - (\alpha_{\nu } + \alpha_{\nu }^{(s)}) I_\nu + j_\nu + \alpha_{\nu }^{(s)} J_\nu 
\,.
\end{align}

Now we cannot give a formal solution anymore: the derivative of the intensity now depends not only on the intensity itself but on its average over all solid angles, \(J_\nu  = \expval{I_\nu }_{\Omega }\).
This will typically pose a problem: scattering is often the dominant phenomenon in radiative transfer for astrophysical systems. 

In order to solve the equation we can resort to numerical methods: for example we can use an iterative relaxation procedure in which we start off by computing the solution without scattering, calculate the mean intensity and plug it as a fixed value into the next iteration, and keep going. 

We can write the transfer equation as 
%
\begin{align}
\dv{I_\nu }{ s} 
= (\alpha_{\nu } + \alpha_{\nu }^{(s)}) 
\qty(- I_\nu + \frac{j_\nu + \alpha_{\nu }^{(s)} J_\nu }{\alpha_{\nu } + \alpha_{\nu }^{(s)}}) 
= (\alpha_{\nu } + \alpha_{\nu }^{(s)}) 
\qty(-I_\nu  + S_\nu )
\,,
\end{align}
%
where we define a new form for the source function: 
%
\begin{align} \label{eq:source-function-with-scattering}
S_\nu = \frac{j_\nu + \alpha_{\nu }^{(s)} J_\nu }{\alpha_{\nu } + \alpha_{\nu }^{(s)}}  
\,,
\end{align}
%
where, as long as Kirkhoff's law holds, we can substitute \(j_\nu = \alpha_{\nu } B_\nu \). 

The optical depth is derived from the \emph{total} absorption coefficient: 
%
\begin{align}
\dd{\tau_{\nu }} = (\alpha_{\nu } + \alpha_{\nu }^{(s)}) \dd{s}
\,.
\end{align}

With these definitions, we can write the transfer equation like before, 
%
\begin{align}
\dv{I_\nu }{\tau_{\nu }} = - I_\nu + S_\nu 
\,.
\end{align}

If there is scattering this is only apparently simple, the formulation only hides the complexity. 

\subsection{Mean free path}

We defined absorption as \(\alpha_{\nu } = n \sigma_{\nu }\): this is just the inverse of the mean free path, so we have (for true absorption):
%
\begin{align}
\ell_{\nu } = \frac{1}{\alpha_{\nu }}
\,,
\end{align}
%
and we can define a mean free path for scattering in the exact same way, with \(\alpha_{\nu }^{(s)}\). 
These will be the mean free paths of a photon before it undergoes that specific process, if we want to compute the mean free path of a photon before it undergoes \emph{either one} then we can just take the inverse of \(\alpha_{\nu } + \alpha_{\nu }^{(s)}\), the total absorption coefficient. 

Now, consider a medium with scattering, emission and absorption. 
Typically, a photon will be emitted, scatter a few times, and then be absorbed. Let us say that between emission and absorption it is scattered \(N\) times, and let us call the spatial intervals it travels between these scatterings \(\vec{r}_i\), where \(i\) goes from 1 to \(N\). 
The total distance travelled will look like \(\vec{R} = \sum _{i} \vec{r}_i\). 

The average of its square value will be 
%
\begin{align}
\expval{R^2} = \sum _{ij} \expval{\vec{r}_i \vec{r}_j}
= \sum _{i} \expval{r_i^2} + 2 \sum _{i<j} \expval{\vec{r}_i \vec{r}_j}
\,.
\end{align}

The mixed averages \(\expval{r_i r_j}\) evaluate to zero, since each scattering is isotropic and independent; on the other hand each of the \(\expval{r_i^2}\) is equal to \(\ell_\nu^2\), the square of the mean free path. 
This means that we have \(\expval{R^2} = N \ell_\nu^2\), meaning that the average distance occurring between each scattering is given by \(\ell_{*} = \sqrt{N} \ell_{\nu }\). 

Now, let us suppose that the photon is being scattered in a medium whose characteristic length is \(L\). Then typically a photon will need to be scattered \(N\) times before escaping the medium, where \(L = \sqrt{N} \ell_{\nu }\). 
This means that 
%
\begin{align}
\sqrt{N} = \frac{L}{\ell_{\nu }} = \alpha_{\nu } L \sim \tau_{\nu }
\,,
\end{align}
%
where the last order-of-magnitude relation comes from the fact that the \emph{differential} of the optical path is given by \(\dd{\tau }_{\nu } = \alpha_{\nu } \dd{s}\). 

This means that, at least in terms of order of magnitude, \(N \sim \tau_{\nu }^2\). This holds as long as the medium is optically thick, that it, \(\tau_{\nu }\) is larger (ideally much larger) than 1. 

What happens instead if the medium is optically thin, with \(\tau_{\nu } < 1\)? 
Then, we can neglect emission and scattering emission, and only consider scattering absorption. Then, the transfer equation will read 
%
\begin{align}
\dv{I_\nu }{\tau_{\nu }} = - I_\nu 
\,,
\end{align}
%
which is solved by \(I_\nu = I_\nu (0) e^{-\tau_{\nu }}\). 
This can also be written, by adding \(I_\nu (0)\) to both sides, as 
%
\begin{align}
I_\nu (0 ) - I_\nu  &= I_\nu (0) \qty(1 - e^{-\tau_{\nu }}) \\
\frac{I_\nu (0) - I_\nu }{I_\nu } &= 1 - e^{-\tau_{\nu }} \approx \tau_{\nu }
\,.
\end{align}

Now, the average number of scatterings will be less than 1 and will correspond to the average relative intensity lost, \(N = \Delta I_\nu / I_\nu (0)\). Therefore, in this case we will have \(N \sim \tau_{\nu }\). 

Putting together these two limiting cases, we can roughly say that we will have 
%
\begin{align}
N \approx \max \qty(\tau_{\nu },  \tau_{\nu }^2)
\,.
\end{align}

After a mean free path the photon can be either scattered or absorbed. 
The probability that it is absorbed as opposed to being scattered is given by 
%
\begin{align}
\epsilon_{\nu } = \frac{\alpha_{\nu }}{\alpha_{\nu } + \alpha_{\nu }^{(s)}}
\,,
\end{align}
%
while the probability that it is scattered is \(1 - \epsilon_{\nu }\), and this last quantity is typically called the \textbf{single scattering albedo}.

The average number of mean free paths travelled before absorption will be\footnote{This can be shown using the identity 
%
\begin{align}
\sum _{i=1}^{\infty } i p^{i-1} = \frac{1}{(1-p)^2}
\,,
\end{align}
%
which can be proven by differentiating the geometric series and bringing the derivative into the sum, which converges absolutely. 

Then, we can compute the average number of scatterings as 
%
\begin{align}
\expval{N} = \sum _{i=1}^{\infty } \epsilon_{\nu } i (1 - \epsilon_{\nu })^{i-1} = \frac{\epsilon_{\nu }}{(1 - (1 - \epsilon_{\nu }))^2} = \frac{1}{\epsilon_{\nu }}
\,.
\end{align}} 
%
\begin{align}
N = \frac{1}{\epsilon_{\nu }}
\,,
\end{align}
%
so we can make the following manipulation to find an explicit expression for the mean path between emission and absorption: 
%
\begin{align}
\ell_{*} = N \ell_{\nu }^2 = \frac{\ell_{\nu }^2}{\epsilon_{\nu }}
= \frac{1}{(\alpha_{\nu } + \alpha_{\nu }^{(s)} )^2} \frac{\alpha_{\nu }+ \alpha_{\nu }^{(s)}}{\alpha_{\nu }} 
= \frac{1}{\alpha_{\nu } (\alpha_{\nu } + \alpha_{\nu }^{(s)})}
\,.
\end{align}

This means that 
%
\begin{align}
\ell_{*} = \frac{1}{\sqrt{\alpha_{\nu } (\alpha_{\nu } + \alpha_{\nu }^{(s)})}}
\,.
\end{align}

Now we make use again of the order-of-magnitude relation \(\tau \sim \alpha L\) where \(L\) is the length scale of the medium; if we multiply \(L\) by \(1 / \ell_{*}\), the effective absorption coefficient, we get 
%
\begin{align}
\tau \sim \frac{L}{\ell_{*}} = \sqrt{L^2 \alpha_{\nu } (\alpha_{\nu } + \alpha_{\nu }^{(s)})} = \sqrt{\tau_{\nu } (\tau_{\nu } + \tau_{\nu }^{(s)})}
\,,
\end{align}
%
which is larger than \(\tau_{\nu }\) alone would be. This is called the \textbf{effective optical depth}; the fact that this is larger than \(\tau_{\nu }\) means that scattering traps the photon in the region for a longer time than it would remain there with absorption alone. 

\section{Radiative diffusion}

This is a way to approximately solve the radiative transfer equation under certain assumptions, the main one being that the medium should be very \textbf{optically thick}.

Another assumption we will make is the \textbf{plane-parallel} approximation: this means that the properties of the medium only vary with respect to one coordinate, which we will call \(z\). 
We will further assume that the properties of the radiation also only vary along \(z\). 

Now, this is about the \emph{spatial} dependence of the quantities, however certain ones like the radiation intensity \(I_\nu \) are also intrinsically vectors, so they also depend on the direction. 
However, we will have cylindrical symmetry for rotations around the \(z\) axis: therefore, the intensity will be a function of \(z\) and of the angle \(\theta \) between the ray and the \(z\) axis. 

The differential length travelled by the ray can be expressed in terms of the distance travelled along the \(z\) axis as
%
\begin{align}
\dd{s} = \frac{ \dd{z}}{\cos \theta } = \frac{ \dd{z}}{\mu }
\,,
\end{align}
%
where we define the usual shorthand, \(\mu  = \cos \theta \).

The radiative transfer equation reads 
%
\begin{align}
\dv{I_\nu }{s} = \qty(- I_\nu + S_\nu ) \qty(\alpha_{\nu } + \alpha_{\nu }^{(s)})
\,,
\end{align}
%
where we can substitute 
%
\begin{align}
\dv{I_\nu }{s} = \mu \pdv{I_\nu }{z}
\,,
\end{align}
%
and using this we can express the radiative transfer equation like 
%
\begin{align}
I_\nu (z, \mu ) = S_\nu - \frac{1}{\alpha_{\nu } + \alpha_{\nu }^{(s)}}
\mu \pdv{I_\nu }{z}
\,.
\end{align}

Now we make use of the assumption that the optical depth is large. This means that \(\tau \sim (\alpha_{\nu }+ \alpha_{\nu }^{(s)}) \ell\) is large, where \(\ell\) is the characteristic scale of the system. 

\todo[inline]{Should we not say explicitly that, besides being optically thick, the medium's thickness should be ``slowly varying''?}

This means that the term proportional to \(\pdv*{I_\nu }{z}\) is small compared to the source function, so we can apply perturbation theory. 

To zeroth order, assuming thermal equilibrium, we will have 
%
\begin{align}
I_\nu^{(0)} \approx S_\nu^{(0)} = B_\nu (T)
\,.
\end{align}

Then, the first-order approximation can be found by inserting the zeroth-order expression into the equation: 
%
\begin{align}
I_{\nu }^{(1)} (z, \mu ) = B_\nu - \frac{\mu }{\alpha_{\nu } + \alpha_{\nu }^{(s)}} \pdv{B_\nu }{z} 
\,.
\end{align}

An important thing to note is the linear relation between \(I_\nu \) and the cosine \(\mu \) of the angle. 

With this first-order result we can compute the flux: if we were to use the zeroth-order one we would get zero, since blackbody radiation is isotropic. 
The flux is given by the integral over the solid angle: the contribution to the integral due to the \(B_\nu \) term vanishes, since it is isotropic
%
\begin{align}
F_{\nu }(z) &= \int I_\nu^{(1)} (z, \mu ) \mu \dd{\Omega }
= 2 \pi \int_0^{\pi } I_\nu^{(1)} (z, \mu ) \cos \theta \sin \theta \dd{\theta }  \\
&= - \frac{2 \pi }{\alpha_{\nu } + \alpha_{\nu }^{(s)}} \pdv{B_\nu }{z} 
\underbrace{\int_{1}^{-1}  \mu^2  (-1) \dd{\mu }}_{= 2/3}  \\
&= - \frac{4 \pi }{3 (\alpha_{\nu } + \alpha_{\nu }^{(s)})} \pdv{B_\nu }{z}  \\
&= - \frac{4 \pi }{3 (\alpha_{\nu } + \alpha_{\nu }^{(s)})} \pdv{B_\nu }{T} \dv{T}{z} 
\,,
\end{align}
%
where we used the fact that the spatial dependence of the \(B_\nu \) only comes through the temperature \(T(z)\). This is useful since, while the derivative of \(B_\nu \) is frequency-dependent, the derivative of the temperature is not: therefore it can be factored out. 

The total flux is given by 
%
\begin{align}
F &= \int_{0}^{\infty } F_\nu \dd{\nu } = 
- \frac{4 \pi }{3} \dv{T}{z} \int_0^{\infty } \frac{1}{\alpha_{\nu } + \alpha_{\nu}^{(s)}} \pdv{B_\nu }{T} \dd{\nu }  \\
&= - \frac{4 \pi }{3} \dv{T}{z}
\frac{\int_0^{\infty } \frac{1}{\alpha_{\nu } + \alpha_{\nu}^{(s)}} \pdv{B_\nu }{T} \dd{\nu }}{
\int_0^{\infty } \pdv{B_\nu }{T} \dd{\nu }
}
\int_0^{\infty } \pdv{B_\nu }{T} \dd{\nu }
\,,
\end{align}
%
which is the \textbf{average value} of the function \(1 / (\alpha_{\nu }+ \alpha_{\nu }^{(s)})\) weighted by the \emph{known} function \(\pdv*{B_\nu }{T}\); the integral at the numerator can be explicitly evaluated: 
%
\begin{align}
\int_0^{\infty } \pdv{B_\nu }{T} \dd{\nu }
= \pdv{}{T} \int_{0}^{\infty} B_\nu \dd{\nu }
= \pdv{}{T} \qty( \frac{\sigma}{\pi } T^{4}) 
= \frac{4 \sigma }{\pi } T^3
\,,
\end{align}
%
which means that the final expression is 
%
\begin{align}
F = - \frac{4 \pi }{3} \frac{4 \sigma T^3}{\pi } \dv{T}{z}  
\frac{\int_0^{\infty } \frac{1}{\alpha_{\nu } + \alpha_{\nu}^{(s)}} \pdv{B_\nu }{T} \dd{\nu }}{
\int_0^{\infty } \pdv{B_\nu }{T} \dd{\nu }
}
\,,
\end{align}
%
which we can better understand by defining the \textbf{Rosseland mean opacity} 
%
\begin{align}
\frac{1}{\alpha_{R}} = \frac{\int_0^{\infty } \frac{1}{\alpha_{\nu } + \alpha_{\nu}^{(s)}} \pdv{B_\nu }{T} \dd{\nu }}{
\int_0^{\infty } \pdv{B_\nu }{T} \dd{\nu }
}
\,,
\end{align}
%
so that the expression reads 
%
\begin{align}
F = - \frac{16}{3} \frac{\sigma T^3}{\alpha_{R}} \dv{T}{z}
\,.
\end{align}

This is called the \textbf{Rosseland approximation}. The result is that we can find a net photon flux which is driven by the temperature gradient. 
This has many applications, for example in stellar astrophysics. 

This equation is very similar to the heat flux equation in one dimension: 
%
\begin{align}
q = - k \pdv{T}{z}
\,,
\end{align}
%
which in our case has the constant (thermal conductivity) 
%
\begin{align}
k = \frac{16}{3} \frac{\sigma T^3}{\alpha_{R}}
\,.
\end{align}

We have assumed that the medium is optically thick, but we can see that if it becomes very optically thick then the photons have a hard time escaping (\(k\) becomes small). 

\section{The Eddington approximation}

In the Rosseland approximation we found the result 
%
\begin{align}
I_\nu = B_\nu (T) + \mu \times \text{stuff} 
\,,
\end{align}
%
where the ``stuff'' is small. We then expanded in this parameter; without it we are in a perfectly isotropic condition, so we can say that we are ``expanding around isotropy''. 

The idea behind the Eddington approximation is similar: we assume that the intensity can be written as 
%
\begin{align}
I_\nu (\tau ) = a_\nu (\tau ) + b_\nu (\tau ) \mu  
\,,
\end{align}
%
so as before the angular dependence is linear. 

Let us compute the moments of the specific intensity: the zeroth moment is the mean intensity, 
%
\begin{align}
J_\nu = \frac{1}{4 \pi } \int I_\nu \dd{\Omega } 
= \frac{2 \pi }{4 \pi } \int_{-1}^{1} (a_\nu + b_\nu \mu) \dd{\mu }
= a_\nu 
\,,
\end{align}
%
while the first moment is equal to 
%
\begin{align}
H_\nu = \frac{1}{4 \pi } \int I_\nu \mu \dd{\Omega }
= \frac{2 \pi }{4 \pi } \int_{-1}^{1} \qty(a_\nu \mu + b_\nu \mu^2) \dd{\mu } = \frac{1}{2} \frac{2}{3} b_\nu = \frac{b_\nu }{3}
\,.
\end{align}

The second moment is
%
\begin{align}
K_\nu = \frac{1}{4 \pi } \int I_\nu \mu^2 \dd{\Omega } = \frac{a_\nu }{3}
= \frac{1}{3} J_\nu 
\,.
\end{align}

The fact that the second moment of the intensity  is related to the zeroth one as \(K_\nu = J_\nu / 3\) is precisely what the Eddington approximation asks. 

The plane-parallel radiative transport equation can be written as  
%
\begin{align}
\mu \pdv{I_\nu }{\tau } = I_\nu - S_\nu 
\,,
\end{align}
%
since \(\dd{\tau } = -(\alpha_{\nu } + \alpha_{\nu }^{(s)}) \dd{s}\) and \(\dd{z} = \mu \dd{s}\). 

\todo[inline]{Why is the optical depth defined with a negative sign here? It is the same in RL \cite[eq.\ 1.115]{rybickiRadiativeProcessesAstrophysics1979}. Is this to be interpreted as the optical depth towards a far-away observer, positioned at large \(z\)?}

In this equation, the source function \(S_\nu \) is assumed to be isotropic. So, if we average over the solid angle (or equivalently in \(\dd{\mu } / 2\)) we get  
%
\begin{align}
\frac{1}{2}\int_{-1}^{1} \mu \pdv{I_\nu }{\tau } \dd{\mu } &= \frac{1}{2}\int_{-1}^{1} I_\nu \dd{\mu }
- \frac{1}{2}\int_{-1}^{1} S_\nu \dd{\mu } \\
\pdv{}{\tau } H_\nu  &= J_\nu - S_\nu 
\,.
\end{align}

We can also take the first moment of the same equation by multiplying the equation by \(\mu \) before integrating: this yields 
%
\begin{align}
\frac{1}{2}\int_{-1}^{1} \mu^2 \pdv{I_\nu }{\tau } \dd{\mu } &= \frac{1}{2}\int_{-1}^{1} I_\nu \mu  \dd{\mu }
- \underbrace{\frac{1}{2}\int_{-1}^{1} S_\nu \mu \dd{\mu }}_{= 0 }  \\
\pdv{K_\nu }{\tau } &= H_\nu 
\,.
\end{align}

However, recall that in the Eddington approximation we have \(K_\nu = J_\nu / 3\): this means that we have the system of coupled differential equations 
%
\begin{align}
\pdv{H_\nu }{\tau } = J_\nu - S_\nu 
\qquad \text{and} \qquad
\frac{1}{3} \pdv{J_\nu }{\tau } = H_\nu  
\,,
\end{align}
%
which we can combine into one second order equation by differentiating the second: this yields 
%
\begin{align}
\frac{1}{3}\pdv[2]{J_\nu }{\tau } = J_\nu - S_\nu 
\,.
\end{align}

Now, the source function can be decomposed in a part depending on \(J_\nu \) and a part depending on \(B_\nu \) according to the single scattering albedo \(\epsilon_{\nu }\) (see equation \eqref{eq:source-function-with-scattering}): 
%
\begin{align}
S_\nu = \epsilon_{\nu } B_\nu + (1 - \epsilon_{\nu }) J_\nu 
\,.
\end{align}

Substituting this in a term simplifies and we finally find 
%
\boxalign{
\begin{align}
\frac{1}{3} \pdv[2]{J_\nu }{\tau } = \epsilon_{\nu } \qty(J_\nu - B_\nu )
\,,
\end{align}}
%
the \textbf{radiative diffusion equation}. 
This can be solved relatively simply; however it does not give us the whole intensity, instead we can only find its mean value \(J_\nu \). 

If we can calculate \(S_\nu \) using this equation, then we can plug what we have found into the radiative transfer equation 
%
\begin{align}
\mu \pdv{I_\nu }{\tau } = I_\nu - S_\nu 
\,.
\end{align}

Now that we have discussed the differential equations at length, let us consider for a moment the \textbf{boundary conditions} we need to impose. 

A possible way to approach the problem is called the \emph{two stream approximation}: we assume that radiation is only travelling along two angles, whose cosines are \(\mu = \pm 1/ \sqrt{3}\), so \(\theta = \pm \SI{55}{\degree}\). 
This choice is less arbitrary than it may seem, there is good reason to suppose that in certain condition this is indeed the angle at which radiation will travel. This will be discussed later. 

Let us introduce the quantities 
%
\begin{align}
I_\nu^{\pm} (\tau ) = I_\nu (\tau , \pm \frac{1}{\sqrt{3}})
\,.
\end{align}

Then, the average intensity will be given by their average: 
%
\begin{align}
J_\nu = \frac{I_\nu^{+} + I_\nu^{-}}{2}
\,.
\end{align}
%
The further moments can also be computed: they will be 
%
\begin{align}
H_\nu = \frac{1}{2 \sqrt{3}} \qty(I_\nu^{+} + I_\nu^{-})
\qquad \text{and} \qquad
K_\nu = \frac{I_\nu^{+} + I_\nu^{-}}{6}
\,,
\end{align}
%
consistently with the Eddington approximation. Note that, if we want to use a two-stream approximation, the only way for Eddington to hold is to use the values of \(\mu = \pm 1/\sqrt{3}\). 

The explicit expressions of the moments allow us to recover the \(I_\nu^{\pm}\) as 
%
\begin{align}
I_\nu^{+} &= J_\nu + \sqrt{3} H_\nu = J_\nu + \frac{1}{\sqrt{3}} \pdv{J_\nu }{\tau } \\
I_\nu^{+} &= J_\nu - \sqrt{3} H_\nu = J_\nu - \frac{1}{\sqrt{3}} \pdv{J_\nu }{\tau }
\,.
\end{align}

Now, let us fix an example for concreteness: a ``slab'' (like, for instance, a stellar atmosphere) extending from \(\tau_0\) to \(\tau =0\). 
The point at \(\tau = 0\) is the top of the atmosphere: so, we want to impose the absence of incoming radiation there (supposing that there are no other nearby stars to influence the process): this is implemented as \(I_\nu^{-} (\tau = 0) = 0\). 
This is called the \emph{non-illuminated atmosphere} condition. 

Also suppose that there is no radiation coming into the bottom of the slab: \(I_\nu^{+} (\tau_0 ) = 0\). 

This means that 
%
\begin{align}
\eval{J_\nu + \frac{1}{\sqrt{3}} \pdv{J_\nu }{\tau }}_{\tau_0 } = 0
\qquad \text{and} \qquad
\eval{J_\nu - \frac{1}{\sqrt{3}} \pdv{J_\nu }{\tau }}_{0} = 0
\,. 
\end{align}

With these boundary conditions, one can solve the equation numerically.

\section{Plane electromagnetic waves}

In a vacuum, Maxwell's equations read 
%
\begin{align}
\nabla \cdot E = \nabla \cdot B =0
\,
\end{align}
%
and 
%
\begin{align}
\nabla \times E &= - \frac{1}{c} \pdv{B}{t}  \\
\nabla \times B &= \frac{1}{c} \pdv{E}{t}
\,.
\end{align}

Using the identity \(\epsilon_{ijk} \epsilon_{klm} = \delta_{jl} \delta_{km} - \delta_{jm} \delta_{kl}\) we can combine these into a wave equation, 
%
\begin{align}
\nabla^2 E - \frac{1}{c^2} \pdv[2]{E}{t} = 0 
\,,
\end{align}
%
which is solved by plane monochromatic waves 
%
\begin{align}
\vec{E}  =\vec{\epsilon} E_0 \exp(i (\vec{k} \cdot \vec{r} - \omega t))
\,,
\end{align}
%
where the angular frequency and wavevector satisfy \(\omega = c \abs{\vec{k}}\).
These are transverse waves: the scalar product of \(\vec{\epsilon}\) and the wavevector \(\vec{k}\) is zero, there is no longitudinal oscillation along the direction of motion. 

These solutions form a basis, generally the radiation we find is not monochromatic. 
The electric field will be time-dependent, \(\vec{E} = \vec{E} (t)\). 

We want to discuss the frequency decomposition of these waves; we start off by assuming that we are dealing with a pulse, such that \(\vec{E}(t) \rightarrow 0\) as \(t \to \pm \infty \).

The Fourier transform of the electric field in the time domain gives us its frequency decomposition: we choose our conventions so that 
%
\begin{align}
\hat{E}(\omega ) = \frac{1}{2 \pi } \int_{- \infty }^{\infty } E(t) e^{i \omega t} \dd{t} 
\,.
\end{align}

Since \(E(t)\) is real, we must have \(\hat{E}^{*}(\omega ) = \hat{E} (- \omega )\). 
This means that we do not need to worry about negative frequencies. 

The fundamental quantity we need to compute is the \textbf{energy spectrum} from the electric field: how is the energy in the EM radiation distributed across the various frequencies? 
The Poynting vector is useful in this context, as it quantifies the amount of energy carried per unit area and time: its modulus is given by 
%
\begin{align}
S = \frac{ \dd{w}}{ \dd{t} \dd{A}} = \frac{c}{4 \pi } E^2(t)
\,,
\end{align}
%
which we need to integrate if we want to find the total energy carried by the pulse: 
%
\begin{align}
\dv{w}{A} = \frac{c}{4 \pi } \int_{- \infty }^{\infty} E^2(t) \dd{t}
\,.
\end{align}

Now, \textbf{Parseval's theorem} is a statement about Fourier transforms, telling us that the energy of the signal is conserved if we Fourier transform, meaning that 
%
\begin{align}
\int_{- \infty }^{\infty } E^2(t) \dd{t} = 2 \pi \int_{-\infty}^{\infty} \abs{\hat{E}(\omega )}^2 \dd{\omega}
\,.
\end{align}

Since this square modulus is the same for the conjugate the integrand is symmetric under \(\omega \to - \omega \), so we can compute it from 0 to \(\infty \) and multiply by 2 to get the same result. This yields 
%
\begin{align}
\dv{w}{A} = \frac{4 \pi c}{4 \pi } \int_{0}^{\infty }\abs{\hat{E}(\omega )}^2 \dd{\omega}
\,,
\end{align}
%
which can be differentiated with respect to \(\omega \) to find the energy density per unit time and frequency:
%
\begin{align}
\frac{ \dd{w}}{ \dd{A} \dd{\omega }} = c \abs{\hat{E}(\omega )}^2
\,.
\end{align}

This concerns the \emph{total} energy of the pulse: we might want to calculate the energy per unit area, time and frequency! 
However, there is an issue in doing this: the uncertainty principle. We cannot sample with arbitrarily small intervals in both \(t\) and \(\omega \), since the inequality \(\Delta \omega \Delta t \geq 1\) must be satisfied. 

However, if our pulse repeats over some comparatively long time \(T\) we can formally define 
%
\begin{align}
\frac{ \dd{w}}{ \dd{A} \dd{\omega } \dd{t}} = \frac{1}{T} c \abs{\hat{E}(\omega )}^2
\,,
\end{align}
%
so, as long as \(T\) is relatively long we can take the limit:
%
\begin{align}
\frac{ \dd{w}}{ \dd{A} \dd{\omega } \dd{t}} = \lim_{T \to \infty  } \frac{1}{T} c \abs{\hat{E}(\omega )}^2
\,.
\end{align}

\subsection{Polarization}

This is a measurable quantity in astrophysics, it is easier to perform polarization measurements for low frequencies but nowadays we are starting to be able to measure polarizations as far as X-rays. 

By definition a monochromatic plane wave is \emph{linearly polarized}: it oscillates along a specific axis. 
Consider a superposition of two electric fields at a specific point in space, for simplicity \(\vec{r} = 0 \): the total electric field will be given by 
%
\begin{align}
\vec{E} = \underbrace{\qty(\vec{x} E_1 + \vec{y} E_2 )}_{\vec{E}_0 } e^{-i \omega t}
\,,
\end{align}
%
where the amplitudes \(E_{1,2} \) are in general complex: \(E_{1, 2} = \xi_{1, 2} e^{i \phi_{1,2} }\). So, we can write the total electric field 
%
\begin{align}
\vec{E} = 
\vec{x} \xi_1 e^{i (\phi_1 - \omega t)}
+
\vec{y} \xi_2 e^{i (\phi_2 - \omega t)}
\,.
\end{align}

The physical field along any direction will be given by the real part of the projection of this complex-valued vector along that component: 
%
\begin{align}
E_x = \Re \qty{\xi_1 e^{i (\phi_1 - \omega t)}} = \xi_1 \cos(\phi_1 - \omega t)
= \xi_1 \cos(\omega t - \phi_1 )
\,.
\end{align}

Now, a short aside: an ellipse can be described parametrically from its principal axes with 
%
\begin{align}
x' = A \cos \beta \cos \omega t
\qquad \text{and} \qquad
y' = - A \sin \beta \sin \omega t
\,,
\end{align}
%
since these expressions satisfy 
%
\begin{align}
\frac{x^{\prime 2} }{A^2 \cos^2\beta }
+\frac{y^{\prime 2} }{A^2 \sin^2\beta }
= 1
\,,
\end{align}
%
so we can parametrize any ellipse with an appropriate choice of \(A \in R^{+}\) and \(\beta \in [- \pi /2, \pi /2]\). 

\end{document}
