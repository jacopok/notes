\documentclass[main.tex]{subfiles}
\begin{document}

\subsection{Polarization of synchrotron radiation}

\marginpar{Saturday\\ 2020-8-29, \\ compiled \\ \today}

Measuring polarization is in general difficult, but still it is important to be able to provide an estimate of how much radiation is actually polarized. 

In order to describe synchrotron radiation we will need some reference unit vectors. Let us define \(\vec{\epsilon}_{\perp}\) and \(\vec{\epsilon}_{\parallel}\) such that, given the observation direction \(\vec{n}\) and the magnetic field \(\vec{B}\), \(\vec{\epsilon}_\perp\) is perpendicular to both, while \(\vec{\epsilon}_\parallel\) is perpendicular to \(\vec{n}\) only, while being in the plane defined by \(\vec{n}\) and \(\vec{B}\). 

These two unit vectors define a basis for the radiation seen in the direction \(\vec{n}\), since as we recall the polarization of electromagnetic radiation is always transverse. 

It can be shown that the power spectrum emitted in each polarization is given by 
%
\begin{align}
\eval{\frac{ \dd{w}}{ \dd{t} \dd{\omega }}}_{\perp} &= \frac{\sqrt{3} q^3 B \sin \alpha }{4 \pi m c^2} \qty(F + G) \\
\eval{\frac{ \dd{w}}{ \dd{t} \dd{\omega }}}_{\parallel} &= \frac{\sqrt{3} q^3 B \sin \alpha }{4 \pi m c^2} \qty(F - G)
\,,
\end{align}
%
where \(F\) is the function of \(\omega / \omega _c\) we defined earlier, 
%
\begin{align}
F \qty(\frac{\omega }{\omega _c}) = \frac{\omega}{\omega _c} \int_{\omega / \omega _c}^{\infty } K_{5/3} (z) \dd{z} 
\,,
\end{align}
%
while the new function \(G\) is defined by 
%
\begin{align}
G \qty(\frac{\omega }{\omega _c}) = \frac{\omega}{\omega _c} K_{2/3} \qty( \frac{\omega}{\omega _c})
\,.
\end{align}

The total emitted power per unit frequency is given by the sum of the two contributions in each polarization; so the terms containing \(G\) simplify, and we are left with the \(F\) term only, the result we found earlier. 

The polarization fraction at each frequency for a single electron will be given by 
%
\begin{align}
\Pi (\omega ) 
= \frac{P_\perp (\omega ) - P_\parallel (\omega )}{P_\perp (\omega ) + P_\parallel (\omega )}
= \frac{G( \omega / \omega _c) }{F(\omega / \omega _c)}
\,,
\end{align}
%
while the polarization fraction for a whole family of electrons whose energies are distributed according to a powerlaw will be given by the ratio of the integrals of \(G\) and \(F\) weighted by the electron distributions:  
%
\begin{align}
\Pi &= \frac{\int_0^{\infty } G (\omega / \omega _c) \gamma^{-P} \dd{\gamma }}{\int_0^{\infty } F(\omega  / \omega _c) \gamma^{-P} \dd{\gamma }}  \\
&= \frac{\int_0^{\infty } G (x) \gamma^{-P} \dd{\gamma }}{\int_0^{\infty } F(x) \gamma^{-P} \dd{\gamma }}
\,,
\end{align}
%
where we must be careful, since \(x\) is a function of \(\gamma \) in the integral. Specifically, \(x = \omega / \omega _c \sim 1/ \gamma^2\), so \(\gamma \sim x^{-1/2}\), therefore \(\dd{\gamma } \propto -1/2 x^{-3/2} \dd{x}\). This yields, up to multiplicative factors which cancel out in the ratio: 
%
\begin{align}
\Pi = \frac{\int_0^{\infty } G(x) x^{(P-3) / 2} \dd{x}}{\int_0^{\infty } F(x) x^{(P-3) / 2} \dd{x}}
\,.
\end{align}

These integrals of special functions are evaluated in the literature, we have general explicit expressions for expressions like \(\int x^{\mu } F(x) \dd{x}\) and similarly for \(G\).

Substituting these in, we find 
%
\begin{align}
\Pi = \frac{1 + P}{P + 7/3}
\,.
\end{align}

This is always \(<1\) as it should be, however with very steep powerlaws (high \(P\)) it can reach values of \(\Pi \) which are arbitrarily close to 1.

If, on the other hand, we are only considering one electron, we can compute the \emph{frequency-integrated} polarization fraction 
%
\begin{align}
\Pi 
= \frac{\int_0^{\infty } G(x) \dd{\omega }}{\int_0^{\infty } F(x) \dd{\omega }}
= \frac{\int_0^{\infty } G(x) \dd{x}}{\int_0^{\infty } F(x) \dd{x }}
\,,
\end{align}
%
which coincides with the polarization fraction from a population of electrons if we take \(P = 3\). Therefore, we find \(\Pi = (1+3) / (3 + 7/3) = 3/4\). 

\section{Einstein coefficients}

In order to understand what the absorption from synchrotron radiation looks like we need to take a step back and consider the Einstein coefficients. 

Let us reconsider Kirkhoff's law: if emitters and absorbers are in thermal equilibrium, then the following relation holds: 
%
\begin{align}
j_\nu = \alpha _\nu B_\nu 
\,.
\end{align}

This law holds at the macroscopic level: 

\end{document}