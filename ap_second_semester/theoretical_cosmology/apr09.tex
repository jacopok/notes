\documentclass[main.tex]{subfiles}
\begin{document}

\marginpar{Thursday\\ 2020-4-9, \\ compiled \\ \today}

We work in the tight coupling limit. 

[around equation 39]

We drop the quadrupole term.

The term multiplying \(\dot{\tau}\) in eq 40 goes to infinity like \(1/ \dot{\tau}\):  we need to replace it by a finite contribution. 

We still need information about \(\dot{\Psi}\) and \(\Phi \). 
This should be provided by the EFE. We approximate them as being independent of \(\theta_0 \) and \(\theta_1 \). 

The trick to get \(\theta_0 \) is to differentiate eq 39 with respect to \(\eta \), and then take \(\dot{\theta}_{1} \) from the other eq.  

So we get eq 42: it describes the evolution of \(\theta_0 \), on its right hand side we have a forcing term which we denote by \(F(k, \eta )\). 
The factor \(R / ( 1+ R)\) is the ratio of the baryon energy density to the total energy density.

This is solved numerically, we want a simple analytic solution. 

We basically have a damped wave equation, with a forcing term [43]. This is solved, as usual, by solving the homogeneous equation first. 
We solve the homogeneous equation with the use of an integrating factor. 

People sometimes drop the viscous term: this is a good approximation if we are within the ``sound horizon'' (which we shall see shortly). 

We replace the \(c_s\), which is dependent on time in principle, by its integral in \(\eta \). (?)
The sound horizon is given by 
%
\begin{align}
r_\eta = 
\,.
\end{align}

Once we have solved the homogeneous equation, we use the Green function method to solve the inhomogeneous equation. 

We use the cosine instead of the sine in order to account for the boundary conditions. 

The wavenumber corresponding to equality is:
%
\begin{align}
k _{\text{eq}} = \num{.073} \SI{}{Mpc^{-1}} \Omega_{0m} h^2
\,.
\end{align}

So we have that \(\theta_0 \sim \cos\), while \(\theta_1 \sim \sin\): they are in opposition of phase in the tight coupling limit. 

Anisotropic stress means neutrinos.
The perturbations enter the horizon in the radiation dominated epoch as long as\dots

Meszaros effect: perturbations which enter the horizon during radiation dominance are damped. 

The gravitational perturbation is constant outside the horizon, it is damped under radiation dominance. 

We now add the quadrupole, but we neglect the octupole. 

In order to solve these equations, we can apply the same procedure as before, or we can use a trick: we get an approximate dispersion relation, compared to the previous solution we get an imaginary term, which correspond to damping in \(\theta_0 \) and \(\theta_1 \). 

\(\lambda_{D}\) tells us how small a perturbation's wavelength can be if it is damped. 

\subsection{Free-streaming}

We wish to describe what we expect to see.
We discovered we can have anisotropies, so we can have hot and cold spots in the radiation content of the universe. 

For photons, the distance in comoving time \(\Delta \eta \) and the comoving distance \(\Delta L\) are equal, since \(\dd{s} = 0\). 

The quantity \(\eta_0 - \eta_{*} \) is usually called \emph{lookback time}. 
A multipole of order \(\ell\) is maximized when \(\ell\)  is of the order \(1/ \theta \), where \(\theta = k / (\eta_0 - \eta_*)\). 

We define a new temporary function \(\widetilde{S}\).

The most important equation is equation 48. 

\end{document}
