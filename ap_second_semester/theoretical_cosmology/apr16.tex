\documentclass[main.tex]{subfiles}
\begin{document}

\marginpar{Thursday\\ 2020-4-16, \\ compiled \\ \today}

Plot: figure 1 from \cite[]{planckcollaborationPlanck2018Results2019}, the power spectrum. Error bars seem small, but consider the factor \(\ell (\ell+1)\) (?).

The resolution could go to \(\ell  = 3000\). 

Temperature anisotropies map, also from Planck: the grey part is reconstructed from galactic absorption. 
Helpix: software to reconstruct the whole sky, which allows us to calculate the power spectrum. 

We can include the polarization, to distinguish E and B-modes.
B-modes are divergence-free, E-modes are curl-free.

We would like to take a statistical average over all observers, so we must average over the angles. 

We define the quantity \(\zeta = - \frac{3}{2} \Phi \), which is gauge invariant and connected to inflation. 

Sachs-Wolfe, Doppler term, integrated Sachs-Wolfe.

We have 
%
\begin{align}
\expval{ \delta (\vec{k}) \delta (\vec{k}')} 
= (2 \pi )^3 \delta^{(3)} (\vec{k} + \vec{k}') P(k)
\,,
\end{align}
%
where we have plus since we are computing the average of \(\delta \delta \) instead of \(\delta^{*} \delta \).

The fluctuation of CDM, \(\delta \), is proportional to \(k^2 \Phi \). This is conventional. 

A powerlaw is scale-invariant: if we multiply by \(\lambda \) the powerlaw is multiplied by \(\lambda^{n}\), but it is not distorted.

\subsection{The free-streaming term}

Dark energy changes \(\Phi \) and \(\Psi \), making them decrease. 



\end{document}
