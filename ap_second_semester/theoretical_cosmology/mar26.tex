\documentclass[main.tex]{subfiles}
\begin{document}

\section{Boltzmann equation for CDM}

\marginpar{Thursday\\ 2020-3-26, \\ compiled \\ \today}

[From yesterday: the distribution function for Compton scattering is general, as long as electrons and photons are in thermal equilibrium.]

Last time we derived the first order Boltzmann equation for the thermal anisotropy. 

Now, however, we need to discuss CDM, and then we will discuss baryons.. 

For Dark Matter the collision term is negligible (since by the observations of galactic dynamics we know that DM is collisionless):
so in the Boltzmann equation we have 
%
\begin{align} \label{eq:Boltzmann-DM}
\bigd{f _{\text{dm}}}{\lambda } = 0
\,.
\end{align}

The key fact is that DM is nonrelativistic. The calculation, except for this, is quite analogous.
The momentum \(P^{\mu }\) has to be normalized: \(P^{\mu }P_{\mu } = g_{\mu \nu } P^{\mu }P^{\nu } = -m^2\), as opposed to \(0\), which we had with the photon. We define the norm of the three-momentum 
%
\begin{align}
p^2 = g_{ij} P^{i} P^{j}
\,,
\end{align}
%
and with it the energy: \(E = \sqrt{p^2+m^2}\). Then, using our perturbed metric we get 
%
\begin{align}
P^{0} = E (1-\Phi )
\qquad \text{and} \qquad
P^{i} = p \hat{p}^{i} \frac{1 + \Psi }{a}
\,,
\end{align}
%
where \(\hat{p}^{i}\) is a unit vector in the direction of \(p^{i}\). 
This makes sense: the velocity, without perturbation, decays like \(a^{-1}\). 

Notice that we can make the derivative with respect to the affine parameter \(\lambda \) to one with respect to \(t\), the factor \(\dv*{t}{\lambda }\) can be moved to the RHS and does not affect our discussion. 
The LHS of the Boltzmann equation \eqref{eq:Boltzmann-DM} then reads 
%
\begin{align}
\bigd{f_{\text{dm}}}{t} = \pdv{f_{\text{dm}}}{t} 
+ \pdv{f _{\text{dm}}}{x^{i}} \dv{x^{i}}{t}
+ \pdv{f _{\text{dm}}}{E} \dv{E}{t}
+ \underbrace{\pdv{f _{\text{dm}}}{\hat{p}^{i}} \dv{\hat{p}^{i}}{t}}_{\text{higher order}}
\,,
\end{align}
%
where the last term is second order since it is a product of factors which are both zero to zeroth order. 
With steps analogous to the massless case we find 
%
\begin{align} 
0= \pdv{f _{\text{dm}}}{t} + \frac{\hat{p}^{i}}{a} \frac{p}{E} \pdv{f _{\text{dm}}}{x^{i}}
- \pdv{f _{\text{dm}}}{E} \qty[
\frac{1}{a} \dv{a}{t} \frac{p^2}{E} - \frac{p^2}{E} \pdv{\Psi }{t} 
+ \frac{\hat{p}^{i} p}{a} \pdv{\Phi }{x^{i}}
]
\,.
\end{align}

The factor \(\dv*{a}{t} / a\) is just \(H\), we write it this way in order not to overload the dot (which represents derivatives with respect to conformal time only in our notation). 

In order to work with this, we shall consider only terms up to first order in \(p / E\), which is justified since DM is non-relativistic. 
Also, we will take moments: in general, a moment is an integral of a function (with angular dependence) multiplied by some power of the cosine of the angle between the direction considered and some fixed other direction. 
In principle, the hierarchy goes to any order: in practice, we can usually truncate the hierarchy to some order according to some physical principle. 
Fortunately, DM behaves like a fluid, which allows us to work to first order only. 

\todo[inline]{Are the NS equations not exact?}

This approach is fully relativistic, but linear. 
In star formation, we use Newtonian approximation, but we account for  nonlinearity.
Having both is impossible.

The lowest (0th) moment is just given by the integral in \(\dd[3]{p} / (2 \pi )^3\) of the distribution, that is, over all angles: we find 
%
\begin{align}
\begin{split}
0&= \pdv{}{t} \int \frac{\dd[3]{p}}{(2 \pi )^3} f _{\text{dm}}
+ \frac{1}{a} \pdv{}{x^{i}} \int \frac{\dd[3]{p}}{(2 \pi )^3} f _{\text{dm}} \frac{p \hat{p}^{i}}{E} \\
&\phantom{=}\ 
- \qty[\frac{1}{a} \dv{a}{t} - \pdv{\Psi }{t}]
\int \frac{\dd[3]{p}}{(2 \pi )^3} \pdv{f _{\text{dm}}}{E} \frac{p^2}{E}
- \underbrace{\frac{1}{a} \pdv{\Phi }{x^{i}} \int \frac{\dd[3]{p}}{(2 \pi )^3} \pdv{f _{\text{dm}}}{E} \hat{p}^{i} p}_{\text{higher order}} 
\marginnote{We take out of the integral any term not depending on the angles in momentum space.}
\,.
\end{split}
\end{align}
%
why is the last term higher order? There is no way to have a nonzero result to zeroth order in the integral by isotropy, and it is multiplied by a first-order perturbation. 

Now, we introduce the dark matter density \(n _{\text{dm}}\) --- for more details, see the notes for the course in Fundamentals of Astrophysics and Cosmology \cite[section 3.2]{tissinoFundamentalsAstrophysicsCosmology2020}, but do note that the number density defined there is global, while the one we are considering here is a local number density, which can be different for different points in spacetime.
Also, we define its velocity \(v^{i}\): 
%
\begin{align}
n _{\text{dm}} &= \int \frac{ \dd[3]{p}}{(2\pi )^3} f _{\text{dm}}  
\marginnote{Note that \(f _{\text{dm}}\) accounts for the number of degrees of freedom \(g _{\text{dm}}\).}\\
v^{i } &= \expval{ \frac{P^{i}}{E}}_{\text{particle}}
= \frac{1}{n _{\text{dm}}} \int \frac{ \dd[3]{p}}{(2 \pi )^3} f _{\text{dm}} \frac{p \hat{p}^{i}}{E}
\,.
\end{align}

\todo[inline]{These are not covariant, right? Since \(n\) is the 0th component of a 4-vector\dots}

In the first and second terms we can insert \(n _{\text{dm}}\) and \(v^{i}\) directly (to first order --- we always do manipulations neglecting second order terms), while for the third term we need to integrate by parts, using \(\dv*{E}{p} = p/E\):\footnote{This comes from \(E = \sqrt{p^2+m^2}\): 
%
\begin{align}
\pdv{E}{p} = \frac{ 2 p }{2 \sqrt{p^2+m^2}} = \frac{p}{E}
\,.
\end{align}
} we obtain: 
%
\begin{align}
\int \frac{\dd[3]{p}}{(2 \pi )^3} \pdv{f _{\text{dm}}}{E} \frac{p^2}{E}
&= \int \frac{ \dd[3]{p}}{(2\pi )^3} p \pdv{f _{\text{dm}}}{p} 
\marginnote{Changed the derivative of the distribution} \\
&= \frac{4 \pi }{(2 \pi )^3} \int_{0}^{ \infty } \dd{p} p^3 \expval{\pdv{f _{\text{dm}}}{p}}  \\
&= -3 \frac{4 \pi }{(2 \pi )^3} \int_{0}^{ \infty } \dd{p} p^2 \expval{f _{\text{dm}}}= -3 n _{\text{dm}}
\,,
\end{align}
%
which is a calculation discussed in the FAC course \cite[]{tissinoFundamentalsAstrophysicsCosmology2020} [I'll insert the section when I get to it].

This yields: 
%
\begin{align}
\pdv{n _{\text{dm}}}{t} + \frac{1}{a} \pdv{}{x^{i}} \qty(n _{\text{dm}} v^{i}) + 3 \qty[\frac{1}{a} \dv{a}{t}  - \pdv{\Psi }{t}] n _{\text{dm}} = 0
\,,
\end{align}
%
which to 0th order yields 
%
\begin{align}
\pdv{n _{\text{dm}}^{(0)}}{t} + 3 \frac{1}{a} \dv{a}{t} n _{\text{dm}}^{(0)} \iff a^3 n _{\text{dm}}^{(0)} = \const
\,.
\end{align}
%
Now, we can write the full number density as 
%
\begin{align}
n _{\text{dm}} = n _{\text{dm}}^{(0)} \qty[1 + \delta (\vec{x}, t)]
\,,
\end{align}
%
where \(\delta \) is the standard notation in cosmology for these kinds of dimensionless fractional perturbations. 

Then, after some simplifications the evolution of this perturbation looks like 
%
\begin{align}
\pdv{ \delta  }{t} + \frac{1}{a} \pdv{v^{i}}{x^{i}} - 3 \pdv{\Psi }{t}=0
\,,
\end{align}
%
which is our first order continuity equation. 

This was for the first moment, for the next one we integrate after multiplying by \(\hat{p}^{j} / E\). 
We neglect a term because it is too relativistic, that is, second order in \(p / E\). 

Integrating by parts, for a different term we get 
%
\begin{align}
\int  \frac{ \dd[3]{p}}{(2\pi )^3} \pdv{f _{\text{dm}}}{p} \frac{p^2 \hat{p}^{i}}{E} &= \int \frac{ \dd{\Omega  } \hat{p}^{j}}{(2 \pi )^3}
\int \dots lni
&= - 4 n _{\text{dm}} v^{i}
\,.
\end{align}

We finally obtain 
%
\begin{align}
\pdv{(n _{\text{dm}} v^{i})}{t} + 4 \frac{1}{a} \dv{a}{t} n _{\text{dm}}v^{j} + \frac{n _{\text{dm}}}{a} \pdv{\Phi  }{x^{i}} =0
\,,
\end{align}
%
so we factor out the background and get, to first order, 
%
\begin{align}
\pdv{v^{j}}{t} + H v^{j} + \frac{1}{a} \pdv{\Phi }{x^{j}}=0
\,.
\end{align}

This is basically a linear Euler equation. In Fourier space, we get 
%
\begin{align}
\widetilde{\dot{\delta}} + i k \widetilde{v} - 3 \widetilde{\dot{\psi}} &= 0  \\
\widetilde{\dot{v}} + \frac{\dot{a}}{a} \widetilde{v} + i k \widetilde{\Phi} &= 0
\,.
\end{align}

\section{Boltzmann equation for baryons}

Which interaction are relevant? 
At recombination, electrons are tightly coupled with photons, while the interaction between protons and photons is suppressed by a factor \((m_{e} / m_p)^2\) with respect to to electron-photon scattering.
We need to deal with Coulomb scattering between electrons and proton. 
So, we can say that the quantity 
%
\begin{align}
\frac{ \rho_{e} - \rho_{p}^{(0)}}{\rho_{e}^{(0)}}  =
% \frac{ \rho_{e} - \rho_{p}^{(0)}}{\rho_{e}^{(0)}}  =
\rho_{b}
\,,
\end{align}
%
is a perturbation. So we write two Boltzmann equations, for electrons and protons, assuming that they move with the same velocity. The collision term looks like 
%
\begin{align}
C_{e \gamma }= (2 \pi)^{4} \delta^{(4)} (p + q - p' - q') 
\frac{ \abs{\mathcal{M}}^2 \qty[f_{e}(q') f_{\gamma }(p') - f_e (q) f_\gamma (p)] }{8 E(p) E(p') E_e(q) E_e(q')}
\,,
\end{align}
%
So the zeroth moment equation for electrons is the same as the one for Dark Matter: both of the terms in the RHS vanish in
%
\begin{align}
\pdv{n_e}{t} + \frac{1}{a} \pdv{}{x^{i}} (n_e v^{i}_{b}) + 3 \qty[H - \pdv{\Psi }{t}]n_e = \expval{C_{ep}} + \expval{C_{e \gamma }}
\,,
\end{align}
%
since they multiply antisymmetric terms in exchange of momenta [clear up]. 

For the first moment we sum the proton and electron equations multiplied by the respective charges. 
The symmetry of the momenta of electrons and protons cancels the term of Coulomb scattering in the first moment equation. So, we get 
%
\begin{align}
\pdv{v^{j}_{b}}{t} + H v_b^{j} + \frac{1}{a} \pdv{\Phi }{x^{j}} 
=\frac{1}{\rho_{b}} \expval{c_{e \gamma } q^{j}}_{pp'q'q}
\,,
\end{align}
%
but by total momentum conservation we can switch the momentum \(q^{j} \) in the equation to a \(p^{j}\). 
We Fourier transform, and multiply by \(\hat{k}^{j}\): then we get \(\vec{p} \cdot \hat{k} = p \mu \), where \( \mu = \cos \alpha \). 

Two pages from Dodelson \cite[]{dodelsonModernCosmology2003}.

The velocities of electrons and protons are the same to any perturbative order in our expansion, because of the tight-coupling assumption. 

In the Planckian, the only dependence is the time.

\end{document}
