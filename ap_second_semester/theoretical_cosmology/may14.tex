\documentclass[main.tex]{subfiles}
\begin{document}

\section{Statistical methods}

\marginpar{Thursday\\ 2020-5-14, \\ compiled \\ \today}

The \textbf{Fair Sample Hypothesis}: we see it from Peebles \cite[]{peeblesLargescaleStructureUniverse1980}.

In cosmology we do observations, not experiments: we cannot do spatial averages, only angular averages. 

Now, the statement is that the statistical properties of the universe are spatially uncorrelated on large enough scales. 

Now we move to the lectures by Martinez. 

The probability density functional is unaffected by translations.
We only can probe a specific realization of the field \(\psi (\vec{x})\), but in theory we discuss the probability \(p[\psi (\vec{x})]\).
We hope that what we observe is large enough to be a statistical sample. 

In the CMB we have \emph{cosmic variance}, due to the fact that we are a single observer.
We assume the ergodic theorem: ensemble averages and spatial averages are equivalent. 
The theorem holds for Gaussian random fields with a continuous power spectrum. 

\textbf{Correlation functions}: we take the joint probability, given two volumes \(\dd{V_1 }\) and \(\dd{V_2 }\) at a distance \(r\), of finding a galaxy in each. This will be 
%
\begin{align}
\dd{P}(r) = \overline{n}^2 \qty(1 + \xi (r)) \dd{V_1 } \dd{V_2 } 
\,.
\end{align}

The correction \(\xi (r)\) is precisely the two-point correlation function. 

Now we move to the notes by U. Natale. 

We look at conditional probability: note the typos in the notes there. 

We can get the mean number of galaxies in a certain sphere. 
So, we get a fractal structure of dimension \(3 - \gamma \). 

We can consider \emph{three-point} correlation functions, beyond two-point ones. We can express this with respect to two-point ones, plus a ``connected part'' \(\zeta \).

The connected part must go to zero when any of the lengths go to zero. 
By the definitions we gave, we can equate \(\xi \) with  \(\expval{ \delta_g (\vec{x})} \delta_{g} (\vec{x} + \vec{r})\).

We can go to Fourier space. Then, we define 
%
\begin{align}
\expval{ \delta_{k_1 } \delta_{k_2 }} = \qty(2 \pi )^3 
\delta^{(3)} (\vec{k}_{1} + \vec{k}_{2}) P_m (k)
\,.
\end{align}

We get an ultraviolet divergence, but it is not an issue: we are describing cosmological structures, so it is fine that we are not able to describe small structures.

\end{document}
