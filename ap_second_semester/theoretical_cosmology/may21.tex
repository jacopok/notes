\documentclass[main.tex]{subfiles}
\begin{document}

\marginpar{Thursday\\ 2020-5-21, \\ compiled \\ \today}

We want to define the notion of a \textbf{functional derivative}. 

The first order in \(\eta \) the difference of the functional applied to \(q\) and \(q + \eta \) is defined to be 
%
\begin{align}
F[q + \eta ] - F[q ] = \int  \fdv{F}{q(y)} \eta (y) \dd{y}
\,.
\end{align}

We can interpret this as 
%
\begin{subequations}
\begin{align}
\fdv{F}{q (y)} = \lim_{\substack{\tau \to 0 \\ q_{i} \to y} } \frac{1}{\tau } \pdv{\hat{F}}{q_{i}}
\,,
\end{align}
\end{subequations}
%
where we compute the functional using infinitesimal ``cells'' \(q_i\)  of volume \(\tau \) in coordinate space.

This behaves like a derivative, and we have 
%
\begin{align}
\fdv{q(x)}{q(y)} = \delta (x- y)
\,.
\end{align}

An example: if \(q(x)\) is a 1d function, define 
%
\begin{align}
F_{n}[q] = \int \prod_i \dd{x_{i}} q(x_{i}) f(x_1, \dots, x_{n})  
\,,
\end{align}
%
where \(f\) is a symmetric function of the coordinates. Then, 
%
\begin{align}
\fdv{F_{n}}{q(y)} = n \int \dots \int f(x_1 , \dots, x_{n-1}, y) q(x_1 ) \dots q(x_{n-1}) \dd{x_{1}} \dots \dd{x_{n-1}}
\,.
\end{align}

We can extend this to 
%
\begin{align}
\fdv[m]{F_{n}}{q(y_1 )\dots}{ q(y_m)}
= \frac{n!}{(n-m)!} \int  \dots \int f(x_1, \dots, x_{n-m}, y_1, \dots y_{m}) \dd{x_1 } \dots \dd{x_{n-m}}
\,.
\end{align}

Now, \textbf{linear functional transformations}
Consider a mapping \(q (x) \to q' (x)\) in the form 
%
\begin{align}
q(x) = \int \dd{y} K(x, y) q' (y)
\,.
\end{align}

If this is invertible, we can write 
%
\begin{align}
q' (x) = \int \dd{y} K^{-1}(x, y) q(y)
\,.
\end{align}

Often in field theory \(K(x, y)\) can be written as \(K(x-y)\) because of translation invariance. If these are compatible, then we must have 
%
\begin{align}
\int \dd{y} K(x, y) K^{-1} (y, z) = \delta (x-z)
\,.
\end{align}

We can do \textbf{Legendre Transforms} in this context as well: if we define 
%
\begin{align}
p(y) = \fdv{F[q]}{q(y)}
\,,
\end{align}
%
then we can write 
%
\begin{align}
G[p] = F[q] - \int q(x) p(x) \dd{x}
\,.
\end{align}

\todo[inline]{This has a different sign than what we do classically, \(H = p \cdot \dot{q} - L\). This has to do with the signature of the metric in QFT, and Wick rotations. In dealing with probability densities we must be general.}

We can also do \textbf{functional integration}: it is formally an integration over an infinite number of variables. 

How do we do changes of variable? We will need the determinant of the matrix \(K\) giving the change of variables. 
We only consider \emph{linear} ones. 

We find that, for a path integral with a quadratic and a linear term, 
%
\begin{align}
\mathscr{Z} [b] = \int \mathcal{D}q \exp(- \frac{1}{2} (B, K^{-1}, b))
\,.
\end{align}
%


\end{document}
