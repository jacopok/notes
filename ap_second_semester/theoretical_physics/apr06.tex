\documentclass[main.tex]{subfiles}
\begin{document}

\section{Dirac equation coupled to an external EM field}

\marginpar{Thursday\\ 2020-4-30, \\ compiled \\ \today}

We can use a complex solution to the Dirac equation to describe a spin \(1/2\) particle.

As we did before, we make the minimal coupling ansatz: 
%
\begin{align}
\partial_{\mu } \to \DD_{\mu } = \partial_{\mu } + i q A_{\mu } 
\,,
\end{align}
%
so that our Dirac equation will read 
%
\begin{align}
\qty(i\slashed{\DD} - M ) \psi = \qty(i \slashed{\partial} - q \slashed{A} - M ) = 0
\,,
\end{align}
%
which in three-vector notation (after the computation in the equations \eqref{eq:dirac-equation-from-ansatz-to-gamma}) reads 
%
\begin{align}
i \partial_0 \psi = \qty[i \vec{\alpha} \cdot \qty(- \vec{\nabla} + i q \vec{A}) + \beta M + q A_0 ] \psi 
\,,
\end{align}
%
where we are using the definition \(\partial_{i} = \vec{\nabla}\), while \(A^{i} = \vec{A} \): the ``natural'' placement for the spatial index of the derivative is lower, while for other vectors it is upper.

\subsection{Nonrelativistic limit of the Dirac equation}

As we did in the case of the Klein-Gordon equation, we start out by factoring out the time-evolution: we define 
%
\begin{align}
\psi (\vec{x}, t) = e^{-i M t} \psi' (\vec{x}, t)
\,,
\end{align}
%
which we can plug into the expression for the Dirac equation.

\end{document}