\documentclass[main.tex]{subfiles}
\begin{document}

\subsection{Classical system with infinite DoF}

\marginpar{Sunday\\ 2020-5-3, \\ compiled \\ \today}

We move from a system of particles to a field, basically. We make the following substitutions: 
\begin{enumerate}
    \item the coordinates \(q_{i}(t)\) become a field, a function of the coordinates \(\varphi (\vec{x}, t)\);
    \item sums over the coordinates, \(\sum _{i}\), become integrals in space: \(\int \dd[3]{x}\);
    \item the Lagrangian \(L(q_{i}, \dot{q}_{i})\) becomes a Lagrangian density \(\mathscr{L} (\varphi , \partial_{\mu } \varphi )\).  
\end{enumerate}

The Lagrangian density is a function of position in space, we can recover the full Lagrangian by integrating: 
%
\begin{align}
L(t) = \int \dd[3]{x} \mathscr{L}(\varphi (\vec{x}, t), \partial_{\mu } \varphi (\vec{x}, t)) 
\,.
\end{align}

The action is now a functional of the fields: 
%
\begin{align}
S [ \varphi ] = \int \dd[4]{x} \mathscr{L} (\varphi, \partial_{\mu } \varphi )
\,,
\end{align}
%
where we integrate in spacetime since we must integrate in \(\dd[3]{x}\) to recover the Lagrangian, and then in time to recover the action from the Lagrangian. 
Note that this yields a number as an output, and takes the whole field in spacetime as an input. As such, it is not a function of the spacetime coordinates.

The domain of integration is usually Minkowski spacetime with some boundary conditions at infinity: we assume that the fields and their derivatives vanish at infinity. 

As in the finite DoF case, we stationarize the action and set \(\delta S =0\). 
What does it mean to vary the field? We assume the variation to be ``synchronous'', that is, 
%
\begin{align}
\varphi '(x) = \varphi (x) + \delta_0 \varphi (x)
\,,
\end{align}
%
so the point in spacetime is not affected: only the field changes.

The variation of the action is given by 
%
\begin{align}
\delta_0 S [\varphi ] &= S [ \varphi + \delta_0 \varphi ] - S[\varphi ]  \\
&= \int \dd[4]{x} \qty{ \mathscr{L} (\varphi + \delta_0 \varphi , \partial_{\mu } \varphi + \delta_0 \partial_{\mu } \varphi ) - \mathscr{L} (\varphi, \partial_{\mu } \varphi )}  \\
&= \int \dd[4]{x } \delta_0 \mathscr{L} (\varphi , \partial_{\mu }\varphi )  \\
&= \int \dd[4]{x} 
\qty{ \pdv{\mathscr{L}}{\varphi } \delta_0 \varphi  + \pdv{\mathscr{L}}{\partial_{\mu } \varphi } \delta_0\partial_{\mu }  \varphi  }  \\
&= \int \dd[4]{x} \qty{\pdv{\mathscr{L}}{\varphi } - \partial_{\mu }\pdv{\mathscr{L}}{\partial_{\mu } \varphi }} \delta_0 \varphi 
\,.
\end{align}

We integrated by parts, and the boundary term vanished because of our boundary conditions on the fields. 
Also, we assumed that \(\delta_0 \qty(\dd[4]{x}) =0 \) in our second step.

In this case, like in the finite-DoF one, \(\qty[\partial_{\mu }, \delta_0 ] =0\). 

Since \(\delta_0 S = 0\) must hold for all \(\delta_0 \varphi \), the rest of the integrand must be equal to zero: these are the Euler-Lagrange equations of the theory, 
%
\begin{align}
\pdv{\mathscr{L}}{\varphi } - \partial_{\mu }\pdv{\mathscr{L}}{\partial_{\mu } \varphi } =0
\,.
\end{align}


\begin{claim}
Two Lagrangians which are equal up to the divergence of a function of the field, \(\partial_{\mu } k^{\mu }(\varphi )\), are equivalent (they have the same EL equations).
\end{claim} 

\begin{proof}
\todo[inline]{We need to make an additional assumpion: that the function of the field vanishes if the field vanishes.}

If we make this assumption, we can calculate the variation of the action we get after adding this term:
%
\begin{align}
\Delta S = \int \dd[4]{x} \partial_{\mu } k^{\mu } 
\,,
\end{align}
%
so we can calculate this in a region whose radius we send to infinity, and apply the divergence theorem: under the critical assumption that the function of the field vanishes like the field does as the field goes to zero, we get zero variation.
\end{proof}

\subsection{Interlude: functional derivatives}

We define a space \(\varphi \) of smooth functions: 
%
\begin{align}
\varphi = \qty{
 f:
\left|
  \begin{array}{rcl}
    M_{4} & \longrightarrow & \mathbb{R} \text{ or } \mathbb{C} \\
    x & \longmapsto & f(x) \\
  \end{array}
\right.
}
\,,
\end{align}
%
and define \textbf{functionals} to be maps from \(\varphi \) to \(\mathbb{R}\) (or \(\mathbb{C}\)), so the space of functionals is the dual of \(\varphi \): 
%
\begin{align}
 F:
\left|
  \begin{array}{rcl}
    \varphi  & \longrightarrow & \mathbb{R} \text{ or } \mathbb{C}   \\
    f & \longmapsto & F[f] \\
  \end{array}
\right.
\,.
\end{align}

We are giving general definitions, but the case we will be interested in is usually the one of real-valued functionals.

We encountered two functionals so far: the Lagrangian is a functional of the field \(\varphi (\vec{x}, \overline{t})\) for a fixed \(\overline{t}\); the action is a functional of the field \(\varphi (x^{\mu })\).



\end{document}
