\documentclass[main.tex]{subfiles}
\begin{document}

\section{Nöther's theorem}

\marginpar{Monday\\ 2020-5-4, \\ compiled \\ \today}

We can relate symmetries and conserved quantities. 

\begin{definition}
A \textbf{symmetry} of the theory is the transformation of the fields and/or of the spacetime coordinates such that the equations of motion do not change.
\end{definition}

This is equivalent to saying that the action is stationary for the transformed fields iff it is stationary for the untransformed ones. 

The Lagrangian's symmetries are different from those of the EoM: for instance, we have shown that \(\mathscr{L} \to \mathscr{L} + \partial_{\mu } k^{\mu }\) preserves them, but the Lagrangian definitely changes.

Symmetries can be classified into \textbf{discrete} and \textbf{continuous} ones, depending on the group they correspond to. 
A \emph{discrete} group is a group which is endowed with the discrete topology --- the one in which all the subsets are open. 
A \emph{continuous} group, for what we will need, is a Lie group: one which is described as a differentiable manifold. 

The discrete groups we use are usually finite, but they can be inifinite: for instance, translations by \(\SI{1}{m}\) along a certain axis are isomorphic to \(\mathbb{Z}\).  
Continuous groups are always uncountably infinite, since at the very least they are described by one real parameter. 
Here, we are interested in continuous symmetries, since Nöther's theorem applies to them.

We classify these based on two characteristics: first, we have the \textbf{global} versus \textbf{local} symmetries.
Local symmetries' parameters depend on the position in spacetime, global symmetries' parameters are constant. 

\textbf{Spacetime} symmetries alter only the spacetime coordinates, \textbf{internal} symmetries alter the fields as well as the spacetime coordinates. 

\begin{enumerate}
    \item Lorentz transformations \(x \to \Lambda x\) are global spacetime symmetries;
    \item general-relativistic diffeomorphisms \(x \to x'(x)\) are local spacetime symmetries;
    \item leptonic and hadronic number symmetries and flavour symmetry are global internal symmetries;
    \item gauge field symmetries like \(A_{\mu } \to A_{\mu } + \partial_{\mu } \Lambda \) are local internal symmetries. 
\end{enumerate}

Nöther's theorem associates a conserved current and charge to every continuous symmetry of the action: the current is \(j^{\mu }_{(a)}\), where the index \(\mu \) is a spacetime one, while \((a)\) labels the generators of the group of symmetry. This current will be conserved: 
%
\begin{align}
\partial_{\mu } j^{\mu }_{(a)} = 0
\,.
\end{align}

A charge can be associated with this current: 
%
\begin{align}
Q_{(a)} = \int \dd[3]{x} j^{0}_{(a)}
\,.
\end{align}

This charge is conserved: \(\partial_{t} Q_{(a)} =0 \) (note that this is a partial derivative but it can also be written as a total one: the total charge only depends on time). This statement follows from the conservation \(\partial_{\mu } j^{\mu }_{(a)} =0 \): if we integrate this law over a cylinder, the product of an interval in time by a spatial sphere with diverging radius, we get 
%
\begin{align}
\int \dd[4]{x} \partial_{\mu } j^{\mu }_{(a)} = \int \dd{t} \dd[3]{x} \qty(\partial_{0} j^{0}_{(a)} + \partial_{i}j^{i}_{(a)}) = Q_{(a)}(t _{\text{fin}}) - Q_{(a)} (t _{\text{in}})
\,,
\end{align}
%
since the integral of the divergence vanishes by the usual boundary conditions. 

\subsection{Nöther's theorem for internal global symmetries}

An internal global symmetry leaves the spacetime coordinates unchanged, and changes the fields as 
%
\begin{align}
\varphi \to \varphi + \delta_0 \varphi 
\qquad \text{where} \qquad
\delta_0 \varphi = \epsilon^{(a)} X_{(a)}(\varphi )
\,.
\end{align}

Here \(\epsilon^{(a)}\) is a vector of small constant parameters, while \(X_{(a)}\) are the generators of the symmetry. 
The index \(a\) runs from 1 to \(n\), the dimension of the symmetry group. 

So, we can take the variation of the perturbed action \(\delta S[\varphi + \delta_0 \varphi ]\) and set it to zero. An important note, though: the variation of the action must be zero \emph{locally} as well as globally, so we can perform our integral in an arbitrary spacetime volume \(V\). We find: 
%
\begin{align}
\eval{\delta S [ \varphi + \delta_0 \varphi ]}_{V} &= \delta  \int_{V} \dd[4]{x} \mathscr{L} \qty(\varphi + \delta_0 \varphi , \partial_{\mu } \varphi + \partial_{\mu } \delta_0 \varphi  )  \\
&= \int_{V} \dd[4]{x} \pdv{\mathscr{L}}{\varphi } \delta_0 \varphi + \pdv{\mathscr{L}}{\partial_{\mu}\varphi } \qty(\partial_{\mu } \delta_0 \varphi )  \\
&= \int_{V} \dd[4]{x} \pdv{\mathscr{L}}{\varphi } \delta_0 \varphi - \partial_{\mu } \qty(\pdv{\mathscr{L}}{\partial_{\mu}\varphi } ) \delta_0 \varphi  + \partial_{\mu } \qty( \pdv{\mathscr{L}}{\partial_{\mu } \varphi } \delta_0 \varphi ) \marginnote{Integrated by parts.}
\,.
\end{align}

Unlike what we usually do when we derive the EL equations, we cannot neglect the boundary term: since we are integrating in an arbitrary volume, there is no reason why the variation of the field should vanish at the boundary (\(\eval{\delta_0 \varphi }_{\partial V} =0\)). 

The terms which are multiplied by \(\delta_0 \varphi \) vanish by the regular Euler-Lagrange equations (which hold everywhere). Therefore, in order for the variation in the action to be zero, we must ask the extra term to vanish: this means 
%
\begin{align}
\int_{V} \dd[4]{x} \partial_{\mu } \qty(\pdv{\mathscr{L}}{\partial_{\mu }\varphi } \delta_0 \varphi ) 
=\epsilon^{(a)} \int_{V} \dd[4]{x} \partial_{\mu } \qty( \pdv{\mathscr{L}}{\partial_{\mu } \varphi } X_{(a)} \varphi )
= 0
\,,
\end{align}
%
which must hold: 
\begin{enumerate}
    \item for any region \(V\);
    \item for any choice of \(\epsilon^{(a)}\).
\end{enumerate}

Therefore, the only possibility is that 
%
\begin{align}
\partial_{\mu  }\qty( \pdv{\mathscr{L}}{\partial_{\mu } \varphi } X_{(a)} (\varphi) ) = 0 
\,,
\end{align}
%
for all choices of \(a\). This is a conservation equation just like the one we mentioned above: so, we define 
%
\begin{align}
j^{\mu }_{(a)} = \pdv{\mathscr{L}}{\partial_{\mu } \varphi } X_{(a)} (\varphi)
\,
\end{align}
%
and we are done.
The conserved charges will look like 
%
\begin{align}
Q_{(a)} = \int \dd[3]{x} j^{0}_{(a)} = \int \dd[3]{x} \pdv{\mathscr{L}}{\partial_{0} \varphi } X_{(a)} (\varphi )
\,.
\end{align}

\subsection{Nöther's theorem for spacetime global symmetries}

Let us consider a more general formulation of the theorem: the statement is the same, but we generalize it to a symmetry which acts both on the spacetime coordinates and on the fields. 

The variations will be 
%
\begin{align}
\delta {x^{\mu }} &= \epsilon^{(a)} Y_{(a)}^{\mu }  \\
\delta \varphi &= \epsilon^{(a)} X_{(a)}(\varphi )
\,.
\end{align}

Here, \(X_{(a)}\) are the generators of the action of the group on \(\varphi \), while \(Y_{(a)}\) are the generators of its action on \(x^{\mu }\). 

The variation is now \(\delta \) instead of \(\delta_0 \): it is not synchronous anymore. To be more explicit: 
%
\begin{align}
\delta \varphi &= \varphi' (x') - \varphi (x)  \\
&= \varphi ' (x') - \varphi (x') + \varphi (x') - \varphi (x)  \marginnote{Added and subtracted.}\\
&\approx  \delta_0 \varphi + (\partial_{\mu } \varphi ) \delta x^{\mu }
\,.
\end{align}

Here \(\varphi \) is a test field, but this reasoning holds for any function. 

Note that, while the synchronous variation \(\delta_0\) commutes with \(\partial_{\mu }\), this is not the case for the total variation \(\delta \)! In fact, we have 
%
\begin{align}
\qty[\partial_{\mu }, \delta ] \varphi = \qty(\partial_{\nu } \varphi ) \partial_{\mu } \delta x^{\nu }
\,.
\end{align}

Also, in general if we transform the coordinates we are also modifying the volume element: 
%
\begin{align}
\dd[4]{x'} \approx \qty(1 + \partial_{\mu } \delta x^{\mu }) \dd[4]{x}
\,.
\end{align}

Therefore, we can write \(\delta \qty(\dd[4]{x}) = \partial_{\mu } \delta x^{\mu } \dd[4]{x}\).

\begin{proof}
This can be shown by using Liouville's formula \cite[eq. 4.9]{winitzkiLinearAlgebraExterior2010}:
%
\begin{align}
\det \exp(M) = \exp(\Tr M)
\,.
\end{align}

The variation of the volume element is given, in full generality, by 
%
\begin{align}
\dd[4]{x'} = \abs{ \pdv{x'}{x}} \dd[4]{x}
\,,
\end{align}
%
where the Jacobian is given by:
%
\begin{align}
\pdv{x^{\prime \mu }}{x^{\nu }} = \delta^{\mu }_{\nu } + \partial_{\nu } \delta x^{\mu } \approx \exp( \partial_{\nu } \delta x^{\mu }) = \exp(M)
\,.
\end{align}

So, we can apply the formula to get: 
%
\begin{align}
\det \qty(\pdv{x^{\prime \mu }}{x^{\nu }}) &= \exp(\Tr M)
= \exp(\Tr(\partial_{\nu } \delta x^{\mu }))   \\
&= \exp( \partial_{\mu } \delta x^{\mu })  \\
&\approx 1 + \partial_{\mu } \delta x^{\mu }
\,,
\end{align}
%
which proves our claim. One plus a small value is positive, so there is no issue with the modulus. 
\end{proof}

So, let us get to the computation: as before, we integrate in a spacetime region \(V\), and we find 
%
\begin{align}
\delta S &= \int_{V} \delta \qty(\dd[4]{x}) \mathscr{L} + \int \dd[4]{x} \delta \mathscr{L}  \\
&= \int_V \dd[4]{x} \qty{ \partial_{\mu } \delta x^{\mu } \mathscr{L} + 
\delta_{0} \mathscr{L} + \qty(\partial_{\mu } \mathscr{L}) \delta x^{\mu}
}  \\
&= \int_V \dd[4]{x}  \partial_{\mu } \qty(\qty(\mathscr{L} \delta x^{\mu }) + \pdv{\mathscr{L}}{\partial_{\mu } \varphi } \delta_0  \varphi )
+ \text{LE} \times \delta_0 \varphi  
\,,
\end{align}
%
where the reasoning for the \(\delta_0 \mathscr{L}\) part is the same as in the internal symmetry case; while the two new additional terms were combined into a single derivative. 

As before, the Lagrange Equations vanish identically, and since we are integrating in a generic region \(V\) the rest of the integrand must be zero everything: so, we have found a conserved current! 

It is, however, still written in terms of \(\delta_0 \), while the variation of the field is not the synchronous one: so, we substitute the expression for \(\delta_0 \varphi \) in terms of \(\delta \varphi \). This yields 
%
\begin{align}
\delta S &=
\int_{V} \dd[4]{x} \partial_{\mu } \qty(\mathscr{L} \delta x^{\mu } + \pdv{\mathscr{L}}{\partial_{\mu } \varphi } \qty(\delta \varphi - \partial_{\rho  } \varphi \delta x^{\rho })  )  \\
&= \int_{V} \dd[4]{x} \partial_{\mu } \qty[ \qty(\mathscr{L} \delta^{\mu }_{\rho } - \pdv{\mathscr{L}}{\partial_{\mu } \varphi } \partial_{\rho } \varphi ) \delta x^{\rho }+ \pdv{\mathscr{L}}{\partial_{\mu }\varphi } \delta \varphi ]  \\
&= \epsilon^{(a)}\int_{V} \dd[4]{x} \partial_{\mu } \qty[ \qty(\mathscr{L} \delta^{\mu }_{\rho } - \pdv{\mathscr{L}}{\partial_{\mu } \varphi } \partial_{\rho } \varphi ) Y^{\rho }_{(a)}+ \pdv{\mathscr{L}}{\partial_{\mu }\varphi } X_{(a)} ]
\,,
\end{align}
%
which, as before, must hold for any volume \(V\) and for any \(\epsilon^{(a)}\), therefore we get that the integrand must be conserved: our final result for the conserved current is 
%
\begin{align}
j^{\mu }_{(a)} = \qty(\pdv{\mathscr{L}}{\partial_{\mu } \varphi } \partial_{\rho } \varphi  - \mathscr{L} \delta^{\mu }_{\rho } ) Y^{\rho }_{(a)}-  \pdv{\mathscr{L}}{\partial_{\mu }\varphi } X_{(a)}
\,,
\end{align}
%
where we changed the global sign; this is purely conventional, as any scalar multiple of the conserved current is also conserved.

\subsection{Application: Poincaré invariance} \label{sec:poincare-invariance-noether}

We want our theories to respect the principles of special relativity. So, the action must be a Lorentz scalar. 
Poincaré transformations are the most general ones in SR, they consist in combinations of translations (4 generators), Lorentz boosts (3 generators) and Lorentz rotations (3 generators).

\subsubsection{Translations}

Translations shift the coordinates by constant amounts, and should not alter the fields. 

We then have 
%
\begin{align}
\delta x^{\mu } &= \epsilon^{\mu } = \epsilon^{\nu } \delta^{\mu }_{\nu } = \epsilon^{(\nu)} Y^{\mu }_{(\nu )}  \\
\delta \varphi &= 0 = X
\,.
\end{align}

So, our conserved current reads 
%
\begin{align}
j^{\mu } _{(\nu), \text{transl}} = \pdv{\mathscr{L}}{\partial_{\mu } \varphi } \partial_{\nu } \varphi  -  \mathscr{L} \delta^{\mu }_{(\nu )} \overset{\text{def}}{=} \widetilde{T}^{\mu }_{\nu }
\,,
\end{align}
%
which is called the canonical stress-energy-momentum tensor.
Note that this tensor is not generally symmetric as written, although there are procedures to make it so. 

The components \(\widetilde{T}^{0}_{\nu }\) are called the \emph{momentum density} associated to the field; their integrals over 3-space are the components of the total momentum \(P^{\mu }\). They are 
%
\begin{align}
P_{\mu } = \int \dd[3]{x} \widetilde{T}^{0}_{\mu } = \int \dd[3]{x} \qty(\pdv{\mathscr{L}}{\partial_{0} \varphi } \partial_{\mu } \varphi  - \mathscr{L} \delta^{0}_{\mu })
\,,
\end{align}
%
so the \(00\) component \(P_{0}\) is precisely the Hamiltonian, which is defined by the Legendre transform: 
%
\begin{align}
H = \int \dd[3]{x} \pdv{\mathscr{L}}{\partial_{0} \varphi } \partial_{0} \varphi - \mathscr{L}
\,.
\end{align}

\subsubsection{Lorentz transformations}

Lorentz rotations and boosts are generated by antisymmetric tensors \(\omega_{\mu \nu  }\). These have six degrees of freedom. They alter the fields as 
%
\begin{align}
x^{\prime \mu } &= x^{\mu } + \omega^{\mu }_{\nu }x^{\nu }  \\
\varphi '(x') &= \varphi (x) - \frac{i}{2} \omega^{\mu \nu } \Sigma_{\mu \nu } \varphi (x)
\,,
\end{align}
%
where \(\Sigma_{\mu \nu }\) are the representations of the generators of Lorentz transformations for our field: for scalar fields they are zero, for a Dirac spinor we found that they can be expressed as 
%
\begin{align}
\Sigma^{\mu \nu } = \frac{i}{4} \qty[\gamma^{\mu }, \gamma^{\nu }]
\,.
\end{align}

So, we can express the transformations in our general form:
%
\begin{align}
\delta x^{\mu } &= \frac{1}{2} \omega^{\nu \rho } Y^{\mu }_{(\nu \rho )}  \\
\delta \varphi  &= \frac{1}{2} \omega^{\mu \nu } X_{(\mu \nu )}
\,,
\end{align}
%
where we included a \(1/2\) in the coordinate transformations since we are summing over two antisymmetric indices, so we will have two copies of every term. In order for these expressions to be equal, we need to set: 
%
\begin{align}
\omega^{\mu }_{\nu }x^{\nu } &= \frac{1}{2} \omega^{\nu \rho } Y^{\mu }_{(\nu \rho )}  \\
\frac{-i}{2} \omega^{\mu \nu } \Sigma_{\mu \nu } &=\frac{1}{2} \omega^{\mu \nu } X_{(\mu \nu )}
\,,
\end{align}
%
which is solved by 
%
\begin{align}
Y^{\mu }_{(\nu \rho )} &= \qty(\delta ^{\mu }_{\rho } \eta_{\nu \sigma } - \delta ^{\mu }_{\sigma } \eta_{\nu \rho })x^{\nu }  \\
X_{(\rho \sigma )} &= -i \Sigma_{\rho \sigma }
\varphi 
\,.
\end{align}

Note that although we are using two indices for the bookkeeping of these transformations there are only six independent ones by antisymmetry. 

The conserved currents are then given directly from the formula: 
%
\begin{align}
J^{\mu }_{(\rho \sigma )} &= 2 x_{[\rho } \widetilde{T}_{\sigma ]}^{\mu } - i \pdv{\mathscr{L}}{\partial_{\mu }\varphi } \Sigma_{\rho \sigma } \varphi  \label{eq:conserved-currents-Lorentz}\\
&= L^{\mu }_{\rho \sigma } + S^{\mu }_{\rho \sigma }
\,.
\end{align}

\todo[inline]{Add first computation step}

So, we have both an ``external'' angular momentum term and a spin term --- as long as the generators \(\Sigma_{\mu  \nu }\) are nonzero! The fact that a scalar field carries no spin can be read off from here. 

\end{document}
