\documentclass[main.tex]{subfiles}
\begin{document}

\section{Relativistic classical field theory}

\marginpar{Friday\\ 2020-5-8, \\ compiled \\ \today}

We want to build a relativistic field theory starting from a lagrangian density \(\mathscr{L}\). 
What conditions should we impose on it?

\begin{enumerate}
    \item The action \(S = \int \dd[4]{x} \mathscr{L}\) must be a real functional, so \(\mathscr{L}\) must be a real function.
    \item The action \(S\) must be adimensional,\footnote{Why? Well, the simplest argument I can think of is that is should have the same dimensionality it has in classical theory. Beyond that, when we write a path integral 
    %
    \begin{align}
    K \sim \int \mathcal{D} x e^{- i S}
    \,
    \end{align}
    %
    the action goes into the argument of the exponential, so it must be adimensional.} so the Lagrangian must have the dimensions of a length to the \(-4\), which is equivalent to a mass to the \(4\)th power.
    \item The action \(S\) must be a Lorentz scalar, therefore (since \(\dd[4]{x}\) is a scalar) the Lagrangian must also be a scalar. 
    \item We want second-order equations of motion, so we ask that \(\mathscr{L} = \mathscr{L}( \varphi , \partial_{\mu }\varphi )\).    
\end{enumerate}

\subsection{Real scalar field}

We consider a real scalar field \(\varphi (x)\). In analogy to classical theory, we choose the Lagrangian 
%
\begin{align}
\mathscr{L} = \underbrace{\frac{1}{2} \qty(\partial_{\mu } \varphi ) \qty(\partial^{\mu } \varphi )}_{\text{kinetic term}} - \underbrace{\frac{1}{2} m^2 \varphi^2}_{\text{mass term}}
\,.
\end{align}

This is a \emph{free particle} Lagrangian: it does not contain powers of \(\varphi \) which are higher than two. These higher terms would represent interactions. 
In order for our conditions to hold, the dimensions of both \(m\) and \(\varphi \) must be those of a mass. 

The Euler-Lagrange equations read: 
%
\begin{align}
\qty(\square + m^2) \varphi = 0
\,,
\end{align}
%
which is the Klein-Gordon equation. 
As we saw in section \ref{sec:solutions-free-KG-equation}, the solutions to this equation are given by 
%
\begin{align}
\varphi (x) = 
\frac{1}{(2 \pi )^{3/2}} 
\int \frac{ \dd[3]{k}}{\sqrt{2 \omega_{k}}}
\qty(a(k) e^{-ikx} + a^{*}(k) e^{ikx})_{k^{0} = \omega_{k}}
\,.
\end{align}

Let us also express this in the Hamiltonian formulation: the momentum is given by 
%
\begin{align}
\pi (x) = \pdv{\mathscr{L}}{\partial^{0} \varphi } = \partial_{0} \varphi 
\,,
\end{align}
%
and using it we can write the Hamiltonian density with the usual formula: 
%
\begin{align}
\mathscr{H} = \pi \partial_{0} \varphi - \mathscr{L} = \frac{1}{2} \pi^2 + \frac{1}{2} \qty(\nabla \varphi )^2 + \frac{1}{2} m^2 \varphi^2
\,.
\end{align}

Note that this is positive definite: so the Hamiltonian, its integral (\(H = \int \mathscr{H} \dd[3]{x}\)) also is.
We can write down the Hamilton equations and see that they are equivalent to the Lagrange ones: they read 
%
\begin{align}
\dot{\varphi}_{t} (\vec{x}) &= \fdv{H}{\pi_{t} (\vec{x})}
 = \pi = \partial_{0} \varphi   \\
\dot{\pi}_{t} &= - \fdv{H}{\varphi_{t}}  \\
&= - \frac{1}{2} \fdv{}{\varphi } \int \pi^2 + \qty(\nabla \varphi)^2 + m^2 \varphi^2 \dd[3]{x}  \\
&= - \fdv{}{\varphi } \int \dd[3]{x} \qty[- \varphi \nabla^2 \varphi + \frac{1}{2} m^2 \varphi ]  \\
&= \nabla^2\varphi  - m^2 \varphi 
\,,
\end{align}
%
where we used the fact that 
%
\begin{align}
\frac{1}{2}\fdv{}{\varphi(x) } \int \qty(\nabla \varphi(y) )^2 \dd[3]{y} 
&= \int \dd[3]{y} \fdv{ \nabla \varphi (y) }{\varphi (x) } \cdot \nabla \varphi (y) \\
&= \int \dd[3]{y} \nabla \qty(\fdv{\varphi (y) }{\varphi( x) } ) \cdot\qty(\nabla \varphi (y) )  \\
&= - \int \dd[3]{y} \fdv{\varphi (y) }{\varphi(x) } \nabla^2 \varphi (y)   \\
&= - \int \dd[3]{y} \delta^{(3)} (x- y) \nabla^2 \varphi (y)
= - \nabla^2\varphi (x)
\,.
\end{align}

By the KG equation, the result can also be written as \(\partial_{0}^2 \varphi\).

Let us now discuss the \textbf{Nöther currents} associated with the symmetries of this Lagrangian. It does not explicitly depend on space nor time, so it has \textbf{Poincaré invariance}: the associated canonical stress-energy-momentum tensor is given by 
%
\begin{align}
\widetilde{T}^{\mu }_{\nu }
= \qty(\partial^{\mu } \varphi ) \qty(\partial_{\nu } \varphi ) - \delta^{\mu }_{\nu }\mathscr{L} 
\,.
\end{align}

Note that, in this case, the stress-energy tensor is symmetric (\(\widetilde{T}^{\mu \nu } = \widetilde{T}^{\nu \mu }\)). This is not a general fact, but it can be shown that the canonical stress-energy tensor can be made into a symmetric tensor in general: this is the so-called Belinfante-Rosenfeld stress-energy tensor. 

The components of the total momentum are given by integrating \(\widetilde{T}^{0}_{\nu }\) over 3-space: they are 
%
\begin{align}
P_{0} &= \int \widetilde{T}^{0}_{0} = \int \dd[3]{x} \qty(\partial_{0}\varphi )^2  - \mathscr{L} = \int \dd[3]{x} \mathscr{H}= H  \\
P_{i} &= \int \widetilde{T}^{0}_{i} = \int \dd[3]{x} \qty(\partial^{0}\varphi) \qty(\partial_{i} \varphi )
\,.
\end{align}

\begin{claim}
We can check explicitly that these currents are indeed conserved, that is, that \(\partial_{\mu } \widetilde{T}^{\mu }_{\nu } =0 \) is an identity. 
\end{claim}

\begin{proof}
The computation goes as such: 
%
\begin{align}
\partial_{\mu } \widetilde{T}^{\mu }_{\nu } &= \partial_{\mu } \qty(\partial^{\mu } \varphi \partial_{\nu } \varphi ) - \partial_{\nu } \mathscr{L}  \\
&= \partial_{\mu } \partial^{\mu } \varphi \partial_{\nu } \varphi 
+ \partial^{\mu }\varphi \partial_{\mu } \partial_{\nu } \varphi 
- \frac{1}{2} \partial_{\nu } \qty( \partial_{\mu } \varphi \partial^{\mu } \varphi  -  m^2 \varphi^2)  \\
&= \partial_{\mu } \partial^{\mu } \varphi \partial_{\nu } \varphi 
+ \partial^{\mu }\varphi \partial_{\mu } \partial_{\nu } \varphi 
- \partial_{\nu } \partial_{\mu } \varphi \partial^{\mu } \varphi 
+ m^2 \varphi \partial_{\nu } \varphi   
\marginnote{The \(1/2\) goes away since we are differentiating squares.}
\\
&= (\square \varphi + m^2 \varphi ) \partial_{\nu }\varphi  = 0
\,,
\end{align}
%
where derivatives are always intended to only act on what's directly in front of them, and where finally we recover the KG equation. 
\end{proof}

The Lagrangian is also invariant under \textbf{Lorentz transformations}: 
since our field is a Lorentz scalar it will not change under them, therefore the generators \(X\) must be zero. 

So, in equation \eqref{eq:conserved-currents-Lorentz} we will have \(\Sigma_{\rho \sigma } = 0\), which means 
%
\begin{align}
J^{\mu }_{(\rho \sigma )} = 2 x_{[\rho } \widetilde{T}^{\mu }_{\sigma ]} = L^{\mu }_{\rho \sigma }
\,,
\end{align}
%
The current density associated with the regular angular momentum. 
The integral of its \(\mu = 0\) component in \(\dd[3]{x}\) yields the total angular momentum. 

\begin{claim}
This is indeed a conserved current: \(\partial_{\mu } J^{\mu }_{\rho \sigma } = 0\). 
\end{claim}

\begin{proof}
We take the divergence explicitly: 
%
\begin{align}
\partial_{\mu } J^{\mu }_{\rho \sigma } &= \partial_{\mu } \qty(2 x_{[\rho } \widetilde{T}^{\mu }_{\sigma ]})  \\
&= 2 \qty(\partial_{\mu } x_{[\rho } ) \widetilde{T}^{\mu }_{\sigma ]}
+ 2 x_{[\rho } \partial_{\mu } \widetilde{T}^{\mu }_{\sigma ]}   \\
&= 2 \eta_{\mu [\rho } \widetilde{T}^{\mu }_{\sigma ]} = \widetilde{T}_{[\rho \sigma ]} = 0 
\,,
\end{align}
%
where we have used the fact that the canonical stress-energy tensor for a real scalar field is conserved: \(\partial_{\mu } \widetilde{T}^{\mu }_{\nu } = 0\), and symmetric. 
\end{proof}

\subsection{Complex scalar field}

Now, let us consider a complex scalar field, such that in general \(\varphi \neq \varphi^{*}\) and \(\varphi ' (x') = \varphi (x)\) under Poincaré transformations.

Then, we can write our Lagrangian as 
%
\begin{align}
\mathscr{L} = \qty(\partial^{\mu } \varphi^{*}) \qty(\partial_{\mu } \varphi )
- m^2 \abs{\varphi }^2
\,,
\end{align}
%
where we dropped the \(1/2\) factor from the real case. 
This is equivalent, since Lagrangians are defined up to a multiplicative factor, but there is a reason for it. 

The complex field \(\varphi_{\mathbb{C}}\) represents two real degrees of freedom: 
%
\begin{align}
\varphi_{\mathbb{C}} (x) = \frac{\varphi_{\mathbb{R}, 1}(1) + i \varphi_{\mathbb{R}, 2}}{\sqrt{2}}
\,,
\end{align}
%
where we included a division by \(\sqrt{2}\) for normalization purposes.

This is what makes us remove the \(1/2\) factor: if we take real fields to be special cases of the complex one, for each one we consider we must divide by \(\sqrt{2}\), and since the Lagrangian is quadratic this yields the global factor. 

We can recover the two real DoF by either adding or subtracting  \(\varphi \) and \(\varphi^{*}\). 
In terms of the real fields we can rewrite the Lagrangian as 
%
\begin{align}
\mathscr{L} &= \qty(\partial_{\mu } \frac{\varphi_1 -i \varphi_2 }{\sqrt{2}})
\qty(\partial^{\mu } \frac{\varphi_1 +i \varphi_2 }{\sqrt{2}})
- m^2 \frac{\varphi_1 -i \varphi_2 }{\sqrt{2}} \frac{\varphi_1 + i \varphi_2 }{\sqrt{2}}  \\
&= \frac{1}{2} \partial_{\mu } \varphi_1 \partial^{\mu } \varphi_1 
+ \frac{1}{2} \partial_{\mu } \varphi_2 \partial^{\mu } \varphi_2 
- \frac{1}{2} m^2 \varphi_1^2 - \frac{1}{2} m^2 \varphi_2^2  \\
&= \mathscr{L}_{1} + \mathscr{L}_{2}
\,.
\end{align}

Therefore, the two scalar degrees of freedom are fully decoupled: each has its own Lagrangian, and there is no interaction between them.

In order to derive the equations of motion we can differentiate with respect to either \(\varphi \) or \(\varphi^{*}\). This yields the Klein-Gordon equation for \(\varphi^{*}\) and \(\varphi \) respectively. 

So, both \(\varphi \) and \(\varphi^{*}\) satisfy the KG equation. 

We have the general form of the solution in terms of the coefficients \(a(k)\) and \(b^{*}(k)\).

We can define the momenta 
%
\begin{align}
\pi (x) &= \pdv{\mathscr{L}}{\partial_0 \varphi} = \partial_{0} \varphi^{*}  \\
\pi^{*} (x) &= \pdv{\mathscr{L}}{\partial_0 \varphi^{*}} = \partial_{0} \varphi  
\,.
\end{align}

In order to fully describe the field, we need both the pair \(\varphi, \pi  \) and \(\varphi^{*}, \pi^{*}\). 
The Hamiltonian density is defined as usual: it is 
%
\begin{align}
\mathscr{H} = \pi^{* } \pi + \qty(\nabla \varphi )^{*} \cdot \qty(\nabla \varphi ) + m^2\varphi^{*} \varphi \geq 0
\,.
\end{align}



\end{document}