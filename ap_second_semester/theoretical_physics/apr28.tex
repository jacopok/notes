\documentclass[main.tex]{subfiles}
\begin{document}

\section{Canonical quantization}

\subsection{System with finite DoF}

\marginpar{Tuesday\\ 2020-5-12, \\ compiled \\ \today}

The classical structure of such a system is completely determined by the equations of motion: 
%
\begin{align}
\dot{q} &= \qty{q, H}  \\
\dot{p} &= \qty{p, H}
\,,
\end{align}
%
and by the commutation relations: 
%
\begin{align}
\qty{q^{i}, q^{j}} &= 0  \\
\qty{p_{i}, p_{j}} &= 0  \\
\qty{q^{i}, p_{j}} &= \delta_{j}^{i}
\,.
\end{align}

Note that these Poisson brackets are to be calculated at a fixed time. 

We can \textbf{quantize} such a system by the following substitutions: 
\begin{enumerate}
    \item the coordinates \(q\) and \(p\), which are functions of phase space in the classical case, become operators: \(X\) and \(P\), which in the Heisenberg picture are functions of time;
    \item the Poisson brackets \(\qty{\cdot, \cdot}\) become commutators: 
    %
    \begin{align}
    \qty{a, b} \to \frac{1}{i \hbar} \qty[A, B]
    \,.
    \end{align}    
\end{enumerate}

The Hamilton equations then read 
%
\begin{align}
\dv{X}{t} &= \frac{1}{i \hbar} \qty[X, H] \\
\dv{P}{t} &= \frac{1}{i \hbar} \qty[P, H]
\,,
\end{align}
%
and the commutation relations are also generalized; the interesting one is the position-momentum commutator which now reads 
%
\begin{align}
\frac{1}{i \hbar} \qty[X,P] = 1 \implies
\qty[X, P] = i \hbar
\,.
\end{align}

\begin{claim}
The Heisenberg and Schrödinger descriptions of QM are equivalent.
\end{claim}

\begin{proof}
To say that the two approaches are equivalent means that they yield the same exact predictions for any observation.

In the Schrödinger approach, the wavefunction evolves as \(\ket{\psi (t)} = U(t) \ket{\psi_0 }\) while the observables are stationary; in the Heisenberg approach the wavefunction is stationary while the operators evolve as \(A(t) = U ^\dag A U\). So, the expectation value is 
%
\begin{align}
\expval{A(t)}_{\psi_0 } &\overset{?}{=} \expval{A}_{\psi (t)}  \\
\bra{\psi_0 } A(t) \ket{\psi_0 } &\overset{?}{=} \bra{\psi (t)} A \ket{\psi (t)} 
\bra{\psi_0 } U ^\dag A U\ket{\psi_0 } &= \bra{\psi_0 } U ^\dag A U \ket{\psi_0 } 
\,.
\end{align}
\end{proof}

\subsection{Field theory}

The evolution of the fields is given in classical field theory as 
%
\begin{align}
\dot{\varphi} &= \qty{\varphi , t}
\dot{\pi} &= \qty{\pi , t}
\,,
\end{align}
%
and we have the commutation relations 
%
\begin{align}
\qty{\varphi (x), \varphi (y)} &=0  \\
\qty{ \pi (x), \pi (y)} &= 0  \\
\qty{\varphi (x), \pi (y)} &= \delta^{(3)}(x-y)
\,,
\end{align}
%
all considered at constant time. 

In order to \textbf{quantize} this system, we replace the fields \(\varphi \) and \(\pi \) by field operators \(\hat{\varphi}\) and \(\hat{\pi}\), and replace the Poisson brackets by commutators as in the finite-DoF case.

The equations of motion will then read 
%
\begin{align}
\dv{\hat{\varphi}}{t} &= \frac{1}{i \hbar} \qty[\hat{\varphi}, H] \\
\dv{\hat{\pi}}{t} &= \frac{1}{i \hbar} \qty[\hat{\pi}, H] 
\,,
\end{align}
%
and the commutation relations will read 
%
\begin{align}
\qty[\hat{\varphi}(x), \hat{\pi }(y)] = i \hbar \delta^{(3)}(x-y)
\,,
\end{align}
%
and the zero ones as usual between \(\varphi, \varphi \) and \(\pi , \pi \).
These commutators are always taken at constant time. 

\subsection{Canonical quantization of a scalar field}

As we have seen before, the field and momentum of the free scalar field, which satisfies the Klein-Gordon equation, can be expressed as 
%
\begin{align}
\varphi (x) &= \frac{1}{(2\pi )^{3/2}} \int \frac{ \dd[3]{k}}{\sqrt{2 \omega_{k}}} \qty[a(k) e^{-ikx} + a^{\dag}(k) e^{ikx}]_{k^{0} =\omega_{k}} \\
\pi (x) &= \frac{1}{(2\pi )^{3/2}} \int \frac{ \dd[3]{k}}{\sqrt{2 \omega_{k}}} (-i \omega_{k}) \qty[a(k) e^{-ikx} - a^{\dag}(k) e^{ikx}]_{k^{0} =\omega_{k}} 
\,,
\end{align}
%
and we will now interpret this as an operator expression: even if we omit the hats, \(\varphi \) and \(\pi \) are intended to be operators. 
Therefore, \(a\) and \(a^{\dag}\) must also be. 

\begin{claim}
We can invert this relation to get \(a\) and \(a^{\dag}\) in terms of \(\varphi \) and \(\pi \): 
%
\begin{align}
    a(k) = \frac{1}{(2 \pi )^{3/2}} \int \frac{ \dd[3]{x}}{\sqrt{2 \omega_{k}}} \qty[\omega_{k} \varphi + i \pi ] \eval{e^{ikx}}_{k^{0}= \omega_{k}} \\
    a^{\dag}(k) = \frac{1}{(2 \pi )^{3/2}} \int \frac{ \dd[3]{x}}{\sqrt{2 \omega_{k}}} \qty[\omega_{k} \varphi - i \pi ] \eval{e^{-ikx}}_{k^{0}= \omega_{k}} \\
    \,.
\end{align}
\end{claim}

\begin{proof}
Let us compute the interesting one: 
%
\begin{align}
\qty[a(k), a ^\dag (p)] &= \frac{1}{(2\pi )^3} \int \frac{ \dd[3]{x} \dd[3]{y}}{2 \omega_{k}}
\qty[\omega_{k} \varphi + i \pi, \omega_{k} \varphi - i \pi ] e^{i(k-p)x}  \\
&= \frac{1}{(2\pi )^3} \int \frac{ \dd[3]{x} \dd[3]{y}}{2 \omega_{k}}
2 i (-) \omega_{k} \qty[ \varphi, \pi ] e^{i(k-p)x}  \\
&= \frac{1}{(2\pi )^3} \int \frac{ \dd[3]{x} \dd[3]{y}}{2 \omega_{k}}
2 i \omega_{k} (-) i \delta^{(3)} (x - y) e^{i(kx-py)}  \\
&= \frac{1}{(2\pi )^{3/2}} \int \dd[3]{x} e^{i(k-p)x}  \\
&= \delta^{(3)} (k-p) \label{eq:commutator-annihilation-creation-operators}
\,.
\end{align}

We have used the fact that, by linearity and antisymmetry of the commutator:
%
\begin{align}
\qty[\omega_{k} \varphi + i \pi, \omega_{k} \varphi - i \pi ] 
&= \qty[\omega_{k} \varphi , \omega_{k} \varphi ]+ \qty[\omega_{k} \varphi , -i \pi ] + \qty[i \pi, \omega_{k} \varphi ] + \qty[i \pi , -i \pi ]  \\
&= -2 \omega_{k} i \qty[\varphi , \pi ]
\,.
\end{align}

For the other two the computation is similar, and the difference comes about in the commutator step: for \(\qty[a, a]\) we get
%
\begin{align}
\qty[\omega_{k} \varphi + i \pi, \omega_{k} \varphi + i \pi ] = 0
\,,
\end{align}
%
and for \(\qty[a ^\dag, a ^\dag]\): 
%
\begin{align}
\qty[\omega_{k} \varphi - i \pi, \omega_{k} \varphi - i \pi ] = 0
\,,
\end{align}
%
since any operator commutes with itself.
\end{proof}

This algebra is the same one we would have for an infinite number of decoupled harmonic oscillators.  
The operators \(a\) and \(a ^\dag\) are called the annihilation and creation operators. 

\subsection{Number density operator}

Since, as we saw, the algebra of our operators resembles that of a harmonic oscillator, we are justified in defining the number density operator
%
\begin{align}
N(k) = a ^\dag  (k) a(k)
\,,
\end{align}
%
and the total number density, 
%
\begin{align}
N = \int \dd[3]{k} N(k)
\,.
\end{align}

\begin{claim}
Both \(N\) and \(N(k)\) are self-adjoint: \(N = N ^\dag\), and they satisfy the following commutation relations: 
%
\begin{align}
\qty[N(k), a(p)] &= - a(k) \delta^{(3)}(\vec{p} - \vec{k}) \\
\qty[N(k), a ^\dag(p)] &= + a ^\dag(k) \delta^{(3)}(\vec{p} - \vec{k}) \\
\qty[N, a(p)] &= - a(p)  \\
\qty[N, a ^\dag(p)] &= + a ^\dag(p)  
\,.
\end{align}
\end{claim}

\todo[inline]{There is a typo in the notes by the professor: the argument of \(a\) and \(a ^\dag\) for the local commutators is \(k\), not \(p\).}

\begin{proof}
The computation goes as follows; I use subscripts as \(a(k) = a_k\) because it makes reading the expressions easier, I think:
%
\begin{align}
\qty[N_k, a_p] &= N_k a_p - a_p N_k  \\
&= a_k a_k ^\dag a_p - a_p a_k a_k ^\dag  \\
&= - a_k a_p a_k ^\dag + a_k a_k ^\dag a_p \marginnote{Commuted \(a_p\)  and \(a_k\) --- their commutator is zero.}  \\
&= - a_k \qty[a_p, a_k ^\dag]  \\
&= - a_k \delta^{(3)} (p - k) \marginnote{Used relation \eqref{eq:commutator-annihilation-creation-operators}.}
\,.
\end{align}

The other computation is similar: 
%
\begin{align}
\qty[N_k, a_p ^\dag] &= N_k a_p ^\dag - a_p ^\dag N_k  \\
&= a_k a_k ^\dag a_p ^\dag - a_p ^\dag a_k a_k ^\dag  \\
&= - a_p ^\dag a_k a_k ^\dag + a_k a_p ^\dag a_k ^\dag  \\
&= - \qty[a_p, a_k ^\dag] a_k ^\dag  \\
&= - a_k ^\dag \delta^{(3)} (p - k)    
\,.
\end{align}
%
while for the total number operator we only need to integrate over \(\dd[3]{k}\): 
%
\begin{align}
\int \dd[3]{k} \qty[N(k), a(p)] &= - \int \dd[3]{k} a(k) \delta^{(3)}(\vec{p} - \vec{k})  \\
\qty[N, a(p)] &= - a(p)
\,.
\end{align}
\end{proof}

\subsection{Hamiltonian density operator}



\end{document}
