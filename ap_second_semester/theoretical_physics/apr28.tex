\documentclass[main.tex]{subfiles}
\begin{document}

\section{Canonical quantization}

\subsection{System with finite DoF}

\marginpar{Tuesday\\ 2020-5-12, \\ compiled \\ \today}

The classical structure of such a system is completely determined by the equations of motion: 
%
\begin{align}
\dot{q} &= \qty{q, H}  \\
\dot{p} &= \qty{p, H}
\,,
\end{align}
%
and by the commutation relations: 
%
\begin{align}
\qty{q^{i}, q^{j}} &= 0  \\
\qty{p_{i}, p_{j}} &= 0  \\
\qty{q^{i}, p_{j}} &= \delta_{j}^{i}
\,.
\end{align}

Note that these Poisson brackets are to be calculated at a fixed time. 

We can \textbf{quantize} such a system by the following substitutions: 
\begin{enumerate}
    \item the coordinates \(q\) and \(p\), which are functions of phase space in the classical case, become operators: \(X\) and \(P\), which in the Heisenberg picture are functions of time;
    \item the Poisson brackets \(\qty{\cdot, \cdot}\) become commutators: 
    %
    \begin{align}
    \qty{a, b} \to \frac{1}{i \hbar} \qty[A, B]
    \,.
    \end{align}    
\end{enumerate}

The Hamilton equations then read 
%
\begin{align}
\dv{X}{t} &= \frac{1}{i \hbar} \qty[X, H] \\
\dv{P}{t} &= \frac{1}{i \hbar} \qty[P, H]
\,,
\end{align}
%
and the commutation relations are also generalized; the interesting one is the position-momentum commutator which now reads 
%
\begin{align}
\frac{1}{i \hbar} \qty[X,P] = 1 \implies
\qty[X, P] = i \hbar
\,.
\end{align}

\begin{claim}
The Heisenberg and Schrödinger descriptions of QM are equivalent.
\end{claim}

\begin{proof}
To say that the two approaches are equivalent means that they yield the same exact predictions for any observation.

In the Schrödinger approach, the wavefunction evolves as \(\ket{\psi (t)} = U(t) \ket{\psi_0 }\) while the observables are stationary; in the Heisenberg approach the wavefunction is stationary while the operators evolve as \(A(t) = U ^\dag A U\). So, the expectation value is 
%
\begin{align}
\expval{A(t)}_{\psi_0 } &\overset{?}{=} \expval{A}_{\psi (t)}  \\
\bra{\psi_0 } A(t) \ket{\psi_0 } &\overset{?}{=} \bra{\psi (t)} A \ket{\psi (t)} \\
\bra{\psi_0 } U ^\dag A U\ket{\psi_0 } &= \bra{\psi_0 } U ^\dag A U \ket{\psi_0 } 
\,.
\end{align}
\end{proof}

\subsection{Field theory}

The evolution of the fields is given in classical field theory as 
%
\begin{align}
\dot{\varphi} &= \qty{\varphi , t}
\dot{\pi} &= \qty{\pi , t}
\,,
\end{align}
%
and we have the commutation relations 
%
\begin{align}
\qty{\varphi (x), \varphi (y)} &=0  \\
\qty{ \pi (x), \pi (y)} &= 0  \\
\qty{\varphi (x), \pi (y)} &= \delta^{(3)}(x-y)
\,,
\end{align}
%
all considered at constant time. 

In order to \textbf{quantize} this system, we replace the fields \(\varphi \) and \(\pi \) by field operators \(\hat{\varphi}\) and \(\hat{\pi}\), and replace the Poisson brackets by commutators as in the finite-DoF case.

The equations of motion will then read 
%
\begin{align}
\dv{\hat{\varphi}}{t} &= \frac{1}{i \hbar} \qty[\hat{\varphi}, H] \\
\dv{\hat{\pi}}{t} &= \frac{1}{i \hbar} \qty[\hat{\pi}, H] 
\,,
\end{align}
%
and the commutation relations will read 
%
\begin{align}
\qty[\hat{\varphi}(x), \hat{\pi }(y)] = i \hbar \delta^{(3)}(x-y)
\,,
\end{align}
%
and the zero ones as usual between \(\varphi, \varphi \) and \(\pi , \pi \).
These commutators are always taken at constant time. 

\subsection{Canonical quantization of a scalar field}

As we have seen before, the field and momentum of the free scalar field, which satisfies the Klein-Gordon equation, can be expressed as 
%
\begin{align}
\varphi (x) &= \frac{1}{(2\pi )^{3/2}} \int \frac{ \dd[3]{k}}{\sqrt{2 \omega_{k}}} \qty[a(k) e^{-ikx} + a^{\dag}(k) e^{ikx}]_{k^{0} =\omega_{k}} \\
\pi (x) &= \frac{1}{(2\pi )^{3/2}} \int \frac{ \dd[3]{k}}{\sqrt{2 \omega_{k}}} (-i \omega_{k}) \qty[a(k) e^{-ikx} - a^{\dag}(k) e^{ikx}]_{k^{0} =\omega_{k}} 
\,,
\end{align}
%
and we will now interpret this as an operator expression: even if we omit the hats, \(\varphi \) and \(\pi \) are intended to be operators. 
Therefore, \(a\) and \(a^{\dag}\) must also be. 

\begin{claim}
We can invert this relation to get \(a\) and \(a^{\dag}\) in terms of \(\varphi \) and \(\pi \): 
%
\begin{align}
    a(k) = \frac{1}{(2 \pi )^{3/2}} \int \frac{ \dd[3]{x}}{\sqrt{2 \omega_{k}}} \qty[\omega_{k} \varphi + i \pi ] \eval{e^{ikx}}_{k^{0}= \omega_{k}} \\
    a^{\dag}(k) = \frac{1}{(2 \pi )^{3/2}} \int \frac{ \dd[3]{x}}{\sqrt{2 \omega_{k}}} \qty[\omega_{k} \varphi - i \pi ] \eval{e^{-ikx}}_{k^{0}= \omega_{k}} \\
    \,.
\end{align}
\end{claim}

\begin{proof}
Let us compute the interesting one: 
%
\begin{align}
\qty[a(k), a ^\dag (p)] &= \frac{1}{(2\pi )^3} \int \frac{ \dd[3]{x} \dd[3]{y}}{2 \omega_{k}}
\qty[\omega_{k} \varphi + i \pi, \omega_{k} \varphi - i \pi ] e^{i(k-p)x}  \\
&= \frac{1}{(2\pi )^3} \int \frac{ \dd[3]{x} \dd[3]{y}}{2 \omega_{k}}
2 i (-) \omega_{k} \qty[ \varphi, \pi ] e^{i(k-p)x}  \\
&= \frac{1}{(2\pi )^3} \int \frac{ \dd[3]{x} \dd[3]{y}}{2 \omega_{k}}
2 i \omega_{k} (-) i \delta^{(3)} (x - y) e^{i(kx-py)}  \\
&= \frac{1}{(2\pi )^{3/2}} \int \dd[3]{x} e^{i(k-p)x}  \\
&= \delta^{(3)} (k-p) \label{eq:commutator-annihilation-creation-operators}
\,.
\end{align}

We have used the fact that, by linearity and antisymmetry of the commutator:
%
\begin{align}
\qty[\omega_{k} \varphi + i \pi, \omega_{k} \varphi - i \pi ] 
&= \qty[\omega_{k} \varphi , \omega_{k} \varphi ]+ \qty[\omega_{k} \varphi , -i \pi ] + \qty[i \pi, \omega_{k} \varphi ] + \qty[i \pi , -i \pi ]  \\
&= -2 \omega_{k} i \qty[\varphi , \pi ]
\,.
\end{align}

For the other two the computation is similar, and the difference comes about in the commutator step: for \(\qty[a, a]\) we get
%
\begin{align}
\qty[\omega_{k} \varphi + i \pi, \omega_{k} \varphi + i \pi ] = 0
\,,
\end{align}
%
and for \(\qty[a ^\dag, a ^\dag]\): 
%
\begin{align}
\qty[\omega_{k} \varphi - i \pi, \omega_{k} \varphi - i \pi ] = 0
\,,
\end{align}
%
since any operator commutes with itself.
\end{proof}

This algebra is the same one we would have for an infinite number of decoupled harmonic oscillators.  
The operators \(a\) and \(a ^\dag\) are called the annihilation and creation operators. 

\subsection{Number density operator}

Since, as we saw, the algebra of our operators resembles that of a harmonic oscillator, we are justified in defining the number density operator
%
\begin{align}
N(k) = a ^\dag  (k) a(k)
\,,
\end{align}
%
and the total number density, 
%
\begin{align}
N = \int \dd[3]{k} N(k)
\,.
\end{align}

\begin{claim}
Both \(N\) and \(N(k)\) are self-adjoint: \(N = N ^\dag\), and they satisfy the following commutation relations: 
%
\begin{align} \label{eq:commutation-relations-number}
\qty[N(k), a(p)] &= - a(k) \delta^{(3)}(\vec{p} - \vec{k}) \\
\qty[N(k), a ^\dag(p)] &= + a ^\dag(k) \delta^{(3)}(\vec{p} - \vec{k}) \\
\qty[N, a(p)] &= - a(p)  \\
\qty[N, a ^\dag(p)] &= + a ^\dag(p)  
\,.
\end{align}
\end{claim}

\todo[inline]{There is a typo in the notes by the professor: the argument of \(a\) and \(a ^\dag\) for the local commutators is \(k\), not \(p\).}

\begin{proof}
The computation goes as follows; I use subscripts as \(a(k) = a_k\) because it makes reading the expressions easier, I think:
%
\begin{align}
\qty[N_k, a_p] &= N_k a_p - a_p N_k  \\
&= a_k a_k ^\dag a_p - a_p a_k a_k ^\dag  \\
&= - a_k a_p a_k ^\dag + a_k a_k ^\dag a_p \marginnote{Commuted \(a_p\)  and \(a_k\) --- their commutator is zero.}  \\
&= - a_k \qty[a_p, a_k ^\dag]  \\
&= - a_k \delta^{(3)} (p - k) \marginnote{Used relation \eqref{eq:commutator-annihilation-creation-operators}.}
\,.
\end{align}

The other computation is similar: 
%
\begin{align}
\qty[N_k, a_p ^\dag] &= N_k a_p ^\dag - a_p ^\dag N_k  \\
&= a_k a_k ^\dag a_p ^\dag - a_p ^\dag a_k a_k ^\dag  \\
&= - a_p ^\dag a_k a_k ^\dag + a_k a_p ^\dag a_k ^\dag  \\
&= - \qty[a_p, a_k ^\dag] a_k ^\dag  \\
&= - a_k ^\dag \delta^{(3)} (p - k)    
\,.
\end{align}
%
while for the total number operator we only need to integrate over \(\dd[3]{k}\): 
%
\begin{align}
\int \dd[3]{k} \qty[N(k), a(p)] &= - \int \dd[3]{k} a(k) \delta^{(3)}(\vec{p} - \vec{k})  \\
\qty[N, a(p)] &= - a(p)
\,.
\end{align}
\end{proof}

\subsection{Hamiltonian and momentum density operators}

The Hamiltonian density for a real scalar field is given by 
%
\begin{align}
\mathscr{H} = \frac{1}{2} \pi^2 + \frac{1}{2} \qty(\nabla \varphi )^2 + \frac{1}{2} m^2\varphi^2
\,,
\end{align}
%
and the total Hamiltonian is given by its integral over 3D space, which is given by 
%
\begin{align}
H = \int \frac{ \dd[3]{k}}{2} \omega_{k} \qty[ a ^\dag (k) a(k) + a(k) a ^\dag (k)]
\,.
\end{align}

This can be derived without commuting \(a\) and \(a ^\dag\) anywhere; we can now get the desired expression in terms of the number operator by inserting a commutator: 
%
\begin{align}
H &= \int \frac{ \dd[3]{k} \omega_{k} }{2} \qty[2 a ^\dag a + \qty[a, a ^\dag]]   \\
&= \int \dd[3]{k} \omega_{k} N(k) + \int \frac{ \dd[3]{k} \omega_{k}}{2} \delta^{(3)} (k - k)
\,,
\end{align}
%
so the second term is proportional to \(\omega_{0} \delta^{(3)} (0)\), which diverges. 
The first term, on the other hand, makes perfect sense: we count how many particles are at each energy, and add their energies to get the total. 

Is the divergence a problem? Not really: Hamilton's equations only depend on the derivatives of \(H\), so a constant is not an issue. 
Intuitively, we can imagine a regular quantum harmonic energy at each (continuous!) value of \(k\), so for each of those we will have a ground energy \( \omega \hbar /2\). 

For the momentum we must integrate the \(T^{0i}\) components of the stress-energy tensor over \(\dd[3]{x}\): so, we find
%
\begin{align}
P^{i} &= \int \dd[3]{x} \pi (x) \partial_{i} \varphi  
\,,
\end{align}
%
and for \(\pi \) and \(\varphi \) we must use the operator expressions in momentum space: so, we find 
%
\begin{align}
\begin{split}
P^{i}&= \frac{1}{(2 \pi)^3} \int \dd[3]{x} 
\qty[
\int \frac{ \dd[3]{k}}{\sqrt{2 \omega_{k}}} 
(-i \omega_{k}) \qty(a(k) e^{-ikx} - a ^\dag (k) e^{ikx})
] \times  \\
&\phantom{=}\ 
\times \qty[
\int \frac{ \dd[3]{k'}}{\sqrt{2 \omega_{k'}}} 
\qty(-i k_i^{\prime } a(k') e^{-ik'x} +i k_i' a ^\dag (k') e^{ik'x})]
\end{split} \\
\begin{split}
&= \int \dd[3]{k'}
\frac{1}{(2 \pi )^{3/2}} \int \dd[3]{x} e^{ik'x}
\frac{1}{(2 \pi )^{3/2}} \int \dd[3]{k} e^{-ikx} \times \\
&\phantom{=}\ \times
\qty[ \frac{-i \omega_{k}}{\sqrt{2 \omega_{k}}} \qty(a(k) - a ^\dag(-k))
\frac{1}{\sqrt{2 \omega_{k'}}} \qty(+i k_i' a(-k') + i k_i' a ^\dag(k'))
]
\end{split} \\
&= \int \frac{\dd[3]{k}}{2} \qty[-i \qty(a(k) - a ^\dag(-k)) \qty(i k_i a(-k) + i k_i a ^\dag (k))] \\ 
&= \int \frac{\dd[3]{k}}{2} k_i\qty{ \qty(a(k) - a ^\dag(-k) ) \qty(a(-k) + a ^\dag (k))}  \\
&= \int \frac{\dd[3]{k}}{2} k_i \qty{
\underbrace{a(k) a (-k)}_{\text{odd}} + a (k) a ^\dag (k) - \underbrace{a ^\dag (-k) a (-k)}_{\text{change } k \to -k} - \underbrace{a ^\dag(-k) a ^\dag(k)}_{\text{odd}}
}
\\
&= \int \frac{\dd[3]{k}}{2} k^{i} \qty{ a (k) a ^\dag (k) + a ^\dag (k) a (k)}
\,.
\end{align}

Now, as before we have the integral of \(a ^\dag a /2 + a a ^\dag /2\), and we can swap them getting a term proportional to \(\delta^{(3)} (0)\). This physically means that the vacuum has an infinite momentum, which we can ignore. 

\todo[inline]{First thing: in the professor's notes the integral is reported as being over \(\dd[3]{x}\), but it is not!}

\todo[inline]{Second thing: now the term being integrated is \(k_i \delta^{(3)}(k-k)\), which diverges, sure, but it is also odd\dots the momentum of the vacuum is ``infinite'' but also zero by symmetry? i guess it doesn't really matter either way.}

If we perform this operation, we get 
%
\begin{align}
P_{i} = \int \dd[3]{k} k_i N(k)
\,.
\end{align}

\subsection{Normal ordering}

This sort of procedure is very common in QFT, so it has been given a name: the \textbf{normal ordering} of operators is the process of reordering them such that the creation operators \(a ^\dag\) are on the \emph{left} while the annihilation operators \(a\) are on the \emph{right}. This corresponds physically to the choice of the positive energy.

The notation we will use for the normal-ordering of a product of operators \(Q\) is \(N[Q]\). One can also find the notation \(: Q :\) used to mean the same thing. 
As an example, consider: 
%
\begin{align}
N[ a ^\dag(m) a(k_2 ) a ^\dag (k_3 ) a (k_4)]
= a ^\dag (m ) a ^\dag (k_3 ) a (k_2  ) a (k_4 )
\,.
\end{align}

This formalizes what we were doing before: 
%
\begin{align}
N \qty[ \frac{a_k ^\dag a_k + a_k a_k ^\dag}{2}] = a_k ^\dag a_k
\,.
\end{align}

Since the order of the product is fixed inside the normal ordering, if we work inside the normal ordering all the creation and annihilation operators commute with each other, for our \textbf{scalar bosonic theory}. 

Taking the normal ordering means we are fixing the energy of the vacuum.

Taking the normal ordering does not alter the harmonic oscillator algebra. 

\subsection{Fock space}

Up until now wee have been writing operators without being really clear about what space they are acting on. 
The space we need is called the \textbf{Fock space}: it is a space containing multiparticle states, and we will now construct it. 

We start from a vacuum state \(\ket{0}\), which must satisfy 
%
\begin{align}
\forall k : 
a(k) \ket{0} = 0
\,,
\end{align}
%
which implies \(N(k) \ket{0} = 0\). This means that there are no particles. 
\todo[inline]{Do we assume uniqueness? What do we do if there are several vacua?}

The Fock space is made up of all the states we can get by repeatedly applying the creation operators \(a ^\dag (k)\) for different values of \(k\).
Physically, each of these operators adds a particle to the state. 

If we have a set of \(N\) momenta \(k_\ell\), and we want the state describing the presence of \(n_\ell\) particles for each, we use the state 
%
\begin{align}
\ket{n_1 \dots n_N} \propto \qty(a ^\dag (k_1 ))^{n_1} \dots
\qty( a ^\dag (k_N))^{n_N} \ket{0}
\,.
\end{align}

The labels \(n_{\ell}\) are the \emph{occupation numbers}, indicating the number of particles in each momentum ``slot''. 
The proportionality sign is there because we have not yet decided on the normalization we want for our states. 

Let us then consider the properties of the states of this Fock space. For starters, the vacuum \(\ket{0}\) obeys 
%
\begin{align}
N \ket{0} = H \ket{0} = \vec{p} \ket{0} = 0
\,,
\end{align}
%
which is consistent with there being no particles, since we have normalized the vacuum energy to zero. 

Now, our one-particle state will look like 
%
\begin{align}
\ket{1(p)} = C a ^\dag_p \ket{0}
\,,
\end{align}
%
for some constant \(C\). 

\begin{claim}
This satisfies: 
%
\begin{align}
N \ket{1(p)} &= 1\ket{1(p)} \\
H \ket{1(p)} &= \omega_{p} \ket{1(p)} \\
\vec{p} \ket{1(p)} &= \vec{p} \ket{1(p)}
\,.
\end{align}
\end{claim}

\begin{proof}
We will need the properties outlined in the equations \eqref{eq:commutation-relations-number}. The computation is as follows: 
%
\begin{align}
N a ^\dag (p) \ket{0} &= \qty( N a ^\dag (p) - a ^\dag (p) N ) \ket{0}  \marginnote{We used the fact that \(N \ket{0} = 0\).}\\
&= \qty[N, a ^\dag (p)] \ket{0}  \\
&= a ^\dag(p) \ket{0} = \ket{1(p)}
\,.
\end{align}

For the energy the computation is similar: 
%
\begin{align}
H a ^\dag (p) \ket{0} &= \int \dd[3]{k} \omega_{k} N(k) a ^\dag (p) \ket{0}  \\
&= \int \dd[3]{k} \omega_{k} \qty[ N(k), a ^\dag (p)] \ket{0}  \\
&= \int \dd[3]{k} \omega_{k} a ^\dag(k) \delta^{(3)} (p-k) \ket{0}  \\
&= \omega_{p} a ^\dag (p) \ket{0}
\,.
\end{align}

For the momentum the steps are almost identical. 
\end{proof}

One can construct an \(n\)-particle state (where all the particles have the same momentum) similarly, by applying \(a ^\dag\) \(n\) times: then the eigenvalues of \(N\), \(H\) and \(\vec{p}\) will be multiplied by \(n\). 

Also, we can have states with many particles with different momenta: the energy, momentum and number will be the sum of the individual ones. 

\end{document}
