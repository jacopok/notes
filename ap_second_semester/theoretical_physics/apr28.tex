\documentclass[main.tex]{subfiles}
\begin{document}

\section{Canonical quantization}

\subsection{System with finite DoF}

\marginpar{Tuesday\\ 2020-5-12, \\ compiled \\ \today}

The classical structure of such a system is completely determined by the equations of motion: 
%
\begin{subequations}
\begin{align}
\dot{q} &= \qty{q, H}  \\
\dot{p} &= \qty{p, H}
\,,
\end{align}
\end{subequations}
%
and by the commutation relations: 
%
\begin{subequations}
\begin{align}
\qty{q^{i}, q^{j}} &= 0  \\
\qty{p_{i}, p_{j}} &= 0  \\
\qty{q^{i}, p_{j}} &= \delta_{j}^{i}
\,.
\end{align}
\end{subequations}

Note that these Poisson brackets are to be calculated at a fixed time. 

We can \textbf{quantize} such a system by the following substitutions: 
\begin{enumerate}
    \item the coordinates \(q\) and \(p\), which are functions of phase space in the classical case, become operators: \(X\) and \(P\), which in the Heisenberg picture are functions of time;
    \item the Poisson brackets \(\qty{\cdot, \cdot}\) become commutators: 
    %
    \begin{align}
    \qty{a, b} \to \frac{1}{i \hbar} \qty[A, B]
    \,.
    \end{align}    
\end{enumerate}

The Hamilton equations then read 
%
\begin{subequations}
\begin{align}
\dv{X}{t} &= \frac{1}{i \hbar} \qty[X, H] \\
\dv{P}{t} &= \frac{1}{i \hbar} \qty[P, H]
\,,
\end{align}
\end{subequations}
%
and the commutation relations are also generalized; the interesting one is the position-momentum commutator which now reads 
%
\begin{align}
\frac{1}{i \hbar} \qty[X,P] = 1 \implies
\qty[X, P] = i \hbar
\,.
\end{align}

\begin{claim}
The Heisenberg and Schrödinger descriptions of QM are equivalent.
\end{claim}

\begin{proof}
To say that the two approaches are equivalent means that they yield the same exact predictions for any observation.

In the Schrödinger approach, the wavefunction evolves as \(\ket{\psi (t)} = U(t) \ket{\psi_0 }\) while the observables are stationary; in the Heisenberg approach the wavefunction is stationary while the operators evolve as \(A(t) = U ^\dag A U\). So, the expectation value is 
%
\begin{subequations}
\begin{align}
\expval{A(t)}_{\psi_0 } &\overset{?}{=} \expval{A}_{\psi (t)}  \\
\bra{\psi_0 } A(t) \ket{\psi_0 } &\overset{?}{=} \bra{\psi (t)} A \ket{\psi (t)} \\
\bra{\psi_0 } U ^\dag A U\ket{\psi_0 } &= \bra{\psi_0 } U ^\dag A U \ket{\psi_0 } 
\,.
\end{align}
\end{subequations}
\end{proof}

\subsection{Field theory}

The evolution of the fields is given in classical field theory as 
%
\begin{align}
\dot{\varphi} &= \qty{\varphi , t} &
\dot{\pi} &= \qty{\pi , t}
\,,
\end{align}
%
and we have the commutation relations 
%
\begin{subequations}
\begin{align}
\qty{\varphi (x), \varphi (y)} &=0  \\
\qty{ \pi (x), \pi (y)} &= 0  \\
\qty{\varphi (x), \pi (y)} &= \delta^{(3)}(x-y)
\,,
\end{align}
\end{subequations}
%
all considered at constant time. 

In order to \textbf{quantize} this system, we replace the fields \(\varphi \) and \(\pi \) by field operators \(\hat{\varphi}\) and \(\hat{\pi}\), and replace the Poisson brackets by commutators as in the finite-DoF case.

The equations of motion will then read 
%
\begin{subequations}
\begin{align}
\dv{\hat{\varphi}}{t} &= \frac{1}{i \hbar} \qty[\hat{\varphi}, H] \\
\dv{\hat{\pi}}{t} &= \frac{1}{i \hbar} \qty[\hat{\pi}, H] 
\,,
\end{align}
\end{subequations}
%
and the commutation relations will read 
%
\begin{align}
\qty[\hat{\varphi}(x), \hat{\pi }(y)] = i \hbar \delta^{(3)}(x-y)
\,,
\end{align}
%
and the zero ones as usual between \(\varphi, \varphi \) and \(\pi , \pi \).
These commutators are always taken at constant time. 

\subsection{Canonical quantization of a scalar field}

As we have seen before, the field and momentum of the free scalar field, which satisfies the Klein-Gordon equation, can be expressed as 
%
\begin{subequations}
\begin{align}
\varphi (x) &= \frac{1}{(2\pi )^{3/2}} \int \frac{ \dd[3]{k}}{\sqrt{2 \omega_{k}}} \qty[a(k) e^{-ikx} + a^{\dag}(k) e^{ikx}]_{k^{0} =\omega_{k}} \\
\pi (x) &= \frac{1}{(2\pi )^{3/2}} \int \frac{ \dd[3]{k}}{\sqrt{2 \omega_{k}}} (-i \omega_{k}) \qty[a(k) e^{-ikx} - a^{\dag}(k) e^{ikx}]_{k^{0} =\omega_{k}} 
\,,
\end{align}
\end{subequations}
%
and we will now interpret this as an operator expression: even if we omit the hats, \(\varphi \) and \(\pi \) are intended to be operators. 
Therefore, \(a\) and \(a^{\dag}\) must also be. 

\begin{claim}
We can invert this relation to get \(a\) and \(a^{\dag}\) in terms of \(\varphi \) and \(\pi \): 
%
\begin{subequations}
\begin{align}
    a(k) = \frac{1}{(2 \pi )^{3/2}} \int \frac{ \dd[3]{x}}{\sqrt{2 \omega_{k}}} \qty[\omega_{k} \varphi + i \pi ] \eval{e^{ikx}}_{k^{0}= \omega_{k}} \\
    a^{\dag}(k) = \frac{1}{(2 \pi )^{3/2}} \int \frac{ \dd[3]{x}}{\sqrt{2 \omega_{k}}} \qty[\omega_{k} \varphi - i \pi ] \eval{e^{-ikx}}_{k^{0}= \omega_{k}} \\
    \,.
\end{align}
\end{subequations}
\end{claim}

\begin{proof}
Let us compute the interesting one: 
%
\begin{subequations}
\begin{align}
\qty[a(k), a ^\dag (p)] &= \frac{1}{(2\pi )^3} \int \frac{ \dd[3]{x} \dd[3]{y}}{2 \omega_{k}}
\qty[\omega_{k} \varphi + i \pi, \omega_{k} \varphi - i \pi ] e^{i(k-p)x}  \\
&= \frac{1}{(2\pi )^3} \int \frac{ \dd[3]{x} \dd[3]{y}}{2 \omega_{k}}
2 i (-) \omega_{k} \qty[ \varphi, \pi ] e^{i(k-p)x}  \\
&= \frac{1}{(2\pi )^3} \int \frac{ \dd[3]{x} \dd[3]{y}}{2 \omega_{k}}
2 i \omega_{k} (-) i \delta^{(3)} (x - y) e^{i(kx-py)}  \\
&= \frac{1}{(2\pi )^{3/2}} \int \dd[3]{x} e^{i(k-p)x}  \\
&= \delta^{(3)} (k-p) \label{eq:commutator-annihilation-creation-operators}
\,.
\end{align}
\end{subequations}

We have used the fact that, by linearity and antisymmetry of the commutator:
%
\begin{subequations}
\begin{align}
\qty[\omega_{k} \varphi + i \pi, \omega_{k} \varphi - i \pi ] 
&= \qty[\omega_{k} \varphi , \omega_{k} \varphi ]+ \qty[\omega_{k} \varphi , -i \pi ] + \qty[i \pi, \omega_{k} \varphi ] + \qty[i \pi , -i \pi ]  \\
&= -2 \omega_{k} i \qty[\varphi , \pi ]
\,.
\end{align}
\end{subequations}

For the other two the computation is similar, and the difference comes about in the commutator step: for \(\qty[a, a]\) we get
%
\begin{align}
\qty[\omega_{k} \varphi + i \pi, \omega_{k} \varphi + i \pi ] = 0
\,,
\end{align}
%
and for \(\qty[a ^\dag, a ^\dag]\): 
%
\begin{align}
\qty[\omega_{k} \varphi - i \pi, \omega_{k} \varphi - i \pi ] = 0
\,,
\end{align}
%
since any operator commutes with itself.
\end{proof}

This algebra is the same one we would have for an infinite number of decoupled harmonic oscillators.  
The operators \(a\) and \(a ^\dag\) are called the annihilation and creation operators. 

\subsection{Number density operator}

Since, as we saw, the algebra of our operators resembles that of a harmonic oscillator, we are justified in defining the number density operator
%
\begin{align}
N(k) = a ^\dag  (k) a(k)
\,,
\end{align}
%
and the total number density, 
%
\begin{align}
N = \int \dd[3]{k} N(k)
\,.
\end{align}

\begin{claim}
Both \(N\) and \(N(k)\) are self-adjoint: \(N = N ^\dag\), and they satisfy the following commutation relations: 
%
\begin{subequations}
\begin{align} \label{eq:commutation-relations-number}
\qty[N(k), a(p)] &= - a(k) \delta^{(3)}(\vec{p} - \vec{k}) \\
\qty[N(k), a ^\dag(p)] &= + a ^\dag(k) \delta^{(3)}(\vec{p} - \vec{k}) \\
\qty[N, a(p)] &= - a(p)  \\
\qty[N, a ^\dag(p)] &= + a ^\dag(p)  
\,.
\end{align}
\end{subequations}
\end{claim}

\todo[inline]{There is a typo in the notes by the professor: the argument of \(a\) and \(a ^\dag\) for the local commutators is \(k\), not \(p\).}

\begin{proof}
The computation goes as follows; I use subscripts as \(a(k) = a_k\) because it makes reading the expressions easier, I think:
%
\begin{subequations}
\begin{align}
\qty[N_k, a_p] &= N_k a_p - a_p N_k  \\
&= a_k a_k ^\dag a_p - a_p a_k a_k ^\dag  \\
&= - a_k a_p a_k ^\dag + a_k a_k ^\dag a_p \marginnote{Commuted \(a_p\)  and \(a_k\) --- their commutator is zero.}  \\
&= - a_k \qty[a_p, a_k ^\dag]  \\
&= - a_k \delta^{(3)} (p - k) \marginnote{Used relation \eqref{eq:commutator-annihilation-creation-operators}.}
\,.
\end{align}
\end{subequations}

The other computation is similar: 
%
\begin{subequations}
\begin{align}
\qty[N_k, a_p ^\dag] &= N_k a_p ^\dag - a_p ^\dag N_k  \\
&= a_k a_k ^\dag a_p ^\dag - a_p ^\dag a_k a_k ^\dag  \\
&= - a_p ^\dag a_k a_k ^\dag + a_k a_p ^\dag a_k ^\dag  \\
&= - \qty[a_p, a_k ^\dag] a_k ^\dag  \\
&= - a_k ^\dag \delta^{(3)} (p - k)    
\,.
\end{align}
\end{subequations}
%
while for the total number operator we only need to integrate over \(\dd[3]{k}\): 
%
\begin{subequations}
\begin{align}
\int \dd[3]{k} \qty[N(k), a(p)] &= - \int \dd[3]{k} a(k) \delta^{(3)}(\vec{p} - \vec{k})  \\
\qty[N, a(p)] &= - a(p)
\,.
\end{align}
\end{subequations}
\end{proof}

\subsection{Hamiltonian and momentum density operators}

The Hamiltonian density for a real scalar field is given by 
%
\begin{align}
\mathscr{H} = \frac{1}{2} \pi^2 + \frac{1}{2} \qty(\nabla \varphi )^2 + \frac{1}{2} m^2\varphi^2
\,,
\end{align}
%
and the total Hamiltonian is given by its integral over 3D space, which is given by 
%
\begin{align}
H = \int \frac{ \dd[3]{k}}{2} \omega_{k} \qty[ a ^\dag (k) a(k) + a(k) a ^\dag (k)]
\,.
\end{align}

This can be derived without commuting \(a\) and \(a ^\dag\) anywhere; we can now get the desired expression in terms of the number operator by inserting a commutator: 
%
\begin{subequations}
\begin{align}
H &= \int \frac{ \dd[3]{k} \omega_{k} }{2} \qty[2 a ^\dag a + \qty[a, a ^\dag]]   \\
&= \int \dd[3]{k} \omega_{k} N(k) + \int \frac{ \dd[3]{k} \omega_{k}}{2} \delta^{(3)} (k - k)
\,,
\end{align}
\end{subequations}
%
so the second term is proportional to \(\omega_{0} \delta^{(3)} (0)\), which diverges. 
The first term, on the other hand, makes perfect sense: we count how many particles are at each energy, and add their energies to get the total. 

Is the divergence a problem? Not really: Hamilton's equations only depend on the derivatives of \(H\), so a constant is not an issue. 
Intuitively, we can imagine a regular quantum harmonic energy at each (continuous!) value of \(k\), so for each of those we will have a ground energy \( \omega \hbar /2\). 

For the momentum we must integrate the \(T^{0i}\) components of the stress-energy tensor over \(\dd[3]{x}\): so, we find
%
\begin{align}
P^{i} &= \int \dd[3]{x} \pi (x) \partial_{i} \varphi  
\,,
\end{align}
%
and for \(\pi \) and \(\varphi \) we must use the operator expressions in momentum space: so, we find 
%
\begin{subequations}
\begin{align}
\begin{split}
P_{i}&= \frac{1}{(2 \pi)^3} \int \dd[3]{x} 
\qty[
\int \frac{ \dd[3]{k}}{\sqrt{2 \omega_{k}}} 
(-i \omega_{k}) \qty(a(k) e^{-ikx} - a ^\dag (k) e^{ikx})
] \times  \\
&\phantom{=}\ 
\times \qty[
\int \frac{ \dd[3]{k'}}{\sqrt{2 \omega_{k'}}} 
\qty(-i k_i^{\prime } a(k') e^{-ik'x} +i k_i' a ^\dag (k') e^{ik'x})]
\end{split} \\
% \begin{split}
% &= \int \dd[3]{k'}
% \frac{1}{(2 \pi )^{3/2}} \int \dd[3]{x} e^{ik'x}
% \frac{1}{(2 \pi )^{3/2}} \int \dd[3]{k} e^{-ikx} \times \\
% &\phantom{=}\ \times
% \qty[ \frac{-i \omega_{k}}{\sqrt{2 \omega_{k}}} \qty(a(k) - a ^\dag(-k))
% \frac{1}{\sqrt{2 \omega_{k'}}} \qty(+i k_i' a(-k') + i k_i' a ^\dag(k'))
% ]
% \end{split} \\
% &= \int \frac{\dd[3]{k}}{2} \qty[-i \qty(a(k) - a ^\dag(-k)) \qty(i k_i a(-k) + i k_i a ^\dag (k))] \\ 
% &= \int \frac{\dd[3]{k}}{2} k_i\qty{ \qty(a(k) - a ^\dag(-k) ) \qty(a(-k) + a ^\dag (k))}  \\
% &= \int \frac{\dd[3]{k}}{2} k_i \qty{
% \underbrace{a(k) a (-k)}_{\text{odd}} + a (k) a ^\dag (k) - \underbrace{a ^\dag (-k) a (-k)}_{\text{change } k \to -k} - \underbrace{a ^\dag(-k) a ^\dag(k)}_{\text{odd}}
% }
% \\
% &= \int \frac{\dd[3]{k}}{2} k^{i} \qty{ a (k) a ^\dag (k) + a ^\dag (k) a (k)}
\begin{split}
&= \int \frac{ \dd[3]{x}}{(2 \pi )^3} \int \frac{ \dd[3]{k}}{2} 
\int \frac{ \dd[3]{k'}}{\sqrt{\omega_{k} \omega_{k'}}} 
(- \omega_{k} k_i')  \\
&\phantom{=}\ 
\bigg[
a(k) a(k') e^{-i (\omega_{k} + \omega_{k}') x_0 + i (\vec{k} + \vec{k}')\cdot \vec{x}} 
-a(k) a ^\dag(k') e^{-i(\omega_{k} - \omega_{k}') x_0 + i (\vec{k} - \vec{k}')\cdot \vec{x}} \\
&\phantom{=}\ -a ^\dag(k) a(k') e^{-i (-\omega_{k} + \omega_{k}') x_0 + i (-\vec{k} + \vec{k}')\cdot \vec{x}} 
+ a ^\dag(k) a ^\dag(k') e^{-i(-\omega_{k} - \omega_{k}') x_0 + i (-\vec{k} - \vec{k}')\cdot \vec{x}} 
\bigg]
\end{split}  \\
\begin{split}
&= \int \frac{ \dd[3]{k}}{2} 
\int \frac{ \dd[3]{k'}}{\sqrt{\omega_{k} \omega_{k'}}} (- \omega_{k} k_i')  \\
&\phantom{=}\ 
\bigg[
a(k) a(k') e^{-i (\omega_{k} + \omega_{k}') x_0} \delta^{(3)} (\vec{k} + \vec{k}') 
-a(k) a ^\dag(k') e^{-i(\omega_{k} - \omega_{k}') x_0} \delta^{(3)}(\vec{k} - \vec{k}') \\
&\phantom{=}\ -a ^\dag(k) a(k') e^{-i (-\omega_{k} + \omega_{k}') x_0}
\delta^{(3)} (\vec{k} - \vec{k}') 
+ a ^\dag(k) a ^\dag(k') e^{-i(-\omega_{k} - \omega_{k}') x_0} \delta^{(3)} (\vec{k} + \vec{k}') 
\bigg]
\end{split}\\
&= \int \frac{ \dd[3]{k}}{2 \omega_{k}} (- \omega_{k} k_i)
\bigg[
- a(k) a(-k) e^{- 2i\omega_{k} x_0}
- a(k) a ^\dag(k) 
- a ^\dag(k) a(k') 
- a ^\dag(k) a ^\dag(-k) e^{2i\omega_{k} x_0} 
\bigg]  \\
&= \int \frac{ \dd[3]{k}}{2} k_i \qty[a(k) a ^\dag (k) + a ^\dag(k) a(k)]
+ \underbrace{k_i \qty( a(k) a(-k) e^{- 2i\omega_{k} x_0} + a ^\dag(k) a ^\dag(-k) e^{2i\omega_{k} x_0} )}_{\text{odd under \(\vec{k} \to - \vec{k}\)}}
\,.
\end{align}
\end{subequations}

Now, as before we have the integral of \(a ^\dag a /2 + a a ^\dag /2\), and we can swap them getting a term proportional to \(\delta^{(3)} (0)\). This physically means that the vacuum has an infinite momentum, which we can ignore. 

\todo[inline]{In the professor's notes the integral is reported as being over \(\dd[3]{x}\), but it is not!}

If we perform this operation, we get 
%
\begin{align}
P_{i} = \int \dd[3]{k} k_i N(k)
\,.
\end{align}

\subsection{Normal ordering}

This sort of procedure is very common in QFT, so it has been given a name: the \textbf{normal ordering} of operators is the process of reordering them such that the creation operators \(a ^\dag\) are on the \emph{left} while the annihilation operators \(a\) are on the \emph{right}. This corresponds physically to the choice of the positive energy.

The notation we will use for the normal-ordering of a product of operators \(Q\) is \(N[Q]\). One can also find the notation \(: Q :\) used to mean the same thing. 
As an example, consider: 
%
\begin{align}
N[ a ^\dag(m) a(k_2 ) a ^\dag (k_3 ) a (k_4)]
= a ^\dag (m ) a ^\dag (k_3 ) a (k_2  ) a (k_4 )
\,.
\end{align}

This formalizes what we were doing before: 
%
\begin{align}
N \qty[ \frac{a_k ^\dag a_k + a_k a_k ^\dag}{2}] = a_k ^\dag a_k
\,.
\end{align}

Since the order of the product is fixed inside the normal ordering, if we work inside the normal ordering all the creation and annihilation operators commute with each other, for our \textbf{scalar bosonic theory}. 

Taking the normal ordering means we are fixing the energy of the vacuum.

Taking the normal ordering does not alter the harmonic oscillator algebra. 

\subsection{Fock space}

Up until now wee have been writing operators without being really clear about what space they are acting on. 
The space we need is called the \textbf{Fock space}: it is a space containing multiparticle states, and we will now construct it. 

We start from a vacuum state \(\ket{0}\), which must satisfy 
%
\begin{align}
\forall k : 
a(k) \ket{0} = 0
\,,
\end{align}
%
which implies \(N(k) \ket{0} = 0\). This means that there are no particles. 
\todo[inline]{Do we assume uniqueness? What do we do if there are several vacua?}

The Fock space is made up of all the states we can get by repeatedly applying the creation operators \(a ^\dag (k)\) for different values of \(k\).
Physically, each of these operators adds a particle to the state. 

If we have a set of \(N\) momenta \(k_\ell\), and we want the state describing the presence of \(n_\ell\) particles for each, we use the state 
%
\begin{align}
\ket{n_1 \dots n_N} \propto \qty(a ^\dag (k_1 ))^{n_1} \dots
\qty( a ^\dag (k_N))^{n_N} \ket{0}
\,.
\end{align}

The labels \(n_{\ell}\) are the \emph{occupation numbers}, indicating the number of particles in each momentum ``slot''. 
The proportionality sign is there because we have not yet decided on the normalization we want for our states. 

Let us then consider the properties of the states of this Fock space. For starters, the vacuum \(\ket{0}\) obeys 
%
\begin{align}
N \ket{0} = H \ket{0} = \vec{p} \ket{0} = 0
\,,
\end{align}
%
which is consistent with there being no particles, since we have normalized the vacuum energy to zero. 

Now, our one-particle state will look like 
%
\begin{align}
\ket{1(p)} = C a ^\dag_p \ket{0}
\,,
\end{align}
%
for some constant \(C\). 

\begin{claim}
This satisfies: 
%
\begin{subequations}
\begin{align}
N \ket{1(p)} &= 1\ket{1(p)} \\
H \ket{1(p)} &= \omega_{p} \ket{1(p)} \\
\vec{p} \ket{1(p)} &= \vec{p} \ket{1(p)}
\,.
\end{align}
\end{subequations}
\end{claim}

\begin{proof}
We will need the properties outlined in the equations \eqref{eq:commutation-relations-number}. The computation is as follows: 
%
\begin{subequations}
\begin{align}
N a ^\dag (p) \ket{0} &= \qty( N a ^\dag (p) - a ^\dag (p) N ) \ket{0}  \marginnote{We used the fact that \(N \ket{0} = 0\).}\\
&= \qty[N, a ^\dag (p)] \ket{0}  \\
&= a ^\dag(p) \ket{0} = \ket{1(p)}
\,.
\end{align}
\end{subequations}

For the energy the computation is similar: 
%
\begin{subequations}
\begin{align}
H a ^\dag (p) \ket{0} &= \int \dd[3]{k} \omega_{k} N(k) a ^\dag (p) \ket{0}  \\
&= \int \dd[3]{k} \omega_{k} \qty[ N(k), a ^\dag (p)] \ket{0}  \\
&= \int \dd[3]{k} \omega_{k} a ^\dag(k) \delta^{(3)} (p-k) \ket{0}  \\
&= \omega_{p} a ^\dag (p) \ket{0}
\,.
\end{align}
\end{subequations}

For the momentum the steps are almost identical. 
\end{proof}

One can construct an \(n\)-particle state (where all the particles have the same momentum) similarly, by applying \(a ^\dag\) \(n\) times: then the eigenvalues of \(N\), \(H\) and \(\vec{p}\) will be multiplied by \(n\). 

Also, we can have states with many particles with different momenta: the energy, momentum and number will be the sum of the individual ones. 

\subsection{Spin-statistics connection}

The procedure we have employed so far implicitly assumed we were using \textbf{Bose-Einstein} statistics, which is consistent with the fact that our field is spin-0.

This can be seen by the fact that in our Fock space we are allowing more than one particle to have the same momentum; also, the state vector does not change sign when we swap the particles, since \(a ^\dag_{p}\) and \(a ^\dag_{k} \) commute.

\subsection{Normalization of Fock states}

We start by setting \(\braket{0}{0}= 1\).
What do we do for states with stuff in them? If we were to set \(\ket{1} = a ^\dag _k \ket{0}\), we would have an issue:  if we wanted to compute the norm we would have 
%
\begin{align}
\braket{1(k)}{1(p)} = \bra{0} a (k) a ^\dag (p) \ket{0} = \delta^{(3)} (k - p)
\,,
\end{align}
%
and we know that the Dirac delta is not covariant.
We can fix this issue: if we impose 
%
\begin{align}
\ket{1(k)} = (2 \pi )^{3/2} \sqrt{2 \omega_{k}} a ^\dag_{k} \ket{0}
\,,
\end{align}
%
so that when we compute the modulus we get 
%
\begin{align}
\braket{1(k)}{1(p)} = (2 \pi )^3 2 \omega_{k} \bra{0} a (k) a ^\dag (p) \ket{0} = % \delta^{(3)} (k - p) =
(2 \pi )^3 2 \omega_{k} \delta^{(3)} (k - p)
\,.
\end{align}

\begin{claim}
This is covariant. 
\end{claim}

\begin{proof}
This is explained very well in Peskin's textbook \cite[section 3.5]{peskinConceptsElementaryParticle2019}.

We want to show that \(\omega_{k}  \delta^{(3)} (\vec{k} - \vec{p})\) is a Lorentz invariant (also a Poincarè invariant, but translation invariance is manifest). In order to do so, we must perform a boost: the system is spherically symmetric, so we choose arbitrarily to perform a boost in the direction \(z\), with parameter \(\beta \). 

Then, we know that since \(p^{\mu }\) is a contravariant 4-vector it transforms like 
%
\begin{subequations}
\begin{align}
E' &= \gamma (E - \beta p_{z}) \\
p_{x}' &= p_{x} \\
p_{y}' &= p_{y} \\
p_{z}' &= \gamma (p_{z} - \beta E)
\,.
\end{align}
\end{subequations}

In the expression \(\delta^{(3)} (\vec{k} - \vec{p})\) the only thing which changes is \(p_{z}\). The triple delta function is the tensor product of three deltas, but two of them are unchanged in the transformation.
So, we must check how \(\delta (k_{z} - p_{z})\) transforms: recall that 
%
\begin{align}
\delta \qty(f(x)) = \frac{1}{\abs{\dv*{f}{x}}} \delta (x) 
\,,
\end{align}
%
as long as \(f(x)\) has a single zero at \(x=0\), which holds in our case since Lorentz transforms are linear.

The conversion factor between the \(\delta \)s will be given by the inverse of the absolute value of
%
\begin{subequations}
\begin{align}
\pdv{p_{z}'}{p_{z}} &= \pdv{}{p_{z}} \qty(\gamma (p_{z} + \beta E))  \\
&= \gamma \qty(1 + \beta \pdv{E}{p_{z}})  \\
&= \gamma \qty(1 + \beta \frac{p_{z}}{E})  \\
&= \frac{\gamma \qty(E + \beta p_z)}{E} = \frac{E'}{E}
\,,
\end{align}
\end{subequations}
%
which means that the product \(E \delta^{(3)} (\vec{k} - \vec{p})\) is covariant. 
\end{proof}

This makes sense: the covariant integration element is \(\dd[3]{p} / 2 E\), so if we want to apply the \(\delta \) linear functional to a Lorentz scalar function \(f(k)\) we can do it simply as 
%
\begin{align}
\int \frac{ \dd[3]{p}}{2E} 2 E \delta^{(3)} (p-k) f (k) 
\,,
\end{align}
%
which yields an invariant number.

\subsection{Field operators and particle interpretation}

Up until now we have been using the operators \(a\) and \(a ^\dag\) in momentum space, since their physical interpretation is more intuitive. 
Let us now consider how to interpret the field operators \(\varphi \) and \(\pi \). 

Recall, the real scalar field is given in general by 
%
\begin{align}
\varphi (x) = \frac{1}{(2 \pi )^{3/2}} 
\int \frac{ \dd[3]{k}}{\sqrt{2 \omega_{k}}}
\qty( a_k e^{-ikx} + a ^\dag_{k} e^{ikx}) = \varphi_{+} (x) + \varphi_{-} (x)
\,.
\end{align}

\begin{claim}
We have the following transition amplitudes: 
%
\begin{subequations}
\begin{align}
\bra{0} \varphi_{+} (x) \ket{1(p)} &= e^{-ipx} \\
\bra{1(p)} \varphi_{-} (x) \ket{0} &= e^{ipx}
\,.
\end{align}
\end{subequations}
\end{claim}

\begin{proof}
In order to compute this we substitute in directly:  
%
\begin{subequations}
\begin{align}
\bra{0} \varphi_{+} (x) \ket{1(p)} &=
\bra{0} \frac{1}{(2 \pi )^{3/2}} 
\int \frac{ \dd[3]{k}}{\sqrt{2 \omega_{k}}} 
e^{-ikx} a(k) \ket{1(p)}  \\
&= \bra{0} \frac{1}{(2 \pi )^{3/2}} 
\int \frac{ \dd[3]{k}}{\sqrt{2 \omega_{k}}} 
e^{-ikx} a(k) (2 \pi )^{3/2} \sqrt{2 \omega_{p}} a ^\dag (p) \ket{0}  \\
&= \braket{0}{0} \int \dd[3]{k} e^{-ikx} \delta^{(3)} (k-p) = e^{-ipx}
\,,
\end{align}
\end{subequations}
%
where we used the fact that \(a_k a ^\dag_p \ket{0} = \qty[a_k, a ^\dag_p] \ket{0}\).

For the other one the computation is similar: we work from the other direction.
\end{proof}

The interpretation for this is that the field operator \(\varphi_{+}\) can annihilate a particle with generic momentum \(p\) at the point \(x\) in space, since when we apply it to \(\ket{1(p)}\) the result is proportional to the vacuum. Similarly, \(\varphi_{-}\) can create a particle with completely undetermined momentum at the position \(x\). 

\subsection{Covariant commutators} \label{sec:covariant-commutators}

In order to perform canonical quantization we started with the commutators at equal time between the fields \(\varphi \) and \(\pi \). 
Are the expressions for these covariant? 

We start by defining 
%
\begin{subequations}
\begin{align}
D(x - y) &= \qty[\varphi (x), \varphi (y)]  \\
&= \qty[\varphi_{+} (x) + \varphi_{-} (x), \varphi_{+} (x) + \varphi_{-} (x)]  \\
&= \qty[\varphi_{+} (x), \varphi_{-} (x)] + \qty[\varphi_{-} (x), \varphi_{+} (x)]  \\
&= D_+ (x-y) + D_- (x-y)
\,.
\end{align}
\end{subequations}


The dependence of \(D\) would generally be on \(x\) and \(y\) separately, but if we assume translational invariance (and we always do) this reduces to \(x-y\).
We have used the fact that \(\varphi_{\pm}\) commute with themselves at different points: this can be computed as 
%
\begin{align}
\qty[\varphi_{-}(x) , \varphi_{-}(y)] &=
\frac{1}{(2\pi )^{3}} \int \frac{ \dd[3]{k}}{\sqrt{2 \omega_{k}}}
\int \frac{\dd[3]{p}}{\sqrt{2 \omega_{p}}} 
\qty[\hat{a}(k), \hat{a}(p)] e^{-ikx} e^{-ipy} = 0
\,,
\end{align}
%
since \(\qty[\hat{a}(k), \hat{a}(p)] = 0\). The computation for \(\varphi_{+}\) is analogous. 

Then, we can compute \(D_{+} (x-y)\) similarly as well: we find 
%
\begin{align}
D_{+} (x-y) &= \frac{1}{(2\pi )^3} \int \frac{ \dd[3]{k} }{2 \omega_{k}}
e^{-ik(x-y)} = - D_{-} (y-x)  
\,.
\end{align}

Then, we can make use of the relation \(\sin(x) = \qty(e^{ix} - e^{-ix}) / 2i\) since we have the difference of two exponentials with opposite arguments: we find 
%
\begin{align}
D(x-y) = \frac{-i}{(2 \pi )^3}
\int \frac{ \dd[3]{k}}{\omega_{k}} \sin(k (x-y))
\,,
\end{align}
%
which is covariant, but not \emph{manifestly} so: we need to manipulate it to see it.

\begin{claim}
The two \(D_{\pm } \) can be written as: 
%
\begin{align} \label{eq:scalar-field-propagator}
    D_{\pm } (x-y) = i \int_{C_{\pm}} \frac{ \dd[4]{k}}{(2 \pi )^{4}} \frac{e^{-ik(x-y)}}{k^2- m^2}
    \,,
\end{align}
%
where \(C_{\pm}\) is a small clockwise contour in the complex plane around \(k_0 = \omega_{k}\). The integral is over all of 3D space --- the coordinates \(\vec{k}\) --- and over a complex contour for the coordinate \(k_0 \). 
\end{claim}

\begin{proof}
The claim we want to prove is 
%
\begin{subequations}
\begin{align}
D_{\pm} (x-y) 
= i \int_{C_{\pm}} \frac{ \dd[4]{k}}{(2\pi )^{4}}
\frac{e^{-ik(x-y)}}{k^2 - m^2}
&\overset{?}{=}
\pm \frac{1}{(2 \pi)^3} \eval{\int \frac{ \dd[3]{k}}{2 \omega_{k}} e^{\mp ik (x-y)}}_{k_0 = \omega_{k}}  \\
\frac{i}{2 \pi }
\int_{C_{\pm}} \dd{k_0 } \frac{e^{-ik(x-y)}}{k^2-m^2}
&\overset{?}{=}
\pm \eval{\frac{e^{\mp ik(x-y)}}{2 \omega_{k}}}_{k_0 = \omega_{k}}
\,.
\end{align}
\end{subequations}

We do not write the \(\dd[3]{k}\) integral anymore and consider \(\vec{k}\) as fixed. 

The complex residue theorem comes to our aid: it states that if we have a complex function \(f(z)\) and a closed contour \(\gamma \), then 
%
\begin{align}
\int_{\gamma } f(z) \dd{z} = 2 \pi i \sum _{j} \Res_{z = a_j} f(z)
\,,
\end{align}
%
where \(a_j\) is a collection of the poles of the function which are contained in the contour \(\gamma \). 

The \emph{residual} of the function \(f(z)\) at a pole \(a\) is the \(-1\)th coefficient in its Laurent series around \(a\), which is the expression 
%
\begin{align}
f(z) = \sum _{j= - \infty }^{ \infty } c_j (z-a)^{j}
\,.
\end{align}

So, we have 
%
\begin{align}
\frac{i}{2 \pi } 
\int_{C_{\pm}} \dd{k_0} 
\frac{e^{-ik(x-y)}}{k^2-m^2} 
= - \Res_{k_0 = \pm \omega_{k}} 
\frac{e^{-ik(x-y)}}{k^2-m^2} 
\,,
\end{align}
%
and we can write \(k^2 - m^2 = k_0^2 - \abs{\vec{k}}^2 - m^2 = k_0^2 - \omega_{k}^2\); so that we need to take the residual of 
%
\begin{align}
\frac{e^{-ik(k-y)}}{(k_0 - \omega_{k}) (k_0 + \omega_{k})}
\,.
\end{align}

We are working near \(k_0 = \omega_{k}\), where the exponential is basically constant.
For a first order pole like this, the residual is simple to compute: 
%
\begin{subequations}
\begin{align}
\Res_{k_0 = \pm \omega_{k}} 
\frac{e^{-ik(x-y)}}{(k_0 - \omega_{k}) (k_0 + \omega_{k})}
&= \lim_{k_0 \to \pm \omega_{k}} 
(k_0 \mp \omega_{k})
\frac{e^{-ik(x-y)}}{(k_0 - \omega_{k}) (k_0 + \omega_{k})}  \\
&= \eval{\frac{e^{-ik(x-y)}}{(k_0 \pm \omega_{k})}}_{k_0 = \pm \omega_{k}}  \\
&= \pm \eval{\frac{e^{\mp ik (x-y)}}{2 \omega_{k}}}_{k_0 = \pm \omega_{k}} 
\,,
\end{align}
\end{subequations}
%
as we desired. Note that we have switched the sign of the \(\vec{k}\) coordinates in the exponent: this would not be possible in general, but we are working implicitly inside of the \(\dd[3]{k} \) integral, so a map \(\vec{k} \to - \vec{k}\) does not pose an issue.
\end{proof}

\subsection{Covariance and microcausality} \label{sec:covariance-microcausality}

We have shown that \(D(x-y)\) are covariant under proper Lorentz transformations. Now we know that the commutator between two 4D points \(x\) and \(y\) is given by 
%
\begin{align}
\qty[\varphi (x), \varphi (y)] = D(x-y)
\,,
\end{align}
%
but if we consider \(x\) and \(y\) to be along the same time-slice (that is, \(x_0 = y_0 \)), then we get back the equal-time commutator, which is equal to zero. 

So, we get that \(D(x-y)\) must be equal to zero if \(x\) and \(y\) are generally spacelike separated, since any spacelike interval can be mapped onto another, and \(D\) must be covariant. 

This is called a \textbf{microcausality} conditions: there cannot be influence between the creation and annihilation of particles if the two events are spacelike separated. 

We cannot directly measure the fields, but observables are functions depending on them; it can be proven from the condition 
%
\begin{align}
(x-y)^2<0 \implies
\qty[\varphi (x), \varphi (y)] = 0
\,,
\end{align}
%
that the commutator of two observables, \(\qty[A(x), B(y)] =0\) as well.

\subsection{Canonical quantization of a complex scalar field}

Similarly to the real case, we interpret the fields \(\varphi \), \(\pi = \partial_0 \varphi ^\dag\), \(\varphi^{*}\), \(\pi^{*} = \partial_{0} \varphi \), as operators acting on a Fock space of states.

The equal time commutators are given by substituting the Poisson brackets with \(-i\) times the commutator. We get: 
%
\begin{align}
\qty[\varphi (\vec{x}, t), \pi (\vec{y}, t)] =
\qty[\varphi^{*} (\vec{x}, t), \pi^{*} (\vec{y}, t)] = i \delta^{(3)} (\vec{x} - \vec{y}) 
\,,
\end{align}
%
while all the other equal-time commutators vanish.

We know that the general solution  is given in terms of the operators \(a\) and \(b\); with its expression we find that the commutation relations between these are 
%
\begin{align}
\qty[a(k), a ^\dag(p)] = \qty[b(k), b ^\dag (p)] &= \delta^{(3)} (k-p)
\,,
\end{align}
%
and all the others between these operators vanish. 
So, we have two independent infinite sets of decoupled harmonic oscillators, describing particles with the same mass and arbitrary momenta. 

\subsection{Summary}

We have the \textbf{number} density operators, \(\mathscr{N}_{a} (k) = a(k) ^\dag a(k)\) and \(\mathscr{N}_{b} (k) = b(k) ^\dag b(k)\). 
Their integrals over \(\dd[3]{k}\) give the total number operators \(N_a\) and \(N_b\). 

The \textbf{Hamiltonian} density is given by 
%
\begin{align}
\mathscr{H}(x) = \pi ^\dag \pi 
+ \qty(\vec{\nabla} \varphi ) ^\dag \qty(\vec{\nabla} \varphi )
+ m^2 \varphi ^\dag \varphi  
\,,
\end{align}
%
and its integral is 
%
\begin{align}
H = \int \dd[3]{x} \mathscr{H} = 
\int \dd[3]{k} \omega_{k} \qty(\mathscr{N}_a (k) + \mathscr{N}_b (k))
\,.
\end{align}

The normal ordering is implied, by \(H\) we really mean \(N[H]\).

The \textbf{momentum} density operator is given by 
%
\begin{align}
\mathscr{P}_{i} = \pi \qty(\partial_{i} \varphi )
+ \qty(\partial_{i} \varphi ) ^\dag \pi ^\dag
\,,
\end{align}
%
and its integral gives the total momentum: 
%
\begin{align}
P_i = \int \dd[3]{x} \mathscr{P}_{i} = \int \dd[3]{k} k_i \qty(\mathscr{N}_a (k) + \mathscr{N}_b (k))
\,.
\end{align}

This is all analogous to the real case, but now we have an additional \(U(1)\) symmetry: this gives us a conserved current, 
%
\begin{align}
J^{\mu }_{U(1)} = i q \qty[ \varphi ^\dag \qty(\partial^{\mu } \varphi )
- \qty(\partial^{\mu } \varphi )^\dag \varphi ] 
\,,
\end{align}
%
whose associated charge is 
%
\begin{align}
Q_{U(1)} = \int \dd[3]{x} J^{0}_{U(1)}
= \int \dd[3]{k} \qty(q \mathscr{N}_a (k) - q \mathscr{N}_b(k))
\,.
\end{align}

This has a direct physical interpretation: the particles which are created by \(a ^\dag\), carry a charge \(+q\), while the particles which are created by \(b ^\dag\) carry a charge \(-q\). 

The \textbf{Fock space} is defined constructively just like the real case: the vacuum must satisfy \(a (k) \ket{0} = b(k) \ket{0} = 0\); we can create any pure Fock state by repeatedly applying \(a ^\dag\) and \(b ^\dag\). 


\begin{claim}
If we denote these states, with the appropriate normalizations, by \(\ket{1(p)} \propto a ^\dag (p) \ket{0}\) and \(\ket{\overline{1} (p)} \propto b ^\dag(p) \ket{0}\), then we have: 
%
\begin{subequations}
\begin{align} \label{eq:fock-one-particle-state-properties}
N \ket{1 (p)} &= 1 \ket{1 (p)} &
N \ket{\overline{1} (p)} &= 1 \ket{\overline{1} (p)}  \\
H \ket{1 (p)} &= \omega_{p} \ket{1 (p)} &
H \ket{\overline{1} (p)} &= \omega_{p} \ket{\overline{1} (p)}  \\
Q \ket{1 (p)} &= +q \ket{1 (p)} &
Q \ket{\overline{1} (p)} &= -q \ket{\overline{1} (p)}  
\,.
\end{align}
\end{subequations}
\end{claim}

\begin{proof}
Let us show it for one of them: 
%
\begin{subequations}
\begin{align}
N \ket{1 (p)} &\propto
\qty(\int \dd[3]{k} \qty(a ^\dag(k) a(k) + b ^\dag(k) b (k))) 
a ^\dag (p) \ket{0}  \\
&= \int \dd[3]{k} \qty( a ^\dag (k) a (k) a ^\dag(p) + b ^\dag(k) b(k) a ^\dag (p)) \ket{0}  \\
&= \int \dd[3]{k} \qty( a ^\dag (k) \qty( a ^\dag(p) a (k) + \qty[a(k), a ^\dag(p)]) + b ^\dag(k) a ^\dag (p) b(k)) \ket{0}  \\
&= \int \dd[3]{k} a ^\dag (k) \qty[a(k), a ^\dag(p) ]  \ket{0}   \\
&= \int \dd[3]{k} \delta^{(3)} (k-p) a ^\dag(k) \ket{0} \\
&= a ^\dag(p) \ket{0}
\,.
\end{align}
\end{subequations}

The basic reasoning for the others is the same. 
\end{proof}

As for the \textbf{commutators}, the same field calculated at different events in spacetime commutes with itself. On the other hand, 
%
\begin{align}
\qty[ \varphi (x), \varphi ^\dag(y)]
= \qty[\varphi_{+} (x), \varphi ^\dag_{-} (y)]
+ \qty[\varphi_{-} (x), \varphi ^\dag_{+} (y)]
= D(x-y)
\,,
\end{align}
%
the same function we found in the real case. 

\begin{claim}
\begin{align}
\qty[\varphi_{\pm} (x), \varphi ^\dag_{\mp}(y)] = D_{\pm} (x-y)
\,.
\end{align}
\end{claim}

\begin{claim}
The commutator between \(\varphi (x)\) and \(\pi (y)\) gives: 
\end{claim}

Just like the real scalar field, we have microcausality. 

\end{document}
