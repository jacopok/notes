\documentclass[main.tex]{subfiles}
\begin{document}

\section{Chiral interactions}

\marginpar{Saturday\\ 2020-6-20, \\ compiled \\ \today}

(Coming from an exercise, 9.1.e)

Consider the following interaction Lagrangian 
%
\begin{align}
\mathscr{L} _{\text{int}} =\varphi \overline{\psi}_{\ell} \qty(a_{\ell} + i b_\ell \gamma_{5}) \psi_{\ell}
\,,
\end{align}
%
which describes the interaction between a real scalar field and a fermion. \(a_\ell\) and \(b_\ell\) are small real parameters.

It is self-adjoint: its conjugate reads 
%
\begin{align}
\mathscr{L} _{\text{int}} ^\dag &= \varphi \psi_{\ell} ^\dag \qty(a_{\ell} - i b_\ell \gamma_{5} ^\dag) \gamma^{0 \dag} \psi_{\ell} ^\dag  \\
&= \varphi \psi_{\ell} ^\dag \qty(a_{\ell} - i b_\ell \gamma_{5}) \gamma_0 \psi_{\ell}   \marginnote{\(\gamma_5 \) and \(\gamma^0 \) are self-adjoint.} \\
&= \varphi \psi_{\ell} ^\dag \gamma^{0} \qty(a_{\ell} + i b_\ell \gamma_{5}) \psi_{\ell}
\,,
\end{align}
%
where we switched the sign when anticommuting \(\gamma_5\) and \(\gamma^{0}\).

\subsection{Vertex contribution}

The Feynman rule for this interaction is given from the interaction Hamiltonian; in our case there are no derivative couplings so that is just minus the interaction Lagrangian. The vertex contribution to the diagram is given by the first-order term in the expansion of the \(S\)-matrix: in momentum space we do not have to include the integral over \(\dd[4]{x}\) so the vertex contribution is simply \(-i \mathscr{H} _{\text{int}} = i \mathscr{L} _{\text{int}}\), with all the fields removed: so it is 
%
\begin{align}
i \qty(a_{\ell} + i b_\ell \gamma_{5})
\,.
\end{align}

We want to consider an initial state of \(\ket{i} = \ket{e^{-}(p)}\), and a final state of \(\ket{f} = \ket{s(q) e^{-}(k)}\). 
The transition amplitude \(\bra{f} S \ket{i}\) will be given by 
%
\begin{align}
S_{fi}
= (2 \pi )^{4} \delta^{(4)} (p - q - k)
i\overline{u}_{s'}(k) \qty(a_e + i b_e \gamma_{5}) u_s (p)
\,,
\end{align}
%
where there is no trace of the scalar: in fact, its contribution is only  \(\varphi_{+} ^\dag (x) \ket{s(p)} = e^{-ipx} \ket{0}\), which only comes up in the momentum conservation. 

\subsection{\(e^{+} e^{-} \to \mu^+ \mu^-\) scalar-mediated scattering}

This is similar to the QED scattering in terms of incoming and outgoing particles, but the coupling is different: the Feynman diagram to second order is shown in figure \ref{fig:scalar-mediator-eemumu}.

\begin{figure}[ht]
\centering
\feynmandiagram[horizontal = y to x]{
e1 -- [fermion] y -- [fermion] e2,
y -- [scalar] x,
mu1 -- [fermion] x -- [fermion] mu2
};
\caption{\(e^{+} e^{-} \to \mu^+ \mu^-\) scattering with a scalar mediator.}
\label{fig:scalar-mediator-eemumu}
\end{figure}

Let us denote \(a = a_e = a_\mu \) and similarly for \(b\).
Then, the unpolarized Feynman amplitude will read: 
%
\begin{align}
\begin{split}
\abs{\mathcal{\overline{M}}}^2
&= \frac{1}{4 (k^2- M^2 (+ i \epsilon))^2 }
\Tr[ (a + i b \gamma_5 ) (\slashed{p} + m_e) \qty(a + i b \gamma_5 ) \qty(\slashed{q} - m_e)] \times \\
&\phantom{=}\ 
\times \Tr[ (a + i b \gamma_5 ) (\slashed{p}' + m_\mu ) \qty(a + i b \gamma_5 ) \qty(\slashed{q}' - m_\mu )]
\end{split}
\,,
\end{align}
%
since the scalar propagator in momentum space is \(1/ (k^2 - m^2)\), and we get two of them from taking the square modulus.
The factor \(1/4\) is due to the average being taken on the initial polarizations. 
\(M\) is the mass of the scalar, \(m_e\) and \(m_\mu\) are those of the electron and muon respectively. 

The traces to compute seem to be many, but actually only three terms survive for each: this is because of the fact that, beyond the fact that traces of an odd number of \(\gamma^{\mu }\) vanish, we also have 
%
\begin{align}
\Tr[\gamma_5 ] 
= \Tr[\gamma_5 \gamma^{\mu }]
= \Tr[\gamma_5 \gamma^{\mu } \gamma^{\nu }]
= \Tr[\gamma_5 \gamma^{\mu } \gamma^{\nu } \gamma^{\rho }]
= 0
\,,
\end{align}
%
while the nonvanishing ones are \cite[section 6.3]{kumerickiFeynmanDiagramsBeginners2016}: 
%
\begin{align}
\Tr[\gamma_5 \gamma_5 ] = 4 \qquad \text{and} \qquad
\Tr[\gamma_5 \gamma^{\mu } \gamma^{\nu } \gamma^{\rho } \gamma^{\sigma }] = -4i \epsilon^{\mu \nu \rho \sigma }
\,,
\end{align}
%
since \(\gamma_{5} \gamma_5  = \mathbb{1} \). 

So, we get: 
%
\begin{align}
\Tr[ (a + i b \gamma_5 ) (\slashed{p} + m_e) \qty(a + i b \gamma_5 ) \qty(\slashed{q} - m_e)] &= 4 \qty( (a^2 + b^2) p \cdot q +  m_e^2 ( b^2 -a ^2))
\,.
\end{align}

This was found by writing all the pieces on paper. Recall that the matrices \emph{anticommute}, so if we need to bring two \(\gamma_5\)s together we have to keep count of the number of swaps. 

Then, our polarization-averaged square amplitude reads: 
%
\begin{align}
\abs{\mathcal{\overline{M}}}^2
&=
\frac{4 \qty( (a^2 + b^2) p \cdot q +  m_e^2 ( b^2 -a ^2)) \qty( (a^2 + b^2) p' \cdot q' +  m_\mu^2 ( b^2 -a ^2))}{ (k^2- M^2 + i \epsilon)^2 }
\,.
\end{align}

This is an \(s\)-channel diagram: \(s = k^2\), and we can also express this Mandelstam variable as \(s = (p + q)^2 = (p'+ q')^2 = 2 m_e^2 + 2p \cdot q  = 2 m_\mu^2 + 2 p' \cdot q'\).

We can assume the electron mass to be zero, since it is small compared to the muon's: 
%
\begin{align}
\abs{\mathcal{\overline{M}}}^2
&=
\frac{4 (a^2 + b^2) p \cdot q \qty( (a^2 + b^2) \qty( p \cdot q - m_\mu^2) +  m_\mu^2 ( b^2 -a ^2))}{ (k^2- M^2 + i \epsilon)^2 }
\,.
\end{align}

Now, we can calculate the cross section: for simplicity, let us move to the center of mass frame. Then, we will have 
%
\begin{align}
\dd{\sigma } = \frac{\abs{\mathcal{\overline{M}}}^2}{4 I_{12}} \dd{\phi }^{(n_f)}
\,,
\end{align}
%
where the phase space element is given by 
%
\begin{align}
\dd{\phi } = (2 \pi )^{4} \delta^{(4)} (p+q-p'-q') 
\frac{ \dd[3]{p'}}{(2 \pi )^3 2 \omega_{p}'} 
\frac{ \dd[3]{q'}}{(2 \pi )^3 2 \omega_{q}'} 
\,,
\end{align}
%
while in our frame 
%
\begin{align}
I_{12} &=  \sqrt{(p \cdot q)^2 - m_e^{4}}  \\
&= \sqrt{\omega_{e}^{4} + 2 \omega_{e}^2 \abs{\vec{p}}^2 + \abs{p}^{4} - m_e^{4}}  \\
&= \sqrt{4 \omega_{e}^2 \abs{p}^2} = 2 \omega_{e} \abs{p}
\,.
\end{align}

We are basically replicating what we have done when deriving the two-output cross section: the result, in the approximation \(m_e \approx 0\), is given by \eqref{eq:zero-initial-mass-cross-section}: 
%
\begin{align}
\dv{\sigma }{\Omega_1 } &= \frac{1}{64 \pi^2}\frac{\abs{\mathcal{\overline{M}}}^2 }{s} \sqrt{1 - \frac{4 m_\mu^2}{s}}  \\
&= \frac{(a^2 + b^2) (s/2) \qty( (a^2 + b^2) \qty( s/2 - m_\mu^2) +  m_\mu^2 ( b^2 -a ^2))}{ (16 \pi^2 s) (s- M^2 + i \epsilon)^2 }
\sqrt{1 - \frac{4 m_\mu^2}{s}}  \\
&= \frac{(a^2 + b^2)\qty( (a^2 + b^2) \qty( s/2 - m_\mu^2) +  m_\mu^2 ( b^2 -a ^2))}{ 32 \pi^2 (s- M^2 + i \epsilon)^2 }
\sqrt{1 - \frac{4 m_\mu^2}{s}}
\,.
\end{align}

There is no angular dependence! Scalar-mediated interactions are \textbf{isotropic}. 

\subsection{Scalar decay rate}

We want to calculate the decay rate of the process with initial state \(\ket{i} = \ket{s(p)}\) and final state \(\ket{f} = \ket{e^{-}(p') e^{+} (q')}\).

The polarization-averaged amplitude is given (in the lab=center of mass frame) by: 
%
\begin{align}
\abs{\mathcal{\overline{M}}}^2 
&= \frac{1}{(k^2 - M^2 + i \epsilon )^2}
\Tr[ (a + i b \gamma_5 ) (\slashed{p} + m_e) \qty(a + i b \gamma_5 ) \qty(\slashed{q} - m_e)]  \\
&= \frac{(a^2 + b^2)(p' \cdot q') + m_e^2 (b^2 - a^2)}{(k^2 -M^2+ i \epsilon)^2}  \\
&=\frac{(a^2 + b^2)(M^2 / 2- m_e^2) + m_e^2 (b^2 - a^2)}{(k^2 -M^2+ i \epsilon)^2}
\,.
\end{align}

Then, we can compute the differential decay rate \eqref{eq:differential-decay-rate-equal-masses}: 
%
\begin{align}
\dv{\Gamma}{\Omega_1 } &= \frac{\abs{\mathcal{\overline{M}}}^2}{64 \pi^2 M}\sqrt{1 - \frac{4 m_e^2}{M^2}}  \\ 
&=\frac{(a^2 + b^2)(M^2 /2- m_e^2) + m_e^2 (b^2 - a^2)}{64 \pi^2 M (k^2 -M^2+ i \epsilon)^2}
\sqrt{1 - \frac{4 m_e^2}{M^2}}  \\
&= \qty[2 a^2 \qty( \frac{M^2}{4 } - m_e^2) + b^2 \frac{M^2}{2}] \frac{1}{64 \pi^2 M (k^2 -M^2+ i \epsilon)^2}
\sqrt{1 - \frac{4 m_e^2}{M^2}}  \\
\Gamma &= \qty[2 a^2 \qty( \frac{M^2}{4 } - m_e^2) + b^2 \frac{M^2}{2}] \frac{1}{16 \pi M (k^2 -M^2+ i \epsilon)^2}
\sqrt{1 - \frac{4 m_e^2}{M^2}}
\,.
\end{align}


\end{document}
