\documentclass[main.tex]{subfiles}
\begin{document}

\section{The Dirac Equation}

\marginpar{Saturday\\ 2020-3-28, \\ compiled \\ \today}

The Klein Gordon equation, which we discussed in the previous lectures, has two main issues: 
\begin{enumerate}
  \item It is a second order equation, since we started from the relativistic dispersion relation \(E^2 = M^2 + \abs{\vec{p}}^2\), so it admits a negative as well as a positive energy solution: \(E = \pm \omega_{p}\). 
  \item The charge associated with its 4-current density \(J^{\mu }\) is not positive definite: 
  %
  \begin{align}
  Q = \int \dd[3]{x} J^{0}(\vec{x}, t) 
  \,
  \end{align}
  %
  can be negative. 
\end{enumerate}

In order to clarify the problem with these, we discussed the Klein Paradox, in which there is a violation of unitarity for a scattering process: the reflection probability was \(>1\) while the transmission probability was negative. 

Dirac, in 1928, tried a different approach. 

\subsection{Historical derivation of the Dirac equation}

We want to build an equation in the form 
%
\begin{align}
i \pdv{}{t} \psi  = H_D \psi 
\,,
\end{align}
%
for some Dirac Hamiltonian \(H_D\), which we require to be first-order in the space derivatives and the mass, so that our equation can be covariant: this in general will be written as 
%
\begin{align}
H_D = -i \vec{\alpha} \cdot \vec{\nabla} + \beta M = \vec{\alpha} \cdot \qty(-i \vec{\nabla}) + \beta M
\,,
\end{align}
%
for some yet to be determined 4 coefficients \(\vec{\alpha}\) and \(\beta \).

This is equivalent to saying that the energy \(E\) is given by \(\vec{\alpha} \cdot \vec{p} + \beta M\), since \(\vec{p} = - i \vec{\nabla}\). 

We are assuming the coefficients to be constant, we shall see that this hypothesis works out. Also, as we shall these cannot be numbers, but instead they are matrices: for now, all this means is that we need to be careful when manipulating them, since in general they will not commute. 

In order to determine \(\vec{\alpha }\) and \(\beta \) we impose the conditions: 
\begin{enumerate}
  \item the Dirac Hamiltonian \(H_D\) is hermitian, since it needs to describe the physical property of energy, therefore it is an observable;
  \item if \(\psi \) solves the Dirac equation, then it also solves the KG equation. 
\end{enumerate}

\todo[inline]{Are KG solutions also Dirac solutions generally?}

The first condition can be written as \(H = H ^\dag\), but to find out what it means for the coefficients \(\vec{\alpha }\) and \(\beta \) we need to know what \(\vec{\nabla} ^\dag\): since we know that momentum \(\vec{p} = -i \vec{\nabla}\) is self adjoint, we must have \(\vec{p} ^\dag = (-i) ^\dag \vec{\nabla} ^\dag = i \vec{\nabla} ^\dag \overset{!}{=} -i \vec{\nabla} = \vec{p} \),
which means that \(\vec{\nabla} ^\dag = - \vec{\nabla}\).
So, we get 
%
\begin{align}
H ^\dag &= (-i) ^\dag \vec{\alpha} ^\dag \cdot \vec{\nabla} ^\dag + \beta ^\dag M ^\dag  \\
&= -i \alpha^{*} \cdot \vec{\nabla} + \beta^{*} M
\,,
\end{align}
%
since \(M\) is real.\footnote{In the professor's notes this is done explicitly by integrating by parts: this method can be adapted to look like that, since one can use integration by parts to show that for any wavefunctions \(\phi \) and \(\psi \) we have \(\braket{\phi }{\nabla \psi } + \braket{\nabla \phi}{\psi } = 0\). Then, the only difference between the approaches is whether we only integrate by parts to get the adjoint of \(\nabla\) or the whole of \(H_D\). }
Therefore, the coefficients must satisfy \(\vec{a}^{*} = \vec{\alpha}\) and \(\beta^{*} = \beta \) themselves. 

The other condition, consistency with the KG equation, means that when we square the Dirac time derivative operator 
%
\begin{align}
\pdv{}{t} = -i H_D
\,
\end{align}
%
we should get the KG square time- derivative operator 
%
\begin{align}
\pdv[2]{}{t} = \nabla^2 + M^2
\,.
\end{align}

So, we find 
%
\begin{align}
\nabla^2 + M^2 &\overset{!}{=} (-i H_D)^2 = - H_D^2  \\
&= -\qty(-i \alpha_{i}  \nabla_{i} + \beta ) \qty(-i \alpha_{j} \nabla_{j} + \beta )  \\
&= \alpha_{i} \alpha_{j} \nabla_{i} \nabla_{j} + i \qty(\alpha_{i} \beta  + \beta \alpha_{i}) \nabla_{i} + \beta^2  \\
&= \frac{1}{2} \qty{\alpha_{i}, \alpha_{j}} \nabla_{i} \nabla_{j} 
+ i \qty{\alpha_{i}, \beta } \nabla_{i} + \beta^2
\,,
\end{align}
%
where we introduced the anticommutator bracket notation: 
%
\begin{align}
\qty{A, B} = AB + BA
\qquad \text{while} \qquad
\qty[A, B] = AB - BA
\,.
\end{align}

So, in order for the equations to be equivalent we need to impose 
%
\begin{align}
\frac{1}{2}\qty{\alpha_{i}, \alpha_{j}} = \delta_{ij}
\qquad \qquad 
\qty{\alpha_{i}, \beta } = 0
\qquad \qquad 
\beta^2 = M^2
\,.
\end{align}
%


\end{document}