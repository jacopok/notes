\documentclass[main.tex]{subfiles}
\begin{document}

\subsection{Covariance of the Dirac Equation}

\marginpar{Sunday\\ 2020-3-29, \\ compiled \\ \today}

How do these new spinors transform under Lorentz transformations? 

We suppose that for a general Lorentz transformation \(\Lambda \) we will have something in the form 
%
\begin{align}
\psi^{\prime } (x') = S(\Lambda ) \psi (x)
\,,
\end{align}
%
where \(S(\Lambda )\) is a \(4 \times 4\) matrix, belonging to the \emph{spinorial representation of the Lorentz group}. 
We do this because if we were to simply impose \(\psi^{\prime }(x^{\prime }) = \psi (x)\) we get a contradiction, so the equation is not covariant. 

Now, let us see what we must require of the matrices \(S(\Lambda )\) so that the Dirac equation holds in the new frame as well as in the old. 
When we do the computation, recall that the \(\gamma^{\mu }\) matrices are \emph{not} a 4-vector, despite the position of the index: they need not be transformed using a \(\Lambda \) matrix; better put, they are a set of 4 Lorentz scalars

So, we must have 
%
\begin{align}
\qty(i \slashed{\partial}' - M) \psi^{\prime } (x') &=
\qty(i \gamma^{\mu } \tensor{\Lambda }{_{\mu }^{\nu }} \partial_{\nu } -M) S(\Lambda ) \psi (x)  
\\
&= S(\Lambda ) \qty[i \tensor{\Lambda }{_{\mu }^{\nu }} S^{-1}(\Lambda ) 
\gamma^{\mu } S(\Lambda ) \partial_{\nu } - M ] \psi (x)  \\
&=S(\Lambda ) \qty[i \slashed{\partial} - M] \psi (x)
\,,
\end{align}
%
where we used the following:
\begin{enumerate}
  \item the matrix \(S(\Lambda )\) is constant with respect to the spatial coordinates, since the Lorentz transformation is fixed, so we can bring it outside of the derivative \(\partial_{\nu }\);
  \item the matrices \(\gamma^{\mu }  \) and \(S(\Lambda )\) do not commute \emph{a priori}, so if we wish to bring the equation in the form \(S(\Lambda ) \times \qty[\text{old-coordinates Dirac eq}]\) we cannot simply commute them, instead we multiply by \(\mathbb{1} = S S^{-1}\) on the left;
  \item we impose the condition 
  %
  \begin{align}
  \tensor{\Lambda }{_{\mu }^{\nu }} S^{-1}(\Lambda ) \gamma^{\mu } S(\Lambda ) \overset{!}{=} \gamma^{\nu }
  \,
  \end{align}
  %
  in order to find the expression we need in order for the equation to be covariant, since as long as the matrix \(S(\Lambda )\) is nondegenerate the equation \(S(\Lambda ) \qty[i \slashed{\partial} - M] \psi =0\) has the same solutions as the Dirac equation.
\end{enumerate}

If we multiply by an inverse Lorentz matrix on both sides we can bring this equation into the form 
% 
\begin{align} \label{eq:spinorial-representation-definition}
S^{-1}(\Lambda ) \gamma^{\mu } S(\Lambda ) = \tensor{\Lambda }{^{\mu }_{\nu }} \gamma^{\nu }
\,,
\end{align}
%
so we can say that the Dirac equation is covariant as long as we find some matrices \(S(\Lambda )\) satisfying these conditions. 
Do note that while this looks like a vector transformation law, the \(\gamma^{\mu }\) matrices do not transform under Lorentz transformations: what we are stating is that it is equivalent to apply a Lorentz matrix to them and to transform them as spinorial matrices. 

In a way, we are asking the transformation laws for spinors and vectors to be compatible. 

\subsubsection{Explicit realization of the spinorial representation}

In order to do this, we consider infinitesimal Lorentz transformations, 
%
\begin{align}
\tensor{\Lambda }{^{\mu }_{\nu }} = \tensor{ \delta }{^{\mu }_{\nu }} + \tensor{\omega}{^{\mu}_{\nu }}
\,,
\end{align}
%
where \(\omega_{\mu \nu } = \omega_{[\mu \nu ]}\) is an antisymmetric tensor.\footnote{The fact that \(\omega \) must be antisymmetric may be derived by imposing the condition \(\eta_{\mu \nu } = \tensor{\Lambda }{_{\mu }^{\alpha }} \tensor{\Lambda }{_{\nu }^{\beta }} \eta_{\alpha \beta }\), with the perturbed \(\Lambda \) we wrote above.}

A rank-2 antisymmetric tensor in 4 dimensions has 6 degrees of freedom: these physically correspond to three rotations and three boosts. 

A rather general ansatz looks like:
%
\begin{align}
S(\Lambda ) = \mathbb{1} - \frac{i}{2} \omega_{\mu \nu } \Sigma^{\mu \nu }
\,,
\end{align}
%
where \(\Sigma^{\mu \nu }\) is the set of the generators of the spinorial representation of the Lorentz group. For fixed \(\mu \) and \(\nu \), these are matrices in the spinor space: if we write all the indices explicitly, they look like \(\Sigma^{\mu \nu }_{\aleph \beth}\). 

This means that to each possible basis Lorentz transformation of spacetime (think boost or rotation) we are associating a \(4 \times 4\) basis spinor transformation matrix. 
This is what finding a representation of the group means: for each element of the Lorentz group we are finding a transformation matrix. 

Do note, however, that we are only working at linear order in \(\omega_{\mu \nu }\), so we are not looking yet at a representation of the whole group, instead we are only considering elements which are close to the identity. 
So, we can insert this expression for \(S(\Lambda )\) into the relation between the \(S(\Lambda )\) and \(\Lambda \), equation \eqref{eq:spinorial-representation-definition}, using the fact that to first order \(S = \mathbb{1} + \epsilon \) is the inverse of \(S^{-1} = \mathbb{1} - \epsilon \) we have, to first order in \(\omega \): 
%
\begin{align}
\qty(\mathbb{1} + \frac{i}{2} \omega_{\rho \sigma } \Sigma^{\rho \sigma }) \gamma^{\mu } \qty(\mathbb{1} - \frac{i}{2} \omega_{\alpha \beta } \Sigma^{\alpha \beta }) &= \gamma^{ \mu} +\tensor{\omega}{^{\mu }_{\nu }} \gamma^{\nu } \\
\gamma^{\mu } + \frac{i}{2} \omega_{\rho \sigma } \Sigma^{\rho \sigma } \gamma^{ \mu } - \frac{i}{2} \gamma^{\mu }\omega_{\alpha \beta } \Sigma^{\alpha \beta } &= \gamma^{\mu } + \omega_{\sigma \nu } \eta^{\mu \sigma } \gamma^{\nu }  \\
-\frac{i}{2} \omega_{\rho \sigma } \qty[\gamma^{\mu }, \Sigma^{\rho \sigma }] &= \omega_{\sigma \nu } \eta^{\mu [\sigma } \gamma^{\nu ]}  \\
\qty[\gamma^{\mu }, \Sigma^{\rho \sigma }] &= i \qty(\eta^{\mu \rho } \gamma^{\sigma } - \eta^{\mu \sigma } \gamma^{\rho })
\,,
\end{align}
%
where we antisymmetrized the indices in \(\eta^{\mu \rho } \gamma^{\sigma }\) since they are multiplied by the antisymmetric tensor \(\omega_{\rho \sigma }\), so any symmetric part would not contribute to the equation. 
We also used the fact that \(\omega_{\rho \sigma } \) is proportional to the identity in the spinor space. 

\begin{claim}
This is satisfied by 
%
\begin{align}
\Sigma^{\mu \nu } = \frac{i}{4} \qty[\gamma^{\mu }, \gamma^{\nu }] = \frac{1}{2} \sigma^{\mu \nu }
\,,
\end{align}
\end{claim}

\begin{proof}
We plug the definitions in: 
%
\begin{align}
\qty[\gamma^{\mu }, \Sigma^{\rho \sigma }] &= \frac{i}{4} \qty(\gamma^{\mu } \gamma^{\rho} \gamma^{\sigma } - \gamma^{\mu } \gamma^{\sigma } \gamma^{\rho } - \gamma^{ \rho } \gamma^{\sigma } \gamma^{\mu } + \gamma^{ \sigma } \gamma^{ \rho } \gamma^{ \mu })  \\
i \qty(\eta^{\mu \rho } \gamma^{\sigma } - \eta^{\mu \sigma } \gamma^{\rho }) &= \frac{i}{2} \qty(\gamma^{\mu} \gamma^{\rho } \gamma^{\sigma } + \gamma^{\rho } \gamma^{\mu } \gamma^{\sigma } -\gamma^{\mu } \gamma^{\sigma }\gamma^{\rho } - \gamma^{\sigma } \gamma^{\mu } \gamma^{ \rho } )
\,,
\end{align}
%
where we used the anticommutation relations \(\qty{\gamma^{\mu }, \gamma^{\nu }} = 2 \eta^{\mu \nu }\). 
Then, we must check whether these two expression are equal, that is, whether 
%
\begin{align}
\frac{1}{2} \qty( \mu  \rho \sigma  - \mu  \sigma  \rho  -  \rho  \sigma  \mu  +  \sigma   \rho   \mu ) &\overset{?}{=} 
\mu \rho  \sigma  + \rho  \mu  \sigma  -\mu  \sigma \rho  - \sigma  \mu   \rho   \\
&=  \qty{\mu, \rho }\sigma - \qty{\mu , \sigma } \rho 
\,,
\end{align}
%
where we only write the indices of the \(\gamma \)s for clarity. 

We can now use the identity: 
%
\begin{align}
[\mu , \rho \sigma ] &= \mu \rho \sigma - \rho \sigma \mu  \\
&= \mu \rho \sigma + \rho \mu \sigma - \rho \mu \sigma - \rho \sigma \mu \marginnote{Added and subtracted}  \\
&= \qty{\mu, \rho } \sigma - \rho \qty{\mu, \sigma }
\,.
\end{align}

Note that in our convention there is no division by 2 in the commutator and anticommutator. 
Then, we can recognize these in the initial claim: we find 
%
\begin{align}
\frac{1}{2} \qty( \mu  \rho \sigma  - \mu  \sigma  \rho  -  \rho  \sigma  \mu  +  \sigma   \rho   \mu )
&= \frac{1}{2} \qty(\qty[\mu, \rho \sigma ] - \qty[\mu, \sigma \rho ])  \\
&= \frac{1}{2} \qty(\qty{\mu, \rho } \sigma - \rho \qty{\mu, \sigma } -
\qty{\mu, \sigma } \rho + \sigma \qty{\mu, \rho })  \\
&= \qty{\mu, \rho } \sigma - \qty{\mu, \sigma } \rho 
\,,
\end{align}
%
since the anticommutators \(\qty{\mu, \sigma }\) are proportional to the metric \(\eta^{\mu \sigma }\), which is proportional to the identity in the spinorial space, which commutes with the gamma matrices, so \(\qty{\mu , \sigma }\) commutes with any gamma matrix. 
\end{proof}

Up until now we have worked ``near the identity'' of our transformation group; if we want to extrapolate these results to general transformations we may use the exponential map, which gives us relations in the form 
%
\begin{align}
S(\Lambda ) &= \exp( -\frac{i}{2} \omega_{\mu \nu } \Sigma^{\mu \nu })  \\
S^{-1} (\Lambda ) &= \exp(\frac{i}{2} \omega_{\mu \nu } \Sigma^{\mu \nu })
\,.
\end{align}
%

\subsection{Dirac conjugate spinor}



\end{document}