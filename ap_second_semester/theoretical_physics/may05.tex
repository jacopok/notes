\documentclass[main.tex]{subfiles}
\begin{document}

\section{Relativistic classical vector field}

\marginpar{Wednesday\\ 2020-6-10, \\ compiled \\ \today}

We want to write a relativistic field theory of particles such as the photon, which are vectors.
So, we require the vector transformation law under Lorentz transformations, 
%
\begin{align}
V^{\prime \mu } (x') = \tensor{\Lambda }{^{\mu }_{\nu }} V^{\nu } (x)
\,,
\end{align}
%
where \(\Lambda \in SO(1, 3)\) is a Lorentz transformation.
This theory will describe a spin-1 particle. 

Now, we know that massive spin-1 particles have three possible \(s_z\) values while massless ones only have two. Will this be an issue? Stay tuned to find out. We shall start with the massive theory. 

\subsection{Massive vector field theory}

We need to make an ansatz for our Lagrangian --- we require it to be of second order in the derivatives and Lorentz invariant. We define the field strength as in the electromagnetic case: \(V_{\mu \nu } = 2 \partial_{[\mu} V_{\nu ]}\). With it we write the Lagrangian:
%
\begin{align}
\mathscr{L} = - \frac{1}{4}  V^{\mu \nu } V_{\mu \nu } + \frac{1}{2} M^2 V^{\mu }V_{\mu }
\,.
\end{align}

Let us compute its EL equations: they read 
%
\begin{align}
\pdv{\mathscr{L}}{V_{\sigma }} - \partial_{\rho } \pdv{\mathscr{L}}{\partial_{ \rho } V_{\sigma }} = 0
\,;
\end{align}
%
so we need to compute these two derivatives. In the derivative with respect to \(\partial_{\rho } V_{\sigma }\) we get: 
%
\begin{align}
\pdv{V_{\mu \nu } V^{\mu \nu }}{\partial_{\rho }V_{\sigma }}
&= 2 V_{\mu \nu } \pdv{V^{\mu \nu }}{\partial_{\rho }V_{\sigma }}  \\
&= 2 V_{\mu \nu }\qty(\eta^{\mu \rho } \eta^{\nu \sigma } - \eta^{\nu \rho } \eta^{\mu \sigma })  \\
&= 2 V^{\rho \sigma } - 2 V^{\sigma \rho } = 4 V^{\rho \sigma }
\,.
\end{align}

Finally then, we find 
%
\begin{align}
M^2 V^{\sigma } + \partial_{\rho } \qty(V^{\rho \sigma }) = \qty(M^2 + \square ) + \partial^{\sigma }\qty(\partial_{\rho }V^{\rho }) = 0
\,.
\end{align}

This is the \textbf{Proca equation}. If we take its divergence, we find the condition \( \partial_{\sigma } V^{\sigma } = 0\): we can then impose this constraint to simplify the equation to 
%
\begin{align}
\qty(\square + M^2 ) V^{\sigma } &= 0   \\
\partial_{\sigma } V^{\sigma } &= 0
\,.
\end{align}

Note that this works \emph{only because} \(M \neq 0\): if the mass vanishes the divergence of the Proca equation vanishes identically, and the condition \(\partial_{\sigma } V^{\sigma }\) is not imposed. 

\subsection{Solving the massive Proca equation}

The procedure we will follow is to write a generic vector solution to the Klein-Gordon equation, and then to impose the vanishing divergence constraint to it. 
In the real-valued case, it will look like 
%
\begin{align}
V^{\mu } (x) &= \int \frac{ \dd[4]{k}}{(2 \pi )^{4}} \qty(f^{\mu }(k) e^{-ikx} + f^{\mu *} (k) e^{ikx}) 
\,.
\end{align}

If we impose the constraint that it be a solution of the KG equation \(\square V^{\sigma } + M^2 V^{\sigma } = 0\), we get 
%
\begin{align}
0= \int \frac{ \dd[4]{k}}{(2 \pi )^{4}} \qty(- k^2 + M^2) \qty(f^{\mu }(k) e^{-ikx} + f^{\mu *} (k) e^{ikx}) 
\,,
\end{align}
%
and like we did before we can force the particle to be on-shell with a delta-function, so we can remove a degree of freedom:
%
\begin{align}
f^{\mu }(k) = (2 \pi )^{5/2} \sqrt{2 \omega_{k}} \delta (k^2 - M^2)
\sum _{\lambda=0}^{3}
\epsilon^{\mu }_{\lambda }(k) a_\lambda (k)
\,.
\end{align}

The four vectors \(\epsilon^{\mu }_{\lambda }\) are called the \textbf{polarization vectors}; they describe the independent degrees of freedom in momentum space.

\end{document}
