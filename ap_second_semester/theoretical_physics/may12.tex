\documentclass[main.tex]{subfiles}
\begin{document}

\section{T-products and Wick's theorem}

\marginpar{Saturday\\ 2020-6-13, \\ compiled \\ \today}

We have shown up to now that in order to describe an interacting QFT it is sufficient to: 
\begin{enumerate}
    \item know the free evolution of the fields, which is governed by \(H_0 \) and described in the Heisenberg picture;
    \item be able to calculate the time-evolution operator up to an arbitrary perturbative order with \eqref{eq:dyson-series}. 
\end{enumerate}

When we write the Hamiltonian, we always mean the \textbf{normal-ordered} Hamiltonian. 

\begin{definition}
The \(S\)-matrix evolution operator is given by:
%
\begin{align}
S = U_I (- \infty , \infty )
&= T \qty[\exp(-i \int_{- \infty }^{\infty } \dd{\tau }H _{\text{int}}^{I} (\tau ))]  \\
&= T \qty[\exp(-i \int \dd[4]{x} \mathscr{H} _{\text{int}}^{I} (x))]
\,.
\end{align}
\end{definition}

This object describes all the interaction that can happen over all of time (this is used when describing actual interactions, temporal infinity is asymptotically the same as the particles flying away and not interacting anymore).

In the calculation of such an object, we will need to compute products such as 
%
\begin{align}
T \qty[\mathscr{H} _{\text{int}}^{I}(x_1 ) \mathscr{H} _{\text{int}}^{I}(x_2 )] 
= T \qty[ N[\overline{\psi} \slashed{A} \psi ]_{x_1 }N[\overline{\psi} \slashed{A} \psi ]_{x_2 }]
\,,
\end{align}
%
where we wrote the QED interaction Hamiltonian density. The way to compute these objects is to make use of \textbf{Wick's theorem}. 

\subsection{Wick theorem for a real scalar field}

We start from the simplest case and then generalize to more complex ones. 

\begin{definition}
The \textbf{Feynman propagator} for a real scalar field is defined as 
%
\begin{align}
D_F (x-y) = \bra{0} T[\varphi (x) \varphi (y)] \ket{0} 
\,.
\end{align}
\end{definition}
\todo[inline]{Also written by connecting two \(\varphi \)s with a line underneath, to figure out how to tex it.}

Now, from the definition of the time-ordered product we have that 
%
\begin{align}
D_F (x-y) &= 
[x_0 > y_0 ]
\bra{0} \varphi (x) \varphi (y) \ket{0}
+
[x_0 < y_0 ]
\bra{0} \varphi (y) \varphi (x) \ket{0}  \\
&= 
[x_0 > y_0 ]
\bra{0} \varphi_{+} (x) \varphi_{-} (y) \ket{0}
+
[x_0 < y_0 ]
\bra{0} \varphi_{+} (y) \varphi_{-} (x) \ket{0}  \\
&= 
[x_0 > y_0 ]
D_+ (x-y)
-
[x_0 < y_0 ]
D_- (x-y) 
\,,
\end{align}
%
since \(\varphi_{+} \ket{0} = 0\) and \(\bra{0} \varphi_{-} = 0 \), as \(\varphi_{+} \sim a\) and \(\varphi_{-} \sim a ^\dag\).

Here \(D_{\pm}\) are the components of the covariant commutator defined in section \ref{sec:covariant-commutators}. 

Then, the crucial statement is that
\begin{claim}
\begin{align}
T [\varphi (x) \varphi (y)] = N[\varphi (x) \varphi (y)] + D_F (x-y)
\,.
\end{align}
\end{claim}

\begin{proof}
We prove this by writing the term \(T[\varphi \varphi ] - N[ \varphi \varphi ]\) explicitly: we get 
%
\begin{align}
& T[\varphi (x)\varphi (y)] - N[\varphi (x) \varphi (y)] = \\
\begin{split}
&= [x_0 > y_0 ] \qty(\varphi (x) \varphi (y)) + [x_0 < y_0 ] \qty(\varphi (y) \varphi (x)) \\
& 
- \qty([x_0 > y_0 ] + [x_0 < y_0 ]) N [\varphi (x)\varphi (y)] 
\end{split}  \\
\begin{split}
&= 
[x_0 > y_0 ] \qty(\varphi_{-} (x) \varphi_{-} (y) + \varphi_{+} (x) \varphi_{+} (y) + \varphi_{-} (x) \varphi_{+} (y) + \hlc{teal}{\varphi_{+} (x) \varphi_{-} (y)}) \\
& 
+ 
[x_0 < y_0 ] \qty(\varphi_{-} (y) \varphi_{-} (x) + \varphi_{+} (y) \varphi_{+} (x) + \varphi_{-} (y) \varphi_{+} (x) + \hlc{teal}{\varphi_{+} (y) \varphi_{-} (x)}) 
\\
& 
-
[x_0 > y_0 ]
\qty(\varphi_{-} (x) \varphi_{-} (y) + \varphi_{+} (x) \varphi_{+} (y) + \varphi_{-} (x) \varphi_{+} (y) + \hlc{teal}{\varphi_{-} (y) \varphi_{+} (x)} ) 
\\
& 
-
[x_0 < y_0 ]
\qty(\varphi_{-} (x) \varphi_{-} (y) + \varphi_{+} (x) \varphi_{+} (y) + \varphi_{-} (x) \varphi_{+} (y) + \hlc{teal}{\varphi_{-} (y) \varphi_{+} (x)} ) 
\end{split}  \\
&= [x_0 > y_0 ] \qty(\varphi_{+} (x) \varphi_{-} (y) - \varphi_{-} (y) \varphi_{+} (x))
- [x_0 < y_0 ] \qty(\varphi_{-} (x) \varphi_{+} (y) - \varphi_{+}(y) \varphi_{-}(x))  \\
&= 
[x_0 > y_0 ]
D_+ (x-y)
-
[x_0 < y_0 ]
D_- (x-y) \\
&= D_F (x-y)
\,.
\end{align}
\end{proof}

Wick's theorem for scalar fields is the generalization of this result to \(n\) fields: 
It states that 
%
\begin{align}
\begin{split}
T [\varphi (x_1 ) \varphi (x_2 ) \dots \varphi (x_n)]
&=
N [\varphi (x_1 ) \varphi (x_2 )\dots \varphi (x_n)] \\
&+ \sum_{i} N[\varphi (x_1 )\dots \varphi (x_{i-1}) D_F (x_i- x_{i+1}) \varphi (x_{i+2}) \dots \varphi (x_n)]  \\
&+ \sum _{ij}
N[\varphi (x_1 )\dots \varphi (x_{i-1}) D_F (x_i- x_{i+1}) \varphi (x_{i+2}) \dots \\
& \dots \varphi (x_{j-1}) D_F (x_j- x_{j+1}) \varphi (x_{j+2}) \dots \varphi (x_n)]  \\
&+ \text{all possible contractions}
\,.
\end{split}
\end{align}

A \textbf{corollary} of Wick's theorem states that contractions of fields evaluated at the same time do not contribute to the \(T\)-product. Formally, this is stated as: 
%
\begin{align}
T \qty[N[\varphi (x) \varphi (x)] \varphi (x_1 )\dots \varphi (x_n)]
= T \qty[\varphi^2(x) \varphi (x_1 )\dots \varphi (x_n)] _{\text{NCET}}
\,,
\end{align}
%
where the left hand side contains No Contractions at Equal Time --- that is, when we compute the time-ordered product using Wick's theorem we ignore the contractions we would have to compute at equal time.

\begin{proof}
The product \(N[\varphi (x) \varphi (x)]\) can be written, by Wick's theorem, as
%
\begin{align}
N[\varphi (x) \varphi (x)] = T[\varphi (x) \varphi (x)] - D_F (x - x)
\,,
\end{align}
%
where \(T[\varphi (x) \varphi (x)] = \varphi (x) \varphi (x)\). So, if we multiply by the field at the other times and take the time-ordering we find 
%
\begin{align}
T \qty[N[\varphi (x) \varphi (x)] \varphi (x_1 )\dots \varphi (x_n)]
= T[\varphi^2(x) \varphi_1 (x) \dots \varphi_n (x)]
- D_F(x-x) T[\varphi_1 (x) \dots \varphi_n (x)] 
\,,
\end{align}
%
and the term containing \(D_F( x-x)\) is precisely a contraction at equal time, which we ignore. 
\end{proof}

This statement can be \textbf{generalized} to complex scalar fields and to vector fields. 

\begin{definition}
For a complex scalar field \(\varphi \), the Feynman propagator is given by 
%
\begin{align}
D_F (x-y) = \bra{0} T [\varphi (x) \varphi ^\dag (x)] \ket{0}
\,.
\end{align}
\end{definition}

\begin{claim}
This is the same function which was calculated for the real scalar field; one can show that the other contractions vanish: 
%
\begin{align}
\bra{0} T[\varphi (x) \varphi (x)] \ket{0} 
= 0 = 
\bra{0} T[\varphi ^\dag(x) \varphi ^\dag (x)] \ket{0} 
\,.
\end{align}
\end{claim}

\begin{proof}
\todo[inline]{Still to state properly, but this is due to the fact that \(a\) (\(a ^\dag\)) and \(b\) (\(b ^\dag\)) commute with each other, whereas \(a\) does not commute with \(a ^\dag\).}
\end{proof}

\begin{definition}
For a real vector field \(A^{\mu }\) and a complex vector field \(\omega^{\mu }\) the Feynman propagator reads: 
%
\begin{align}
D^{\mu \nu }_{F, \text{ real}} (x-y) &= \bra{0} T[A^{\mu }(x) A^{\nu }(y) ]\ket{0} \\
D^{\mu \nu }_{F, \text{ complex}} (x-y) &= \bra{0} T[\omega^{\mu }(x) \omega^{\nu \dag}(y) ]\ket{0}
\,.
\end{align}
\end{definition}

\begin{claim}
In the Feynman gauge \(\xi = 1\) we have that 
%
\begin{align}
D^{\mu \nu }_{F} (x-y) = [x_0 > y_0 ]
D^{\mu \nu }_{+ } (x-y) 
+
[x_0 < y_0 ]
D^{\mu \nu }_{-}(x-y) 
= - \eta^{\mu \nu } D_F (x-y)
\,.
\end{align}
\end{claim}

Note that for a complex vector field the propagators between \(\omega^{\mu } \omega^{\nu }\) and \(\omega^{\mu \dag} \omega^{\nu \dag}\) vanish. 

The Wick theorem generalizes to the complex scalar, and to the real and complex vector: we have 
%
\begin{align}
T[\varphi (x) \varphi ^\dag (y)] &= N[\varphi (x) \varphi ^\dag (y)] + D_F (x-y) \\
T[A^{\mu } (x) A^{\nu } (y)] &= N[A^{\mu } (x) A^{\nu } (y)] + D_F^{\mu \nu } (x-y) \\
T[\omega^{\mu } (x) \omega^{\nu  \dag} (y)] &= N[\omega^{\mu } (x) \omega^{\nu \dag } (y)] + D_F^{\mu \nu } (x-y) 
\,.
\end{align}

\subsection{Feynman propagator for fermions}

For fermionic fields the discussion gets a little more complicated because of the minus signs coming from the anticommutators. 

\begin{definition}
The Feynman propagator for fermions is defined as 
%
\begin{align}
S^{F}_{\alpha \beta } (x-y)
&= \bra{0}T[\psi_{\alpha }(x) \overline{\psi}_{\beta }(y)] \ket{0}
= - \bra{0} T[\overline{\psi}_\alpha (x) \psi_{\beta }(y)] \ket{0}
\,.
\end{align}
\end{definition}

\begin{claim}
We have the following expression for the propagator: 
%
\begin{align}
S_F(x-y) &= [x_0 > y_0 ] S_+ (x-y) - [x_0 < y_0 ] S_- (x-y)  \\
&= \qty(i \slashed{\partial} + m) D_F (x-y)
\,.
\end{align}
\end{claim}

\todo[inline]{Missing some indices?}

\begin{claim}
Wick's theorem holds in this case as well: 
%
\begin{align}
T[\psi_{\alpha }(x) \overline{\psi}_{\beta }(y)]
= N [\psi_{\alpha }(x) \overline{\psi}_{\beta }(y)]
+ S^{F}_{\alpha \beta }
\,.
\end{align}
\end{claim}

\begin{proof}
Let us write explicitly the difference \(T[\psi \overline{\psi}] - N[\psi \overline{\psi}]\). We find something that is similar to the scalar case, but there is a crucial difference: instead of commuting operators when we switch them around for normal or time ordering, we anticommute them, so we swap their positions and change the sign. This then yields: 
%
\begin{align}
&T[\psi_{\alpha } (x) \overline{\psi}_{\beta } (y)] - N[\psi_{\alpha } (x) \overline{\psi}_{\beta } (y)] = \\
\begin{split}
&= 
[x_0 > y_0 ] \qty(
    \psi_{\alpha }^{-} (x) \overline{\psi}^{-}_{\beta }(y)+
    \psi_{\alpha }^{+} (x) \overline{\psi}^{+}_{\beta }(y)+
    \psi_{\alpha }^{-} (x) \overline{\psi}^{+}_{\beta }(y)+
    \psi_{\alpha }^{+} (x) \overline{\psi}^{-}_{\beta }(y)
) \\
&-
[x_0 < y_0 ]\qty(
    \overline{\psi}_{\beta }^{-} (y) \psi^{-}_{\alpha } (x)+
    \overline{\psi}_{\beta }^{+} (y) \psi^{+}_{\alpha } (x)+
    \overline{\psi}_{\beta }^{-} (y) \psi^{+}_{\alpha } (x)+
    \overline{\psi}_{\beta }^{+} (y) \psi^{-}_{\alpha } (x)
)  \\
&- 
[x_0 > y_0 ] \qty(
    \psi_{\alpha }^{-}(x) \overline{\psi}_{\beta }^{-}(y)+
    \psi_{\alpha }^{+}(x) \overline{\psi}_{\beta }^{+}(y)+
    \psi_{\alpha }^{-}(x) \overline{\psi}_{\beta }^{+}(y)-
    \overline{\psi}_{\beta }^{-}(y) \psi_{\alpha }^{+}(x) 
)\\
&- 
[x_0 < y_0 ] \qty(
    \psi_{\alpha }^{-}(x) \overline{\psi}_{\beta }^{-}(y)+
    \psi_{\alpha }^{+}(x) \overline{\psi}_{\beta }^{+}(y)+
    \psi_{\alpha }^{-}(x) \overline{\psi}_{\beta }^{+}(y)-
    \overline{\psi}_{\beta }^{-}(y) \psi_{\alpha }^{+}(x) 
)
\end{split}\\
\begin{split}
&= [x_0 > y_0 ]\qty(
    \psi_{\alpha }^{-} (x) \overline{\psi}^{+}_{\beta }(y)+
    \psi_{\alpha }^{+} (x) \overline{\psi}^{-}_{\beta }(y)-
    \psi_{\alpha }^{-}(x) \overline{\psi}_{\beta }^{+}(y)+
    \overline{\psi}_{\beta }^{-}(y) \psi_{\alpha }^{+}(x) 
) \\ 
&+
[x_0 < y_0 ]\bigg(-
    \overline{\psi}_{\beta }^{-} (y) \psi^{-}_{\alpha } (x)-
    \overline{\psi}_{\beta }^{+} (y) \psi^{+}_{\alpha } (x)-
    \overline{\psi}_{\beta }^{-} (y) \psi^{+}_{\alpha } (x)-
    \overline{\psi}_{\beta }^{+} (y) \psi^{-}_{\alpha } (x)+\\
    &\phantom{=}\ 
    -\psi_{\alpha }^{-}(x) \overline{\psi}_{\beta }^{-}(y)-
    \psi_{\alpha }^{+}(x) \overline{\psi}_{\beta }^{+}(y)-
    \psi_{\alpha }^{-}(x) \overline{\psi}_{\beta }^{+}(y)+
    \overline{\psi}_{\beta }^{-}(y) \psi_{\alpha }^{+}(x) 
\bigg)
\end{split}
\,.
\end{align}

\todo[inline]{Man, this is a tedious calculation}
\end{proof}

\end{document}