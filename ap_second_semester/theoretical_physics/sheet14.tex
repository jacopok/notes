\documentclass[main.tex]{subfiles}
\begin{document}

\section{QED \(S\)-matrix expansion}

\marginpar{Monday\\ 2020-6-15, \\ compiled \\ \today}

As an explicit example of the theory we developed for an interacting QFT, we consider Quantum ElectroDynamics.

The Lagrangian is written as 
%
\begin{align}
\mathscr{L} = 
- \frac{1}{4} F^{\mu \nu } F_{\mu \nu }
+ \mathscr{L} _{\text{gauge-fixing}}
+ i q \overline{\psi} \qty(i\slashed{\DD} - M) \psi  
\,,
\end{align}
%
where \(q\) is the charge of the fermion --- for instance, electrons have \(q = - \abs{e}\). 
This can be decomposed as \(\mathscr{L} = \mathscr{L}_{0} + \mathscr{L} _{\text{int}}\), with 
%
\begin{align}
\mathscr{L} _{\text{int}} = - q \overline{\psi} \gamma^{\mu } \psi A_{\mu } = - \mathscr{H} _{\text{int}}
\,.
\end{align}

Notice that the interaction Lagrangian and Hamiltonian are opposite of each other: this holds in general, as long as there are no derivative terms --- basically, we have an interaction \emph{potential}. 

We are working in the \textbf{interaction picture}. 

The \(S\)-matrix operator \eqref{eq:s-matrix} for QED reads: 
%
\begin{align}
S _{\text{QED}}
&= T \qty[\exp(-iq \int \dd[4]{x} N [\overline{\psi}(x) \slashed{A} (x) \psi (x)])]  \\
&= \sum _{n=0}^{ \infty }
\frac{(-iq)^{n}}{n!}
\int \dd[4]{x_1 } \dots \dd[4]{x_{n}}
T \qty[N[\dots]_{x_1} \dots N[\dots]_{x_n}]
\,.
\end{align}

\subsection{Expansion in position space}

\subsubsection{0th order term}

It is just the identity, since it corresponds to no interaction:
%
\begin{align}
S^{(0)} = \mathbb{1}
\,.
\end{align}

\subsubsection{1st order term}

It is given by 
%
\begin{align}
S^{(1)} &= - iq \int \dd[4]{x} T \qty[N[\overline{\psi}(x) \slashed{A} (x) \psi (x)]]  \\
&= - iq \int \dd[4]{x} N[\overline{\psi}(x) \slashed{A} (x) \psi (x)]
\,.
\end{align}

In order to remove the time-ordering we have used the corollary of Wick's theorem. 

\todo[inline]{It is not clear to me how the corollary applies here. We have something like \(T[N[A(x)B(x)]]\), and the corollary tells us that it is equivalent to \(T[A(x)B(x)]_{\text{NCET}} = A(x)B(x)\): this is fine, but we lose the normal-ordering! Why does he still write it? I think it might not be actually there.}

We can split \(\overline{\psi} \), \(\slashed{A}\) and \(\psi \) into their \(+\) and \(-\) components: we get eight contributions, 
%
\begin{align}
N \qty[
    \qty(\overline{\psi}_{+} + \overline{\psi}_{-})
    \qty(\slashed{A}_{+} + \slashed{A}_{-})
    \qty(\psi_{+} + \psi_{-})
]
= N \qty[
    \sum \overline{\psi}_{\pm} \slashed{A}_{\pm} \psi_{\pm}
]
\,,
\end{align}
%
which we can represent using Feynman diagrams.
All of these diagram only have one vertex at \(x\), and they have a fermion, an antifermion and a photon being created/annihilated there (all the 8 possible combinations). As an example, we show: 
%
\begin{align}
\overline{\psi}_{+} \slashed{A}_{-} \psi_{+}
= \feynmandiagram[inline=(c.base), horizontal = c to p]{
    c [particle=\(x\)] -- [photon] p [particle=\(\gamma \)],
    a [particle=\(e^{-}\)] -- [fermion] c -- [fermion] b [particle=\(e^{+}\)]
};
\end{align}
%
here we have an electron and a positron annihilating to form a photon.

\todo[inline]{Incidentally, this is the only vertex which is wrong in the professor's notes.}

In general, particles which are created (so, with subscript \(-\)) go rightward, particles which are annihilated come from the left.

\end{document}