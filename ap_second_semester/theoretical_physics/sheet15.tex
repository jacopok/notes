\documentclass[main.tex]{subfiles}
\begin{document}

\section{Observables: decay rate and cross section}

\marginpar{Wednesday\\ 2020-6-17, \\ compiled \\ \today}

In the last section we have shown how to calculate perturbatively 
%
\begin{align}
S_{fi} = (2 \pi )^{4} \delta^{(4)}(p_i - p_f) \mathcal{M}_{fi}
\,,
\end{align}
%
using Feynman rules. 
Now we want to connect the result we have gotten with experimentally observable quantities: 
\begin{enumerate}
    \item the \textbf{decay rate} describes the case in which we have 1 incoming particle and \(n_f\) outgoing ones;
    \item the \textbf{cross section} describes the case in which the initial state consists of 2 particles, which scatter to produce \(n_f\) particles.
\end{enumerate}

We will discuss how to connect these to the \(S\)-matrix; but first, we must make sure we are using the correct normalizations. 

\subsection{Normalizations}

\subsubsection{Covariant versus canonical}

In deriving the expression for the conservation of probability for the \(S\)-matrix evolution we have used the fact that the initial and final states were \textbf{canonically}s normalized (i.\ e.\ to 1). 

However, in the last section our states were \textbf{covariantly} normalized, as 
%
\begin{align}
\braket{1(p)}{1(k)} = (2 \pi )^3 2 \omega_{p} \delta^{(3)} (\vec{p} - \vec{k}) 
\,,
\end{align}
%
since the canonical normalization is not covariant. 

The relation connecting the two normalizations is: 
%
\begin{align}
S^{CN}_{fi} = \bra{\psi_{f}}_{CN} S \ket{\psi_{i}}_{CN}
= \frac{\bra{\psi_{f}} S \ket{\psi_{i}}}{\abs{\psi_{f}} \abs{\psi_{i}}} = \frac{S_{fi}}{\abs{\psi_{f}} \abs{\psi_{i}}}
\,,
\end{align}
%
where if we do not specify \(CN\) we mean that the states are covariantly normalized. 
The covariant-normalization \(S_{fi}\) is the one we derived in the last section. 
In terms of the Feynman amplitude, this reads 
%
\begin{align}
S^{CN}_{fi} = (2 \pi )^{4} \delta^{(4)} (p_i - p_f)
\frac{\mathcal{M}_{fi}}{\abs{\psi_{i}} \abs{\psi_{f}}}
= (2 \pi )^{4} \delta^{(4)} (p_i - p_f) \mathcal{M}_{fi}^{CN}
\,.
\end{align}

\subsubsection{The normalization issue}

In order to investigate the normalization issue we use a discrete system --- we will take the continuum limit at the end. 
We consider a cubic box of side \(L\) and volume \(V = L^3\).

From quantum mechanics we know that the momenta of the particles inside the box are quantized according to 
%
\begin{align}
p_{i} = \frac{2 \pi }{L} n_i
\,,
\end{align}
%
where \(i = 1, 2, 3\) and \(n_i \in \mathbb{Z}\). 

The continuum expressions we wrote can be converted into discrete ones with: 
%
\begin{align}
\int \dd[3]{p} f (\vec{p})  &\to \sum _{\vec{n}} \qty(\frac{2 \pi }{L})^3 f_{\vec{n}}  \\
\delta^{(3)} (p-k) & \to \qty(\frac{L}{2 \pi })^3 \delta_{\vec{n}, \vec{m}}
\,.
\end{align}

The relation \(\int \dd[3]{p } \delta^{(3)} (p-k) = 1\) then becomes \(\sum _{\vec{n}} (2\pi / L)^{3-3} \delta_{\vec{n}, \vec{m}}  = 1\). 
This is all consistent. 

Then, this discrete version of the delta is an actual function, which can be evaluated at zero: 
%
\begin{align}
\delta^{(3)}(0) \to \qty( \frac{L}{2 \pi })^{3}
\qquad \text{and} \qquad
\delta^{(4)}(0) \to \qty( \frac{L}{2 \pi })^{3} \frac{T}{2 \pi }
\,,
\end{align}
%
so now the covariant normalization reads: 
%
\begin{align}
\braket{1(p)}{1(p)} &= 2 \omega_{p} (2 \pi )^3 \delta^{(3)}(0) = 2 \omega_{p} V  \\
\ket{1(p)} &= \sqrt{2 \omega_{p} V} \ket{1(p)}_{CN}
\,.
\end{align}

So, if we have \(n_i\) particles initially and \(n_f\) particles later, the relation between the Feynman amplitudes is 
%
\begin{align}
\mathcal{M}_{fi}^{CN} = \prod_{i=1}^{n_i} \frac{1}{\sqrt{2 \omega_{i}V}}
\prod_{j=1}^{n_f} \frac{1}{\sqrt{2 \omega_{j}V}} \mathcal{M}_{fi}
\,.
\end{align}

\todo[inline]{By the way this is used later, it is written improperly: for multiparticle states, we divide by \(V\) a single time, so it should be outside of the product!}

\subsection{Decay rate}

\todo[inline]{What is the graph about? Do we turn on and then off the interaction Hamiltonian?}

It describes the decay of an unstable particle. There are no decays in QED, because of the fact that the photon is massless they are not kinematically allowed. 

Decays, however, occur in theories which have massive vectors or scalars (and the standard model has them). 

The probability (actually, a probability \emph{density} in momentum space) of the decay is given by 
%
\begin{align}
\mathbb{P} = \abs{S^{CN}_{fi}} 
&= \abs{(2 \pi )^{4} \delta^{(4)} (p_i - p_f) \mathcal{M}^{CN}_{fi}}^2  \\
&= (2 \pi )^{4} \delta^{(4)} (p_i-p_f) \underbrace{VT}_{(2 \pi )^{4} \delta^{(4)}(0)} \abs{\mathcal{M}_{fi}^{CN}}^2  \\
&=  (2 \pi )^{4} \delta^{(4)} (p_i-p_f) VT \frac{1}{2 \omega_{i} V} \prod_{j=1}^{n_f} \frac{1}{2 \omega_{f} V} \abs{\mathcal{M}_{fi}}^2   \\
&= (2 \pi )^{4} \delta^{(4)} (p_i - p_f) \frac{T}{2 \omega_{i}}
\prod_{j=1}^{n_f} \frac{1}{2 \omega_{f} V} \abs{\mathcal{M}_{fi}}^2
\,,
\end{align}
%
which goes to zero as \(V \to \infty \), which reflects the fact that the number of states with an exact momentum \(p_f\) goes to zero. 

So, instead of using total probabilities we want to move to probability densities, and accordingly we will not look at a final state with an exact momentum \(p_f\); instead, we will consider states with a momentum \(p \in (p_f, p_f + \dd{p_f})\). 

In the ``particle in a box'' language, this means we consider 
%
\begin{align}
\dd{n_i} = \frac{L}{2 \pi } \dd{p_i}
\,,
\end{align}
%
so that 
%
\begin{align}
\dd[3]{n} = \qty( \frac{L}{2 \pi })^3 \dd[3]{p} 
= \frac{V}{(2\pi )^3} \dd[3]{p} 
\,.
\end{align}

\todo[inline]{The uncertainty principle argument is not really convincing to me. This is a statement about probability densities, it should hold for classical systems as well\dots}

Since this is the differential \emph{number} of states, we multiply by it to get the \textbf{transition probability}:  
%
\begin{align}
\dd{\omega_{fi}} = \abs{S^{CN}_{fi}}^2 \prod_{j=1}^{n_f} \frac{V \dd[3]{p_j}}{(2\pi )^3 2 \omega_{j}}
\,,
\end{align}
%
where \(p_f = \sum_j p_j\); we also can divide by the decay time \(T\)

\todo[inline]{Ok, but properly speaking shouldn't we differentiate with respect to \(T\) to get the probability per unit time? The result is the same since the expression is linear, but still\dots}

to get 
%
\begin{align}
\dd{\Gamma_{fi}} &= \frac{ \dd{\omega_{fi}}}{T} =
(2\pi )^{4} \delta^{(4)} (p_i - p_f) 
\frac{\abs{\mathcal{M}_{fi}}^2}{2 \omega_{i}}
\prod_{j=1}^{n_f} \frac{\dd[3]{p_j}}{(2\pi )^3 2 \omega_{j}}  \\
&= \frac{\abs{\mathcal{M}_{fi}}^2}{2 \omega_{i}} \dd{\phi}^{(n_f)}
\,,
\end{align}
%
where we define the \(n_f\)-particles \textbf{phase space} element: 
%
\begin{align}
\dd{\phi}^{(n_f)}
=
(2\pi )^{4} \delta^{(4)} (p_i - p_f) 
\prod_{j=1}^{n_f} \frac{\dd[3]{p_j}}{(2\pi )^3 2 \omega_{j}}
\,.
\end{align}

Note that in the continuum limit the differential decay rate \(\dd{\Gamma }_{fi}\) and the phase space element \(\dd{\phi }^{(n_f)}\) are \textbf{finite}. 

The \textbf{total decay rate} is defined as the integral over all of the possible phase space of the differential decay rate: 
%
\begin{align}
\Gamma_{fi} = \int \dd{\Gamma}_{fi}
\,.
\end{align}

This has a direct physical interpretation: it is the probability per unit time that the initial particle will decay into the particles \(f\).

The number of particles \(i\) will then exponentially decay according to 
%
\begin{align}
\dv{N_i}{t} = - \Gamma N_i = - \frac{N_i}{\tau }
\,.
\end{align}

The quantity \(\tau  = 1/ \Gamma \) is called the \textbf{lifetime} of the particle. 

Note that the dimensions of \(\Gamma \) are those of an energy in natural units; it is \emph{not} Lorentz invariant! Under a boost of Lorentz factor \(\gamma \) from the rest frame it transforms like 
%
\begin{align}
\Gamma \to \frac{\Gamma}{\gamma } < \Gamma 
\,,
\end{align}
%
which corresponds to the lifetime \(\tau \) being longer in the boosted frame. 
This corresponds to the observed decay of muons in the usual example of time dilation. 

\subsubsection{Example: differential decay rate with two products}

We put ourselves in the rest frame of the initial particle: so
we have a particle of 4-momentum \((M, 0)\) decaying into two ones with 4-momenta 
%
\begin{align}
q_{j} = (\omega_{j}, \vec{q}_{j}) 
\,,
\end{align}
%
for \(j=1, 2\), and by momentum conservation we have 
%
\begin{align}
M &= \omega_{1} + \omega_2  \\
0 &= \vec{q}_{1} + \vec{q}_{2}
\,.
\end{align}

The masses of the two products are \(m_j\), they are potentially different. 

The general formula for the decay rate in the \emph{rest frame} reads: 
%
\begin{align}
\dd{\Gamma} &= \frac{\abs{\mathcal{M}_{fi}}^2}{2M} \dd{\phi }^{(2)}  \\
&= \frac{\abs{\mathcal{M}_{fi}}^2}{2M} (2 \pi )^{4} \delta^{(4)}(p - q_1 - q_2 ) 
\frac{ \dd[3]{q_1 }}{(2\pi )^3 2 \omega_1}
\frac{ \dd[3]{q_2 }}{(2\pi )^3 2 \omega_2}
\,.
\end{align}

We have six integration variables and four constraints, so really there will be two integrals to do. In general for \(n\) outgoing particles we will have \(3n - 4\) integrals. 

We integrate the \(\dd[3]{q_2 }\) first, as is the convention. This yields 
%
\begin{align}
\dd{\phi}_{(2)}' 
= \frac{(2\pi )^{4}}{4 \omega_1 \omega_2 (2 \pi )^3 } \delta (M - \omega_1 - \omega_2 ) \frac{ \dd[3]{q_1 }}{(2 \pi )^3} 
= \frac{1}{(2\pi )^2} \frac{1}{4 \omega_1 \omega_2 } \delta (M - \omega_1 - \omega_2 ) \dd[3]{q_1 }
\,.
\end{align}

We can now move to angular coordinates for \(q_1 \): we insert 
%
\begin{align}
\dd[3]{q_1 } = \abs{q_1 }^2 \dd{ \abs{q_1 }} \dd{\Omega_1 }
\,,
\end{align}
%
and we have still one integral to do, to remove the energy delta.
Since the energies are fixed if we set \(\abs{q_1 }\), we do that integral: 
%
\begin{align}
\dd{\phi }_{(2)}'' = \frac{1}{(2\pi )^2}
\dd{\Omega_1 }
\int \frac{\delta (M - \omega_1 - \omega_2 )}{4 \omega_1 \omega_2 }\abs{q_1 }^2 \dd{ \abs{q_1 }}
\,,
\end{align}
%
and now we must apply the properties of the \(\delta\) to integrate this: recall that 
%
\begin{align}
\delta (f(x)) = \sum _{x_i \text{ zero of } f(x)}
\frac{ \delta (x- x_i)}{\abs{f'(x_i)}}
\,.
\end{align}

Renaming \(\abs{q_1 } = x\) for convenience, we must rewrite the following delta function: 
%
\begin{align}
\delta \qty(M - \sqrt{m_1^2 + x^2} - \sqrt{m_2^2 + x^2}) = \delta (f(x))
\,.
\end{align}

The derivative of \(f\) is:
%
\begin{align}
\dv{f}{x} = - \frac{x}{\omega_1 } - \frac{x}{\omega_2 } = - x \qty( \frac{1}{\omega_1 } + \frac{1}{\omega_2 }) 
= - x \frac{\omega_1 + \omega_2 }{\omega_1 \omega_2 }
\,.
\end{align}

How many zeroes does \(f\) have in the region which interests us? only one, since it is strictly decreasing. 
Let us call this zero \(\hat{x}\), and denote quantities calculated with \(x = \hat{x}\) with a hat as well. Then we can remove the integral: 
%
\begin{align}
\dd{\phi }_{(2)}'' &= \frac{1}{(2\pi )^2}
\dd{\Omega_1 }
\frac{1}{4 \hat{\omega}_{1} \hat{\omega}_{2}} \frac{\hat{\omega}_{1} \hat{\omega}_{2}}{\hat{x} (\hat{\omega}_{1} + \hat{\omega}_{2})}
\hat{x}^2  \\
&= \frac{1}{(2\pi )^2}
\dd{\Omega_1 }
\frac{1}{4M} \hat{x} 
= \frac{1}{16 \pi^2} \frac{\abs{\hat{q}_{1}}}{M} \dd{\Omega_1 }
\,,
\end{align}
%
since the energies of the two outgoing particles always add to \(M\). 

The explicit expression for \(x\) in terms of the masses is 
%
\begin{align}
x = \frac{\sqrt{-2 M^2 m_1^2-2 M^2 m_2^2+M^4-2 m_1^2 m_2^2+m_1^4+m_2^4}}{2 M}
\,.
\end{align}
%


\end{document}