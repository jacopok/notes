\documentclass[main.tex]{subfiles}
\begin{document}

\section{QED process at the lowest order}

\marginpar{Wednesday\\ 2020-6-17, \\ compiled \\ \today}

We will now study \textbf{tree level} processes in QED: this means that we only consider QED processes which have \emph{no loops}.
This usually corresponds to considering the leading order. 

We will look at three processes explicitly: 
\begin{enumerate}
    \item \(e^{+} e^{-} \to \mu^+ \mu^-\) scattering;
    \item \(e^{-}\) scattering by an external EM field;
    \item \(e^{-} \gamma \to e^{-} \gamma \) (Compton) scattering.
\end{enumerate}

The techniques we will apply here will generalize to any tree-level QED process. 

\subsection{Two-flavoured QED Lagrangian}

In order to consider the \(e^{+} e^{-} \to \mu^+ \mu^-\) process we need a QFT description of the muon as well as the electron: so, we write a Lagrangian with both terms. It reads: 
%
\begin{align}
\mathscr{L}^{(2)}_{\text{QED}} = - \frac{1}{4} F^{\mu \nu } F_{\mu \nu }
- \frac{1}{2 \xi } \qty(\partial_{\mu } A^{\mu })^2
+ \overline{\psi}_{e} \qty(i \slashed{\DD}_e - m_e) \psi_{e}
+ \overline{\psi}_{\mu} \qty(i \slashed{\DD}_\mu - m_\mu) \psi_{\mu}
\,,
\end{align}
%
where the covariant derivative associated with the particle species \(j\) contains the charge of that particle, \(q_j\):
%
\begin{align}
\slashed{\DD}_j = \gamma^{\mu } \qty(\partial_{\mu } + i q_j A_{\mu })
\,.
\end{align}

This holds in general, but in our case \(q_e = q_\mu = -e\). 
The Lagrangian \(\mathscr{L}^{(2)} _{\text{QED}}\) has a \(U(1)\) gauge symmetry, whose action is 
%
\begin{align}
A^{\mu }(x) & \to A^{\mu }(x ) - \partial_{\mu } \alpha (x)  \\
\psi_{e, \mu }(x) & \to e^{iq \alpha (x)} \psi_{e, \mu }(x)
\,;
\end{align}
%
and a \(U(1)_{e} \times U(1)_{\mu }\) global symmetry, whose action is 
%
\begin{align}
A^{\mu }(x) &\to A^{\mu }(x)  \\
\psi_{e, \mu } (x ) & \to e^{i \beta_{{e, \mu }}} \psi_{e, \mu }
\,,
\end{align}
%
where \(\beta_{e, \mu }\) are two (possibly different) constants. 

\subsection{\(e^{+}e^{-} \to f^{+} f^{-}\) unpolarized scattering}

Here, \(e^{\pm}\) are electron and positron, while \(f\) is a generic spin \(1/2\) fermion. 

We make the assumption that the couplings are \textbf{diagonal} in flavour space: this means that there is no term in the Lagrangian which couples \(e\) and \(f\) explicitly, instead they are both coupled to the EM field and interact in that way. 

\todo[inline]{Is this the correct interpretation?}

Then, the QED vertex for this interaction looks like 
%
\begin{align}
\feynmandiagram[baseline=(v.base), horizontal=v to p]{
    f1 [particle = {\(f^{-}\)}] -- [fermion] v [label= left : \(\mu \)],
    f2 [particle = {\(f^{+}\)}] -- [anti fermion] v,
    v -- [photon] p [particle = \(\gamma \)]
};
= - i q_f \gamma^{\mu }
\,.
\end{align}

This holds for the generic fermion \(f\), so for the electron as well. 
So, if \(f \neq e\) we only have one diagram contributing to this interaction, at second order and at tree level: 
\begin{figure}[ht]
\centering
\feynmandiagram[layered layout, extra large, horizontal = ia to ib]{
e1 [particle = {\(e^{-}_{r}\)}] -- [fermion, momentum = \(\vec{p}\)] ia [label = -45: \(-i q_e \gamma_{\mu }\)] -- [photon, momentum = {\(\vec{k}= \vec{p}+\vec{q}\)}] ib [label = 235 : {\(-iq_f \gamma_{\nu }\)}] -- [fermion, momentum = {\(\vec{p}'\)}] e2 [particle = {\(f^{-}_{r'}\)}],
p1 [particle = {\(e^{+}_{s}\)}] -- [anti fermion, momentum' = \(\vec{q}\)] ia,
ib -- [anti fermion, momentum' = \(\vec{q}'\)] p2 [particle = {\(f^{+}_{s'}\)}]
};
\caption{Tree level diagram for electron-fermion interaction.}
\label{fig:electron-fermion-tree-level}
\end{figure}

Also, we denote the mass of the electron as \(m\) and the mass of the other fermion as \(M\).

The whole diagram can just as well be flipped: the incoming and outgoing direction of the momenta is arbitrary. 
Because of momentum conservation, \(\vec{k} = \vec{p} + \vec{q} = \vec{p}' + \vec{q}'\).

The Feynman amplitude is given by the Feynman rules: 
%
\begin{align}
\mathcal{M}_{ef} = (-i q_f) (-i q_e)
\overline{u}_{r'} (p') \gamma_{\nu } v_s' (q')
\overline{v}_{s} (q) \gamma_{\mu} u_r (p)
\widetilde{D}^{\mu \nu } (k)
\,,
\end{align}
%
where the photon propagator in the gauge-fixing Lagrangian is given by 
%
\begin{align}
\widetilde{D}^{\mu \nu } (k) = \frac{-i}{k^2 + i \epsilon }
\qty(\eta^{\mu \nu } - \frac{k^{\mu }k^{\nu }}{k^2} (1 - \xi ))
\overset{\xi =1}{=}
\frac{-i}{k^2 + i \epsilon }
\eta^{\mu \nu }
\,.
\end{align}

Observables do not depend on the choice for \(\xi \), so we can set it to one without worry. 

\todo[inline]{But the EOM for the EM field depend on \(\xi \) explicitly!}

Also, we will now start implying the \(+i \epsilon \) term.

Now, in order to write the Feynman amplitude more explicitly we introduce the Mandelstam variables: 
%
\begin{align}
s &= (p+q)^2 = (p' + q')^2 \\
t &= (p-p')^2 = (q - q')^2 \\
u &= (p-q')^2 = (q - p')^2 
\,,
\end{align}
%
which are a useful covariant scalar description of the kinematics. 
With these conventions, the Feynman amplitude reads: 
%
\begin{align}
\mathcal{M}_{ef} = \frac{i q_e q_f}{s} 
\overline{u}_{r'} (p') \gamma^{\mu } v_s' (q')
\overline{v}_{s} (q) \gamma_{\mu} u_r (p)
\,,
\end{align}
%
since \(k^2 = s\), as \(k = p+q\).
Now the real matrix-crunching starts: we want to find \(\abs{\mathcal{M}}^2 = \mathcal{M} \mathcal{M}^{*}\), which will be needed in the computation of the cross section, so we make it explicit: 
%
\begin{align}
\abs{\mathcal{M}}^2 = \qty(\frac{q_e q_f}{s})^2 
\qty(\overline{u} (p') \gamma^{\mu } v (q')) \qty(\overline{v}(q) \gamma_{\mu } u(p)) 
\qty(\overline{u} (p') \gamma^{\nu } v (q'))^{*} \qty(\overline{v}(q) \gamma_{\nu } u(p))^{*}
\,,
\end{align}
%
which we can manipulate: consider the conjugate bits
%
\begin{align}
\qty(\overline{u} (p') \gamma^{\nu } v (q'))^{*} 
&= u_{\alpha }(p') \qty( \gamma^{0 *} \gamma^{\nu *})_{\alpha \beta } v^{*}_{\beta } (q')  \\
&= v^{*}_{\beta }(q')\qty( \gamma^{0 *} \gamma^{\nu *} \gamma^{0} \gamma^{0})_{\alpha \beta }
u_\alpha (p')    \\
&= v^{*}_{\beta }(q')\qty( \gamma^{\nu \top} \gamma^{0})_{\alpha \beta }
u_\alpha (p')  \\
&= v^{*}_{\beta } (q') (\gamma^{0 \top} \gamma^{\nu })_{\beta \alpha } u_\alpha (p')  \\
&= \overline{v}_{\beta } (q') \gamma^{\nu  } u_\alpha (p')
\,,
\end{align}
%
where we used the identity \eqref{eq:gamma-matrices-identity}, noticing the fact that it can be adapted into 
%
\begin{align}
\gamma^{0} \gamma^{\nu , \top} \gamma^{0} = \gamma^{\nu *}
\,,
\end{align}
%
since \(\gamma^{0} = \gamma^{0 \dag} = \gamma^{0 \top} = \gamma^{0 *}\). The symbol \(\top\) means transpose, so that \(\top * = \dag\). Also, \(\qty(\gamma^{0})^2 = \mathbb{1}\) so a pair of them can be inserted anywhere.

This all means that our Dirac spinor construction is well-behaved under conjugation: \((\overline{u} \gamma v)^{*} = \overline{v} \gamma u\), as it should be. 
This means that we can write the probability as: 
%
\begin{align}
\abs{\mathcal{M}}^2 = \qty(\frac{q_e q_f}{s})^2 
\underbrace{\qty(
\overline{u}_{r'}(p') \gamma^{\mu } v_{s'}(q') \overline{v}_{s'} (q') \gamma^{\nu } u_{r'}(p')
)}_{\text{muon}}
\underbrace{\qty(
\overline{v}_{s}(q) \gamma_{\mu} u_r (p) \overline{u}_{r}(p) \gamma_{\nu } v_s(q)
)}_{\text{electron}}
\,.
\end{align}

We want to consider the \textbf{unpolarized} \(\abs{\mathcal{M}}^2\), so we need to average over the 4 initial polarizations (\(r, s\)), and sum over the final polarizations (\(r', s'\)).
The factor \(4\) is calculated as \((2s_{e^{-}}+1) (2s_{e^{+}}+1)\). 

We make the spinorial indices explicit and find: 
%
\begin{align}
\abs{\mathcal{M}}^2 = 
\qty( \frac{q_e q_f}{2 s})^2 
\underbrace{\sum _{r'} u_{r'}^{ \delta'} \overline{u}^{\alpha '}_{r'}
\sum _{s'} v_{s'}^{ \beta'} \overline{v}^{\gamma '}_{s'}}_{\text{muon, mass }M}
(\gamma^{\mu })_{ \alpha ' \beta '}
(\gamma^{\nu })_{ \gamma ' \delta '}
\underbrace{\sum _{r} u_{r}^{ \delta} \overline{u}^{\alpha }_{r}
\sum _{s} v_{s}^{ \beta} \overline{v}^{\gamma }_{s}
}_{\text{electron, mass }m}
(\gamma_{\mu })_{ \alpha  \beta }
(\gamma_{\nu })_{ \gamma  \delta }
\,.
\end{align}

Now, objects in the form \(\sum _{r} u_r \overline{u}_{r}\) are \textbf{projectors} in spinor space: specifically, they are the ones defined in equation \eqref{eq:energy-projectors-spinor} (without the mass in the denominator). 
So, we can write the ones corresponding to the electron as: 
%
\begin{align}
\sum _{r} u_r (p) \overline{u}_r (p) &= 2 m \Lambda_{+} (p) = \slashed{p} + m \\
\sum _{s} v_s (p) \overline{v}_s (p) &= -2 m \Lambda_{-} (p) = \slashed{p} - m
\,,
\end{align}
%
and similarly for the ones of the other fermion.
So, writing all the terms in order we find: 
%
\begin{align}
\begin{split}
\abs{\mathcal{M}}^2 &= 
\qty( \frac{q_e q_f}{2 s})^2 
(\gamma^{\mu })_{ \alpha ' \beta '}
(\slashed{q}' - M)_{\beta ' \gamma '}
(\gamma^{\nu })_{ \gamma ' \delta '}
\qty(\slashed{p}' + M)_{ \delta ' \alpha '} \\
&\phantom{=}\ 
(\gamma_{\mu })_{ \alpha  \beta }
(\slashed{q} - m)_{\beta  \gamma }
(\gamma_{\nu })_{ \gamma  \delta }
\qty(\slashed{p} + m)_{ \delta  \alpha }
\end{split}  \\
&= 
\qty( \frac{q_e q_f}{2 s})^2 
\Tr [\gamma^{\mu } (\slashed{q}' - M) \gamma^{\nu } (\slashed{p}' + M)]
\Tr [\gamma_{\mu } (\slashed{q} - m) \gamma_{\nu } (\slashed{p} + m)]
\,,
\end{align}
%
so we have the result: summing over the initial and final polarizations corresponds to taking a trace over the fermionic lines. 

This makes sense: there are no free spinorial indices in the probability, so we must trace them all away. 

\todo[inline]{He then makes an example in which we have a fermionic loop which yields a trace, but this was not the case here... How do these connect? Do we need a loop or not?}

This is similar to the Feynman rules: instead of the propagator \(\widetilde{S}(p)\) we plug in the projector \(\Lambda_{\pm }(p)\). 

\begin{claim}
The traces are equal to: 
%
\begin{align}
\Tr [\gamma_{\mu } (\slashed{q} - m) \gamma_{\nu } (\slashed{p} + m)] &= 
4 \qty[\qty(p_{\mu } p_{\nu } + q_{\mu } p_{\nu }) - (m^2 + p \cdot q) \eta_{\mu \nu}] \\
\Tr [\gamma^{\mu } (\slashed{q}' - M) \gamma^{\nu } (\slashed{p}' + M)] 
&= 4 \qty[\qty(p_{\mu }' p_{\nu }' + q_{\mu }' p_{\nu }') - (M^2 + p' \cdot q') \eta_{\mu \nu}] 
\,.
\end{align}
\end{claim}

\begin{proof}
We do the first one (so we do not have to write many primes), the other one is precisely analogous. 
We make the product explicit, using the linearity of the trace:
%
\begin{align}
\Tr [\gamma_{\mu } (\slashed{q} - m) \gamma_{\nu } (\slashed{p} + m)]
&= \Tr[ \gamma_{\mu } \slashed{q} \gamma_{\nu } \slashed{p}]
-m\Tr[\gamma_{\mu } \gamma_{\nu } \slashed{p}]
+m\Tr[\gamma_{\mu } \slashed{q} \gamma_{\nu }]
-m^2 \Tr[\gamma_{\mu } \gamma_{\nu }]
\,.
\end{align}

Now, we have terms with two, three and four gamma matrices. 

\begin{claim}
The trace of an odd number of gamma matrices vanishes.
\end{claim}

\begin{proof}
Recall the definition of the matrix \(\gamma5 \) \eqref{eq:gamma5-definition}. Its square is the identity: so, we can insert it anywhere. So, let us take: 
%
\begin{align}
\Tr[ \prod_{i=1}^{2n+1} \gamma^{\mu_{i}} ] &= \Tr[\gamma^{5} \gamma^{5}\prod_{i=1}^{2n+1} \gamma^{\mu_{i}}]  \\
&= \Tr[\gamma^{5}\prod_{i=1}^{2n+1} \gamma^{\mu_{i}} \gamma^{5}] \marginnote{Cyclicity of the trace.}  \\
&= (-)^{2n+1} \Tr[\gamma^{5} \gamma^{5}\prod_{i=1}^{2n+1} \gamma^{\mu_{i}} ]  \marginnote{Anticommutation, \(\qty{\gamma^{5}, \gamma^{\mu }} = 0\).}\\
&= (-)^{2n+1} \Tr[\prod_{i=1}^{2n+1} \gamma^{\mu_{i}} ] \marginnote{Cyclicity of the trace.}
\,.
\end{align}

So, the trace is equal to minus itself, so it vanishes. 
\end{proof}

With this, we can throw away the terms with an odd number of \(\gamma\) matrices: so, we are left with 
%
\begin{align}
\Tr [\gamma_{\mu } (\slashed{q} - m) \gamma_{\nu } (\slashed{p} + m)]
&= \Tr[ \gamma_{\mu } \slashed{q} \gamma_{\nu } \slashed{p}]
-m^2 \Tr[\gamma_{\mu } \gamma_{\nu }]
\,,
\end{align}
%
and we can evaluate the two pieces separately. 
For the second, we have 
%
\begin{align}
\Tr[\gamma_{\mu } \gamma_{\nu }] &= \Tr[\qty{\gamma_{\mu }, \gamma_{\nu }}] - \Tr[\gamma_{\nu } \gamma_{\mu }]  \\
&= 2 \eta_{\mu \nu } \Tr[\mathbb{1}]
- \Tr[\gamma_{\mu } \gamma_{\nu }] \marginnote{Cyclicity of the trace.}  \\
2 \Tr[\gamma_{\mu }\gamma_{\nu }] &= 2 \eta_{\mu \nu } \times 4 \\
\Tr[\gamma_{\mu }\gamma_{\nu }] &= 4 \eta_{\mu \nu } 
\,.
\end{align}

For the first, instead, we use the following:
\begin{claim}[Unproven]
\begin{align}
\Tr[\gamma_{\mu } \gamma_{\nu }\gamma_{\rho } \gamma_{\sigma }]
= 4 \qty(\eta_{\mu \nu } \eta_{\rho \sigma }
+ \eta_{\mu \sigma } \eta_{\nu \rho }
- \eta_{\mu \rho }\eta_{\nu \sigma })   
\,.
\end{align}
\end{claim}

Combining these, we get: 
%
\begin{align}
\Tr[ \gamma_{\mu } \slashed{q} \gamma_{\rho } \slashed{p}]
-m^2 \Tr[\gamma_{\mu } \gamma_{\rho }]
&= q^{\nu }p^{\sigma } \Tr[\gamma_{\mu } \gamma_{\nu }\gamma_{\rho } \gamma_{\sigma }]
- m^2 \Tr[\gamma_{\mu } \gamma_{\rho }]  \\
&= 4 q^{\nu }p^{\sigma } 
\qty(\eta_{\mu \nu } \eta_{\rho \sigma }
+ \eta_{\mu \sigma } \eta_{\nu \rho }
- \eta_{\mu \rho }\eta_{\nu \sigma })
- 4 m^2 \eta_{\mu \rho }  \\
&= 4 \qty(q_{\mu } p_{\rho } + p_{\mu }q_{\rho } - \eta_{\mu \rho } \qty( p \cdot q + m^2) )
\,.
\end{align}
\end{proof}

The masses are \(p^2 = q^2 = m^2\) and \(p^{\prime 2} = q^{\prime 2} = M^2\); in certain cases (like in electron-muon decay) we can assume that since \(M \gg m\) the mass \(m\) is negligible: \(m \approx 0\). 

With this simplification, several terms cancel and we find: 
%
\begin{align}
\abs{\mathcal{M}}^2 \approx \qty(\frac{8 q_e^2 q_{\mu}^2}{s^2})
\qty[(p \cdot p') (q \cdot q') + (p \cdot q')(q \cdot p') + M^2 (p \cdot q)]
\,.
\end{align}

\begin{claim}
If, instead, we do not want to neglect \(m\) we find: 
%
\begin{align}
\abs{\mathcal{M}}^2 = \qty(\frac{8 q_e^2 q_{\mu}^2}{s^2})
\qty[(p \cdot p') (q \cdot q') + (p \cdot q')(q \cdot p') + M^2 (p \cdot q) 
+ m^2 (p' \cdot q') + 2 m^2M^2]
\,.
\end{align}
\end{claim}

\begin{proof}
\todo[inline]{To do.}
\end{proof}

Everything we have derived so far is Lorentz-invariant, and the probability is a Lorentz scalar. 
However, we can specify these results to a specific frame in order to make them more concretely useful. 
We choose the center of mass frame. 

\subsubsection{COM frame}

In this frame, the momenta of the electron and positron will look like: 
%
\begin{align}
p = (\omega, \vec{p}) 
\qquad \text{and} \qquad
q = (\omega, -\vec{p})
\,,
\end{align}
%
while the ones of the other two fermions will be 
%
\begin{align}
p' = (\omega, \vec{p}') 
\qquad \text{and} \qquad
q' = (\omega, -\vec{p}')
\,.
\end{align}

The Mandelstam variable \(s\) is equal to \(4 \omega^2\). 
Notice that the energies are equal even though the mass of the incoming particles, \(m\), is possibly different from the one of the outgoing ones, \(M\).  

Let us write out the scalar products between the momenta explicitly: 
%
\begin{align}
p \cdot p' &= \omega^2 - \abs{\vec{p}} \abs{\vec{p}}' \cos \theta \overset{m \approx 0}{\approx} \omega ( \omega - \abs{\vec{p}}' \cos \theta ) \\
p \cdot q' &= \omega^2 + \abs{\vec{p}} \abs{\vec{p}}' \cos \theta \overset{m \approx 0}{\approx} \omega ( \omega + \abs{\vec{p}}' \cos \theta )  \\
p \cdot q &= \omega^2 + \abs{\vec{p}}^2 \overset{m \approx 0}{\approx}
2\omega^2 \marginnote{Used \(\abs{\vec{p}} \approx \omega \).} \\
p' \cdot q' &= \omega^2 + \abs{\vec{p}'}^2 = 2 \omega^2 - M^2 
\marginnote{Used \(\abs{\vec{p}'}^2 = \omega^2-M^2\).}
\,.
\end{align}

This tells us that the primed-frame momentum is given by 
%
\begin{align}
\abs{\vec{p}'}^2 = \omega^2 \qty(1 - \frac{M^2}{\omega^2})
\,.
\end{align}
%
\todo[inline]{The last one is exact! He writes that it is approximated\dots}

Substituting these in, in the \(m \approx 0 \) approximation we find: 
%
\begin{align}
\abs{\mathcal{M}}^2_{CM} &\approx \frac{8 q_e^2 q_f^2}{s^2}
\qty[
2 \omega^{4} 
+ 2 \omega^2 \abs{\vec{p}'}^2 \cos^2\theta + 2 M^2\omega^2
] \\
&\approx e^{4} \qty[
\qty(1 + \frac{M^2}{s}) +   \qty(1 - \frac{4M^2}{s}) \cos^2\theta 
]
\,.
\end{align}

We can substitute this into the expression we know for the differential cross section \eqref{eq:zero-initial-mass-cross-section}: we find 
%
\begin{align}
\dv{\sigma }{\Omega_1 }
&\approx
\frac{1}{64 \pi^2}
\frac{\abs{\vec{p}'}}{\abs{\vec{p}}}
\abs{\mathcal{M}}^2_{CM}  \\
&= \frac{\alpha^2}{4 s} \sqrt{1 - \frac{4M^2}{s}}
\qty[ \qty(1 + \frac{M^2}{s}) +   \qty(1 - \frac{4M^2}{s}) \cos^2\theta ]
\,,
\end{align}
%
where we have introduced \(\alpha_{EM} = e^2 / 4 \pi^2\). 

\begin{enumerate}
    \item This expression is well-defined as long as \(\sqrt{s} > 2M\), which reminds us that if this is not the case the process is kinematically not allowed. 
    \item The \(\cos^2\theta \) term means we have a maximum at the poles \(\theta = 0, \pi \) (that is, the directions from which the electrons were coming) and a minimum at the equator, \(\theta = \pi /2\). This is similar to other vector-fermion interactions. 
    \item Angular momentum is conserved: two spin \(1/2\) particles become one spin-1 one. 
    \item In the ultrarelativistic limit \(\sqrt{s} \gg M, m\) we have 
    %
    \begin{align}
    \dv{\sigma }{\Omega_1 } \approx \frac{\alpha^2}{4 s} \qty(1 + \cos^2 \theta )
    \qquad \text{and} \qquad
    \sigma = \frac{4 \pi }{3} \frac{\alpha^2}{s} 
    \,.
    \end{align}
\end{enumerate}

\subsubsection{Angular momentum conservation}

In order to get angular momentum conservation we need to calculate the polarized cross sections --- those in which we do not average over the polarizations, but instead keep them fixed.

The ones in which angular momentum would not be conserved vanish. The ones which are nonzero in the \emph{ultrarelativistic limit} are
%
\begin{align}
\abs{\mathcal{M}(-+, - + )}^2 &= \abs{\mathcal{M}_{LL}}^2 = 
e^{4} \qty(1 + \cos \theta )^2 \\
\abs{\mathcal{M}(+ -, + -)}^2 &= \abs{\mathcal{M}_{RR}}^2 = 
e^{4} \qty(1 + \cos \theta )^2 \\
\abs{\mathcal{M}(-+, + -)}^2 &= \abs{\mathcal{M}_{LR}}^2 = 
e^{4} \qty(1 - \cos \theta )^2 \\
\abs{\mathcal{M}(+ -, - + )}^2 &= \abs{\mathcal{M}_{RL}}^2 = 
e^{4} \qty(1 - \cos \theta )^2 
\,,
\end{align}
%
where the four signs denote the spins of the incoming and outgoing particles.

\begin{figure}[ht]
\centering
\includegraphics[width=\textwidth]{figures/angular_distribution.png}
\caption{Angular distribution: the dotted line is \(LL\) or \(RR \), the dashed line is \(LR\) or \(RL\). }
\label{fig:angular_distribution}
\end{figure}

We can see in figure \ref{fig:angular_distribution} that in all the cases \(\theta = 0\) and \(\theta = \pi \) correspond to either zero or the maximum probability. 
For the spin-preserving interactions, we have that \(\theta = \pi \) is forbidden, for the spin-flipping ones \(\theta = 0\) is. 

In order to understand this, let us consider the first of the two: we have two incoming electrons with opposite helicities, so their spins will be aligned. 
If, say, we measure spin with respect to the \(z\) axis which is aligned with \(e^{-}\)'s trajectory, we will have \(s_z = -1/2 - (+ 1/2) = -1\) for the total initial spin if the initial spin configuration is \(+ -\).

If the final spin configuration is \(- +\) and the angle is \(\pi \), that means that the fermion \(f^{-}\) with helicity \(-\) is going along positive \(z\), and the fermion \(f^{+}\) with helicity \(+\) is going along negative \(z\), so the total spin will be \(s_z = +1\). 

\todo[inline]{Hold on, something is not super clear: are these quantum numbers actually helicities?}

The unpolarized probability is found by adding all of the \(\abs{\mathcal{M}}^2\) together. 

\begin{claim}
The unpolarized cross sections for the processes \(e^{-} f \to e^{-} f\) and \(e^{-} \overline{f} \to e^{-} \overline{f}\) in the ultrarelativistic limit are respectively: 
%
\begin{align}
\dv{\sigma }{\Omega_1 }
&= \frac{\alpha^2}{4 t} \qty(1 + \cos^2 \theta ) \\
\dv{\sigma }{\Omega_1 }
&= \frac{\alpha^2}{4 u} \qty(1 + \cos^2 \theta )
\,.
\end{align}
\end{claim}

\begin{proof}
The Feynman diagram is the same (rotated by \(\pi /2\) if we follow the conventions), only the kinematics change. 
So, the momentum conservation equation which was 
%
\begin{align}
p + q = p' + q'
\,,
\end{align}
%
now should be written as 
%
\begin{align}
p - p' = q' - q 
\qquad \text{or} \qquad
p - q' = p' - q
\,,
\end{align}
%
where either \(-p'\) or \(-q'\) is the ``new second incoming momentum''. This is equivalent to switching to the other Mandelstam variables.
\end{proof}

\end{document}