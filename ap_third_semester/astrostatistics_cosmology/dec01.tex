\documentclass[main.tex]{subfiles}
\begin{document}

\subsection{Galaxy surveys}

\marginpar{Tuesday\\ 2020-12-1, \\ compiled \\ \today}

Nowadays we can have 3D surveys, which make use of photometric and spectroscopic data to give distance information. 
The galaxy density field we observe is non-Gaussian, and we would expect it to be so even starting from Gaussian initial conditions since gravitational evolution is nonlinear. 

Therefore, not all the information lies in the power spectrum. 
This means that it is hard to do the analysis on arbitrary scales; on the other hand if we average over \(\gtrsim \SI{10}{Mpc}\) the field is indeed close to Gaussian, so we can analyze it in a simpler way.
This is due to the hierarchical nature of the gravitational collapse, which is ``bottom up''. 

So, on these large scales we can linearize. 
We work in Fourier space, defining: 
%
\begin{align}
\delta _g (k \ll 1) = \int \dd[3]{x} \delta _g (\vec{x}) W_\lambda e^{-i \vec{k} \cdot \vec{x}}
\,,
\end{align}
%
where \(W_\lambda \) is our filter function, which has a wavelength of \(\gtrsim \SI{10}{Mpc}\).

The power spectrum is defined as 
%
\begin{align}
\expval{ \delta _g (\vec{k}_1) \delta _g^{*} (\vec{k}_2)} = (2 \pi )^3
\delta^{(3)} (\vec{k}_1 - \vec{k}_2) P_g (k)
\,,
\end{align}
%
where \(k = \abs{\vec{k}_1} = \abs{\vec{k}_2}\).

The density will really be defined as 
%
\begin{align}
\Delta _i = \int \dd[3]{x} \psi _i (\vec{x}) \qty(\frac{n(\vec{x}) - \overline{n}}{\overline{n}})
\,.
\end{align}

The function \(\psi \) can be chosen in real space to be the count-in-cells one: 
%
\begin{align}
\psi^{\text{CIC}}_{i} (\vec{x}) = \overline{n}  [\vec{x} \in i \text{-th cell}]
\,,
\end{align}
%
or the Fourier one: 
%
\begin{align}
\psi_{\vec{k}} (\vec{x}) = \frac{1}{V} e^{i \vec{k} \cdot \vec{x}} [\vec{x} \in V]
\,.
\end{align}

In the latter case, we get 
%
\begin{align}
\Delta (\vec{k}) = \frac{1}{V \overline{n}} \int \dd[3]{x} e^{i \vec{k} \cdot \vec{x}} n(\vec{x}) - \frac{1}{V} \underbrace{\int \dd[3]{x} e^{i \vec{k} \cdot \vec{x}}}_{ =\delta_{\vec{k}, 0} } 
\,,
\end{align}
%
where \(\delta_{\vec{k}, 0}\) is a Kronecker delta between \(\vec{k}\) and \(0\). 
From \(\Delta \) we want to compute the power spectrum. 

The galaxy density field can be written as 
%
\begin{align}
n(\vec{x}) = \sum _i \delta^{(3)} (\vec{x} - \vec{x}_i)
\,,
\end{align}
%
therefore 
%
\begin{align}
\Delta (\vec{k}) = \frac{1}{\overline{n} V} \sum _{i} e^{i \vec{k} \cdot \vec{x}_i} - \delta_{\vec{k}, 0}
\,.
\end{align}

We make a grid that is so fine that each cell contains at most 1 galaxy: then, we can write 
%
\begin{align}
\Delta (\vec{k}) = \frac{1}{\overline{n} V} \sum _{\alpha }^{N _{\text{cells}}} N_\alpha  e^{i \vec{k} \cdot \vec{x}_\alpha } - \delta_{\vec{k}, 0}
\,,
\end{align}
%
where \(N_\alpha = 0, 1\) for each \(\alpha \). 

Then, we can compute the power spectrum as 
%
\begin{align}
\begin{split}
\expval{\Delta (\vec{k}_1) \Delta^{*} (\vec{k}_2)} &= \frac{1}{(\overline{n} V)} \sum _{\alpha , \beta } \expval{N_\alpha N_\beta } e^{i \vec{k}_1 \cdot \vec{x}_\alpha  - i \vec{k}_2 \cdot \vec{x}_\beta } + \delta_{\vec{k}_1, 0} \delta_{\vec{k}_2, 0 } \\
&\phantom{=}\ 
- \frac{1}{\overline{n} V} \delta_{\vec{k}_2, 0} \sum _{\alpha } \expval{N_\alpha } e^{i \vec{k}_1 \cdot \vec{x}_\alpha }
- \frac{1}{\overline{n} V} \delta_{\vec{k}_1, 0} \sum _{\beta } \expval{N_\beta } e^{i \vec{k}_2 \cdot \vec{x}_\beta  }
\,.
\end{split}
\end{align}

Now, the last two terms can be computed making use of the fact that in a small cell of volume \(\delta V\) the expected number of galaxies is \(\overline{n} \delta V\); therefore we can write 
%
\begin{align}
\frac{1}{\overline{n} V} \delta_{\vec{k}_2, 0} \sum _{\alpha } \expval{N_\alpha } e^{i \vec{k}_1 \cdot \vec{x}_\alpha }
&= \frac{1}{\overline{n} V} \delta_{\vec{k}_2, 0} \sum _{\alpha } \overline{n} \delta V e^{i \vec{k}_2 \cdot \vec{x}_\alpha }  \\
&\approx \delta_{\vec{k}_2, 0} \qty( \frac{1}{V} \int \dd[3]{x} e^{i \vec{k}_1 \cdot \vec{x}_\alpha }) = \delta_{\vec{k}_2, 0} \delta_{\vec{k}_1, 0} 
\,,
\end{align}
%
which means that 
%
\begin{align}
\expval{ \Delta (\vec{k}_1) \Delta^{*} (\vec{k}_2)} = \frac{1}{(\overline{n} V)^2} \sum _{\alpha , \beta } \expval{N_\alpha N_\beta } e^{i \vec{k}_1 \cdot \vec{x}_\alpha - i \vec{k}_2 \cdot \vec{x}_\beta } - \delta_{\vec{k}_1, 0} \delta_{\vec{k}_2, 0}
\,.
\end{align}

If the positions of galaxies were purely random and unclustered we would have 
%
\begin{align}
\expval{N_\alpha N_\beta } = \expval{N_\alpha } \expval{N_\beta } =(\overline{n}V)^2
\,,
\end{align}
%
so we define 
%
\begin{align}
\expval{N_\alpha N_\beta } = (\overline{n} \delta V)^2 \qty(1 + \expval{ \delta _g (\vec{x}_1) \delta _g (\vec{x}_2)} )
= (\overline{n} \delta V)^2 \qty(1 + \xi _{12} (\abs{\vec{x}_1 - \vec{x}_2}))
\,.
\end{align}

Also, we know that the autocorrelation can be written through the fact \(N_\alpha = N_\alpha^2\), since \(N_\alpha = 0, 1\). 
Therefore, 
%
\begin{align}
\expval{ \Delta(\vec{k}_1 ) \Delta^{*} (\vec{k}_2)} = \frac{1}{(\overline{n} V)} \qty(\sum _{\alpha } \expval{N_\alpha } e^{i (\vec{k}_1 - \vec{k}_2) \cdot \vec{x}_\alpha } + \sum _{\alpha \neq \beta } \expval{N_\alpha N_\beta } e^{i \vec{k}_1 \cdot \vec{x}_\alpha - i \vec{k}_2 \cdot \vec{x}_\beta })
\,,
\end{align}
%
where the cross-term is given by 
%
\begin{align}
\frac{1}{(\overline{n} V)^2} 
\sum _{\alpha \neq \beta } \expval{N_\alpha N_\beta } e^{i \vec{k}_1 \cdot \vec{x}_\alpha - i \vec{k}_2 \cdot \vec{x}_\beta }
= \expval{ \delta _g (\vec{k}_1) \delta _g^{*} (\vec{k}_2)} = \frac{1}{V } P_g(k_1) \delta_{\vec{k}_1, \vec{k}_2}
\,,
\end{align}

The final term represents shot noise, so this finally yields 
%
\begin{align}
\expval{ \Delta(\vec{k}_1 ) \Delta^{*} (\vec{k}_2)}
= \frac{1}{V} \qty(P_g (k) + \frac{1}{\overline{n}}) \delta_{\vec{k}_1, \vec{k}_2}
\,.
\end{align}

The shot noise term would be there even for a pure random process, while the \(P_g (k)\) is our signal. 

[Argument through characteristic functions as to why the total number of galaxies in each volume is Poisson distributed.]

For a small volume 
%
\begin{align}
Z_s(k) = 1 + \rho \delta V (e^{ik} - 1)
\,,
\end{align}
%
so for a large volume 
%
\begin{align}
Z_l(k) = \qty[Z_s(k)]^{V / \delta V} \approx \exp[\rho V \qty(e^{ik}- 1)]
\,,
\end{align}
%
therefore 
%
\begin{align}
Z_l(k)  = \sum _{n=0}^{\infty } \frac{\lambda^{n} e^{-\lambda }}{n!} e^{ikn}
\,,
\end{align}
%
where \(\lambda = \rho V\); therefore the probability can be recovered as an inverse Fourier transform:
%
\begin{align}
P_n = \frac{e^{-\lambda } \lambda^{n}}{n!}
\,.
\end{align}

Then, we can compute the Fisher matrix from our general expression: 
the process is zero mean, so the mean-dependent terms cancel, so the expression is 
%
\begin{align}
F_{\alpha \beta } = \frac{1}{2} C_{ij, \alpha } C^{-1}_{jm} C_{mn, \beta } C^{-1}_{ni}
\,,
\end{align}
%
where the covariance matrix and its inverse are diagonal:
%
\begin{align}
C_{ij} &= C(\vec{k}_i, \vec{k}_j) = \delta_{ij} \qty[P_g (k_i) + \frac{1}{\overline{n}}]  \\
C^{-1}_{ij} &= \frac{ \delta_{ij}}{P_g(k) + 1/ \overline{n}}
\,.
\end{align}

The derivative of the covariance matrix is 
%
\begin{align}
C_{ij, \alpha } = \pdv{C_{ij}}{P_\alpha } = \delta_{ij} \pdv{P_g(k_i)}{P_\alpha } = \delta_{ij} d_{i \alpha }
\,,
\end{align}
%
where \(d_{i \alpha } = [\abs{\vec{k}_i} \in [k_\alpha, k_\alpha + \Delta k]]\) (using the Iverson bracket).

This then yields 
%
\begin{align}
F_{\alpha \beta } &= \frac{1}{2} \delta_{ij} d_{i \alpha } \frac{ \delta_{im}}{P_g (k_j) + 1/ \overline{n}} ( \delta_{mn} d_{n \beta }) \frac{ \delta_{ni}}{P_g (k_i) + 1/\overline{n}} 
&=
\frac{1}{2} \frac{\sum _{i} d_{i \alpha }}{(P_\alpha + 1/\overline{n})^2}
\,.
\end{align}
%

Now, we are working with a finite survey of volume \(V\), therefore 
%
\begin{align}
\sum _{i} d_{i \alpha } = \frac{4 \pi k_\alpha^2 \Delta k}{1/V} = 4 \pi k_\alpha^2 \Delta k V
\,.
\end{align}
%
This yields 
%
\begin{align}
F_{\alpha \beta } = \frac{1}{2} \delta_{\alpha \beta } \frac{4 \pi k_\alpha^2 \Delta k V}{(P_g (k_\alpha ) + 1/ \overline{n})^2}
\,,
\end{align}
%
so the Cramer-Rao bound is 
%
\begin{align}
\sigma _\alpha = \sqrt{F_{\alpha \alpha }^{-1}} = \qty(P_g (k_\alpha ) + \frac{1}{\overline{n}}) \frac{1}{\sqrt{2 \pi k_\alpha^2 \Delta k V}}
\,.
\end{align}

So, in order to decrease the error bar we want to increase \(\overline{n}\) as much as possible (i.e.\ measure very dim galaxies), and increase the number of \(k\) modes: this is achieved by taking a survey with a volume which is as large as possible.

\end{document}
