\documentclass[main.tex]{subfiles}
\begin{document}

\chapter{Model selection}

\marginpar{Monday\\ 2020-12-14, \\ compiled \\ \today}

The Bayes factor between two models \(M'\) and \(M\), depending on parameters \(\theta'\) and \(\theta \) respectively, having observed the data \(x\), is 
%
\begin{align}
B = \frac{\int \dd{\theta '} \mathbb{P}(x | \theta ' M') \mathbb{P}(\theta ' | M')}{\int \dd{\theta } \mathbb{P}(x | \theta  M) \mathbb{P}(\theta  | M)}
\,,
\end{align}
%
which, if the parameters are uniformly distributed in some range \(\Delta \theta _i\), we will have 
%
\begin{align}
B = \frac{\int \dd{\theta '} \mathbb{P}(x | \theta ' | M')}{\int \dd{\theta } \mathbb{P}(x | \theta , M)} \frac{\Delta \theta _1 \dots \Delta \theta _n}{\Delta \theta '_1 \dots \Delta \theta '_{n'}}
\,.
\end{align}

This can account for having nested models, in which \(n < n'\) and the prime parameters are a superset of the non-prime ones.
The ratio of the likelihoods will favor the more general model (with \(\theta '\)), while the ratio of the \(\Delta \theta \) will reduce to \(\Delta \theta'_{n+1} \dots \Delta \theta '_{n'}\), which will favor the simpler model. 

In general computing the integrals must be done numerically, typically through MCMC.
We will instead take the analytic approach. 
We consider the likelihoods as being MVNs around their maxima: 
%
\begin{align}
\mathcal{L}^{(i)} = \mathcal{L}_0^{(i)} \exp[
    \qty(\vec{\theta}_i - \vec{\theta}_i^{ML})^{\top}
    \qty(\nabla \nabla \chi^2_i)
    \qty(\vec{\theta}_i - \vec{\theta}_i^{ML})
]
\,.
\end{align}

Now, we can calculate the integrals analytically: 
%
\begin{align}
\int \dd{\theta } \mathcal{L} = (2 \pi )^{N/2} \qty(\det (\nabla \nabla \chi^2))^{-1/2} \mathcal{L}_0
\,,
\end{align}
%
so the likelihood ratio is just 
%
\begin{align}
\frac{\int \dd{\theta _1} \mathcal{L}_1}{\int \dd{\theta _2} \mathcal{L}_2} = (2 \pi )^{(N_2 - N_1) / 2 } \frac{\mathcal{L_0^{(1)}}}{\mathcal{L_0}^{(2)}} \qty[\frac{\det (\nabla \nabla \chi^2_{(2)})}{\det (\nabla \nabla \chi^2_{(2)})}]^{1/2}
\,.
\end{align}

Therefore, for nested models 
%
\begin{align}
B_{12} = (2 \pi )^{(N_2 - N_1) / 2 } \frac{\mathcal{L_0^{(1)}}}{\mathcal{L_0}^{(2)}} \qty[\frac{\det (\nabla \nabla \chi^2_{(2)})}{\det (\nabla \nabla \chi^2_{(2)})}]^{1/2}
\Delta \theta _2^{N_1 +1 } \dots \Delta \theta _2^{N_2 }
\,.
\end{align}

\subsection{Savage-Dickey density ratio}

This is an analytical tool which can be applied as long as we have nested models, with separable priors, and the more complex model only has one extra parameter. 

Let us say that the likelihood for the complex model is \(p_2 (\vec{x} | \phi , \psi )\), while the likelihood of the simpler model is \(p_1 (\vec{x} | \phi ) = p_2 (\vec{x} | \phi , \psi_{*})\).
The Bayes factor is then 
%
\begin{align}
B_{12}  =\frac{\int \dd{\phi } p(\vec{x} | \phi , \psi_*) \pi_1 (\phi )}{\int \dd{\phi } \dd{\psi } p(\vec{x} | \phi , \psi ) \pi_2 (\phi , \psi )}
\,.
\end{align}

We call the bottom integral \(q\). 
Let us consider \(p(\psi_* | \vec{x})\): by Bayes' theorem, it is 
%
\begin{align}
p(\psi _* | \vec{x}) = \frac{\int \dd{\phi } p(\vec{x} | \phi , \psi_* ) \pi _2 (\phi, \psi_*)}{q} 
\,,
\end{align}
%
therefore 
%
\begin{align}
B_{12} &= p(\psi _*, \vec{x}) \frac{\int \dd{\phi } p(\vec{x} | \phi , \psi_*) \pi_1 (\phi )}{p(\psi _*, \vec{x}) }  \\
&= \frac{\int \dd{\phi } p(\phi, \psi _* | \vec{x})}{\pi_2 (\psi _*)}
\,.
\end{align}

\todo[inline]{Missing some calculation steps}



\end{document}
