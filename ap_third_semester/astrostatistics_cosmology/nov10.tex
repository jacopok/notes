\documentclass[main.tex]{subfiles}
\begin{document}

\subsection{Autocorrelation}

\marginpar{Tuesday\\ 2020-11-10, \\ compiled \\ \today}

Successive points in a chain are correlated, and the Monte Carlo error scales like \(1/\sqrt{N}\), where \(N\) is the number of \emph{uncorrelated} points. 
How do we calculate the effective number of uncorrelated points? 

The autocorrelation for points \(\tau \) steps apart is 
%
\begin{align}
\rho_\theta (\tau ) = \frac{n}{n - \tau } \frac{\sum _{i=1}^{n-\tau } (\theta _i - \overline{\theta}) (\theta _{i+\tau } - \overline{\theta})}{\sum _{i=1}^{n-\tau } (\theta _i - \overline{\theta})^2}
\,.
\end{align}

Then, the variance of the parameter \(q\) is 
%
\begin{align}
\var{\theta } = \frac{\tau _f}{N} \sigma_\theta^2
\,,
\end{align}
%
where \(\tau _f\) is the integrated autocorrelation time: 
%
\begin{align}
\tau _f = \sum _{\tau =- \infty }^{+ \infty } \rho _\theta (\tau )
\approx 1 + 2 \sum _{\tau = 1}^{N} \rho _\theta (\tau )
\,.
\end{align}

So, \(N / \tau _f\) is the effective number of points. 
A note on numerical stability: in computing the previous expression one must be careful to not extend the sum to a region in which the estimation of \(\rho _\theta (\tau )\) is imprecise: for very large \(\tau \) we have very few samples of these large delays.  

It is best if the proposal distribution \(Q\) is as close as possible to the true distribution. 
Preprocessing should include estimating the covariance matrix of the data, or even running a short chain.

Gibbs sampling does not work well if the parameters are heavily correlated, since it moves along the conditional distributions. 

Examples of libraries.

\chapter{Real datasets}

\section{The CMB}

At roughly \SI{3e5}{yr} after the Big Bang the universe had a temperature of roughly \SI{3e3}{K}, and electrons and protons combined into hydrogen atoms.
Since the cross-section of Compton scattering is proportional to the inverse square mass of the particle, the optical depth of the gas decreased drastically: it became transparent. 

We observe that light as a near-blackbody spectrum at \(T _{\text{CMB}} \approx \SI{2.7255}{K}\). 
We see slight anisotropies, of the order of \(\Delta T / T \sim \num{e-5}\). 

These fluctuations are very sensitive to inflationary cosmological parameters. 

The quantum process which generates the anisotropies is inherently random. We must treat it as a random field. 
Given a random field \(X(\vec{t})\) (where \(\vec{t}\) is a position vector) with a pdf \(P(X)\) we can compute its average value as 
%
\begin{align}
\mathbb{E}_X(\vec{t}) = \int X(\vec{t}) P(X) \dd{X}
\,,
\end{align}
%
and its auto-covariance and auto-correlation functions: 
%
\begin{align}
C(\vec{t}, \vec{s}) &= \text{cov} \qty( X(\vec{t}), X(\vec{s}))  \\
\rho (\vec{t}, \vec{s}) &= \frac{C(\vec{t}, \vec{s})}{\sqrt{C(\vec{t}, \vec{t}) C(\vec{s}, \vec{s})}}
\,.
\end{align}

The hypothesis of stationarity can be stated by saying that \(C(\vec{t}, \vec{s})\) is only a function of \(\vec{t}-\vec{s}\).  
Isotropy means that it is only a function of \(\abs{\vec{t} - \vec{s}}\). 

A Gaussian random field is such that the joint probability of the field attaining certain values at certain points is a multivariate Gaussian. 

By homogeneity and isotropy many fields in cosmology can be treated as Gaussian, stationary and isotropic: they can then be described through their power spectrum alone. 

We Fourier transform the field \(F(\vec{x})\) into \(F(\vec{k})\), and define the power spectrum \(P_F\) as 
%
\begin{align}
\expval{F(\vec{k}_1) F(\vec{k}_2)} = (2 \pi )^3 P_F(k) \delta^{(D)} (\vec{k}_1 + \vec{k}_2)
\,,
\end{align}
%
where \(k = \abs{k_1 } = \abs{ k_2 }\).

By the \textbf{Wiener-Kintchine} theorem the power spectrum is the 3D Fourier transform of the two-point correlation function. 

We must work in spherical harmonics: 
%
\begin{align}
\frac{\Delta T}{T} (\theta , \varphi ) = \sum _{\ell>2} \sum _{\abs{m} \leq \ell} a_{\ell m} Y_{\ell m} (\theta ,\varphi )
\,.
\end{align}

The angular frequency \(\ell\) gives  us the angular scale of the oscillations: roughly speaking, \(\theta \sim \pi / \ell\). 
Then we define the power spectrum as 
%
\begin{align}
\expval{a^{*}_{\ell_1 m_1 } a_{\ell_2 m_2 }} = C_{\ell_1} \delta_{\ell_1 \ell_2 } \delta_{m_1 m_2 }
\,.
\end{align}

We skip the monopole and the dipole: the monopole is just an offset, and the Doppler is given by the motion of the Solar System through the galaxy, and of the Galaxy through the CMB reference. 

\end{document}

