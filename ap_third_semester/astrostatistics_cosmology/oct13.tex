\documentclass[main.tex]{subfiles}
\begin{document}

\marginpar{Tuesday\\ 2020-10-13, \\ compiled \\ \today}

Homework: we have on Moodle a sheet with the same homework as last year.
This year's will likely be quite similar, although a couple exercises might change.

Coming back to two-parameter estimation:
what is interesting now is to give error-bars on our parameters. 

If our posterior for a single parameter is a Gaussian, then the errorbar for that parameter will be the \(\sigma \) for that Gaussian.
We can compute the \(\sigma \) through the FWHM or any other height-interval. 

For a multivariate Gaussian the reasoning is the same. 
We can intercept the distribution with a plane at a given height, and project the interception (which for a perfect Gaussian will be an ellipse) onto the plane of the two variables. 

We can thus find the region corresponding to a \SI{68}{\percent}, \SI{95}{\percent} or whatever percent probability contained inside: a credible region.

We have a probability distribution \(P = \mathbb{P}(x, y | \vec{D}, I)\), where \(\vec{D}\) is the data vector while \(I\) represents previous information. 
We usually work with \(L = \log P\).

The MAP estimate is the pair \((x_0 , y_0 )\) such that 
%
\begin{align}
\eval{\vec{\nabla} L}_{(x_0 , y_0 )} = \vec{0}
\,.
\end{align}

We can then Taylor expand in two variables: the first derivative vanishes. Let us denote the parameter vector as \(\vec{t} = (x, y)\)
%
\begin{align}
L(x, y) &= L(x_0 , y_0 ) 
+ \frac{1}{2} \eval{\pdv[2]{L}{x}}_{\vec{t}_0} (x - x_0)^2
+ \frac{1}{2} \eval{\pdv[2]{L}{y}}_{\vec{t}_0} (y - y_0)^2
+ \eval{\pdv{L}{y}{x}}_{\vec{t}_0} (x-x_0 )(y - y_0)  \\
&= L(x_0 , y_0 ) 
+ \frac{1}{2} (\vec{t} - \vec{t}_0)^{\top} \eval{H}_{\vec{t}_0 } (\vec{t} - \vec{t}_0)
\,,
\end{align}
%
where \(H\) is the Hessian matrix, \(H_{ij} = \partial_{i} \partial_{j} L\). 
Then, the probability reads 
%
\begin{align}
P = \exp(L) = \const \times  \exp[ \frac{1}{2} (\vec{t} - \vec{t}_0)^{\top} \eval{H}_{\vec{t}_0 } (\vec{t} - \vec{t}_0)]
\,.
\end{align}

This is precisely a multivariate Gaussian, with a covariance matrix \(\Sigma \) such that \(H = - \Sigma^{-1}\).

Let us denote \(Q = (\vec{t} - \vec{t}_0)^{\top} \eval{H}_{\vec{t}_0 } (\vec{t} - \vec{t}_0)\). 

This is the argument of the exponential, so having constant \(P\) means having constant \(L\) means having constant \(Q\). 
Note that \(H\) is both \emph{negative} definite and symmetric, since the covariance matrix must be positive definite and symmetric. 

The matrix \(H\) will have two eigenvectors: these define the principal axes, knowing which we will be able to draw the contours.
If we impose \(Q = k\) for some real number \(k\), the lengths of the eigenvectors \( \vec{e}_i\) will satisfy \(\abs{\vec{e}_i} = k / \lambda _i\), where \(\lambda _i\) are the corresponding eigenvalues.

If \(H_{12} = 0\), then the principal axes are aligned with the coordinate axes: the parameters are \emph{uncorrelated}.

How do we compute an errorbar for a \textbf{marginalized} parameter? 
We start by marginalizing: 
%
\begin{align}
\mathbb{P}(x) = \int \exp(L) \dd{y}
\,,
\end{align}
%
which will be a function of \(x\) only. 
The errorbar is then given by \(\sigma _x^2 = \expval{(x - x_0 )^2} = \int \mathbb{P}(x) (x-x_0 )^2 \dd{x}\). So, the final expression is given by 
%
\begin{align}
\sigma^2_x = \int (x-x_0)^2 \int \exp(L) \dd{y} \dd{x}
\,.
\end{align}

For a bivariate Gaussian, this yields \(\sigma^2_x = (-H^{-1})_{11}\). 
Note that there is a big difference between \((H_{11})^{-1}\) and \((H^{-1})_{11}\).

We can also compute the covariance: 
%
\begin{align}
\operatorname{Cov}(x, y) = \expval{(x-x_0 )(y-y_0 )}
= \int (x-x_0 ) (y - y_0 ) \exp(L) \dd{x} \dd{y}  = - \qty(H^{-1})_{12}
\,.
\end{align}

Now, let us discuss how \(\sigma _x = \sqrt{(- H^{-1})_{11}}\) changes with the other parameters: in this bivariate case we can write it explicitly, it comes out to be
%
\begin{align}
\sigma _x &= \sqrt{\frac{- H_{yy}}{\det H}}  \\
&= \sqrt{\frac{- H_{yy}}{H_{xx} H_{yy} - H^2_{xy}}}
\,.
\end{align}

If the two parameters are uncorrelated, this simplifies to \(\sigma _x = \sqrt{- 1 / H_{xx}}\): the same as the one-parameter case, since the Hessian is minus the inverse of the covariance matrix.
This is expected: if the parameters are uncorrelated the probability factors.

If, instead, we have correlation then the error on \(x\) increases.

If the determinant vanishes, the single-parameter errorbar diverges: the parameters are completely \emph{degenerate}. 
This happens when the effect of the noise is undistinguishable from the effect of the signal. 
We can, though, often perform a change of variables in order to constrain some combination of the parameters. 

The CMB has a temperature average of \(\expval{T(\hat{n})} \approx \SI{2.725}{K}\), but there are anisotropies. 
These can be expanded in spherical harmonics, whose coefficients encode all the information: these represent the \emph{angular power spectrum}.

The coefficients we compute are \(\expval{\abs{a_{\ell m}}^2} = C_\ell\) (for any \(m \in (- \ell, \ell)\)). 


The problem is that the \(C_\ell\) are sensitive to cosmological parameters. The features of the spectrum we change are different depending on which parameter we change. 
Changing \(\Omega _{0b}\) raises the odd-numbered peaks and lowers the even-numbered ones, \(\Omega _{0m}\) raises the third peak compared to the second\dots 

We start off with a power spectrum \(P(k) = A k^{n}\), and then we have oscillations. Changing \(A\) has the effect of raising the whole spectrum. 

After recombination there is \emph{reionization}, when the first stars form. At \(z \sim 6 \divisionsymbol 10\) the universe becomes ionized again, so there are free electrons messing up the CMB. This damps the primordial fluctuations. 

The optical depth due to reionization is denoted as \(\tau \). 
Therefore, the parameters \(A\) and \(\tau \) are very degenerate!
The degeneracy is not complete, since \(\tau \) is causal: it can not affect the first 10-so multipoles, while \(A\) can.

What must be done is finding some observable which breaks the degeneracy: one option is looking at CMB polarization.
\(\tau \) has a polarizing effect, \(A\) does not. 

\end{document}
