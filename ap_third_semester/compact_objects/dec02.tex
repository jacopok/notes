\documentclass[main.tex]{subfiles}
\begin{document}

\marginpar{Wednesday\\ 2020-12-2, \\ compiled \\ \today}

We have discussed the temperature dependence for URCA and MURCA neutrino-driven cooling of NSs, as well as radiative photon cooling.

Let us see when (slow) neutrino cooling and photon cooling have equal magnitude: 
%
\begin{align}
L_\nu^{s} = L_\gamma  \implies N^{s} T^{8} = s T^{2 + 4 \alpha }
\,,
\end{align}
%
which yields (neglecting \(\alpha \)) a transition temperature of 
%
\begin{align}
T_{\text{trans}}^{s} = \qty(\frac{s}{N^{s}})^{1/6} \sim \SI{e8}{K}
\,;
\end{align}
%
and we can do a similar thing for fast neutrino cooling: here 
%
\begin{align}
T _{\text{trans}}^{f} = \qty(\frac{s}{N^{f}})^{1/4} \sim \SI{e6}{K}
\,,
\end{align}
%
which means that the transition time is of the order of \(\num{e4} \divisionsymbol \SI{e6}{yr}\). 

This means that we can make a rough plot of \(\log T\) against \(\log t\) for the slow-cooling case; in the first phase slow cooling dominates and the temperature decreases quickly; then photon cooling takes over and the cooling accelerates. 

In the fast-cooling case the cooling is faster initially, and the photon-cooling phase is reached earlier. 

Real cooling curves can be computed in a much more sophisticated way.

\textbf{Fast cooling is not supported by observations, since we observe old NSs}. 

\subsection{Neutron star spectra}

The most common observational manifestation of NSs are radio pulsars. 
We know about around 2500 of those. 
The thermal emission from the surface is not the only one. 

At the beginning of the 90s the first satellite observing the soft X-rays (ROSAT) between \SI{.1}{keV} and \SI{10}{keV} discovered NS sources which emit X-rays but \emph{not} radio waves. 
We know of 7 of those, the ``magnificent seven''. 
Their temperature is of the order of the hundreds of eV, and their spectrum looks to be purely thermal. 

New satellites like CHANDRA and XMM-Newton can give us spectra like \(\log F\) as a function of \(\log E\).

The first thing one may try to fit is a blackbody: 
%
\begin{align}
F_{BB} = A \frac{E^3}{e^{E / k_B T} - 1}
\,.
\end{align}

The parameter \(A\) is proportional to \(R^2 / D^2\), where \(R\) is the radius of the NS while \(D\) is the distance from it, since
the total flux is proportional to \(L / 4 \pi D^2\), and \(L \propto R^2 T^{4}\).

There are reasons to suspect that the surface of a cooling NS is not homogeneous. 
Energy is transferred to the envelope from the interior mostly through electron conduction. 
We can make a good model for the crust in the plane-parallel approximation, since its depth is \(L \sim \SI{100}{m}\), while the radius of the NS is \(R \sim \SI{10}{km}\).
So, we approximate it as a slab.

The \(\hat{z}\) axis is vertical from the surface, and the magnetic field \(\vec{B}\) will generally not be aligned with it: we denote as \(\phi \) the angle between it and the \(\hat{x}\) axis --- taking \(\hat{x}\), \(\hat{z}\) and \(\vec{B}\) as coplanar. 

Heat flux is usually described by Fourier's law: 
%
\begin{align}
\vec{q} = - k \vec{\nabla} T
\,,
\end{align}
%
where the scalar \(k\) is called the thermal conductivity. 
It being a scalar means that conduction is in principle isotropic --- the medium can conduct in any direction equally.
This is \emph{not} the case if there is a magnetic field: electrons move along field lines in a much easier way than across them. 

Therefore, we expect \(k_\parallel \gg k_\perp\), where parallel and perpendicular are referred to the direction of \(\vec{B}\). 
The conductivity we need is a tensorial one: the law is written as usual, however \(k\) is now a tensor.

We define a ``primed'' system of reference in which the \(\hat{x}\) axis is parallel to \(\vec{B}\): here, (in two dimensions since there is rotational symmetry around the \(\vec{B}\) axis as far as conduction is concerned)
%
\begin{align}
k' = \left[\begin{array}{cc}
k_\parallel & 0 \\ 
0 & k_\perp
\end{array}\right]
\,,
\end{align}
%
and the law reads 
%
\begin{align}
q_i = - k_{ij} \pdv{T}{x_j}
\,.
\end{align}

The conductivity tensor will transform with the Jacobian of the rotation: 
%
\begin{align}
k_{ij} = \pdv{x_i}{x'_k} \pdv{x_j}{x'_\ell} k'_{k \ell}
\,,
\end{align}
%
where 
%
\begin{align}
\pdv{x_i}{x'_k} = \left[\begin{array}{cc}
\cos \phi  & \sin \phi  \\ 
- \sin \phi & \cos \phi 
\end{array}\right]
\,.
\end{align}

% [Calculations]

\end{document}
