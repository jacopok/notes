\documentclass[main.tex]{subfiles}
\begin{document}

\marginpar{Wednesday\\ 2020-11-18, \\ compiled \\ \today}

Recall that 
%
\begin{align}
\Sigma = \num{5.2} \alpha^{-4/5} \dot{M}_{16}^{7/10} M^{1/4} R^{-3/4} \SI{}{g / cm^2}
\,.
\end{align}

Is it true that \(M_d = M _{\text{disk}} \ll M\), as we assumed? 
We can calculate it as 
%
\begin{align}
M_d &= \int_{R _{\text{in}}}^{R _{\text{max}}} 2 \pi R \Sigma \dd{R}  \\
&\approx \num{e-8} \alpha^{-4/5} M_{16}^{7/10} M^{1/2} M_{\odot}
\,,
\end{align}
%
assuming that \(R _{\text{max}} \approx 10 R _{\text{in}}\).
This is indeed many orders of magnitude below the mass of the stellar source. 

\subsubsection{Regions of the disk}

The \(\alpha \) parameter comes from the very rough assumption \(\nu _{\text{turb}} = \alpha c_s H\), it is a weak point of this model.
Asking that \(\alpha = \const\) is just plain wrong. 

Further, we assumed that \(P = P _{\text{gas}}\), and that the Rosseland mean opacity \(\kappa _R\) is only given by free-free absorption. 

We know that for Thompson scattering the cross-section is \(\kappa _R^{s} = \sigma _T / m_p = \SI{.4}{cm^2 / g}\). 
When is this smaller than the free-free opacity? In dimensionless terms, the equation for \(\kappa = \tau / \Sigma \) reads
%
\begin{align}
\kappa_{R}^{\text{ff}} &> \kappa_R^{s} \\
\num{6.3} \dot{M}_{16}^{-1/2} M^{1/4} R_{10}^{-3/4} f^2 &> \num{.4}  \\
R_{10} &> \num{.5e-2} \dot{M}_{16}^{2/3} M^{1/3} f^{8/3} \\
R &> \num{.5e8} \dot{M}_{16}^{2/3} M^{1/3} f^{8/3} \SI{}{cm}
\,.
\end{align}

For a white dwarf, this is always the case; we can tell in general that for \(R \lesssim \SI{e8}{cm}\) electron scattering dominates.

The temperature decreases with radius as \(T \propto R^{-3/2}\), so for high enough radii it can drop below \SI{e4}{K}, at which point recombination can occur: at that point free-free absorption cannot occur anymore, and we must account for free-bound and bound-bound transitions. 

If a NS or a BH is accreting, on the other hand, we can have a scattering-dominated internal region.

So, in terms of the \textbf{main type of matter-radiation interaction} we will have three regions: going outwards, there is domination of electron scattering, free-free absorption, bound-free/bound-bound absorption. 

Does \(P _{\text{gas}}\) dominate over \(P _{\text{rad}}\)? their ratio is indeed
%
\begin{align}
\frac{P _{\text{rad}}}{P _{\text{gas}}} = \frac{ \frac{1}{3} a T^{4}}{\frac{k T_c \Sigma H}{\mu m_p}} \approx \num{3e-3} \alpha^{1/10} \dot{M}_{16}^{7/10} R_{10}^{-3/8} f^{7/5} \ll 1
\,.
\end{align}

As \(R\) decreases, \(P _{\text{rad}}\) becomes ever more relevant. 
Is there an equality radius? It will definitely be smaller than \SI{3e8}{cm}. 
Doing the calculation, we find 
%
\begin{align}
R  _{\text{equality}} \approx \num{24} \alpha^{2/21} \dot{M}_{16}^{16/21} f^{9/21} \SI{}{km}
\,.
\end{align}

This may sometimes be attained in the innermost region of the disk, right before the ISCO.

So, in terms of the \textbf{nature of most of the pressure}, we may have a radiation-dominated region in the innermost part of the disk, but mostly there will be gas pressure domination. 

\paragraph{The shape of the disk}

We calculate the shape of the disk, \(H(R)\), in the innermost radiation-dominated region.
The sound speed under radiation domination is
%
\begin{align}
c_s^2 = \frac{P}{\rho } = \frac{1}{3} \frac{a T_c^{4}}{\rho} = \frac{1}{3} \frac{4 \sigma }{c} \frac{T_c^{4}}{\rho }
\,,
\end{align}
%
and we know that 
%
\begin{align}
\frac{4}{3} \frac{\sigma T_c^{4}}{\tau } = \frac{3 GM \dot{M}}{8 \pi R^3} f
\,,
\end{align}
%
which means that 
%
\begin{align}
c_s^2 = \frac{3 GM \dot{M} \tau f}{8 \pi R^3 \rho c}
\,,
\end{align}
%
and using the fact that 
%
\begin{align}
\tau = \kappa _R^{s} \Sigma = \frac{\sigma _T}{m_p} \rho H
\,,
\end{align}
%
we find 
%
\begin{align}
c_s^2 =  
\frac{3 GM \dot{M} \sigma _T \rho H f}{8 \pi R^3 \rho  c m_p} =
\frac{3 GM \dot{M} \sigma _T  H f}{8 \pi R^3   c m_p}
\,,
\end{align}
%
but we also know that \(H = c_s R (R / GM)^{1/2}\); using this fact we can calculate the sound speed \(c_s = (H/R)(GM/R)^{1/2} \). 
Using this (see equation \eqref{eq:speed-of-sound-disk-shape}), we get 
%
\begin{align}
\frac{H^2}{R^2} \frac{GM}{R} &= \frac{3 GM \dot{M} \sigma _T  H f}{8 \pi R^3   c m_p} \\
H &= \frac{3 \sigma _T \dot{M} }{8 \pi c m_p}f 
\,,
\end{align}
%
which is nearly independent of \(R\): the only dependence is inside the factor \(f\), which depends on \(R\) quite weakly. 
The shape of the disk is slab-like in the inner region, and concave in the outer part but still quite flat. 

\paragraph{The Eddington limit}

The Eddington luminosity for electron scattering is 
%
\begin{align}
L _{\text{Edd}} = \frac{4 \pi G M m_p c}{\sigma _T}
\,,
\end{align}
%
and the corresponding accretion rate is \(\dot{M} _{\text{Edd}} = L _{\text{Edd}} / c^2\). 

The critical accretion rate is the one which produces an Eddington luminosity, after accounting for efficiency: 
%
\begin{align}
\eta \dot{M} _{\text{crit}} c^2 = L _{\text{Edd}}
\,,
\end{align}
%
so \(\dot{M} _{\text{crit}}\) is larger than \(\dot{M} _{\text{Edd}}\). 
Using this, the height of the disk is given by 
%
\begin{align}
H &= \frac{3}{2} \dot{M} f \frac{\sigma _T}{4 \pi c m_p } = \frac{3}{2} \dot{M} f \frac{GM}{L _{\text{Edd}}} \\
&= \frac{3}{2} \underbrace{\frac{GM}{R _{\text{in}}c^2}}_{\eta } R _{\text{in}} \frac{\dot{M}}{\dot{M} _{\text{Edd}}} f  \\
&= \frac{3}{2} \eta R _{\text{in}} \frac{\dot{M}}{\dot{M} _{\text{Edd}}} f
\,,
\end{align}
%
so 
%
\begin{align}
\frac{H}{R _{\text{in}}} = \frac{3}{2} \frac{\dot{M}}{\dot{M} _{\text{crit}}} f
\,,
\end{align}
%
and we can see that \(H < R _{\text{in}}\) iff \(\dot{M} < M _{\text{crit}}\). 
Then, we see that \textbf{the accretion rate must be subcritical as long as we want to keep the disk thin}. 

\subsection{The multicolor blackbody}

A final point about disks: each layer of the disk emits roughly a blackbody, which means that the total spectrum is a superposition of several blackbodies, this is called a multicolor blackbody.

The spectrum emitted by each annulus, as usual, is described by a  Planck function: 
%
\begin{align}
I_\nu = \frac{2h}{c^2} \frac{\nu^3}{\exp(\frac{h \nu }{k_B T}) - 1}
\,,
\end{align}
%
as long as there is no reprocessing of the radiation from stuff around the disk.
This is an interesting process, but for simplicity we will not discuss it. 

What we measure is the flux: 
%
\begin{align}
F_\nu = \int_{4 \pi } I_\nu \cos \theta \dd{\Omega }
\,,
\end{align}
%
where \(\dd{\Omega } = 2 \pi R \dd{R}/D^2\), where \(D\) is the distance from us. 
The integral to compute is 
%
\begin{align}
F_\nu = \frac{2 \pi }{D^2} \cos \iota \int_{R _{\text{in}}}^{R _{\text{out}}} R \dd{R} \frac{\nu^3}{\exp(\frac{h \nu }{k_B T(R)}) - 1}
\,.
\end{align}

This integral can be computed numerically, but we can already gather its main characteristics. 
As we have seen earlier \eqref{eq:temperature-accretion-disk}, the temperature looks like 
%
\begin{align}
T(R) = T _{\text{in}} \qty( \frac{R _{\text{in}}}{R})^{3/4}
\,.
\end{align}

Let us now estimate the integral in three limits, comparing \(h \nu     \) to \(k_B T(R _{\text{in}})\) and \(k_B T (R _{\text{out}})\) respectively.

The first interesting limit is the low-energy one: \(h \nu \ll k_B T (R _{\text{out}}) < k_B T(R)\) for any \(R\).
Then, the flux is proportional to 
%
\begin{align}
F_\nu \propto \int R \dd{R} \frac{\nu^3}{h \nu / k_B T(R)} \propto \nu^2
\,.
\end{align}

The opposite, high energy limit \(h \nu \gg k_B T(R _{\text{in}})\) yields an exponential cutoff: 
%
\begin{align}
F_\nu \propto \nu^3 \exp(- \frac{h \nu }{k_B T _{\text{in}}})
\,.
\end{align}

In the intermediate region, \(k_B T _{\text{out}} < h \nu < k_B T _{\text{in}}\). 

Defining \(x = h \nu / k_B T(R)\), we find 
%
\begin{align}
F_\nu \propto \int \frac{\nu^3 }{e^{x} - 1} R \dd{R}
\,,
\end{align}
%
but \(x \propto \nu / T \propto \nu R^{3/4}\), therefore \(R \propto x^{4/3} \nu^{-4/3}\), which means that (since \(\nu \) is constant in the context of the integral) \(\dd{R} \propto x^{1/3} \nu^{-4/3} \dd{x}\) 
%
\begin{align}
F_\nu \propto \int \frac{\nu^3 \nu^{-8/3}}{e^{x}-1} x^{5/3} \dd{x} 
\,,
\end{align}
%
which means that 
%
\begin{align}
F_\nu \propto \nu^{1/3} \int \frac{x^{5/3}}{e^{x}-1} \dd{x}
\,.
\end{align}

The integral is approximately one from 0 to \(\infty \), a number. 
The \(F \propto \nu^{1/3}\) signature intermediate region is a characteristic of accretion disks.  

\end{document}
