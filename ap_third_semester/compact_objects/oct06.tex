\documentclass[main.tex]{subfiles}
\begin{document}

\marginpar{Tuesday\\ 2020-10-6, \\ compiled \\ \today}

As we said earlier, massive stars go type-2 supernova: this corresponds to Core-Collapse. 

Compact objects are quite common in the galaxy. 

% Non si può preparare un matrimonio con i fichi secchi. 

A compact object is one for which the ratio of the gravitational radius \(R_g = GM / c^2\) is comparable to the radius of the true object.
For a white dwarf, the ratio is of the order of \num{e3}.

We then need GR in order to deal with them. 
Let us quickly go over exact solutions of the Einstein Field Equations.

\section{The Schwarzschild external solution}

This lecture, we consider the vacuum Schwarzschild solution.
The most general line element which is spherically symmetric (invariant under spatial rotations) 
must be made up of elements which are themselves invariant under spatial rotations. 
We will use spherical coordinates: \(r, \theta , \varphi , t\). 

In flat spacetime, the line element reads 
%
\begin{align}
\dd{s^2} = - c^2 \dd{t^2} + \dd{r^2} + r^2 \qty( \dd{\theta^2} + \sin^2 \theta \dd{\varphi^2})
\,,
\end{align}
%
and our Schwarzschild solution will need to reduce to this in some limit.

The spatial line element is given by 
%
\begin{align}
\dd{\vec{r}} \cdot \dd{\vec{r}} =  \dd{r^2} + r^2 \qty( \dd{\theta^2} + \sin^2 \theta \dd{\varphi^2}) = g_{ij} \dd{x^{i}} \dd{x^{j}}
\,.
\end{align}

Then, the most general spherically symmetric line element will read 
%
\begin{align}
\dd{s^2} =  F(r ,t) \dd{t^2} + M(r, t) \dd{r^2} + G(r, t) \dd{r} \dd{t} + C(r, t) r^2 \qty( \dd{\theta^2} + \sin^2 \theta \dd{\varphi^2})
\,,
\end{align}
%
however, we can redefine the radial coordinate in order to remove the function multiplying the angular term, so we get 
%
\begin{align}
\dd{s^2} = F \dd{t^2} + M \dd{r^2}  + G \dd{r} \dd{t} 
+ r^2 \qty(\dd{\theta^2} + \sin^2\theta \dd{\varphi^2}) 
\,.
\end{align}

We can also introduce a new time variable: 
%
\begin{align}
\dd{t'} = \dd{t} + \psi (r, t) \dd{r} 
\,.
\end{align}

If \(\psi = 2 G / F\), then we remove the mixed term, and then we are left with the expression 
%
\begin{align}
\dd{s^2} = -B (r, t) \dd{t^2} + A(r, t) \dd{r^2} + r^2 \dd{\Omega^2}
\,.
\end{align}

However, we have not yet determined the two functions, and we have not said anything about the Einstein Field Equations, which are 
%
\begin{align}
R_{\mu \nu } - \frac{1}{2} g_{\mu \nu } R = 8 \pi G T_{\mu \nu }
\,.
\end{align}

In vacuo, the stress-energy tensor vanishes. 
The curvature scalar must vanish (we can show this by contracting the EFE with the inverse metric), so the equations reduce to \(R_{\mu \nu } = 0\). 
We restrict ourselves to the static case. 

The linearly independent components of the Ricci tensor read 
%
\begin{align}
R_{0}^{0} = \frac{B''}{2AB} - \frac{A' B'}{4 A'' B}
- \frac{B^{\prime 2}}{4AB^2} + \frac{B'}{rAB} &= 0 \\
R_{1}^{1} = \frac{B''}{2AB} - \frac{A' B'}{4 A'' B}
- \frac{B^{\prime 2}}{4AB^2} + \frac{A'}{rA^2} &= 0 \\ 
R_{2}^{2} = \frac{1}{4rA} \qty(\frac{B'}{B} - \frac{A'}{A})
+ \frac{1}{r^2} \qty( \frac{1}{A} - 1)
\,.
\end{align}

Computing \(R^{0}_{0} - R^{1}_{1} = 0\) we find 
%
\begin{align}
\frac{1}{rA} \qty(\frac{B'}{B} + \frac{A'}{A} )&= 0  \\
\dv{\log (AB)}{r} &= 0
\,,
\end{align}
%
so \(AB\) is constant.
Without losing generality we can take \(A = 1/B\), since if this is not the case we can just rescale the radial or temporal coordinate until it is.

Then, we can compute 
%
\begin{align}
R^{2}_{2} = \frac{B}{2r} \qty(\frac{B'}{B} + \frac{B'}{B}) + \frac{1}{r^2} (B-1) &= 0  \\
B' + \frac{B}{r} - \frac{1}{r} &= 0  \\
\dv{}{r} \qty(rB) &= 1
\,,
\end{align}
%
so \(r B(r) = r + C\) for some constant \(C\), or equivalently 
%
\begin{align}
B (r) = \frac{C}{r} + 1
\,.
\end{align}

After this, we can already substitute into the metric: 
%
\begin{align}
\dd{s^2} = - \qty(1 + \frac{C}{r}) \dd{t^2} 
+ \frac{1}{1 + C/r} \dd{r^2} + r^2 \dd{\Omega^2}
\,.
\end{align}

For any value of \(C\), \(B \to 1\) as \(r \to \infty \): the metric reduces to the flat one asymptotically. 
Right now \(C\) is an arbitrary constant, however in the weak field limit it is known that 
%
\begin{align}
g_{00} = - \qty(1 + 2 \frac{\phi }{c^2})
\,,
\end{align}
%
where \(\phi = - GM / r\) is the Newtonian gravitational field.
Equating this expression to the one for \(g_{00} \), we find 
%
\begin{align}
g_{00} = - \qty(1 - \frac{2 GM}{rc^2})
\implies C = - \frac{2GM}{c^2}
\,.
\end{align}

The constant \(M\) in the classical case is the mass of the source, however we are computing a vacuum solution. This is the mass we would compute if we were to measure the orbits of objects around the compact object.

% \todo[inline]{This is then surely a \emph{gravitational} mass, is it also an \emph{inertial} mass? Can we show this in GR?}

Then, we can write the Schwarzschild metric: 
%
\begin{align}
\dd{s^2} = - \qty(1 - \frac{2GM}{c^2r}) \dd{t^2}
+ \qty(1 - \frac{2GM}{rc^2})^{-1} \dd{r^2} + r^2 \dd{\Omega^2}
\,.
\end{align}

This is derived by assuming time-independence, however the result is the same even in the time-dependent case by the Jebsen-Birkhoff theorem (which we will prove in a moment). 
The element \(g_{rr} \) diverges as \(r \to R_g = 2GM/c^2\), however this does not represent any physical divergence: no component of the Riemann tensor \(R_{\mu \nu \rho \sigma }\) diverges there, while the Ricci tensor \(R_{\mu \nu }\) is identically zero by hypothesis.
On the other hand, the origin is a true singularity, since the scalar \(R^{\mu \nu \rho \sigma } R_{\mu \nu \rho \sigma } \propto r^{-6}\) diverges there. 

There are coordinates which do not diverge near the horizon: one classical choice employs the ``tortoise'' coordinates, which are the same for \(r\), \(\theta \), \(\varphi \) as the Schwarzschild ones, while the time becomes (setting \(G = c =  1\))
%
\begin{align}
t = t' - 2M \log \qty(1 - \frac{r}{2M})
\,.
\end{align}

\begin{claim}
Substituting this into the metric yields (dropping the primes for clarity):
%
\begin{align}
\dd{s^2} = - \qty(1 - \frac{2M}{r}) \dd{t}^2
+ \frac{4M}{r} \dd{r} \dd{t} + \qty(1 + \frac{2M}{r}) \dd{r^2}
+ r^2 \dd{\Omega^2}
\,.
\end{align}
\end{claim}

\begin{proof}
The new time differential after the change of coordinates reads 
%
\begin{align}
\dd{t} &= \pdv{t}{t'} \dd{t'} + \pdv{t}{r} \dd{r}  \\
&= \dd{t'} - 2M \frac{(-1/2M)}{1 - r/2M} \dd{r} = \dd{t}' + \frac{ \dd{r}}{1 - r / 2M}  \\
\dd{t}^2 &= \dd{t}^{\prime 2} 
+2 \frac{ \dd{r} \dd{t'}}{1 - r/2M} 
+ \frac{ \dd{r^2}}{(1 - r / 2M)^2}
\,,
\end{align}
%
therefore the new metric reads (omitting the angular part for brevity):
%
\begin{align}
\dd{s}^2 
&= - \qty(1- \frac{2M}{r}) \dd{t}^2 + \frac{ \dd{r}^2}{1 - 2M / r}  \\
&= - \qty(1- \frac{2M}{r}) \qty(\dd{t}^{\prime 2} 
+2 \frac{ \dd{r} \dd{t'}}{1 - r/2M} 
+ \frac{ \dd{r^2}}{(1 - r / 2M)^2}) + \frac{ \dd{r}^2}{1 - 2M / r}  \\
&= - \qty(1 - \frac{2M}{r}) \dd{t^{\prime 2}} + \frac{4M}{r} \dd{r} \dd{t'} + \qty(1 + \frac{2M}{r}) \dd{r^2}
\,,
\end{align}
%
where we have used the following manipulations: setting \(x = 2M / r\), the coefficients of the \(\dd{r} \dd{t}'\) and \(\dd{r^2}\) terms are respectively
%
\begin{align}
2\frac{1 - x}{1 - x^{-1}} &= -2x  \\
-\frac{1 -x}{(1- x^{-1})^2} + \frac{1}{1-x} &= \frac{1}{1-x} \qty(-\qty(\frac{1-x}{1- x^{-1}})^2 + 1) =  \frac{1-x^2}{1-x} = 1+x
\,.
\end{align}
\end{proof}

There is no pathology at \(r = 2M\) anymore, so it was not a physical divergence. 
The temporal coefficient \(g_{00} \) is the same: it can be shown that it is an invariant under coordinate transformations. 

If we take two points which are very close along a particle trajectory, they must be separated by an interval \(\dd{s}^2 < 0\). 

If we consider a radial path (not necessarily geodesic) described by \(r(t)\), we can compute the corresponding line element by neglecting the angular part: 
%
\begin{align}
\dd{s^2} &= - \qty(1 - \frac{2M}{r}) \dd{t}^2
+ \frac{4M}{r} \dd{r} \dd{t} + \qty(1 + \frac{2M}{r}) \dd{r^2}  \\
\qty( \dv{s}{t})^2 &= - \qty(1- \frac{2M}{r})
+ \frac{4M}{r} \dv{r}{t} + \qty(1 + \frac{2M}{r} ) \qty(\dv{r}{t})^2
\,.
\end{align}

Now, the question we ask is: is it possible for the particle trajectory to be timelike or lightlike (\(\dd{s}^2 \leq 0\)) and outgoing (\(\dv*{r}{t} > 0\)) under these conditions? If this is the case, the signs of the three terms read 
%
\begin{align}
\underbrace{\qty(\dv{s}{t})^2}_{< 0 ?}
= 
- \qty(1- \frac{2M}{r})
\underbrace{+ \frac{4M}{r} \dv{r}{t}}_{> 0 } \underbrace{+ \qty(1 + \frac{2M}{r} ) \qty(\dv{r}{t})^2}_{> 0}
\,,
\end{align}
%
so we can see that the equality can be satisfied (a positive number cannot equal a negative one!) as long as the first term on the right-hand side is negative, which means \(r > 2M\). 
If \(r \leq 2M\), on the other hand, this cannot be the case: a radial trajectory below the horizon \emph{cannot} be outward.

This is what ``horizon'' means: it is a \emph{semi-permeable} membrane, particles can surpass it only in one direction. 

\paragraph{Jebsen-Birkhoff}

This theorem states that the Schwarzschild solution also describes the spacetime around an object in the spherically-symmetric but time-\emph{dependent} case. Let us give a sketch of its proof, omitting some tedious calculations. 
If we write out the components of the Ricci tensor, we find something in the form 
%
\begin{align}
R^{0}_{0} &= \eval{R^{0}_{0}}_{\text{static}} + \dot{A} \qty(\dots)\\
R^{1}_{1} &= \eval{R^{1}_{1}}_{\text{static}} + \dot{A} \qty(\dots)
\,,
\end{align}
%
while \(R^{2}_{2}\) and \(R^{3}_{3}\) are the same. Also, the term \(R^{1}_{0}\) does not vanish unlike the static case, and is equal to 
%
\begin{align}
R^{1}_{0} = - \frac{\dot{A}}{r A^2} = 0
\,,
\end{align}
%
so \(\dot{A} = 0\): the equations then are the same as the static case ones!
This, however, is not the end, since now the equation 
%
\begin{align}
\frac{A'}{A} + \frac{B'}{B} = 0
\,
\end{align}
%
is not necessarily solved by \(\log A = - \log B\), since a prime denotes a \emph{partial} derivative with respect to \(r\), so in general we will have \(\log A + \log B = f(t)\), some generic function of time. 
The metric will then read 
%
\begin{align}
\dd{s}^2 = - \qty(1 - \frac{2M}{r}) f(t)\dd{t^2} + \qty(1 - \frac{2M}{r}) \dd{r^2} + r^2 \dd{\Omega}^2
\,,
\end{align}
%
but we can simply rescale the time coordinate to \(t \to \sqrt{f} t\) in order for this to reduce to the usual expression.
This theorem was originally discovered by the Norwegian physicist Jebsen, and only later popularized in a textbook by Birkoff \cite{johansenDiscoveryBirkhoffTheorem2005}. 

The source of the geometry can change in time while leaving the outside spacetime unperturbed, however this holds only as long as the variation remains spherically symmetric: collapse, expansion or pulsation. 
Any asymmetry can lead to the emission of gravitational radiation.  

\end{document}
