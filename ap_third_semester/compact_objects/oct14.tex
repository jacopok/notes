\documentclass[main.tex]{subfiles}
\begin{document}

\section{The Kerr solution}

\marginpar{Wednesday\\ 2020-10-14, \\ compiled \\ \today}

Some history: the Schwarzschild was found in 1915 while S.\ was serving in the army; the second exact solution was found in 1963 by a PhD student in New Zeland \cite[]{kerrGravitationalFieldSpinning1963}.
In Cartesian coordinates the metric is hideous; nowadays we use the coordinates defined by Boyer and Lyndquist: starting from a flat set of Cartesian coordinates \(x, y, z\) we define \(r, \theta , \varphi \) as
%
\begin{align}
x &= \sqrt{r^2 + a^2} \sin \theta \sin \varphi   \\
y &= \sqrt{r^2 + a^2} \sin \theta \cos \varphi  \\
z &= r \cos \theta 
\,.
\end{align}

These are then not \emph{spherical} but \emph{spheroidal} coordinates. 
Constant-\(r\) spheroids are oblate ellipsoids in the \(z\) direction.

The Kerr solution, for who wants to see the computation, is derived in ``The mathematical theory of Black Holes'' by Chandrasekhar \cite[]{chandrasekharMathematicalTheoryBlack1998}.  
The metric reads 
%
\begin{align}
\dd{s^2} = - \qty(1 - \frac{2Mr}{\Sigma }) \dd{t^2}
+ \frac{\Sigma }{\Delta } \dd{r^2} 
+ \Sigma \dd{\theta ^2} 
+ \qty(r^2 + a^2 + \frac{aA}{\Sigma }) \sin^2 \theta \dd{\varphi^2} 
- \frac{2A}{\Sigma } \dd{t} \dd{\varphi }
\,,
\end{align}
%
where: 
%
\begin{align}
\Delta = r^2 - 2Mr + a^2
&&
\Sigma = r^2 + a^2 \cos^2 \theta 
&&
A = 2Mar \sin^2 \theta 
\,,
\end{align}
%
while \(a = J / M\) is the \emph{specific angular momentum} of the black hole. 

It describes the spacetime outside an axially symmetric, stationary body. 
In the Schwarzschild spacetime we have identified an horizon at \(r = 2M\) by the properties that \(g_{00} \to 0\) and \(g_{rr} \to \infty \). 

What about Kerr? Are there horizons? In this case, the two conditions do not happen in the same place. 
The region \(g_{00} = 0\) is known as the \emph{limit of staticity}: if it is the case, an observer's worldline cannot be both \emph{timelike} and \emph{stationary} (meaning time-directed). 

The condition \(g_{00} > 0\) is equivalent to \(\Sigma < 2Mr\).
Now, consider the \(\dd{r^2}\) coefficient: \(\Sigma / \Delta \): this is positive always. 

If we are in the region \(\theta = \pi /2\), at \(g_{00} > 0\) we can still have the metric's signature be \(- +++\), since we have the \(\dd{t} \dd{\varphi }\) term. 

As long as \(a \dd{\varphi } > 0\), that term in the metric is negative, allowing the signature of the metric to be preserved. 

This means that \textbf{the particle must be co-rotating} with the black hole if it is in the region \(g_{00} > 0\).
 
As long as this is the case, however, a particle can remain at fixed \(r\) and even escape. This is, then, \emph{not} an horizon. 

The horizon, instead, is found when \(g_{rr}\) diverges: this is equivalent to \(\Delta \to 0\), which means 
%
\begin{align}
r^2 - 2Mr + a^2 = 0 \implies r = M \pm \sqrt{M^2 - a^2}
\,.
\end{align}

Both of these radii correspond to an horizon. 
Let us denote \(r_+ = r_H\), since the inner horizon cannot really affect any observations. 

In order for these two solutions to be real, we must have \(a < M\). 
There is an horizon as long as \(a/ M < 1\). 

If \(a > M\), we have a \textbf{naked singularity}, since the singularity at \(\Sigma  = 0\) is still there.
This singularity, in any case, is not shaped like a point: we only reach it along the equatorial plane. 

Penrose proposed the cosmic \textbf{censorship hypothesis}: the universe is a ``prude'', it always hides singularities with horizons.
There are good theoretical reasons to believe that this is verified. 

Where is the limit of staticity? The equation is 
%
\begin{align}
r^2 + a^2 \cos^2 \theta - 2Mr = 0
\,,
\end{align}
%
which  is solved by 
%
\begin{align}
r_{\pm} = M \pm \sqrt{M^2- a^2 \cos^2 \theta }
\,.
\end{align}

There are two of these surfaces as well, and we consider the outer one as before: \(r_E = r_+\). If the horizon exists, then this region also exists. 

This region is called the \textbf{ergosphere}, and the region between \(r_H < r < r_E\) is called the \textbf{ergoregion}. 

See \textcite[sec. 3]{heinickeSchwarzschildKerrSolutions2015} for an in-depth discussion of the shape of these regions; figure 12 there also shows them graphically. 

The name comes from the fact that we can extract rotational energy from the BH. 
``Ergo'' means energy. 

There has been a long debate about whether the Penrose process actually occurs in a realistic astrophysical setting: the consensus is that the trajectory a particle must take in order for this to happen is way too peculiar. 

Note that in this case we also have the cyclic coordinates \(\varphi \) and \(t\). We have two constants of motion like in Schwarzschild. 

It can be shown that there exists a third constant of motion, beyond \(E\) and \(L_z\): \(Q\), called Carter's constant. 

For motion in the equatorial plane we can write the following expression for the energy integral: 
%
\begin{align}
E = \frac{r^{3/2} - 2 r^{1/2} \pm a M^{1/2}}{r^{3/2} \qty(r^{3/2}- 3M r^{1/2} \pm 2a M^{1/2})^{1/2}}
\,,
\end{align}
%
where \(\pm\) refers to whether the particle moves along a prograde or retrograde trajectory.

The expression for the last stable (LS, or sometimes MS for ``marginally stable'') circular orbit is \cite[eq.
\ 28]{puglieseEquatorialCircularMotion2011}:
%
\begin{align}
r_{LS} = M \qty[ 3 + z_2 \mp \qty[(3 - z_1 ) (3 + z_1 +2z_2 )]^{1/2}]
\,,
\end{align}
%
where 
%
\begin{align}
z_1 &= 1 + \qty(1 - \frac{a^2}{M^2})^{1/3} \qty[\qty(1 + \frac{a}{M})^{1/3} + \qty(1 - \frac{a}{M})^{1/3}]  \\
z_2 &= \qty(3 \frac{a^2}{M^2} + z_1^2)^{1/2}
\,.
\end{align}

This reduces to \(r_{LS} = 6 M\) in the \(a = 0\) case, since then we have \(z_1 = z_2 = 3\). 

What happens for an extreme Kerr BH, with \(a = M\)? 
Then \(z_1 = 1\), \(z_2 = 2\): so, 
%
\begin{align}
r_{LS} = M \qty[3 + 2 \mp \sqrt{2 \times 8}] = M \qty[5 \mp 4]
\,,
\end{align}
%
which yields \(r_{LS} = M\) in the corotating case, and \(r_{LS} = 9M\) in the counter-rotating case.

We expect that the efficiency of a Kerr BH in the extraction of energy from matter will be higher than the Schwarzschild solution. 
Let us use the expression we found for the specific energy \(E\), with respect to the parameter \(x = r/M\): 
%
\begin{align}
E = \frac{x^{3/2} - 2 x^{1/2} \pm a/M}{x^{3/2} \sqrt{x^{3/2} - 3x^{1/2} \pm 2a /M}}
\,,
\end{align}
%
which in the extreme case becomes 
%
\begin{align}
E = \frac{x^{3/2} - 2 x^{1/2} \pm 1}{x^{3/2} \sqrt{x^{3/2 }- 3 x^{1/2} \pm 2}}
\,,
\end{align}
%
so we can take the limit: in the corotating direct case, sending \(x \to 1\) we get \(E = 1 / \sqrt{3} \approx \num{.577}\). 

The efficiency is given by 
%
\begin{align}
\eta = \frac{E_\infty  - E}{E_{\infty }} = 1 - 1 / \sqrt{3} \approx \SI{42}{\percent}
\,.
\end{align}

This is a huge amount of energy: we do not expect real black holes to be extreme. 
There are estimates of the spins of the black holes. 
What is found is that they seem to cover the whole range \(0 < a/ M < 1\). 

An issue regarding a wide-spread misconception: the parameter \(M\) in the interior Schwarzschild solution is the same as the \(M\) in the corresponding \emph{exterior} solution. Can we apply the same kind of reasoning for Kerr?
Is there an interior Kerr solution? 
We don't know, but most probably not. 
Many efforts were put into seeking it, and they all failed. 
The properties of the matter and radiation inside are weird. 

The wide-spread misconception is to claim that the Kerr spacetime describes the spacetime around a rotating star. 
It sounds reasonable, but it's wrong. 
Numerically we can derive the true form of the spacetime outside something like a neutron star: it is very different from the Kerr spacetime. 

A qualitative argument: a star will generally have a quadrupole moment and emit GWs, while Kerr does not. 
Kerr is a Petrov-type-D spacetime, which is \emph{nonradiating}. 

For an in-depth discussion see \textcite[]{bertiRotatingNeutronStars2005}, which refers to \textcite[]{bakerMakingUseGeometrical2000} for its tools to measure ``non-type-D-ness''.

The Petrov classification refers to the symmetries of the Weyl tensor --- the ``deviatoric'' part of the Riemann tensor, which describes deformation but not expansion/compression. 
Kerr spacetime can be analytically shown to be type D, which means its Weyl tensor is ``highly symmetric'', and the consequence of this is that the radiation it can emit decays like \(r^{-3}\), like the tidal tensor of regular Newtonian gravity. 
This prevents the emission of long-range \(r^{-1}\) gravitational radiation, which we know NSs emit (see, for example, the Hulse-Taylor pulsar). 

% Next time, we will discuss Equations of state and degenerate gasses.

\end{document}
