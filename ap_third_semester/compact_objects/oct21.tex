\documentclass[main.tex]{subfiles}
\begin{document}

\marginpar{Wednesday\\ 2020-10-21, \\ compiled \\ \today}

Back to what we were saying: we have an explicit expression for the pressure of a degenerate electron gas. 

Let us consider the ultrarelativistic limit first, \(x_F \gg 1\): 
%
\begin{align}
P &\approx \frac{\pi m^4 c^{5}}{h^3}
\qty[x_F^2 \frac{2}{3} x_F^2 + \log \dots]  \\
&\approx \frac{\pi m^4 c^{5}}{h^3} \frac{2}{3} x_F^{4} \propto n^{4/3} \propto \rho^{4/3}
\,.
\end{align}

In the opposite limit, \(x_F \ll 1\), we would need to expand up to fifth order to see through all the cancellations: skipping all that mess, we find
%
\begin{align}
P \approx \frac{8}{15} \frac{m^4 c^5}{h^3} x_F^{5} \propto n^{4/3}  \propto \rho^{5/3}
\,.
\end{align}

The results are similar: in both cases, \(P \propto \rho^{\gamma }\), with \(\gamma = 5/3\) and \(4/3\) respectively.

This allows us to compute the maximum mass that a spherical equilibrium configuration can reach if it is supported by the degeneracy pressure of the electron gas alone: the \textbf{Chandrasekhar mass}.
This is the maximum mass of a white dwarf. 

We need to start with the \textbf{Lane-Emden equation}, which is also due to work by Chandrasekhar. 

We have a spherical star with no nuclear burning. 
The only forces are due to gravity and pressure. This is a good description of a white dwarf. 
Sometimes white dwarfs can have some burning on their surface if they are in binary systems, if they accrete fresh mass. 

The equation of hydrostatic equilibrium reads 
%
\begin{align}
\dv{P}{r} = - \frac{G m(r) \rho }{r^2}
\,,
\end{align}
%
where \(m(r)\) is the mass contained within a shell of radius \(r\):
%
\begin{align}
m(r) = \int_0^{r} 4 \pi r^2 \rho \dd{r} 
&& 
\dv{m}{r} = 4 \pi r^2 \rho 
\,.
\end{align}

We want to couple these two equations in order to get a single one: we will find a second-order equation. 
We start from 
%
\begin{align}
\frac{r^2}{\rho }\dv{P}{r} &= - G m  \\
\dv{}{r} \qty(\frac{r^2}{\rho }\dv{P}{r}) &=
- G \dv{m}{r} = - 4 \pi G r^2 \rho  \\
\frac{1}{r^2} \dv{}{r} \qty( \frac{r^2}{\rho } \dv{P}{r}) &= - 4 \pi G \rho 
\,.
\end{align}

We also need to specify an equation of state, which we will have in the form \(P(\rho )\): we assume \(P = K \rho^{\gamma }\), a \textbf{polytropic} EoS, with constant \(\gamma \).
Sometimes this is also written through \(\gamma = 1 + 1/n\), where \(n\) is called the polytropic index. 

A convenient way to solve the equation is to substitute \(\rho = \lambda \phi^{n}\): then, \(P = \lambda^{1+1/n} K \phi^{n+1}\). 
Then, if we assume that the function \(\phi \) is dimensionless and such that \(\phi (0) = 1\), we have \(\lambda = \rho _c\). 

We need to compute 
%
\begin{align}
\dv{P}{r} &= \dv{}{r} \qty( K \lambda^{1 + 1/n} \phi^{n+1})  \\
&= K \lambda^{1 + 1/n} (n+1) \phi^{n} \dv{\phi }{r} 
\,.
\end{align}

Then, the equation reads 
%
\begin{align}
K \lambda^{1/n} (n+1) \frac{1}{r^2} \dv{}{r} \qty(r^2 \dv{\phi }{r} )
&= - 4 \pi G \lambda \phi^{n}  \\
a^2 \frac{1}{r^2} \dv{}{r} \qty(r^2 \dv{\phi }{r}) &= - \phi^{n} 
\,,
\end{align}
%
where 
%
\begin{align}
a^2 = \frac{K (n+1) \lambda^{-1 + 1/n}}{4 \pi G}
\,.
\end{align}

The physical dimensions of \(a\) are those of a length. 
We then introduce a new radial coordinate \(\xi = r /a\), and we have the right amount of \(a\)s on the left-hand side to adimensionalize everything: 
%
\begin{align}
\frac{1}{\xi^2} \dv{}{\xi } \qty(\xi^2 \dv{\phi }{\xi }) = - \phi^{n}
\,,
\end{align}
%
which is the usual formulation of the Lane-Emden equation. 

What are the boundary conditions we need to set? If we assume that \(\rho (r=0) = \rho _c\), we can set \(\lambda = \rho _c\) and \(\phi (0) = 1\).

We also need to set \(\dv*{P}{r} = 0\) at \(r = 0\), which also means that \(\dv*{\phi }{\xi } =0 \) there as well. 

% \todo[inline]{Is that not achieved through an argument of differentiability?}

After prescribing these conditions, we can solve the equation: the solution will depend on \(n\). 
We have analytical solutions for \(n = 0, 1, 5\) (note that however \(n\) is not necessarily an integer). 

The radius of the star described by the equation is given by the first zero: \(\xi_1 \). 
As a function of \(n\), 

\begin{figure}
\centering
\begin{tabular}{cc}
\(n\) & \(\xi_1\)\\
\hline
0 & \(\sqrt{6}\)  \\
\(3/2\) & \num{3.65}  \\
5 & \(\infty \)
\end{tabular}
\label{tab:zero-crossing-lane-emden}
\caption{First zero crossing as a function of \(n\).}
\end{figure}

Let us calculate the total mass of the star: it is given by an integral we can simplify inserting the Lane-Emden equation:
%
\begin{align}
M &= \int_{0}^{\xi_1 } 4 \pi \lambda \phi^{n} \xi^2 a^3 \dd{\xi }  \\
&= -4 \pi \lambda a^3 \int_{0}^{\xi_1 } \frac{\xi^2}{\xi^2} \dv{}{\xi } \qty( \xi^2 \dv{\phi }{\xi }) \dd{\xi }  \\
&= - 4 \pi \lambda a^3 \eval{\xi^2 \dv{\phi }{\xi }}_{0}^{\xi_1 }  \\
&= - 4 \pi \lambda a^3 \xi_1^2 \eval{\dv{\phi }{\xi }}_{\xi_1 }
\,,
\end{align}
%
where \(\dv*{\phi }{\xi}\) at \(\xi_1 \) is necessarily negative, since \(\xi_1\) is the first zero-crossing while \(\phi = 1\) at \(\xi= 0\). 

We want expressions for \(R\) and \(M\) in terms of \(\lambda \), the central density, and the polytropic index. 

Now, disregarding the constant numbers we have \(M \propto \lambda a^3\), while \(a^2 \propto \lambda^{-1 + 1/n}\), therefore \(a^3 \propto \lambda^{3 (-1 + 1/n) /2}\). 

This yields a dependence of the mass \(M\) on the central density \(\rho _c = \lambda \) as follows: 
%
\begin{align}
M \propto \lambda^{- \frac{3}{2} + \frac{3}{2n} + 1} = \lambda^{- \frac{1}{2} + \frac{3}{2n}}
\,.
\end{align}

On the other hand, the radius scales like 
%
\begin{align}
R = a \xi_1 \propto \lambda^{- \frac{1}{2} + \frac{2}{n}}
\,.
\end{align}

The polytropic index for a nonrelativistic gas is given by \(1 + 1/n = 5/3\), meaning \(n = 3/2\); for an ultrarelativistic gas we have \(1 + 1/n = 4/3\), meaning \(n = 3\). 

In general, for a polytropic index \(\gamma \) the number \(n\) is given by \(n = (\gamma - 1)^{-1}\).

Then, we want to find solutions with varying \(\rho _c\): let us start with relatively-small central densities, \(\rho _c < \SI{e6}{g / cm^3}\), and with the star being supported by electron degeneracy pressure, in the \textbf{nonrelativistic} case.
Then, we increase the central density. 

In this case we have \(M \propto \rho _c^{1/2}\). Increasing the central density increases the mass, in a polynomial fashion. 
On the other hand, \(R \propto \rho _c^{- 1/6}\). The radius decreases as we increase the central density. 

Increasing \(\rho _c\), sooner or later we move towards the \textbf{relativistic} region of the equation of state: then, we need to change \(n\) from \(3/2\) to \(3\): this means \(M \propto \rho _c^{0} = \const\) and \(R \propto \rho _c^{- 1/3}\). \(M \propto \rho _c^{0}\) means, qualitatively,  that ``\(\rho _c \propto M^{\infty }\)'': the central density is extremely dependent on the mass, and it will diverge with an asymptote, while the radius decreases. 
Then, we cannot go beyond this threshold, which we can compute using the first zero of the Lane-Emden equation: 
%
\begin{align}
M _{\text{Ch}} &= - 4 \pi \qty[\frac{K (n+1) \rho _c^{-1 + 1/n}}{4 \pi G}]^{3/2} \rho _c \eval{\qty(\xi^2 \dv{\phi }{\xi })}_{\xi_1 }
\,,
\end{align}
%
which we can evaluate with the correct numbers for the constants \(M _{\text{Ch}} = \num{5.81} / \mu _e^2 M_{\odot} \approx \num{1.44} M_{\odot}\) in the case of \(\mu _e \approx 2\), which is a good approximation for the cores of massive stars, which have endured several nuclear burning phases: for example, \ce{^{56}Fe} has 26 protons and 30 neutrons, so we can approximate \(n_n \approx n_p = n_e\), therefore \(n_e \approx n_b /2\).

% \todo[inline]{So we need to be careful when applying this to the iron cores of supernovae!}

\end{document}
