\documentclass[main.tex]{subfiles}
\begin{document}

\marginpar{Wednesday\\ 2020-10-28, \\ compiled \\ \today}

We have seen why \(\dot{M}\) is fixed as long as \(a_\infty \) and \(\rho _\infty \) are.
Let us then write the Bernoulli equation at the sonic radius: 
%
\begin{align}
\frac{1}{2} a_s^2 + \frac{a_s^2}{\Gamma -1} - \frac{GM}{r_s} &= \frac{a_\infty^2}{\Gamma -1}  \\
\frac{1}{2} a_s^2 + \frac{a_s^2}{\Gamma -1} -2 a_s^2 &= \frac{a_\infty^2}{\Gamma -1}
\,,
\end{align}
%
so we can calculate the sonic speed if we know \(a_\infty\): 
%
\begin{align}
a_s^2 \qty( \frac{1}{2} - 2 + \frac{1}{\Gamma -1}) &= \frac{a_\infty^2 }{\Gamma -1}  \\
a_s &= a_\infty \qty( \frac{2}{5 - 3 \Gamma })^{1/2}
\,,
\end{align}
%
so, since the proportionality constant is of order unity, we can say that the speed of sound at infinity and at the sonic point are of the same order of magnitude. 
We can also calculate the sonic radius: 
%
\begin{align}
r_s = \frac{GM}{2 a_s^2} = \frac{GM}{2 a_\infty^2} \qty(\frac{5 - 3 \Gamma }{2}) \sim \frac{GM}{a_\infty^2}
\,.
\end{align}

This relates the chemical potential at infinity and the gravitational potential energy: when these contributions are of the same order of magnitude we find the sonic transition. 

We are close to being able to write an explicit expression for the accretion rate \(\dot{M}\); we only need to use the continuity equation: 
%
\begin{align}
\dot{M} &= \eval{4 \pi r^2 \rho_\infty \qty( \frac{a}{a_\infty })^{2 / (\Gamma -1)} u  }_{r = r_s}  \\
&= 4 \pi r_s^2 \rho _\infty \qty(\frac{a_s}{a_\infty })^{2 / (\Gamma -1)} a_s  \\
&\propto \frac{M^2 \rho _\infty }{a_\infty^3}
\,.
\end{align}

What happens when \(\Gamma = 5/3\)? it would seem that then the sonic speed diverges; this is not the case, as was clarified relatively recently. 

What happens if the massive object is moving through the cloud, in which \(u_\infty \neq 0\)? We lose spherical symmetry, but we can still get a solution, in the form 
%
\begin{align}
\dot{M} \propto \frac{M^2 \rho _\infty }{(u_\infty^2 + a_\infty^2)^{3/2}}
\,.
\end{align}

Numerically, we have 
%
\begin{align}
\frac{\dot{M}}{\SI{e11}{g/s}} \approx \qty(\frac{M}{M_{\odot}})^2 \frac{n_\infty}{\SI{}{cm^{-3}}} \qty( \qty( \frac{u_{\infty}}{\SI{10}{km/s}} ) ^2 + \qty( \frac{a_{\infty}}{\SI{10}{km/s}} ) ^2 )^{-3/2} 
\,.
\end{align}
%

The equation  we found earlier is 
%
\begin{align}
\frac{a^2- u^2}{\rho } \dv{\rho }{r} = \frac{2u^2}{r} - \frac{GM}{r^2}
\,.
\end{align}

How do \(u\) and \(\rho \) depend on \(r\), roughly speaking? 
Let us give a qualitative argument. First, we assume that \(u \ll a\) (which also means \(r \gg r_s\)).

Then, everything on the right-hand side is approximately zero, while \(a^2-  u^2 \approx a^2\): this means 
%
\begin{align}
\frac{a^2}{\rho } \dv{\rho }{r} \approx 0
\,,
\end{align}
%
or, \(\rho \equiv \rho _\infty \). This happens if we are far from the star, in the \emph{hydrostatic region}. 
If \(\rho \approx \const\), then by continuity \(u \propto r^{-2}\).

\todo[inline]{Insert plot}

In the opposite limit, we have \(u \gg a\), \(r \ll r_s\).
Then, substituting in from the continuity we get
%
\todo[inline]{But \(u^2/r\) is \emph{not} negligible compared to \(GM / r^2\)! This does not change the substance, the proportionality still works, however we must be careful. }
\begin{align}
- \frac{1}{\rho } u^2 \dv{\rho }{r} &\approx - \frac{GM}{r^2}  \\
\frac{1}{\rho } \qty( \frac{1}{r^2 \rho })^2 \dv{\rho }{r} &\propto \frac{1}{r^2}  \\
\frac{1}{\rho^3 }\dv{\rho }{r} &\propto r^2  \\
\frac{1}{\rho^2} &\propto r^3
\,,
\end{align}
%
therefore \(\rho \propto r^{-3/2}\). This is the crucial aspect: in the supersonic region the density increases with decreasing \(r\). 
We know that \(u \propto 1/ r^2 \rho \propto r^{-1/2}\).  
This is the same thing we would have found by considering free fall: 
%
\begin{align}
\frac{1}{2} m v^2 = \frac{GMm}{r}
\,.
\end{align}

The full GR treatment of spherical accretion onto a Schwarzschild BH is quite close to what we have found here \cite[fig.\ 2, top left]{nobiliSphericalAccretionBlack1991}. 
What is the emitted \textbf{luminosity}? We can express in terms of the \emph{efficiency} \(\eta \): \(L = \eta \dot{M} c^2\). 

Let us give a qualitative, Newtonian argument about the maximum accretion efficiency of a Neutron Star. 
The surface of the NS is rigid, when matter falls upon it it basically \emph{stops}, so \(u(r_*) = 0\), and let us assume that all the impact energy is radiated away.
Then, the efficiency is the 
%
\begin{align}
\eta_{NS} = \frac{mc^2 - \qty( mc^2 -GMm / r_*)}{mc^2}=  \frac{GM}{c^2 r_*} = \frac{r _{\text{Schw}}}{2 r_*}
\,.
\end{align}

In GR this is slightly different, but not by much. 
Typical values are generally 
%
\begin{align}
\eta = \frac{3 (M / M _{\odot}) \SI{}{km}}{2 \times 10 (R / R _{\odot})} \approx \num{.15}
\,.
\end{align}

With typical values, we get 
%
\begin{align}
L = \num{.15} \times \num{e11} \times 9 \times \num{e20} \SI{}{erg/s} \sim \SI{e31}{erg/s}
\,,
\end{align}
%
100 times lower than the solar luminosity.
The average number density in interstellar space is of the order of \SI{1}{cm^{-3}}, the region in which we live has a lower number density, one or two orders of magnitude less.
It was swept away by multiple supernova events. 

People have looked for the luminosity of NS accretion, without result. 

NSs are strongly magnetized: radio pulsars have magnetic fields of the order of \SI{e7}{T}, magnetars reach \SI{e10}{T}. 
These magnetic fields can inhibit any accretion, through the \emph{propeller effect}.

If we consider a rotating dipolar field,\footnote{It's kind of like making homemade mayonnaise\dots} it propels the particles away. 

What about black holes? There is now no solid surface, and what we can do is to compare the typical timescales for production of radiation, \(\tau _{\text{rad}}\), and the typical dynamical (free-fall) timescale \(\tau _{\text{dyn}}\). If \(\tau _{\text{rad}} \gg \tau _{\text{dyn}}\), then hardly any radiation will be produced before the plasma can fall in. 
This is what happens if we do the proper modelling of the flow. 
Typically, \(\num{e-8} \lesssim \eta \lesssim \num{e-2}\). This is at least 10 times larger. 

The luminosity produced by a solar-mass BH would be even lower.
Isolated BHs are typically larger, but not by \emph{that much}. The very massive BHs can only come from very massive stars in the universe, where the low metallicity allowed for low stellar winds and high remnant masses. 

\section{Roche Lobe Overflow}

What happens in a \textbf{binary system} with a Compact Object and another \emph{donor} star? 

They will revolve around their common center of mass. 
If the star is massive, then there will be a strong stellar wind.
A fraction of the matter expelled will fall onto the CO, however this will not be a large fraction. 

The other possibility is \emph{Roche lobe overflow}. 

The idea is: consider a test particle (or, really, a test fluid element, since we will consider gasses) moving under the action of two centrally condensed (``point-like'') masses. This is basically a problem in celestial mechanics.
Let us denote the two masses by \(M_{1, 2}\). They will orbit around a common center of mass, and by Kepler's third law we can link their orbital separation and the period of the motion: 
%
\begin{align}
4 \pi^2 a^3 = G (M_1 + M_2 ) P^2
\,.
\end{align}

Solving the Roche problem means that we have to write down Newton's second law; it is convenient to do so in a system which is \emph{co-rotating} with the two stars, and centered in the center of mass.
Then, we will need to account for the fictitious forces for this noninertial reference system. 

The Euler equation (we start with this directly, for a single test particle the effects are the same) will read 
%
\begin{align}
\pdv{\vec{v}}{t} + \underbrace{\qty(\vec{v} \cdot \vec{\nabla}) \vec{v} }_{\text{convective derivative}}
= - \frac{\vec{\nabla} P}{\rho } - 
\underbrace{2 \vec{\omega} \wedge \vec{v}}_{\text{Coriolis}} 
 - \underbrace{\omega \wedge \qty(\vec{\omega} \wedge \vec{r})}_{\text{centrifugal}} - \vec{\nabla} \phi_G 
\,.
\end{align}

We can write down a potential for the centrifugal term: 
%
\begin{align}
\phi _C = - \frac{1}{2} \qty(\vec{\omega} \wedge \vec{r})^2
\,,
\end{align}
%
and we can then introduce an effective potential: the \emph{Roche} potential, \(\phi _R = \phi _C + \phi _G\). 

This potential reads 
%
\begin{align}
\phi _R = \frac{GM_1}{\abs{\vec{r} - \vec{r}_1} } + \frac{GM_2}{\abs{\vec{r} - \vec{r}_2} } + \frac{1}{2} \qty(\vec{\omega} \wedge \vec{r})^2
\,.
\end{align}

\todo[inline]{Add plot!}

If we plot the contour lines of this potential, they are determined by the mass ratio \(q = M_2 / M_1 \), while the linear scale of the system is only determined by \(a\).

\end{document}
