\documentclass[main.tex]{subfiles}
\begin{document}

\marginpar{Wednesday\\ 2020-12-2, \\ compiled \\ \today}

We will consider another kind of baryogenesis mechanism, connected to the reheating phase. 

The baryon asymmetry generated by the out-of-equilibrium decays of \(X\) particles is of the order of 
%
\begin{align}
B \approx \frac{\epsilon}{g_*}
\,,
\end{align}
%
where \(g_* \approx \num{e2}\) for the SM, or even \(g_* \approx \num{e3}\) BTSM.

A possible approach is to identify the \(X\) particle as the inflaton \(\varphi \). 

We will use the usual notations: \(n_X\), \(n_B\), and \(\epsilon \) for the amount of \(CP\) violation.
The Boltzmann equation reads 
%
\begin{align}
\dot{n}_X + 3H n_X = - \Gamma _X n_X
\,,
\end{align}
%
which is equivalent to 
%
\begin{align}
\dot{\rho}_\varphi + 3 H \rho _\varphi = - \Gamma _\varphi \rho _\varphi 
\,,
\end{align}
%
with the identification \(\rho _\varphi = m_X n_X\), where \(m_X\) is the effective of the inflaton field before the start of oscillations. 

The right-hand side should read \(- \Gamma _X (n_X - n_X^{\text{eq}})\), but we approximated it; the BE for baryon asymmetry on the other hand is 
%
\begin{align}
\dot{n}_B + 3 H n_B = +\epsilon \Gamma _X n_X
\,,
\end{align}
%
and we also need to write the Friedmann equation 
%
\begin{align}
H^2 = \frac{8 \pi G}{3} \qty(\rho _\varphi +\rho _R)
\,.
\end{align}

The number density of \(X\) decays like 
%
\begin{align}
n_X = n_{X, i} \qty(\frac{a_i}{a})^3 e^{-\Gamma _\varphi (t - t_i)}
\,,
\end{align}
%
where \(t_i\) is the ``oscillation time''. 

As usual, we solve the BE for baryon asymmetry by writing the LHS as 
%
\begin{align}
\frac{1}{a^3} \dv[]{}{t} (n_B a^3) 
\,,
\end{align}
%
therefore 
%
\begin{align}
n_B a^3 = \epsilon \Gamma _X \int_{t_i}^{t} n_{X, i} \qty(\frac{a_i}{a})^3 a^3e^{- \Gamma _\varphi (t - t_i)} \dd{t}
\,.
\end{align}

Doing the integral we get, 
%
\begin{align}
n_B a^3 = \epsilon n_{X, i} a_i^3 \qty(1 - e^{- \Gamma _\varphi (t - t_i)})
\,.
\end{align}

At late times, we will get 
%
\begin{align}
n_B a_f^3 = \epsilon n_{X, i} a_i^3
\,.
\end{align}

This has a very simple physical meaning: the left-hand side counts the number of net (signed) baryons in a comoving volume, while the right-hand side is \(\epsilon \) multiplied by the number of \(X\) particles in a comoving volume. 

The final \(B\) is given by 
%
\begin{align}
B_f = \eval{\frac{n_B}{s}}_{t_f} = \frac{n_B a_f^3}{s a_f^3} = \frac{\epsilon n_{X, i} a_i^3}{S_f}
\,,
\end{align}
%
but what is \(S_f\), the total entropy at a late time?
There are two ways to compute it: one is to solve the thermodynamic differential equations, and the alternative which we will adopt.

We write it as 
%
\begin{align}
S_f =s_f a_f^3
\,,
\end{align}
%
and we know that in general for radiation domination 
%
\begin{align}
s = \frac{2 \pi^2}{45} g_* T^3
\,,
\end{align}
%
and also 
%
\begin{align}
\rho _R = \frac{\pi^2}{30} g_* T^{4} \implies T = g_*^{-1/4} \rho _R^{1/4} \qty(\frac{30}{\pi^2})^{1/4}
\,,
\end{align}
%
therefore 
%
\begin{align}
S = s a^3 = g_*^{1/4} \rho _R^{3/4} a^3
\,.
\end{align}

Using this estimate, we can write 
%
\begin{align}
B_f = \frac{\epsilon n_{X, i} a_i^3}{g_{*f}^{1/4} \rho _{R, f}^{3/4} a_f^3}
\,.
\end{align}

We can build a hierarchy of timescales, and consider a time \(t\) such that: \(t_i = t _{\text{oscill}} < t \leq t_\varphi = \Gamma _\varphi^{-1}\). 
In this phase, \(\varphi \) dominates.
Then, with the assumption that there is complete conversion of the energy between \(\varphi \) and radiation, we can write 
%
\begin{align}
\rho_{R, f}^{3/4} = \rho _{X, i}^{3/4} \qty(\frac{a_i}{a_f})^{9/4}
\,,
\end{align}
%
so, since \(a \propto t^{2/3}\) (matter domination):
%
\begin{align}
B_f &= \frac{\epsilon n_{X, i} a_i^{3/4}}{g_{*f}^{1/4} \rho _{X, i}^{3/4} a_f^{3/4}}  \\
&= \frac{\epsilon n_{X, i}}{g_*^{1/4} \rho _{X, i}^{3/4}} t_i^{1/2} \Gamma _\varphi^{1/2}
\,,
\end{align}
%
therefore in Friedmann's equation we have
%
\begin{align}
H_i^2 = \frac{8 \pi }{3} \frac{\rho_{X,i}}{m_P^2 } \approx \frac{1}{t_i^2}
\,,
\end{align}
%
therefore \(t_i \approx m_P \rho _{X, i}^{-1/2}\).
Plugging this in and using \(\rho _{X, i} = m_X n_{X, i}\), we get 
%
\begin{align}
B_f &\approx \frac{\epsilon g_*^{-1/4} m_P^{1/2} \Gamma _\varphi^{1/2}}{m_X}  \\
&\approx \epsilon \frac{T_{RH}}{m_X}
\,,
\end{align}
%
where we recognize the approximate expression for the reheating temperature \eqref{eq:reheating-temperature}.

This expression should be compared with \(B_f = \epsilon / g_*\), the alternative. Which is larger?
For \(k \ll 1\), the mass of the bosons violating baryon number must be quite large: \(m_X \gtrsim \SI{e10}{GeV}\). 
In order for these to be relativistic, the temperature must then also be very large: \(T \gtrsim m_X \sim \SI{e10}{GeV}\), which constrains the reheating temperature. 

In this alternative scenario, with the inflaton violating baryon number conservation, we are allowed to violate this constraint. 

Are \(B\), \(C\) and \(CP\) violated in the SM? Yes, in some nonperturbative electroweak processes, but not enough. 

\section{Dark matter production}

We will start with the ``freeze-out'' mechanism. 
From the latest Planck data, we know that at a \SI{68}{\percent} CL:
%
\begin{align}
\Omega_{DM} h^2 &= \num{.120(1)}   \\
h &= \num{.674(5)}
\,,
\end{align}
%
therefore \(\Omega_{DM} \approx \SI{26.4}{\percent}\).
Dark Matter as measured here must be a non-relativistic pressureless fluid.

Consider a massive particle \(\psi \), with mass \(m_\psi \). 
Let us define 
%
\begin{align}
y = n_\psi / s
\,.
\end{align}

As long as \(\Gamma _\psi < H\), the abundance of \(\psi \) ``freezes out'' when \(z = m_\psi / T \) reaches 1. 

The Boltzmann equation for a process \(\psi \overline{\psi} \leftrightarrow X \overline{X}\) where \(X\) is in equilibrium with the plasma reads 
%
\begin{align}
\dot{n}_\psi + 3 H n_\psi = - \expval{\sigma \abs{v}} \qty[n_\psi^2 - (n_\psi^{\text{eq}})^2]
\,.
\end{align}

Then, 
%
\begin{align}
\dot{y} = - \expval{\sigma \abs{v}} s \qty[y^2 - y^2 _{\text{eq}}]
\,,
\end{align}
%
therefore, because
%
\begin{align}
\dv{z}{t} = zH
\,,
\end{align}
%
we can express the same with derivatives with respect to \(z\), denoted with primes: 
%
\begin{align}
y' &= - \frac{\expval{\sigma \abs{v}} s}{z H} \qty[y^2 - y^2 _{\text{eq}}]  \\
&= - \frac{z\expval{\sigma \abs{v}} s}{H(z=1)} \qty[y^2 - y^2 _{\text{eq}}]
\,,
\end{align}
%
since \(z^2 H = H(z=1)\), which corresponds to the moment at which \(m_\psi = T\).


%
\begin{align} \label{eq:y-eq-freezeout}
y _{\text{eq}} = \frac{n_\psi^{\text{eq}}}{s} = \begin{cases}
    \num{.278} \frac{g _{\text{eff}}}{g_{*s}} & z \ll 1  \\
    \num{.145} \frac{g_\psi }{g_{*s}} z^{3/2} e^{-z} & z \gg 1
\,.
\end{cases}
\end{align}

In the \(z \gg 1\) limit inverse processes are suppressed.

The effective number of degrees of freedom is given by 
%
\begin{align}
g _{\text{eff}} = \begin{cases}
    g_\psi & \text{boson} \\
    \frac{3}{4} g_\psi & \text{fermion} 
\,.
\end{cases}
\end{align}

We can write the derivative of \(y\) with respect to \(z\) as 
%
\begin{align}
y' &= - y^2 _{\text{eq}} \frac{\expval{\sigma \abs{v}} s}{zH} \qty[ \qty(\frac{y}{y _{\text{eq}}})^2 - 1] \\
\frac{z}{y _{\text{eq}}} y' &= - \underbrace{n^\psi  _{\text{eq}} \frac{\expval{\sigma \abs{v}} s}{H}}_{\Gamma_A / H} \qty[ \qty(\frac{y}{y _{\text{eq}}})^2 - 1] 
\,.
\end{align}

If \(\Gamma _A / H <1\), then the right-hand side becomes small, so the left-hand side (which is basically \(\Delta y / y\) in a comoving volume) must also be small, meaning that \(y\) is roughly constant (frozen out): we have produced a relic abundance of \(\psi \) particles.

\subsection{Hot Dark Matter relics}

These particles are called ``hot'' since they decouple while still being relativistic. This is what we would expect for neutrinos, for example. 

This particle decouples in the upper plateau of the curve \(y(z)\), so we expect little dependence for it on \(z_f\). 

Here we will have 
%
\begin{align}
y_\infty &= y(z \to \infty ) \approx y _{\text{eq}} (z_f)  \\
&\approx \frac{\zeta (3) g _{\text{eff}}T_f^3 / \pi^2}{(\pi^2 / 45) g_{*s } (z_f) T_f^3} \approx \num{.278} \frac{g _{\text{eff}}}{g_{*s} (z_f)}
\marginnote{Used \eqref{eq:y-eq-freezeout}.}
\,.
\end{align}

Now, 
%
\begin{align}
\rho_{\psi 0 } = m_\psi n_{\psi 0 } = m_\psi y_\infty s_0 
\,,
\end{align}
%
where \(s_0 \approx 1.8 g_{*s} (t_0 ) n_{\gamma 0 }\), where \(n_{\gamma 0} \approx \SI{422}{cm^{-3}}\). 

Assuming that there is one relativistic neutrino species (this is conservative, we know there to be at least 2), we can write 
%
\begin{align}
g_{*s} (t_0) \approx \num{2.63}
\,,
\end{align}
%
therefore \(s_0  \approx \num{2e3}\). 
This means that 
%
\begin{align}
\rho_{0, \text{crit}} = \SI{1.054e4}{eV cm^{-3}} h^2
\,,
\end{align}
%
and 
%
\begin{align}
\Omega_{0 \psi } h^2 \approx \num{5.56e-2} \times \frac{m_\psi }{\SI{}{eV}} \frac{g _{\text{eff}}}{g_{*s} (z_f)}
\,.
\end{align}

From Planck data, including lensing of the CMB and BAO, we know that at a \SI{95}{\percent} CL: 
%
\begin{align}
\sum _{\nu } m_\nu < \SI{.12}{eV}
\,.
\end{align}

They will surely be relativistic when they decouple at \(T_f \sim \SI{}{MeV}\). 
This tells us that 
%
\begin{align}
\Omega_{0 \nu  \overline{\nu}} h^2 = \frac{\qty(\sum _{\nu } m_\nu  / \SI{}{eV})}{128} \lesssim \num{e-3}
\,.
\end{align}

Thus, we learn that neutrinos cannot constitute the majority of DM. 
Despite this, they have a characteristic signature effect on structure formation. 

The free-stream scale is given by 
%
\begin{align}
\lambda_{fs}(t) = a(t) \int_0^{t} \frac{v_\nu }{a(\widetilde{t})} \dd{\widetilde{t}}
\,.
\end{align}

Below this scale, neutrinos move too fast and damp the power spectrum of matter. 

\end{document}
