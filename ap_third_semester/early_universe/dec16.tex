\documentclass[main.tex]{subfiles}
\begin{document}

\marginpar{Wednesday\\ 2020-12-16, \\ compiled \\ \today}

We will focus on the linear-order perturbations, thus setting \(r = 1\).
We can decompose our perturbation into its scalar, vector and tensor contributions. 
The scalar contributions are \(\psi (\vec{x}, \eta )\) and \(\phi (\vec{x}, \eta )\); they represent a GR extension of Newtonian gravitational potentials. 
The vector contributions are related to transverse (aka divergence-free, or vertical, or solenoidal) vector fields.
We can make the decomposition \(\omega _i (\vec{x}, \eta ) = \partial_i \omega^{\parallel} + \omega _i^{\perp}\), where \(\partial^{i} \omega _i^{\perp} = 0\). This is known as the Helmholtz decomposition, and it amounts to considering modes \(\omega (\vec{k})\) which are parallel and orthogonal to \(\vec{k}\) in Fourier space.
By counting the number of constraints or by reasoning geometrically in Fourier space one can see that the term \(\omega^{\parallel}\) includes 1 degree of freedom, while the term \(\omega^{\perp}_i\) includes two.

On the other hand, the tensor component can be decomposed into 
%
\begin{align}
\chi_{ij} = \underbrace{D_{ij} \chi^{\parallel}}_{\text{scalar}} + \underbrace{2 \chi^{\perp}_{(i, j)}}_{\text{vector}} + \underbrace{\chi_{ij}^{T}}_{\text{tensor}}
\,,
\end{align}
%
where \(D_{ij} = \partial_i \partial_{j} - (1/3) \delta_{ij} \nabla^2\) is a traceless derivative operator.
We assume that \(\partial^{i} \chi_{i}^{\perp} = 0\), while the tensor \(\chi_{ij}^{T}\) is both traceless and transverse: \(\chi_{i}^{i, T} = 0 = \partial^{i} \chi_{ij}^{T}\).

The last term contains gravitational waves. 
We can write the stress-energy tensor as 
%
\begin{align}
T_{\mu \nu } = \rho u_\mu u_\nu + p h_{\mu \nu  }
\,,
\end{align}

where \(u_\mu \) is the four-velocity of the fluid, while \(\rho \) and \(p\) are the energy density and isotropic pressure, and \(h_{\mu \nu } = g_{\mu \nu} + u_\mu u_\nu \) is a projection tensor onto the hypersurfaces orthogonal to the four-velocity.

The energy density is a function of position, which can be written as 
%
\begin{align}
\rho (\vec{x}, \eta ) = \rho^{(0)}(\eta ) + \sum _{r=1}^{\infty } \frac{1}{r!} \delta^{(r)} \rho (\vec{x}, \eta )
\,,
\end{align}
%
and an analogous expression can be written for the isotropic pressure \(p(\vec{x}, \eta )\). 

The perturbation of the pressure can be written as 
%
\begin{align}
\delta p &= \underbrace{\pdv{p}{\rho } \delta \rho }_{\text{adiabatic}} 
+ \underbrace{\pdv{p}{S} \delta S}_{\text{non-adiabatic}}  \\
&= c_s^2 \delta \rho + \delta p _{\text{non-ad}}
\,.
\end{align}

On the other hand, the four-velocity of a comoving observer is 
%
\begin{align}
u^{\mu } = \frac{1}{a} \qty(\delta^{\mu }_{0} + \sum _{r=1}^{\infty } \frac{1}{r!} v^{\mu }_{(r)})
\,,
\end{align}
%
since the unperturbed four-velocity is just \(u^{\mu }_{(0)} = \delta^{\mu }_0  / \sqrt{- g_{00} } = \delta^{\mu }_{0} / a\).
As for the perturbations, it can be shown (from the normalization condition \(u^2= -1\)) that at linear order \(v^{0} = - \psi \). 
Then, we can just consider the peculiar spatial velocity, and Helmholtz-decompose it: 
%
\begin{align}
v^{i} = \partial^{i} v^{\parallel} + v^{i}_\perp
\,,
\end{align}
%
where, as usual, \(\partial_{i} v^{i}_\perp = 0\). 
This latter term describes vorticity. 

As we discussed earlier, any tensor \(T\) can be gauge-transformed at linear order as 
%
\begin{align}
\widetilde{\Delta T} = \Delta T + \mathscr{L}_\xi (T_0 )
\,,
\end{align}
%
along the gauge transformation represented by the vector field \(\xi \). 
We write this vector field as 
%
\begin{align}
\xi^{0} &= \alpha (\vec{x}, \eta )  \\
\xi^{i} &= \partial^{i} \beta + d^{i}
\,,
\end{align}
%
where \(\partial^{i} d_i = 0\). 

Denoting derivatives with respect to conformal time with a prime, the perturbations transform like 
%
\begin{align}
\widetilde{\psi} &= \psi + \alpha ' + \frac{a'}{a} \alpha   \\
\widetilde{\omega}_i &= \omega _i - \alpha_{, i} + \beta'_{, i} + d _i'   \\ 
\widetilde{\phi} &= \phi - \frac{1}{3} \nabla^2 \beta - \frac{a'}{a} \alpha d
\widetilde{\chi}_{ij} &= \chi_{ij} + 2 D_{ij} \beta + 2d_{(i, j)} 
\,,
\end{align}
%
which follows from the equation we derived earlier, \(\widetilde{\Delta T}  = \Delta T + \mathscr{L}_\xi T_0  \)
and we can decompose these as 
%
\begin{align}
\widetilde{\omega}^{\parallel} &= \omega^{\parallel} - \alpha + \beta '  \\
\widetilde{\omega}^{\perp}_i &= \omega^{\perp}_i + d'_i  \\
\widetilde{\chi}^{\parallel } &= \chi^{\parallel} + 2 \beta   \\
\widetilde{\chi}^{\perp}_i &= \chi^{\perp}_i + d _i  \\
\widetilde{\chi}^{T}_{ij} &= \chi^{T}_{ij}
\,.
\end{align}

The traceless tensor part of the perturbation is gauge invariant. 

The energy density perturbation can be shown to transform like 
%
\begin{align}
\widetilde{\delta \rho } &= \delta \rho + \mathscr{L}_\xi \rho^{(0)} (\eta ) \\ 
&= \delta \rho + \rho^{(0), \prime} \alpha 
\,,
\end{align}
%
showing that \(\delta \rho \) is \emph{not} a scalar.
 
The velocity transforms like 
%
\begin{align}
\delta u^{\mu } = \frac{1}{a} v^{\mu } 
\,,
\end{align}
%
so 
%
\begin{align}
\widetilde{\delta u}^{\mu } = \delta u^{\mu } + \mathscr{L}_\xi u^{\mu , 0}
\,,
\end{align}
%
meaning that 
%
\begin{align}
\widetilde{v}^{0} &= v^{0} - \frac{a'}{a} \alpha - \alpha '  \\
\widetilde{v}^{i} &= v^{i} - \beta^{, i\prime} - d '
\,,
\end{align}
%
so 
%
\begin{align}
\widetilde{v}_\parallel &= v_\parallel - \beta '  \\
\widetilde{v}^{i}_{\perp} &= v^{i}_\perp - d^i
\,.
\end{align}

The split of the cosmological perturbation is useful since (at linear order) the evolutions of the scalar, vector and tensor modes is decoupled.

The gauge is determined by \(\xi^{\mu }\), which amounts to choosing the scalars \(\alpha \) and \(\beta \), plus the vector \(d^{i} \): a total of four degrees of freedom, since \(d^{i} \) has one constraint. 

A possible gauge choice is the \textbf{Poisson gauge}:\footnote{Also known as the longitudinal or Newtonian gauge, since it reproduces the Newtonian treatment of the gravitational potential perturbations.}
%
\begin{align}
\widetilde{\omega}^{\parallel} = \widetilde{\chi}^{\parallel} = 0 = \widetilde{\chi}^{T}_i 
\,.
\end{align}

In this case, the lapse function quite closely matches the behaviour of Newtonian gravitational perturbations. 

In order to impose these condition we need to set
%
\begin{align}
0 = \widetilde{\chi}^{\parallel} = \chi^{\parallel} + 2 \beta = 0 \implies \beta = - \frac{1}{2} \chi^{\parallel}
\,.
\end{align}
%
% \todo[inline]{What does the bar indicate?}

This gauge is also known as the \textbf{orthogonal zero-shear gauge}.

If we consider constant-\(\eta \) spatial hypersurfaces, and take a vector \(N_\mu \propto \dv*{\eta }{x^{\mu }}\) perpendicular to them (this will often correspond to the four-velocity), we can make the following decomposition: 
%
\begin{align}
N_{\mu ; \nu } = \frac{1}{3} \theta h_{\mu \nu } + \omega_{\mu \nu } + \sigma_{\mu \nu } - a_\mu N_\nu 
\,,
\end{align}
%
where \(\theta = N^{\mu }_{;\mu }\) represents the local compression of the fluid, in FLRW \(\theta = 3 H\).
The term \(\omega_{\mu \nu }\) is given by 
%
\begin{align}
\omega_{\mu \nu } =  h^{\alpha }_{\mu } h^{\beta }_{\nu } u_{[\alpha ; \beta ]}
\,
\end{align}
%
and it is known as the \emph{vorticity tensor}. 
If indeed \(N_\mu = u_\mu \), then \(\omega_{\mu \nu } = 0\). 

The tensor \(\sigma_{\mu \nu }\) instead represents shear:
%
\begin{align}
\sigma_{\mu \nu } =   h^{\alpha }_{\mu } h^{\beta }_{\nu } u_{(\alpha ; \beta )} - \frac{1}{3} \theta h_{\mu \nu }
\,,
\end{align}
%
and finally \(a^{\mu } = u^{\nu } u_{\mu ; \nu }\).
If \(u_{\mu} = N_\mu\), \(\sigma_{\mu \nu } \) is the geometric shear, so \(\sigma_{\mu 0} = 0\), meaning that we can only work with 
%
\begin{align} \label{eq:zero-shear-gauge}
\sigma_{ij} &= D_{ij} \sigma + \sigma^{\perp}_{(i, j)} + \sigma^{T}_{ij}  \\
\sigma &= - \omega^{\parallel} + \frac{1}{2} \chi^{\parallel}
\,.
\end{align}

This term vanishes in the Poisson gauge.

The vector \(N\) reads:
%
\begin{align}
 N^{\mu } =(1/a) \qty[1 - \psi,  - \omega^{i}]
\,.
\end{align}
%

Another possible choice is the \textbf{synchronous}, or \textbf{time-orthogonal gauge}: 
%
\begin{align}
\widetilde{\psi} = \widetilde{\omega}^{\parallel} = \widetilde{\omega}^{\perp}_i = 0
\,,
\end{align}
%
meaning that \(\delta g_{0\mu } = 0 \). 
This amounts to only perturbing the spatial part of the metric.
If we take observers at fixed spatial coordinates, their subjective time in this gauge will be 
%
\begin{align}
\dd{\tau } = \dd{t} \qty(1 + \psi ) = \dd{t}
\,,
\end{align}
%
so they will all be synchronized. 

We can write 
%
\begin{align}
\widetilde{\psi} = \psi + \alpha ' + \frac{a'}{a} \alpha = \psi + \frac{(a \alpha )'}{a} = 0
\,,
\end{align}
%
so we must have 
%
\begin{align}
a \alpha = -\int a \psi \dd{\eta } + X(\vec{x})
\,,
\end{align}
%
so we have not completely fixed the gauge. 
Similarly to electrodynamics, we have residual gauge freedom. 

Another possible choice is the \textbf{comoving gauge}, 
defined by 
%
\begin{align}
v^{i} = 0
\,,
\end{align}
%
meaning that \(v^{\parallel} = v^{i}_\perp = 0\). This basically amounts to setting the coordinate system as comoving with the fluid.

We have an additional condition, which we can use to impose that constant \(\eta \) spatial hypersurfaces must be orthogonal to \(u^{\mu }\): this leads to the condition \(v^{\parallel} + \omega^{\parallel} = 0\), which tells us that \(\omega^{\parallel} = 0\) because of the previous condition. 

This is because at leading order we can lower the index of \(N^{\mu }\) and \(u^{\mu }\), and use \(v^{0}= \psi \), to get 
%
\begin{align}
N_{\mu } &= \qty[- a (1 + \psi ), 0]  \\
u_{\mu } &= a\qty[ - (1 + \psi ), v_i + \omega _i]
\,,
\end{align}
%
therefore the equation \(N_\mu = u_\mu \) reads \(v_i + \omega _i = 0\). 
This condition also leads to \(T^{0}_{i} = 0\): the momentum flux vanishes. 
This is also called the comoving orthogonal gauge. 

% \todo[inline]{is the condition not just \(\omega^{\parallel} = 0\)?}

Other options are the \textbf{spatially flat gauge}, also known as the uniform curvature gauge: this is found by selecting constant-\(\eta \) spatial hypersurfaces where the spatial metric is left unperturbed, at least for scalar and vector perturbations. 

This means 
%
\begin{align}
\widetilde{\phi} = \widetilde{\chi}^{\parallel} = \widetilde{\chi}^{\perp}_i = 0
\,,
\end{align}
%
and the reason for the name is that the intrinsic spatial curvature (Ricci scalar) of the constant-\(\eta \) hypersurfaces is 
%
\begin{align}
^{(3)}R = \frac{6k}{a^2} + \frac{rk}{a^2} \hat{\phi} + \frac{4}{a^2} \nabla^2 \hat{\phi}
\,,
\end{align}
%
where \(\hat{\phi} = \phi + (1/6) \nabla^2 \chi^{\parallel}\). 
This is why \(\hat{\phi}\) is sometimes called the \emph{curvature perturbation} itself, as \(R \propto \hat{\phi}\) in Fourier space.

The \textbf{uniform energy density gauge} is defined by \(\delta \widetilde{\rho} = 0\). 

There are two approaches we can take with regard to gauge freedom: we can choose a gauge and work within it in a self-consistent way. 
This is fine, but residual gauge freedom can create issues. 

The alternative is to only use gauge-invariant perturbations. 
A paper by Bardeen \cite[]{bardeenGaugeinvariantCosmologicalPerturbations1980a} showed how to do so.
Starting with scalars, we would have 4 of them: \(\psi, \phi , \omega^{\parallel}, \chi^{\parallel}\), however only 2 degrees of freedom can remain.
We define 
%
\begin{align}
2 \Psi  _A &= 2 \psi + 2 \omega^{\parallel, \prime} + 2 \frac{a'}{a} \omega^{\parallel} - \qty(\chi^{\parallel, \prime \prime} + \frac{a'}{a} \chi^{\parallel, \prime})  \\
2 \Phi_H &= -2 \phi - \frac{1}{3} \nabla^2 \chi^{\parallel} + 2 \frac{a'}{a} \omega^{\parallel} - \frac{a'}{a} \chi^{\parallel, \prime}
\,,
\end{align}
%
and it can be shown (exercise) that these are indeed gauge invariant. 
These quantities have a physical meaning: if we go to the zero-shear gauge \eqref{eq:zero-shear-gauge}, then \(\Phi _H =- \hat{\phi} \), while \(\Psi _A = \psi \).
These are known as the Bardeen gauge-invariant potentials, since they mimic Newtonian potentials. 

\end{document}
