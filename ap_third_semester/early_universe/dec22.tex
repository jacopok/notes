\documentclass[main.tex]{subfiles}
\begin{document}

\marginpar{Tuesday\\ 2020-12-22, \\ compiled \\ \today}

Today we will give a summary of certain topics we discussed in the course, mentioning some more advanced topics.
We have mentioned the evolution of a curvature perturbation on uniform energy density hypersurfaces \(\zeta \).
We defined it as the local fluctuation of the number of \(e\)-folds: \(\zeta = \delta N = - H \delta \varphi / \dot{\varphi}\). 
This is the \(\delta N\) formalism. 

It is an alternative method to study cosmological perturbations which works on super-horizon scales --- alternative to standard perturbation theory. 

Consider two disconnected regions with a typical length scale \(\lambda _s \gtrsim H^{-1}(t)\), and take their separation to be \(\lambda \gg \lambda _s\). 
In this viewpoint, \(\zeta \) takes into account how much one of these regions has expanded relative to the other.

We want to recover the result \(\zeta = \delta N\) and generalize it. 
The regions are separated by a distance much larger than the Hubble scale, so we expect them to be completely disconnected. 

The number of \(e\)-folds at a spatial position \(\vec{x}\) is given by 
%
\begin{align}
N = N(t, \vec{x}) = N(t, \varphi_*(\vec{x}))
\,,
\end{align}
%
where \(\varphi_* (\vec{x}) = \varphi^{0}_{*} + \delta \varphi (\vec{x})\) is the inflaton field, evaluated at the same initial time \(t_*\) corresponding to the time of horizon crossing for a generic fluctuation mode \(\lambda \sim 2 \pi / k\). 

% \todo[inline]{Does this not depend on the specific \(\lambda \)?}

We can Taylor expand: 
%
\begin{align}
\delta N = \eval{\pdv{N}{\varphi _*}}_{\varphi_*^{0}} \delta \varphi _*
+ \frac{1}{2} \pdv[2]{N}{\varphi _*} (\delta \varphi _*)^2
\delta \varphi^2_*
\,.
\end{align}

We can write the number of \(e\)-folds as 
%
\begin{align}
N = \int_{t_0 }^{t} H \dd{t} 
= \int_{\varphi _*}^{\varphi (t)}  \frac{H}{\dot{\varphi}} \dd{\varphi }
\,,
\end{align}
%
therefore 
%
\begin{align}
\pdv{N}{\varphi _*} = - \eval{\frac{H}{\dot{\varphi}}}_{\varphi _*} 
\implies
\delta N = \underbrace{- \eval{\frac{H}{\dot{\varphi}}}_{\varphi _*} \delta \varphi _*}_{\text{constant on superhorizon scales}}
\,.
\end{align}

This way we see that the term we wrote is the first order one, but we can also go beyond: the second-order term can be useful to study primordial non-Gaussianity. 

The lowest order statistic which can be used to characterize it is the three-point function, in the form \(\expval{\zeta \zeta \zeta }\). 

This can be computed in this formalism as 
%
\begin{align}
\qty(\pdv{N}{\varphi _*})^3 \expval{ \delta \varphi _{k_1 }^{*}\delta \varphi _{k_2 }^{*}\delta \varphi _{k_3 }^{*}} 
+ 
\qty(\pdv{N}{\varphi _*})^2 \pdv[2]{N}{\varphi _*}
\expval{
    \delta \varphi ^*_{k_1 }
    \delta \varphi ^*_{k_2 }
    (\delta \varphi _* \delta \varphi _*)_{k_3 }
} + \dots
\end{align}

Expectation values like these can arise from self-interactions in the field itself, or from interactions with other fields; in any case, nonlinear physics.
The notation \((\delta \varphi _* \delta \varphi _*)_{k_3 }\) denotes convolution in Fourier space.

A period of accelerated expansion will necessarily yield gravitational waves: \(\chi_{ij}^{T} = h_{ij}\), with two polarizations 
%
\begin{align}
h_{+, \times} = \sqrt{16 \pi G } \phi_{+, \times }
\,,
\end{align}
%
evolving according to 
%
\begin{align}
u_k'' + \qty(k^2 - \frac{a''}{a}) u_k = 0
\,.
\end{align}

The \(a'' / a\) term sources GWs from the quantum vacuum. 
Inflationary GWs would show up as a \(B\)-mode on the CMB polarization. 
However, we must be careful since there could be other sources of early GWs, such as cosmic strings or phase transitions. 

The excursion of \(\phi \) is related to the tensor-to-scalar ratio: 
%
\begin{align}
\frac{\Delta \phi }{M_P} \approx \qty( \frac{r}{\num{.01}})^{1/2}
\,.
\end{align}

Large field models, which are superPlanckian, have small \(r\), and it small field models have small \(r\). 

\todo[inline]{Add some more comments.}

What about exams? In January and February they will still be online oral examinations.
We have two options: we can do a standard orderly examination in which we are asked about the standard parts of the course, or (for those who have attended the course) choosing a macro-topic to prepare in detail --- ideally even in more detail than was done in the course --- and then talk mostly about it. Even if we choose this second option, a few broad questions will be asked about the rest of the course. 
The presentation should be a blackboard one.

Exams can be done outside the usual session. 

There will be a couple of extra lectures in January, possibly not by professor Bartolo. 

\end{document}
