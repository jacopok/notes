\documentclass[main.tex]{subfiles}
\begin{document}

\subsection{IN-IN formalism}

\marginpar{Monday\\ 2021-1-11, \\ compiled \\ \today}


In cosmology we directly observe correlation functions, as opposed to particle accelerators. 
In particle physics speak, they are VEVs of products of fields: here we will be interested in 
%
\begin{align}
Q(t) = \delta 
\phi _{\vec{k}_1}(t)
\phi _{\vec{k}_2}(t)
\phi _{\vec{k}_3}(t)
\,,
\end{align}
%
where \(\delta \phi \) is the inflaton perturbation. This is the \textbf{bispectrum}.

The IN-IN formalism gives us a formula like 
%
\begin{align}
\ev{Q(t)}{\Omega }
= \bra{0}
\qty[\overline{T} \exp(i \int_{t_0 }^{t} \dd{t'} H' _{\text{int}} (t'))]
Q'(t)
\qty[\overline{T} \exp(-i \int_{t_0 }^{t} \dd{t'} H' _{\text{int}} (t'))]
\ket{0}
\,.
\end{align}

Short description of the interaction picture for a field described by  Hamiltonian \(H = H_0 + H _{\text{int}}\).
Fields evolve according to the free time-evolution operator only. 
Then, we get 
%
\begin{align}
\expval{Q(t)}
&= \ev{Q(t)}{\Omega } = \ev{U^{-1}(t, t_0 ) Q (t_0 ) U(t, t_0 )}{\Omega } \\
&= \ev{U^{-1} (t, t_0 ) U _{\text{free}} (t, t_0 ) Q' (t) U^{-1} _{\text{free}} (t, t_0 ) U(t, t_0 )}{\Omega }  \\
&= \ev{F^{-1}(t, t_0 ) Q'(t) F (t, t_0 )}{\Omega }
\,,
\end{align}
%
where 
%
\begin{align}
\dv{F(t, t_0 )}{t} = -i H _{\text{int}} F(t, t_0 )
\,.
\end{align}

We can get a formal solution as the exponential of a time-ordered integral;
using Wick's theorem we can expand the exponential. 

How do we have \(\ket{0}\), the vacuum of the free theory, instead of \(\ket{\Omega }\)?

In QFT we usually have the \(S\)-matrix formalism, which is an ``in-out'' formalism; here instead we use an ``in-in'' one. 
We turn off the interaction theory in the far past; we can then replace \(\ket{\Omega }\) with \(\ket{0}\). 
We use the Bunch-Davies vacuum as \(\ket{0}\): we say that the quantum fluctuation on small scales are approximately plane waves.

As an example, we will compute the bispectrum for a model 
%
\begin{align}
S = \int \dd[4]{x} \sqrt{-g}
\qty[ \frac{M _{\text{Pl}}^2}{2} R 
- \frac{1}{2} g^{\mu \nu } \partial_{\mu } \phi (t, \vec{x}) \partial_{\nu } (t, \vec{x}) 
- V[\phi (t, \vec{x})]]
\,,
\end{align}
%
where \(M _{\text{Pl}}\) is the reduced Planck mass. 
For simplicity, let us take \(g_{\mu \nu } = a^2(\tau) \eta _{\mu \nu }\).

We expand 
%
\begin{align}
\phi (t, \vec{x}) &= \phi_0 (t) + \delta \phi (t, \vec{x})  \\
V(\phi ) &= V(\phi_0 ) 
+ \eval{\dv{V}{\phi }}_{\phi_0 } \delta \phi 
+ \frac{1}{2} \eval{\dv[2]{V}{\phi }}_{\phi_0 } \delta \phi^2  
+ \frac{1}{3!} \eval{\dv[3]{V}{\phi }}_{\phi_0 } \delta \phi^3  
\,.
\end{align}

We substitute these into the action (see slides) and the first term yields the unperturbed EOM: 
%
\begin{align}
\ddot{\phi}_0 + 3 H \dot{\phi}_0 + \dv{V}{\phi_0 } = 0
\,.
\end{align}

The second term vanishes.
We define the parameters 
%
\begin{align}
\eval{\dv[2]{V}{\phi}}_{\phi_0 } &= m_{ \delta \phi}^2 &
\eval{\dv[3]{V}{\phi}}_{\phi_0 } &= - \lambda  
\,.
\end{align}

We impose \(m_{ \delta \phi } = 0\) to simplify our discussion.
The Lagrangian density then becomes 
%
\begin{align}
\mathcal{L} = \underbrace{\frac{a}{2} \dot{\delta}_\chi^2
- \dot{a} \dot{\delta}_\chi \delta_\chi 
+ \frac{1}{2} \frac{\dot{a}^2}{a} \delta_{ \chi  }^2
- \frac{\delta^{ij}}{2a} \partial_{i} \delta _\chi \partial_{j} \delta _\chi }_{\mathcal{L} _{\text{free}}}
+ \underbrace{\frac{\lambda}{3!} \delta_\chi^3}_{\mathcal{L} _{\text{int}}}
\,,
\end{align}
%
in terms of \(\delta _\chi = a \delta \phi \). 

The EOM read
%
\begin{align}
\dots
\,.
\end{align}

The bispectrum is connected to the three-point function since we lose one independent momentum because of the requirement of momentum conservation. 

What follows is a long discussion on how to perform the integrals involved in the computation of the bispectrum.

\end{document}
