\documentclass[main.tex]{subfiles}
\begin{document}

Let us continue with our discussion of the power spectrum: the \textbf{spectral index} \(n_s\) is defined as\footnote{Note that the dimensionless power spectrum is sometimes denoted as \(\Delta^2\) and sometimes as \(\Delta \): here (in this section) we use the latter definition.}
%
\begin{align}
n_s - 1 = \dv{\log \Delta (k)}{\log k}
\,.
\end{align}

The index \(s\) means ``scalar''. 
In general this will depend on the wavenumber \(k\); it is a convenient description of the shape of the power spectrum. 

If \(n_s\) were a constant, then we would have a \emph{powerlaw} spectrum: \(\Delta (k) = \Delta (k_0 ) (k / k_0 )^{n_s - 1}\). 
If \(n_s = 1\), we have the \textbf{Harrison-Zel'dovich} power spectrum, for which \(\Delta \) does not depend on \(k\).
This would be a \emph{scale-invariant} power spectrum.

In a quantum-mechanical formalism, we will calculate the power spectrum as 
%
\begin{align}
\bra{0} \delta \varphi _{\vec{k}_1} \delta \varphi _{\vec{k}_2} \ket{0}
\,,
\end{align}
%
which will be written in terms of creation and annihilation operators: we have 
%
\begin{align}
\bra{0} a a \ket{0}=
\bra{0} a ^\dag a  \ket{0}=
\bra{0} a ^\dag a ^\dag \ket{0}= 0
\,,
\end{align}
%
while 
%
\begin{align}
\bra{0} a a ^\dag \ket{0} &= \underbrace{\bra{0} \qty[a, a ^\dag] \ket{0}}_{ \delta^{(3)} (\vec{k}_1 - \vec{k}_2)} - \underbrace{\bra{0} a ^\dag a \ket{0}}_{= 0} 
\,,
\end{align}
%
so 
%
\begin{align}
\expval{ \delta \varphi _{\vec{k}_1} \delta \varphi _{\vec{k}_2}} = (2 \pi )^3 \abs{ \delta \varphi _{\vec{k}_1}}^2 \delta^3(\vec{k}_1 - \vec{k}_2)
\,,
\end{align}
%
where \(\delta \varphi _{kl} = u_{kl} / a\). 
Therefore, the power spectrum is given by 
%
\begin{align}
P(k) = \abs{ \delta \varphi _k}^2
\,.
\end{align}

Recall that in the superhorizon case we found 
%
\begin{align}
\abs{ \delta \varphi _k} = \frac{H}{\sqrt{2 k^3}} \qty( \frac{k}{aH})^{3/2 - \nu }
\,,
\end{align}
%
where \(\nu^2 = 9/4 + 9 \epsilon - 3 \eta _V\), so \(\nu \approx 3/2 + 3\epsilon - \eta _V \), meaning that the index is \(3/2 - \nu = \eta _V - 3 \epsilon \). 

Then, the power spectrum reads 
%
\begin{align} \label{eq:scalar-field-power-spectrum}
\Delta_{ \delta \varphi }(k) = \frac{k^3}{2 \pi^2} \abs{ \delta \varphi _k}^2 = \qty( \frac{H}{2 \pi })^2 \qty( \frac{k}{aH})^{3-2\nu }
\,.
\end{align}
%
There will be a weak scale dependence proportional to the slow-roll parameters: \(3 - 2 \nu = 2 \eta _V - 6 \epsilon \).

\section{From \(\delta \varphi \) to primordial density perturbations}

The first Friedmann equation will read 
%
\begin{align}
H^2 = \frac{8 \pi G}{3} \rho_\varphi  \approx \frac{8 \pi G}{3} V(\varphi )
\,,
\end{align}
%
so the density fluctuation can be written as 
%
\begin{align}
\delta \rho _\varphi \approx V' (\varphi ) \delta \varphi \approx - 3 H \dot{\varphi} \delta \varphi 
\marginnote{See equation \eqref{eq:approx-EOM-scalar-field-flat-FLRW}}
\,.
\end{align}

Recall that we can define a time shift \(\delta t = - \delta \varphi / \dot{\varphi} \).
This means that we will have perturbations in the expansion of the universe from place to place.

The number of \(e\)-folds is given by 
%
\begin{align}
N = \log \qty( \frac{a(t)}{a (t_*)}) = \int_{t_*}^{t} H(\widetilde{t}) \dd{\widetilde{t}}
\,.
\end{align}

The fluctuations will perturb the number of \(e\)-folds by 
%
\begin{align} \label{eq:delta-N-definition}
\zeta = 
\delta N = H \delta t = - H \frac{ \delta \varphi }{\dot{\varphi}} \approx - H \frac{ \delta \rho _\varphi }{ \dot{\rho}_\varphi }
\,.
\end{align}

This is called the ``\(\delta N\) formalism'' for the study of large-scale perturbations.
The last equality in \eqref{eq:delta-N-definition} comes from the fact that 
%
\begin{align}
\dot{\rho}_\varphi = - 3H \qty(\rho _\varphi + P_\varphi ) = - 3 H \dot{\varphi}^2
\marginnote{From equation \eqref{eq:energy-momentum-scalar-field}.}
\,,
\end{align}
%
so indeed 
%
\begin{align}
H \frac{ \delta \rho _\varphi }{\dot{\rho}_\varphi } = \frac{- 3 H^2 \varphi  \delta \varphi }{-3 H \dot{\varphi}^2} = H \frac{ \delta \varphi }{\dot{\varphi}}
\,.
\end{align}

The quantity \(\delta N = \zeta\) is \textbf{gauge invariant}. It is written as 
%
\begin{align}
\zeta = - \hat{\Phi} - H \frac{ \delta \rho }{\dot{\rho} }
\,,
\end{align}
%
where \(\hat{\Phi}\) is related to scalar perturbations  of the spatial part of the metric, \(g_{ij}\). We shall explore this later. 

This \(\zeta\) is called the \textbf{curvature perturbation on uniform energy density hypersurfaces}.
Why did we write this with \(\rho \) instead of \(\rho _\varphi \)? This definition is completely general; it can be applied at any time and for a generic evolution of the universe. 
We can then specify it to 
%
\begin{align}
\zeta_\varphi \approx - H \frac{ \delta \rho _\varphi }{\dot{\rho}_\varphi }
\,.
\end{align}

What we will show is that on superhorizon scales \(\zeta\) remains constant (for single-field inflation, at least). 
Therefore, this keeps a sort of ``record'' of what happened after horizon crossing.

Let us denote as \(t^{(1)}_H(k)\) the time of horizon crossing during inflation for perturbations with wavenumber \(k\), and \(t^{(2)}_H(k)\) the time \emph{after inflation} of the second horizon crossing, when the perturbation comes back inside the horizon.  
The value of \(\zeta \) at these two times will be the same.
\todo[inline]{This can be seen since\dots}

We know that \(\delta \varphi \sim H / 2 \pi \) (statistically), and \(H^2 \approx 8 \pi G V(\varphi ) /3\): then, specifying the potential gives a prediction for the power spectrum. 

Suppose that the perturbation re-enters during the radiation-dominated epoch. Then, 
%
\begin{align}
H \frac{ \delta \rho }{\dot{\rho} } \approx \frac{H \delta \rho _\gamma }{-4 H \rho _\gamma } = - \frac{1}{4}\frac{ \delta \rho _\gamma }{\rho _\gamma }
\,,
\end{align}
%
since 
%
\begin{align}
\dot{\rho} = \dv{\rho }{a} \dot{a} = \frac{\rho}{a} \dv{\log \rho }{\log a} \dot{a} = \frac{\rho}{a} (-4) \dot{a} = - 4 \rho H
\,.
\end{align}
%

So, the dimensionless power spectrum is 
%
\begin{align}
\Delta_{ \delta \rho / \rho } (k)= 
\eval{\frac{H^2}{ \dot{\varphi}^2} \Delta_{ \delta \varphi } (k)}_{t^{(1)}_H (k)} 
\,,
\end{align}
%
and 
%
\begin{align}
\Delta_{ \delta \varphi } (k) = \qty( \frac{H}{ 2 \pi })^2 \qty(\frac{k}{aH})^{3-2 \nu }
\,,
\end{align}
%
so 

\todo[inline]{Recover a few minutes}

\end{document}
