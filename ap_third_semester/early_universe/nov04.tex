\documentclass[main.tex]{subfiles}
\begin{document}

\marginpar{Wednesday\\ 2020-11-4, \\ compiled \\ \today}

Last lecture we found 
%
\begin{align}\label{eq:density-perturbation-power-spectrum}
\Delta_{ \delta \rho / r } (k) 
= \qty(\frac{H^2}{2 \pi \dot{\varphi}})^2 \qty(\frac{k}{aH})^{3-2\nu }
\,,
\end{align}
%
where \(3 - 2 \nu = 2 \eta _V - 6 \epsilon = n_s -1\). 
The last equality comes from the definition 
%
\begin{align}
n_s - 1 = \dv{\log \Delta }{\log k}
\,.
\end{align}
%

A power spectrum with \(n_s = 1\) is called a Harrison-Zel'dovich spectrum, and it is scale-invariant.
If \(n_s > 1\) the spectrum is ``blue'', while if \(n_s < 1\) the spectrum is ``red'' (or, respectively, red or blue ``tilted''). 
In the two cases, we have either more energy at longer or shorter wavelengths.

This deviation from \(n_s = 1\) is a specific prediction of single-field models.
If we had \(n_s = 1\) exactly, we would expect there to be some symmetry which prevents \(n_s \neq 1\). 
The detection of a spectral index \(n_s \neq 1\) is a good indication of an inflation-like process. 

From CMB data \cite[eq.\ 21]{planckcollaborationPlanck2018Results2019} we have \(n_s = \num{.9649(42)}\) at a \SI{68}{\percent} CL.

Recall the definition of \(\zeta \): for \(k \ll aH\) it is constant, and it is given by 
%
\begin{align}
\zeta = \frac{1}{4} \frac{ \delta \rho _\gamma }{\rho _\gamma }
\,,
\end{align}
%
but we also know that \(\rho _\gamma \propto T^{4}\), so \(\zeta \sim \delta T / T\). 

\subsection{Power spectrum of primordial gravitational waves}

The two polarization states evolve like two scalar fields:
%
\begin{align}
h_{+, \times } = \sqrt{32 \pi G} \phi_{+, \times}
\,,
\end{align}
%
so 
%
\begin{align}
\Delta_{h_{+, \times }} (k) = 32 \pi G \Delta_{\phi_{+, \times }}
\,,
\end{align}
%
and we know that for a massless scalar field \(\phi  \)
%
\begin{align}
\Delta_{\phi} = \qty(\frac{H}{2 \pi })^{2} \qty(\frac{k}{aH})^{-2\epsilon }
\marginnote{See equation \eqref{eq:scalar-field-power-spectrum}.}
\,,
\end{align}
%
therefore 
%
\begin{align}
\Delta_{h_{+, \times }} (k)
&= \frac{32 \pi }{M_P^2} \qty(\frac{H}{2 \pi })^2 \qty(\frac{k}{aH})^{-2 \epsilon }  \\
&= \frac{8}{\pi M_P^2} H^2 \qty(\frac{k}{aH})^{-2 \epsilon }
\,.
\end{align}

Once again, we find a powerlaw as a function of \(k\). 
The amplitude is of the order \(H^2 / M_P^2\), but \(H\) is determined by the vacuum energy density of the inflaton, so 
%
\begin{align}
\frac{H^2}{M_P^2} = \frac{8 \pi }{3} \frac{V(\varphi )}{M_P^4}
\,.
\end{align}

If we define the characteristic energy of inflation as \(E_{\text{inf}} = V^{1/4}\) (recall that \(V\) is an energy density, with dimensions of a mass to the fourth power) we find that the amplitude of the GW spectrum is of the order \(E _{\text{inf}} / M_P\). 
Measuring the primordial SGWB amplitude gives us directly the energy scale of inflation. 

The ``tensor spectral index'' is defined as 
%
\begin{align}
n_T = \dv{\log \Delta _T}{\log k} = - 2 \epsilon 
\,,
\end{align}
%
where \(\Delta _T\) is the total tensor amplitude: we just by 2 to account for the two polarizations,
%
\begin{align}
\Delta_T = \frac{16}{\pi M_P^2} H^2 \qty(\frac{k}{aH})^{-2\epsilon }
\,.
\end{align}

In single-field models \(\epsilon > 0\) always, therefore \(n_T \) will always be red-tilted. 

We can also write 
%
\begin{align}
\Delta_{h_{+, \times }} = \frac{8}{M_P^2 \pi } H^2_*
\,,
\end{align}
%
where the star means we calculate the Hubble rate at the time of horizon crossing during inflation. 

This is because the fluctuation remains constant when it is on superhorizon scales: so, its value is set by whatever it is at horizon crossing, \(k = aH\).
The term \(k / aH\) then simplifies, and we find the aforementioned expression.

We need to define a few observationally relevant quantities. 
The \textbf{tensor-to-scalar ratio} \(r\) is defined as 
%
\begin{align}
r = \frac{\Delta _T}{\Delta _\zeta }
= \frac{ \frac{16}{M_P^2 \pi } H^2_*}{\Delta _\zeta }
\,.
\end{align}

This measures the level of primordial GWs compared to the amount of primordial scalar perturbations. 
From CMB temperature anisotropies we can measure \(\zeta \), since at large scales \(\zeta \sim \delta T / T \sim \num{e-5}\). Therefore, \(\Delta_\zeta \sim \zeta^2 \sim \num{e-10}\) (this is known to a better precision, we just give an order of magnitude here).

The energy scale of inflation then can be written as 
%
\begin{align}
E _{\text{inf}} \approx \SI{e16}{GeV} \qty(\frac{r}{\num{.01}})^{1/4}
\,.
\end{align}

Measuring \(r\) can allow us to probe extremely high energy scales.
The energy scale is reminiscent of Grand Unification Theories. 
In modern models the connection to GUT theories is less strong.

The latest measurements, using \(B\)-mode polarization of CMB photons, which is characteristic of tensor perturbations, give a bound \(r < \num{.044}\) at a \SI{95}{\percent} CL. This number also includes \(B\)-mode data.
In the future we hope to be able to measure \(r \sim \num{e-2} \divisionsymbol \num{e-3}\). 

The overall amplitude of tensor perturbations is 
%
\begin{align}
\Delta _T = \frac{16}{\pi M_P^2} H^2_*
\,,
\end{align}
%
so 
%
\begin{align}
\Delta_{\zeta } = \frac{H^2}{4 \pi^2} \frac{H^2}{\dot{\varphi}^2} 
\marginnote{See equation \eqref{eq:density-perturbation-power-spectrum}.}
\,,
\end{align}
%
and \(\dot{H} = - 4 \pi G \dot{\varphi}^2\) (see \eqref{eq:derivative-Hubble-parameter-scalar-field}). 
\todo[inline]{Check: this is an exact expression, we just differentiate \(H^2= \frac{8 \pi G}{3} \qty( \dot{\varphi}/2 + V(\varphi ))\) and the use the equation of motion.}
This means that 
%
\begin{align}
\epsilon  = - \frac{\dot{H}}{H^2} = + 4 \pi G \frac{\dot{\varphi}^2}{H^2}
\,.
\end{align}

So, 
%
\begin{align}
\frac{H^2}{\dot{\varphi}^2} = \frac{4 \pi }{M_P^2} \frac{1}{\epsilon } \implies 
\Delta_\zeta = \frac{H^2}{4 \pi } \frac{4 \pi }{M_P^2} \frac{1}{\epsilon } = \frac{H^2}{\pi M_P^2 } \frac{1}{\epsilon }
\,.
\end{align}

This allows us to express the tensor-to-scalar ratio as 
%
\begin{align}
r =\frac{\Delta _T}{\Delta _\zeta } = 16 \epsilon 
\,. 
\end{align}

This gives us the \emph{consistency relation} for single-field models: \(n_T = - 2 \epsilon\), therefore \(r = - 8 n_T\). 
It relates two observable variables: it can be used as a check. 
This is why it is called the ``Holy Grail of cosmology''. If we saw it holds, we would be rather sure that inflation actually occurred. 

This is extremely difficult to measure, it would be very hard to see even \(r\), \(n_T\) typically will have much larger errorbars. 


\subsection{Base observational predictions}

We predict a scalar power spectrum 
%
\begin{align}
\Delta_{\zeta } (k) = \Delta_\zeta (k_0 ) \qty( \frac{k}{k_0 })^{n_s-1}
\,,
\end{align}
%
where \(k_0 \) is the \emph{pivot scale}. 
In the Planck data release \cite[]{planckcollaborationPlanck2018Results2019} the value of \(r\) is reported as \(r_{\num{.002}}\), where the index corresponds to the pivot scale \(k_0  = \SI{.002}{Mpc^{-1}}\).
The spectral index as usual is \(n_s - 1 = 2 \eta _V - \epsilon \). 

Also, 
%
\begin{align}
\Delta _T (k) = \Delta _T (k_0 ) \qty(\frac{k}{k_0 })^{n_T}
\,,
\end{align}
%
where \(n_T = - 2 \epsilon \). 
At the pivot scale we have 
%
\begin{align}
\Delta _\zeta (k_0 ) &= \eval{\frac{H^2}{\pi M_P^2 \epsilon }}_{k_0 }  \\
\Delta _T (k_0 ) &= \eval{\frac{16 H^2}{\pi M_P^2} }_{k_0 }
\,.
\end{align}

The value of \(H\) is determined by the potential \(V(\varphi )\), while the value of \(\epsilon \) depends on the derivative \(V' (\varphi )\), and the value of \(\eta _V\) depends on the second derivative \(V''(\varphi )\): recall 
%
\begin{align}
H^2 &= \frac{8 \pi G}{3} V(\varphi )  \\
\epsilon &= \frac{1}{16 \pi G} \qty(\frac{V'}{V})^2  \\
\eta _V &= \frac{1}{8 \pi G} \qty(\frac{V''}{V})
\,.
\end{align}
%

If we are able to constrain the two (scalar and tensor) amplitudes and spectral indices we can reconstruct the inflationary potential.  
We can reduce this four-parameter space: on the largest angular scales, \(\Delta T / T \sim \Delta _\zeta + \Delta _T\). This is very imprecise, there are transfer functions involved, however the basic idea is that there are contributions from both terms. 

But \(\Delta T / T\) is well-known, so instead of the two amplitudes we can just measure \(r = \Delta _T / \Delta _\zeta \). 
If we assume that the consistency relation holds, we have \(r = - 8 n_T\)
which reduces the parameter space to two parameters only: \((r, n_s)\). 
This is commonly used in order to visualize inflationary models, since we can plot contours in a plane. 
There, we can show observational constraints as in the lower row of figure 26 in the Planck 2018 release \cite[]{planckcollaborationPlanck2018Results2019}.

Large-field models have \(0 < \eta _V < 2 \epsilon \), while small field models have \(\eta _V < 0\). 
% Recall that \(\eta _V = V'' / (3 H^2)\), so 
Generally, small-field models predict low levels of GWs: we can express 
%
\begin{align}
n_s -1 = 2 \eta _V - 6 \epsilon = 2 \eta _V - \frac{3}{8} r
\,,
\end{align}
%
so we can express \(r = r (n_s, \eta _V )\). 
For more details see \textcite[]{kinneyNewConstraintsInflation2000}.
Small field models have \(V'' < 0\), large field models have \(V'' > 0\). 

\subsubsection{Some examples of inflationary models}

Large field models could be \(V \propto \phi^{p}\) (a chaotic inflation scenario) or \(V \propto \exp(\phi / \mu )\) (powerlaw inflation).\footnote{Note that inflation cannot be exactly De Sitter! It would never end.} These have \(0 < \eta _V < 2 \epsilon \). 

Small field models could be \(V \propto
1 - (\phi / \mu )^{p}\) (from SSB of, say, axion models) or \(V \propto 1 + (\phi / \mu )^{p}\) (from SUSY, often involving a second field). 
These have \(\eta _V < 0 \). 

There are also hybrid models, mixing small-field and large-field characteristics.
These are among the earlier examples of two-field models: \(\phi \) starts off similarly to large-field models, and moves towards a minimum with a nonzero VEV, provided by a second field \(\psi \), which is stuck to the minimum of its potential.

This second field is necessary for the model but it is not dynamical in this first phase, however it becomes unstable at the end of inflation, triggering it. 
These models have \(\eta _V > 2 \epsilon \). 

We can have these kinds of models arising from adding an \(R^2\) term to the Einstein-Hilbert action. 
This corresponds to a new degree of freedom, with a transformation \(g_{\mu \nu } \to e^{-2 \omega } g_{\mu \nu }\) and a field \(\phi  \propto \omega \). 
The potential is in the form 
%
\begin{align}
V \propto \qty(1 - e^{-2 \omega })^2
\,.
\end{align}

We also have \emph{natural inflation}, with \(V \propto 1 - \cos(\phi / \mu )\): this is related to a shift symmetry \(\phi \to \phi + c\). 
If this symmetry were exact we would have \(V = \const\), but if instead it is broken we have the aforementioned potential. 
These are axion-like models: the shift symmetry arises naturally if \(\phi \) is a phase, corresponding to a \(U(1)\) symmetry like the Peccei-Quinn one.
The inflaton would not be the axion itself, but they would have similar properties. 



\end{document}
