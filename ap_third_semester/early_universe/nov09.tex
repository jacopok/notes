\documentclass[main.tex]{subfiles}
\begin{document}

\section{An overview of early inflationary models}

\marginpar{Monday\\ 2020-11-9, \\ compiled \\ \today}

The main reason to study these despite their issues is that they informed the modern viewpoint on inflation, which we will understand better after studying them.

\subsection{Old inflation}

This is a model first proposed by Guth in 1981.
The idea was to directly link inflationary dynamics to some high temperature SSB phase transition. 

We consider the usual SSB potential  
%
\begin{align}
V(\varphi ) = \frac{\lambda}{4} \qty(\varphi^2 - \sigma^2)^2
\,,
\end{align}
%
for a real scalar field \(\varphi \): the symmetry \(\varphi \to -\varphi \) is spontaneously broken in the vacuum for positive \(\sigma^2\). 
Accounting for one-loop quantum corrections at a finite temperature \(T\), we find the effective potential 
%
\begin{align}
V(\varphi , T) = V(\varphi ) 
+ \alpha  \varphi^2 T^2
+ \gamma (\varphi^2)^{3/2} T
+ \beta T^{4} 
\,.
\end{align}

The potential is written in a funny way in order for it to still be explicitly symmetric under \(\varphi \to - \varphi \).  
At high temperatures these corrections will dominate over \(V(\varphi )\); at low temperatures the most important correction will be the quadratic one in \(T\). 

As the temperature decreases, beyond the minimum at \(\varphi = 0\) we get another one at \(\varphi = \pm \varphi_c\). 
This minimum gets lower and lower, and at \(T = T_c\) it goes below \(\varphi (0)\), becoming the true vacuum. 

\todo[inline]{Insert figure(s)}

The vacuum at \(\varphi = 0\) is called the ``false vacuum'', while the one at \(\varphi = \pm \sigma \) is called the ``true vacuum''.

Starting at a high temperature, we need to consider the energy density of radiation, which we know is given by 
%
\begin{align}
\rho _r (T) = \frac{ \pi^2}{30 } g_* (T) T^{4}
\,,
\end{align}
%
together with the energy density of the scalar field, which is initially trapped at \(\varphi = 0\), so that \(\rho_0 = \lambda \sigma^{4} /4\). 
Then, the first Friedmann equation will read 
%
\begin{align}
H^2= \frac{8 \pi G}{3} \qty(\rho_0 + \rho _r)
\,,
\end{align}
%
ant the contribution \(\rho_0 \), which acts like an effective cosmological constant, will eventually come to dominate as the temperature decreases: when \(\rho_0 > \rho_r\) inflation starts its De Sitter phase. 
As this begins, the energy density of radiation will decrease exponentially. 

When the potential barrier becomes low enough, the field \(\varphi \) can undergo quantum tunneling, moving to the true vacuum. 
This kind of phase transition is called a \emph{first order} phase transition, as opposed to \emph{second order} ones.\footnote{Roughly speaking, a first-order transition involves a discontinuity in the energy, while in the second-order case it is merely nondifferentiable, while remaining continuous.}

These models have the so-called \textbf{graceful exit problem}.  
The transition to the true vacuum will generally take place in bubbles, which are moving in an exponentially-expanding false-vacuum background. 
So, they may never meet, thus being unable to explain observations. 

Also, after inflation we need reheating: this may be explained by the latent heat of the field having fallen in the true vacuum, \(\Delta V = V(0 ) - V(\sigma )\). 
Still, though, this heat is trapped in the bubbles, and the universe cannot thermalize efficiently.

\subsection{New inflation}

These problems were solved by new inflation, proposed by Linde in 1982 and by Albrecht and Steinhardt in the same year.
Here, the transition becomes a \textbf{second order} one: there is no potential barrier. 

During the phase transition, as the derivative \(V'' (\varphi =0)\) changes sign the field can start \emph{slow-rolling} towards its minimum. 

Since there is no barrier, there is no nucleation of bubbles either.
What happens instead is described by \textbf{spinodal decomposition}: basically, there are different SSB in different domains which then inflate to encompass the universe.

What are the \textbf{problems} of new inflation?
We would like to ask for \(\expval{ \delta \varphi^2} \ll \varphi_0^2\): the fluctuations of the scalar field should be small, in order to be treated perturbatively. 
This is hard to ensure. This is reflected in the scale of density perturbations, \(\delta \rho / \rho \). 

For these models of inflation we find \(\delta \rho / \rho \sim \order{\lambda^{1/2}}\), where \(\lambda \) is the coupling constant between the inflaton field and the thermal bath.

Typically people considered values for \(\lambda \) between \num{.1} and \num{1}, which means (also accounting for the constant in front) we would get anisotropies of the order of 100:
way too large, since from CMB anisotropies we know that \(\delta \rho / \rho \sim \num{e-5}\).
In order to match this observational value we would need \(\lambda \sim \num{e-10}\), however this is not compatible with us being sure to have thermal equilibrium between the inflaton reheating and its thermal bath. 

Thermal equilibrium and sufficiently small density perturbations seem to be in conflict with each other. 

This is the reason why \textbf{chaotic inflationary models} (Linde 1983): disconnecting inflation from phase transitions. 
Suppose we have a flat potential \(V(\varphi ) \equiv V_0 \). 
This would lead to unending De Sitter inflation with \(- \infty < \varphi < \infty \); so instead of it we consider a roughly flat potential \(V(\varphi ) = (\lambda /4) \varphi^{4}\), with \(\lambda \ll 1\). 

The constraint we need to require is that \(V(\varphi ) \lesssim M_P^{4}\), which ensures that quantum gravity corrections are not needed.
Then, we must constrain \(\varphi \) to lie between plus and minus \(M_P / \lambda^{1/4}\). 
But if \(\lambda \ll 1\), \(\varphi \) will be able to attain initial values larger than the Planck mass: this allows for slow-roll inflation. 

These models are called chaotic because the values of \(\varphi \) are taken to be randomly distributed, and in some regions inflation will not take place. 
At the Planck time \(t = t_P\) this is required: there, \(\Delta E \Delta T \gtrsim 1\), so the potential \(V(\varphi )\) must have an uncertainty larger or equal than \(M_P\). 

Now having \(\lambda \sim \num{e-10}\) is not an issue: the inflaton is not coupled, and we are fine with this. 

The high temperature corrections to this potential are negligible.

As an example we consider the \textbf{Coleman-Weinberg potential}: 
%
\begin{align}
V(\varphi ) = \frac{B \sigma^{4}}{2}
+ B \varphi^{4} \qty[\log \qty(\frac{\varphi^2}{\sigma^2}) - \frac{1}{2}]
\,,
\end{align}
%
a typical potential which can characterize symmetry breaking. 
The peculiarity of this potential is the logarithm: it comes from the one-loop corrections for the interactions between the field and other particles. 

This has a minimum at \(\varphi = \sigma \), while near the origin it is quite flat: we can approximate it as 
%
\begin{align}
V(\varphi ) \approx \frac{B \sigma^{4}}{2} - \lambda \frac{\varphi^{4}}{4} 
\,,
\end{align}
%
where \(\lambda = \abs{4 B \log (\varphi^2 / \sigma^2)} \approx \qty(\num{10} \divisionsymbol \num{100}) B\). 

Very close to the origin, we will have \(V(\varphi ) \approx B \sigma^{4} / 2 = \const\), so 
%
\begin{align}
H^2 \approx \frac{8 \pi G}{3} V(\varphi ) \approx \frac{4 \pi G}{3} B \sigma^{4}
\,.
\end{align}

In the original model of new inflation, this was considered with \(B \approx \num{e-3}\), and \(B = \frac{25}{16} \alpha^2 _{\text{GUT}}\), and \(\alpha _{\text{GUT}} = g _{\text{GUT}}^2 / 4 \pi \). 

Since it pertained to a GUT, we also had \(\sigma \sim T_C \sim \SI{e15}{GeV}\), therefore the Hubble parameter is approximately given by 
%
\begin{align}
H^2 \approx \frac{8 \pi }{3 M _{\text{Pl}}^2} V 
\approx \frac{4 \pi }{3} B \frac{\sigma^{4}}{M _{\text{Pl}}^2}
\approx \num{e-2} \frac{\qty(\SI{e15}{GeV})^{4}}{\qty(\SI{e19}{GeV})^{2}} \approx \qty(\SI{e10}{GeV})^2
\,.
\end{align}

Do these kinds of models actually work? The first question is: can they solve the horizon problem? The number of \(e\)-folds is given by 
%
\begin{align}
N = \int _{t_i}^{t_F} H \dd{t} = \int_{\varphi _i}^{\varphi _f} \frac{H}{\dot{\varphi}} \dd{\varphi }
\,,
\end{align}
%
and recalling \(3 H \dot{\varphi} \approx V' (\varphi )\), plus \(H^2= 8 \pi G V(\varphi ) /3\), we find 
%
\begin{align}
N = - 8 \pi G \int_{\varphi _i}^{\varphi _f} \frac{V(\varphi )}{V'(\varphi )} \dd{\varphi }
\,.
\end{align}

Using the near-origin approximation, we get \(V'(\varphi ) \approx - \lambda \varphi^3\), so 
%
\begin{align}
N &\approx  - 3 H^2 \int_{\varphi _i}^{\varphi _f} \frac{ \dd{\varphi }}{- \lambda \varphi^3}   \\
&\approx 3 \lambda^{-1} H^2 \int_{\varphi _i}^{\varphi _f} \frac{ \dd{\varphi }}{\varphi^3} = \frac{3}{2} \frac{H^2}{\lambda } \qty( \frac{1}{\varphi _i^2} - \frac{1}{\varphi_f^2})
\,.
\end{align}

When does inflation end?
We know that slow-roll holds as long as \(\abs{V''} < H^2\), so we can say that \(\varphi _f\) is determined by \(\abs{V''(\varphi _f)} \sim 10 H^2\), and we know that \(V''(\varphi ) = - 3 \lambda \varphi^2\): 
therefore, \(\abs{\varphi _f} = H / \sqrt{3 \lambda }\). 

If \(B \approx \num{e-3}\), this means 
%
\begin{align}
\varphi _f^2 \approx \frac{3 H^2}{\lambda } \approx \qty(30 \divisionsymbol 300) H^2 \gtrsim \qty(\SI{e11}{GeV})^2
\,.
\end{align}

Then, an initial condition which works is \(\varphi _i \approx \num{e8} \divisionsymbol \SI{e9}{GeV}\). 
This means that \(\varphi _i \sim H / 10\), which means that \(N \gtrsim 1000\) easily.
So, we can solve the horizon and flatness problems without issue. 

However, we will see that there are indeed problems with this model. 
The amplitude of the primordial fluctuations is too large, the quantum fluctuations of the inflaton are too large, invalidating the semiclassical approach. 

\end{document}
