\documentclass[main.tex]{subfiles}
\begin{document}

\marginpar{Wednesday\\ 2020-11-11, \\ compiled \\ \today}

Last time we discussed the Coleman-Weinberg potential: 
%
\begin{align}
V(\varphi ) = \frac{B \sigma^{4}}{2} - B \varphi^{4}
\qty[\log \qty(\frac{\varphi^2}{\sigma^2}) - \frac{1}{2}]
\,,
\end{align}
%
where typically \(B \sim \num{e-3}\). 
It can be approximated for \(\varphi \ll \sigma \) as 
%
\begin{align}
V(\varphi ) \approx \frac{B \sigma^{4}}{2} - \frac{\lambda}{4} \varphi^{4}
\,,
\end{align}
%
where \(\lambda \sim (10 \divisionsymbol 100 )B \sim \num{.1}\), and we have seen that if the initial value of \(\varphi \) is of the order of \num{e8} to \SI{e9}{GeV}, corresponding to roughly \(H/10\) then we can get a total number of \(e\)-folds of more than 1000. 

Let us now move to \textbf{problems} with this model. 
We must guarantee that the variance of the fluctuations of the scalar field must be smaller than the classical trajectory: 
%
\begin{align}
\expval{ \delta \varphi^2} \ll \varphi_0^2 (t)
\,,
\end{align}
%
otherwise we cannot meaningfully treat the problem in a perturbative way. 
This is violated, since \(\delta \varphi \sim H / 2\pi \) is comparable to \(\varphi _i \sim H / 10\). 
This might destroy inflation. 

We have 
%
\begin{align}
\frac{ \delta \rho }{\rho } \approx \frac{H^2}{\dot{\varphi}} \approx \frac{3 H^3}{V'(\varphi )} \approx \frac{3}{\lambda } \qty(\frac{H}{\varphi })^3
\marginnote{See equation \eqref{eq:density-perturbation-power-spectrum}.}
\,,
\end{align}
%
since \(\dot{\varphi} = - V'(\varphi ) / 3H\). When \(\varphi \ll \sigma \), \(V' (\varphi ) \approx - \lambda \varphi^3\).

This will be of order unity at least! 

Let try to make it so the number of \(e\)-folds \(N_\lambda (\varphi \to \varphi _f)\) is at least 50 or 60: this is given by 
%
\begin{align}
N_\lambda (\varphi \to \varphi _f) &= \frac{3}{2} \frac{H^2}{\lambda } \qty( \frac{1}{\varphi^2} - \underbrace{\frac{1}{\varphi _f^2}}_{\text{negligible}})  \\
\frac{H}{\varphi } &\approx  \lambda^{1/2} \qty( \frac{2}{3})^{1/2} N_\lambda^{1/2}
\,,
\end{align}
%
so the density perturbation will be 
%
\begin{align}
\frac{ \delta \rho }{\rho } \approx \lambda^{1/2} N_\lambda^{3/2} (8/3)^{1/2}
\,,
\end{align}
%
so if \(\lambda \approx \num{.1}\) and indeed \(N_\lambda \sim 50 \divisionsymbol 60 = N _{\text{CMB}}\) we get 
%
\begin{align}
\frac{ \delta \rho }{\rho } \sim 100
\,.
\end{align}

This is in stark contrast to the CMB observation: \(\delta \rho /\rho \sim \num{e-5}\). 

So, we get a contradiction: if \(\lambda \) is relatively large the density perturbations are huge, and in order to have the observed density perturbations we would need to have it tiny (\(\lambda \sim \num{e-10}\)), which would prohibit thermal equilibrium, which we assumed. 

\textbf{Chaotic} models of inflation attempted to solve these problems: they used 
%
\begin{align}
V(\varphi ) = \frac{\lambda}{4} \varphi^{4}
\,,
\end{align}
%
and they are \emph{large-field} models. 
These models do not require coupling between the inflaton and other fields; however here we also find a relation 
%
\begin{align}
\frac{ \delta \rho }{\rho } \sim \lambda^{1/2} N_\lambda^{3/2}
\,.
\end{align}

Something about the Helmholtz free energy in the finite-temperature corrections to the potential \(V(\varphi ; T)\). 

Hybrid models of inflation: we take a potential in the shape 
%
\begin{align}
V(\varphi ) = V_0 \qty[1 + \qty(\frac{\varphi }{\mu })^{p}]
\,,
\end{align}
%
so that the potential is rolling down a potential towards a nonzero vacuum energy. 

These were the first models for which the potential depends on two fields, so that the aforementioned potential for \(\varphi \) is an effective one: the full potential is 
%
\begin{align}
V(\varphi , \psi ) = \frac{1}{2} m^2\varphi^2+ \frac{\lambda}{4} \qty(\psi^2-M^2)^2 + \frac{\lambda'}{2} \varphi^2 \psi^2
\,.
\end{align}

If we evaluate it at \(\psi = 0\) we find 
%
\begin{align}
V(\varphi , \psi = 0) = \frac{1}{2} m^2\varphi^2 + \frac{\lambda}{4} M^{4}
\,,
\end{align}
%
in the same form as the aforementioned potential, with \(V_0 = \lambda M^{4} /4\). 

The effective mass of \(\psi \), only considering the quadratic coupling, is
%
\begin{align}
\pdv[2]{V}{\psi } = m^2_\psi = \lambda ' \varphi^2 = m^2_\psi (\varphi )
\,,
\end{align}
%
so the square mass of \(\psi \) reads 
%
\begin{align}
\eval{m^2_\psi}_{\psi = 0} = - M^2 + \lambda ' \varphi^2
\,,
\end{align}
%
and it can be positive or negative! The critical value of \(\varphi \) for which it has zero mass is given by 
%
\begin{align}
\varphi _c = \qty( \frac{\lambda }{\lambda'})^{1/2} M
\,,
\end{align}
%
for \(\varphi > \varphi _c\) we have \(\eval{m^2_\psi }_{\psi = 0} > 0\).

When \(\varphi \) becomes such that \(\varphi < \varphi _c\) then the point \(\psi = 0\) becomes unstable, so \(\psi \) moves from zero to either \(+\) or \(-M\).  

We get a \emph{second-order} phase transition for \(\psi \). 
\todo[inline]{Add graph for \(V (\psi , \phi )\).}

\subsection{Some remarks on inflation model building}

What are the loop contributions to the tree-level potential?
The tree-level potential can look like 
%
\begin{align}
V(\varphi ) = V_0 \pm \frac{1}{2} m^2 \varphi^2 + \frac{\lambda}{4} \varphi^{4} + \frac{\lambda '}{\Lambda^2} \varphi^{6} + \dots
\,,
\end{align}
%
but the UV completion (the sixth-power term and beyond) must be suppressed with a large cutoff scale \(\Lambda \). 

Loop corrections can change the parameters, but it can also add whole new terms (like the log in the Coleman-Weinberg potential). 
These can be useful but also dangerous, for example spoiling the flatness needed for slow-roll inflation. 

Typically, we will have something like 
%
\begin{align}
\Delta V^{\text{one-loop}}(\varphi )
= \sum _{i} \frac{\pm N_i}{64 \pi^2}
M_i^{4} (\varphi ) \log \qty(\frac{M_i^2 (\varphi )}{\mu^2})
\,,
\end{align}
%
where \(M_i^2(\varphi )\) is the effective mass of the \(i\)-th particle species which interacts with \(\varphi \). 
This logarithm generates the one in the Coleman-Weinberg potential.

The vacuum energy is given by 
%
\begin{align}
E^{\psi } _{\text{vacuum}} = \frac{1}{2} \int  \frac{ \dd[3]{p}}{(2 \pi )^3}
\sqrt{p^2 + M^2_\psi (\varphi )}
\,.
\end{align}

Let us make some examples: if we have two scalar fields 1 and 2 with which \(\varphi \) interacts we might get terms in the form 
%
\begin{align}
\Delta V(\varphi ) \sim m_1^2 m_2^2 \log \qty( \frac{\varphi}{\mu })
\,,
\end{align}
%
so starting from a tree-level flat potential \(V(\varphi ) = V_0 \) we might get 
%
\begin{align}
V(\varphi ) = V_0 \qty(1 + \alpha  \log (\frac{\varphi }{\mu }))
\,,
\end{align}
%
where 
%
\begin{align}
\alpha = \frac{m_1^2 m_2^2}{V_0 }
\,.
\end{align}
%

Another example which is similar to Coleman-Weinberg is  
%
\begin{align}
\Delta V(\varphi ) \sim \frac{c}{2} \qty(m_1^2+ m_2^2) \varphi^2 \log \frac{\varphi }{\mu }
\,,
\end{align}
%
so 
%
\begin{align}
V(\varphi ) &= V_0 + \frac{\varphi^2}{2} \qty[m^2 + c \widetilde{m}^2 \log \frac{\varphi }{\mu }]
\,,
\end{align}
%
which can arise from a tree-level potential like \(V_0 + m^2 \varphi^2 / 2\), where we defined \(\widetilde{m}^2 = m_1^2 + m_2^2\). 

This might be observationally detected through a \textbf{running of the spectral index}: \(n_s = n_s (k)\). 

\paragraph{The \(\eta \) problem}

Suppose we have a potential \(V\) depending on \(n\) complex scalar fields: 
%
\begin{align}
V = V(\varphi _n, \overline{\varphi} _n) = e^{k(\varphi _n, \overline{\varphi}_n) / M _{\text{pl}}^2} \widetilde{V} (\varphi _n, \overline{\varphi}_n)
\,,
\end{align}
%
which is commonly called the Kähler potential --- it arises in the context of supergravity theories. 
It arises from a Lagrangian in the form 
%
\begin{align}
\mathscr{L} = \partial_{\mu } \overline{\varphi}_{m} K_{\overline{m} n} \partial^{\mu } \varphi_n
\,,
\end{align}
%
where 
%
\begin{align}
K_{\overline{m}} = \pdv{K}{\overline{\varphi}_m} &&
K_{n} = \pdv{K}{\varphi_n} 
\,.
\end{align}

If \(K = \sum _{m} \abs{\varphi _m}^2\) we find simply \(K_{\overline{m} n} = \delta_{nm}\); then 
%
\begin{align}
\eta _V = \frac{1}{3} \frac{m^2}{H^2} = \frac{1}{3} \frac{V''(\varphi )}{H^2}
\,.
\end{align}

Let us identify \(\varphi \), the inflaton, with the real part of one of the fields: \(\varphi = \Re \varphi _n\).  

Then, 
%
\begin{align}
V = e^{K / M_P^2} \widetilde{V}
\,,
\end{align}
%
so that 
%
\begin{align}
\pdv{V}{\varphi } = e^{K / M_p^2} \frac{2 \varphi }{M_P^2} \widetilde{V} + e^{k / M_P^2} \widetilde{V}'
\,,
\end{align}
%
therefore the square mass of the field reads 
%
\begin{align}
m_\varphi^2
= \eval{\pdv[2]{V}{\varphi }}_{\varphi = 0} 
= \frac{2V}{M_P^2} + e^{K / M_P^2} \widetilde{V}''
\,.
\end{align}

Then, using \(H^2 \approx \frac{8 \pi G}{3} V(\varphi )\) we get 
%
\begin{align}
\eta _V = \frac{1}{3} \frac{m^2}{H^2} \sim \frac{m^2 M_P^2}{V}
\,.
\end{align}

This yields 
%
\begin{align}
\eta _V = 1 + M_P^2 \frac{\widetilde{V}''}{\widetilde{V}}
\,,
\end{align}
%
which is a problem if we want slow-roll inflation. 
\todo[inline]{Possibly not a 1 here.}

If we have a potential \(V _{\text{sr}}\) for the slow-roll, we might add a term 
%
\begin{align}
V = V _{\text{sr}} + \frac{\varphi^2}{\Lambda^2} V _{\text{sr}}
\,,
\end{align}
%
so if \(\Lambda \leq M_P\), we might get \(\eta _V = V'' M_P^2 / V \approx 1\). 
The main point is that inflation is very sensitive to high-energy physics. 

The inflationary potential can in general be an effective one, which works for energies \(E \leq \Lambda \) for some cutoff \(\Lambda \). 
We can use some additional symmetries imposed on the UV completion. 

%
\begin{align}
\mathscr{L} _{\text{eff}} = \frac{1}{2} \partial_{\mu } \varphi \partial^{\mu } \varphi + \frac{1}{2} m^2 \varphi^2 - \frac{\lambda}{4} \varphi^{4}
+ \sum _{p=1}^{\infty } \qty[\lambda _p \varphi^{4} + \nu _p \partial_{\mu } \varphi \partial^{\mu }\varphi ] \qty( \frac{g}{\Lambda } \varphi )^{2p}
\,,
\end{align}
%
which contains nonrenormalizable terms, but this is fine since it is an effective theory. 
We have employed the reflection symmetry \(\varphi \to - \varphi \). 

We can impose that \(\varphi \ll \Lambda \), so that the tower of terms does not spoil the flatness. This means that we have a small-field model of inflation. 
Alternatively, we may employ some symmetry like a shift symmetry \(\varphi \to \varphi + c\). 
In that case, the tower of terms cannot spoil the flatness, therefore we may see large-field models of inflation, which predict a relatively high tensor-to-scalar ratio. 

\end{document}
