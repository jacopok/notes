\documentclass[main.tex]{subfiles}
\begin{document}

\chapter{Reheating}

\section{Radiation from the inflaton}

\marginpar{Monday\\ 2020-11-16, \\ compiled \\ \today}

This is what happens in the transition between the inflationary phase and the usual radiation-dominated epoch. 

We will give a simplified treatment, which however captures the main characteristics of the model.

Consider the typical inflationary potential: flat at \(\varphi \sim 0\), sloping down towards a minimum. At \(\varphi \sim \varphi _f\) the field starts ``falling down'' towards the minimum quickly.
The condition is \(V''(\varphi ) \gtrsim H^2\), meaning that \(\eta _V \gtrsim 1\), which also implies that quickly we will find \(\epsilon \gtrsim 1\). 

When the field falls down it will \textbf{oscillate}, however its oscillations will be damped.
This is due to two factors: the expansion of the universe and the coupling of the field to other particles. 

The damped oscillations are described by a coupled Klein-Gordon equation: 
%
\begin{align}
\ddot{\varphi} + 3 H \dot{\varphi} + \Gamma _\varphi \dot{\varphi} = - V' (\varphi )
\,,
\end{align}
%
where \(\Gamma _\varphi \) is the \textbf{decay rate} of the inflaton field into other kinds of particles. 
This has the same form as the expansion term: it is a damping term as well.

The energy density of the scalar field can be differentiated:
%
\begin{align}
\rho _\varphi &= \frac{1}{2} \dot{\varphi}^2 + V(\varphi )  \label{eq:scalar-field-energy-density}\\
\dot{\rho}_\varphi &= \dot{\varphi} \ddot{\varphi} + V' (\varphi ) \dot{\varphi}
\,,
\end{align}
%
into which we can substitute into the KG equation to get
%
\begin{align}
\dot{\rho}_\varphi + (3 H + \Gamma _\varphi ) \dot{\varphi}^2 = 0
\,.
\end{align}

The timescale of the oscillations of \(\varphi \) will be much smaller than \(H^{-1}\), the timescale of the expansion of the universe. 

Oscillations at \(\varphi = \sigma \) will have a frequency \(\omega^2 = V''(\sigma )\), which is also the effective mass of the inflaton there, \(m^2_\varphi (\sigma )\). 

Since \(\omega^2 \gg H^2\) (this is true since \(\eta _V \gg 1\)) we can take averages over a period of the relevant quantities: 
%
\begin{align}
\expval{\dot{\varphi}^2} _{\text{period}} = \rho _\varphi 
\,,
\end{align}
%
since in general \(\expval{\dot{\varphi}^2 / 2} = \expval{V}\) by the virial theorem: therefore, substituting into \eqref{eq:scalar-field-energy-density} we get \(\expval{\dot{\varphi}^2} = \rho _\varphi \). 
Then our equation becomes 
%
\begin{align}
\dot{\rho}_\varphi + 3 H \rho _\varphi =  - \Gamma _\varphi \rho _\varphi 
\,,
\end{align}
%
which, neglecting \(\Gamma _\varphi \), looks like the continuity equation for nonrelativistic matter, which yields \(\rho _\varphi \propto a^{-3}\). 
This is expected: we have used the fact that \(m^2_\varphi (\sigma ) \gg H^2\), which is saying that the scalar field is very massive. 

In the spirit of keeping things simple, we will use a toy model and assume that all the decay products of \(\varphi \) are relativistic, whose continuity equation is 
%
\begin{align}
\dot{\rho} _R + 4 H \rho _R = + \Gamma _\varphi  \rho _\varphi 
\,,
\end{align}
%
by conservation of energy. This equation comes from \(\nabla_\nu T^{\mu \nu } = 0\). 
Gravity will then be described by the first Friedmann equation: 
%
\begin{align}
H^2 = \frac{8 \pi G}{3} \qty(\rho _\varphi + \rho _R)
\,.
\end{align}

We have a simple and exact solution: 
%
\begin{align}
\rho _\varphi = M^{4} \qty(\frac{a}{a _{\text{osc}}})^{-3} \exp(- \Gamma _\varphi \qty(t - t _{\text{osc}}))
\marginnote{See \cite[eq. 8.30]{kolbEarlyUniverse1994}.}
\,,
\end{align}
%
where \(t _{\text{osc}}\) and \(a _{\text{osc}}\) correspond to the time at which oscillations start, and \(M^{4} = \rho _\varphi ( t _{\text{osc}})\).
As a first approximation, this is the height of the potential the field is ``falling from'': \(M^{4} \sim V(\varphi = 0)\). 

Let us consider the evolution up to the time \(t \approx \Gamma _\varphi^{-1}\). Then, the decay will not have been very efficient yet, and the universe will still be nonrelativistic matter (\(\varphi \)) dominated, so we will have \(a \propto t^{2/3}\).

The time of the start of oscillation will be 
%
\begin{align}
t _{\text{osc}} \approx H^{-1} = \frac{M_P}{M^2}
\,,
\end{align}
%
since at the start of oscillations
%
\begin{align}
H^2 \approx \frac{8 \pi G}{3} M^{4} \approx \frac{M^2}{M_P}
\,.
\end{align}

So, 
%
\begin{align}
\dot{\rho}_R + 4 H \rho _R &= \Gamma _\varphi M^4  \qty( \frac{a}{a _{\text{osc}}})^{-3} \\
\dot{\rho}_R + \frac{8}{3} \frac{\rho _R}{t} &=
\Gamma _\varphi M^{4} \qty( \frac{t}{t _{\text{osc}}})^{-2} 
\,,
\end{align}
%
since \(a \propto t^{2/3}\) (we are in a matter-like dominated phase), therefore \(H = (2/3) t^{-1}\).
With a powerlaw ansatz \(\rho _R = B t^{\alpha }\) we get 
%
\begin{align}
\alpha t^{\alpha -1} + \frac{8}{3} \frac{t^{\alpha }}{t} = \frac{\Gamma _\varphi }{B} M^{4} \qty(\frac{t}{ t _{\text{osc}}})^{-2}
\,,
\end{align}
%
the homogeneous solution is given by \(\alpha = - 8 /3\), while the particular has \(\alpha = -1\). 

The initial condition we set is \(\rho _R (t _{\text{osc}}) = 0\), since before reheating inflation was taking place, diluting the energy density of radiation. 
This yields, in the pre-radiation-domination epoch:
%
\begin{align}
\rho _R \approx \Gamma _\varphi M_P^2 \frac{9}{40 \pi } \frac{1}{t} \qty[ 1 - \qty( \frac{t}{t _{\text{osc}}})^{-5/3}]
\,.
\end{align}

\todo[inline]{Where does the \(\pi \) come from?}

% \todo[inline]{So, we are basically approximating the decaying exponential with a constant function for a small region of time, and then 0?}

Starting from this solution, and using \(a \propto t^{2/3}\), we find 
%
\begin{align}
\rho _R = \frac{\num{.4}}{\pi^{1/2}} \Gamma _\varphi M_P M^2 \qty(\frac{a}{a _{\text{osc}}})^{-3/2} \qty[1 - \qty(\frac{a}{a _{\text{osc}}})^{- 5/2}]
\,.
\end{align}

The maximum energy density of radiation will be roughly given by \(\rho _R^{\text{max}} \approx \Gamma _\varphi M_P M^2\). 
\marginnote{The \(a\)-dependent part has a maximum value of roughly \num{.35}.}

The radiation energy density will be given by \(\rho _R = \frac{\pi^2}{30} g_* T^{4}\), so the maximum temperature will be 
%
\begin{align}
T^{\text{max}} = g_*^{-1/4} \rho _R^{\text{max}, 1/4} \sim g_*^{-1/4} \qty(\Gamma _\varphi M_P M^2)^{1/4}
\,.
\end{align}

In the reheating phase the energy density scales like \(\rho _R \propto a^{- 3/2}\): it is \emph{decreasing}, but much \emph{slower} than the usual \(\rho _R \propto a^{-4}\). 

Let us also discuss the entropy: 
%
\begin{align}
^*S = s a^3 &&
s = \frac{2 \pi^2}{45} g_{*s} T^3
\,,
\end{align}
%
so, since \(\rho _R \propto a^{-3/2}\) and \(\rho _R \propto T^{4}\) we have \(s \propto \rho _R^{3/4} \propto a^{- 9/8}\), therefore
%
\begin{align}
S \propto a^3 a^{-9/8} = a^{15/8}
\,.
\end{align}

It makes sense that this is increasing. 

What is the \emph{reheating temperature}? We want to match the inflationary solution and the radiation-dominated solution. 

Since we are in the radiation-dominated phase, we have \(a \propto t^{1/2}\): so
%
\begin{align}
H^2 = \frac{8 \pi G}{3} \rho _R = \frac{8 \pi }{3} \frac{1}{M_P^2} \frac{\pi^2}{30} g_* T^{4} = \frac{1}{4t^2}
\,.
\end{align}

The reheating temperature can be computed as the temperature at \(T_{RH} = T (t \approx \Gamma _\varphi^{-1})\). 
Plugging this in, we get 
%
\begin{align} \label{eq:reheating-temperature}
T_{RH} \approx \num{.55} g_*^{-1/4} \sqrt{\Gamma _\varphi M_P}
\,.
\end{align}

What is interesting to note here is that there is no memory of the energy scale \(M\), the vacuum energy scale of inflation. 
\todo[inline]{Add comment about the redshift of the inflaton's energy}

Only if \(\Gamma _\varphi \gg H _{\text{osc}}\) we would have had \(T_{RH} \sim M\); in that case the reheating would have been \SI{100}{\percent} efficient, the harmonic oscillator would have been overdamped. 

Note that the maximum temperature reached in the reheating phase is different from the reheating temperature. 

See \cite[fig.\ 8.3]{kolbEarlyUniverse1994}. 

\section{Boltzmann equation applications}

A usual rule of thumb is given in terms of the interaction rate \(\Gamma = n \sigma \abs{v}\): if \(\Gamma \gtrsim H\) thermal equilibrium can be established, while if \(\Gamma < H\) interactions become inefficient. 

If, roughly speaking, \(T \propto a^{-1}\), then \(\dot{T} / T = - H\). 

When \(\Gamma \sim H\), we need the Boltzmann equation in order to find out what exactly is going on. 
Let us give two examples. 

\(2 \leftrightarrow 2\) scattering between relativistic particles may be mediated by a massless boson, such as the photon, or by a massive boson, such as the \(W^{\pm}\) or \(Z^{0}\) boson. 

In the first (massless boson) case, we have 
%
\begin{align}
\sigma \sim \frac{\alpha^2}{T^2} && \alpha = \frac{g^2}{4 \pi }
\,,
\end{align}
%
while in the second (massive boson) case we have 
%
\begin{align}
\sigma \sim G_X^2 T^2 && G_X = \frac{\alpha }{m_X^2}
\,.
\end{align}

Roughly speaking, 
%
\begin{align}
\sigma \sim \alpha^2\abs{\text{propagator}}^2 \frac{q^{4}}{E^2}
\,,
\end{align}
%
where \(q\) is the spatial momentum of the interacting particles, while \(E\) is the center of mass energy, so for relativistic particles \(q^{4}/ E^2 \sim E^2 \sim T^2\).

In the massless boson case, the propagator looks like 
%
\begin{align}
\text{propagator} \sim \frac{-i g_{\mu \nu }}{p^2}
\,,
\end{align}
%
where \(p^2 =(q_1 + q_2 )^2\). 
Then, the cross-section looks like \(\sigma \sim \alpha^2 / T^2\) typically. 

In the massive boson case, we have 
%
\begin{align}
\text{propagator} \sim \frac{- i g_{\mu \nu } + p_{\mu } p_{\nu } / m_X^2}{p^2-  m_X^2}
\,,
\end{align}
%
so \(\sigma \sim \alpha^2 T^2 / m_X^{4}\). 

In the massless boson case, then, 
%
\begin{align}
\Gamma = n \sigma \abs{v} \approx n \sigma 
\,,
\end{align}
%
so if \(n \sim T^3\) (which is the case if we have radiation domination and equilibrium) we get \(\Gamma = \alpha^2 T\). We want to know when this will be larger than \(H\), using the fact that 
%
\begin{align}
H \sim \frac{T^2}{M_P} && H^2 = \frac{8 \pi G}{3} \underbrace{\rho _R}_{\propto T^{4}}
\,,
\end{align}
%
so \(\Gamma \gtrsim H\) when \(T < \alpha^2 M_P \sim \SI{e15}{GeV}\). 

For the massive gauge boson, at temperatures \(T \ll m_X\) we get \(\Gamma \gtrsim H\) when 
%
\begin{align}\label{eq:massive-gauge-boson-cross-section}
\Gamma = G_X^2 T^{5} &\gtrsim \frac{T^2}{M_P}  \\
T &\gtrsim G_X^{-2/3} M_P^{-1/3}
\,.
\end{align}

With normalization reflecting the case of weak interactions, and using \(G_X \sim \alpha / m_X^2\), we find 
%
\begin{align}
T \gtrsim \qty(\frac{m_X}{\SI{100}{GeV}})^{4/3} \SI{1}{MeV}
\,,
\end{align}
%
which can be used to figure out when neutrinos decouple. 

\end{document}
