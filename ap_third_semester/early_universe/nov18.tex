\documentclass[main.tex]{subfiles}
\begin{document}

\section{The Boltzmann equation}

\marginpar{Wednesday\\ 2020-11-18, \\ compiled \\ \today}

The phase space distribution function is denoted as \(f(x^{\mu }, p^{\mu })\); the Boltzmann equation reads 
%
\begin{align}
\hat{L}[f] = \hat{C}[f] 
\,.
\end{align}

If \(\hat{C} = 0\), then Liouville's theorem tells us that \(\dv*{f}{t} = 0\). 
The nonrelativistic expression for \(\hat{L}[p]\) is 
%
\begin{align}
\hat{L}[f] = \qty[ \pdv{\vec{x}}{t} \cdot \nabla_{\vec{x}} + \pdv{\vec{v}}{t} \cdot \nabla_{\vec{v}} + \pdv{}{t}]f
\,,
\end{align}
%
while the general relativistic expression is 
%
\begin{align}
\hat{L}[f] = \qty[p^{\mu } \pdv{}{x^{\mu }} - \Gamma^{\alpha }_{\beta \gamma } p^{\beta } p^{\gamma } \pdv{}{p^{\alpha }}]f
\,,
\end{align}
%
since by the geodesic equation 
%
\begin{align}
\dv[2]{x^{\mu }}{\lambda} 
= - \Gamma^{\mu}_{\alpha \beta } \dv{x^{\alpha }}{\lambda } \dv{x^{\beta }}{\lambda }  
= - \Gamma^{\mu }_{\alpha \beta } p^{\alpha } p^{\beta }
\,.
\end{align}

In the homogeneous and isotropic case we will have \(f = f (\abs{p}, t)\), since the dependences on \(\abs{p}\) and \(E\) are related. 
The only nonzero Christoffel symbols for a flat FLRW metric are 
%
\begin{align}
\Gamma^{0}_{ij} = \delta_{ij} a \dot{a} && \Gamma^{i}_{0j} = \Gamma^{j}_{i0}
= \delta^{i}_{j} \frac{\dot{a}}{a}
\,.
\end{align}

Therefore, the Liouville operator in our case can be written in terms of \(p^2 = g_{ij} p^{i}p^{j} = a^2 \delta_{ij} p^{i} p^{j}\), the square of the \emph{three}-momentum, which scales like \(1/a\):
%
\begin{align}
\hat{L}[f(E, t)] &= E \pdv{f}{t} - \frac{\dot{a}}{a} p^2 \pdv{f}{E}   
\,.
\end{align}

The number density reads 
%
\begin{align}
n = \frac{g}{(2 \pi )^3} \int \dd[3]{p} f
\,,
\end{align}
%
so the Boltzmann equation can be manipulated by rewriting the Liouville operator as 
%
\begin{align}
\underbrace{\pdv{}{t} \qty( \frac{g}{(2 \pi )^3} \int \dd[3]{p} f)}_{\dot{n}(t)} - \frac{\dot{a}}{a} \frac{g}{(2 \pi )^3}\int \dd[3]{p} \frac{p^2}{E} \pdv{f}{E} &= 
\frac{g}{(2 \pi )^3} \int \dd[3]{p} \frac{C[f]}{E}
\,.
\end{align}
%

Manipulating the second term in the equation we find
%
\begin{align}
\frac{g}{(2 \pi )^3} \int \dd[3]{p} p^2 \pdv{f}{E} 
= \frac{4 \pi g}{(2 \pi )^3} \int \dd{p} \frac{p^{4}}{E} \pdv{f}{E}
\,,
\end{align}
%
and since \(E^2 = p^2 + m^2 \implies E \dd{E} =p \dd{p}\) we get 
%
\begin{align}
\frac{4 \pi g}{(2 \pi )^3} \int \dd{p} p^3 \pdv{f}{p} 
= -\frac{4 \pi g}{(2 \pi)^3} \int \dd{p} (3 p^2) f(p)
= -3 n(t)
\marginnote{Integrated by parts}
\,.
\end{align}

Then our equation reads 
%
\begin{align}
\dot{n}(t) + 3 H n(t) = \frac{g}{(2 \pi )^3} \int \dd[3]{p} \frac{\hat{C}[f]}{E}
\,.
\end{align}

This, as expected, gives \(n \propto a^{-3}\) if there are no collisions.

Let us apply this to a process \(1 + 2 \leftrightarrow 3 + 4\), of which we are interested in the density \(n_1 (t)\): the collision terms reads 
%
\begin{align}
\begin{split}
\frac{g_1 }{(2 \pi )^3} \int \frac{\hat{C}[f_1 ]}{E_1 } \dd[3]{p_1 } &=
\int \dd{\Pi_1 }\dd{\Pi_2 }\dd{\Pi_3 }\dd{\Pi_4 }
(2 \pi )^{4} \delta^{(4)}(p_1 + p_2 - p_3 - p_4 ) \times \\
&\phantom{=}\ 
\qty[
    \abs{\mathcal{M}}^2_{3 + 4 \to 1 + 2} f_3 f_4 (1 \pm f_1 ) (1 \pm f_2 ) 
    -
    \abs{\mathcal{M}}^2_{1 + 2 \to 3 + 4} f_1 f_2 (1 \pm f_3 ) (1 \pm f_4 ) 
    ]
\end{split}
\marginnote{The minus in the second term is because in that case we are annihilating particle 1.}
\,,
\end{align}
%
where 
%
\begin{align}
\dd{\Pi _i} = \frac{g_i \dd[3]{p_i}}{(2 \pi )^3 2 E_i} 
\,,
\end{align}
%
while \(\mathcal{M}\) is the Feynman amplitude.
The terms \(1 \pm f_i\) are given by the Pauli blocking (\(-\)) or Bose enhancement (\(+\)) coming from the statistics of the chosen particle.

This equation must be written for all the particle species, in general yielding a complicated integrodifferential equation system. 
Because of time reversal symmetry (our first assumption) we will have \(\abs{\mathcal{M}}^2_{1 + 2 \to 3 + 4} = \abs{\mathcal{M}}^2_{3 + 4 \to 1 +2} \equiv \abs{\mathcal{M}}^2\). 

We can further postulate that all the particle species are in kinetic equilibrium (our second assumption): then, we say that they are distributed according to the Bose-Einstein or Fermi-Dirac distribution, 
%
\begin{align}
f^{i}_{BE - FD} = \qty[\exp(\frac{E - \mu}{T}) \mp 1]^{-1}
\,.
\end{align}

In this case we write \(\mp\) since the \(-\) corresponds to Bosons, while the \(+\) is for Fermions. 

This works well as long as the scatterings are fast enough. 
This parametrization is quite useful, since it makes our integrodifferential equations into simple(r) ODEs. 

Kinetic equilibrium is not just local thermodynamic equilibrium: it also requires local \emph{chemical} equilibrium, which implies the following for the chemical potentials:
%
\begin{align}
\mu _1 + \mu _2 = \mu _3 + \mu _4
\,.
\end{align}

Since we can find such linear relations for any possible reaction, the true number of independent chemical potentials will dramatically shrink. 
We expect these to be related to conserved quantities. 
Since these are mostly small, we can approximate \(\mu _i = 0 \forall i\). 
For the CMB photons, this is confirmed with \(\mu _\gamma / T \lesssim \num{e-4}\).

Further, we make the semiclassical approximation (our third assumption): \(1 \pm f_i \approx 1\), so that the distribution just becomes \(f = \exp((\mu - E) /T)\). 
With all these considerations, we find 
%
\begin{align}
\dot{n}_1 (t) + 3 H n_1 (t)
&= \int \dd{\Pi_1 }\dd{\Pi_2 }\dd{\Pi_3 }\dd{\Pi_4 } (2 \pi )^{4}\delta^{(4)} (p_1 +p_2 - p_3 - p_4 )
\abs{\mathcal{M}}^2 \qty(f_3 f_4 - f_1 f_2 )\\
\begin{split}
&= \int \dd{\Pi_1 }\dd{\Pi_2 }\dd{\Pi_3 }\dd{\Pi_4 }
(2 \pi )^{4}\delta^{(4)} (p_1 +p_2 - p_3 - p_4 ) \\
&\phantom{=}\ 
\abs{\mathcal{M}}^2 \exp(- \frac{E_1 + E_1 }{T})
\qty(\exp(\frac{\mu_3 + \mu _4 }{T}) - \exp( \frac{\mu _1 + \mu _2 }{T}))
\end{split}
\marginnote{Used \(E_1 + E_2 = E_3 + E_4 \).}
\,.
\end{align}

This yields 
%
\begin{align}
n_i (t) &= \frac{g_i}{(2 \pi )^3} \int \dd[3]{p_i} f_i  \\
&= g_i \exp( \frac{\mu_i}{T}) \int \frac{ \dd[3]{p_i}}{(2 \pi )^3} 
\exp(- \frac{E_i}{T})
\,.
\end{align}

At equilibrium, and using the approximation that \(\mu _i = 0\), this yields 
%
\begin{align} \label{eq:equilibrium-number-density}
n^{\text{eq}}_i = 
g_i \int \frac{ \dd[3]{p_i}}{(2 \pi )^3} 
\exp(- \frac{E_i}{T}) \approx \begin{cases}
    g_i \qty(\frac{m_i T}{2 \pi })^{3/2} \exp(- \frac{m_i}{T}) & \text{nonrelativistic}  \\
    \frac{g_i}{\pi^2} T^3 & \text{relativistic}
\end{cases}
\,.
\end{align}

Also, 
%
\begin{align}
f_3 f_4 - f_1 f_2 = \exp(- \frac{E_1 + E_2 }{T}) 
\qty[ e^{ \frac{\mu_3}{T}} e^{ \frac{\mu_4}{T}} - e^{ \frac{\mu_1}{T}} e^{ \frac{\mu_2}{T}} ] 
= \exp(- \frac{E_1 + E_2 }{T}) \qty[
    \frac{n_3 n_4 }{n_3^{\text{eq}} n_4^{\text{eq}}} -
    \frac{n_1 n_2 }{n_1^{\text{eq}} n_2^{\text{eq}}}
]
\,.
\end{align}

The thermally averaged cross-section is 
%
\begin{align}
\expval{\sigma \abs{v}} = 
\frac{1}{n_1^{\text{eq}} n_2^{\text{eq}}}
\int \dd{\Pi_1 }\dd{\Pi_2}\dd{\Pi_3 }\dd{\Pi_4 }
(2 \pi )^{4} \delta^{(4)} (p_1 + p_2 - p_3 - p_4 )
\abs{\mathcal{M}}^2 \exp(- \frac{E_1 + E_2 }{T})
\,.
\end{align}

Then, 
%
\begin{align}
\dot{n}_1(t) + 3 H n_1 (t) = n_1^{\text{eq}} n_2^{\text{eq}}
\expval{\sigma \abs{v}}
\qty[
    \frac{n_3 n_4 }{n_3^{\text{eq}} n_4^{\text{eq}}} -
    \frac{n_1 n_2 }{n_1^{\text{eq}} n_2^{\text{eq}}}
]
\,.
\end{align}

For a standard DM particle --- a process in the form \(\psi \overline{\psi} \leftrightarrow X \overline{X}\), with \(\psi \) and \(\overline{\psi }\) in local thermal equilibrium, and \(n_\psi = n_{\overline{\psi}}\) as well as \(n_X = n_{\overline{X}}\)---, we get 
%
\begin{align}
\dot{n}_1 (t) + 3 H n_1 (t) = \expval{\sigma \abs{v}} \qty[ (n_1^{\text{eq}})^2 - n_1^2]
\,.
\end{align}

The left-hand side is roughly \(n_1 / \tau \sim n_1 H\); while the right-hand side is roughly \(n_1^2 \expval{\sigma v}\): then, we are comparing \(H\) and \(\Gamma = n_1 \expval{\sigma v}\). 
If \(\Gamma \gg H\), then \(n_1 \) must tend towards its equilibrium value in the limit. 

\end{document}
