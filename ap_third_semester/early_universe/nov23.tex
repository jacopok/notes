\documentclass[main.tex]{subfiles}
\begin{document}

\chapter{Baryogenesis and Dark Matter production}

\section{Baryogenesis}

\marginpar{Monday\\ 2020-11-23, \\ compiled \\ \today}

The net baryon number density is denoted as 
%
\begin{align}
n_B = n_b - n_{\overline{b}}
\,,
\end{align}
%
where \(n_b\) and \(n_{\overline{b}}\) are the number densities of baryons and antibaryons respectively. 
We can estimate it as 
%
\begin{align}
n_B = n_N \approx \SI{1.38e-5}{cm^{-3}} \Omega_{0b} h^2
\,,
\end{align}
%
since
%
\begin{align}
n_N = (m_p n_N) \frac{1}{m_p} = \frac{\rho_{0b}}{m_p} 
= \underbrace{\frac{\rho_{0b}}{\rho _{\text{0, crit}}}}_{\Omega_{0b}} \frac{\rho _{\text{0, crit}}}{m_p}
\,,
\end{align}
%
and recalling that 
%
\begin{align}
\rho _{\text{crit}} = \frac{H_0^2}{8 \pi G} = \frac{1}{8 \pi G} \qty(2 h \SI{e-42}{GeV} ) 
\,.
\end{align}

Another useful quantity to have is 
%
\begin{align}
B = \frac{n_B}{s}
\,,
\end{align}
%
where \(s\) is the entropy density.
This is interesting to us since \(s \propto a^{-3}\), just like \(n_B\) will do if there is no baryon annihilation or generation. 
What we are doing amounts to counting the amount of baryons in a comoving volume: if none are generated, \(B\) is constant. 

This is given by 
%
\begin{align}
B \approx \num{7e-9} \Omega_{0b} h^2
\,,
\end{align}
%
where \(\Omega_{0b} h^2 \approx \num{.0224(1)}\) with the current measurements, therefore \(B \approx \num{e-10}\).
    
Then, we can solve for \(s_0 \) in the expression: recall that in general
the entropy is given in terms of the total relativistic degrees of freedom \(g_{*s}\):
%
\begin{align}
s = \frac{2 \pi^2}{45} g_{*s} T^3
\,,
\end{align}
%
and 
%
\begin{align}
n_\gamma = \frac{\zeta (3) g_\gamma T^3}{\pi^2}
\,,
\end{align}
%
where \(g_\gamma = 2\) is the number of polarizations of a photon, therefore 
%
\begin{align} \label{eq:entropy-approximate}
s &\approx \num{1.8} g_{*s} n_\gamma  \\
s_0 &\approx \num{1.8} g_{*s}(t_0 ) n_\gamma (t_0 )
\,.
\end{align}

The \textcite{kolbEarlyUniverse1994} does the following calculation with three massless neutrinos, however we now know that at least two neutrino species are nonrelativistic today. 
So, with one relativistic neutrino we have \(n_\gamma \approx \SI{422}{cm^{-3}}\), and we can give the conservative estimate:
%
\begin{align}
g_{*s}(t_0 ) \approx \num{2.63}
\,,
\end{align}
%
and using the fact that \( T / T_\nu = (11/4)^{1/3}\) we get 
%
\begin{align}
s_0 = \num{4.7} n_\gamma (t_0 )
\,.
\end{align}

We can measure the baryon number \(\Omega_{0b}\) from a different source from CMB anisotropies: primordial nucleosynthesis. 
This is in very good agreement with the CMB data. 

The \textbf{annihilation catastrophe}: in a symmetric baryonic universe baryons and antibaryons remain in thermal equilibrium down to \(T \sim \SI{20}{MeV}\), and at these temperatures \(n_b / s \sim \num{e-20}\). 
This is around 10 orders of magnitude smaller than what we observe. 
This occurs because baryons and antibaryons keep annihilating. 

Therefore, at \(T \gg \SI{20}{MeV}\) an initial asymmetry between baryons and antibaryons must be present. 

So, we are looking for the \textbf{mechanism} which, starting from symmetric initial conditions, is able to generate an initial \(B \approx \num{e-10}\).

We have the \textbf{Sakharov conditions} telling us what we need in order for this to occur: 
\begin{enumerate}
    \item violation of baryon number conservation;
    \item \(C\) and \(CP\) violation;
    \item out of equilibrium conditions.
\end{enumerate}

If we are in thermal equilibrium, the chemical potentials \(\mu \) can all be taken to be zero; then the baryon and antibaryon distribution functions will look like 
%
\begin{align}
f_b = f_{\overline{b}} = \frac{1}{\exp(\frac{E \pm \mu }{T}) + 1} = \frac{1}{\exp(\frac{E}{T}) + 1}
\,,
\end{align}
%
and the energies are indeed the same: 
%
\begin{align}
E_{b / \overline{b}} = \sqrt{p^2 + m_{b / \overline{b}}^2}
\,,
\end{align}
%
but by CPT symmetry \(m_b = m_{\overline{b}}\), so \(E_b = E_{\overline{b}}\): then indeed \(f_b = f_{\overline{b}}\), which means \(n_b = n_{\overline{b}}\), so \(B = 0\).

Consider a particle \(X\) which decays into either \(qq\), with BR \(r\), or into \(\overline{q} \ell\) with branching ratio \(1 - r\). 
These generate baryon number \(2/3\) and \(-1/3\) respectively.

The antiparticle \(\overline{X}\) can decay into \(\overline{q} \overline{q}\) or \(q \overline{\ell}\), with branching ratios \(\overline{r}\) and \(1 - \overline{r}\) respectively. 
\(C\) violation tells us that we can have \(r \neq \overline{r}\).

We can then calculate the net baryon number generated by the decay of an \(X\) particle: 
%
\begin{align}
B_X = \frac{2}{3} r - \frac{1}{3} (1 - r) = r - \frac{1}{3}
\,,
\end{align}
%
while the net baryon number for the decay of an anti-\(X\) will be 
%
\begin{align}
B_{\overline{X}} = -  \frac{2}{3} \overline{r} + \frac{1}{3} (1 - \overline{r}) = - (\overline{r} - \frac{1}{3})
\,.
\end{align}

The total, for an \(X\)-\(\overline{X}\) pair, is
%
\begin{align}
B_X + B_{\overline{X}} = r - \overline{r} = \epsilon 
\,.
\end{align}

This tells us that we need \textbf{CP violation}.

We also need departures from thermal equilibrium. 
Suppose we want to consider the decay of a particle \(X\), and we start at \(T \gg m_X\). 
At this stage, we expect \(n_X \approx n_{\overline{X}} \approx n_\gamma\). 

On the other hand, if \(T \ll m_X\) we have  \(n_X = (m_X T)^{3/2} \exp(- m_X / T)\), which decays quickly. 

\todo[inline]{Add plot of \(n_X / n_\gamma \) versus \(z = m_X / T\).}

If \(H \gg \Gamma \) the particle can go out of equilibrium. 
We can have decay of \(X\), or \(B\)-violating scattering, or processes mediated by \(X\). 

The annihilation rate in any case is \(\Gamma _{\text{ann}} \propto n_X\). 

Then, considering a process in the form \(qq \to \overline{q} + \ell\) mediated by \(X\) (like \(W^{+} \to \ell^{+} + \nu _e\), \(H \to \ell \overline{\ell}\))
we have 
%
\begin{align}
\Gamma _D = 
\begin{cases}
    \alpha  m_X & T \lesssim m_x, \alpha = g^2 / 4 \pi \\
    \alpha  m_X ( \frac{m_X}{T}) & T \geq m_X
\end{cases}
\,,
\end{align}
%
or, for the \(\Gamma \) of Inverse Decay:
%
\begin{align}
\Gamma_{ID} = \begin{cases}
    \Gamma _D & T \geq m_X \\
    \sim \Gamma _D e^{- m_X / T} & T \lesssim m_X
\end{cases}
\,.
\end{align}

If we have \(2 \leftrightarrow 2\) scattering, mediated by \(X\), we will see 
%
\begin{align}
\Gamma _s \approx n \sigma \abs{v} \approx T^3 \frac{\alpha^2 T^2}{(T^2 + m_X^2)^2}
\marginnote{\(\abs{v} \approx 1\).}
\,.
\end{align}

This interpolates between the \(\alpha^2 / T^2\) case, in which \(X\) is massless, and the \(G_X^2 T^2\) case, in which \(X \) is massive, and \(G_X \sim \alpha / m_X^2\). 

Now, in this early universe stage the Hubble rate will look like \(H \sim \sqrt{g_*} T^2 / m_P\), without any dependence on \(m_X\).

So, the quantity \(\Gamma / H\) at the time in which \(m_X / T = 1\) will tell us whether the decays are efficient. 
Let us calculate it explicitly: 
%
\begin{align} \label{eq:K-boltzmann-equation}
K 
= \eval{\qty(\frac{\Gamma _D}{H})}_{T = m_X} 
= \frac{\alpha  m_X}{m_x^2 \sqrt{g_*} / m_P} 
= \frac{\alpha  m_P}{m_X \sqrt{g_*}}
\,.
\end{align}

Let us take the regime \(T \lesssim m_X\). Here, 
%
\begin{align}
\frac{\Gamma_{ID}}{H} \sim \frac{e^{-m_X / T} \Gamma _D}{H} \sim K e^{-m_X / T}
\,,
\end{align}
%
therefore 
%
\begin{align}
\frac{\Gamma_{S}}{H} \sim\frac{ \frac{\alpha^2}{m_X^{4}} T^{5}}{ \frac{T^2}{m_P} \sqrt{g_*}} 
\sim \alpha \frac{\alpha m_P}{\sqrt{g_*} m_X} \qty( \frac{T}{m_X})^{3}
\sim \alpha K \qty( \frac{T}{m_X})^{3}
\,.
\end{align}

If \(K \ll 1\), then for sure all the possible processes will be inefficient, and the particle \(X\) will be out of equilibrium. 

Then the net baryon number will be 
%
\begin{align}
n_B = \epsilon \eval{n_X}_{t_D} \approx \epsilon n_\gamma 
\,.
\end{align}

So, using the expression for the entropy given in \eqref{eq:entropy-approximate} we get
%
\begin{align} \label{eq:baryon-number}
B = \frac{n_B}{s} = \frac{\epsilon n_\gamma }{g_{*s} n_\gamma } \sim \frac{\epsilon}{g_{*s}}
\,.
\end{align}

Since \(B \sim \num{e-10}\) and \(g_* \sim \num{e2} \divisionsymbol \num{e3}\), we will need \(\epsilon \sim \num{e-7} \divisionsymbol \num{e-8}\). 

The condition \(K \ll 1\) is equivalent to \(m_X \gg g_*^{-1/2} \alpha m_P\). 
So, the particles we need to consider are very massive: on the order of \SI{e16}{GeV}. These can be realized in some models of GUTs.

\todo[inline]{Plot the curve \(n_X / n_\gamma \): roughly 1 for a large time, then when \(m_X \sim T\) it drops. }

If on the other hand \(K \gg 1\), because of thermal equilibrium we will have \(B \approx 0\). 
If, instead, we had \(K \lesssim 1\) (close to 1), we need to use the full Boltzmann equation. 

Recall that for \(T \geq m_X\), we had 
%
\begin{align}
\Gamma _D = \alpha  m_X \qty( \frac{m_X}{T})
\,.
\end{align}

Where does the suppression factor \(m_X / T \) come from? 
It is a time dilation effect: \(t_D \sim \Gamma_D^{-1}\), but we need to multiply \(t_D\) by the Lorentz factor \(\gamma \), so the real decay rate is \(\Gamma _D \gamma^{-1} \), but 
%
\begin{align}
E_X \sim \abs{q} = m_X \abs{\vec{v}} \gamma \approx m_X \gamma 
\,,
\end{align}
%
so \(\gamma \sim E_X / m_X \sim T / m_X\). This yields 
%
\begin{align}
\Gamma _D \gamma^{-1} = (\alpha  m_X) \frac{m_X}{T}
\,.
\end{align}



\end{document}
