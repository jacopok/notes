\documentclass[main.tex]{subfiles}
\begin{document}


\marginpar{Wednesday\\ 2020-11-25, \\ compiled \\ \today}

Again we consider the problem of \(X\) and \(\overline{X}\) decaying into the relativistic particles \(b\) and \(\overline{b}\), which carrying baryon number \(\pm 1/2 \) respectively. This is a simplification, but not a large one.

The Feynman amplitudes can be parametrized as 
%
\begin{align}
\Circled{1} = \abs{\mathcal{M}(X \to b b )}^2 &=
\abs{\mathcal{M}(b b \to X )}^2 =
+ \frac{1}{2} (1 + \epsilon ) \abs{\mathcal{M}_0}^2 \\
\Circled{2} = \abs{\mathcal{M}(X \to \overline{b} \overline{b} )}^2 &=
\abs{\mathcal{M}(\overline{b} \overline{b} \to X )}^2 =
+ \frac{1}{2} (1 - \epsilon ) \abs{\mathcal{M}_0}^2 
\,,
\end{align}
%
which means that  
%
\begin{align}
\frac{\Circled{1} - \Circled{2}}{\Circled{1} + \Circled{2}} = \epsilon 
\,.
\end{align}

We suppose we are in kinetic equilibrium, and neglect quantum effects: 
%
\begin{align}
f_b (E) &= \exp(- \frac{(E-\mu )}{T})  \\
f_{\overline{b}} (E) &= \exp(- \frac{(E + \mu )}{T}) \\
f_{X} (E) &= \exp(- \frac{(E - \mu_X )}{T}) 
\,.
\end{align}

Chemical equilibrium, together with the process \(b + \overline{b} \leftrightarrow 2 \gamma \) and \(\mu _\gamma = 0\), would tell us \(\mu _b = - \mu _{\overline{b}}\).

This yields, confusing \(g_{*s}\) for \(g_*\) (which works well in the high-energy regime):
%
\begin{align}
B = \frac{n_B}{s} =  \frac{1}{2 }\frac{n_b - n_{\overline{b}}}{\num{1.8} g_{*s} n_\gamma } =
\frac{g}{2 \times \num{1.8} g_*} \times \sinh( \frac{\mu}{T} )
\,,
\end{align}
%
where the hyperbolic sine comes from the fact that we must consider the distribution functions, which translates to 
%
\begin{align}
f_b - f_{\overline{b}} \sim 
\exp(+ \frac{\mu}{T}) - \exp(- \frac{\mu }{T}) \sim \sinh(\frac{\mu}{T})
\,.
\end{align}

We can expand the hyperbolic sine as \(\mu / T\) since \(\mu / T\) is small. 

Using the aforementioned reactions, we get \(\mu _X = 2 \mu _b = 2 \mu \), and also \(\mu _X = - 2 \mu \), so both are zero, which means that \(B = 0\). 
This illustrates how\dots

The expression for \(B\) can also be written as 
%
\begin{align}
B = \frac{n_B}{\num{1.8} g_* n_\gamma } 
\,,
\end{align}
%
where we used the approximated expression 
%
\begin{align}
n_\gamma \approx \frac{2 T^3}{\pi^2}
\,;
\end{align}
%
this tells us 
%
\begin{align}
\frac{\mu}{T} \approx \frac{n_B}{n_\gamma }
\,.
\end{align}

Now, let us consider the processes \(X \to bb\) and \(X \to \overline{b} \overline{b}\) using the Boltzmann equation in the semiclassical approximation: 
%
\begin{align}
\begin{split}
\dot{n}_X + 3 H n_X &= 
\int \dd{\Pi_X } \dd{\Pi_1 } \dd{\Pi _2} (2 \pi )^{4} \delta^{(4)}(p_X - p_1 - p_2)\\
&\phantom{=}\ 
\qty[-f_X \abs{\mathcal{M}_0}^2 
+ f_{\overline{b}}^{(1)} f_{\overline{b}}^{(2)} \frac{1}{2} (1 + \epsilon ) \abs{\mathcal{M}_0}^2
+ f_{b}^{(1)} f_{b}^{(2)} \frac{1}{2} (1 - \epsilon ) \abs{\mathcal{M}_0}^2
]
\end{split}
\,,
\end{align}
%
where by \(f^{(i)}\) we mean \(f(p_i)\), and
%
\begin{align}
\dd{\Pi _i} = \frac{g_i \dd[3]{p_i}}{(2 \pi )^3 2 E_i}
\,.
\end{align}

The products 
of distribution functions read 
%
\begin{align}
f_{b/ \overline{b}}^{(1)} f_{b/ \overline{b}}^{(2)}  &= \exp(\frac{-(E_1 + E_2 )}{T}) \exp(\pm \frac{2 \mu }{T})  \\
&\approx f_X^{\text{eq}} (E_X) \exp(\pm \frac{2 \mu }{T})  \\
&\approx f_X^{\text{eq}} (E_X) \qty(1 \pm \frac{2\mu}{T})  \\
&\approx f_X^{\text{eq}} (E_X) \qty(1 \pm 2 \frac{n_B}{n_\gamma })
\,.
\end{align}

We are working to first order in both \(\epsilon \) and \(n_B / n_\gamma \), products of these would be higher order. 
The BE then reads 
%
\begin{align}
\begin{split}
\dot{n}_X + 3 H n_X &= 
\int \dd{\Pi_X } \dd{\Pi_1 } \dd{\Pi _2} (2 \pi )^{4} \delta^{(4)}(p_X - p_1 - p_2)\\
&\phantom{=}\ 
\qty[- f_X + f_X^{\text{eq}}] \abs{\mathcal{M}_0}^2
\end{split}
\,,
\end{align}
%
but 
%
\begin{align}
f_X = \exp( \frac{\mu_X}{T}) f_X^{\text{eq}} = \frac{n_X}{n_X^{\text{eq}}} f_X^{\text{eq}}
\,.
\end{align}

The square bracket in the BE then reads 
%
\begin{align}
-\frac{f_X^{\text{eq}}}{n_X^{\text{eq}}} \qty[- n_X + n_X^{\text{eq}}]
= -\frac{f_X^{\text{eq}}}{n_X^{\text{eq}}}
\expval{M_0}^2
\,.
\end{align}

Recall that \(n_B = (n_b - n_{\overline{b}}) / 2\) (since the baryon number they carry is \(\pm 1/2\)): then, we need to write another Boltzmann equation here, without a \(1/2\) factor since there are two baryons
%
\begin{align}
\begin{split}
\dot{n}_b + 3 H n_b &=
\int \dd{\Pi_X } \dd{\Pi_1 } \dd{\Pi _2} (2 \pi )^{4} \delta^{(4)}(p_X - p_1 - p_2)\\
&\phantom{=}\ 
\qty[- (1 - \epsilon ) f_b^{(1)} f_b^{(2)} +
(1 + \epsilon ) f_X
] \abs{\mathcal{M}_0}^2
\end{split}
\,.
\end{align}

However, we must also account for \(X\)-mediated scatterings \(bb \to \overline{b} \overline{b}\) and vice versa: this means we need to add a new term to the right-hand side of the BE, 
%
\begin{align}
2 \int \dd{\Pi_1 } \dd{\Pi_2 } \dd{\Pi_3 } \dd{\Pi_4 } \delta^{(4)}(p_1 + p_2 - p_3 - p_4)
\qty[-f_b^{(1)} f_b^{(2)} \abs{\mathcal{M}' (bb \to \overline{b} \overline{b})}^2 
+
f_{\overline{b}}^{(1)} f_{\overline{b}}^{(2)} \abs{\mathcal{M}' (\overline{b} \overline{b} \to bb) }^2
]
\,.
\end{align}

The quantity \(\mathcal{M}'\) is the Feynman amplitude of the scattering process where the physical intermediate \(X\) has been removed; we have already counted processes in which a \(X\) was formed and then decayed.
Instead, the processes we were missing and which we are now accounting for are the ones in which \(X\) is virtual (i.\ e.\ not on shell).
Then, 
%
\begin{align}
\abs{\mathcal{M} (bb \to \overline{b} \overline{b})}^2= 
\abs{\mathcal{M}' (bb \to \overline{b} \overline{b})}^2
+ \abs{\mathcal{M} _{\text{ris}} ( bb  \to \overline{b} \overline{b})}^2
\,.
\end{align}

In the end the equation for \(X\) is written as 
%
\begin{align}
\dot{n}_X + 3 H n_X = - \Gamma _D (n_X - n_X^{\text{eq}}) 
\,.
\end{align}

We then find that the bracket to be integrated in the BE for \(n_B = (n_b - n_{\overline{b}}) / 2\) is
%
\begin{align} \label{eq:BE-term}
\frac{1}{2} \qty[- (1-\epsilon ) f_b^{(1)} f_b^{(2)} + (1 + \epsilon ) f_X 
+ (1 + \epsilon ) f_{\overline{b}}^{(1)} f_{\overline{b}}^{(2)} ]\abs{\mathcal{M}_0}^2
\,,
\end{align}
%
and we can make the estimate 
%
\begin{align} \label{eq:estimate-density-functions-boltzmann}
f_{b/\overline{b}}^{(1)}
f_{b/\overline{b}}^{(2)} \approx 
f_X^{\text{eq}} \qty(1 \pm 2 \frac{n_B}{n_\gamma })
\,.
\end{align}

This allows us to write the expression in \eqref{eq:BE-term} as 
%
\begin{align}
\qty[\epsilon f_X^{\text{eq}} + \epsilon f_X ] \abs{\mathcal{M}_0}^2
- 2 f_X^{\text{eq}} \frac{n_B}{n_\gamma } \abs{\mathcal{M}_0}^2
\,.
\end{align}

Every time \(n_X^{\text{eq}}\) appears, it is coming from inverse decay processes. 
What we have found is a Boltzmann equation for \(n_B = (n_b / n_{\overline{b}}) / 2\). 
There seems to be a problem: in what we have found we have something like \(n_X + n_X^{\text{eq}}\), which is weird! 
This is actually fixed if we account for scatterings. 

The scattering term yields 
%
\begin{align}
4 \times \qty[
     - f_b^{(1)} f_b^{(2)} \abs{\mathcal{M}' (bb \to \overline{b} \overline{b})}^2 + 
     f_{\overline{b}}^{(3)} f_{\overline{b}}^{(4)} \abs{\mathcal{M}' (\overline{b} \overline{b} \to bb)}^2  
     ]
\,.
\end{align}

There can be a certain part of scattering processes which do not violate \(CP\); for these, the backward and forward amplitudes are equal: 
%
\begin{align}
\abs{\mathcal{M}' (bb \to \overline{b} \overline{b})}^2
=
\abs{\mathcal{M}' (\overline{b} \overline{b} -> bb)}^2
\,,
\end{align}
%
therefore, using equation \eqref{eq:estimate-density-functions-boltzmann}, we get 
%
\begin{align}
- 4 \frac{n_B}{n_\gamma } f_X^{\text{eq}} \abs{\mathcal{M}' (\overline{b} \overline{b} -> bb)}^2
\,.
\end{align}

Now, instead, let us consider the \(CP\) violating part of the processes: neglecting higher order terms, we find 
%
\begin{align}
- 2 f_X^{\text{eq}} \qty(- \abs{\mathcal{M}'(\overline{b} \overline{b} \to bb)}^2 + \abs{\mathcal{M}' (bb \to \overline{b} \overline{b})}^2)
\,.
\end{align}

Finally then, 
%
\begin{align}
\abs{\mathcal{M}'}^2 = \abs{\mathcal{M}}^2 - \abs{\mathcal{M}_{\text{ris}}}^2
\,,
\end{align}
%
and using \(CPT\) plus unitarity we can say that for the \emph{total amplitude} we have \(\abs{\mathcal{M}(bb \to \overline{b}\overline{b})}^2 = \abs{\mathcal{M}(\overline{b}\overline{b} \to bb)}^2\).

Then we get 
%
\begin{align}
-2 f_X^{\text{eq}} \qty[\qty(- \abs{\mathcal{M}_{\text{ris}}(\overline{b} \overline{b} \to bb)}^2 + \abs{\mathcal{M}_{\text{ris}} (bb \to \overline{b} \overline{b})}^2)]
\,.
\end{align}

Here there would be another two pages of computations to do; what we find is 
%
\begin{align}
\abs{\mathcal{M}_{\text{ris}}(bb \to \overline{b}\overline{b})}^2
\sim 
\abs{\mathcal{M}(bb \to X)}^2
\abs{\mathcal{M}(X \to \overline{b}\overline{b})}^2
\propto \frac{1}{4} (1 - \epsilon )^2 \abs{\mathcal{M}_0}^4
\,,
\end{align}
%
so finally we will get 
%
\begin{align}
-2 f_X^{\text{eq}} \epsilon \abs{\mathcal{M}_0}^2
\,.
\end{align}

The thermally averaged decay rate reads 
%
\begin{align}
\Gamma _D = \frac{1}{n_X^{\text{eq}}} \int \dd{\Pi _X} \dd{\Pi_1 } \dd{\Pi _2} \delta^{(4)} (p_X - p_1 - p_2 ) f_X^{\text{eq}} \abs{\mathcal{M}0 }^2
\,.
\end{align}

So, the Boltzmann equation for \(n_B\) reads
%
\begin{align}
\dot{n}_B + 3 H n_B = + \underbrace{\epsilon \Gamma _D \qty(n_X - n_X^{\text{eq}})}_{X \text{ decay}}
- \underbrace{2 \Gamma _D \qty( \frac{n_X^{\text{eq}}}{n_\gamma }) n_B }_{X \text{ inverse decay}}
- \underbrace{4 n_B n_\gamma \expval{\sigma \abs{v}}}_{\text{scattering}}
\,,
\end{align}
%
where 
%
\begin{align}
\expval{\sigma \abs{v}} = \frac{1}{n_\gamma^2}
\int \dd{\Pi_1 } \dd{\Pi_2 } \dd{\Pi_3 } \dd{\Pi_4 }
f_X^{\text{eq}} \abs{\mathcal{M}' (bb \to \overline{b}\overline{b})}^2
\,.
\end{align}

\end{document}
