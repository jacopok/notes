\documentclass[main.tex]{subfiles}
\begin{document}

\marginpar{Wednesday\\ 2020-10-21, \\ compiled \\ \today}

We come back to the dynamics of slow-roll inflation. 
We defined \(\epsilon = - \dot{H} / H^2 \), which in our model corresponds to \(\dot{\varphi}^2 / V \). 
So, the conditions of the potential being flat (\(V'\) being small) and the kinetic energy being small compared to the potential are seen to correspond to \(\epsilon \ll 1\). 

On the other hand, \(\eta = - \ddot{\varphi} / (H \dot{\varphi}) = \eta_V - \epsilon \), where 
%
\begin{align}
\eta _V = \frac{1}{3} \frac{V''}{H^2} = \frac{1}{8 \pi G} \frac{V''}{V}
\,.
\end{align}

What we ask is that all three of these parameters be small. 

This means that \(H\) must change slowly over time. 
We can manipulate the second derivative of the scale factor so that \(\epsilon \) appears:
%
\begin{align}
\ddot{a} = \dv{}{t} \qty(Ha) = a \dot{H} + \dot{a} H  = a \qty(\dot{H} + H^2) = a H^2 \qty(1 + \frac{\dot{H}}{H^2}) = a H^2 \qty(1 - \epsilon )
\,.
\end{align}

We can see that \(\ddot{a} > 0\) only if \(\epsilon < 1\). 
What is then the relevance of \(\eta < 1\) then? 
We must have not only a phase of accelerating expansion, but a phase of accelerating expansion which lasts \emph{sufficiently long}.
In order for this to happen, we need \(\epsilon \sim \const\). Since \(\epsilon \sim \dot{\varphi}^2\) while \(\eta \sim \ddot{\varphi}\), requiring \(\eta \ll 1\) ensures this. 

Also, \(\eta \ll 1\) is needed in order to neglect the acceleration term in the Klein-Gordon equation, so that we move towards an attractor solution in the friction-dominated regime. 

There could be arbitrarily many more slow-roll parameters, defined in terms of higher-order derivatives \cite[pag.\ 406]{guzzettiGravitationalWavesInflation2016}: 
%
\begin{align}
\xi  ^2 = \frac{1}{8 \pi G} \qty(\frac{V' V'''}{V''})
\,,
\end{align}
%
and we will always expand in these parameters; in this course we will usually stop at second order (keeping \(\epsilon \) and \(\eta \)).

We will then be able to consider both \(\epsilon \) and \(\eta\) as approximately constant: their time derivatives are of higher order in them, specifically \(\dot{\epsilon}, \dot{\eta} = \order{\epsilon^2, \eta^2}\).

\begin{claim}
An example: the following relation holds: 
%
\begin{align}
\frac{\dot{\epsilon}}{H} = 2 \epsilon (\epsilon - \eta )
\,.
\end{align}
\end{claim}

\begin{proof}
Let us take the derivative explicitly: 
%
\begin{align}
\frac{\dot{\epsilon}}{H} &= \frac{1}{H} \dv{}{t} \qty(- \frac{\dot{H}}{H^2}) = - \frac{\ddot{H}}{H^3} + 2 \underbrace{\frac{\dot{H}^2}{H^{3} H}}_{= \epsilon^2}
\,.
\end{align}

Now, we need to use the fact that, as we have shown earlier \eqref{eq:hdot-phidot-relation}, \(\dot{H} \propto \dot{\varphi}^2\). In terms of logarithmic derivatives this can be written as 
%
\begin{align}
\frac{\ddot{H}}{\dot{H}} = 2\frac{\ddot{\varphi}}{\dot{\varphi}}
\,,
\end{align}
%
which we can substitute in the aforementioned expression to get 
%
\begin{align}
- \frac{\ddot{H}}{H^3} = - \frac{1}{H^3} 2 \frac{\ddot{\varphi}}{\dot{\varphi}} \dot{H} = -2 \qty(- \frac{\dot{H}}{H^2}) \qty(-\frac{\ddot{\varphi}}{H \dot{\varphi}}) = -2 \epsilon \eta 
\,.
\end{align}
\end{proof}

Forgetting about the precise coefficients, let us consider a model such that \(V(\varphi ) \propto \varphi^{\alpha }\): then, the parameter \(\epsilon \) reads
%
\begin{align}
\epsilon \sim \frac{1}{\pi G} \qty(\frac{V'}{V})^2 \sim \frac{\alpha^2 M _{\text{Pl}}^2}{\varphi^2}
\,,
\end{align}
%
meaning that in order to have \(\epsilon \ll 1\) we need \(\varphi \gtrsim M _{\text{Pl}}\). 
Note that we are talking about the \emph{background} solution, dropping the index \(0\) for simplicity.

These are then called \emph{large-field} models, since the value of \(\varphi \) must be very large. 
Another kind of potential is in the form 
%
\begin{align}
V(\varphi ) = V_0 \qty[1 - \qty(\frac{\varphi }{\mu })^{p } + \dots]
\,,
\end{align}
%
where \(\varphi < \mu < M _{\text{Pl}}\), while \(p > 2\). 
% \todo[inline]{Possibly 1?}

The dots (higher order terms) are important for the end of inflation, not for most of it. 
This potential has a plateau near \(\varphi = 0\), and \(V(\varphi = 0) = V_0 \).
In this region, 
%
\begin{align}
\epsilon \sim \frac{1}{\pi G} \qty(\frac{V'}{V})^2
\sim \frac{p^2}{\pi G} \frac{\varphi^{2p-2}}{\mu^{2p}} 
\sim \frac{p^2 \varphi^{2p} M _{\text{Pl}}^2}{\varphi^2 \mu^{2p}} 
\,,
\end{align}
%
so \(\epsilon \to 0\) as \(\varphi \to 0\) (the exponent of \(\varphi \) is \(2p -2 > 0\)): these are called \emph{small-field models} of inflation, since we can have small \(\epsilon \) even with small \(\varphi \).

Why do we need the condition \(\varphi < \mu < M _{\text{Pl}}\)? 

\begin{claim}
In the case \(p = 2\) and in the case \(\mu > M _{\text{Pl}}\) this model becomes a \emph{large-field} one. 
\end{claim}
\begin{proof}
In that case, the condition we must ask for \(\epsilon < 1\) is 
%
\begin{align}
4 \frac{M _{\text{Pl}}^2 \varphi^2}{\mu^{4}} \ll 1
\,,
\end{align}
%
which can be satisfied by \(\varphi \gtrsim M _{\text{Pl}}\). 
\end{proof}

There is also a third category (high gravitation models), but it is mostly excluded by data. 

The quantity we are interested in is the \emph{excursion} \(\Delta \varphi \) of the field in the \emph{observable window}: the difference between its value when the horizon crosses the largest observable scales (\(\varphi _{\text{CMB}}\) --- called so since we can measure it from CMB observations) and the value of the field at the end of inflation, \(\varphi _{\text{end}}\): \(\Delta \varphi = \varphi _{\text{CMB}} - \varphi _{\text{end}}\). 
This interval corresponds to the \(\sim 60\) \(e\)-folds of inflation; inflation likely lasted much more, however the earliest parts of it, which correspond to scales much larger than the horizon today, are hardly observable. 

We can compute \(\Delta \varphi \) as 
%
\begin{align}
\Delta \varphi  = \int_{\varphi _{\text{CMB}}}^{\varphi _{\text{end}}} \dd{\varphi } = \int_{t _{\text{CMB}}}^{t _{\text{end}}} \dot{\varphi} \dd{t} \approx \frac{\dot{\varphi}}{H} \int_{H t _{\text{CMB}}}^{H t _{\text{end}}} \dd{(Ht)} = \frac{\dot{\varphi}}{H} 
N _{\text{CMB}} \sim N _{\text{CMB}} \sqrt{\epsilon } M _{\text{Pl}}
\,,
\end{align}
%
where \(N _{\text{CMB}} \sim 60 \divisionsymbol 70\) is the number of \(e\)-folds in the sub-horizon part of inflation. 
We used the fact that \(\ddot{\varphi}\) is negligible compared to \(\dot{\varphi} / H^{-1}\), so \(\dot{\varphi}\) can be taken to be constant; also, we used the first alternative expression for \(\epsilon \) in equation \eqref{eq:slow-roll-epsilon-alternative}. 

If \(\epsilon \) is of the order of \(1 / N _{\text{CMB}}\), as happens in large-field models, the excursion of the field becomes \(\Delta \varphi \sim \sqrt{N _{\text{CMB}}} M _{\text{Pl}} \gtrsim M _{\text{Pl}}\).

\todo[inline]{Where does this \(\epsilon \sim 1/N\) come from?}

% GOT HERE

On the other hand, if we have \(\epsilon \to 0\) we can have smaller \(\Delta \varphi \lesssim M _{\text{Pl}}\). 

Do we have to account for quantum gravity if \(\Delta \varphi \gtrsim M _{\text{Pl}}\)? No, since the condition required is actually \(V \leq M _{\text{Pl}}^{4}\).
However, a large (transPlanckian) excursion of the scalar field can constitute a problem, especially if we try to include these models in a ``UV-complete'' theory, so we must be careful.

We have often taken a De Sitter or quasi-De Sitter phase of expansion, but this is not necessarily the case: recall that 
%
\begin{align}
\ddot{a} = a H^2 \qty(1 + \frac{\dot{H}}{H^2}) = a H^2 \qty(1 - \epsilon )
\,,
\end{align}
%
so if \(\dot{H} = 0\) (which defines De Sitter expansion) we do indeed have accelerated expansion, but \(\dot{H} \neq 0 \) does not prevent it, as long as \(\abs{\dot{H}} < H^2\) for negative \(\dot{H}\), or even more easily with \(\dot{H} > 0\) (which corresponds to \(\ddot{a} > \dot{a}^2 / a  > 0\)). 
% \todo[inline]{Check}

In our models we usually had 
%
\begin{align} \label{eq:derivative-Hubble-parameter-scalar-field}
\dot{H} = - 4 \pi G \dot{\varphi}^2 < 0
\,.
\end{align}

This is to say that inflation is not De Sitter, although it can be close to it in certain phases: for starters, it must end at a certain point. 
What kind of fluid could correspond to these other two solutions? 

In general, for a spatially flat FLRW universe, we have 
%
\begin{align}
a(t) = a_* \qty(1 + \frac{1}{\alpha } H_*(t- t_*))^{\alpha }
&& \alpha = \frac{2}{3 (1+w)}
\,,
\end{align}
%
with \(w = \const\). This is a good approximation, at least for the subhorizon phase of inflation.
Accelerated expansion is achieved for \(-1 < w < - 1/3\): for these values the scale factor expands like a powerlaw, this is called \emph{powerlaw inflation}. 

The De Sitter case is one we know. 
The case in which \(\dot{H} >0\) is realized with \(w < -1\); this goes like 
%
\begin{align}
a(t ) \propto (t-t _{\text{asymptote}})^{-\alpha } 
\,.
\end{align}
%

This is called \emph{pole inflation}: it is \emph{superexponential}. 

In the simple models we are considering we always have \(\dot{H} < 0 \), so this is not relevant. 

We have always worked with equations like \(H^2 = \frac{8 \pi G}{3} V(\varphi )\), which starts us off with an \emph{unperturbed} FLRW metric.

How can we be sure that the solution we find is indeed an attractor? It seems like we are requiring very specific initial conditions, not general at all! 

We need the \textbf{cosmic no-hair principle} (sometimes called ``theorem'', although it is not one).
This tells us that starting from very general initial conditions inflation moves us towards a flat FLRW metric. 
This is discussed in \textcite[]{kolbEarlyUniverse1994}. 

We do not have a full GR solution to describe an anisotropic inhomogeneous universe; we consider instead a homogeneous but anisotropic spacetime: a \emph{Bianchi model}. 
The Bianchi classification distinguishes between different kinds of universes like these. 

Bianchi class I universes expand at different rates in different directions: 
%
\begin{align}
\dd{s^2} = - \dd{t}^2 + \sum _{i} a_i^2 (t) \dd{x_i}^2
\,.
\end{align}

We can then define a volume \(V = a_1 a_2 a_3 \), and an average scale factor \(\overline{a} (t) \propto V^{1/3}\): this yields an \emph{averaged} Friedmann equation like 
%
\begin{align}
\overline{H}^2 = \qty(\frac{\dot{\overline{a}}}{\overline{a}})^2 = \frac{1}{9} \qty(\frac{\dot{V}}{V})^2
 = \frac{8 \pi G}{3} \qty[\rho _\varphi + \rho _m + \rho _r]
 + F(a_1, a_2, a_3 )
\,,
\end{align}
%
where the term \(F\) represents the dynamical effects of the anisotropic expansion of the universe. 
It can also include the effects of curvature, with a term like \(- k / \overline{a}^2\).

Note that we could also write a full set of equation from the Einstein ones, to describe the evolution of the three anisotropic scale factors \(a_i\): the point is that writing the averaged equation we did we can explicitly see the effects of the anisotropy on the mean scale factor. This allows us 

Suppose we have a model of inflation with a scalar field \(\varphi \) and a potential \(V(\varphi )\), which drives inflation if it is considered without the anisotropic term \(F\). 

The question is: if we account for \(F\), does it \textbf{``destroy'' inflation}? 
This is to say: under these more general anisotropic initial conditions, can inflation move us towards a flat FLRW universe? 

The evolution of \(\varphi \) will be governed by the Klein-Gordon equation:
%
\begin{align}
\ddot{\varphi} + 3 \overline{H} \dot{\varphi} = - \pdv{V}{\varphi }
\,.
\end{align}

We can then show that usually there is no problem from the anisotropy. 
Let us start with the simplest model, in which we take \(\rho _\varphi = \const\), a cosmological constant. 
% Here we have Wald's theorem. 
% \todo[inline]{What does it state?}

Typically, \(F \propto \overline{a}^{\alpha }\), where \(\alpha \leq -2\). This then means that, if \(\rho _\varphi \) is constant, eventually \(\rho _\varphi \) will dominate. 
Everything else will be exponentially suppressed. 

Under certain assumptions this can be proven formally, and is called Wald's Theorem. 

There can be exceptions: for example, a closed universe with high \(\Omega _k\) initially can collapse before inflation starts. 

We, however, do not have a cosmological constant but a field: does \(\varphi \) roll to the minimum \emph{before} inflation starts, if the anisotropic term \(F\) is present?
It does not. The effect of \(F\) is to give an \textbf{additional contribution to} \(H\), usually increasing it, which modifies the ``friction'' term, which allows for the slow-roll inflation to still occur.

What about completely general \emph{inhomogeneous} as well as anisotropic spacetimes? 
Various analyses have shown (still, not as a mathematical theorem!) that also in this case typically we evolve towards a flat FLRW metric.

\end{document}