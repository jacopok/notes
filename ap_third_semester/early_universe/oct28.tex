\documentclass[main.tex]{subfiles}
\begin{document}

\subsection{Quasi De Sitter solutions}

\marginpar{Wednesday\\ 2020-10-28, \\ compiled \\ \today}

Today we will see how to explicitly solve the equation 
%
\begin{align}
u''_k (\tau )
+ 
\qty[k^2 - \frac{a''}{a} + a^2 m^2] u_k (\tau ) = 0
\,,
\end{align}
%
where \(\tau \) is the conformal time, while \(m^2 = \pdv*[2]{V}{\varphi }\). 
The stage in which \(m^2 = 0\) is the quasi de-Sitter stage. 

Now, we will not assume that \(H\) is constant, and instead we will consider the effect of a nonzero slow-roll parameter \(\epsilon = - \dot{H} / H^2 \ll 1\). 
We will need the relation
%
\begin{align}
\frac{\ddot{a}}{a} = H^2 ( 1- \epsilon )
\,,
\end{align}
%
which can be proven by a straightforward manipulation.

\begin{claim}

Integrating the conformal time definition (we need to also integrate by parts) we can find 
%
\begin{align}
\tau = - \frac{1}{a H (1 - \epsilon )}
\,.
\end{align}
\end{claim}

\begin{proof}
In this quasi-De-Sitter scenario the scale factor will be given, to first order, by 
%
\begin{align}
a &= \exp(\int H \dd{t}) \approx \exp(\int H(t_i) + \dot{H}_i t \dd{t})  \\
&\approx \exp(\int H(t_i) - \epsilon H^2(t_i) t \dd{t} ) = \exp(H_i t - \epsilon H_i^2 t^2 /2)
\,,
\end{align}
%
so the conformal time will read 
%
\begin{align}
\tau &= \int \frac{ \dd{t}}{a} = \int \exp(- H_i t + \epsilon \frac{H_i^2t^2}{2}) \approx \eval{\tau }_{\epsilon =0} + \epsilon \eval{\dv{\tau }{\epsilon }}_{\epsilon = 0}
\,.
\end{align}

The zeroth order term is just \(\tau^{(0)} = - 1/ aH\), while the derivative in the second term is 
%
\begin{align}
\eval{\dv{\tau }{\epsilon }}_{\epsilon = 0} &= \eval{\dv{}{\epsilon } \int \exp(- H_i t + \epsilon \frac{H_i^2t^2}{2}) \dd{t}}_{\epsilon =0}   \\
&= \frac{1}{H_i} \eval{\dv{}{\epsilon }}_{\epsilon = 0} \int \exp(-x + \epsilon \frac{x^2}{2}) \dd{x}  \\
&= \frac{1}{H_i} \int \eval{\exp(- x + \epsilon \frac{x^2}{2})}_{\epsilon = 0 } \frac{x^2}{2} \dd{x}  \\
&= \frac{1}{H_i} \int e^{-x} \frac{x^2}{2} \dd{x} 
= - \frac{1}{H_i} \int e^{-x} x \dd{x} = \frac{1}{H_i} \int e^{-x} \dd{x} = - \frac{1}{a H_i}
\,,
\end{align}
%
therefore the approximation reads 
%
\begin{align}
\tau \approx - \frac{1}{a H_i} \qty(1 + \epsilon ) \approx - \frac{1}{a H_i (1 - \epsilon )} 
\,.
\end{align}
\end{proof}

Using this, we find that to first order \((aH)^2 \approx (1 + 2 \epsilon ) / \tau^2\), therefore
%
\begin{align}
\frac{a''}{a} = a^2 H^2 (2- \epsilon ) 
=2 \frac{(1 + 2 \epsilon)}{\tau^2} \qty(1 - \frac{\epsilon}{2})
\approx \frac{2}{\tau^2} \qty(1 + \frac{3}{2} \epsilon )
\,.
\end{align}

Also, we have the expression \(\epsilon = 2 - a'' a / a^{\prime 2}\) --- all of these can be proven from the definition of \(H\) and \(\epsilon \).

Substituting this for \(a'' /a\) the equation becomes 
%
\begin{align}
u''_k(\tau ) + \qty[k^2 - \frac{\nu^2 - 1/4}{\tau^2}] u_k (\tau )= 0 
\,,
\end{align}
%
where \(\nu^2 = 9/4 + 3 \epsilon \). 
This is a Bessel equation: these equations are generally in the form 
%
\begin{align}
z^2 y'' (z) + z y' (z) + (z^2 - \nu^2) y(z) = 0
\,.
\end{align}

The solutions are called Hankel functions, and the general solution will read 
%
\begin{align}
u_k (\tau ) = \sqrt{- \tau } \qty[ c_1 (k) H_\nu^{(1)} (-k\tau ) + c_2 (k) H_\nu^{(2)} (-k\tau )]
\,,
\end{align}
%
where \(H^{(2)}_\nu = H^{(1), *}_\nu\). 
We want to impose the asymptotic behavior of the solution: let us start with the sub-horizon case \(k / aH \gg 1\). 
This means that 
%
\begin{align}
u_k (\tau ) = \frac{1}{\sqrt{2k}} e^{-ik \tau }
\,,
\end{align}
%
and the asymptotics of the Hankel functions are
%
\begin{align}
H^{(1)}_\nu  (x) \sim \sqrt{ \frac{2}{\pi x}} \exp(i \qty(x - \frac{\pi}{2} \nu - \frac{\pi}{4})) \sim \frac{e^{ix}}{\sqrt{x}}
\,
\end{align}
%
for \(x \gg 1\). This works well for us: we can set \(c_2 (k) = 0\) and only use \(c_1 (k)\). 

We must choose 
%
\begin{align}
c_1 (k) = \frac{\sqrt{\pi }}{2} \exp(i \qty(\nu + \frac{1}{2}) \frac{\pi}{2})
\,
\end{align}
%
in order to have the correct normalization. 
Then, our solution will read 
%
\begin{align}
u_k (\tau ) = \frac{\sqrt{\pi }}{2} \exp(i ( \nu + \frac{1}{2}) \frac{\pi}{2}) \sqrt{-\tau } H^{(1)}_\nu (- k \tau )
\,,
\end{align}
%
so in the \textbf{superhorizon} \(k /aH \ll 1\) (or \(-k \tau \ll 1\)) regime we have the asymptotic \(x \ll 1\) expansion of the Hankel function:
%
\begin{align}
H^{(1)}_\nu (x) \sim \sqrt{ \frac{2}{\pi }} e^{- i \pi /2}
2^{\nu - 3/2} \frac{\Gamma (\nu )}{\Gamma (3/2)} x^{-\nu } \sim x^{-\nu }
\,,
\end{align}
%
therefore 
%
\begin{align}
u_k (\tau ) \approx \exp(i (\nu - \frac{1}{2}) \frac{\pi}{2})
2^{\nu - 3/2}
\frac{\Gamma (\nu )}{\Gamma (3/2)}
(- k \tau )^{1/2 - \nu } \frac{1}{\sqrt{2 k}} 
\,,
\end{align}
%
so if we want the modulus \(\abs{ \delta \varphi _k} = \abs{u_k } / a\) we can make use of the fact that \(\tau \sim - 1/aH \propto 1/a\), therefore \(\abs{ \delta \varphi _k} \approx - H \tau \abs{u_k}\). 
Inserting the expression we have for \(u_k\) yields 
%
\begin{align}
\abs{ \delta \varphi _k} 
&\approx 2^{\nu - 3/2} \frac{\Gamma (\nu )}{\Gamma (3/2)} \frac{H}{\sqrt{2 k^3}} \qty(\frac{k}{aH})^{3/2 - \nu }  \\
&\approx \frac{H}{\sqrt{2 k^3}} \qty( \frac{k}{aH})^{3/2 - \nu }
\approx \frac{H}{\sqrt{2 k^3}} \qty( \frac{k}{aH})^{-\epsilon }
\marginnote{Expanded to first order in \(\epsilon \), so that \(3/2 - \nu \approx - \epsilon \).}
\,,
\end{align}
%
which holds at super-horizon scales. 
In the De Sitter case, the \textbf{exact} result reads
%
\begin{align}
u_k (\tau ) \propto \sqrt{- \tau } H^{(1)}_{3/2} (- k \tau )
= \frac{e^{-ik \tau }}{\sqrt{2k}} \qty(1 - \frac{i}{k \tau })
\,.
\end{align}

The next step is to generalize to a scalar field which is not massless anymore: we shall consider a mass \(m^2 = \pdv*[2]{V}{\varphi } \ll H^2\), so a ``light'' scalar field. 
We require this because one of the slow-roll parameters is \(\eta _V = m^2 / (3 H^2) \ll 1\): a massive field (compared to \(H^2\)) would not be able to drive slow-roll inflation. 

Then, the \(m^2 a^2\) term in the equation is nothing but 
%
\begin{align}
m^2 a^2 = 3 \eta _V a^2 H^2 = \frac{3 \eta _V}{\tau^2}
\,,
\end{align}
%
but we also know that 
%
\begin{align}
\frac{a''}{a} = \frac{2}{\tau^2 } \qty[ 1 + \frac{3}{2} \epsilon ]
\,,
\end{align}
%
so we can still recast the equation into the same Bessel form: only the coefficient of the \(\tau^{-2}\) term changes.

The equation will read 
%
\begin{align}
u''_k (\tau ) + \qty[ k^2 - \frac{\nu^2 - 1/4}{\tau^2}] u_k (\tau ) = 0
\,,
\end{align}
%
with \(\nu^2 = 9/4 + 3 \epsilon - 3 \eta _V\). 
Then, the solution has exactly the same form as before: on super-horizon scales, with \(k \ll aH\), we have
%
\begin{align}
\abs{ \delta \varphi _k} = \frac{H}{\sqrt{2 k^3}} \qty( \frac{k}{aH})^{3/2 - \nu }
&&
\nu \approx \frac{3}{2} + \epsilon - \eta _V
\,.
\end{align}

Note that we did not require the field to be anything but light: the computation we have done can apply to an inflaton field but also to any other scalar field evolving in this phase of the expansion of the universe. 
If it is another scalar field, why should we require it to be light? It can be shown that if \(m^2\) is \textbf{large} compared to \(H^2\) the superhorizon-scale fluctuations of that field have a very hard time being excited: the field basically remains in its vacuum state. 


The Klein-Gordon equation reads 
%
\begin{align}
\ddot{\varphi} (\vec{x}, t)
+ 3 H \dot{\varphi} (\vec{x}, t)
- \frac{\nabla^2 \varphi (\vec{x},t)}{a^2}
= - \pdv{V}{\varphi }
\,,
\end{align}
%
and we have obtained is starting from the expression 
%
\begin{align}
\square \varphi = \frac{1}{\sqrt{-g}} \partial_{\nu } \qty(\sqrt{-g} g^{\mu \nu } \partial_{\mu } \varphi )
\,,
\end{align}
%
where the metric is taken to be a FLRW one. We have neglected a crucial component: we should also \textbf{perturb the spacetime}, beyond the FLRW metric, if we want to have a consistent discussion. 

This is important to do for the inflaton especially since \(\varphi \) dominates the energy density of the universe. 
The Einstein equations, 
%
\begin{align}
G_{\mu \nu } = 8 \pi G T_{\mu \nu }^{(\varphi )}
\,,
\end{align}
%
are perturbed with a variation in the field, \(\delta \varphi \), which perturbs the energy-momentum tensor \(\delta T_{\mu \nu }^{(\varphi )}\), which means that we must also perturb the Einstein tensor \(\delta G_{\mu \nu} \) and then finally the metric \(\delta g_{\mu \nu }\). 

We can define the so-called Sasaki-Mukhanov variable \(Q_\varphi \) associated to \(\varphi \):
%
\begin{align}
Q_\varphi = \delta \varphi + \frac{\varphi }{H} \hat{\Phi}
\,,
\end{align}
%
where \(\hat{\Phi} = \Phi + (1/6) \nabla^2 \chi^{\parallel}\), where \(\Phi \) and \(\chi^{\parallel}\) are scalar perturbations of \(g_{ij}\): 
%
\begin{align}
g_{ij} = a^2(\tau ) \qty[(1 - 2 \Phi ) \delta_{ij} + \chi_{ij}]
\,.
\end{align}

The perturbation \(\chi^{\parallel}\) is given in terms of \(\chi_{ij}\) as we shall see in equation \eqref{eq:traceless-perturbation-decomposition}.
The quantity \(\hat{\Phi}\) is gauge invariant.

The variable \(\hat{Q}_\varphi  = Q_\varphi / a\) obeys an equation in the form 
%
\begin{align}
\hat{Q}_\varphi '' + \qty[ k^2 - \frac{a''}{a} + M^2a^2] \hat{Q}_\varphi  = 0
\,.
\end{align}

In this case we have 
%
\begin{align}
\frac{M^2}{H^2} \approx 3 \eta _V - 6 \epsilon 
\,,
\end{align}
%
therefore the parameter of the Bessel equation reads
%
\begin{align}
\nu^2 &= \frac{9}{4} + 9 \epsilon - 3 \eta _V  \\
\frac{3}{2} - \nu  &\approx \eta _V - 3 \epsilon 
\,.
\end{align}

We then have the most general expression for the field perturbation, accounting for the gravitational perturbation to first order as well:
%
\begin{align}
\abs{ Q_\varphi} \approx \frac{H}{\sqrt{2 k^3}} \qty( \frac{k}{aH})^{3/2 - \nu }
\,.
\end{align}

We want to show how primordial GW are generated from the inflaton perturbations.
We consider a tensor-perturbed FLRW metric 
%
\begin{align}
\dd{s^2} = a^2 (\tau )
\qty[- \dd{\tau}^2 + \qty(\delta_{ij} + h_{ij} (\vec{x}, \tau )) \dd{x^{i}} \dd{x^{j}}]
\,.
\end{align}

We are neglecting all scalar and vector perturbations for simplicity. 
The symmetric tensor \(h_{ij}\) can be chosen to be traceless: \(h^{i}_{i} = \partial_{i}^{\text{BG}} h^{ij} = 0 \), where the derivative is that constructed from the background unperturbed spacetime. 
This is basically the TT-gauge, but in a cosmological context the conditions come about naturally. 

The equations of motion read 
%
\begin{align}
h_{ij}'' + 2 \frac{a'}{a} - \nabla^2 h_{ij} = 0
\,,
\end{align}
%
where a prime denotes a derivative with respect to the conformal time \(\tau \).
If we wanted to use cosmic time instead, the equation would read 
%
\begin{align}
\ddot{h}_{ij} + 3 H \dot{h}_{ij} - \frac{\nabla^2 h_{ij}}{a^2} = 0
\,.
\end{align}

We can see that this is the same as \(\square^{\text{BG}} h_{ij} = 0\). The equation is the same as a \textbf{massless, minimally coupled scalar field}. 
We already know the solution for these equations: the mechanism of quantum vacuum amplification still works. 
We have a 0 on the right-hand side since we are working up to linear order, in full generality there will be other source terms, commonly denoted as \(\pi^{T}_{ij}\) (\(T\) for ``source of tensor modes''). 
This term describes the tensor component of the anisotropic stresses, from the quadrupole up.

Measuring these primordial GWs would be the first probe of the quantum nature of gravity.

GWs start out with 6 degrees of freedom, but 4 are gauged away, so we are left with only 2. 
We can then expand \(h_{ij}\) in a plane wave basis: 
%
\begin{align}
h_{ij} (\vec{x}, \tau )
= \int  \sum_{\lambda = +, \times }
\frac{ \dd[3]{k}}{(2 \pi )^3}
e^{i k \cdot x}
h_\lambda (\vec{k}, \tau ) \epsilon^\lambda_{ij} (\vec{k})
\,,
\end{align}
%
so that in Fourier space the equations of motion for the coefficients read
%
\begin{align}
h_\lambda'' + 2 \frac{a'}{a} h_\lambda' + k^2 h_\lambda = 0
\,.
\end{align}

This is exactly the same as the equations of motion for the scalar massless minimally coupled field. 
We can then see that if \(k \gg aH\) we get 
%
\begin{align}
h_\lambda \sim \frac{e^{-ik \tau }}{a}
\,,
\end{align}
%
while for \(k \ll aH\) the gravitational perturbation has a constant mode and a decaying mode. 
We can treat the two coefficients like \(h_{+, \times } = \phi_{+, \times } \sqrt{32 \pi G}\), \marginnote{The normalization comes from the action.}
so that \(\phi \) has the dimension of a mass while \(h\) is dimensionless. 
Then, like we saw before in the superhorizon regime \(k\ll aH\) we get
%
\begin{align}
\abs{h_{+, \times }} = \frac{H}{\sqrt{2 k^3}}
\qty( \frac{k}{aH})^{-\epsilon } \sqrt{32 \pi G}
\,.
\end{align}

We can distinguish GWs from inflation from those produced by single astrophysical events: inflation yields a \emph{stochastic background} of GWs. 
These GWs are a crucial target of future experiments, both in interferometry and in CMB maps. 

We need to introduce the power spectrum of perturbations: 
the two-point correlation function, in real space, is given in terms of a generic stochastic perturbation field \(\delta (\vec{x})\): we can calculate 
%
\begin{align}
\expval{ \delta (\vec{x} + \vec{r}) \delta (\vec{x})} = \xi (r)
\,.
\end{align}

In Fourier space, we define the power spectrum as the Fourier transform of the two-point correlation function: the field is written as 
%
\begin{align}
\delta (\vec{x}) = \frac{1}{(2\pi )^3} \int \dd[3]{k} e^{i \vec{k} \cdot \vec{x}} \delta (\vec{k})
\,,
\end{align}
%
and the power spectrum \(P(k)\) is defined by
%
\begin{align}
\expval{ \delta (\vec{k}) \delta (\vec{k}')} = (2\pi )^3 \delta^{(3)} (\vec{k} + \vec{k}') P(k)
\,.
\end{align}

The plus is there since we wrote \(\delta (k) \delta (k')\) instead of \(\delta(k) \delta^{*}(k')\): if we included the conjugate than the argument of the \(\delta \) would be \(k - k'\), due to the fact that since \(\delta (x)\) is real we have \(\delta^{*}(k) = \delta (-k)\). 

We have 
%
\begin{align}
\sigma^2 &= \expval{ \delta^2(x)}
= \frac{1}{2 \pi^2} \int \dd{k} k^2 P(k)  \\
&= \frac{1}{2 \pi^2} \int \frac{ \dd{k}}{k} k^3 P(k)  \\
&= \int \dd{\log k } \Delta^2_\delta  (k)
\,,
\end{align}
%
where we define the dimensionless power spectrum: 
%
\begin{align}
\Delta^2_\delta (k) = \frac{k^3}{2 \pi^2} P(k)
\,.
\end{align}

We can show that \(P(k)\) is indeed the Fourier transform of \(\xi (r)\): 
%
\begin{align}
\expval{ \delta (x+r) \delta (x)} &= \frac{1}{(2 \pi )^6}
\int \dd[3]{k_1 } e^{i k_1 (x+r)} \int \dd[3]{k_2 } e^{i k_2 x} 
\expval{\delta (k_1 ) \delta (k_2 )}  \\
&= \frac{1}{(2 \pi )^3} \int \dd[3]{k_1 } \dd[3]{k_2 } \delta^{(3)} (k_1 + k_2 ) P(k_1 ) e^{i k_1 (x+r) + i k_2 x}  \\
&= \frac{1}{(2 \pi )^3} \int \dd[3]{k_1 } e^{i k_1 r} P(k_1 )
\,,
\end{align}
%
as we wanted to show. 
In the \(r = 0\) case we recover the expression from above for the variance. 

\end{document}
