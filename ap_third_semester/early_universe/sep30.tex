\documentclass[main.tex]{subfiles}
\begin{document}

\marginpar{Wednesday\\ 2020-9-30, \\ compiled \\ \today}

\subsection*{Introduction}

Professor Nicola Bartolo, \url{bartolo@pd.infn.it}. 
Office 236 in the DFA department.

Live lectures will be in room P1A, usually at the blackboard, sometimes with slides.
There will be notes uploaded to Moodle.

\paragraph{Course content}

An up-to-date overview of the physics of the Early Universe.
The goal is to be able to understand and analyze the problems from both a theoretical and observational point of view. 

There are three main parts: 
\begin{enumerate}
    \item inflationary models: the issue of the inital conditions;
    \item cosmological perturbations, GW of cosmological origin;
    \item baryogenesis, production of DM particles.
\end{enumerate}

All of these will be connected with observations: nowadays cosmology is data-driven. 

There are connections with: cosmology, astroparticle physics, astrophysics, GR, theoretical physics, field theory, multimessenger astrophysics, GW. 

We will do ``blended learning'', with shifts in the classroom. 
We will do 48 hours of lectures. 

Textbooks: \textcite[]{lythPrimordialDensityPerturbation2009} ``The primordial Density Perturbation'', \cite[]{liddleCosmologicalInflationLargeScale2000} ``Cosmological inflation and Large-Scale Structure''.

There will be 2 exam dates for each session. The exam is an oral one. 
Office hours can be arranged anytime. 

We should limit questions in the break, prefer asking them during the lecture itself.

\chapter{Inflationary models}

\section{A general introduction}

The basic issue is to find what initial conditions would produce the universe as we currently observe it.

Observational probes of the Hot Big Bang model include the Hubble diagram (redshift against distance for galaxies), Big Bang nucleosynthesis, the CMB. 

On large scales we observe a \textbf{smooth universe}. However, that is a ``zeroth-order'' approximation: there are structures and anisotropies. 
All the structures need initial conditions to start from and then grow through gravitational instability. 

We have several observables to probe the anisotropies: CMB, LSS, clusters of galaxies, weak gravitational lensing. 
There are \textbf{initial fluctuations} on the order of 
%
\begin{align}
\frac{ \delta \rho }{\rho } \sim \frac{ \delta T}{T} \sim \num{e-5}
\,.
\end{align}

What is the initial time and temperature at which these perturbations start? Is there a \textbf{dynamical} mechanism which produces the perturbations? 
How do the perturbations evolve exactly? 
How do they relate to baryogenesis?

Under a Newtonian treatment, relative density perturbations grow like \(\delta _m \propto a(t)\). 
The problem we will address here is how the initial value of \(\delta _m\) comes about. 

The CMB is a very good blackbody, without spectral distortions except for the Sunyaev-Zel'dovich effect (inverse Compton scattering from high-energy electrons in galaxies up-scattering the CMB photons). 

% The number density of photons is of the order of \(n_\gamma \approx \SI{422}{cm^{-3}}\). 
We recall some basic concepts about the smooth model of the universe: critical density, Hubble parameter and so on. 

The standard \(\Lambda \)CDM model does predict a small deviation, \(\mu / T \sim \num{1.9e-8}\), from the Planckian, whose phase space distribution is: 
%
\begin{align}
f = \qty[\exp(\frac{h \nu - \mu }{k_B T}) - 1]^{-1}
\,.
\end{align}

The number density of photons can be extracted from this distribution: 
%
\begin{align}
n_\gamma = \int f \dd[3]{p} = \frac{4 \pi }{c} \int \frac{I_\nu}{h \nu } \dd{\nu } \approx \SI{422}{cm^{-3}}
\,.
\end{align}
%

Currently we have upper bounds: \(\mu / T < \num{9e-5}\) at \SI{95}{\percent} CL. 

The CMB radiation is also highly, but not perfectly, isotropic. the scale of the temperature angular anisotropies are of the order \(\Delta T / T \sim \num{e-5}\) (the quoted value for \(\Delta T/T\) is a root-mean-square, since the average of \(\Delta T\) is zero). 
This is to say: in each direction we observe a very good blackbody, whose characteristic temperature changes slightly depending on the direction.

Planck 2018 had an angular resolution of 5 arcminutes, and it also measured the polarization of the CMB. 

We also have redshift galaxy surveys like the Sloan Digital Sky Survey. We map galaxies in redshift space. 
There is a statistical pattern of the galaxies, which is connected to the origin of the inhomogeneities. 

The idea is that the seeds of the perturbations are quantum mechanical, coming from the inflaton scalar field, which are made into galaxies and galaxy clusters from gravitational instabilities.

At \(z \sim 20\) the DM distribution was quite smooth, it then clustered.\footnote{\url{https://www.youtube.com/watch?v=FBkYIqtYb0I}.}

The components of the \(\Lambda \)CDM model are: 
\begin{enumerate}
    \item dark energy \SI{68}{\percent};
    \item dark matter \SI{26}{\percent};
    \item hydrogen and helium gas \SI{4}{\percent};
    \item stars \SI{.5}{\percent};
    \item neutrinos \SI{.26}{\percent};
    \item metals \SI{.025}{\percent};
    \item radiation \SI{.005}{\percent}.
\end{enumerate}

These numbers are expressed as fractions of the critical energy density, 
%
\begin{align}
\rho _{0,\text{ crit}} = \frac{3 H_0^2}{8 \pi G}
\,,
\end{align}
%
and the fact that they approximately add up to 1 means that the total energy budget of the universe is compatible with spatial flatness. 

We also need seed perturbations and baryo-leptogenesis.
We will see phases in which the universe is not in thermal equilibrium.

We want to find information about energies up to \SI{e16}{GeV}: we will see that the inflationary phase corresponds to this epoch.

GW from inflation travel basically unimpeded from inflation to us. 

Today, we have \textbf{radiation} with \(w = 1/3\), \(\rho \propto a^{-4} \propto T^{4} \), so, Tolman's law \(Ta = \const\). 

\textbf{Baryonic matter} has \(\Omega_b h^2 = \num{.0224(1)}\).
Its equation of state is \(P = nT \ll nm\). So,  \(\rho \propto a^{-3}\). 
Dark matter is also nonrelativistic, with \(P \approx 0\), and \(\Omega_{DM} \approx h^2 \num{.120(1)}\).

The \textbf{cosmological constant} has \(P = - \rho \), and \(\Omega _\Lambda = \num{.6847(73)}\). 

Neutrinos have \(\sum m_\nu < \SI{.12}{eV}\), and \(\Omega _\nu h^2 < \num{.0012}\). Both values are at \SI{95}{\percent} CL. 

Spatial curvature has \(\Omega _k = 1- \Omega_0 = \num{.001(2)}\), from Planck, Baryon Acoustic Oscillations, local measurements.

The presence of discordance can surely signal systematics, but also new physics. There are certain discordances. 

Beyond galactic rotation curves, we also have evidence for DM from the power spectrum of inhomogeneities. 
We Fourier-transform the density perturbation field \(\delta \) to get \(\delta _{\vec{k}}\); then we can calculate the power spectrum 
%
\begin{align}
\Delta^2 (k) = \pdv{ \sigma^2}{\log k} \propto k^3 \abs{ \delta _{\vec{k}}}^2 \propto k^{3+n} T^2 (k)
\,.
\end{align}

The Poisson equation reads 
%
\begin{align}
4 \pi G \overline{\rho} \delta = \nabla^2 \Phi \implies \delta _{\vec{k}} \propto k^2 \Phi_{\vec{k}}
\,.
\end{align}

If the gravitational perturbation is written as 
%
\begin{align}
\Phi _{\vec{k}} = \Phi _{\vec{k}}^{\text{primordial}} T(k) \times \text{growth function}
\,,
\end{align}
%
and the primordial field perturbation squared is \(\abs{\Phi _k^{\text{primordial}}}^2 \propto k^{n-4}\), where the power spectral index reads \(n =\num{.9600(42)}\).
The growth function is commonly denoted as \(g(a)\). 
This explains the last proportionality sign we wrote earlier, the density power spectrum includes information about the power-spectral index of the initial conditions. This index measures the amplitude of the inhomogeneities in DM density. 

Also, if we only had baryons \textbf{without dark matter} the power spectrum of the matter density perturbations would look \textbf{very different} from what we see.

Baryon Acoustic Oscillations are an oscillatory imprint in the power spectrum, they have been measured today.

Inflation is an early epoch in the history of the universe during which expansion is accelerated. 
The basic predictions of inflation are so far confirmed, however we have not detected the SGWB from it, which would be a ``smoking gun''.

\end{document}
