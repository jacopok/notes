\documentclass[main.tex]{subfiles}
\begin{document}

\marginpar{Wednesday\\ 2020-4-8, \\ compiled \\ \today}

\section{``In Viaggio verso le Stelle'', l'astronomia nel pensiero cinese tradizionale}

Sezione tenuta dal professor Ghilardi. 

Il pensiero cinese ci appare ``uniformato'' se guardato dall'occidente, ma è variato tanto quanto quest'ultimo. 

Il Tao è un punto centrale di questo pensiero, che è differente dalla nozione occidentale di filosofia, non c'è un'ontologia, non c'è una distinzione fisico-metafisico. 
Letteralmente tradotto, significa ``via'' come sostantivo, o ``andare'' o ``proferire'' come verbo. 

Tao è sia la physis, generazione delle cose, che il metodo umano di procedere. 

Le scuole filosofiche cinesi si sviluppano fra il sesto e il (?) secolo BCE,
``epoca degli stati combattenti'', prima dell'Impero. 

Nessuno parla di vita contemplativa qui: si conosce il mondo attraverso il saper fare, che permette di armonizzarsi con la natura. 
I movimenti della natura sono armonici: l'essere umano è l'unico ente che rischia di deviare da questo corso. 

Si parte dal termine \emph{Xiaoyaoyou}: la capacità di evolvere, come un uccello che si muove in cielo. 

Questo è direttamente opposto al mito dell'eroe occidentale che va contro. 
Il saggio è colui che sa evolversi, ``surfare sulle onde'' della realtà. 
Se si opponesse alla potenza del mare, ne verrebbe travolto. 

Questo non è fatalismo: è intelligenza dello sfruttare le correnti, divenendo tutt'uno con il Tao. 

Immagine: antico diagramma del taiji. 
Questo simbolo ha un migliaio d'anni, non esisteva la tempo di Confucio. 
Dicotomia e complementarietà fra yin e yang. 
Il Tao è il movimento del taiji, l'accadere dei fenomeni. 

L'equilibrio dello yin e dello yang è sempre dinamico. 
A partire dall'indistinzione originale, si crea la realtà e con essa gli effetti del Tao.
Vi sono cinque ``elementi'', che sono fasi di un processo.

C'è una continua transitività fra visibile e invisibile. 

Citazione di Saxl, \emph{Storia della biblioteca Warburg}, in E. Gombrich, \emph{Aby Warburg}, Feltrinelli 2003, pp 169-170.

L'atto fondamentale della conoscenza umana è orientarsi di fronte al caos attraverso la posizione di immagini e segni. 
Ci sono due modalità di orientamento: trarre a sè il cosmo, e contenere, governare la distanza tramite un linguaggio particolare, come quello matematico. 

Fino al 1400, la superiorità cinese non è stata utilizzata per scopo di conquista o scoperta, mentre la superiorità europea nell'età moderna ha avuto l'``andare oltre'' come paradigma. 

Il pensiero cinese non ha un'impostazione di causa ed effetto, bensì tratta i flussi reciproci. 

La modalità non matematizzante è meno efficace, ma tiene insieme umano e naturale: non ``dualizza''. 

\subsection{Cosmogonie taoiste}

Daodejing: ``Il Tao\footnote{Talvolta si scrive Dao, è solo una traslitterazione diversa.} genera l'uno, l'uno genera il due, il due genera il tre, il tre genera le diecimila cose.
Le diecimila cose si lasciano dietro lo \emph{yin} e vanno verso lo \emph{yang}.''

Non vi è un Dio creatore, il Tao non è un'entità cosciente. 
Il Tao è un movimento infinito, le cui onde sono i fenomeni. 

Huinanzi: ``Del Tao è detto pertanto: il suo inizio è nell'uno; l'uno non può generare, perciò si divise trasformandosi in \emph{Yin} e \emph{Yang}. 
Dall'unione armonica dello \emph{Yin} e dello \emph{Yang} hanno avuto origine tutte le cose.
Per questo è detto: l'uno dà origine al tue, il due al tre e il tre a tutte le cose.

Questo è in un certo senso simile a quello che scrivono i presocratici, ma questa cosmogonia è sempre volta al ``cosa posso fare io''?

\subsection{Miti cosmogonici: il mito di Pangu}

Non ha senso interrogarlo in senso logico.

All'inizio del tempo c'era l'oscurità, il mondo era un uovo che conteneva il caos. 
Dentro all'uovo c'è il gigante Pangu con una scure. 
Le due parti dell'uovo vanno a formare lo yin e lo yang, e il gesto di Pangu è di separarle, tenerle separate: 
il tema della scissione è ricorrente, anche il termine tedesco per ``giudizio'' significa ``scissione''. 

Quando il gigante decide di morire, il suo corpo diventa la Terra. 
L'uomo non è, come nel mito biblico, centrale nella creazione: al contrario, è eccentrico in quanto gli uomini si formano dalle pulci sul corpo di Pangu. 

Un'altro mito, questa volta \textbf{cosmologico} (non cosmogonico) è quello dell'arciere di Yi. 
La terra è arsa dal calore dei dieci Soli, e l'arciere, un semidio, deve riportare l'equilibrio. 

Dal punto di vista del Tao, il fatto che una specie si estingua o meno è indifferente. 
I saggi taoisti dicono che se siamo in grado di assecondare lo scorrere naturale della natura allora riusciremo a preservare la nostra. 

\subsection{La cosmologia cinese}

La cosmologia cinese è testimoniata da reperti archeologici, derivanti da oggetti utilizzati per la scapulomanzia: gusci di tartaruga incisi, lasciati a seccare; si interpretavano le loro crepe. 

Il cielo era suddiviso in 28 case lunari (28 come i giorni del ciclo lunare); divisi in 4 palazzi celesti, abbinati ai 4 punti cardinali e le 4 stagioni: 
\begin{enumerate}
    \item Tigre bianca (W);
    \item Tartaruga nera (N);
    \item Drago azzurro (E);
    \item Fenice rossa (S).
\end{enumerate}

In più, l'Orsa Maggiore: Drago Giallo. 
Ci sono riferimenti a queste nei manga (es. maestro Tartaruga in Dragonball).

L'oroscopo cinese non si basa sui dodici mesi dell'anno, ma invece ha un ciclo di dodici anni. 
Questi sono poi moltiplicati per i 5 elementi (fuoco, terra, acqua, legno, metallo) per ottenere un ciclo di sessant'anni.

La costellazione di Orione era considerata una sagoma umana anche per i cinesi. 

Quando la società cinese inizia a studiare le stelle è già burocratizzata, quindi abbiamo nomi come ``funzionario minore'' e altre cose legate alla quotidianità. 

La scienza in Cina ha studiato diversi fenomeni: il movimento dei corpi celesti, la medicina, zoologia e botanica, musica e magnetismo. 
Quest'ultimo, in particolare, ha una polarità intrinseca quindi era interessante al tempo. 

Il termine ``scienza'' non esiste in cinese. 

Approccio degli studiosi cinesi di fine 1800: come si relaziona il \emph{daotong} (visione complessiva dell'uomo) cinese con la scienza?
È possibile distaccare il \emph{daotong} cinese classico dalla scienza tradizionale cinese?

Il concetto tradizionale di ``natura'' cinese, \emph{ziran}, corrisponde a ``ciò che spontaneamente si dà''. 

Non si pensa in termini di sostanze, ma di mutamenti, processi.
Non esiste l'``ente''. 

Il sapere è sempre sapere per fare. 

Il cinque è un numero ricorrente. 
Non si sviluppa un sistema assiomatico: si trova un analogo del teorema di Pitagora, ma non viene ``dimostrato'' con l'approccio euclideo.

La dimensione mitica c'è anche dal lato occidentale: anche Keplero cercava connessioni di stampo mitologico. 
C'è una necessità di un paradigma comune, uno sfondo indiscusso, come in Omero era il fatto che gli dèi abitino gli uomini per influenzarne le azioni. 

Concetto di tempo: ci sono i concetti di momento-occasione, durata, ``spazio-tempo'' nel senso di universo, o firmamento. 
Più recentemente, si forma il concetto formale di tempo ``momento-intervallo''. 
Nella concezione classica, il tempo è informale, stagionale, qualitativo.

Come si distinguono la scienza, la filosofia dalla mitologia? C'è una differenza della tenuta logica dei discorsi, ``rendere conto'' delle affermazioni. 
La scienza in senso moderno si basa sull'esperimento.
C'è la questione della matematizzazione, e della formazione di connessione fra cause ed effetti. 

Il mito, invece, non è messo in questione.

La scrittura ideografica cinese non si presta all'algebra.
La scrittura alfabetica permette più facilmente di sganciare il significato dalle lettere. 

In che lingua parlano gli studiosi di fisica, astronomia etc? 

Francois Jullien: ``L'invenzione dell'ideale e il destino dell'Europa''.

\end{document}
