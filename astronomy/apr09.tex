\documentclass[main.tex]{subfiles}
\begin{document}

\marginpar{Thursday\\ 2020-4-9, \\ compiled \\ \today}

Svastica simbolo cinese prima che nazista: \emph{wan}, diecimila cose, uncino - dieci, quattro uncini \(10^{4}\).

Motore immobile: il perno della ruota sta fermo, e con ciò permette che la ruota giri. 
Vuoto è la massima possibilità, capacità di accoglienza.
Riempire di cose la propria vita è peggio che ``farsi vuoti''.

Le stelle ``accadono'', non hanno volontà. L'uomo deve naturalizzarsi ed imitarle. 

\subsection{Modelli cosmologici}

\textbf{Gai Tian}: ``cielo a ombrello'', Terra quadrata, cielo emisferico a coperchio. Stella Polare punto di riferimento fondamentale. 

Le stelle non sono ``incollate'' alla volta celeste, invece si muovono, perché tutto è in costante trasformazione. 

La Cina è comunque centrale, asse del mondo, e non è interessata ad esplorare le periferie. 

Questa teoria viene scartata in quanto una Terra quadrata non si adatta bene ad un cielo tondo. 

Una nuova teoria, lo \textbf{Hun Tian}, prevede una Terra tonda. 

L'idea qui è che il cielo è grande, la Terra è piccola.
Questa polarità è chiave nel pensiero cinese classico: il pensare bene, per i cinesi, non è concatenare argomentazioni ma trovare dicotomie, opposizioni. 

La Terra galleggia nel vapore, e c'è un tentativo di ragionare: il disco terrestre è innalzato e abbassato fra estate e inverno, quindi la temperatura cambia. 

\todo[inline]{Maree? Le conoscevano, ne tenevano conto?}

\textbf{Xuan ye}: ``notte che si espande''. Questa è la più vicina alla modernità: nega l'esistenza di un cielo con forma e sostanza, chiarore e oscurità celesti sono fenomeni apparenti, il cielo non ha davvero sostanza e colore e l'universo è infinito. 

Questo non ha provocato un ``crollo narcisistico'' come quello causato  in Occidente dalle teorie eliocentrica ed evoluzionistica. 

Nel mondo cinese non c'è un analogo sentore di disfatta: l'uomo non era mai stato pensato come il fine della creazione. 

C'è inoltre uno sviluppo della pratica: il popolo cinese non dava valore allo studio contemplativo in sè.
Un esempio è la \emph{sfera armillare}, si conoscono sfere armillari dal primo secolo BCE.

Sono inventati anche i primi orologi ad acqua. 

Incontro con la scienza europea: Padre Matteo Ricci vive in Cina gli ultimi anni della propria vita, poi c'è anche Padre Ferdinand Verbiest.
Questi attorno all'inizio del 1600: sono gesuiti. 

In Cina non c'è nulla di simile a quello che accade nel Giappone raccontato nel \emph{Silence} di Scorsese. 

Il corpo umano nel taoismo è metafora del cosmo, e viceversa. C'è un cielo anteriore, \emph{xiantian}, e un cielo posteriore, \emph{houtian}.

La posizione degli astri del cosmo, in questa interpretazione, ha un'influenza su come sarà una persona. Questo è il cielo anteriore.

Il cielo posteriore, invece, consiste in quello che effettivamente accade nella vita. 

C'è idea di regolazione invece che predizione. 

Non c'è una parola, propriamente, per dire ``creazione'': c'è la modifica, tu non hai ``fatto'', hai modulato qualcosa di preesistente. 

Dallo Zhang-zi: non puoi nemmeno pretendere di osservare il cielo, se non conosci l'unità del Tao. 

L'amore verso la propria psiche è ciò da cui bisogna allontanarsi.

Tien è contemporaneamente cielo nel senso di \emph{sky} e nel senso di \emph{heaven}. 

\end{document}
