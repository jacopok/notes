\documentclass[main.tex]{subfiles}
\begin{document}

\marginpar{Thursday\\ 2020-4-16, \\ compiled \\ \today}

Dall'\emph{Iliade}: le armi di Achille brillano come stelle, nello specifico come Sirio, il cane di Orione. 

Il viaggio dell'Odissea è ritorno guidato dalle, e verso le stelle.
Odisseo seduto al timone guida abilmente, senza addormentarsi, e guarda le stelle. 
Orsa l'unica che non si bagna in Oceano.

Nell'età ellenistica l'approccio è diverso: i viaggi sono di ``fuga'' verso le stelle.

Distinzione concettuale:
\begin{enumerate}
    \item osservazione del cielo;
    \item imposizione di nome alle stelle e ai gruppi di stelle;
    \item dottrina delle influenze degli astri sugli eventi terreni.
\end{enumerate}

Questa dottrina astrologica diventa rilevante nella cultura greca nel periodo ellenistico.
Alessandria culla di questo tipo di conoscenze.

\emph{Fenomeni} di Eudosso di Cnido: attorno al 370--360 BCE, descrive tutte le costellazioni con descrizione delle singole stelle. Modello geometrico, con sfera armillare.

Vi sono miti riguardanti sia costellazioni che, talvolta, singole stelle. 

Arato di Soli, fra il 315 e il 245 BCE, è un poeta che attinge a Eudosso. Secondo la tradizione, Antigono Gonata di Macedonia gli affida un lavoro: rendere più accattivanti le conoscenze di Eudosso, divulgarle.

Molti autori latini lo riprendono: Cicerone, Germanico, Avieno. 
Anche autori carolingi, che condannano le \emph{fabul\ae\ poetarum}, o \emph{gentilium deliramenta} (delirî pagani).

Tuttavia devono riprendere questi testi, in quanto la conoscenza è intimamente legata alla forma poetica e alla mitologia.

Erudito è chi è in grado di tradurre (e anche talvolta tradire) Eudosso.
L'opera di Eudosso inizia celebrando Zeus, non come \emph{primus inter pares} come nella Teogonia di Esiodo tuttavia: questo Zeus è invece più stoico, esercita una sorta di ``provvidenza'' rispetto alla realtà umana.
Arato, nel proemio, dice che Zeus è dappertutto. Orienta la vita umana, ed egli stesso fissa i segni nel firmamento. 
Per questo lo richiama, chiamandolo thauma, grande portento.

La costellazione della Vergine è cruciale per Arato, perché rappresenta Dike, la giustizia. Un tempo questa era concittadina degli uomini, sedeva in mezzo a loro, radunava gli anziani e là cantava sentenze eque per il popolo. In quell'epoca gli uomini non conoscevano ancora la dolorosa contesa né il fragore della guerra.

Giustizia era portatrice di doni legittimi, e forniva tutto in abbondanza.
Quando venne l'età della stirpe d'argento, che iniziò a comportarsi diversamente dalla stirpe d'oro, Giustizia iniziò a frequentare meno gli uomini. 
Ammonì gli uomini, e tornò nelle montagne. 
Gli uomini di bronzo furono i primi a mangiare la carne.
Nel mondo greco classico l'uccisione non sacrificale del bue aratore era paragonata all'omicidio.
Allora Giustizia si trasferisce per il loro comportamento fra le stelle.

È un mito eziologico: trova l'origine, la causa.

In età imperiale Amato Marcellino parla di Giuliano citando Arato, e dicendo che l'imperatore Giuliano è colui che riporta Giustizia dalle stelle.

Arato esordisce parlando delle Orse: esordisce premettendo che la storia è narrata, con ``Se si tratta di cosa veritiera\dots''

Spesso i miti delle Orse riguardano l'infanzia di Zeus, Zeus infante che la madre Rea doveva proteggere da Crono. 

Cinosura ed Elice sono le Orse che lo allevano mentre si nasconde a Creta, in attesa di spodestare il padre.
Le Orse salgono al cielo per volontà di Zeus successivamente.

Nel testo di Arato c'è il cosiddetto \emph{katasterismos}: collocazione in stella, la raffigurazione in stella.

Il tema riappare nei \emph{Catasterismi} di Eratostene: introduzione generale all'astronomia e ai suoi problemi scientifici, con origini mitiche delle costellazioni. 
A noi è rimasta solo un'Epitome di questo testo.

Quest'Epitome preserva una classificazione sistematica delle figure rappresentate, il racconto mitico (con le varianti!), cataloga le stelle anche per luminosità, considerando anche i nomi delle stelle e le attività umane ad esse collegate.

Molti miti sono miti di \emph{stabilizzazione}: storie di Zeus, Eracle. 

Più di metà delle costellazioni ruotano attorno alle vicende di Zeus e dei suoi figli. Inoltre, ci sono storie degli amori di Zeus.
Vi è un ciclo dedicato a chi ha sfidato l'ordine cosmico: Cassiopea, Orione, il Corvo, Serpente.
Inoltre vi sono miti di \emph{hybris}.

Inoltre ci sono miti \emph{eziologici}, nei quali l'uomo ``civilizza''.

Spesso i miti, invece che di amore, raccontano di violazione, di ratti: ad esempio, il ratto di Proserpina ad opera di Ade, reso dal Bernini. 
Il tema è spesso quello della figura maschile che tenda di prendere la figura femminile, che cerca di sfuggire.

Igino restituisce una serie di varianti rispetto ai miti astrali. 

Inizia con 6 varianti della storia dell'Orsa Maggiore.
Dice che Esiodo dice che si chiamava Callisto (ovvero bella).
Entrò nel seguito di Diana, tuttavia poi fu violata da Giove e per questo fu punita essendo trasformata in orsa.

In un'altra versione, Callisto accusa Diana e per questo è punita.

Da che punto ha assunto carattere importante anche l'aspetto ludico del mito?
È difficile trovare un punto in cui il mito sia \emph{solo} \emph{divertissement} nella storia romana.

Anche successivamente il mito viene rivalutato e reinterpretato ma non diventa davvero intrattenimento.

La riscrittura del mito è sia necessaria conseguenza dell'evoluzione sociale, che del volere dei potenti. 

Esempio: faccio un sacrificio rituale come aveva fatto Prometeo. 
Combattiamo Saddam come avevamo combattuto Hitler.

\end{document}
