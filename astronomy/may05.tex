\documentclass[main.tex]{subfiles}
\begin{document}

\section{Astronomia sferica e pratica}

\marginpar{Tuesday\\ 2020-5-5, \\ compiled \\ \today}

La \textbf{volta celeste} è la superficie semisferica sulla quale appaiono proiettate le stelle. Ha un raggio indeterminato, non abbiamo modo di determinarne la dimensione: talvolta è comodo pensarla come di raggio infinito, talvolta invece la pensiamo come di raggio unitario.

La caratteristica più evidente della volta celeste è la \textit{Via Lattea}; nell´emisfero australe è possibile vederne il centro. 

\todo{Vedere mito greco della sua nascita, con Giunone, Ercole e Giove.}

Inoltre, vediamo le due Nubi di Magellano: le due galassie a noi più vicine. 

Opera di James Turrell: ``Celestial Vault'', ci si può stendere in una piattaforma con un orizzonte artificiale che scherma dall'inquinamento luminoso. 

Il \textit{Primo Moto} di Raffaello mostra una precisa configurazione astrale: quella del 31 ottobre 1503, anno in cui il suo pontefice entra in carica. 

\end{document}