\documentclass[main.tex]{subfiles}
\begin{document}

\subsection{Coordinate terrestri}

\marginpar{Monday\\ 2020-5-18, \\ compiled \\ \today}

Il modello più semplice per la forma della Terra è una sfera.
Un modello un po' più accurato è un ellissoide oblato: questo è il solido di rotazione che si ottiene ruotando un'ellissi lungo il suo asse minore. Matematicamente, in coordinate cartesiane \(x\), \(y\) e \(z\) è il luogo dei punti che soddisfano 
%
\begin{align}
\frac{x^2}{a^2} + \frac{y^2}{b^2} + \frac{z^2}{c^2} = 1
\,,
\end{align}
%
dove, se l'asse della Terra è lungo \(z\), abbiamo \(a = b > c\).

Per la sfera abbiamo una precisione \(\sim \SI{10}{km}\), con l'ellissoide \(\sim \SI{100}{m}\) (a meno delle montagne).

Eratostene aveva stimato il raggio della Terra: Alessandria - Siene sono \num{5000} stadi, l'angolo dell'ombra dell'obelisco di Alessandria è \(1/50\) di giro. 
Uno stadio sono circa \SI{160}{m}, quindi la stima di Eratostene era estremamente vicina al valore accettato (entro l'errore con il quale conosciamo il valore dello stadio in metri).

Per mettere delle coordinati sulla superficie terrestre scegliamo un asse, quello di rotazione, che definisce l'equatore come cerchio massimo normale, e fissiamo un meridiano di riferimento: comunemente si utilizza Greenwich. 
In questo modo abbiamo due angoli: la latitudine è l'angolo dall'equatore al punto, misurato lungo il meridiano (che è il cerchio massimo definito dal punto e i due poli); la longitudine è l'angolo fra il nostro meridiano e quello di riferimento. 

Un grado di latitudine corrisponde a circa \SI{111}{km}, per la longitudine dobbiamo inserire un fattore \(\cos(\theta )\). 

Un miglio nautico è definito come l'angolo sotteso da un primo d'arco. 

Con un orologio, un sestante e delle tavole astronomiche possiamo misurare la posizione di una nave in mare: la latitudine si vede traguardando il polo celeste Nord, la longitudine è più articolata ma con l'orologio si piò fare.

Considerando la Terra come ellissoide, possiamo definire delle coordinate \emph{geodetiche}.
Il sistema migliore, però, è il WGS84: la forma che avrebbe una superficie d'acqua se le terre emerse non ci fossero, ma la loro gravità sì. 

\end{document}