\documentclass[main.tex]{subfiles}
\begin{document}

\section{Astronomical accuracy through the ages}

Astronomy has historically been among the first of the \emph{exact sciences}, necessarily dealing with precise measurements, numbers and geometry.
Its objects of interest were often tied with religion, as one might expect given the strong connection between ritualism, repetition and regularity in human culture.
Stars' and planets' positions were often interpreted as omens, whose interpretation and prediction was crucial, and the starting point for predictions is naturally observation; \emph{precise} observation is needed in order to make accurate predictions. 

In this section we will briefly explore some aspects of the advancement of the techniques used for the measurement of angles from the pre-magnification era to the current day.
In the next section, we will discuss two examples of applications of astronomical knowledge to the study of history and literature. 

\subsection{Pre-magnification angle measurement}

The unaided eye can generally resolve objects displaced by an angle of the order of \SI{2}{^\prime} \cite[pag.\ 56]{karttunenObservationsInstruments2017}.\footnote{This holds for people with good eyesight, the unfortunate bespectacled ones such as myself can hardly distinguish Mizar from Alcor in the constellation of Ursa Major, and their separation is roughly \SI{12}{^\prime}. }
However, precise measurement of the positions of the stars requires not only visual acuity but also precision instrumentation and, crucially, \emph{timekeeping}. 
% Nevertheless, in certain specific cases ancient measurements were made with remarkable accuracy. 
% One example is Hipparchus' estimate of the Earth-Moon distance through the observation of an eclipse \cite[fig.\ 56]{northCosmosIllustratedHistory2008}. 

Hipparchus and Erathostenes measured the obliquity of the ecliptic to within \SI{10}{^\prime} of the true value \cite[pag.\ 109]{northCosmosIllustratedHistory2008}. 

One might suspect that no one would have been able to reach precisions below the eye's resolution before the invention of telescopes; however this might not have been the case, at least insofar as solar observation went.

\textcite[]{sattarovMainInstrumentUlugh2009} suggested that astronomers from the 1400s in Samarquand (Uzbekistan) could have measured the obliquity of the ecliptic with an extrremely low error, using an instrument known as a Fahri sextant.
This is basically a very large structure with a pinhole and an excavated circular arc of radius \(R \sim \SI{40}{m}\) with markings; through the basic principle of the \emph{camera obscura} the position of the Sun could have been measured with staggering accuracy, potentially even down to \SI{5}{\arcsec}.
Historic reports of the measurements have been found to have an error of \SI{32}{\arcsec}: perhaps not as good as the instrument could have achieved, but still a remarkable result without any optical aid. 

\subsection{Modern era improvements}

The accuracy of angular measurements skyrocketed in the modern era: between the 1500s and the 1800s there was roughly an order of magnitude improvement in accuracy per century \cite[]{chapmanAccuracyAngularMeasuring1983}.

In the 1500s Tycho Brahe introduced diagonal lines between the markings of his instruments: with the same basic principle as a Vernier scale, this allowed for precise measurements, which can be said to have reached the limit of naked-eye accuracy, at around \SI{1}{^\prime}. 

In the 1600s the invention of telescopic sights and micrometry combined to yield a very large improvement in accuracy, reaching accuracies in the tens of arcseconds. 

The next big leap, which allowed astronomers to breach the arcsecond barrier in the 1800s, consisted of improvements in the construction of scales, and a movement from quadrants to full graduated circles. 

\subsection{Current state of the art}

The limit nowadays, for space-based instruments like the Hubble Space Telescope which are not affected by the atmosphere, is the intrinsic wavelike nature of light: the equation for the \emph{diffraction limit} is 
%
\begin{align}
\theta \approx \num{1.22} \frac{\lambda }{D}
\,,
\end{align}
%
where \(\theta \) is the minimum angular distance of two points we can distinguish,\footnote{This is to be interpreted in the precise sense that two points are \emph{distinguished} if the first diffraction minimum of one coincides with the peak of the diffraction pattern of the other.} \(D\) is the diameter of our telescope, while \(\lambda \) is the wavelength of the light we are considering.
For example, the HST has a diameter of \SI{4.2}{m}, so for visible green light at \SI{500}{nm} its diffraction limit is of around \SI{30}{mas} (milli-arcseconds).
Beyond this limit objects cannot be resolved. 
Telescopes on Earth are commonly limited not by this problem but, instead, by the distorting effect of the atmosphere. 

Higher resolutions can be obtained by using shorter wavelength light, or larger telescopes. 
For example, the Event Horizon Telescope \cite[]{theeventhorizontelescopecollaborationFirstM87Event2019} used the technique of Very Long Baseline Interferometry in order to achieve an effective telescope diameter of the scale of the Earth, which combined with its observation wavelength of \SI{1.3}{mm} (in the radio band) meant that its diffraction limit fell to the order of \SI{25}{\micro arcseconds}.

\section{Equinox precession as a dating tool}

The Earth's axis of rotation is tilted by an angle of around \SI{24}{\degree} with respect to the \emph{ecliptic}, which is the plane in which the Earth's orbit around the Sun lies. 

In the time it takes for the Earth to complete a revolution around the Sun the orientation of this axis with respect to the distant stars is roughly constant --- this can be seen easily in the night sky from the fact that the sky appears to revolve around a fixed axis, which in the Northern Hemisphere is right now close to the star Polaris.\footnote{Currently the separation of the star from the northern celestial pole --- which corresponds to \(\SI{90}{\degree} - \delta \), where \(\delta \) is the declination of Polaris in an equatorial coordinate system at a chosen equinox --- is around \SI{39}{^\prime}, and it is decreasing, it will reach a minimum of \SI{27}{^\prime} in the year 2102DC \cite[pag.\ 58]{barbieriLezioniDiAstronomia2003}.}

However, it was noticed as early as 129BC by Hypparcos that the axis of rotation does not remain stationary over the years. 

In an ecliptic coordinate system we find that the ecliptic latitude of stars stays constant, while their ecliptic longitude increases by around \SI{50.3}{\arcsec} per year, which corresponds to a full revolution every \SI{25772}{yr}.
This motion is called the \emph{precession of the equinoxes}
\cite[]{barbieriLezioniDiAstronomia2003}. 

Because of it, the Celestial North Pole appears to rotate clockwise around the Ecliptic North Pole. The appearance of the ecliptic grid in the modern night sky is shown in figure \ref{fig:ecliptic-grid}. 

\begin{figure}[ht]
\centering
\includegraphics[width=\textwidth]{figures/ecliptic_grid_athens_now.png}
\caption{Ecliptic grid on the night sky in 2020. One can see that the Ecliptic North Pole is situated in the constellation of Draco, around \SI{24}{\degree} away from the Celestial North Pole. Image created in Stellarium \cite[]{stellariumcontributorsStellariumAstronomySoftware2020}.}
\label{fig:ecliptic-grid}
\end{figure}

This means that the appearance of the night sky changes over the centuries in a way that is detectable even without high angular precision measurements. 

\subsection{Dating the Farnese Atlas}

The Farnese Atlas is a Roman statue from the second century AD, depicting the Titan Atlas who, after the Titanomachy (the war of the Titans against the Olympians) was punished by Zeus with the task of holding up the sky.

He is depicted as a hunched, muscular man holding up the celestial sphere. On the sphere is a map of constellations, as well as several important circles: the celestial equator, the Tropics of Cancer and Capricorn, the Artic and Antarctic Circles; the equinoctial and solstitial colures \cite[table 4]{schaeferEpochConstellationsFarnese2005}. 

Art historians had long debated the source of the positions of the sculpted constellations, with potential references dating from 1150BC to 150AD; a 2005 study by \textcite[]{schaeferEpochConstellationsFarnese2005} allowed for the matter to be settled: the positions of the stars refer to the year 125BC, \(\pm \SI{55}{yr}\) (at \SI{68}{\percent} C.\ L.).

This matches very well with the stars' positions being taken from the catalog by Hipparchus, who indeed lived in the second century BC. 
The finding is further corroborated by several details about the constellation being only consistent with Hipparchus' work \cite[sec.\ 2]{schaeferEpochConstellationsFarnese2005}.

How was the confidence interval calculated? First, Schaefer accurately measured the position of the recognizable points in the constellations from photographs of the Atlas taken at a known distance.
From these positions we can determine the epoch of the atlas: if the incisions were precisely accurate, we could only look at the position of the vernal equinox (whose right ascension is clearly marked by the presence of the colure on the atlas).

On the Atlas, the constellation of Aries (the Ram) ends precisely on the colure, so we can tell that the sky would have looked roughly as in figure \ref{fig:aries}, which refers to the year 184BC. 

% This allows us to make a first order-of-magnitude estimate. The position of the equinox compared to the 

\begin{figure}[ht]
\centering
\includegraphics[width=\textwidth]{figures/aries_on_colure_184_BC.png}
\caption{The constellation of Aries is shown, as well as the vernal equinox, on an equatorial grid; this is how the sky would have looked in 184BC. The colure is the vertical line passing by the equinox (marked as \(\aries\)), and one can see that it passes by the right side of Aries; this is what appears in the Farnese Atlas \cite[fig.\ 2]{schaeferEpochConstellationsFarnese2005}. Image created in Stellarium \cite[]{stellariumcontributorsStellariumAstronomySoftware2020}.}
\label{fig:aries}
\end{figure}

This, however, is too imprecise. A detailed analysis of the positions of all the stars shows that they are incompatible among themselves, in a way which is compatible with a Gaussian random error on the position of each point. The standard deviation of this Gaussian is roughly \SI{3.5}{\degree} for the points which lie on the aforementioned celestial circles, and \SI{5}{\degree} for ones which do not. 
% These positions are compared among themselves 

This might seem like a rather large error to be able to draw any kind of conclusion.
Indeed, we can directly relate this accuracy to an uncertainty in years using the rate of precession of the equinoxes, \SI{50.3}{\arcsec / yr}: a \SI{3.5}{\degree} uncertainty on the position of a star corresponds to a \SI{250}{yr} uncertainty in the epoch. 
Fortunately, we can measure the position of more than one star, and roughly speaking measuring \(N\) of them reduces the error by a factor \(\sqrt{N}\). 

Schaefer measured \(N = 70\) different points, and employed a \(\chi^2\) minimization procedure in order to find which year corresponded to the least global error in positioning of the constellations, which yielded the final estimate for the epoch of the catalog from which the constellations were drawn. 

The uncertainty of the positions must have been due to both intrinsic errors in the original catalog, and mistakes or imprecision by the sculptor.
Because of this, Schaefer estimates the precision of Hipparchus' atlas to have been \(\lesssim \SI{2}{\degree}\).

\subsection{Ursa Maior in the Iliad}

The following passage of the Iliad appears in the description of the magnificent shield wrought by Hephaestus for Achilles in order for him to go avenge Patroclus, who was slain by Hector.
On the shield, among other things, are depicted the constellations \cite[XVIII, 483--490]{murrayIliad1924}: 
%
\begin{quotation}
``Therein he wrought the earth, therein the heavens therein the sea, and the unwearied sun, and the moon at the full, and therein all the constellations wherewith heaven is crowned—the Pleiades, and the Hyades and the mighty Orion, and the Bear, that men call also the Wain, that circleth ever in her place, and watcheth Orion, and alone hath no part in the baths of Ocean.''
\end{quotation}

\begin{figure}[ht]
\centering
\includegraphics[width=\textwidth]{figures/ursa_maior_today.png}
\caption{The constellation of Ursa Major as seen in 2020 from Athens, Greece. Image created in Stellarium \cite[]{stellariumcontributorsStellariumAstronomySoftware2020}.}
\label{fig:ursa-now}
\end{figure}

Homer (or whoever wrote the Iliad) is stating that the Bear --- Ursa Maior, also known as the Wain (which means ``Cart'') --- alone never ``bathes in the Ocean'', meaning that it never goes below the horizon. 

This statement has a close connection to the latitude and epoch at which the observations leading to the passage were made. 
If one is at a latitude \(\varphi \) in the Northern Hemisphere, then they will see the North Celestial Pole at an altitude of \(\varphi \) above the horizon. In the night the sky will appear to rotate around the NCP, and stars whose declination is \(\delta \geq \SI{90}{\degree} - \varphi \) will appear never to set below the horizon. 

In figure \ref{fig:ursa-now} one can see how the Great Bear will appear in the night sky of 2020 from Athens: the ``Big Dipper'' asterism hardly sets, but it only represents the body of the bear, whose legs definitely go below the horizon. 

\begin{figure}[ht]
\centering
\includegraphics[width=\textwidth]{figures/ursa_names.png}
\caption{Ursa Major: names of all the main stars in the constellation. The stars Tania Australis and Alkaphrah are mentioned in figure \ref{fig:ursa} but they do not appear in this image: they correspond to the ends of the ``legs'' of the bear, on the lower right of \(\psi \) UMa and Alhaud V respectively. Image created in Stellarium \cite[]{stellariumcontributorsStellariumAstronomySoftware2020}.}
\label{fig:ursa-names}
\end{figure}

The Iliad verse, however, is completely reasonable when compared with observation, once again due to the precession of the equinoxes. 
The time when the Iliad was written is not precisely known, however it surely must be prior to the seventh century BC, since there are source referencing it starting from that time.
For our purposes, it is enough to say that it was composed in the period between 1000BC and 500BC.
In figure \ref{fig:ursa-names} a picture of the Ursa constellation is shown with labels for the stars.

Eight stars of the ``lower part'' of the constellation --- four from the Big Dipper asterism, and four from the ``legs'' of the bear --- are chosen, and for each of them we compute the declination at different times from the year 2000BC to the year 2000AD.\footnote{The computations were done with the aid of the Astropy Python package \cite[]{astropycollaborationAstropyProjectBuilding2018}; the specific code which was used can be found at \url{https://github.com/jacopok/notes/blob/master/astronomy/project/ursa/ursa.ipynb}.}
From the declination we compute the quantity \(\varphi _\delta  = \SI{90}{\degree} - \delta \), which corresponds to the minimum latitude from which that star is seen never to set below the horizon. 

\begin{figure}[ht]
\centering
\includegraphics[width=\textwidth]{figures/Ursa.pdf}
\caption{The minimum latitude at which certain selected stars from the constellation of Ursa Major would have appeared not to touch the horizon at varying times, from 2000BC to 2000AD. Stars from the Wain are shown in blue, stars from the ``legs'' of the bear are shown in green. Shown as dotted black lines are the latitudes of Troy, Athens and Crete (in order, from top to bottom).}
\label{fig:ursa}
\end{figure}

The results are shown in figure \ref{fig:ursa}. For reference, we also show the latitudes of relevant geographic locations: Troy, which is the main setting of the Iliad; Athens, and Crete. 
We can see that in the relevant time period, in the North Aegean Sea (at latitudes between that of Athens and that of Crete), the Great Bear would indeed have seemed never able to bathe in the waters of the Ocean.

\end{document}
