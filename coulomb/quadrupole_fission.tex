\documentclass{article}

\usepackage[utf8]{inputenc}

\usepackage{textcomp}
\usepackage[T1]{fontenc}
\usepackage{multirow}
\usepackage{float}
\usepackage[caption = false]{subfig}
\usepackage{longtable}
\usepackage{listings}
\usepackage{mathtools}
\DeclareMathOperator{\tr}{Tr}
\usepackage{commath}
\usepackage{bbold}
\usepackage{xcolor}
\usepackage{physics}
%\usepackage[margin=1.8cm]{geometry}

\usepackage{tikz-cd}
\usepackage{amsmath}
\usepackage{amsfonts}
\usepackage{amssymb}
\usepackage{amsthm}
\usepackage{graphicx}
\usepackage[colorinlistoftodos]{todonotes}
\usepackage[colorlinks=true, allcolors=blue]{hyperref}
\usepackage{siunitx}
\sisetup{separate-uncertainty=true}

\usepackage[sc]{mathpazo}
\linespread{1.05}         % Palladio needs more leading (space between lines)
\usepackage[T1]{fontenc}

\newcommand{\diag}[1]{\text{diag}\qty(#1)}
\newcommand{\const}{\text{const}}
\newcommand{\sign}[1]{\text{sign}\qty(#1)}
\renewcommand{\H}{\mathcal{H}}
\renewcommand{\dim}{\text{dim}}
\newcommand{\C}{\mathbb{C}}
\newcommand{\R}{\mathbb{R}}
\newcommand{\N}{\mathbb{N}}
\newcommand{\Z}{\mathbb{Z}}

\renewcommand{\var}[1]{\text{var} \qty(#1)}
\newcommand{\average}[1]{\langle #1 \rangle}
\newcommand{\defeq}{\ensuremath{\stackrel{\text{def}}{=}}}

\newcommand\mybox[1]{%
  \fbox{\begin{minipage}{0.9\textwidth}#1\end{minipage}}}


\begin{document}

If the Hamiltonian for the quadrupole oscillations is given by

\begin{equation}
    H = \sum_{\mu} \frac{1}{2B} \abs{\pi_{2\mu}}^2 + \frac{C}{2} \abs{\alpha_{2\mu}}^2
\end{equation}

with

\begin{equation}
    C = \frac{4 R_0^2 a_s}{4 \pi r_0^2} - \frac{3Z^2 e^2}{10 \pi R_0}
\end{equation}

we can see what condition is imposed on the nucleus by \(C<0\): if this is the case, the nucleus will gain energy by deforming.

Recall \(R_0 = r_0 A^{1/3}\), then

\begin{subequations}
\begin{align}
  \frac{4 r_0^2 A^{2/3} a_s}{4 \pi r_0^2} &< \frac{3Z^2 e^2}{10 \pi r_0 A^{1/3}}  \\
  \frac{10 a_s r_0}{3 e^2} &< \frac{Z^2}{A}  \\
  2a_s  \qty(\frac{3e^2}{5r_0})^{-1} = \frac{2 a_s}{a_C} &< \frac{Z^2}{A}
\end{align}
\end{subequations}




\begin{flushright}
    Jacopo Tissino, 21 june 2019
\end{flushright}


\end{document}
