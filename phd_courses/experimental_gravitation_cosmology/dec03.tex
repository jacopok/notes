\documentclass[main.tex]{subfiles}
\begin{document}

\subsection{Quantum noise}

\marginpar{Friday\\ 2021-12-3}

The measurement process can be thought of as ``counting photons'', 
although we cannot really determine their exact number,
but this is relevant since it means the measured intensity fluctuates with \(\sqrt{N}\). 

If \(\Delta N / \Delta t\) is constant (we get a fixed number of photons per unit time),
we have a \textbf{Fock state} \(\hat{n} \ket{n} = n \ket{n}\), an eigenvector of the 
photon number operator \(\hat{n} = a ^\dag a\). 

If this were the case, we would have no fluctuations in the readout. 
But, besides the fact that photodetectors do not count photons, 
we do not produce Fock states. 

It is convenient to describe this process in the Heisenberg picture. 
If we use it, the operators we use are all functions of \(t\) while the states are unchanged. 
The state we will use will always just be the vacuum state \(\ket{0}\). 

Classically, the output field will be given by a certain linear transformation of the 
output field: 
%
\begin{align}
\vec{E} _{\text{out}} (\vec{x}, t) = \mathscr{L} \qty[ \vec{E} _{\text{in}}(\vec{x}, t)]
\,,
\end{align}
%
but this will also be what we do in our quantum-mechanical, Heisenberg-picture treatment. 

The difficulty comes from the fact that there are actually many input fields: 
at various places, like the mirrors, we do not have ``nothing'' since there is 
always at least the vacuum. 

The transmissivity \(\tau \) of a mirror is the ratio between the incoming and transmitted
field magnitudes. 
All mirrors have a transmissivity \(\neq 0\), so we always have a coupling to the vacuum 
state on the other side of the mirror. 

If there are losses, we cannot have a unitary process: so, 
there will never be ``one-way'' losses, and a loss will always be associated 
with an extraneous input from the environment. 

The transmissivity and reflectivity satisfy \(\tau^2 + \rho^2 = 1\);
the transmissivity can be interpreted as a loss parameter, so if \(\tau = 0\) 
all the field is reflected back into the system, but we will never have this condition. 

So, if \(A\) is our input, some \(\tau A\) is leaving our system; 
because of what we were saying above the field going back will not be just \(\rho A\) but
instead \(\rho A + \tau V\), where \(V\) denotes the vacuum state on the other side of the port. 

A very important field is the one coming from the photodiode, which 
will often look like a thermal state. 
However, its temperature will be very low compared to the laser light: 
that is near-infrared, 100 to \SI{1000}{THz}, corresponding to \(\gtrsim \SI{700}{K}\)
of thermal temperature. 
So, we can take it to be the vacuum. 

Only about \SI{.5}{\percent} of the laser light at the beamsplitter reaches the 
photodiode. 
Therefore, changing the field at the laser will have little effect on the diode; 
on the other hand, \SI{99.5}{\percent} of the field from the photodiode will 
come back to the laser. 

Only in the eighties someone properly described an interferometer in QFT. 
Before, people just used Poissonian statistics to calculate the quantum noise. 

The first important insight is that we need to act on the field at the photodiode. 

Let us neglect the vector character of the field for simplicity; 
the fields are all linearly polarized, and the quantum mechanics for the 
electric and magnetic fields are the same.  
The field will be 
%
\begin{align}
\hat{E}(x, t) = \sqrt{\frac{2 \pi \hbar}{\mathcal{A} c}} \int_{- \infty }^{\infty } \frac{ \dd{\omega }}{2 \pi } \sqrt{\omega }
\qty( \hat{a} (\omega ) e^{- i \omega (t - \vec{x} \cdot \vec{\epsilon} /c) } + \hat{a} ^\dag e^{i \omega (t - \vec{x} \cdot \vec{\epsilon} /c)})
\,,
\end{align}
%
where we are fixing the propagation direction by fixing \(k = \omega / c\); 
the creation and annihilation operators will satisfy 
%
\begin{align}
\qty[ \hat{a}(\omega ),  \hat{a} ^\dag (\omega ')] = 2 \pi \delta (\omega - \omega ')
\,.
\end{align}

If we apply the annihilation operator to the vacuum we get \(\hat{a} \ket{0} = 0 \ket{0}\); 
since the photon number operator is \(\hat{n} = \hat{a} ^\dag \hat{a}\)  we also have 
\(\hat{n}  \ket{0} = 0 \ket{0}\). 

A \textbf{coherent state} is an eigenstate of the annihilation operator: \(\hat{a} \ket{\alpha } = \alpha \ket{\alpha }\), 
where the eigenvalue \(\alpha \) is a generic complex number. 
This is ``close to classical'', it describes the output of a laser quite well, 
and it has Poissonian statistics. 

What is the distribution of the number of photons in a coherent state? 
%
\begin{align}
p(n) = \abs{\braket{n}{\alpha }}^2 = e^{- \abs{\alpha }^2} \frac{\abs{\alpha }^{2n}}{n!}
\,.
\end{align}

This is a Poissonian with average photon number \(\expval{n} = \abs{\alpha }^2\). 

The energy in the field is given by \(E = \hbar \omega_0 n\), so this value also corresponds
to the mean energy in the field. 

In the Heisenberg picture, one finds that the laser is mapping \(\hat{a} \to \hat{a} + \alpha\). 
In a sense, all ``classical'' things are fixed complex numbers as opposed to operators. 

The full operator is \(\hat{a} + h + \alpha \), where \(\hat{a}\) describes the quantum fluctuations,
\(h\) describes the GW signal, while \(\alpha \)describes the laser. 

We have not yet defined what is our canonical pair of observables. 
We can write 
%
\begin{align}
\hat{E} (x, t) = 
E_1 (x, t) \cos (\omega_{0} t - kx)
+
E_2 (x, t) \sin (\omega_{0} t - kx)
\,.
\end{align}

Why are we picking a single frequency \(\omega_0 \), while our field has many? 
It is convenient since the laser frequency (\(\omega_0 \)) is the main one, but there are also things fluctuating at all other frequencies. 
We expect quantum fluctuations to have a white spectrum, so for those it is the same; however the sidebands from the GW signal will be close to \(\omega_0 \). 

In the photocurrent we measure, \(I _{\text{ph}} (t) = \abs{E(x _{\text{photodiode}}, t)}^2\), we would have product terms between the laser frequency and the GW frequency. 

The Fourier transform of the photocurrent allows us to compute a Power Spectral Density \(S (I _{\text{ph}}, \omega )\). 

The Fourier transform looks like 
%
\begin{align}
\widetilde{I} _{\text{ph}} (\omega ) \sim \underbrace{\abs{\alpha }^2}_{\omega_0 , \omega_0 } + \underbrace{\Re(\alpha \hat{a})}_{\omega_0 , \omega  } +  \underbrace{\Re (\alpha h)}_{\omega_0 \pm \Omega , \omega_0  } + \underbrace{\Re (h \hat{a})}_{\omega_0 \pm \Omega , \omega}
\,.
\end{align}

The term which does not fluctuate, \(\abs{\alpha }^2\), is often called a DC component or DC offset: since we bandpass the photocurrent signal, any low-frequency component like that will vanish. 

The noise term will oscillate at \(\omega_0 - \omega = \Omega \); in the end the photocurrent we will actually have left after the bandpassing we get 
%
\begin{align}
\hat{I} _{\text{ph}}(\omega ) \sim h(\Omega ) + \hat{a} (\Omega )
\,.
\end{align}

What we are basically doing is extracting out the fast-oscillating term, and we can focus on the slow, audio-band oscillations we care about. 

The fluctuations in the photocurrent will contain the square moduli of the \emph{quadratures} \(E_1 \) and \(E_2 \), which form a Heisenberg pair. 

How do we actually manipulate the vacuum field at the photodiode? 
A squeezer is introduced, and it passes a polarizing beamsplitter, which is always either fully transmissive or fully reflective depending on the polarization of the light. 

The unpolarized vacuum passes the polarizer in some fraction, but when it comes back it is fully reflected. 

The squeezer emits little power, and some of it is in green light as opposed to infrared! 

The power in the Fabry-Perot cavity is very high (\(\sim \SI{200}{kW}\)), but a small amount of squeezing power is enough to improve the sensitivity significantly. 

\end{document}
