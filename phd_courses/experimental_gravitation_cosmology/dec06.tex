\documentclass[main.tex]{subfiles}
\begin{document}

\marginpar{Tuesday\\ 2021-12-7 \\ (missed lesson) \\ notes by Silvia}

As mentioned last time, the measured intensity is proportional to the \emph{low-pass} filtered square of the electric field: \(\hat{I} \propto \abs{E}^2_{LP}\). 

The two quadrature fields \(E_1 \) and \(E_2 \) can be written in terms of the \(\hat{a}_{1, 2}\) creation and annihilation operators: 
%
\begin{align}
\hat{E}_{1, 2} (\hat{r}, t) = \sqrt{\frac{4 \pi \hbar}{\mathcal{A}c}} 
\int \frac{ \dd{\Omega }}{2 \pi } \left[
    \hat{a}_{1, 2} (\Omega ) e^{-i \Omega + i \vec{k} \cdot \vec{r}}
    +
    \hat{a} ^\dag_{1, 2} (\Omega ) e^{+i \Omega - i \vec{k} \cdot \vec{r}}
\right]
\,,
\end{align}
%
where 
%
\begin{align}
\hat{a}_{1} (\Omega) &=
\underbrace{\sqrt{\frac{\omega_0 + \Omega}{2 \omega_0 }}}_{\approx 1/2 \text{ when } \Omega \ll \omega_0 } \hat{a}(\omega_0 + \Omega  ) +
\sqrt{\frac{\omega_0 - \Omega}{2 \omega_0 }} \hat{a}(\omega_0 + \Omega  ) \\
\hat{a}_{2} (\Omega) &=
-i \underbrace{\sqrt{\frac{\omega_0 + \Omega}{2 \omega_0 }}}_{\approx 1/2 \text{ when } \Omega \ll \omega_0 } \hat{a}(\omega_0 + \Omega  ) 
+ i\sqrt{\frac{\omega_0 - \Omega}{2 \omega_0 }} \hat{a}(\omega_0 + \Omega  )
\,.
\end{align}

When we compute the modulus \(\abs{E}^2\) we get a \(\sim \abs{E_1 }^2 \cos^2(\omega_0 t)\) term, a \(\abs{E_2 }^2 \sin^2(\omega_0 t)\) term and a \(2 E_1 E_2 \sin(2 \omega_0 t)\) term.
The last of these averages to zero over a few periods the laser pulsation \(2 \pi / \omega_0 \) (so, a very short time) while the square cosines and sines both average to \(1/2\). 
This means we get 
%
\begin{align}
\abs{E}^2 = \frac{1}{2} \left( \abs{E_1 }^2 + \abs{E_2 }^2 \right)
\,.
\end{align}

A field quadrature can be written in terms of a baseline \(A\) plus some oscillations with period \(\Omega \): 
%
\begin{align}
\hat{E}_1 = A + \sqrt{\frac{4 \pi \hbar}{\mathcal{A}c}} \int \frac{ \dd{\Omega }}{2 \pi }
 \left[ 
     \underbrace{\hat{a}_1 e^{-i \Omega t + i \vec{k} \cdot \vec{r}}}_{\hat{c}_1} + \text{h. c. }
 \right]
\,.
\end{align}

This quantum noise is constant in \(\Omega \): 
%
\begin{align}
S(\hat{a}_1 , \Omega ) = \lim_{T \to \infty } \frac{\expval{\hat{a}_1 \hat{a}_1 ^\dag}}{T} = 1 (?)
\,.
\end{align}

The Heisenberg principle for these is \(S(a_1 ) S(a_2 ) \geq  1\).

When we do homodyne detection we are measuring a product in the form \(I(\Omega ) \propto E _{\text{LO}} \cdot E _{\text{signal}}\); we can determine amplitude and phase of the local oscillator field, then we can squeeze the vacuum in the direction parallel to it and allow it to expand in the other. 

Typically, the wavelength corresponding to the pump laser, at \(2 \omega_0 \), is green light (\(\lambda = \SI{532}{nm}\)). 
This is then passed to an Optical Parametric Amplification crystal which is basically an SPDC crystal plus amplification. 

The squeezing procedure can be schematically represented as 
%
\begin{align}
\left[\begin{array}{c}
\hat{s}_1  \\ 
\hat{s}_2 
\end{array}\right]
= \underbrace{\left[\begin{array}{cc}
\cos \varphi  & - \sin \varphi  \\ 
\sin \varphi  & \cos \varphi 
\end{array}\right]\left[\begin{array}{cc}
e^{- \sigma } & 0 \\ 
0 & e^{\sigma }
\end{array}\right]}_{\Sigma (\sigma , \varphi )}
\left[\begin{array}{c}
\hat{a}_1  \\ 
\hat{a}_2 
\end{array}\right]
\,.
\end{align}

The transformation matrix \(\Sigma \) determines the power spectral density: 
%
\begin{align}
S(s_1 , s_2 , \omega ) = \Re ( \Sigma  \Sigma ^\dag) \underbrace{S(a_1 , a_2, \Omega )}_{\equiv \mathbb{1}_2}
\,,
\end{align}
%
where these power spectral densities are to be interpreted as covariance matrices. 


\end{document}