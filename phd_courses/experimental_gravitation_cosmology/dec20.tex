\documentclass[main.tex]{subfiles}
\begin{document}

\marginpar{Monday\\ 2021-12-20}

At high frequency we also lose sensitivity because our 
detector responds less well to GWs, as well as shot noise. 

In the simplest version, squeezing reduces shot noise but increases
radiation pressure noise. 

RP noise is at low frequency mostly. 
The resonant frequency of the suspensions for our detectors is typically 
at a few \SI{}{Hz}; therefore for practical purposes we are above it and we can treat
the mirror as a free mass. 

The radiation pressure noise can be described by relating the fluctuation in 
applied power to the fluctuation in position:
%
\begin{align}
-m \omega^2 \delta x &= 2 \frac{ \delta P}{c}  \\
\delta x &= - \frac{2 \delta P}{mc \omega^2} = -2 \frac{\sqrt{E \omega \delta P}}{mc \omega^2} \hat{a}_1 (\omega )
\,.
\end{align}

We suppose that the amplitude quadrature contains the power, as \(E_1 = \sqrt{P} + \dots\),
while the phase quadrature \(E_2 \) contains the GW signature. 

The expression for the fluctuation is due to the fact that 
the quadrature also contains an integral over \(\hat{a}_1\). 

The phase of the reflected signal is modulated by the change in position of the mirror: 
%
\begin{align}
\frac{ \delta \phi }{2 \pi } &= 2 \frac{ \delta x}{\lambda } = 2\frac{ \delta x \omega_0 }{c }  \\
\delta \phi &= \frac{4 \pi \omega_0 }{c} \delta x
\,,
\end{align}
%
where the factor 2 is due to the fact that the new path is added both forward and back. 

The quadrature reads 
%
\begin{align}
\sqrt{P}\cos(\omega_0 t + \delta \phi) &= \sqrt{P} \left( \cos( \omega_0 t) \underbrace{\cos(\delta \phi )}_{\sim 1} - \sin( \omega_0 t) \delta \phi \right)
\,.
\end{align}

Therefore, the modulation  is roughly \(E_1 \approx \sqrt{P} \delta \phi \). 

The incoming field will be described by \(\vec{a} = (a_1, a_2 )^{\top}\), while the 
outgoing one will be 
%
\begin{align}
\vec{b} = \left[\begin{array}{c}
b_1  \\ 
b_2 
\end{array}\right]
= \underbrace{\left[\begin{array}{cc}
1 & 0 \\ 
-k & 0
\end{array}\right] }_{T}
\vec{a}
\,,
\end{align}
%
where \(k = 4 \omega_0 P / mc^2 \omega^2\). 
The spectral density of \(b\) will read 
%
\begin{align}
S(\vec{b}) = \Re (T T ^\dag) \underbrace{S(\vec{a})}_{\mathbb{1}}
\,.
\end{align}

The real part of \(T T ^\dag\) reads 
%
\begin{align}
\Re (T T ^\dag) = \left[\begin{array}{cc}
1 & -k \\ 
-k & 1+k^2
\end{array}\right]
\,.
\end{align}

The off-diagonal terms mean that the noise in the amplitude and phase quadratures
is correlated. 

We can plot \(1 + k^2\) against \(f\) with the expression we have for \(k\). 
It is roughly \(1 / f^2\) and then flat.

How can we show that this is still a minimum uncertainty state? We need to
see that the determinant of the transfer matrix is 1, which holds. 

There is a quadrature which has reduced quantum noise with respect to the vacuum, 
at the expense of another which has larger noise. 

People tried to do squeezing in this way; the main thing to do is to use 
tiny mirrors so that \(k\) becomes very large. 

This effect will rotate the ellipse, even if we already send in squeezed light!

This is the reason why the RP noise worsens the noise curve when we send in squeezed light! 

The solution to this problem is to pre-rotate the ellipse so that this effect is 
compensated. 

At the quantum level, this is analogous to thinking of regularizing photon arrival times; 
the statistics of this squeezing operator should be the same as what we get with an SPDC. 
Of course, that's not really the physics of the measurement we make. 

Some really big names were working on quantum technologies GW detection. 
Braginsky, Khalili, Kip Thorne, Alessandra Buonanno. 

The things we did today appeared in a paper ``Ponderomotive effects in electromagnetic radiation''. 

A detuned power recycling cavity allows for moving the resonance 
frequency of the Fabry-Perot cavity into the observation band, which
means we have very low noise there. 

Squeezed noise is not stationary in the photocurrent! 

When we squeeze there is also an addition of some extra noise,
we do not stay minimum-uncertainty. 

Squeezing and changing the power are equivalent in the respect of overcoming the Standard Quantum Limit! 

Frequency-dependent squeezing is achieved through the ponderomotive 
effect: an input cavity rotates the ellipse by \SI{90}{\degree} between high and low frequency.

One can do an output filter, alternatively: doing the same thing 
after the round-trip in the arms. 
This turns out to be much better, but it is more sensitive to losses in the filtering cavity. 

\todo[inline]{What does the configuration look like there?}

So, the idea is that the mirror's fluctuation will rotate the noise among the quadratures, so we do that in the opposite direction to the squeezed light (?). 

Mirror coatings can create issues! Even if a requirement on the amplitude is satisfied, if the pattern is regular there can be problems: it acts like a diffraction grid. 

\todo[inline]{Where does the \(\Delta \sim \SI{}{MHz}\) frequency shift come from?}

We can do measurements of entangled photons coming from down-conversion to reduce noise. 

The OPA is where we get a pump with an offset. 
This allows us to make an EPR filter with frequency-dependent filtering! 

The \SI{}{MHz} photons experience no frequency offset: there is no radiation pressure there! 



\end{document}