\documentclass[main.tex]{subfiles}
\begin{document}

\marginpar{Tuesday\\ 2021-12-21}

The radiation pressure term is written as 
%
\begin{align}
\text{RP} = \frac{1}{c} \left(
    P + 2 \sqrt{P} \sqrt{\hbar \omega_0 } \hat{a}_1 (\Omega )
\right)
\,,
\end{align}
%
with quadratic, \(\order{\hat{a}^2}\) corrections. 

This corresponds to a displacement \(\text{RP} / m \Omega^2 = \delta x (\Omega) \). 

This displacement is connected to a phase fluctuation 
%
\begin{align}
\delta \phi (\Omega ) = 2 \frac{ \delta x (\Omega ) \omega_0 }{c}
\,.
\end{align}

The thing is then that the modulation can be written as 
%
\begin{align}
E_1 &= \sqrt{P} \cos(\omega_0 t + \delta \phi (\Omega ))  \\
&= \sqrt{P} \cos(\omega_0 t) - \delta \phi (\Omega ) \sqrt{O} \sin(\omega_0 t)
\,,
\end{align}
%
while the phase quadrature reads 
%
\begin{align}
E_2 = \sqrt{P} \delta \phi (\Omega ) \propto - \frac{\sqrt{P}}{\Omega^2}
\,.
\end{align}

This is the reason why the \SI{}{MHz} sidebands 
are not affected by radiation pressure noise. 

\subsection{Wiener filtering}

If we have an estimated signal \(\hat{y}  = F(x)\) and an observed signal \(y\), 
we want to minimize \(\expval{(y - \hat{y})^2} = e^2\).

We choose this by determining a maximum, with \(\pdv*{e^2}{F} = 0\) and \(\pdv*{e^2}{F} < 0\). 
The optimal choice is then \(F = \expval{xy} / \expval{x^2}\).

In the frequency domain, a filter looks like \(\hat{y}(\Omega ) = F(\Omega ) x(\Omega )\).

There are ways to implement this and keep track of continuous variations of the environment (Kanman filtering?).

\subsection{Environmental noise}

Part of the site selection process is looking at seismic noise. 
We need to have noise globally below a certain RMS value, 
even microseismic peaks at a few Hz create issues since they can throw 
the interferometer out of alignment, 
even though we do not detect signals at those frequencies. 

A big issue is also Newtonian gravitational noise, 
gravitational coupling of the environment to the test mass, which scales like \(f^{-2}\). 

Temperature gradients of sufficiently small scales in the atmosphere can also create issues. 

The acoustic noise in the cave also creates issues.
One can subtract the measurement of noise actively, but this is 
very expensive for ET since we would need many new boreholes! 

Measuring acoustic fields in the atmosphere is hard: eddies can form around the 
microphone if the wind is too high. 
Lasers are quite good for this purpose.

There are magnetic disturbances correlating all the way around the world. 
These are Schumann resonances. 

\subsection{Thermal noise}

Material science is very poorly understood.
We are not able to compute things like Young's modulus from first principles. 

The fundamental theorem describing this is the fluctuation-dissipation theorem: 
%
\begin{align}
S_x(\Omega ) = \frac{ 8 \pi k_B T}{\Omega^2} \frac{W _{\text{diss}}}{F_p^2}
\,.
\end{align}

This describes the thermal noise spectrum due to mechanically dissipated power. 
The thermal noise we measure is average over the beam, so low-wavelength 
thermal noise are not really a problem. 

Some high-order Laguerre-Gauss modes have been proposed as a way to moderate this issue.

Heat links can be a shortcut for vibrations! 
The way coating of the mirror is done is relevant. 
Tunneling is a big source of dissipation in materials. 

The study of internal friction of materials is often studied by 
perturbing them and looking at the ringdown. 

\end{document}