\documentclass[main.tex]{subfiles}
\begin{document}

\section{Multimessenger Astrophysics}

\marginpar{Wednesday\\ 2021-12-22}

This course is more on the astrophysics side. 

Marica worked on radio, AGNs, and then moved to gravitational waves. 

Overview: 
\begin{enumerate}
    \item main GW discoveries;
    \item astrophysical sources of GWs;
    \item modelling of EM emission of GW transients;
    \item observations and data analysis;
    \item astrophysics of compact objects;
    \item future perspectives.
\end{enumerate}

Neutrinos will mostly be in a short course, here we discuss GW and EM emission. 

The LIGO-Virgo-KAGRA collaboration is thinking to start observations in December 2022. 

\subsection{Gravitational Wave discoveries}

Having a network is important to do localization! 

History of GW150914, GW170817. 

A self-gravitating system has a natural frequency 
%
\begin{align}
f _{\text{dyn}} = \sqrt{\frac{G \overline{\rho}}{4 \pi }}
\,,
\end{align}
%
where \(\overline{\rho}\) is the mean density of the object. 

This yields \SI{1.9}{kHz} for \(1.4M_{\odot}\) within \SI{10}{km} (BNS), 
\SI{1}{kHz} for \(10M_{\odot}\) within \SI{30}{km} (BBH), 
while for a SMBH we get about \SI{4}{mHz}. 

The GW amplitude scales like 
%
\begin{align}
h \sim 2 \left(\frac{GM}{Rc^2}\right)
\left(\frac{GM}{rc^2}\right)
\,,
\end{align}
%
where \(R\) is the system scale, \(r\) is the distance to the source. 

We measure the \emph{amplitude} of GWs! not the flux. 

We can make a plot of lines in the mass-radius plane. 
Observation bands with fixed-frequency, lifetimes left for binary systems\dots

The energy emitted in a BNS merger is around \(\num{e-2} M_{\odot} c^2\). 

The waveforms for supernovae and for NS instabilities are not known. 
There are no simulations that can make a supernova explode completely. 

GW150914 was detected with CWB! 

Unmodelled searches just search for correlated excess power. 

\todo[inline]{Is there a systematic review of GW modelling degeneracies?}

The chirp mass has an analytic dependence on the frequency and frequency derivative, 
we can do this calculation for 150914 to get the chirp mass, which translates to 
an upper bound on the total mass. This yields a Schwarzschild radius of \SI{210}{km}; 
but the orbital frequency reached \SI{75}{Hz} which corresponds, with Kepler's law, 
to \SI{350}{km}! 
These objects must have been compact. 

Compactness \(C\) is the Newtonian orbital separation divided by the sum of their smallest possible (Schwarzschild) radii. 

In the non-spinning, circular orbit case we get \(C \approx 1.7\). 

For NSs we get a number between 2 and 5 (accounting for the uncertainty on the NS mass). 
With a line of reasoning accounting for the mass ratio, we can also exclude the BHNS case. 



\end{document}
