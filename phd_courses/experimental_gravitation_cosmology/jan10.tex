\documentclass[main.tex]{subfiles}
\begin{document}

\marginpar{Monday\\ 2022-1-10}

As we discussed, from \(f\) and \(\dot{f}\) in a GW signal 
we can extract the chirp mass. 

The maximum orbital frequency of GW150914 was \SI{75}{Hz}; 
using Kepler's law we get a separation of \SI{350}{km}. 

The compactness of the system is defined as the ratio between the 
Newtonian orbital separation divided by the sum of the possible radii (Schwarzschild radii).

For unequal masses we have a compactness of the order of 
%
\begin{align}
C \approx \frac{3.0 q^{2/5}}{(1 + q)^{4/5}}
\,.
\end{align}

We can plot \(C\) as a function of mass ratio \(q\) and eccentricity, then using
the \(C = 1\) line as a boundary we get a limit beyond which the object would be inside 
their respective Schwarzschild radii. 

\todo[inline]{Plot of \(\sqrt{S(f)}\) against \(2 \abs{h(f)} / \sqrt{f}\)? from Abbott 2016?}

We can measure the source-frame masses, chirp mass, effective aligned spin,
radiated energy, distance and redshift, but the redshift is not estimated directly. 

What people thought from EM observations was that stellar-mass black holes would be below 
20 solar masses. 

The first evidence for BH existence was in the 1970s, when the mass of the X-ray binary 
Cygnus X-1 exceeded the maximum mass of a neutron star. 
We see about 22 of these systems we detect in X-rays; we can look at their orbital motion.

The formula is 
%
\begin{align}
\frac{PK^3}{2 \pi G} = \frac{M \sin^3 \iota }{(1 + q)^2}
\,,
\end{align}
%
where \(P\) is the orbital period, which we measure directly, 
\(K\) is the radial velocity amplitude of the companion, 
\(\iota \) is the binary inclination and \(q\) is the mass ratio. 

Doing these estimates we get 5 to 20 solar masses. 

Between 8 and 25 solar masses for the progenitor we have NS formation;
above 40 solar masses we can have direct collapse to a BH, ``failed supernovae''. 

Roughly, NS masses are constrained to lie somewhere between 1 and 3 solar masses. 
The lower limit is the Chandrasekhar mass (for NS formation), 
the upper limit comes from the Oppenheimer-Volkoff equation 
(above a certain mass pressure cannot counteract gravity).

Theoretically, knowing the mass distribution of BH masses is hard, 
because of the uncertainties in the mass loss with stellar winds and supernova explosions. 

A plot of remnant mass against ZAMS mass of the progenitor; 
varying the metallicity makes a huge difference. 
For \(Z = Z_{\odot} = 0.02\) we always get remnants below \(20 M_{\odot}\); 
the other extreme \(Z = Z_{\odot}  / 100\) yields prompt collapse, 
while in an intermediate region \(Z \sim Z_{\odot} / 10\) we get BHs 
compatible with the observed population by LVC. 

Metallicity affects stellar winds due to line-driven winds. 

The mass loss scales roughly like \(\dot{M} \propto Z^{\alpha }\), with \(\alpha \) around 0.5 to 0.9.

Heavy BH formation is complicated, but the main idea is to have low metallicity (\(< Z_{\odot} / 2\)), meaning that stellar winds are quenched leading to larger pre-collapse mass, and then a good probability of direct collapse. 

What are the mechanisms for SN formation?
One is a \emph{pair-production supernova}: if the Helium core is larger than \(64M_{\odot}\), there is efficient production of \SI{1}{MeV} or harder photons, which produce pairs leading to a decrease of photon pressure. 

This leads to a collapse during oxygen burning, a runaway thermonuclear reaction without an iron core. No remnant is left. 

A lighter version of this is a pulsation-pair instability supernova, in which the star oscillates due to the same mechanism. 

This leads to a \emph{mass gap}, between 60 to \(120M_{\odot}\). 

In O3 we saw GW190521: a component had \(85M_{\odot}\), in the mass gap! 

Massive black holes can form in the galaxy field or, more probably, in a globular cluster, young star cluster or near an AGN. 

Dynamical interactions favor the formation of lower mass BHs! 

Globular clusters are older, more compact, typically observed in the bulk of the galaxy, and most stars are of the same age, in elliptical galaxies. 
Young star clusters are less graitationally bound, found in both spiral and elliptical galaxies, and they typically evolve. 
Nuclear star clusters are the ones which are closer to the center of the galaxy. 

BBHs are typically ejected from the cluster at the beginning. 

The BBHs we are looking at could come from isolated binaries, or from dynamical interactions in a dense environment: which is the main formation channel?

It is very common to have stars in a binary system. 
In order to form an original binary we need two massive stars; about \SI{70}{\percent} of massive stars have a companion. 

Many evolutionary processes can destroy our binary! 

One very hard thing to model is the common envelope phase. 
This happens when the separation is about 1000 to 10000 solar radii; the mass transfer becomes unstable forming the common envelope. 

What about dynamical binaries? 
This becomes important when we have more than 1000 stars per cubic parsec. 

A star can do a flyby of a binary, acquiring kinetic energy and thereby hardening the binary. 
Further, we can have exchanges!
More than \SI{90}{\percent} of binaries in young star clusters are formed by exchanges. 

How can we characterize the exchange-origin population? 
These will have 
\begin{enumerate}
    \item very massive BHs;
    \item high eccentricity;
    \item misaligned BH spins.
\end{enumerate}

The first detections of BBHs had \(\chi _{\text{eff}}\) compatible with zero, which could both mean small aligned spins or large misaligned spins. 

Now we can do population studies!

\end{document}
