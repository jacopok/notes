\documentclass[main.tex]{subfiles}
\begin{document}

\marginpar{Thursday\\ 2022-1-13}

Last time we discussed how to make an effective search for an EM counterpart. 

The search for the most likely galaxies within the 
uncertainty in volume was used already in GW170817.

The ways to rank galaxies are various: one is the blue luminosity; 
another is the K-band luminosity. 
Blue luminosity is a proxy for young stars and star formation, 
K-band luminosity is a proxy for stellar mass. 

How do we rank galaxies? 
One way is through theoretical simulations: 
combining population synthesis models with cosmological simulations. 

Simulations show that, roughly, the trend for BNS is of a strong correlation between total mass and number of binary systems; 
there is a dependence on metallicity, but it is weak. 

For BHNS, lower-mass galaxies seem to have a higher number of systems 
per unit mass.

One of the typical counterparts is a short GRB, 
and there is also a dependence on the mass of the rate of short GRBs\dots
this approach might be slightly flawed. 

How do we proceed? 

We take the 3D sky localization map, compute the localization probability for each pixel, and combine it with a luminosity-derived probability estimate in the form 
%
\begin{align}
\mathbb{P} _{\text{lum}} = \frac{L_{K, b}}{\sum L_{K, b}}
\,.
\end{align}

We can also optimize the observation strategy by 
taking into account the estimated masses, spins, equations of state! 

We can even try to estimate the best \emph{time} at which to observe, 
since different lightcurves peak at different times. 

We can evaluate a posterior detectability distribution: 
what are the fraction of fluxes based on the posterior distribution 
which exceed the limit flux for this telescope? 

(See Salafia et al 2017)

The observation strategy also includes the amount of 
exposure time to use --- short exposures at first, longer exposures
later. 

\subsection{Detectable sources}

We have already seen the three combinations of NSs and BHs; 
the other interesting objects in this GW band
are CCSNe and SN instabilities. 

What is a short Gamma Ray Burst?
It is a very energetic, beamed emission, due to an ultra-relativistic jet. 

For a kilonova, we also expect to see sub-relativistic dynamical effects, connected to dynamical outflows, as well as disk-wind outflow and spin-down luminosity. 

What is the EM emission of a supernova? 
The short breakout is typically seen in X-ray and UV. 
Then, there is emission in all bands. 

What are the EM counterparts to isolated NS instabilities? 
They could be ``soft gamma-ray repeaters''\dots

Magnetar flares were also associated with the newly-discovered fast radio bursts. 

After a BNS merger, an accretion disk is formed! 

There are full GR simulation by the Jena group which treat well the microphysics. 
There is a large fraction of unbound mass; the geometry of this mass heavily 
determines the EM emission. 

For a NSBH binary, the interesting quantity is the tidal disruption radius to the 
ISCO of the BH; if this is less than 1 the NS is swallowed by the BH; 
if it is larger than 1 the NS is tidally disrupted, and we have long spiral arms.

Before the merger, the BH is fully defined by its mass and spin, which determine the ISCO, going from 9 with maximal antialigned spin to 1 with maximal aligned spin to the orbital angular momentum. 

The spin of the NS is expected to be small; what matters is the BH spin. 

Tidal disruption is defined as happening when the tidal force of the BH is 
stronger than the self-gravity of the NS: this yields 
%
\begin{align}
d _{\text{tidal}} \sim R _{\text{NS}} \left(\frac{3 M _{\text{BH}}}{M _{\text{NS}}}\right)^{1/3}
\,.
\end{align}

We can make a bivariate plot with \(\chi _{\text{BH}}\) and \(q\), and color it according to 
the fraction of ejected material. 

The mass-radius relation is uniquely given by the underlying EoS. 

Stiff EoS means the NS is \emph{less} compact. 
These are harder to compress, and easier to disrupt. 

Using spin, mass from the GW signal and amount of ejecta from the EM counterpart, 
we can constrain the equation of state! 

For the EM emission, it is also important to know: what is the remnant? 
If the mass is relatively small, we can have unstable 
supermassive (lifetime of an hour) or hypermassive (lifetime of a second) neutron stars. 

\subsection{Electromagnetic emission}

Some basic relations: 
the total luminosity \(L\) is given in terms of the monochromatic luminosity \(L(\nu )\) as 
%
\begin{align}
L = \int L (\nu ) \dd{\nu }
\,.
\end{align}

The most relevant processes are blackbody, bremsstrahlung, synchrotron, and inverse Compton. 

We can define a brightness temperature: the temperature of a blackbody which would 
emit that amount of radiation at that frequency.

\todo[inline]{this never exceeds the kinetic temperature of the source}

Synchrotron emission yields a powerlaw.

\todo[inline]{Particle energy is \(E = m c^2 \gamma^2\)? }

The energy radiated scales with \(\gamma^2 B\). 
The superposition of the synchrotron spectra from electrons at various energies 
gives the slope of the overall spectrum. 

Synchrotron self-absorption also happens at low frequencies: 
at low frequency the electron medium is optically thick, since the brightness temperature 
is close to the electron temperature. 

\subsubsection{The history of gamma ray bursts}

To know the intrinsic luminosity of the source of a GRB, we should 
know what is the width of the cone. 

The emission of GRBs really starts in the X-ray band, near \SI{10}{keV}. 

They were discovered serendipitously in the 1960s, since the US people 
were looking for nuclear tests of the Russians and found these flashes 
happening every 2 or 3 days. 

The satellite BATSE was the first to be sent to study them; 
and it showed an isotropic distribution. 

This was an indication that these objects were extragalactic, cosmological. 
It could also correspond to nearby stars, which are also nearly isotropic. 

A new satellite was BeppoSAX; it showed an afterglow emission in the X-rays, 
so they were able to also identify the emission galaxy and associate it to a cosmological redshift. 

The first observation of the GRB optical afterglow was done in Campo Imperatore! 
But, the first paper which was published was by Dutch people. 

SWIFT is a current telescope, and it is very important for X-ray observation. 
We also have XRT with a sky localization of arcseconds. 

An ultraviolet optical telescope can also observe the optical afterglow. 



\end{document}