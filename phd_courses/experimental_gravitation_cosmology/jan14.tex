\documentclass[main.tex]{subfiles}
\begin{document}

\marginpar{Friday\\ 2022-1-14}

The particle energy distribution from synchrotron emission has a spectral index \(\delta \), 
while the index of the spectral distribution is \(\alpha = ( \delta - 1 ) / 2\). 
This is due to the superposition of many emission spectra. 

The frequency of the emitted energy scales like \(\nu \propto \gamma^2 B\). 

The cutoff at low frequencies, \(S_\nu \sim \nu^{5/2}\), happens when the electron 
spectral temperature gets close to the electron kinetic temperature. 
There, the plasma is optically thick, while
in the high-energy regime the plasma is optically thin and we get 
a powerlaw \(S_\nu \sim \nu^{-\alpha }\). 

Roughly, we observe a GRB every couple of days, and as we discussed earlier 
they were found to be cosmological. 

If we plot the distribution of GRB durations, it is bimodal;
there is a peak at just less than a second, and a peak of 10 to \SI{100}{s}. 

The distribution of the hardness ratio (hard photons vs soft photons)
is also bimodal: short GRBs also have a higher hardness ratio. 

Long, soft GRBs are connected to CCSNe. 
Short, hard GRBs are not connected to SNe, and they typically have a larger 
distance from their host galaxy center (5 to \SI{10}{kpc}, but even more). 


They seemed to be connected with the timescale of accretion after a binary merger with a NS. 
This is all consistent with what we expect for a binary system of NSs.

The fact that these are further from the galaxy center is consistent with 
the possibility that they drifted away from the star-forming 
galactic center due to the SN kick. 

In optical, the flux seems to decrease over time according to a powerlaw. 
For lGRBs, we then have a bump at the end, corresponding to a SN. 
\todo[inline]{Why is the bump there in lGRBs? is the SN not the start?}

It is interesting to study the galaxies the GRBs are coming from. 

The distribution of sGRBs seems to match up with field galaxies, 
while lGRBs typically are fainter, more star-forming and with lower metallicity
than field galaxies. 

Certain GRBs seem to be hostless: they can go as far as \SI{1}{Mpc}. 
We can have an independent estimation of distance with the evaluation
of the absorption. 

Only \SI{20}{\percent} of sGRBs are in early type galaxies. 
Many more are in late-type, spiral galaxies. 

Elliptical galaxies are early-type. 
They have older stars, maybe they formed with the mergers of spiral galaxies. 

When we go distant, we use color: spiral galaxies are bluer. 

We observe sGRBs mostly at low redshift, this is partly expected 
because it takes time to make a compact binary, 
and partly because of observational bias. 

\subsubsection{GRB formation}

No simulation is currently able to produce jets. 
The current best scenario is to produce a jet thanks to a high magnetic field,
which allows for the extraction of angular momentum. 

We have prompt emission in \(\gamma \) within seconds, while the afterglow in lower-energy
bands can last much longer. 
The scenario with the most consensus is the ``fireball'' model. 

We get a jet with bulk Lorentz factor \(\Gamma \gtrsim 100\), so that 
particles undergo Fermi acceleration, leading to prompt emission. 
When the jet interacts with the medium we have a shock. 

We also have a reverse shock moving backwards.

The idea is that particles are initially very relativistic, but as the jet decelerates
in the ISM we find the lower-energy afterglow.

Typically, the decay of the optical lightcurve is like \(t^{-\alpha }\) with \(\alpha \) between 
1 and 1.5. 

Long GRBs have higher energy, their lightcurves are higher up. 

The relativistic beaming is of the order of \(\theta \sim 1 / \Gamma \) where \(\Gamma \) is 
the jet bulk Lorentz factor. 
However, the jet decelerates in the ISM and its Lorentz factor decreases. 

An off-axis observer cannot observe the prompt emission, but as the jet slows down 
the angle increases. 

People tried to observe off-axis GRBs, but it's hard! There are not gamma-rays to 
guide us. 

Off-axis means that even the highest-energy photons we observe are not in the gamma range. 

A lower ISM density means the emission is fainter and peaks later,
even though the total energy is the same.

We only have about 30 sGRBs with an associated redshift. 

Before SWIFT, people looked at the prompt emission, at the X-ray, but 
not in the middle! 
When it came out it showed that some sources can have flares
in the range of a few minutes to a few hours, or a plateau. 

We don't know why there is a plateau in X-rays but not in the optical! 

The plateau might originate in the slow region of the jet! 
GW170817 showed us that jets have structure. 

\subsubsection{Thermal emission}

The thermal emission associated to a BNS is called the \emph{kilonova emission}. 
The almost-relativistic material which gets unbound in the merger is very neutron-rich:
it is the perfect place for \(r\)-process nucleosynthesis to occur. 

We have expectations about which elements may form ---
lanthanides and actinides are very opaque, and they may emit in the infrared.  

In the interface between the two neutron stars we have shock-heated ejecta. 
These are typically orthogonal to the tidal tail. 

In the ejecta we have weak interactions which increase the electron fraction;
this decreases the neutron fraction and prevents the formation of very heavy elements. 
The emission of these is on the bluer side of the spectrum. 

What is \(r\)-process nucleosynthesis?

How do we form elements higher than iron? 
\(s\) (slow) -process happens when the neutron capture timescale is faster 
than the decay time. 

The basic reactions are \(\beta \)-decay and neutron capture. 

This miiiight happen in supernovae, but simulations show it will not happen a lot,
while in BNS mergers we have very dense environments. 

Some peaks we observe in the Sun are only compatible with \(r\)-process, 
some others are only compatible with \(s\)-process. 

The electron fraction, \(Y_e = n_p / (n_n + n_p)\), is inversely correlated 
with the neutron fraction. 

The number of neutrons is damped by positron capture, which yields protons,
as well as by neutrino capture, which yields protons and electrons. 

In the end, we don't have enough neutrons to form really heavy elements. 

We have different ejecta types: 
\begin{enumerate}
    \item tidal --- equatorial, cold, with low \(Y_e \lesssim 0.1\), and with low mass;
    \item shock-heated --- polar; hot, with high  \(Y_e \gtrsim 0.1\), and with low mass;
    \item neutrino-driven winds --- polar, with high \(Y_e \gtrsim 0.1\), and with low mass;
    \item secular --- isotropic, with a broad range of \(Y_e\), and with high mass.
\end{enumerate}

Low mass means about \SI{1}{\percent} of the disk mass, which is about \(0.3 M_{\odot}\) 
typically.

The total mass of these components, for GW170817, is quite well known, 
while its breakdown is more uncertain. 

If the electron fraction is \(Y_e \lesssim 0.15\) we can form actinides, 
with \(0.15 \lesssim Y_e \lesssim 0.25\) we can form lanthanides, 
with more we cannot go beyond iron. 
These heavy elements are opaque since they have a lot of atomic lines. 

The main parameters are the opacity \(\kappa \), the mass of the ejecta \(m _{\text{ej}}\),
the velocity of the ejecta \(v _{\text{ej}}\). 

We have empirical fitting formulas connecting the peak time of the 
lightcurve and its luminosity to these quantities. 

\todo[inline]{Which come from simulations, right?}

A further important parameter is the radioactive heating rate. 

\(r\)-process kilonovae have suppression of the UV curves, 
and they have more broad spectra. 

Decay for nuclei formed in \(r\)-process can happen through \(\alpha \), \(\beta \) decay or fission,
the radioactive heating rate is the rate of energy deposited by these. 

Also, a problem is due to the fact that we don't know all the energy levels of 
very heavy elements. 

The remnant type is also very important: if we have a direct BH collapse the 
spectra are quite short-lived, while if we form a magnetar
we get longer-lived spectra. 

People started to talk about kilonovae after 2010! 
The first lightcurve only came in 2013. 

\todo[inline]{Are the ejecta then still orbiting the remnant?}

The material propagating outward can then have an effect in the radio band. 
However, the long-lasting radio remnant was not really observed for 170817. 

\todo[inline]{What about accretion of the ejecta?}

\end{document}
