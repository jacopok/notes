\documentclass[main.tex]{subfiles}
\begin{document}

\marginpar{Monday\\ 2022-1-17}

We were talking about the various types of ejected mass in compact binary mergers
which contain a neutron star. 

The ejected mass leaves a rather empty region near the remnant; 
as it goes out it shocks the ISM, leading to a radio remnant. 
Its peak energy is heavily dependent on the ejecta energy, 
which is thought to be of the order of magnitude
of \SI{e50}{erg} to \SI{e51}{erg}. 

If a jet is more energetic, it is also more collimated, 
so an off-axis observer will see it later. 

Another interesting type of emission is X-ray emission from a long-lived NS remnant: 
this can be stable or unstable. 
This emission is expected to be very bright, in the soft X-rays (half to \SI{10}{keV}), and close to isotropic. 
It should be peaked \num{e2} to \SI{e4}{s} after the merger. 

The plateau we observe in the time evolution of GRBs could be due to the presence of a neutron star. 
What we know for sure is that a NS which formed will emit GWs, although 
we currently do not have detectors which can detect them. 

There was a nomenclature discussion about the term "kilonova": before GW170817 people 
thought that their energies were not a thousand times that of a nova, so 
the term ``macronova'' was preferred. 

The moral of the story is this: we have emission in every band and at different 
timescales, so we need a network of detectors! 

For a BBH, we do not expect to see a detectable EM counterpart; 
however Fermi observed some data associated with one GW event, so people have 
started to think about scenarios. 

There could be emission 
\begin{enumerate}
    \item from the remnants of the stellar progenitors;
    \item from the tidal disruption of a star in a triple system;
    \item from the environment of binaries in AGNs. 
\end{enumerate}

The sky localization was around 30 square degrees.
There were only about 50 galaxies to be observed.
A signal, blue at first and then red 10 days later, was observed! 

The signal was in a blind spot of Virgo. 

For almost three months every instrument was pointing at that galaxy. 

Then, astronomers worked in small groups, but 
now European astronomers decided to associate into ENGRAVE. 

The source galaxy was already catalogued, and it was exactly at the 
distance predicted by GWs. 

The mass distribution predicted for 170817 heavily depends 
on the prior spin distribution we assume.  

We can also measure the tidal polarizability of the stars. 

A supermassive NS is sustained by uniform rotation and lasts for about an hour;
a hypermassive NS is sustained by differential rotation and lasts for about a second.

With the same total mass, but a soft EOS, we can have direct collapse into a BH; 
however with a hard EOS we can have a higher likelihood of a NS remnant. 
This is because a hard EOS means a higher maximum mass threshold. 

The postmerger signal, in any case, is emitted at a few \SI{}{kHz}; 
too much for current GW detectors. 

GRB 170817A was precisely consistent with a short GRB; however if we look 
at it compared to the population of sGRBs we see something interesting.

We look at the isotropic energy emission --- the luminosity we would have 
if the emission was isotropic --- 
and we see that the luminosity of this event was very faint compared 
to the sGRB population. 

Is it an off-axis GRB, or an intrinsically faint emission of another kind? 
Initially people thought of the first, but there are alternatives. 

An isotropic outflow would also explain the data: if a jet is formed but it is choked. 

In any case, people expected a decline in the flux; however 
what was observed was a rise in flux for several months! 

The spectrum observed was perfectly consistent with a powerlaw (if the kilonova was removed). 
Powerlaw means nonthermal synchrotron: again, there are two scenarios. 

In the jet case, we have a structured jet with a fast core and slower outer parts; 
in the case of isotropic outflow we also have a layered structure with a faster outer part. 

The rise of luminosity is shallow, roughly with \(t^{0.8}\). 

The way to resolve this degeneracy was \emph{radio high resolution imaging}! 
The VLBI network looked at the size of the source. 
The measured size ruled out mildly isotropic outflow: it was very small, 
meaning that there was a jet. 

It seemed that the viewing angle was about \SI{15}{\degree}, outside the jet core.

The point which is still a bit controversial is the GRB observation itself; 
is it the shock breakout or the slow part of the jet? 

Next time we will look at the thermal, kilonova emission.

\end{document}
