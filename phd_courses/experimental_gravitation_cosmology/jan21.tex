\documentclass[main.tex]{subfiles}
\begin{document}

\marginpar{Friday\\ 2022-1-21}

VLBI works with resolution \(\phi \sim \lambda / D\) where \(D\) is the distance among the 
antennas. 
Quasars are typically discovered in the optical band, but then 
radio observations can allow us to measure distances. 

A source moving with angle \(\theta \) at speed \(\beta c\) will have moved by a 
displacement \(\Delta x = \beta c \Delta t \sin \theta \); 
the time interval on the other hand is \(\Delta t (1 - \beta \cos \theta )\), so 
the apparent velocity is given by 
%
\begin{align}
v _{\text{app}} = \frac{\beta \sin \theta }{1 - \beta \cos \theta }
\,.
\end{align}

\subsection{Thermal emission}

The red kilonova is expected to peak around 1 week after the merger, 
and it is due to tidal ejecta; 
the blue kilonova is given by shock-heated ejecta, and it peaks sooner, 
1--2 days after merger.  

The fraction of lanthanides determines the color and timeseries --- 
they absorb a lot, so a lot of them means later and redder. 

The spectrum is initially very close to a (blue-ish) blackbody, 
then absorption spectral features appear after a few days! 

The temperature decay for a kilonova is much faster than 
that of a supernova. 

Chemical signatures are hidden by the velocity distribution. 
It is therefore not possible to identify single elements.

There were, however, predictions for a kilonova spectrum, 
which made some assumptions about the spectral lines and 
these were validated by 170817! 

Enrico Cappellaro in Padua said he'd never seen something like that in his life!

The ejected mass can be found through a fit; it is on the order of
0.03 to \(0.05 M_{\odot}\). 

Bolometric luminosity as a function of time 
depends on the \(r\)-process mass: 
we observe two different components at different time, the 
blue and red kilonova. 

The temperature initially decreases from \SI{e4}{K}, 
but after a few days it stabilizes to \SI{2500}{K}. 
This is due to the fact that the photosphere moves inward in the ejecta. 

Models are currently not able to consistently reproduce the observed spectral features. 

It is hard to identify single elements, but people thought they found Cesium and Tellurium. 

Nowadays people are seeing strontium lines. 

We can constrain the BNS rate both from GW observations, 
and from heavy elements' abundance! 

EM observations exclude very soft EoS, \(\widetilde{\Lambda} \lesssim 400\)! 
GW observations exclude \(\widetilde{\Lambda} \gtrsim 800\), so this is great! 

In the case of prompt collapse we expect a red KN powered by the tidal ejecta; 
For a supermassive NS we expect it to impart a lot of energy 
to the jet. 
In the end, therefore, we think that the middle situation, a HMNS is the 
one which is consistent with observation. 

The host galaxy, NGC4993,  showed some spiral characteristics even though it was mostly lenticular. 
It was consistent with the merger of two galaxies less than \SI{1}{Gyr} ago. 

There is no evidence of YSC or GSC there. 

No neutrinos were detected in a window around the merger by IceCube, Auger and others.

\end{document}
