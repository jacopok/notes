\documentclass[main.tex]{subfiles}
\begin{document}

\marginpar{Tuesday\\ 2022-1-25}

Prompt collapse is not compatible with the amount of ejecta we observed 
for 170817, as well as the blue kilonova component, 
while a long-lived SMNS is also not compatible with observations.

We know that \(1.2 M _{\text{max}} \lesssim M _{\text{170817}} \lesssim 1.6 M _{\text{max}}\).

\subsection{Observational cosmology and gravitational waves}

Hubble was the first who thought about extragalactic astronomy and observational cosmology. 

He measured the recessional velocities using the Doppler shift of the spectral lines. 
He assumed that galaxies were standard candles.

Why was Hubble's estimate of \(H_0 \) so wrong? 
Nearby galaxies are very contaminated with peculiar velocity. 
A large contribution is attraction to the Virgo cluster. 

The law was renamed Hubble-Lemaître law, since Lemaître discovered the law two years 
before Hubble. 

Luminosity distance is \(d_L = d (1+ z)\), where \(z\) is the redshift.

Supernovae Ia are heavily used as standard candles: we don't know if they
refer to two white dwarfs or a white dwarf together with a companion. 

They are not that standard really: brighter supernovae are longer. 
However, we can calibrate to correct for this issue. 
This allows us to estimate their intrinsic luminosity, therefore 
we can get the distance.

In supernovae we also see spectra, thereby estimating the spectrum. 

There are some systematics which are still not completely under control: 
some variation depending on the galaxy is still there. 

From the phase of the waveform we can estimate the redshifted mass \(\mathcal{M} (1+z)\); 
on the other hand the normalization scales with \((1+z) \mathcal{M} / d_L\), so we can 
estimate with good accuracy (to Newtonian order) both \((1+z) \mathcal{M}\) and \(d_L\). 

One big problem is the degeneracy between distance and inclination. 

Increasing the statistics will surely help. 
Breaking the inclination-degeneracy fluctuation 
can aid in measuring the surface brightness fluctuation. 

Also, we could have more information on the inclination angle with kilonova and afterglow. 

The jet gives a lot of information. 

We could also try to give a statistical estimate of the redshift 
by statistically weighting all the galaxies in the error box. 

LSS cosmology with ET BHs. 

\end{document}