\documentclass[main.tex]{subfiles}
\begin{document}

\marginpar{Wednesday\\ 2021-11-17}

GW detection! 
This part is given by Jan Harms. 

Marica will talk more about GW science and the multimessenger approach.

The first slide is the north arm of Virgo, Einstein Telescope and the LGWA. 

Outline: 
\begin{enumerate}
    \item overview of GW detection;
    \item detector responses to GW;
    \item noise spectra, filtering, SNR, transfer functions;
    \item fluctuation-dissipation, thermal noise;
    \item quantum noise \& squeezing;
    \item passive and active seismic isolation, Newtonian noise;
    \item GW detector concepts.
\end{enumerate}

The interferometer is operated at the dark fringe
\todo[inline]{but not exactly there?}

At a frequency scale of \(\sim H_0 \) we can do measurements thanks to CMB observations. 

At the \SI{}{nHz} scale (corresponding to \SI{}{lyr} wavelengths) we can use Pulsar Timing Arrays,
with radio telescopes.
Pulsars being the ``best clocks in the universe'' is not really true. 
For now, they've discovered red noise, which could potentially be a GW signal. 

At the \SI{}{mHz} scale we might have space detectors such as LISA: 
here the GW wavelengths are of the order of \SI{1}{AU}. 

At the \SI{100}{Hz} scale we have ground-based detectors. 

Current infrastructure is expected to reach \(z \sim 2\); we expect future infrastructure
to reach \(z \sim 100\). 
This goes all the way back to the dark ages in the Universe's history.

This means that ET + CE would see something like \SI{90}{\percent} of BNS mergers, and 
basically all BBH mergers. 

It is good to have a long baseline. 

Jan disses Cosmic Explorer a bit. \SI{40}{km} is ambitious because of the 
cross-couplings due to the fact that the mirrors will not be perpendicular 
to the suspension system, which will point straight down at both edges. 
Also, with such lengths there is a loss in sensitivity in the \SI{}{kHz} band. 

High performance sensing will require cold temperatures, which however 
also means that we need to redesign basically all the infrastructure. 

The dissipation in the system dictates the thermal noise. 

We need in-vacuum optical systems; people were extremely worried before the 
construction of LIGO about this. 
Now in Livingston they found a hole. 
The idea to find it is like finding a hole in a bike tire, with liquid helium
around the detector and helium sensors inside. 

Current detectors have a finite lifetime. Virgo is slowly sinking into the ground,
since it is built on soft soil. 

There are disturbances caused by powerful lightning strikes. 
It is crucial to monitor the environment. 
In the future, there will be online background removal. 

Einstein Telescope needs very big underground chambers,
with roofs on the order of \SI{25}{m}. 
Do we have enough people in Europe to build ET? Not clear.

Kagra taught us some lessons.
The spring melt filled up water reservoirs there, 
so there was a waterfall inside the arms there. 
That means a lot of humidity. 
The water stream near the test masses was gravitationally coupled to them; 
Newtonian noise is bad. 

If ET is built in Sardinia it will be really dry; maybe not so for the Netherlands. 
We need to be careful that the ventilation and cryogenic systems do not create noise. 

Ventilation is needed for both humidity and radioactivity management. 
The experiments in LNGS, say, are much more sensitive. 

LISA will be able to locate sources thanks to their amplitude modulation during the year. 
At these frequencies going underground would not help. 

There is a requirement for drag-free navigation, though. 

The beam to another satellite is on the order of \SI{25}{km} in size. 

Now we talk about LGWA. 
Weber invented everything when it comes to GW detection technologies. 
He invented the resonant bar detectors. 

He tried detecting GWs from quadrupolar seismic vibrations on the Earth! 
We now know that their amplitude is much too small for that. 
They did, however, put the first upper limits on the amplitude of GWs. 

Apollo 17 brought the Lunar Surface Gravimeter. A design flaw limited its sensitivity. 
The problem of having 14 days of lunar night without solar power is large. 

A Russian team is probing microwave beaming for the transmission of power. 
Nuclear power is possible, but how do we shield from it? 
They used plutonium there. However, nobody's producing plutonium anymore\dots 
But it might be produced again thanks to a decision by the Trump administration. 

That gravimeter failed on the Moon because of an arithmetic mistake which 
failed to account for the decreased gravity there. 

One can bring stuff to the Moon at a price of about \(\$ \num{e6} / \SI{}{kg}\). 
This might decrease in the future. 

LGWA: four seismometers at \(\sim \SI{}{km}\) separation. 

LSGA: interferometers connected to the ground, measuring the deformation of the surface of the Moon.
They'd need to be deployed at \SI{10}{km} distance, but the Moon curves: maybe put them 
at the rims of a crater? this is very difficult.  

GLOC: basically Cosmic Explorer on the Moon.
By Jani and Loeb, two theorists.
These detectors do require a lot of maintenance. 

The spectrum of noise on the Moon is on the order of \SI{e-10}{m / \sqrt{Hz}} between \SI{0.1}{Hz} and \SI{1}{Hz}. 

This is much lower than what is observed on the Earth, and which is mostly due to the ocean. 

Micro-meteoroid impacts have a small impact. 

The Moon might be the quietest place in the Solar System: 
it being tidally locked helps a lot. 
It is also near the coldest: there are permanently shadowed regions at the poles. 
These have never seen sunlight, neither direct nor indirect, for a billion years or more. 

It might even be colder than Uranus or Neptune: the absence of radioactivity helps a lot. 

We can use superconductors for free there. 

How do we get power there? 
An option is beaming from solar panels at the rim. 
Another option is beaming from satellites in orbit. 

Nuclear power? Europe will not do it, since ESA does not launch for Europe (as opposed to NASA and the Chinese space agency): therefore, any issues would be dumped onto a South American country, a huge political issue. 

The presence of gravitational waves from inflation, at the tensor-to-scalar ratio given by simple, single-field inflationary models, would be incredibly important. 

A direct observation for this kind of thing would be the Big Bang Observer. 
This would be a proof of quantum gravity! 

This would be a 12-satellite configuration, two triangles and a hexagon, smaller than LISA, sensitive to the deciHertz regime. 
Why not the milliHertz band? LISA is also limited by GW foregrounds! 

Foreground removal is computationally difficult. 

\end{document}
