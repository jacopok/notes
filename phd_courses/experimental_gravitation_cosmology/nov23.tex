\documentclass[main.tex]{subfiles}
\begin{document}

\marginpar{Tuesday\\ 2021-11-23}

We have the following relation for the power spectral density: 
%
\begin{align}
\expval{\widetilde{y}(f) \widetilde{y}^{*}(f')}  
&= \int \dd{t} e^{2 \pi i ft}
\int \dd{t'} e^{-2 \pi i f' t'} \underbrace{\expval{y(t) y(t')}}_{C(y, \tau)}  \marginnote{\(\tau = t' - t\).}\\
&= \int \dd{t} e^{2 \pi i (f - f') t} \int \dd{\tau } e^{-2 \pi i f \tau } C(y, \tau )  \\
&= \int \dd{t} e^{2 \pi i (f - f') t} S ( y, f  ) 
\\
&= \delta (f - f') S (f)
\,.
\end{align}

The fact that if \(f \neq f'\) this vanishes is a crucial property of stationary noise. 

Any linear transformation of such a stationary-noise timeseries will leave it stationary. 

If we have a force timeseries \(F(t)\) applied to a pendulum, we will get a response like 
%
\begin{align}
\widetilde{x}(f) = \frac{\widetilde{F}(f)  / m}{(2 \pi )^3 (f_0^2 - f^2)} 
\qquad \text{where} \qquad
f_0 = \frac{1}{2 \pi } \sqrt{ \frac{g}{L}}
\,,
\end{align}
%
but this \emph{linear} approximation will fail if the force is very strong. 
Nonlinear dynamics can turn stationary noise into nonstationary noise. 
There is a deep connection between linearity and stationarity. 

In the time domain, the application of a linear filter looks like 
%
\begin{align}
y'(t) = \int \dd{t'} K(t - t') y(t')
\,,
\end{align}
%
which modifies the Fourier transform as 
%
\begin{align}
\widetilde{y}' (f) = \widetilde{K}(f) \widetilde{y}(f)
\,.
\end{align}

Therefore, the spectral density of the transformed signal reads 
%
\begin{align}
\widetilde{S}(y', f) = \abs{\widetilde{K}(f)}^2 S(y, f)
\,.
\end{align}

\subsection{Matched filtering}

How do we define the best linear statistic? 
Specifically, we'd want to have something which has the largest amplitude variation between the presence of a signal and the absence of it. 

If the system is linear, we can approximate the data as 
%
\begin{align}
d(t) = s(t) + n(t)
\,,
\end{align}
%
and in the operation of the detector one tries very hard to make this as close to true as possible. 

Suppose we define a filter and apply it to the noise and signal: 
%
\begin{align}
s_F (t) &= \int \dd{t'} F(t-t') s(t') \\
n_F (t) &= \int \dd{t'} F(t-t') n(t') 
\,.
\end{align}

The variance of the filtered noise will then read 
%
\begin{align}
\expval{n_F^2(t)} = \int \dd{f} \abs{\widetilde{F}(f)}^2 S(n, f) 
\,.
\end{align}

The filtered signal timeseries can also be written as 
%
\begin{align}
s_F(t) = \int \dd{f} e^{2 \pi i ft} \widetilde{s}(f) \widetilde{K}(f)
\,.
\end{align}

We define the SNR timeseries through 
%
\begin{align}
\text{SNR}^2 &= \frac{s_F^2(t_0 )}{\expval{n_F^2}}  \\
&= \frac{\abs{\int \dd{f} e^{2 \pi i f t_0 } \widetilde{s}(f) \widetilde{K}(f)}^2}{\int \dd{f} \abs{\widetilde{K}}^2 S(n, f)} \\
&\leq \int \dd{f} \frac{\abs{\widetilde{s}}^2}{S_n}
\,.
\end{align}

We exploited the Schwarz inequality: 
%
\begin{align}
\abs{\int \dd{f} e^{2 \pi i ft_0 }\widetilde{s}\widetilde{K}}^2 &= \abs{\int \dd{f} e^{2 \pi i f t_0 } \widetilde{K} \sqrt{S_n} \frac{\widetilde{s}}{\sqrt{S_n}}}  \\
&\leq \int \dd{f} \abs{\widetilde{K}}^2 S_n \times  \int \dd{f} \frac{\abs{\widetilde{s}}^2}{S_n}
\,.
\end{align}

What this means is that any linear filter can reach, at a maximum, this value. 
If we can then write a filter which reaches this value we know it is the best! 

The optimal filter reads 
%
\begin{align}
\widetilde{K}(f) = e^{-2 \pi i f t_0 } \frac{\widetilde{s}(f)}{S_n(f)}
\,.
\end{align}

Intuitively, one puts less weight on the region of higher noise. 

The kernel looks a bit like the elephant in the Little Prince. 

\section{Perspectives for GW}

GW echoes: if the horizon is quantized, the GWs can be reflected if they are not at the right frequencies. 
For any known quantum gravity theories, the frequencies at which these are reflected are precisely the ones we can probe. 

NS equations of states can be probed by the high-frequency region of the waveform. 

SN explosions can emit GWs if they have anisotropies. 

Stochastic backgrounds can be emitted by cosmological processes. 

How do we calibrate our detector? 
Blind injections were done by moving the test masses. 
The first GW signal was thought to be a blind injection at first, but then we found that nobody had done it. 

The issue is that we cannot continuously characterize the detector in this way. 
The alternative is modelling the detector. 

There are controls which keep the masses stationary at low frequencies. 
So, at low frequencies we look at the \emph{control signal}! 

The problem is most difficult at low frequencies. 
The control interferes with the GW response below \SI{50}{Hz}. 

The equation for the strain, including this, is 
%
\begin{align}
h(t) = \frac{1}{L} \qty[\mathcal{C}^{-1} d _{\text{err}}(t) + \mathcal{A} d _{\text{ctrl}}(t)]
\,.
\end{align}

The inverse of the SNR gives roughly an order of magnitude for the required calibration error. 
Currently, the LIGO-Virgo collaboration has reached roughly 2 to \SI{3}{\percent}.

There are very few places which can give a laser to be used in pushing to calibrate\dots

\subsubsection{Unmodelled signals}

Modelling a SN GW signal is very hard, simulations do not agree with each other. 

The signal roughly increases in frequency from a  hundred to a thousand Hz within about a second. 

The first signal we detected was detected through a burst search: templates with masses as high as \(30 M_{\odot}\) were not used because those masses were not though to be likely. 

Convolving a signal with wavelets allows for a time-frequency plot. 

A type II supernova would have to happen in our galaxy for us to be able to see it with current instruments, and future detectors will not improve this by much. 

See \url{gwburst.gitlab.io}.

\subsubsection{Continuous signals}

A pulsar which is not axisymmetric will emit GWs. 
These are signals which are definitely modelled: we know basically everything. 

Computational cost scales with \(T^{6}\). 

Glitches are a problem in this case.

\todo[inline]{Why is there such a large cluster of millisecond pulsars?}

We know from these GW observations that 
most of the braking in the NS spin-down is \emph{not} due to GW emission. 
We have bounds on the order of \(h \sim \num{e-26}\). 

\end{document}
