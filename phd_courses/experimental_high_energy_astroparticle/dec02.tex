\documentclass[main.tex]{subfiles}
\begin{document}

\marginpar{Thursday\\ 2021-12-2}

We can approximate the interaction of a high energy particle with matter as a continuous energy loss. 

There can also be radiative processes: a decelerated particle will also produce EM radiation. 

This will also often produce secondary particles. 

A standard, main feature is just Coulomb force, but there are many other things to consider if one wants to get a precise prediction. 

This main contribution will be described by the notorious Bethe-Bloch formula. 

We will write \(\dv*{E}{x}\) to mean \(\expval{ \dv*{E}{x}}\): the mesoscopic average value of the energy loss per unit length. 

The Bethe-Bloch formula tells us that this is proportional to certain quantities: 
%
\begin{align}
\dv{E}{x} \propto \rho \qty( \frac{Z}{A}) \mathscr{Z}^2 \frac{1}{\beta^2} \qty[\log f(\beta , \gamma ) + \dots]
\,,
\end{align}
%
where \(\rho \) is the mass density of the medium, \(Z\) and \(A\) refer to the nuclei in the medium,  \(\mathscr{Z}\) and \(\beta \) are the charge and velocity of the particle.

The dependence on \(\beta \) is relatively weak, since \(\beta \sim 1\) will hold in most cases we are interested in. 

At high energies, as a first approximation, \(\dv*{E}{x} \sim f(\beta \gamma )\), so we often plot this as a function of \(\beta \gamma \). 
A minimum can be seen around \(\beta \gamma \approx 3\), where the value of the energy loss per unit density is about \SI{1.6}{MeV / (g cm^{-2})}. 

We know, however, that the number of interactions is given by \(N _{\text{int}} \propto (\rho \Delta x) \sigma \), so the relevant quantity is the \emph{grammage} \(\rho \Delta x\), measured in \SI{}{g cm^{-2}}. 
Confusingly, this is often denoted as 
%
\begin{align}
\frac{1}{\rho } \dv{E}{x} \overset{\text{def}}{=} \dv{E}{X}
\,.
\end{align}

The plateau at high \(\beta \gamma \) is not much higher than the minimum: it is at around \SI{2}{MeV cm^2 g^{-1}}. 

A \emph{minimum ionizing particle} is one which is coming in at the minimum value for \(\beta \gamma \). 

Since \(\gamma = E  / m\) and \(\beta = \abs{\vec{p}} / E\), we have \(\beta \gamma = \abs{\vec{p}} / m\): this means that \(\beta \gamma = 3\) corresponds to \(\abs{\vec{p}} \approx 3 m\). 

Integrating this we get a plot for \(E\) against \(x\): we expect a slow decrease and a sharp drop, but we must keep in mind the fact that this is just an average. 
The low part will not be as sharply decreasing, since the approximation of the BB formula gets worse as \(\beta \) approaches 0. 

We can also compute the \emph{range} of the particle, defined as the average depth the particle reaches in the material. 

The range of a particle with energy \(E\) is given by 
%
\begin{align}
R(E) 
= \int_{E_0 }^{mc^2} \frac{ \dd{E}}{- \dv*{E}{x}}
= \int^{E_0 }_{mc^2} \frac{ \dd{E}}{\dv*{E}{x}} 
\,.
\end{align}

Consider a cosmic muon at \SI{10}{GeV}. 

The atmospheric grammage is roughly \SI{e3}{g cm^{-2}}. 

Protons produce \(\ce{p} \ce{p} \to \pi^{+} \pi^{0} \pi^{-}\), the pions produce \(\mu^{+} \overline{\nu}_\mu \), the muons produce \(e^{+} \overline{\nu}_\mu \nu _e\). 

At sea level, we have roughly \SI{300}{muons / m^2 s}. 

The muon flux increases rapidly in the upper atmosphere, and then stays roughly constant. Why is this?
If we start with a \SI{}{TeV} proton, we expect roughly a \SI{100}{GeV} muon; they then lose about \SI{2}{MeV cm^2/ g }, so over \SI{e3}{g / cm^2} they lose about \SI{2}{GeV}.

This does not take into account the fact that they might decay. A muon's decay time is \SI{2.2}{\micro\second}. 
It travels about \SI{660}{m} in that time, but its Lorentz factor is about \(\gamma \approx 1000\), so its decay time in our reference frame is about \SI{2.2}{\milli\second}, in which it travels \SI{660}{km}.

Less energetic muons will also be part of the distribution, and those might start to decay within \SI{20}{km}, but the effect as a whole is small at sea level.

The ranges of the particles can differ by a lot in certain ranges. 

Suppose we have a beam with \(\abs{\vec{p}} = \SI{200}{MeV} \) with both kaons and muons. 

The range of kaons there is about \SI{6}{g /cm^2}, while the range of muons is about \SI{60}{g / cm^2}, while the density of lead is about \SI{11}{g / cm^3}. 

So, the range in lead for kaons is about \SI{.5}{cm}, while the range for muons is about \SI{7}{cm}. 

If we put a \SI{1}{cm} barrier, the muons will lose about \SI{12}{MeV}, so they will remain relativistic, while most muons will be stopped. 

For electrons, radiative energy losses (bremsstrahlung) are very significant, since the energy emitted in that way has a strong inverse dependence on the mass. 

In that case, we will have a dependence like 
%
\begin{align}
\abs{ \dv{E}{x}} \propto E
\,,
\end{align}
%
therefore there will be some \(x_0 \), which we call the \emph{radiation length}, such that 
%
\begin{align}
\dv{E}{x} = - \frac{E}{x_0 } \implies 
E(x) = E_0 e^{- x / x_0 }
\,.
\end{align}

The constant \(x_0 \) only depends on the medium, and it scales like 
%
\begin{align}
x_0 = 180 \frac{A}{Z^2} \SI{}{g / m^2}
\,.
\end{align}

In water, this is about \SI{36}{g / cm^2}.

Adding this \(\propto E\) term to the Bethe-Bloch formula means we will have a \emph{critical energy} at which the two cross. 

For electrons, this will be of the order of 
%
\begin{align}
E_c \approx \frac{\SI{600}{MeV}}{Z + 1}
\,.
\end{align}

For water, \(E_c \approx \SI{78}{MeV}\). 

What happens to the photons?
At low energies, we have the photoelectric effect. 
At few \SI{}{eV} this is possible, it will then depend on the binding energies of the atoms there. 
This means there will be a few peaks in the \(\sigma \) against \(E\) plot in the \SI{}{eV} to \SI{10}{eV} range.

For Compton scattering, we can compute the Klein-Nishina cross-section, and we get a ``shoulder'' at mid-energies of \SI{100}{eV} to a few \SI{}{keV}. 

After the \SI{1}{MeV} kinematic threshold for pair production, the cross-section for that is roughly constant. 

The typical length for pair production is \(\Lambda_{pp} = 1/ (n \sigma_{pp})\), which is roughly constant with energy when the photon energy is above a few \SI{}{MeV}. 

It comes out that this \(\Lambda_{pp}\) is similar to the radiation length \(x_0 \) for electrons, even though they are describing very different physics.
So, typically, the photons emitted by bremsstrahlung emit a second generation of electrons and positrons on the same length scale as the bremsstrahlung itself. 

This is the process underlying an electromagnetic shower. 
In this toy model, we have \(e^{-} \to \gamma e^{-} \to (e^{+} e^{-}) (e^{-} \gamma ) \to \dots\), so typically the number of particles scales like 
%
\begin{align}
N ( t = x / x_0 ) \sim 2^{t} 
\qquad \text{and} \qquad
E(t = x/ x_0 ) \sim \frac{E_0}{2^{t}} 
\,.
\end{align}

This is the \textbf{Heitler model}. 
After the energy drops too low the secondary particles will start all being absorbed, so the actual curve will have a maximum \(N _{\text{max}} = N (t _{\text{max}})\). 

A good approximation for this is 
%
\begin{align}
t _{\text{max}} = \log_2 \qty( \frac{E}{E_c})
\,.
\end{align}

The logarithmic dependence is good if we want to build a detector:
increasing the energy by an order of magnitude does not mean we need to increase the size of the detector by an order of magnitude!

This all holds both for a high energy \(e^{\pm}\) and for a high energy \(\gamma \), but it is a good model also for other cosmic rays. 
The qualitative behavior is the same for higher-mass particles, but the critical energy is much higher. 

The radiation length scales like 
%
\begin{align}
x^{\mu }_{0} = x^{e}_{0} \qty(\frac{m_\mu }{m_e})^2
\,,
\end{align}
%
and so does the critical energy. 

The critical energy for muons is \(\num{4e4} \) times the one for electrons: roughly \SI{4}{TeV}, which means we only rarely have to deal with bremsstrahlung induced by muons. 

\end{document}
