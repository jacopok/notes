\documentclass[main.tex]{subfiles}
\begin{document}

\marginpar{Friday\\ 2021-12-17}

We were dealing with the interaction of radiation with matter, and 
we saw the Bethe-Bloch formula. 

We have found a way to determine the radiation length 
%
\begin{align}
x_0 = 180 \frac{A}{Z^2} \SI{}{g/cm^2}
\,,
\end{align}
%
as well as seeing that the maximum penetration length is \(x \propto \log (E_0 / E_i)\). 

The final angle of the particle after interaction with the medium will be 
distributed according to a Gaussian with: 
%
\begin{align}
\sigma _\theta \approx \frac{\SI{21}{MeV}}{p \beta } \sqrt{ \frac{x}{x_0 }} \left(1 + \num{2e-3} \log (\dots)\right)
\,,
\end{align}
%
where the logarithmic correction is small, of the scale of its prefactor. 

The Molière radius is defined as 
%
\begin{align}
R _{\text{Molière}} = \frac{\SI{21}{MeV}}{E_c} x_0 
\,,
\end{align}
%
and it characterizes the transverse scale of the beam. 

What happens if a proton is incoming? 
The length before it interacts will typically be 
%
\begin{align}
\Lambda \approx \frac{1}{n \sigma }
\,,
\end{align}
%
and we have already seen what the plot for \(\sigma_{pp}\) looks like.
The oscillations mostly happen at low energies, ad high energies there is an increase,
while in a wide region we have \(\sigma_{pp} \approx \SI{40}{mb}\). 

The hadronic cross-section is quite universal: the length scale for 
the interaction of a pion is similar to that for a proton. 

The longitudinal shower development is always peaked at few interaction lengths, like \(5 \Lambda \), 
always proportional to \(\log E\). 

The distribution of the angle of the products after an hadronic interaction
will look like \(\theta \approx p_T / p_ \parallel\), with \(e^{- p_T / p_0}\). 

Let us assume that \(p_0 \) is close to \SI{300}{MeV}; 
the scale of the transverse momentum can be estimated with \(\Delta x \Delta p \sim \hbar\), 
where \(\Delta x\) is the length scale of the interaction cross-section. 

Typically, in a material \(\Lambda > x_0 \) as long as \(Z > 10\) in the medium. 
Therefore, the hadronic shower typically is deeper than the electromagnetic one. 

Suppose we are in LNGS and we have cosmic rays in the atmosphere. 
What is the minimum energy the muons must have to reach the detector? 

Gran Sasso rock has a density of roughly \SI{2.3}{g/cm^3} and a 
depth of roughly \SI{1500}{m}. 
This is equivalent to about \SI{3}{km} of water. 

So, we want to require that \(R_\mu (E) \geq \SI{3200}{mwe}\). 
This is defined as 
%
\begin{align}
R_\mu (E) = \int_{m_\mu c^2}^{E} \frac{ \dd{E}}{\abs{ \dv*{E}{x}}}
\,.
\end{align}

How does the energy loss scale as a function of energy? 
We will have several contributions, so 
%
\begin{align}
\dv{E}{x} = f_{\text{bethe-bloch}} (E) + f _{\text{bremsstrahlung}}(E)
\,,
\end{align}
%
but the bremsstrahlung for muons will be suppressed by \((m_\mu / m_e)^2\) compared to that of electrons: the threshold for electrons is roughly \SI{100}{MeV}, so that for muons is \SI{4}{TeV}. 

So, unless we deal with very high-energy muons the contribution can be neglected; we will be able to write the radiative energy loss by bremsstrahlung as \(bE\) if needed (above a few hundred \SI{}{GeV}). 

The Bethe-Bloch term is roughly constant for a very large energy range, so the integral is rather easy. 
In general, if 
%
\begin{align}
\abs{\dv{E}{x}} = a + bE
\,
\end{align}
%
then we will have 
%
\begin{align}
R(E) = \int_{mc^2}^{E} \frac{ \dd{E}}{a + bE} \approx \int_{mc^2}^{E_c} \frac{\dd{E}}{a} + \int_{E_c}^{E} \frac{ \dd{E}}{a+ bE}
\,,
\end{align}
%
where the \(bE\) term is dominant in the high-energy range. 
This means that when \(E < E_c\) we simply have \(R(E) = (E-mc^2) / a\), while for \(E > E_c\) we get 
%
\begin{align}
R(E) \approx \frac{E_c - mc^2}{a} + \frac{1}{b} \log \frac{E}{E_c}
\approx  \left( 2.5 + 1.2 \log \frac{1 + 4 E / \SI{}{TeV}  }{3} \right)
\,.
\end{align}

The value of \(b\) when \(E\) is expressed in \SI{}{MeV} is approximately \(b \approx \SI{8e-6}{MeV g^{-1} cm^2}\). 
This means that the minimum energy for muons is about \SI{1.4}{TeV}. 
Not very many muons have such an energy. 
Suppose we have a pion with energy \SI{10}{GeV}. 

The  minimum energy loss rate is \(\SI{2}{MeV g^{-1} cm^2}\), this means that the range is 
%
\begin{align}
R = \frac{E}{\Delta E / \Delta x} \approx \SI{5e3}{g / cm^2}
\,,
\end{align}
%
which translates to \SI{50}{m} in water, \SI{7}{m} in iron, \SI{50}{km} in air. 

These are generally quite long; what about the hadronic interaction length \(\Lambda \)? 
For the cross-section we can use \(\sigma \sim \SI{40}{mb}\), for the number density we have 
%
\begin{align}
n = N_A \frac{\text{atoms}}{\text{mol}} \frac{A \text{nucleons / atom}}{A \text{grams}/ \text{mol}} \rho \approx N_A \frac{\rho}{\SI{}{g/cm^3}} \text{targets / } \SI{}{cm^3}
\,.
\end{align}

This comes out to 
%
\begin{align}
\Lambda \approx \frac{\SI{0.4}{m}}{\rho / (\SI{}{g/cm^3})}
\,.
\end{align}

When looking up the proper numbers, we have that interaction dominates: in iron, it happens after only \SI{1.6}{cm}, in water after \SI{8.6}{cm}, in air after about \SI{700}{m}. 

We also need to look at decay! The decay time is \SI{26}{ns} for a pion, meaning that the decay length is \(c \tau \gamma \beta \approx \gamma \SI{7.8}{m} \approx \SI{540}{m}\). 

This scales linearly with the energy! 

\section{Detectors}

One of the simplest detectors one can make is a Geiger counter:
a chamber with gas and a wire. 

A particle comes through and generates ionization; 
since there is an electric field the electrons and ions are separated from each other. 

The velocity of the drift is typically plotted as a function of \(E / P\); it saturates to a certain plateau value, typically of the order of a few \SI{}{cm/\micro s} for electrons, while ions are a factor 1000 slower. 

The variation in the voltage is related to a variation in the charge by \(\delta V = \delta q / C\). 
The rise time of the voltage is roughly \(v _{\text{drift}} R\), where \(R\) is the length scale of the detector. 

The HV of the detector is quite large compared to \(\delta V\); so we need a high-pass filter, such that \(RC \lesssim v _{\text{drift}} R\).

Typically one uses \(RC = (\SI{50}{\ohm}) (\dots)\).
Also, one puts a strong resistor before the detector, to reduce the current.

The value of \(\delta q\) will be \(N _{\text{pairs}} e\); the number of pairs can be computed as 
%
\begin{align}
N _{\text{pairs}} = \frac{\Delta E}{W}
\,,
\end{align}
%
where \(W\) is the energy needed to release one pair, while \(\Delta E\) is the total energy lost in the detector. 
Typically, in gas detectors one has \(W \approx \SI{10}{eV}\). 

What is \(\Delta E\)? Since \(\dv*{E}{x} \approx \SI{2}{MeV g^{-1} cm^2}\) we have \(\Delta E = \SI{2}{keV}\) in a \SI{}{cm} of material. 
This means that we release about 200 pairs, or \SI{3e-17}{C}. 

What is the capacitance of a cylinder with a given length and a radius of \SI{1}{cm}?  

\end{document}