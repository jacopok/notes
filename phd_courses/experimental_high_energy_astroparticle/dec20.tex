\documentclass[main.tex]{subfiles}
\begin{document}

\marginpar{Monday\\ 2021-12-20}

\begin{extracontent}
    What is the capacitance of a cylinder with a wire?
    %
    \begin{align}
    C = \frac{2 \pi \epsilon  L}{\log R _{\text{cylinder}} / R _{\text{wire}}}
    \,.
    \end{align}
\end{extracontent}

Last time we were looking at some particle detectors. 
We are looking at a detector with a cylindrical shape --- a wire within a cylinder. 

We saw that for a \(\sim \SI{2}{MeV}\) particle there will be roughly 200 emitted electrons. 

The drift time for a \SI{1}{cm} detector will roughly be \SI{200}{\nano\second}, 
since the drift time is roughly \SI{5}{cm/\micro\second}. 

This means that the shape of the pulse, after an RC high-pass filter, is 
%
\begin{align}
\overline{V}(t) = \overline{R} \overline{C} \dv{V(t)}{t} \approx \overline{R} \overline{C}
\frac{ \delta V}{T_D} = \overline{R}\frac{\overline{C}}{C} \frac{ \delta q}{T_D}
\,.
\end{align}

The current will roughly be \(\delta q / T_D \approx \SI{0.16}{nA}\).
The change in tension is roughly \(\Delta V \sim I R \approx \SI{8}{nV}\). 

So, we need an amplification mechanism. 

The anodic wire needs to be very thin, so that we have a large electric field, 
which means that we get amplification by the emission of secondary electrons. 
This gain can be a factor \num{e5}! 
This means we get a charge \(\delta q \approx N _{\text{pair}} e \times \text{gain}\). 

We get \(\Delta V \sim \SI{0.8}{mV}\); with better systems we typically get millivolt currents. 

We should also aim to reduce the capacitance of the detector! 

At low values for the high voltage we are in the ionization regime; this means that we 
see exactly the number of charges produced by the particle. This is the ionization regime. 
\(\delta q (V) = \text{constant}\). 

At higher voltages, we will be getting a certain gain: this is the proportional regime. 
\(\delta q(V) = \alpha \delta q\).

At even higher voltages, we will get a nonlinear profile, with saturation. 
This is the limited proportional regime.

At even higher voltages, we fully saturate: this is the Geiger-Muller regime. 
Here we do not have any information about the particle, we just know that 
something has passed through. 

This all depends on the charge of the particle: for \(\alpha \) particle the \(\delta q\) is 
always \(Z^2 = 4\) times higher. 

We can use many of these gas detectors to figure out where a particle passed through. 

If this is the case, we can also get trajectory information! 
The delay between the arrival time of the particle ionization in different wires tells us about
how close to the center of each of them it passed. 
If we have a star, trigger that's even better! 

This kind of thing is used in LHC. 

An alternative is to have many HV wires in the same chamber. 
This is called a \emph{wire chamber}. 

The spatial resolution can be computed as \(\sigma _x \approx v _{\text{drift}} \sigma _t\), 
so if we have millisecond timing we can get a few hundreds of microns. 

Another component is a \emph{quencher}. 
Some electrons and ions produced might recombine, producing UV photons. 
Those can then ionize the gas again, in a place which is unrelated. 
So, we need to quench these UV photons (absorb them) so that they do not mess up 
our measurement.

If we also have a magnetic field, the trajectory of the particle passing through 
will be bent; this allows us to measure the momentum and charge of the particle, but
it will also curve the trajectories of the electrons. 

What to do when the interaction cross-section is very low?
We can increase the density of the gas, or even make it a liquid.

Those can be for example liquid Argon or liquid Xenon Time Projection Chambers.

\subsection{Scintillators}

A fraction of the energy from a particle is emitted as light. 
Atoms are ionized and then recombine; the UV light they emit thus is 
called \emph{scintillation light}. 

If the total energy lost is \(\Delta E\), and \(W\) 
is the energy required to produce a scintillation photon (\(W \approx \SI{100}{eV}\)). 

For a \SI{2}{MeV} particle, we get about 20000 photons. 
The wavelength of this light depends on the energy states of the material. 
Typically the emission of these happens within a few nanoseconds. 

We need the material to be transparent to the scintillation photons, 
but we need to contain it within a reflective material so that the photons are not
going everywhere. 

The mechanism for the absorption of these visible or UV photons will be the photoelectric effect. 

The quantum efficiency of the detector describes the fraction of the photons which are detected. 

We need to give energy to the electron emitted by the photon!
It will have low, order-\SI{}{eV} energy. 
The way this is typically done is through a series of dynodes. 
We connect the HV supply to a series of resistors, and have a uniformly decreasing 
voltage between the dynodes. 
The gain will depend on the number of secondary electrons produced by the dynode chain
per incident electron, which is denoted as \(\alpha \). 

The dependence is roughly a powerlaw as a function of the HV. 
The system is roughly saturated when the current due to the secondary electrons
roughly equals the HV/\(R\) current. 

These are \emph{photomultipliers}. 

We will have a certain efficiency for our collection \(\epsilon _{\text{coll}}\), as well as 
a quantum efficiency \(\epsilon _q\).

The signal will be 
%
\begin{align}
N_\gamma \epsilon _q \epsilon _{\text{coll}} \times \text{gain}
\,.
\end{align}

Typical numbers are about \(N_ \gamma \approx \num{2e4}\), and both efficiencies at \SI{20}{\percent}. 

With this, we get about \(N = 800\) photoelectrons. 

The time required for the detection is a few nanoseconds for the emission, 
a few nanoseconds for the propagation, and a further few nanoseconds for 
the collection by the photomultiplier.
In the end, the current will be roughly \(I \approx eN / (\SI{100}{ns})\); 

With a \SI{50}{\ohm} resistor, this means a voltage of \SI{70}{nV} times the gain. 

In order to get to the millivolt regime, we need a gain on the order of a million. 
So, if each dynode allows for a doubling of the electrons we need about 20. 

With these kinds of detectors we can do nuclear spectroscopy. 

\end{document}
