\documentclass[main.tex]{subfiles}
\begin{document}

\marginpar{Tuesday\\ 2021-12-21}

Another type of particle detectors are based on the properties of semiconductors.

If a forward bias is passed through the diode, we have very high current, 
while if we put a reverse diode there is very little current, up until a 
point where we get a breakdown. 

A diode can be a particle detector if it works in reverse bias. 
If an ionizing particle passes through the depletion region, 
we get a voltage we can measure.

In practice one can typically make a single \(n\)-doped region, with several small 
\(p\)-doped sections. 

The energy needed to generate a signal is \(W \approx \SI{3}{eV}\), and we also have 
a relatively small amount of fluctuations. 
The relative fluctuations \(\sim 1/\sqrt{N}\) due to Poisson statistics cannot be eliminated, 
but since \(W\) is low we have larger \(N\) and therefore smaller relative fluctuations. 

Operating the diode at low temperatures is convenient so that we get less noise 
in the measurement of the total energy, if the particle can be completely stopped
inside the detector. 
These detectors have good energy resolution when operated in this way. 

We can make \(p\)-strips which are 50 or even \SI{20}{\micro\metre} wide, this means we
get very good localization. 

A grid of these detectors can do a lot. 
Silicon PN junctions can do localization as well as energy measurement. 

In summary: 

\begin{enumerate}
    \item Gas detectors have \(W \approx \SI{10}{eV}\), 
    they can be used for tracking with \(\sigma _x \approx \SI{300}{\micro m}\), 
    they can measure energy loss \(\dv*{E}{x}\);
    \item scintillator detectors have \(W \approx \SI{100}{eV}\), 
    they can measure the energy released very well, 
    they have very good, order-nanosecond time resolution;
    \item semiconductors have \(W \approx \SI{3}{eV}\), 
    they have good statistics since the number of electrons is very large, 
    they can be made in large numbers and used for tracking with \(\sigma _x \approx \SI{20}{\micro m}\),
    they can measure the energy output. 
\end{enumerate}

Most detectors' basic operating principle are based on these technologies.

We can make sub-detectors with these! 
We will have an incoming particle with a momentum \(\vec{p}\); 
it will have a track length \(L\), and it will be deflected by a bending
angle \(\theta \) by our magnetic field. 
We define the \emph{saggitta} \(s\) as the maximum distance between the trajectory of the particle and the straight line between its enter and exit points in our detector. 

If the track is circular, and \(\rho \) is its radius of curvature, we get 
%
\begin{align}
s = \rho \frac{1 - \cos(\theta /2)}{2} \approx \frac{\rho \theta^2}{8}
\,.
\end{align}

The radius of curvature will be given by \(\rho = p / qB\); 
since 
%
\begin{align}
\frac{L/2}{\rho } \approx \frac{\theta }{2}
\,
\end{align}
%
we get \(\theta = L/ \rho \). 
This means that 
%
\begin{align}
s \approx \frac{1}{8} \rho \frac{L^2}{p^2} = \frac{1}{8} \frac{L^2}{\rho } \approx \frac{L^2}{8} \frac{qB}{p}
\,,
\end{align}
%
therefore the saggitta is inversely proportional to the momentum of the 
particle. 

What does the uncertainty calculation look like? \(B\) is typically well-known, so we get 
%
\begin{align}
\frac{\sigma_p}{p} \approx \frac{\sigma _s}{s} \approx \frac{8 p}{qBL^2} \sigma _s \propto p
\,,
\end{align}
%
so the relative accuracy on the momentum \emph{decreases} as the momentum rises. 

Typically, because of this, one looks at the quantity \(\sigma _p / p^2\). 

We want to make a solenoid for our magnetic field. 
The relation \(B = B(I)\) as a function of the current in the solenoid. 

Hysteresis cycle for \(B\) as \(I\) changes. 
There is a saturation around \SI{1}{T} to \SI{2}{T}; if we use superconductors we can reach a few Teslas. 

We therefore only measure momenta up to 
%
\begin{align}
p _{\text{max}} &= \frac{q B L^2}{8 \sigma _s}  
 \\
\frac{p _{\text{max}}}{\SI{2}{TeV} / c} &\approx \left( \frac{q}{e} \right)
\left( \frac{B}{\SI{1}{T}}\right) 
\left( \frac{L}{\SI{1}{m}}\right)^2
\frac{1}{8} \left( \frac{\sigma_s}{\SI{20}{\micro m}}\right)^{-1} 
\,.
\end{align}

This is an order of magnitude, and we get it with \SI{100}{\percent} uncertainty at \(1 \sigma \). 

The rigidity \(R = p / q\) is actually the quantity we are able to measure.\footnote{The name is behind the ``rigidity'' of the track to being bent.}

The Maximum Detectable Rigidity for this detector is \(\SI{2}{TeV/e} = \SI{2}{TV}\). 

This is a good approximation for the AMS detector. 
For larger ions we can reach much higher momenta. 

When the high-energy particle passes through a medium it is deflected with a root-mean-square angle of 
%
\begin{align}
\sigma _\theta = \frac{\SI{2.1}{MeV}}{p \beta } \sqrt{\frac{x}{x_0 }}
\,,
\end{align}
%
which will limit our resolution on the saggitta with a contribution 
%
\begin{align}
\frac{\sigma _s}{s} &= 2 \frac{\sigma _\theta }{\theta } = 2 \frac{p}{L} \sigma _\theta  \\
&= \frac{\SI{2.1}{MeV}}{qBL \beta } \sqrt{ \frac{L}{x_0 }}
\,.
\end{align}

This needs to be added in quadrature: 
%
\begin{align}
\frac{\sigma _p}{p} = \frac{8 \sigma _s}{qBL^2} p \oplus \frac{\SI{2.1}{MeV}}{qB \beta } \frac{1}{\sqrt{L x_0 }}
\,.
\end{align}

The resolution as a function of the momentum is limited at low momenta as well. 

\end{document}
