\documentclass[main.tex]{subfiles}
\begin{document}

\section{Detectors}

\marginpar{Thursday\\ 2022-1-13}

We can show all-particle and by-particle spectra corresponding to the state of the art. 
At Monte Aquila they mounted particle detectors. 

The things we can measure about a particle are
\begin{enumerate}
    \item its momentum, with a  magnetic spectrometer;
    \item its velocity, with time-of-flight, Cherenkov angle \(\cos \theta = 1 / \beta n\), transition radiation when \(\gamma \gtrsim 1000\);
    \item its energy loss, through the Bethe-Bloch formula;
    \item its total energy \(E = \gamma m_0 c^2\) through calorimetry.
\end{enumerate}

It's hard to measure momentum! 
Coulomb scattering worsens the resolution because it's stochastic noise. 

To measure time of flight, the best thing to do is to use scintillator detectors. 

The differences between the times of flight of different-mass particles 
will scale like 
%
\begin{align}
\Delta t \approx \frac{Lc}{2 p^2} (m_1^2 - m_2^2)
\,.
\end{align}

If our time resolution \(\sigma _t\) is fixed, we will only be able to measure
a mass difference \(\Delta m^2\) up to a maximum momentum which scales like 
%
\begin{align}
p _{\text{max}} \propto \sqrt{\frac{L \Delta m^2}{\sigma _t}}
\,.
\end{align}

Ionization energy losses are different for different particles, 
which allows us to distinguish them. 

The problem is that the Bethe-Bloch formula is just an average: 
there are fluctuations. 
These are Landau fluctuations, and they are worse if the material is thin.

Cherenkov emission was discovered serendipitously, 
when they were looking at flashes of light in non-scintillating materials. 

There is aberration in the emission! 
Emission is characterized by \(\cos \theta _c = 1 / n \beta \); 
the number of photons emitted looks like 
%
\begin{align}
\frac{ \dd{N}}{ \dd{x} \dd{\lambda }} = \frac{2 \pi z^2\alpha }{\lambda^2} \sin^2 \theta _c
\,,
\end{align}
%
therefore emission is peaked towards the blue end of the spectrum. 

There is a high-energy cutoff, but typical detectors cannot reach it. 

Since there is a threshold for Cherenkov emission, we can have different 
blocks and use the information about which of them we saw light in 
to identify particles. 

The technique of Ring Imaging Cherenkov (RICH) was used to 
find antiprotons by Segré and collaborators: 
the idea is to measure the Cherenkov light on a plane with photodetectors. 

\paragraph{Transition radiation detectors}

When a particle goes through the boundary between media with different 
indices of refraction, it emits this kind of radiation, typically in the X-ray range. 

The effective threshold for a measurable emission is \( \gamma \gtrsim 1000\). 

Many particle identification methods are available, and a good 
particle detector will use several. 

\paragraph{Calorimetry}

We measure the total energy dumped by a particle. 

One can parametrize the length of the EM shower; it scales logarithmically with the energy.

The error would ideally scale like \(\sigma \sim \sqrt{E}\), but in reality we have 
%
\begin{align}
\sigma / E = \frac{a}{\sqrt{E}} \oplus b \oplus \frac{c}{E}
\,,
\end{align}
%
where \(a\) is the stochastic term, \(b\) is the constant term, while \(c\) is the noise term. 

The interaction length of a hadronic shower is typically much larger than the radiation length. 
So, it is much easier to measure the energy of an electromagnetic shower. 

\subsection{Space exploration}

The first satellite sent in space with a particle detector was Sputnik II, with Geiger-Muller counters. 

The belts, however, are called Van Allen belts since Sputnik only beamed signals to Earth
when it was above the USSR. 

The Explorer missions from the US, instead, measured the belts. 
They measured artificially injected electrons' permanence in certain bands in the high atmosphere. 

The Proton missions in the early 60s also measured the first hints 
of the logarithmic rise of the hadronic cross-section.

\end{document}