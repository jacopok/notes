\documentclass[main.tex]{subfiles}
\begin{document}

\section{Gamma ray astronomy}

\marginpar{Monday\\ 2022-1-17}

We want to observed photons with energies \(\gtrsim \SI{1}{MeV}\). 
The atmosphere is transparent in the optical and radio bands, but not really anywhere else. 

High energy photons (more than UV) cannot penetrate the atmosphere without 
interacting, typically. 

There are many motivations to do gamma-ray astronomy: 
both from astronomy and cosmology, and from astroparticle physics. 

What are their sources? particle interactions at sites of cosmic ray acceleration, mostly. 
The spectra are, therefore, powerlaws. 

They can come from accelerated electrons or accelerated protons. 

We can have pointlike and extended sources, as well as diffuse galactic emission, as well
as an isotropic background. 
We could also have light from dark matter! 
For example, a gamma line from annihilation of DM particles. 

We can plot a gamma ray horizon: if the kinematic threshold for the interaction 
with CMB is met exactly, we can get a very short mean free path (on the scale of the Milky Way); 
if, instead, the photons are much more energetic there is no issue. 

We would like to have a space-based detector, but the fluxes are very small! 
For Vela, we only get a few photons per day. 

With a space-based detector, we have a very high charged particle background: 
the detection mechanism is typically pair production. 

Angular resolution and energy resolution are both very important. 

Already in the sixties the OSO-3 mission found evidence for diffuse \(\gamma \)-ray emission 
from the galaxy. 

Definitions: take a point source with a differential flux \(\dv*{F}{E}\), measured in \SI{}{m^{-2} s^{-1} GeV^{-1}}. 

The differential count spectrum will be given by 
%
\begin{align}
\dv{N}{E} = A _{\text{eff}} \dv{F}{E}
\,,
\end{align}
%
where the effective area is close to the geometric area but it depends on the energy; 
it is given by 
%
\begin{align}
A _{\text{eff}} (E, \theta , \varphi ) = A _{\text{geo}} (E, \theta , \varphi ) \epsilon _{\text{det}} (E, \theta , \varphi) \epsilon _{\text{sel}} (E, \theta , \varphi ) 
\,.
\end{align}

If we are dealing with an isotropic flux, instead, we measure the flux in \SI{}{m^{-2} s^{-1} sr^{-1} GeV^{-1}}. 

The differential count spectrum will then be given by 
%
\begin{align}
\dv{N}{E} = GF \dv{J}{E}
\,,
\end{align}
%
where the geometric factor \(GF\) is a number given by 
%
\begin{align}
GF(E) = \int A _{\text{eff}} (E, \theta , \varphi ) \dd{\Omega }
\,.
\end{align}

These might all also depend on time: in that case, we should look at the exposure
factor 
%
\begin{align}
EF(E) = \int GF(E) \dd{t}
\,,
\end{align}
%
which is measured in \SI{}{m^2 sr yr}. 
This is analogous to an integrated luminosity in particle physics. 

The Field of View is defined as 
%
\begin{align}
\text{FoV} (E) = \frac{GF(E)}{A _{\text{eff}} (E, \theta = 0)}
\,,
\end{align}
%
so that, for example, in the case of a planar detector \(\text{FoV} = \pi \sin^2 \theta _{\text{max}}\). 
This is because the effective area is just \(A _{\text{eff}} (\theta ) = S \cos \theta \). 

The old instrument was EGRET, now we have FERMI-LAT. 
Its angular resolution scales with energy: it is on the order of single degrees below a \SI{}{GeV},
less than a degree above a \SI{}{GeV}.

DAMPE was made for electrons, but it can also measure photons. 

What about detection on the ground?
In space we are limited at an energy of about a \SI{}{TeV}, to go above we need 
ground-based detectors. 

It is better to go to high-altitude sites.
The thickness of the atmosphere is about \(30 X_0 \). 
If we go high enough, we can get about half an atmosphere in terms of column density:
this will mean \(15 X_0 \). 

We can then try to detect particles with a Particle Detector Array.

Even if none of the secondary particles reach the ground, though, we still have
the Cherenkov light from the relativistic secondaries: 
this is not absorbed in any significant part. 

These Imaging Atmospheric Cherenkov Telescopes only work at night. 

IACTs have good angular and energy resolution, as well as source selection, 
but poor FoV. 

We can plot shower size (which means number of particles) as a function of atmospheric depth. 
This shows that having a detector on top of Mount Everest would be roughly ideal. 

The minimum energy detectable is determined by altitude, sampling, trigger energy deposited;
the maximum energy is determined by the total area, and the dynamic range of the detector. 

A shower is called young or old depending on when we measure it.
The age is parametrized by a parameter \(s\), defined so that \(s=1\) at the maximum of the shower.

In the 1950s there was a key theoretical prediction: the Crab Nebula should 
produce many \SI{}{TeV} photons. 

The ARGO-YBJ detector in Tibet, an Italian-Chinese collaboration, measured these. 
It was basically a grid of capacitor pixels, which were very dense. 

Hadron showers are typically more spread than photon showers. 
This is a way to identify them, the other is to look at 
anisotropies in the angular distributions: protons are isotropic, photons 
come from specific sources typically. 



\end{document}
