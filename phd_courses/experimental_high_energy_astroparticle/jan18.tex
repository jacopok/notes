\documentclass[main.tex]{subfiles}
\begin{document}

\marginpar{Tuesday\\ 2022-1-18}

The first lights of IACT. 

To a first approximation, Cherenkov light is collinear with the particles,
and we can detect Cherenkov light even if the shower was completely 
absorbed in the upper atmosphere. 

The first experiments about this were done in the 1950s. 
They saw about an event every two minutes. 

The Crab Nebula was observed in 1989 in gamma rays. 

There are two main working regimes: 
the low-energy \(\gamma \)s are many with low light yield --- we want small ground area but large mirror area, while 
the high-energy \(\gamma \)s are few with high light yield --- we want large ground area but small mirror area.

The resolution is quite good with current detectors. 

\section{Primary electron or positron detectors}

We mostly observe matter in cosmic rays, but 
there is also a fraction of antimatter. 
These are thought to be mainly the secondary
products of other high-energy cosmic rays. 

The total grammage crossed by CRs in the galaxy is something like \SI{5}{g / cm^2}.

The bremsstrahlung energy loss for electrons means the electron spectrum goes down much faster with energy. 

We can measure these with either calorimetry or spectroscopy. 
We want to know incoming direction, energy, particle type. 

Why would we use a calorimeter as opposed to a magnet? 
It is the only possibility to reach the highest energies. 

There is a \emph{positron anomaly}: the positron flux \emph{increases} at high energy! 

DAMPE was calibrated at CERN, but that can only work up to a certain 
energy, while we need to extrapolate into the high-energy regime.

The positron excess is not matched in terms of antiprotons. 

\end{document}
