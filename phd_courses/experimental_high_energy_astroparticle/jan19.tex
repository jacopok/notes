\documentclass[main.tex]{subfiles}
\begin{document}

\subsection{Neutrino telescopes}

\marginpar{Wednesday\\ 2022-1-19}

\begin{enumerate}
    \item Muon neutrinos: hadronic shower + muon track;
    \item electron neutrinos: hadronic and electromagnetic shower;
    \item tau neutrinos: two separate hadronic showers;
\end{enumerate}

If the neutrino spectrum has a slope \(E^{-2}\), we expect a neutrino rate 
%
\begin{align}
R = \int _{E_{\text{min}}}^{ \infty } k E^{-2} \sigma (E) \rho N_A \dd{E}
\,,
\end{align}
%
and we would like \(R V \SI{1}{yr}\) to be at least a few. 
The required size comes out to be a few hundred meters. 

We will not be able to distinguish neutrinos from antineutrinos: there
is no way to have such a large magnetic field.  

IceCube saw an excess compared to the expected neutrino background. 

The low-energy regime is up to \SI{100}{GeV}, then up to \SI{0.1}{TeV} is the mid-energy regime, while up from there is high-energy. 

How to distinguish ultra high-energy neutrino-induced events from proton-induced ones? 
The idea is to look near the horizon: a proton or nucleus could not have 
crossed that much atmosphere. 

\subsection{Hadrons}

\dots

\end{document}
