\documentclass[main.tex]{subfiles}
\begin{document}

\section{Ultra High Energy Cosmic Rays}

\marginpar{Friday\\ 2022-1-21}

Current experiments include the Telescope Array and the Pierre Auger observatory. 

The flux is ultra-low: on the order of 1 particle per kilometer squared per
century per year.
The mean free path of a photon as a function of energy rises after the cutoff. 

The modification of the trajectories of ultra-high energy cosmic rays 
by galactic magnetic fields becomes very small around \(E \gtrsim \SI{e19}{eV}\).

The galactic magnetic field is about 3 orders of magnitude larger than the extra-galactic one. 

There are experiments with very high area. 

One can estimate the position of the shower core, and from it get a 
determination of the primary energy. 
This is done passing through the value of the lateral particle distribution 
at a certain lateral distance. 

If we have two ``eyes'' observing fluorescence from  the same shower we can get a better 
shower reconstruction.  

Auger has both fluorescence detectors and particle detectors.
The particle detectors can be calibrated in energy using the subset of events 
which were taken at night. 

The depth of the shower maximum scales logarithmically with the energy;
if we fix the energy the depth changes with \(\log A\). 

Auger detected a large scale anisotropy above \SI{8e18}{eV}!



\end{document}