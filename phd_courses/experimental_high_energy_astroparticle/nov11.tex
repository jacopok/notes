\documentclass[main.tex]{subfiles}
\begin{document}

\paragraph{A \(2 \to 3\) process}

\marginpar{Thursday\\ 2021-11-11}

Before looking at three-body decays, let us consider a \(2 \to 3 \) process. 

Consider 
%
\begin{align}
\ce{K}^{-} + p \to \pi^{+} + \pi^{-} + \Lambda 
\,,
\end{align}
%
where the masses are \(m_\pi \approx \SI{1}{GeV}\), \(m_ \pi \approx \SI{140}{GeV}\), \(m_K \approx \SI{500}{MeV}\), and \(m_\Lambda \approx \SI{1.1}{GeV}\). 

The experimental setup is a kaon beam impacting upon a proton target. 

The invariant mass of the starting particles will be 
%
\begin{align}
m_{Kp} = \frac{1}{c} \sqrt{(\sum  p_i)^2}
 = \frac{1}{c} \sqrt{(p_k + p_p)^2}
\,.
\end{align}

We could also define a corresponding energy \(m_{Kp} c^2\). 

If we extend this sum to all the particles in the initial state, \(m _{\text{initial}} c^2 =  m_{\text{final}}c^2 = \sqrt{s}\) is the total invariant mass. 
The variable \(s\) is written as \(s = c^2 p^2 _{\text{tot}}\). 
Its meaning is that it is the total energy in the center-of-mass frame. 

We can then think of this as an equivalent decay of a particle with \(m_{Kp}\), therefore the kinematic threshold is \(m_{Kp} \geq m_{\pi^{+}} + m_{\pi^{-}} + m_\Lambda \). 
This is a general statement: 
%
\begin{align}
\sqrt{s} \geq \sum _{i \in \text{products}} m_i c^2
\,.
\end{align}

Let us approximate the protons as stationary, and we have a kaon with \(E_K\).

The total four-momentum reads 
%
\begin{align}
p _{\text{tot}} = (E_K + m_p, \vec{p}_k)^{\top}
\,,
\end{align}
%
so we get 
%
\begin{align}
\sqrt{s} &= \sqrt{(E_k + m_p)^2 - \abs{\vec{p}_k}^2}  \\
&=\sqrt{m_K^2 + m_p^2 + 2 E_k m_p}
\,.
\end{align}

We can then straightforwardly compute the energy threshold for \(E_k\). 
In this case we have \(E_k^{\text{threshold}} < m_k\), so we have no real lower limit. 

This reasoning can be applied, for example, when looking at the reaction
%
\begin{align}
\ce{p} + \ce{p} \to 3 \ce{p} + \overline{p}
\,.
\end{align}

The threshold for the energy of the single proton impacting on the target comes out to be \(7 m_p \approx \SI{6.6}{GeV}\).

This is counterintuitive! It is due to the fact that we lose a lot of energy when boosting back to the CoM frame. 

Let us go back to the \(2 \to 3\) reaction. 
We can always compute the invariant mass of a subsystem such as \(m_{\pi^{\pm}\Lambda }\): 
%
\begin{align}
m_{\pi^{\pm \Lambda }} = \sqrt{(E_{\pi^{\pm}} + E_\Lambda )^2 - \abs{\vec{p}_{\pi^{\pm }} + \vec{p}_\Lambda }^2}
\,,
\end{align}
%
which we can compute once we have the measurement from our detector. 

\todo[inline]{See ``dalitz-pi-lambda'' figure in the drive.}

We can look at the histogram of \(m_{\pi^{+}\Lambda }^2\) versus \(m_{\pi^{-}\Lambda }^2\). 

There are peaks! 
This suggests the presence of something decaying into \(\pi^{\pm }\Lambda \).
This particle is a \(\Sigma^{\pm}\), and what happened was 
%
\begin{align}
\ce{K}^{-} + p &\to \Sigma^{\pm} + \pi^{\mp}  \\
\Sigma^{\pm} &\to \pi^{\pm} + \Lambda 
\,.
\end{align}

We have not detected the \(\Sigma \) particle, which was short-lived. 

We know that there should be a range \(m_2 + m_2 \leq m_{23} \leq M - m_1 \). 

This can be extended to a constraint on the two invariant masses. 

This is called a \textbf{Dalitz plot}, a common trick to find peaks for new particles. 

The (true?) width of the peak is about \SI{36}{MeV}, which corresponds to \SI{1.8e-23}{s}. 
Is this a \(\Sigma (1385)\)? 

\todo[inline]{But we should also consider experimental error in the determination of the energy\dots}

An alternative way to measure lifetimes is to look at track length: the path length we expect is 
%
\begin{align}
\lambda = \beta c \tau \gamma 
\,,
\end{align}
%
where we need the Lorentz factor since the particle decays in time \(\tau \) in its own rest frame. 

In this case the track length would be about \SI{5.5}{fm}; however we also have the \(\gamma \) factor: if that is large enough we may get something measurable. 

We'd want to see at least a few points. 
If \(\sigma _x\) is our spatial resolution, we need a track length which is at least \(\gtrsim 5 \sigma _x\). 

The best spatial trackers are silicon detectors or emulsion trackers, on the order of \SI{10}{\micro m}. 
This means that we need to increase our path length by a factor \(\num{e10}\)\dots 
The \(\gamma \) factor is surely not that large: it would mean having a beam with energy on the order of \SI{e19}{eV}.

Let us consider another example: \(\ce{p} + \gamma \to \Delta^{+} \), which can decay into \(\pi^{+} + \ce{n}\) or \(\pi^{0} + \ce{p}\). 

What is the kinematic threshold for this reaction? 
This is a cosmic ray proton interacting with a CMB photon. 

These photons have an average energy of \(\SI{2.7}{K} \approx \SI{230}{\micro eV}\). 

This does not precisely match \(\Omega_{0 \gamma } \rho _c c^2\) when multiplied by \(n_{0 \gamma }\), since we'd need to integrate the Planckian. 

The value of \(\sqrt{s}\) is 
%
\begin{align}
\sqrt{s} &= \sqrt{m_p^2 + 2 \qty(E_p E_\gamma - \vec{p}_p \cdot \vec{p}_\gamma )}  \\
&= \sqrt{m_p^2 + 2 E_p E_\gamma \qty(1 - \cos \theta_{p \gamma })} \geq m_{\Delta^{+}}
\,.
\end{align}

This means that 
%
\begin{align}
E_p \geq \frac{m^2_{\Delta^{+}} - m_p^2}{2 E_\gamma \qty(1 - \cos \theta_{p \gamma })}
\,.
\end{align}

If the collision is head-on, we get \(1 - \cos = 2\), while in the other case the threshold diverges since the proton cannot reach the photon.

We should average the formula over all values of \(\theta_{p \gamma }\). 

The threshold is about \SI{e19}{eV}. 
This is the GZK cutoff, or GZK effect. 
In order to properly study this phenomenon we would need
to also look at the probability: \cite[]{groupReviewParticlePhysics2020} says it is of the order of \(\SI{e-1}{mb}\) for the \(\gamma p\) process. 

We can study this in the lab simply, we only need \SI{200}{MeV} photons colliding on stationary protons. 

With this, we can find that a typical proton above the threshold will travel for only tens of \SI{}{Mpc} before losing energy to this process. 

The products from protons being annihilated in this way will produce many secondary particles: photons, neutrinos, electrons, positrons, muons and more, all being very energetic. 

Let us do another possibility: could photons from a high-energy source be absorbed by the CMB? 
For example, you can have pair production. 

We need \(\sqrt{s}\) for the process to be larger or equal than about \SI{1}{MeV}.
This comes out to be about \SI{e14}{eV}. 

\end{document}
