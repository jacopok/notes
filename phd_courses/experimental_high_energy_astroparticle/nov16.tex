\documentclass[main.tex]{subfiles}
\begin{document}

\marginpar{Tuesday\\ 2021-11-16}

Last time we compute the threshold for reactions with a CMB photon interacting with some high-energy particles. 

We can plot the mean free path with varying energy. 
It is roughly infinite below a certain threshold (we can have some interactions with the tails of the distribution near it), then it asymptotes to \(\Lambda \sim \SI{100}{Mpc}\). 

Around \SI{5e19}{eV} the \(\ce{p} + \gamma _{\text{CMB}} \to \Delta^{\pm}\) reaction kicks in, while at around \SI{e14}{eV} we get  the \(\gamma + \gamma  _{\text{CMB}}\to e^{+} e^{-}\) reaction. 
There, however, the energy affects the cross-section of the reaction, so we expect to see a minimum. 

The all-particle energy spectrum has an ultra-high-energy cutoff around \SI{e20}{eV} because of these reactions. 

At LHC we have center-of-mass energies of \SI{14}{TeV}, which is found by adding together the energies of the bins.
This is true in this particular case since we have a symmetric configuration, meaning that the total momentum in the lab frame as well vanishes. 

In a fixed target context (such as a cosmic ray coming in) instead we get 
%
\begin{align}
\sqrt{s} = \sqrt{2 m (E + m)}
\,,
\end{align}
%
so the correct comparison reads 
%
\begin{align}
\sqrt{2 m (E +m)} = 2 E^{*}
\,,
\end{align}
%
where \(E^{*} = \SI{7}{TeV}\). Therefore, 
%
\begin{align}
E = \frac{(2 E^{*})^2}{2m} - m
\,.
\end{align}

It depends on \(E^{*}\) quadratically! 
In our case, with protons, we get \(E \sim \SI{e17}{eV}\).

How are the angles changed? 
The invariant mass of the total system is 
%
\begin{align}
m _{\text{inv}} = \sqrt{\qty(\sum _{i}E_i)^2 - \abs{ \sum _i p_i}^2}
\,.
\end{align}

We also know that \(\gamma = E / m\).
The total \(\gamma \) can be computed by looking at this expression with the total energy \(E _{\text{tot}}\) and \(\sqrt{s} = m _{\text{inv}}\) as the total mass. 

In the case of our cosmic rays, this reads 
%
\begin{align}
\gamma _{\text{CMS}} \sim \sqrt{ \frac{\SI{e17}{eV}}{\SI{2e9}{eV}}} \approx \SI{7e3}{}
\,,
\end{align}
%
which corresponds to an angle of \(1 / \gamma _{\text{CMS}}\sim \SI{30}{\arcsec}\). 

This is a good thing, in a way: instead of having to make a detector all around the event, we can make a smaller one in the forward direction only. 

We can define a new kinematic variable in order to better understand these angles. 

Consider a Lorentz transformation with a velocity \(v_0 = \beta_0 c\) along the \(x\) axis. 
The transformation for a velocity \(v\) reads 
%
\begin{align}
v = v' + v_0 
\,
\end{align}
%
in the NR case, while for the relativistic case it will be 
%
\begin{align}
v = \frac{v' + v}{1 + v' v_0 / c^2}
\,.
\end{align}

This only holds for the \(x\) component, along the boost.

Therefore, the velocities are not additive.
A trick is to introduce a new quantity, the \emph{rapidity}:
%
\begin{align}
y' = \frac{1}{2} \log \frac{1 + \beta '}{1 - \beta '} = \operatorname{arctanh} (\beta ')
\,,
\end{align}
%
and rapidities are indeed additive. 
These are typically considered only looking at the component parallel to the boost: 
%
\begin{align}
y = \operatorname{arctanh} (\beta_{\parallel})
\qquad \text{where} \qquad
\beta_{\parallel} = \beta \cos \theta = \frac{p_{\parallel}}{E}
\,.
\end{align}

This formulation holds in 3D as well. 
The rapidity can also be written as 
%
\begin{align}
y = \frac{1}{2} \log \qty(\frac{E + p_{\parallel}}{E - p _{\parallel}})
\,.
\end{align}

The Lorentz parameters can be written in terms of hyperbolic functions: 
%
\begin{align}
\beta &= \tanh(y)  \\
\gamma &= \cosh(y)  \\
\beta \gamma &= \sinh(y)
\,,
\end{align}
%
so that the Lorentz matrix reads 
%
\begin{align}
\left[\begin{array}{cc}
\gamma  & - \beta \gamma  \\ 
- \beta \gamma   &  \gamma 
\end{array}\right]
= \left[\begin{array}{cc}
\cosh(y) & -\sinh(y) \\ 
-\sinh(y) & \cosh(y)
\end{array}\right]
\,.
\end{align}

In the ultrarelativistic limit, the rapidity reads 
%
\begin{align}
y &\approx \log \sqrt{\frac{1 + \cos \theta }{1 - \cos \theta }} \\
&\approx \log \frac{\sqrt{(1 + \cos \theta ) / 2}}{\sqrt{( 1 - \cos \theta ) / 2}}  \\
&\approx - \log \tan(\theta / 2 ) = \eta 
\,,
\end{align}
%
where \(\eta \) is called the pseudo-rapidity. 

\todo[inline]{make figure}

At LHC we can only detect particles which are not emitted very forward or very backward. 

What about the rapidities of the particles for the case of a \SI{e17}{eV} cosmic ray? 
They are additive, so \(y _{\text{lab}} = y' + y_0 \), where 
%
\begin{align}
y_0 &= \frac{1}{2} \log \qty(\frac{1 + \beta _{\text{CMS}}}{1 - \beta _{\text{CMS}}})  \\
\beta _{\text{CMS}} &= \frac{\abs{p'}}{E+m} \approx \frac{E}{E+m} \\
y_0 &\approx \frac{1}{2} \log \qty(\frac{2E + m}{m}) \approx \frac{1}{2} \log \frac{2E}{m} 
\,.
\end{align}

Let us consider an energy \(E \approx \SI{e15}{eV}\). Then, \(y_0 \approx 7\). 

We can divide the CMS depending on the sign of \(\eta^{*}\),the pseudo-rapidity in the CMS. Then the total rapidity will read \(\eta = \eta^{*} + y_0 \). 

The boundary between these in the lab frame is found when \(\eta^{*} = 0\), so we get \(\eta = - \log \tan \theta /2\): \(\theta \) is about 6 arcminutes for \(\eta = 7\). 

For a collision \SI{20}{km} in the air, we expect about half of the particles to be found within a \SI{40}{m} radius. 

Collider experiments typically cover \(\abs{\eta } \lesssim 5\) (the figure refers to ATLAS);
this only covers the central region, we do not get all of them. 

The definition of the pseudo-rapidity is convenient because PDFs in the form \(\dv*{N}{\eta }\) are only shifted under Lorentz boosts. 

The \textbf{transverse mass} can also be defined: 
%
\begin{align}
E^2 = m^2 + p^2 = \underbrace{m^2 + p_T^2}_{m_T^2} + p^2_\parallel
\,,
\end{align}
%
and \(m_T^2 = m^2 + p_T^2\) defined as above is Lorentz invariant under boosts in the \(\parallel\) direction. 
We can also write this relation like 
%
\begin{align}
\qty( \frac{E}{m_T})^2 - \qty( \frac{p_\parallel}{m_T})^2 &= 1  \\
\cosh^2 y - \sinh^2 y &= 1  
\,.
\end{align}

The rapidity can also be expressed as 
%
\begin{align}
y = \frac{1}{2} \log \qty(\frac{E + p_\parallel}{m_T})
\,.
\end{align}

Exercise: suppose we have an incoming particle with mass \(M\) impacting upon a particle with mass \(m\); what is the maximum transfer of energy from particle \(M\) to particle \(m\), if the collision is elastic? 

The result should be 
%
\begin{align}
\Delta E = \frac{2 m \gamma^2 \beta^2}{1 + 2 \gamma m/M + (m/M)^2}
\,.
\end{align}

The case we are interested in will be a proton interacting with a medium and interacting with an electron. There, and for a relativistic proton, this will read \(\Delta E \approx 2 m \gamma^2\). 

Next time, we will discuss some matter-radiation interactions. 

\end{document}
