\documentclass[main.tex]{subfiles}
\begin{document}

\section{Matter-radiation interaction}

\marginpar{Tuesday\\ 2021-11-23}

The binding energy of electrons to atoms 
is typically of the order of \(1 \divisionsymbol \SI{10}{eV}\); 
the maximum change in energy of an electron struck by a high-energy particle is \(\Delta E \approx 2 m_e \beta^2 \gamma^2\). 

This is a lot for the electron, but a small amount compared to the initial energy of the high-energy particle. 

The trajectory of the large particle is also deflected but only by a small amount. 
The quantity we are interested in computing is the average change in energy per unit length: \(\dv*{E}{x}\). 

Suppose we have a beam of particles with a certain flux \(\Phi \), defined as 
%
\begin{align}
\Phi = \frac{\text{\# incoming particles}}{\text{time} \times \text{surface}}
\,.
\end{align}

After hitting a target, these particles will go along a trajectory defined by angles \(\theta \), \(\varphi \) from the interaction point. 

The differential amount of particles emitted in a certain direction, \(\Delta N_s (\theta , \varphi )\), will satisfy 
%
\begin{align}
\Delta \dot{N}_s (\theta , \varphi ) \propto \Phi 
\,,
\end{align}
%
so we define 
%
\begin{align}
\frac{\Delta \dot{N}_s}{\Delta \Omega } = \Phi \frac{\Delta \sigma }{\Delta \Omega }
\,,
\end{align}
%
where \(\Delta \Omega \) is a small solid angle. 
This allows us to compute the differential cross-section: 
%
\begin{align}
\dv{\sigma }{\Omega } =  \frac{1}{\Phi } \dv{\dot{N}_s(\theta , \varphi )}{\Omega }
\,.
\end{align}

We can further define the total, or integral cross-section: 
%
\begin{align}
\sigma = \int \frac{1}{\Phi } \dv{\dot{N}_s}{\Omega } \dd{\Omega }
\,.
\end{align}

This is all written with respect to a single target, but in practice we will have a certain slab of matter with thickness \(\delta x\), and a particle flux impacting on it within an area \(A\). 
In this case, then, we also need to account for the number of targets \(N_T\): 
%
\begin{align}
\dv{\dot{N}_s}{\Omega } = \Phi N_T \dv{\sigma }{ \Omega }
\,.
\end{align}

If \(n\) is the number density of the targets, and \(\rho \) is the matter density there, the number of targets will read \(N_T = n A \delta x\). 
With this, we find 
%
\begin{align}
\dv{\dot{N}_s}{\Omega } = \underbrace{\Phi A}_{\dot{N}_b} n \delta x \dv{\sigma }{\Omega }
\,,
\end{align}
%
where we can identify the incoming beam particle rate \(\dot{N}_b\). 
The definitions for the differential and total cross-sections will then read 
%
\begin{align}
\dv{\sigma }{\Omega } &= \frac{1}{\dot{N}_b} \frac{1}{n \delta x} \dv{\dot{N}_s}{\Omega } (\theta , \varphi ) \\
\sigma &= \frac{1}{\dot{N}_b} \frac{1}{n \delta x} \underbrace{\int  \dv{\dot{N}_s}{\Omega } (\theta , \varphi ) \dd{\Omega }}_{ = \dot{N}_s}
\,.
\end{align}

There is an alternative way to compute this in terms of QFT quantities, but we will not concern ourselves with it. 

If we know the cross-section for, say, electron-proton interactions, we can compute the number of scatterings as 
%
\begin{align}
\dot{N}_s = \dot{N}_b (n \dd{x}) \sigma 
\,.
\end{align}

These quantities all depend on \(n \delta x\), never on the number density or the thickness singularly. 

How does the intensity of a particle beam decrease as it travels through a medium? We know \(I (x= 0)\), and we want to compute the dependence \(I(x)\). 
In a certain thickness \(\Delta x\), we will have a certain number of  interactions \(\Delta N_s = N_b (n \Delta x) \sigma \), so we can define the probability of interaction 
%
\begin{align}
\frac{\Delta N_s}{N_b} = \Delta P = n \sigma \Delta x
\,.
\end{align}

The probability of \emph{not} having an interaction is \(1 - n \sigma \Delta x\). 
The survival probability \(P(x)\) is the probability of a particle surviving at least until \(x\). 
We can write a relation for \(P(x + \Delta x)\): 
%
\begin{align}
P(x + \Delta ) = P(x) (1 - n \sigma \Delta x)
\,,
\end{align}
%
since this means that in this \(\Delta x\) the particle still has not interacted. 
Then, 
%
\begin{align}
P(x + \Delta x) - P(x) &= \Delta P = - P(x) n \sigma \Delta x  \\
\dv{P}{x} &= - n \sigma P \\
P(x) &\propto \exp(- n \sigma x)
\,.
\end{align}

We can also compute the average \(x\) at which an interaction occurs: 
%
\begin{align}
\expval{x} = \frac{\int_{0}^{\infty } x \exp(- n \sigma x) \dd{x}}{\int_0^{\infty} \exp(- n \sigma x) \dd{x}} = \frac{1}{n \sigma } = \Lambda 
\,,
\end{align}
%
which can therefore be interpreted as the \emph{mean free path}.\footnote{The mean free path before the \emph{first} interaction - after that the particle may change, so this simple expression may not hold anymore.}

Protons coming into a medium with atoms may interact with the electrons through EM interactions, but also through with nucleons through hadronic interactions. 

Since these hadrons are very heavy, the deflection by hadronic interactions is much heavier. 

What is the order of magnitude for the cross-section of proton-proton, or neutron-proton interactions? 

Roughly, the radius of a nucleus is \(R(A) \sim r_0 A^{1/3}\), where \(r_0 \sim \SI{1.2}{fm}\).

A rough estimate of the geometric cross-section is \(\sigma \sim \SI{4}{fm^2}\). 
This can be written as \SI{40}{mb}, where \(\SI{1}{barn} = \SI{e-24}{cm^2}\). 

We can look at the proton-proton cross-section plot from the PDG \cite[]{groupReviewParticlePhysics2020}. 
There is some structure at low energies, but between 1 and \SI{100}{GeV} the cross-section is roughly \SI{40}{mb} indeed. 

At higher energies, roughly \(\sqrt{s} \sim \SI{10}{TeV}\), this increases up to \SI{100}{mb}.\footnote{See figure 52.6 in \url{http://pdg.ge.infn.it/2021/reviews/rpp2020-rev-cross-section-plots.pdf}.}
These numbers have an order of magnitude which is a good approximation for all hadronic interactions. 

A barn is a large cross-section! 

Let us consider a beam of \SI{5}{GeV} protons going toward a \SI{10}{cm} thick block of graphite. 
The total number of particles going to the target is called protons-on-target, and in this case it is \num{e16}. 

How many interactions will we have? 
First, we need the \(\sqrt{s}\) for the interaction, in order to compute \(\sigma \). 
It comes out to \(s = (E + m)^2 - \abs{\vec{p}}^2 = 2m^2 + 2Em \approx \SI{3.5}{GeV}\)
(the nucleons can be considered to be stationary: the order of magnitude of their kinetic energies is tens of \SI{}{MeV}, so their momenta are \(p^2 / 2m_p \sim \SI{250}{MeV}\)). 

What is \(n\) for graphite? The molar mass is \SI{12}{g / mol}, while \(\rho \approx \SI{2.2}{g/cm^3}\), so we get 
%
\begin{align}
n = N_A \frac{A}{\text{molar mass}} \rho \approx \SI{1.3e24}{cm^{-3}}
\,.
\end{align}

This means that 
%
\begin{align}
n \sigma \delta x = \frac{N _{\text{interactions}}}{N _{\text{protons on target}}}\approx \num{.53}
\,.
\end{align}

The mean free path is therefore roughly \SI{20}{cm}. 
The target size is typically chosen so that its length is roughly \(\Lambda \): this way, we often have one interaction but rarely two. 

What happens if, instead, we have a neutrino beam? 
The cross-section is directly proportional to the energy of the beam, with 
%
\begin{align}
\sigma_{\nu N} \sim \num{.6e-38} \qty(\frac{E}{\SI{}{GeV}}) \SI{}{cm^2}
\,.
\end{align}

At high energies the process is called \emph{deep inelastic scattering}. 

The charged current process for the interaction of a neutrino with a nucleus looks like 
%
\begin{align}
\nu_\mu + \ce{n} \to \ce{p} + \mu^{-}
\,.
\end{align}

Unless the energy is extremely high, this cross-section is extremely low. 
This means that the mean free path is much higher. 
What is the probability of a neutrino being able to cross the Earth? 

The density of the Earth is on average roughly \SI{5}{g / cm^3}. It changes a lot through its volume, from \(\sim \SI{2.6}{g / cm^3}\) to \(\lesssim \SI{15}{g/cm^3}\). 

For \(E = \SI{1}{GeV}\), we get roughly one thousand times the Earth-Moon distance, \(\Lambda = 1/  n \sigma \approx \SI{3e8}{km}\).

Thus, the Earth is almost transparent to neutrinos at these energies! 
For an energy of \SI{25}{TeV}, though, the Earth's diameter is close to the mean free path of the neutrino.

What about neutrions coming from pion decay, \(\pi^{+} \to \mu^{+} + \nu _\mu \) and then \(\mu^{+} \to e^{+} + \overline{\nu} _\mu  + \nu _e\), in the atmosphere? 

Typically, the order of magnitude of the energy of these secondary neutrinos is about a factor of 10 less than the energy of the proton initiating the shower. 

So, these can be very high-energy neutrinos, up to \SI{e19}{eV}, and their spectrum will roughly mirror the \(\sim E^{-3}\) powerlaw for cosmic rays. 

The flux of neutrinos produced from astrophysical sources, on the other hand, scales like \(E^{-2}\). 
At lower energies, therefore, we have atmospheric neutrino domination, while at high energies the astrophysical neutrinos dominate. 

The crossing point happens at energies of roughly \SI{1}{TeV}. 


\end{document}
