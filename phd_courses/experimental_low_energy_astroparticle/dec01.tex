\documentclass[main.tex]{subfiles}
\begin{document}

\subsection{Background}

\marginpar{Wednesday\\ 2021-12-1}

% Ferroni

It either comes from inside or outside the detector. 

The external background is mostly made of 
\begin{enumerate}
    \item \(\gamma \)s from natural decay chains;
    \item Radon;
    \item cosmic muons;
    \item neutrons.
\end{enumerate}

Very energetic muons can go through a few kilometers, so we can even go underground. 

The big problem are not the muons which go through the detector - 
those are easily rejected since their tracks are recognizable,
while the ones which pass nearby and interact are terrible! 

The internal background is mostly made of 
\begin{enumerate}
    \item cosmogenic events;
    \item bulk and surface materials;
    \item two-neutrino double beta decay.
\end{enumerate}

We can either insulate from the external background, or measure it so that 
we can reject some part of the data acquisition. 

Cosmic rays interact for the nuclei of the crystal which 
is being made for the detector.

The best level of purification which was reached, by the Borexino experiment, 
was of the order of \(\num{e-16}\). 

\emph{Irreducible background} is that which has the same physics
of the thing we are studying; in the \(0 \nu 2 \beta \) this is the \(2 \nu 2 \beta \) decay. 
The tail of that distribution will be very large! 

There are many reduction strategies. 
\begin{enumerate}
    \item High \(Q\)-value of the decay: this way, we are above the highest 
    \(\gamma \)s from naturally occurring radioisotopes;
    \item Improving the energy resolution;
    \item underground operation to stop cosmic rays;
    \item shielding;
    \item active veto;
    \item radiopure materials;
    \item particle identification;
    \item identification of daughter nuclei;
    \item minimization of exposure to cosmic rays. 
\end{enumerate}

The depth underground is measured in ``meters of water equivalent'', normalized 
to the average density of the material: LNGS is \SI{1.3}{km} deep,
but \SI{3.5}{km}-equivalent deep. 

One can either put an experiment inside a mine when digging a highway tunnel, 
or put it in an abandoned mine. 

Fermi got the Nobel prize for his technique of hitting elements with slow neutrons
in order to transmute them. 
He did this systematically for the periodic table, but when he hit Uranium
instead of finding element 93 he found Barium 
Actually, this was found by some chemists a bit later. 
Lise Meitner studied this while in Sweden, where she had escaped was since 
she was Jewish and Austrian. 

The Nobel prize was also given to the chemist Han Strassmann.  

Could we look for Barium? It would be the final product of \(0 \nu 2 \beta \)\dots

\todo[inline]{And not the product of \(2 \nu 2 \beta \)? }

The relevant ratio is 
%
\begin{align}
R_{0 \nu / 2 \nu } \propto \qty(\frac{Q_{\beta \beta }}{\Delta })^6 \frac{t_{2 \nu }^{1/2}}{t^{1/2}_{0 \nu }}
\,.
\end{align}

The width of the \(0 \nu \) peak is very small.
Having a very good energy resolution \(\Delta \) is especially crucial if we have an isotope with a short \(2 \nu \) lifetime. 

The positions of photomultipliers under the sea is monitored with sonar, and that sound signal is also used for listening to whales. 

After \SI{15}{km} of water-equivalent the muon flux stabilizes around \SI{e-13}{cm^{-2} sr^{-1} s^{-1}}, because of neutrino interactions. 

At LNGS, instead, we have about \SI{e-8}{cm^{-2} sr^{-1} s^{-1}}. This corresponds to about \SI{3e-4}{m^{-2} s^{-1}}, the neutron flux is \SI{3e-6}{cm^{-2}s^{-1}} for slow neutrons and \SI{.9e-6}{cm^{-2} s^{-1}}. 

It's a big number of neutrons! 
Fast neutrons can excite lead nuclear levels, which have energies similar to the \(Q\)-values of double \(\beta \) decay. 

Cosmogenic activation is a problem in which radioactive nuclei are produced by cosmic ray interaction while the detector is being built on the surface.

Some isotopes have half-lives of a few days or weeks so that's fine, but not all: \(^{60} \ce{Co}\) has about \SI{5}{yr}, longer than the typical lifespan of the detector. 

Besides the \(\gamma \)s we also have background from \(\alpha \)s: here, in the \SI{2.2}{MeV} to \SI{5}{MeV} range we have a continuum around \SI{.5}{KeV^{-1} kg^{-1}yr^{-1}} with a few peaks. 

There might be a Platinum line: crystals are grown in a rotating crucible made of Platinum. 
It's popular choice.
One can \emph{rent} it instead of buying it. 
It's better than Iridium and Graphite. 

\end{document}