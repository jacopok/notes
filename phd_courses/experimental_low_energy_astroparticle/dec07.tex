\documentclass[main.tex]{subfiles}
\begin{document}

\section{Bolometers}

\marginpar{Tuesday\\ 2021-12-7}

We need cryogenic detectors to be able to detect such small temperature increases. 

Thermistors are the most mature technology, but they have limited sensitivity. 

Transition edge sensor are operated at the edge of superconductivity. 
The readout from a TES can be accomplished with a Superconducting Quantum Interference Device. 

We look at COSINUS, at CRESST, at CDMS. 

Experiments are starting to give up the possibility to do particle identification 
and discrimination in order to lower the energy threshold  for the DM particle. 

Neganov-Luke effect: while electron-hole pairs are separated by the \(\vec{E}\) field, 
and they start moving, they can radiate some energy as phonons. 

With this, they reach an energy threshold of \SI{50}{eV}, by raising the potential
by a lot. As mentioned before, this means that particle identification cannot be done anymore. 



\end{document}
