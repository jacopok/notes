\documentclass[main.tex]{subfiles}
\begin{document}

\marginpar{Monday\\ 2021-12-13}

Directionality detection is a way to have a ``smoking gun'' DM detection. 

It allows a DM detector to double as a solar neutrino detector. 

The directionality of recoils is broader than the WIMP directionality distribution. 
Still, there is a strong asymmetry. 
We do not need a very good resolution: even 30 degrees of spread would be fine, since 
we only need to characterize the dipole and reject isotropy. 

This would allow us to go beyond the neutrino floor! 

People are getting audacious and thinking of going beyond the neutrino floor. 

The ``sausage'' feature in the phase space distribution of stars
in the Milky Way would yield a DM ``hurricane''. 

Supernovae could produce MeV-scale, relativistic dark matter.

SNe in the galactic center happen roughly twice a century, but 
the emitted DM has a velocity spread which means we receive it over several thousands
of years (\(\sim\) the distance to the galactic center); therefore
we expect a flux which is roughly constant in time. 

2D localization with directionality is better than 3D localization without directionality. 

Measuring localization is messed up by diffusion. 

Negative ion drift is a way to minimize drift by having electrons captured by 
electronegative gas ions (typically fluorine). 
This allows us to reduce the drift down to the thermal limit. 

There is a way to fiducialize detectors in this case, 
since typically different species are ionized, which drift at different velocities. 

\ce{SF6} is very heavy but non-toxic. 

The relevant parameter for Gas Electron Multipliers is \(E / P\), electric field
over pressure. 
These make micron-scale strong electric fields, to amplify signals. 

Directionality is important.

\end{document}
