\documentclass[main.tex]{subfiles}
\begin{document}

\marginpar{Tuesday\\ 2021-11-9, \\ compiled \\ \today}

This part of the course, ``LE-EXP2'' is held by Elisabetta Baracchini. 

\todo[inline]{Complete}

\paragraph{Cosmic Microwave Background}

\paragraph{Big Bang Nucleosynthesis}

BBN depends a lot on the balance between baryonic and non-baryonic matter.
With it, we can look at the Universe \emph{before} the CMB. 

Hydrogen is \SI{25}{\percent} of total matter ... ?
But this does not refer to Helium. 

Still, the estimated amount of baryonic matter from BBN is much lower 
than the total mass. 

\subsubsection{Modified Newtonian Dynamics}

The idea is that we have never tested Newtonian gravity in a very low-acceleration 
regime. 

MOND was developed to explain galactic rotation curves, but other 
phenomena are not well-explained by it. 
Right now, no one has made a MOND theory which can explain all the data. 

These lectures will focus on DM as a \emph{particle} which we might be able 
to detect. 

The proposal of DM as a WIMP has been strongly questioned. 

Dark Matter may:
\begin{enumerate}
    \item not exist;
    \item not be detectable --- only interact gravitationally, or have very suppressed non-gravitational interaction;
    \item interact with the atmosphere and therefore not reach the ground;
    \item be incredibly under-dense\dots
\end{enumerate}

However, the WIMP hypothesis is not completely ruled out.
There is a region in the parameter space we have not explored yet. 

Also, developing the instruments used to search for DM is useful for other areas of 
physics as well. 

\subsection{Dark Matter candidates}

The things we know about it (if it is a particle) are, roughly: 
\begin{enumerate}
    \item it is non-baryonic;
    \item it is dark (does not interact with electromagnetic radiation) and neutral;
    \item it is stable, or it has a lifetime which is long compared to the age of the universe;
    \item it is, at most, weakly interacting;
    \item it is either cold or warm, not hot;
    \item we have data about its abundance. 
\end{enumerate}

Are there any SM candidates? 
The obvious candidate is the neutrino; however given the limits we have on their mass 
we have put limits on their relic density (\(\Omega_\nu h^2 \lesssim \num{.07}\)). 

This is just one among the many known problems with the Standard Model. 

Therefore, we look for candidates beyond the standard model. 
There is a whole ``zoo'' of candidates. 
A convenient way to plot these is on a cross-section versus mass log-log plot.

Some interesting candidates are those which come from other fields: 
for example, the axion emerged as a solution for the strong CP problem. 

\subsection{Weakly interacting massive particles}

The assumption is that there was no dark matter asymmetry in the early Universe,
but two DM particle can annihilate into two SM particles. 

The freeze-out mechanism is crucial for the \emph{WIMP miracle}: 
The decoupling happens when we start to have \(\Gamma \lesssim H\). 

In the hot, early Universe we have thermal equilibrium between SM and DM particles;
then the Universe started to cool, and we only had decay of DM particles into SM ones;
then finally both channels decoupled. 

Changing the annihilation strength changes the resulting abundance. 
As we increase the annihilation strength, we decrease the resulting abundance.

There is a quantitative way to discuss this with the Boltzmann equation, the 
resulting equation is 
%
\begin{align}
\Omega_X \propto \frac{1}{\abs{\sigma v}} \sim \frac{m_X^2}{g_X^{4}}
\,.
\end{align}

The ``miracle'' is that to get the observed density we need 
to have an interaction on the order of the weak scale, on the order of 
\(\sim \SI{100}{GeV}\). 

The SM itself is an effective theory, and it needs new physics 
at the \(\sim \SI{}{TeV}\) scale. 
Supersymmetry provides DM candidates, as well as solving this problem. 

Supersymmetric models are many, and they have many parameters. 

We also have Universal Extra Dimensions theories. 
Here gravity propagates in \(3+1+n\) dimensions, reconciling
the difference between the electroweak and the Planck scale. 

The compactification of these dimensions happens in ``Kaluza-Klein towers''.
Here we also get DM candidates.

\end{document}
