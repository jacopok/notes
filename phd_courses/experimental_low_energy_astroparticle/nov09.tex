\documentclass[main.tex]{subfiles}
\begin{document}

\section{Direct dark matter searches and experimental challenges}

\marginpar{Tuesday\\ 2021-11-9, \\ compiled \\ \today}

This part of the course, ``LE-EXP2'' is held by Elisabetta Baracchini. 
% It is about direct dark matter searches and experimental challenges.

An archaic example of the search for something ``dark'' is given by 
Neptune and Vulcan. 

The presence of Neptune was theorized by Le Verrier in the 1800s 
thanks to its influence on the orbit of Uranus. 
In this case, the observed anomalies were indeed caused by something ``dark''.

On the other hand, Le Verrier also attributed the anomalies in Mercury's orbit
to a new inner solar system planet, Vulcan; however this was never observed, 
since the corrections were instead to be attributed to the precession induced by 
general-relativistic effects. 

The idea behind these anecdotes is that, \emph{a priori}, Dark Matter may very well
exist, but it may also be an effect of an incomplete theory of gravity: 
both have happened in the past. 

\subsection{Evidence for dark matter}

\paragraph{Galactic rotation curves}

The Keplerian velocity of test particles moving around a central mass 
(which is a good approximation for, say, the Solar system)
looks like \(v=  \sqrt{GM / r}\).\footnote{This can be easily computed, 
say, through the virial theorem: \(2T  +V = 0\), where \(T = mv^2 / 2\) 
and \(V = - GMm/r\).}

This is roughly what we would expect for the motion of stars at the edges 
of the galaxy, where little luminous mass is stored; instead of this \(v \propto r^{-1/2}\) 
decay we observe flat rotation curves, indicating the presence of large amounts of mass
even at the edges where the luminous mass fades. 

This is, in a sense, the most ``classic'' and oldest indication of the presence
of something we now call dark matter. 

\paragraph{Galaxy clusters}

We can give an estimate for the mass of a galaxy cluster by measuring its
\emph{velocity dispersion} and applying the virial theorem, which can be simplified to 
%
\begin{align}
\expval{v^2} \simeq GM \expval{ \frac{1}{r}}
\,.
\end{align}

This measurement of the mass can also be validated through other techniques: 
we can look at gravitational lensing, by which light is deflected by an angle 
%
\begin{align}
\Delta \phi = \frac{4GM}{bc^2}
\,
\end{align}
%
to first perturbation order in GR, where \(b\) is the impact parameter. 

Also, we can look at the X-ray emission by the gas in the cluster. 
This tells us that the average temperature is \(T \sim \SI{10}{keV}\). 
We can use this together with an assumption of hydrostatic equilibrium, which yields a relation in the form 
%
\begin{align}
k_B T \approx (1.3 \divisionsymbol 1.8)\SI{}{keV}
\qty(\frac{M_r}{\SI{e14}{M_{\odot}}}) 
\qty(\frac{\SI{1}{Mpc}}{r})
\,,
\end{align}
%
to estimate the mass. 

% \todo[inline]{\dots which we can use together with the virial theorem?}

Invisible gas is not the culprit: its mass can be estimated to account for around 
\SI{10}{\percent} of the total, while the observed ratio of virial mass to luminous 
mass is on the order of 300.

\todo[inline]{This number does not really seem to jibe with the 
\SI{5}{\percent} and \SI{25}{\percent} figures by Planck: is this a true mismatch 
or are we comparing things which should not be compared?}

\paragraph{Types of gravitational lensing}

This is a quick aside: 
\begin{enumerate}
    \item \emph{strong} lensing refers to the case in which a massive source 
    deflects the light from a source behind it, which we can see as a distortion or as an 
    Einstein ring or cross;
    \item \emph{weak} lensing refers to the combination of several minor lensing episodes
    in the path taken by the light from its source to us;
    \item \emph{micro} lensing refers to an episode of strong lensing in which the lens
    has a low mass, and there is relative motion which allows us to see a variation
    in the lensing. 
    This has applications in the search for exoplanets. 
\end{enumerate}

\paragraph{Mergers of superclusters}

Superclusters are the largest gravitationally bound systems we observe, and 
we have been able to observe their mergers. 
Here, we can tell that the visible matter is not aligned with the gravitating matter. 

\paragraph{Cosmic Microwave Background}

There are several effects impacting the multipolar decomposition of the CMB: 
the main ones are 
\begin{enumerate}
    \item the Sachs-Wolfe effect, in which radiation is red-shifted 
    by coming out of an over-dense region, so \(-\Delta T / T \sim \Delta \rho / \rho \);
    \item the Doppler effect, in which radiation is red-shifted depending on
    the velocity of the matter, so \(- \Delta T / T \sim \Delta \rho / \rho \);
    \item the Sunyaev-Zel'dovich effect, in which radiation is affected by 
    scattering on hot electrons
    \item the integrated Sachs-Wolfe effect, in which radiation goes in and out
    of a gravitational well, but because of the expansion of the universe it takes 
    longer to get out than it does to get in, resulting in an overall red-shift. 
\end{enumerate}

We can model the dependence of the peaks in the CMB spectrum on the presence of
baryonic and non-baryonic matter in the early universe, which is what allows us to get 
very tight constraints on these parameters.

\paragraph{Big Bang Nucleosynthesis}

BBN depends a lot on the balance between baryonic and non-baryonic matter.
With it, we can look at the Universe \emph{before} the CMB. 

\todo[inline]{Hydrogen is \SI{25}{\percent} of total matter ... ?
But this does not refer to Helium. 
ASKED}

Still, the estimated amount of baryonic matter from BBN is much lower 
than the total mass. 

\subsubsection{Modified Newtonian Dynamics}

The idea is that we have never tested Newtonian gravity in a very low-acceleration 
regime. 

MOND was developed to explain galactic rotation curves, but other 
phenomena are not well-explained by it. 
Right now, no one has made a MOND theory which can explain all the data. 

These lectures will focus on DM as a \emph{particle} which we might be able 
to detect. 

The proposal of DM as a WIMP has been strongly questioned, and also dark matter may:
\begin{enumerate}
    \item not exist;
    \item not be detectable, meaning that it only interacts gravitationally, or it has very suppressed non-gravitational interactions;
    \item interact with the upper atmosphere and therefore never reach the ground;
    \item be incredibly under-dense\dots
\end{enumerate}

However, the WIMP hypothesis is not completely ruled out.
There is a region in the parameter space we have not explored yet. 

Also, developing the instruments used to search for DM is useful for other areas of 
physics as well. 

\subsection{Dark Matter candidates}

The things we know about it (if it is a particle) are, roughly: 
\begin{enumerate}
    \item it is non-baryonic;
    \item it is dark (does not interact with electromagnetic radiation) and neutral;
    \item it is stable, or it has a lifetime which is long compared to the age of the universe;
    \item it is, at most, weakly interacting (meaning that its interaction cross-section is at most at the weak scale);
    \item it is either cold or warm, not hot;
    \item we have data about its abundance. 
\end{enumerate}

Are there any SM candidates? 
The obvious candidate is the neutrino; however given the limits we have on their mass 
we have put limits on their relic density (\(\Omega_\nu h^2 \lesssim \num{.07}\)). 

This is just one among the many known problems with the Standard Model. 

Therefore, we look for candidates beyond the standard model. 
There is a whole ``zoo'' of candidates. 
A convenient way to plot these is on a cross-section versus mass log-log plot.

Some interesting candidates are those which come from other fields: 
for example, the axion emerged as a solution for the strong CP problem. 

\subsection{Weakly interacting massive particles}

The assumption is that there was no dark matter asymmetry in the early Universe,
but two DM particle can annihilate into two SM particles. 

The freeze-out mechanism is crucial for the \emph{WIMP miracle}: 
The decoupling happens when we start to have \(\Gamma \lesssim H\). 

In the hot, early Universe we have thermal equilibrium between SM and DM particles;
then the Universe started to cool, and we only had decay of DM particles into SM ones;
then finally both channels decoupled. 

Changing the annihilation strength changes the resulting abundance. 
As we increase the annihilation strength, we decrease the resulting abundance.

There is a quantitative way to discuss this with the Boltzmann equation; the abundance of dark matter, as measured with the density rescaled by the entropy \(Y = n / s\), at infinity (so, now) reads 
%
\begin{align}
Y_{\infty } = \sqrt{\frac{45 G}{\pi g_*}} \frac{1}{T_F} \frac{1}{\expval{\sigma _{\text{ann}} v}}
\,,
\end{align}
%
where \(g_*\) is the effective number of degrees of freedom computed at freezeout, \(G\) is Newton's gravitational constant, \(T_F\) is the temperature at freezout, while \(\sigma _{\text{ann}}\) is the cross-section for the annihilation of these DM particles. 


The resulting abundance, as measured with \(\Omega _X = \rho _X / \rho _c\) (where \(\rho _c = 3 H^2 / (8 \pi G)\) is the critical density for the universe), is 
%
\begin{align}
\Omega_X \propto \frac{1}{\abs{\sigma v}} \sim \frac{m_X^2}{g_X^{4}}
\,.
\end{align}

The ``miracle'' is that to get the observed density we need 
to have an interaction on the order of the weak scale, on the order of 
\(\sim \SI{100}{GeV}\). 

The SM itself is an effective theory, and it needs new physics 
at the \(\sim \SI{}{TeV}\) scale. 
Supersymmetry provides DM candidates, as well as solving this problem. 

Supersymmetric models are many, and they have many parameters. 

We also have Universal Extra Dimensions theories. 
Here gravity propagates in \(3+1+n\) dimensions, reconciling
the difference between the electroweak and the Planck scale. 

The compactification of these dimensions happens in ``Kaluza-Klein towers''.
Here we also get DM candidates.

In the next lecture we will give a quick introduction to axion-like dark matter (Weakly Interacting Slim Particles), but a more complete overview will follow in the last lecture of the course. 

 

\end{document}
