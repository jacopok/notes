\documentclass[main.tex]{subfiles}
\begin{document}

\marginpar{Thursday\\ 2021-11-11}

The name ``axion'' comes from a brand of detergent. 
%
\begin{align}
\mathscr{L} _{\text{QCD}}
= - \frac{1}{4} G^{a}_{\mu \nu }
G^{a \mu \nu }
+ \sum _{q} 
\overline{q} \qty(i \gamma_{\mu } D^{\mu } - \mathcal{M}_q) q 
+
\underbrace{\frac{\theta g^2}{32 \pi^2} G^{a}_{\mu \nu }
\widetilde{G}^{a \mu \nu }}_{\text{CP - violating}}
\,.
\end{align}


This term would give an electric dipole moment to the neutron 
%
\begin{align}
\abs{d_n} = \frac{e m_u m_d}{(m_u + m_d) m_n^2} \overline{\theta}
\,,
\end{align}
%
where \(\overline{\theta} = \theta + \arg \det \mathcal{M}\), \(\mathcal{M}\) being the quark mass matrix. 

The CP-violating term is allowed in the QCD Lagrangian, but we do not experimentally observe it: \(\overline{\theta} < \num{5e-11}\) \cite{chadha-dayAxionDarkMatter2021} (see also \textcite[]{kolbEarlyUniverse1994}).

\todo[inline]{Computing this value does lead to the correct result: it is off by an order of magnitude. 
% m_u = 2.16 * u.MeV
% m_d = 4.67 * u.MeV
% (m_u * m_d / (m_u + m_d) / (ac.m_n * ac.c**2)**2 * ac.c * ac.hbar).cgs

\textcite[]{chadha-dayAxionDarkMatter2021} report \(\abs{d_n} = \num{3.6e-16} \overline{\theta} e \SI{}{cm}\), while this computation yields \num{3.3e-17}!
}

Why is this CP-violating phase so (``unnaturally'') small? 

The Peccei-Quinn mechanism promotes this phase to a dynamical field \(a\), with its own kinetic term \(\partial_{\mu } a \partial^{\mu } a  / 2\), 

If we define the dimensionless parameter \(\theta = a / f_a\), this is minimized for \(\theta _{\text{eff}} = \theta + \expval{a} / f_a\).

This field is driven to 0 under the spontaneous breaking of a new global \(U(1)\) symmetry.

If we assume the SM gauge group, the mass of this axion will be on the order of the \SI{}{\micro eV}. 

This axion may couple to photons, or to gluons, or to fermions. 
The photon coupling would allow for a \(a \to \gamma \gamma \) process.

\section{WIMP-like DM experimental detection}

What may DM couple to? 
\begin{enumerate}
    \item nuclear matter (quarks, gluons)?
    \item leptons (electrons, muons, taus, neutrinos)?
    \item photons, or other \(W\), \(Z\), \(h\) bosons?
    \item other dark particles?
\end{enumerate}

We don't know, so we try them all. 
We must do it in different contexts, both astrohysical and from particle physics. 

Consider the Feynman diagram for a process \(\chi \chi \leftrightarrow q q \), where \(q\) means ``quark'' but it could be substituted for any SM particle really. 
We can look at 
\begin{enumerate}
    \item efficient annihilation now: \(\chi \chi \to q q\), so indirect detection (in the sky);
    \item efficient scattering now: \(\chi q \to \chi q\) (underground);
    \item efficient production: \(q q \to \chi \chi \) (in particle colliders).
\end{enumerate}

In particle colliders a \(\chi \) may be produced, but it would be among a huge amount of other things. 
There, the particle physics people also must do trigger selection, so they may lose DM. 

Particle detectors measure the energy of the final state particles with calorimeters, as well as their momentum and trajectory with a tracking detector equipped with a magnetic field. 

An explanation of the characteristics of ATLAS and CMS.
Neutrinos are indirectly measured by the missing energy. 

\todo[inline]{Why are the uncertainties reported as \(\sigma (E) / E \sim x \% / \sqrt{E}\)?}

Conservation of energy does not really work, since the quarks are moving around inside the protons, while the conservation of transverse momentum can be used. 

Search paradigms include: 
\begin{enumerate}
    \item mono-X searches, a SM particle recoiling against nothing;
    \item mediator searches, a DM particle acting as a mediator, so it yields a bump in the mass spectrum of SM particle pairs;
    \item Higgs portal: if DM couples to it, a Higgs can decay into DM. 
\end{enumerate}

Several model-dependent bounds have been given, most of which are below \SI{1}{TeV}. 

Indirect DM searches involve DM particles annihilating into SM particles somewhere in the universe, and these SM particles are then detected by us.  

Photons: large background from astrophysical sources; protons/positrons: deviated by magnetic fields, and we don't know the background; neutrinos: too small cross-section\dots

\todo[inline]{
There is a section I did not understand about an ``invrse problem'': it seems to be about
the parameter space for the backgrounds in DM searches being very large, 
and thus able to accommodate any new observations.
}

\subsection{Direct DM searches}

Detect the DM bumping into something we can see. 

The Solar System is moving towards the Cygnus constellation. 
Therefore, we expect to see an apparent wind of DM in that direction.

% \todo[inline]{Cygnus is \emph{in our galaxy}, though!}

Our signal is WIMPs bumping into nuclei, with recoil velocity \(v/c \sim \num{7e-4}\), and recoil energy \(E_R \sim \SI{10}{keV}\).

Our background is both electromagnetic (photons bumping into nuclei) and neutral (neutrons or neutrinos bumping into nuclei). 

The expected rate is 
%
\begin{align}
R = N_N \phi_0 \sigma_{WN} = \frac{N_A}{A} \frac{\rho_0 }{m_W} \expval{v} \sigma_{WN}
\,,
\end{align}
%
where \(\rho_0 \approx \SI{.3}{GeV / cm^3}\) is the local DM density, the mean velocity is \(\expval{v} \approx \SI{220}{km/s} \approx \num{.75e-3} c\), but the cross-section is \(\sigma_{WN} \lesssim \SI{e-38}{cm^2}\).

Therefore, we get \(R \sim \num{0.13}\) events per kg per year. 

The interaction rate is very low, while backgrounds are very high.

A single banana, on the other hand, yields \(\sim \SI{100}{events/kg/s}\), or about a billion times more.

% \todo[inline]{Calcare is ``limestone''}

Environmental natural radioactivity is a mess. 
At LNGS we have more \(\gamma \)s than at Boulby in the UK, because that is a salt mine. 

\todo[inline]{Why is the neutron flux reported as single data points?}

We need to shield the detector from all possible backgrounds. 

The chain is \(\alpha \) easier to block then \(\beta \), then \(\gamma \) (where we need steel plates), then finally neutrons (for which we need a water tank). 

This is because hydrogen has the best kinematic match: same-mass moderators for neutrons are the best. 

Roman lead! the production process makes lead radioactive, 
but if it was produced 2000 years ago we are fine. 

Another approach is to have active shielding: have detectors which are sensitive to backgrounds around the detector, and then actively remove the background. 

\end{document}
