\documentclass[main.tex]{subfiles}
\begin{document}

\marginpar{Monday\\ 2021-11-15}

Using pure germanium spectrometers allows one to not have a radioactive detector. 

The bigger the detector, the more radioactivity one has to worry about. 

We see a PMT sample's spectrum from the XENON collaboration. 

We can have radiogenic or cosmogenic activation of the detector. 
Even if we go under the mountain, some cosmic ray flux remains, which can produce neutrons by spallation (this is cosmogenic). 
We can also have radiogenic activation: if there is a radioisotope left in our detector it can produce some new neutrons and \(\alpha \) particles. 

The typical lifetime of a detector is mostly spent trying to minimize its radioactivity. 

We also have RPRs, radon progeny recoils: radon has a decay chain producing both \(\alpha \) and \(\beta \) particles.
Radon is in the air, and the polonium it produces gets stuck to the detector. 

If the \(\alpha \) decay happens at the edge of the detector, the polonium can escape it, leaving only the \(\alpha \) particle inside it, which mimicks a DM signal. 

The way to avoid this problem is \emph{fiducialization}: using only signals coming from far away from the wall of the detector (in a ``fiducial region'').
In order to do this, we need 3D localization of each event. 

Another evil background is that given by neutrinos from the Sun. 
These typically come from the \(pp\) chain: 
%
\begin{align}
\ce{p} + \ce{p} &\to {}^{2}\ce{H} + e^{+} + \nu _e \\
\ce{p} + e^{-} + \ce{p} &\to {}^{2}\ce{H} + \nu _e 
\,,
\end{align}
%
and more.

Neutrino interactions will look just like the ones we are looking for, and there is no way to shield from them. 
We also have a diffuse SN neutrino background, as well as atmospheric neutrinos. 

Experiments are currently getting close to the sensitivity at which this becomes an irreducible background. 
This is called the \textbf{neutrino floor}. 

At \SI{}{GeV} to \SI{}{TeV} WIMP masses, the main contributions are neutrinos from Boron or Beryllium. 

One can look at the differences between the spectrum of energy from these neutrinos, but we need a lot of statistics for this. 

Another approach is to look at the directionality:
fortunately, Cygnus (from which we expect DM to be coming) never overlaps with the direction of the Sun. 

The event rate per unit energy reads 
%
\begin{align}
\dv{R}{E_R} = N_N \frac{\rho_0  }{ m_W} \int [v _{\text{min}} < \abs{\vec{v}} < v _{\text{max}}] \dd{\vec{v}} f(\vec{v}) v \dv{\sigma }{E_R}
\,,
\end{align}
%
where \(\rho_0 \) is the local DM density, while \(m_W\) is its mass; \(\vec{v}\) is the velocity of the DM, and we can compute 
%
\begin{align}
v _{\text{min}} = \sqrt{\frac{m_N E_R}{2 \mu^2_N}}
\,,
\end{align}
%
where \(\mu _N\) is the reduced mass of the interaction. 
This is the minimum velocity needed to produce a recoil with energy \(\geq E_R\). 

The local DM density and its velocity distribution come from astrophysics. 
We expect the DM halo to extend much further than the galactic disk. 

Exclusion limits are traditionally computed setting \(\rho_0 = \SI{.3}{GeV / cm^3}\), which is roughly the mean of our estimates, which are however rather uncertain, ranging from \num{.2} to \num{.56}.  

On average the halo is stationary, not corotating with the galactic center. 
Its velocity distribution is typically modelled as a Maxwellian, with \(\sigma \sim \sqrt{3/2} v_e\), where \(v_e\) is the escape velocity for the galaxy. 

We must also account for the motion of the Sun around the  galaxy (which is written in terms of a ``local standard of rest''), and the Earth's motion around the Sun. 

We have knowledge of large-scale DM distribution, but it could be non-smooth on the milliparsec scale! 
The Gaia survey suggested the presence of \emph{streams}\dots

The DM-nucleus cross-section \(\dv*{\sigma }{E_R}\) is not known. 
The order of magnitude of the momentum transfer can be used to estimate it, through \(\lambda \sim \hbar / p \gtrsim r_0 A^{1/3}\).

In an effective field theory approach, we can write a Lagrangian like 
%
\begin{align}
\mathscr{L} _{\text{eff}} = \frac{1}{\Lambda^2} \qty( \widetilde{\chi} \Gamma _{\text{dark}} \chi ) \qty( \widetilde{\psi} \Gamma _{\text{vis}} \psi )
\,.
\end{align}

We will have scalar-scalar, vector-vector interactions which are spin-independent, and enhanced by a factor \(A^2\)\dots

With these assumptions, we can write an expression like 
%
\begin{align}
\dv{\sigma }{E_r} &\sim \frac{2 m_N A^2 (f^{p})^2}{\pi v^2} F^2(E_R)  \\
F^2 (E_R) &= \qty(\frac{3 j_R q R_S}{q R_1 })^2 \exp(- q^2 s^2)
\,
\end{align}
%
for the spin-independent contribution. 

The spin-dependent contribution is proportional to \(J (J+1)\), which is nonzero only if there is an unpaired nucleon. 

The important parameter is \(A^2\) for the spin-dependent interaction. 

The final term is the kinematic one. 
The scattering is very much nonrelativistic, since \(m_W \sim \SI{10}{GeV} \divisionsymbol \SI{1}{TeV}\); the typical velocity of such a nucleus is \(v \sim \SI{220}{km/s} \sim c / 1400\). 

The recoil energy is therefore 
%
\begin{align}
E_R = \frac{p^2}{2 m_N} = \dots
\,.
\end{align}

The kinematics of the process means that heavier nuclei cross-sections are suppressed with respect to lighter ones as the energy rises, even though the curves are higher overall because of the \(A^2\) enhancement. 

This law would be a strong confirmation that what we are looking at is indeed a DM signal. 

The signal is a decaying exponential; but the background is also a decaying exponential. 
This poses a large problem for identification. 

A more robust signature would be the temporal dependence of the rate: our velocity with respect to the center of the galaxy is modulated as we go around the Sun. 
This will yield a law like 
%
\begin{align}
\dv{R}{E} (E, t) = S_0 (E) + S_m (E) \cos(\frac{2 \pi (t - t_0 )}{T})
\,.
\end{align}

DAMA claimed to have seen this kind of signal, but there might be another explanation. 

The rotation around the Earth's axis also changes the apparent direction of Cygnus. 

We can write a distribution depending on the direction of nuclear recoil: 
%
\begin{align}
\frac{ \dd{R}}{ \dd{E} \dd{\cos \gamma }}
\,.
\end{align}

This directional asymmetry is a tool to have a true, positive identification of dark matter. 

If we measure the event rate, we can give a plausible region of \(m_W\) and \(\dv*{\sigma }{E_R}\). 

In order to go to low DM mass we need light nuclei and a low threshold, in order to go to high DN mass we need heavy nuclei and large exposure (integration time). 

We can also use the material response to a signal to figure out what it is: we can detect \emph{charge} (ionization), \emph{light} (scintillation) and \emph{heat} (phonons). 

DM will scatter only once, neutrinos as well. 
Neutrons might scatter more than once, which allows one to throw them out. 

The distribution of energy between the three channels is a possible path for event discrimination. 

A nuclear recoil will mostly produce heat, while an electron will mostly yield ionization. 
The ratio between the visible energies is known as the Quenching Factor: 
%
\begin{align}
\text{QF} = \frac{E _{\text{visible}}(\SI{}{keV} ee)}{E _{\text{visible}} (\SI{}{keV} r)}
\,.
\end{align}

Experiments can be then classified in terms of the three detection channels. 

\end{document}
