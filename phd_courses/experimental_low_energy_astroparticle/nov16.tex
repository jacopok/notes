\documentclass[main.tex]{subfiles}
\begin{document}

\subsubsection{Scintillation-based detectors}

\marginpar{Tuesday\\ 2021-11-16}

DM experiments are classified by which signals they are sensitive to. 
We start with experiments able to detect light: scintillating crystals
and liquid noble-gas detectors. 

The idea is that the energy lost as \(\dv*{E}{x}\) is converted into 
visible light, which is then detected with photo sensors (such as photomultipliers). 
The human eye was also historically used. 

\todo[inline]{Is the claim ``the human eye is sensitive to a single photon'' true?}

We hope that our detector is efficient in the conversion of excitation energy to 
fluorescent radiation, that it is transparent enough to the emitted light 
(the Stokes shift is the one between the absorption peak and the emission peak,
we'd like this to be large). 

The decay time should also be short, so that we have a short response time. 

\paragraph{Photo multipliers}

The photomultiplier includes a photo cathode, where light releases an electron through 
the photoelectric effect; this electron is then accelerated towards a dynode and moves
forward to generate secondary emission on successive dynodes. 

The quantum efficiency is not very large: typically on the order of \SI{10}{\percent} to \SI{30}{\percent}. 

Silicon photomultipliers are arrays of avalanche photodiodes in Geiger-Muller regime: 
they include a p-n junction, which has a depletion region. 
In this depletion region charges are free and subjected to an electric field, 
therefore it can be used as a radiation detector. 

PMTs can be quite large, while SiPMs are typically small --- \SI{1}{cm} of diameter is 
already on the large side. 

A comparison of PMTs and SiPMs follows. 
\todo[inline]{Include this from the slides.}

SiPMs are quite noisy if they are not cooled. 

We measure the quantity 
%
\begin{align}
\text{PDE} = \text{FF} \times \text{QE} \times \text{AP}
\,,
\end{align}
%
the product of the Filling Factor, the Quantum Efficiency and the Avalanche Probability. 

\paragraph{Scintillators}

There is \emph{quenching}: highly ionizing particles create defects in the 
atoms or molecules in the medium, thus resulting in quenching. 

The emitted light per unit length reads 
%
\begin{align}
\dv{L}{x} = \frac{S \dv*{E}{x}}{1 + k B \dv*{E}{x}}
\,,
\end{align}
%
where \(k\) is the quenched fraction, while \(B\) is called the Birks constant. 
This is called Birks' law. 

In certain scintillators we have a two-exponential decay: 
%
\begin{align}
N(t) = A \exp( - \frac{t}{\tau _f}) + B \exp(- \frac{t}{\tau _s})
\,.
\end{align}

In principle we can do particle identification through \emph{pulse shape analysis},
figuring out \(A\) and \(B\) in the aforementioned formula. 

For DM searches, we typically use either liquefied noble gases or inorganic crystals. 
Liquefied noble gases have a high yield, a fast response. 
We can do pulse shape discrimination with them. 

However, the light they emit is typically UV. We need ways to shift their wavelengths
to detectable ones. 

They are very sensitive to impurities, even at the order of \num{e-6}. 
The best ones (like Xenon) emit several tens of photons per \SI{}{keV}. 

Electron recoils have sparse ionization, ion recoils have dense ones. 
The response to highly ionizing particles, such as alphas, is different
from the one to minimum ionizing particles such as electrons: 
they can also be discriminated through PSD, since they excite triplet and singlet
states in different proportions. 

We compare \(\alpha \) particles to electrons --- these are the things we can reliably
make in a lab setting, ideally we would like to measure the recoils of heavy nuclei,
but it's very hard to make these with reproducible, low energies. 

Still, the idea is that the response to these \(\alpha \)s is roughly similar to that from nuclei. 

The difference between triplet and singlet decays is the difference between fluorescence
and phosphorescence. 

What one can do is then just integrate the starting part of the signal and the end part: 
this ratio approximates the ratio of the exponential fit components, and 
allows for the discrimination of nuclear to electron recoils. 

These detectors are typically spherical, in order to have \(4 \pi \) coverage. 
Examples are XMASS at Kamioka, and DEAP-3600 at SNOLAB. 

This is the DEAP-CLEAN family. 
As the mass scale of these detectors grows above the ton, experimental efforts 
tend to merge. 

We look at DEAP-3600. In order to have a fiducial region, we go from \SI{3.6e3}{kg} to 
\SI{e3}{kg}. 

The argon used is radioactive! Contaminated with \(^{39}\ce{Ar}\). 
Thanks to pulse shape, they can suppress electron recoils by a factor \num{e10}. 

\todo[inline]{What is the wavelength shifter made of?}

The prompt fraction of the light is used as a discriminator. 
Nuclear recoils have a larger \(F _{\text{prompt}}\), of the order of \SI{70}{\percent},
while electron recoils have something like \SI{30}{\percent}. 

Typical efficiency values selected for nuclear recoils is of the order of \SI{50}{\percent}. 

Fiducialization is the process of reconstructing the direction the signal was coming from.
It can be time-based (measuring the arrival delays), or charge-based (measuring the signal amount).

Surface background, such as radon and polonium decays, also populate the signal region for \(F _{\text{prompt}}\), however there are other methods.

The effect of the sphere's neck is also relevant, and modelled by current detectors.

The number of photoelectrons detected can be directly mapped to the number of \SI{}{keV} in the recoil. 

They didn't find dark matter, sadly.

\subsubsection{Charge + light: double-phase detectors}

The scheme for the ionization of the noble gas is the same, but now we also look at the
electron which is ejected when the xenon is ionized. 
The light signal in the slow and fast channels is called S1, while the electron is called S2. 

The ratio of S1 to S2 allows for further discrimination. 

The light and charge contributions to the total energy are anti-correlated: 
if we add them together we reduce the fluctuation in energy by a lot.

We get energy resolutions of the order of \(\sigma / E \sim \SI{2}{\percent}\). 

These detectors are typically shaped like cylinders, as opposed to spheres. 
It's very difficult to make a uniform, radial \(\vec{E}\)-field in a sphere. 

Even in a cylinder geometry, a ``field cage'' is used, which prevents the field lines from exiting the cylinder. 
Instead of a single big voltage, we get a ladder of small voltages. 

The first signal, S1, is detected early and after a certain time the charge drifts and 
allows us to measure the stronger S2 signal. 

Electric charge does electroluminsescence and is detected by the same PMTs which detected S1. 



\end{document}