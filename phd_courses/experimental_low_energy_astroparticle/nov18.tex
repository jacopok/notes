\documentclass[main.tex]{subfiles}
\begin{document}

\section{Neutrinoless double beta decay}

\marginpar{Thursday\\ 2021-11-18}

This part of the course is given by Fernando Ferroni. 

In the seventies and eighties the Standard Models was created, 
putting quarks and leptons in separate families. 

We know why the photon is massless, in that it mediates an \(r^{-2}\) force. 
Neutrinos were thought to be massless (by very strong people, such as Glashow!). 

When Pauli postulated the neutrino, he thought it was massive! 
When they were making the SM, instead, 
they had moved to thinking that the neutrino was massless. 

Majorana's idea does not apply is the neutrino was massless.
In this course we will see how we might in principle prove or disprove Majorana's hypothesis. 

For the examination, he will do it jointly with Elisabetta:
a presentation about some technique discussed in each of the two courses.

\subsection{History of the neutrino}

In 1930, Pauli introduces an ``invisible'' neutral particle 
to resolve the issue of energy non-conservation in beta decay.

A two-body process imposes a definite energy to both products! 
So, with the observed energy spectrum there needed to be a light,
neutral particle which did not interact with the detector. 

A different hypothesis, by Bohr, was that energy conservation might 
not hold for each single process, but only on average. 

Pauli had a strong social life. 

\todo[inline]{Copy letter by Pauli}

He called these particles neutrons! 

The true neutron was discovered in 1932, two years later,
by Chadwick. 
He had \(\alpha \) particles hit beryllium atoms, which sent neutrons out,
which then hit the low-\(Z\) paraffin, emitting protons. 
This would not have worked with no paraffin, nor with a high-\(Z\) element. 

Chadwick did publish his results. 
Majorana said ``Idiots: they discovered a particle that they didn't understand''.

Fermi put these facts all together in 1934. 
After him, we know how to compute the decay time: 
%
\begin{align}
\lambda = \frac{2 \pi }{\hbar} \abs{M_{ij}}^2 p_f
\,.
\end{align}

The decay time depends on the ``form factor''. 
The matrix element is basically the overlap of the parent and daughter wavefunctions. 

The decay from which one computes the Fermi constant is muon decay: 
%
\begin{align}
\frac{1}{\tau _\mu } = \frac{G_F^2 m_\mu^{5}}{192 \pi^3} \qty(1 + \Delta q)
\,,
\end{align}
%
where one finds \(G_F \approx \SI{1.1e-5}{GeV^{-2}}\). 

However, all the cross-sections will scale like \(\sigma \approx G_F^2E^2\): 
this diverges at high energies! 
This is a problem since it violates unitarity. 
Therefore, Fermi's theory was known to be incomplete. 

The paper was initially rejected, as a ``solution to an irrelevant problem''. 

The solution to that issue was in the Standard Model: unifying EM and weak interactions, 
mixing photons with the other vector bosons. 

In 1935, it was hypothesized that two \(\beta \) decays would occur for the same nucleus: 
it is known that odd \(Z\) nuclei have more energy. 
So, it is possible that from an even nucleus we can go to another even one, 
passing through an odd one with a higher energy. 

This can happen, but it cannot be frequent: the second Fermi constant 
heavily suppresses the process. 

There are 35 isotopes which can undergo this process. 
This has a lifetime \(T_{1/2}  (2 \nu \beta \beta ) \sim (\num{e18} \divisionsymbol \num{e21}) \SI{}{yr}\),
however we can use many nuclei! 

In one mole, roughly \(A\) grams, of the nuclide we will have \(N_A\) isotopes, meaning \(\num{e2} \divisionsymbol \num{e6}\) decays per year.

The process \(^{Z} \ce{A} \to ^{Z+2} \ce{A} + 2 e^{-}\) is often energetically allowed, 
while \(^{Z} \ce{A} \to ^{Z+2}\ce{A} + 2 e^{-} + 2 \overline{\nu}_e\) is observed. 

Majorana wrote a new fundamental paper, ``Teoria simmetrica dell'elettrone e del positrone''. 
He didn't like Dirac's interpretation of a sea of negative energy states. 
He wanted to avoid the negative energy states.
In reality, his theory only apply to neutral particles. 

His theory, he said, allowed one to not require the existence of antineutrons and antineutrinos. 
(He was wrong about neutrons, since they are not fundamental particles). 

This would allow one to use the same neutrino for \(\beta^{\pm}\) decay. 

Racah replied quickly with ``Sulla simmetria tra particelle e antiparticelle'' (in Italian!).
A Majorana field could describe the situation in which particles and antiparticles coincide,
leading however to different physical predictions! 

When theorists know that something cannot be tested they do not pay it so much attention.
In 1939 W.\ Furry said that it can be shown that Majorana theory does not make different predictions
from Dirac theory in the case of regular \(\beta \) decay, while the prediction do differ in the case of 
double \(\beta \) decay! 

This is through the fact that in the process \(^{Z} \ce{A} \to ^{Z+2} \ce{A} + 2 e^{-}\) 
the neutrino/(anti)neutrino pair is only virtual. 

\todo[inline]{Add a loop diagram for \(0 \nu 2 \beta \)! Through a \(W^{+}\) boson.}

The issue here is that the neutrino must act as its own antiparticle. 

Dirac would say that \(\nu^{0}_R\) is mapped by a CPT transformation to a \(\overline{\nu}^{0}_L\). 

Again the story of the decay of the pion into electron being suppressed. 

After Furry's paper there was complete silence until the 90s. 
In all those times it was inconceivable that \(0 \nu  2 \beta \) would be measured.

A massless particle cannot flip its helicity, and the SM was built with massless neutrinos. 

However, neutrinos are massive, albeit with a very small mass. This was discovered by the Kamioka experiments, which found neutrino oscillations. 

At LNGS they saw some \(\tau \)s from a beam of muon neutrinos! This is also evidence of flavor mixing. 

The differences of the masses were thus measured: \(\Delta m^2 \sim \SI{}{eV}\). 
There is an impressive mass gap between these and all the other SM particles! 

If we have an experiment looking for proton decay, a significant background will be given by atmospheric electron neutrinos.

The processes will be \(\pi \to \mu \nu _\mu \), and then \(\mu \to e \nu _e \nu _\mu \).
The ratio of muon to electron neutrinos will be 2 to 1. 
But we do not measure 2! 
Also, the number changes if we look in different directions. 

This can be fit with a model including oscillations. 
Trying to do this for the purpose of removing the background for proton decay was the reason why people measured this stuff. 

One can write a Dirac mass term for the neutrinos. 
This term preserves lepton number. 

In a Majorana mass term, that symmetry is no longer enforced. 

A hybrid mass term is also allowed. 

A see-saw mechanism allows for three energy scales: \(m_D\) at the electroweak scale, \(m_L\) is approximately 0 (since we have \(2 \nu \beta \beta \)), and \(m_R\) is determined as \(m_D^2/ m_\nu \), so if \(m_\nu \sim \SI{50}{meV}\) we get \(m_R \sim \SI{e15}{GeV}\). 



\end{document}
