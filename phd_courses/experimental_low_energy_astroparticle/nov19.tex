\documentclass[main.tex]{subfiles}
\begin{document}

\marginpar{Friday\\ 2021-11-19}

How do we tell whether a neutrino is Dirac or Majorana? 
The SM has been built assuming a massless neutrino, 
and in it (anti)neutrinos are always (right)left-handed; 
however, since they have mass this cannot be exactly true. 

Majorana's hypothesis is as follows:
in their rest frame, the only distinction between neutrinos and antineutrinos
is their spin, not a Lorentz invariant concept.

This would imply a violation of \(L\) at the order \(m_\nu / p_\nu \). 
In most cases we observe ultrarelativistic neutrinos, for which this value is tiny. 

The weak interaction is \(V-A\), while Dirac conserves \(L\). 

How do we test this? we need a neutrino beam from the decay of, say, \(\pi^{+}\) or Kaons. 
We would like to have a pure neutrino beam without antineutrinos. 
We can deflect the charged particles away, but we have an issue: antineutrinos
will pollute our beam coming from neutral particles. 

The suppression of the \(\pi^{-} \to e^{-} \overline{\nu}_e\) over the muon decay
because of the necessity of a spin flip
is quantified by the matrix element \(M_{fi} \approx m / (m_\pi + m)\). 

For the difference between massive and massless neutrinos, we get 
%
\begin{align}
\frac{\Gamma_{m_\nu = 0}}{\Gamma_{m_\nu \neq 0}} \sim 1 + \frac{m_\nu ^2}{m_e^2} \dots
\,.
\end{align}

For the expected \(m_\nu \sim \SI{10}{meV}\) values we get an order \(\num{e-12}\) suppression. 

Another possibility would be measuring the magnetic dipole moment of the neutrino. 
This is due to the interaction \(\nu + \gamma \nu \) with a \(W^{+} + e^{-}\) loop; 
for a Dirac neutrino we get a nonzero moment, while for a Majorana the prediction 
is exactly zero. 


The prediction is 
%
\begin{align}
\mu^{D}_\nu 
\approx \num{3.2e-19} \frac{m_\nu}{\SI{1}{eV}} \mu _B
\,,
\end{align}
%
but the current experimental limits are \(\mu _\nu < \num{2.9e-11} \mu _B\) at a \SI{90}{\percent}. 

\todo[inline]{Make diagram!}

Even measuring this very precisely, if we still measure zero, means that 
we do not know anything yet. 

Since the neutrino mass is very small it is very hard to tell 
whether they are Dirac or Majorana.

What about \(0 \nu 2 \beta \)? What is the prediction for its lifetime? 
The nice thing is that this is a monochromatic process: 
the energies of the electrons add to the mass difference of the nuclei.

We do know \(2 \nu 2 \beta \) very well, it is an easy process to measure. 
We need a magnetic field, otherwise the two \(e^{-}\) would look the 
same as pair-production \(e^{+} e^{-}\). 

The values of \(E_1 + E_2\) for the two electrons are binned with \SI{100}{keV} bins. 
The end point for this should be where we see \(0 \nu 2 \beta \). 

From the Nemo measurements, in the Frejus tunnel, we see that the ratio of
the lifetimes needs to be at least a million. 

And, \(^{100} \ce{Mo}\) is the fastest of the isotopes doing \(2 \beta \)!

The lifetimes are written in terms of certain terms which are not fully known
%
\begin{align}
\Gamma^{2 \nu } &= G^{2 \nu } (Q_{\beta \beta }, Z) \abs{M^{2 \nu }}^2  \\
\Gamma^{0 \nu } &= G^{0 \nu } (Q, Z) \abs{M^{0 \nu }}^2 \frac{\abs{m_{ \beta \beta }^2}}{m_e^2}
\,.
\end{align}

What is \(m_{\beta \beta } \) in the suppression term? 
We cannot write ``neutrino mass'', since we do not know which neutrinos appear. 
We need to figure out which masses appear in the process. 

Also, even seeing this decay might also be an indication of new physics! 
(Tales we tell ourselves to have more hope)

Are the other terms beside the suppression factor in the lifetimes the same?
The phase space term \(G\) is computable, it's a bunch of integrals. 
One must integrate over all possible energies. 

The integral comes out to something that fluctuates a lot between isotopes, 
by several orders of magnitude, for the \(2 \nu \) case. 
These are of the order of \(G^{2 \nu } \sim \SI{e-21}{yr^{-1}}\).

The reason for this is that, to lowest order, the phase space term is \(G^{2 \nu } \propto Q_{\beta \beta }^{11}\), 
where \(Q\) is the \(Q\)-value of the decay. 

For the \(0 \nu \) case we have \(G^{0 \nu } \propto Q^{5}_{\beta \beta }\). 
These are typically of the order of \(G^{0 \nu } \sim \SI{e-15}{yr^{-1}}\). 

The nuclear matrix element is a mess.
Nobody has ever done a proper \emph{ab initio} calculation.

This must be done taking into account the possibility of passing by excited states. 
There are several models such as 
\begin{enumerate}
    \item the Nuclear Shell Model;
    \item the Quasiparticle Random Phase Approximation;
    \item the microscopic Interacting Boson Model (like Cooper pairs in superconductivity);
    \item the Interacting Shell Model. 
\end{enumerate}

These are hard, there are only few people who do this. 

These do not accurately reproduce the measured matrix elements: 
we should have accounted for the fact that the 
axial coupling constant \(g_A\) may be different from its value 
for a free neutron, \(g_A = 1.269\). 

Even normal \(\beta \) decay tells us that \(g_A\) varies. 
The coupling constant seems to scale like \(g \sim g_A^{(0)} A^{-0.18}\). 
But what will \(g_A\) for \(0 \nu 2 \beta \) be? 

This will not be a pure Gamow-Teller transition, we might have a Fermi (vector) 
term or a tensor one. 

We don't know what the \(g_A\) behaviour will be. 
The calculation methods do not really agree with each other, but they
are all within a factor of \(2 \divisionsymbol 3\). 

Considering all these factors, we can make some estimates: they are typically 
of the order of \(T_{1/2}^{0 \nu } \sim \SI{e24}{yr}\) for \(m_\nu = \SI{1}{eV}\). 

And this scales quadratically with \(1 / m_\nu \)! For \(m_\nu \sim \SI{10}{meV}\) we get \(\sim \SI{e28}{yr}\)\dots

At least the predictions for these isotopes tell us which isotopes to use. 

The last part, which is more well-known than the rest, are weak interactions. 

We know the \(\Delta m^2\) for different neutrino species, 
from atmospheric neutrinos and from solar neutrinos. 
One is on the order of \SI{2e-3}{eV^2}, the other \SI{7e-5}{eV^2}. 

Specifically, we care about the electron neutrino mass. Is it on the ``upper side'' 
or not? 

The crucial parameters for an oscillation are 
%
\begin{align}
1.267 \frac{\Delta m_{31}^2 L}{L_\nu }
\,.
\end{align}

So, measuring the oscillation at LNGS for a \SI{2}{GeV} beam 
or after only a \SI{}{km} at CERN for a \SI{3}{MeV} beam should be the same. 

\end{document}
