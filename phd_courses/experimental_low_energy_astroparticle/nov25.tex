\documentclass[main.tex]{subfiles}
\begin{document}

\subsubsection{Dual-phase noble-liquid detectors}

\marginpar{Thursday\\ 2021-11-25}

There is a liquid and a gaseous part: the
charges are extracted from the liquid with an electric field and 
then scintillate in the gas part. 

The time difference from the initial scintillation pulse in the
liquid region and the electron signal in the gas allows us to measure the \(z\) 
coordinate; combined with the measurement from all the 
sensors we can achieve a 3D localization. 

If we use a field of only a few \SI{}{kV/cm}, as opposed to several tens
of \SI{}{kV/cm}, instead of ionizing the atoms the electron just excites them.

We do not get an avalanche then, and we can get a better energy resolution. 

We can plot the number of scintillating photons produced per primary event
as a function of the electric field divided by the gas pressure. 
There is a linear regime, and then a dramatic rise. 
Between 4 and \SI{8}{kV/cm} there is a minimum in the energy resolution. 

We can discriminate electron recoils and nuclear recoils. 
We make a bivariate plot: the energy deposited in \(S_2 \) 
over that deposited in \(S_1 \). 

Argon detectors have better discrimination 
between electron recoils and nuclear recoils
than Xenon-based detectors.

With Xenon1T they reached less than \SI{100}{events / ton / yr / keV_{ee}}. 

The quenching factor is around \SI{20}{\percent}: \SI{5}{keV_{nr}} corresponds to \SI{1}{keV_{ee}}. 
Typically, higher atomic numbers correspond to higher quenching.

The detector must be kept very clean: even ppm-scale impurities would absorb the light. 
The detector is also surrounded by a water Cerenkov muon veto. 

Measuring both light and charge provides an absolute energy calibration: we can add them together. 

There is a background of radiogenic neutrons: an \(\alpha \) decay induces an \(\alpha \)-neutron event. 
The radon decay chain keeps creating problems.

Those events are spatially located at the border of the detector. 

They get very stringent limits in the high-mass region (\SI{100}{GeV} range).
They did see a few candidate WIMP events. 

\(S_2 \) only analyses: give up the non-amplified \(S_1 \) signal. 
Do only radial analysis, without considering the depth.
The gain is the ability to lower the energy threshold. 

Electron recoil analyses: there seemed to be an excess of electron recoil. 
The idea is to not look at nuclear recoil at all! 
DM may scatter on electrons as well. 

They just look at the energy spectrum of electron recoils.  
There is a \(\sim 3.5 \sigma \) excess at low (\(\sim \SI{}{keV}\)) energies. Is it tritium? 

\end{document}
