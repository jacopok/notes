\documentclass[main.tex]{subfiles}
\begin{document}

\marginpar{Monday\\ 2021-11-29}

The three coupling constants for electromagnetism, weak and strong theory are thought to reach similar values around \SI{e15}{eV}, which because of the seesaw mechanism 
corresponds to small energy scales. 

The three masses could be 
%
\begin{align}
m_1 &= m  \\
m_2 &= \sqrt{m^2 + \Delta m^2}  \\
m_3 &= \sqrt{m^2 + \Delta m^2 + \delta m^2 /2 }
\,,
\end{align}
%
or the inverse. 

The CKM matrix, defining the mixing of quarks, is rather close to diagonal, while the neutrino mixing matrix is quite ``democratic'', with all components having rather large values. 

The effective Majorana mass is 
%
\begin{align}
m_{ee} = \abs{\sum _{i=1, 2, 3} U^2_{\beta e} m_i}
\,,
\end{align}
%
but due to the phase terms cancellations can occur. 

We can make an exclusion plot, with the exclusion range varying the lightest neutrino mass. 
In the inverted hierarchy case the \(m_{ee}\) is around \SI{20}{meV}, in the normal hierarchy case it could be around \SI{2}{meV} or even less. 

There is this corner we cannot really eliminate. 
But, we have cosmological bounds for the sum of the neutrino masses. 
In a \(\Lambda CDM\) model one can include \(\sum _{\nu }m_\nu \) as a parameter. 

The half-life for \(0 \nu  2 \beta \) heavily depends on this  \(m_{ee}\) mass. 
In the inverted hierarchy case, we have \SI{40}{meV} which means \(T_{1/2} \sim \SI{2.5e26}{yr}\). 
In the normal hierarchy case this is much worse, and we \(T_{1/2} \sim \SI{2.5e28}{yr}\).

The normalized half-life for the \(0 \nu 2 \beta \) of \(^{100} \ce{Mo}\), instead, is \(T_{1/2} \sim \SI{4e23}{yr}\).

One can look at the original papers from Majorana, Goeppert-Mayer, Furry; the kinematics and nuclear matrix elements for the ones by Iachello, for the weak part the ones by Francesco Vissani. 

From a calorimetric point of view, the signal will look like a peak at \(Q = \SI{2.039}{MeV}\) in the energy spectrum (that \(Q\) value refers to germanium). 
On the other hand, the \(2 \nu 2 \beta \) decay spectrum is quite broad. 

What is quite important to distinguish this peak is the energy resolution of the detector. 

The risk of background from \(2 \nu 2 \beta \) may be mitigated by using a tracking particle detector: thin foils of emitter and then tracking gas. 
The problem, though, is that then we cannot have a large
mass of emitter. 

This, on the other hand, is a rather good technique to measure \(2 \nu 2 \beta \) itself.

The energy spectrum of a single electron is not very informative in the \(0 \nu 2 \beta \) case.
In the \(2 \nu 2 \beta \) case, though, it can tell us about nuclear transitions. 

The tracking detectors do not really have good energy resolution. 

If the electrons' energy is too low, then they are not detected. 

The factors going into the number of detections are the half-life \(T^{0 \nu }_{1/2}\), the number of available nuclei \(N _{\text{nuclei}}\), the duration of the experiment \(t\), and the efficiency of the detector \(\epsilon \): 
%
\begin{align}
N _{\text{detections}} \approx \epsilon N _{\text{nuclei}} (1 - 2^{- t / T_{1/2}^{0 \nu }}) \approx \epsilon N _{\text{nuclei}} \qty( \frac{t \log 2}{T_{1/2}^{0 \nu }})
\,.
\end{align}

The time for which we can run the experiment is limited by the human factor; but also the background increases nonlinearly with the time. 
Therefore, the SNR increases with \(\sqrt{t}\), as opposed to linearly with \(t\). 

If the calorimeter were to coincide with the emitter, it could allow us to reach a very high efficiency. 

We can assume that the number of nuclei scales with the mass, but that is not necessarily true! 
We will often have different isotopes in the same sample; actually, it is often the case that the interesting \(0 \nu 2 \beta \) isotope is not even the majority of the sample. 

The solution to this problem is \emph{isotopic enrichment}. 
How does that work? 

There are two methods (let us discuss it for Uranium): the very painful one is to take the Uranium and put it in a cyclotron; since \(\vec{B}\) is constant particles with different energies are separated. 

The painful one is to make a Uranium gas (by making the chemical compound Uranium Hexafluoride), put it in a centrifuge which will separate by mass density. 

The centrifuge will spin at a few thousand turns per second. 
Each centrifuge has an awful efficiency, and the process must be repeated many times. 
The cost in terms of electricity is very high. 

The cost is about \(100 \SI{}{\text{€}/g}\). 
This is due to both the electricity and the chemistry. 

There are cases in which we can easily make a molecule: Xenon is already a gas! 
Therefore, the chemistry cost for Xenon is zero. 

The case in which we save on electricity, on the other hand, is when we have a better isotopic abundance initially. 

Suppose we have bought some \SI{95}{\%} enriched germanium. 
Now we have about \SI{e27}{atoms} within only \SI{125}{kg}.
It will take a while: there are not that many centrifuges for scientific isotope enrichment. 

Germanium has a very good energy resolution. 
However, the phase space for germanium is very bad. 

Tellurium oxide, \(\ce{TeO2}\), has a quite good isotopic abundance. 

Isotopically pure detectors, however, are much better in terms of background. 

\(\beta^{+} \beta^{+}\) decay does not have many good candidates, but it would be nice since it produces \(4 \gamma \) through annihilation. 
These are an interesting signature.
The best candidate seems so be Rutenium, but its transition energy is quite low. 

The three things we'd want are an abundance of \(\gtrsim \SI{15}{\%}\), a \(Q_{\beta \beta } \gtrsim \SI{2.7}{MeV}\), a long lifetime of \(2 \nu 2 \beta \). 

There seems to be a bit of a correlation between the \(\mathcal{M}^2_{0 \nu }\) and the specific \(\mathcal{G}_{0 \nu }\). 

Isotopes basically have the same decay rate per unit mass, at any given value of \(m_{\beta \beta }\). 

The ``black line'' for \(Q_{\beta \beta }\) at around \SI{2.6}{MeV} marks the end of Earth's radioactivity. 
The decay of Uranium and Thorium, in the background, yields two decay chains. 
There is a photon at \SI{2.6}{MeV} from \(^{208} \ce{Tl}\), and below it there are a lot of other things.
So, we try to find \(0 \nu 2 \beta \) candidates which decay at above these energies.

In a calorimeter the signature from a single photon and two \(\beta \)s is the same. 



\end{document}
