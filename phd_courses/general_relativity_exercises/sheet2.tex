\documentclass[main.tex]{subfiles}
\begin{document}

\section{Sheet 2}

\subsection{Exercise 1}

The motion of a photon in Schwarzschild spacetime can be described, 
assuming without of generality it lies in the \(\theta = \pi / 2\) plane, 
by the normalization of its 4-velocity %
\begin{align} \label{eq:photon-velocity-normalization}
0 = u^2 = - \left(1 - \frac{2M}{r}\right) \dot{t}^2 + \left(1 - \frac{2M}{r}\right)^{-1} \dot{r}^2 + r^2 \dot{\varphi}^2
\,,
\end{align}
%
where dots denote derivatives with respect to an affine parameter \(\lambda\). 

We can use two Killing vectors of this spacetime, \(\partial_t\) and \(\partial_\varphi\), 
to compute integrals of motion: %
\begin{align}
e &= - u \cdot \partial_t = \left(1 - \frac{2M}{r}\right) \dot{t}  \\
l &= u \cdot \partial_\varphi = r^2 \dot{\varphi}
\,.
\end{align}

These are conserved across the photon's motion, therefore it is convenient to use them to 
eliminate the functions \(t(\lambda )\) and \(\varphi (\lambda )\) in equation \eqref{eq:photon-velocity-normalization}, 
so that we get a differential equation for \(r(\lambda )\) only, parametrized by \(e\) and \(l\).

The computation goes as follows; let us denote \(A =1 - 2M/r\) for brevity: 
%
\begin{align}
0 &= - A \left( \frac{e}{A}\right)^2 + \frac{\dot{r}^2}{A} + r^2 \left( \frac{l}{r^2}\right)^2   \\
0 &= - \frac{e^2}{A} + \frac{\dot{r}^2}{A} + \frac{l^2}{r^2}  \\
e^2 &= \dot{r}^2 + \frac{l^2}{r^2} A \\
\underbrace{\frac{\dot{r}^2}{2}}_{\text{kinetic}} 
\underbrace{+ \frac{l^2}{2r^2} - \frac{l^2M}{r^3}}_{\text{potential}} &= \underbrace{\frac{e^2}{2}}_{\text{total}}
\,.
\end{align}

We found something in the form of an energy conservation equation, 
with a kinetic term \(\dot{r}^2 / 2\) and an effective potential. 

For large \(r\) the potential will approach 0 from above, like \(1/r^2\), 
while for small \(r\) it will approach negative infinity like \(- 1/ r^3\).
Therefore, it will have a maximum, which is easily computed to lie at \(r = 3M\). 
There, the value of the potential \(V _{\text{eff}}\) is 
%
\begin{align}
l^2 \left( \frac{1}{2 (3M)^2} - \frac{M}{(3M)^3}\right) 
= \frac{l^2}{M^2} \left( \frac{1}{18} - \frac{1}{27} \right) = \frac{l^2}{54M^2}
\,.
\end{align}
%

For photons starting at \(\mathscr{I}^-\), 
this potential barrier might be insurmountable depending on \(M\), \(l\) and \(e\): 
the condition to check for insourmountability is whether 
%
\begin{align}
\frac{e^2}{2} < \frac{l^2}{54 M^2}
\,,
\end{align}
%
which means %
\begin{align}
\frac{l^2}{e^2} > 27M^2
\,.
\end{align}

What is the physical meaning of the ratio \(l/e\)?
it can be shown to correspond 
(up to a \(\pm\) sign, which depends on whether the motion is clockwise or counterclockwise) 
to the impact parameter \(b\), 
the distance between the asymptotic trajectory of the photon and the 
center of the black hole. 
The proof goes as follows: %
\begin{align}
\frac{l}{e} = r^2 \frac{\dot{\varphi}}{\dot{t}} = r^2 \dv{\varphi}{t}
\,,
\end{align}
%
which we can compute at radial infinity: %
\begin{align}
\dv{\varphi}{t} = \dv{\varphi}{r} \underbrace{\dv{r}{t}}_{= -1 } = 
- \dv{r} \arcsin (b / r) \approx - \dv{r} ( \pm b/r) = \mp \frac{b}{r^2}
\,,
\end{align}
%
therefore \(l/e = \mp b\) --- the linear order approximation is justified since
we are only interested in the asymptotic behaviour.

Therefore, photons with an impact parameter \(b\) larger than \(\sqrt{27} M \approx 5.2M\)
will not be able to surmount the potential barrier and "bounce off" it, 
going back to radial infinity, while the ones with smaller \(b\) 
will cross the potential barrier, moving further towards the horizon and then the singularity.

The \(b = \sqrt{27}M\) case is degenerate, as it describes three distinct trajectories: 
photons approaching from \(\mathscr{I}^-\), which spiral forever, 
asymptotically approaching \(r = 3M\); photons stuck in an unstable 
orbit at \(r = 3M\); and finally photons which reach the singularity at
some finite time after having orbited since past infinity. 

\subsection{Exercise 2}
\subsection{Exercise 3}
\subsection{Exercise 4}
\subsection{Exercise 5}

\end{document}