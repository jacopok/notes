\documentclass[main.tex]{subfiles}
\begin{document}

\section{Sheet 2}

\subsection{Exercise 1}

The motion of a photon in Schwarzschild spacetime can be described, 
assuming without of generality it lies in the \(\theta = \pi / 2\) plane, 
by the normalization of its 4-velocity %
\begin{align} \label{eq:photon-velocity-normalization}
0 = u^2 = - \left(1 - \frac{2M}{r}\right) \dot{t}^2 + \left(1 - \frac{2M}{r}\right)^{-1} \dot{r}^2 + r^2 \dot{\varphi}^2
\,,
\end{align}
%
where dots denote derivatives with respect to an affine parameter \(\lambda\). 

We can use two Killing vectors of this spacetime, \(\partial_t\) and \(\partial_\varphi\), 
to compute integrals of motion: %
\begin{align}
e &= - u \cdot \partial_t = \left(1 - \frac{2M}{r}\right) \dot{t}  \\
l &= u \cdot \partial_\varphi = r^2 \dot{\varphi}
\,.
\end{align}

These are conserved across the photon's motion, therefore it is convenient to use them to 
eliminate the functions \(t(\lambda )\) and \(\varphi (\lambda )\) in equation \eqref{eq:photon-velocity-normalization}, 
so that we get a differential equation for \(r(\lambda )\) only, parametrized by \(e\) and \(l\).

The computation goes as follows; let us denote \(A =1 - 2M/r\) for brevity: 
%
\begin{align}
0 &= - A \left( \frac{e}{A}\right)^2 + \frac{\dot{r}^2}{A} + r^2 \left( \frac{l}{r^2}\right)^2   \\
0 &= - \frac{e^2}{A} + \frac{\dot{r}^2}{A} + \frac{l^2}{r^2}  \\
e^2 &= \dot{r}^2 + \frac{l^2}{r^2} A \\
\underbrace{\frac{\dot{r}^2}{2}}_{\text{kinetic}} 
\underbrace{+ \frac{l^2}{2r^2} - \frac{l^2M}{r^3}}_{\text{potential}} &= \underbrace{\frac{e^2}{2}}_{\text{total}}
\,.
\end{align}

We found something in the form of an energy conservation equation, 
with a kinetic term \(\dot{r}^2 / 2\) and an effective potential. 

For large \(r\) the potential will approach 0 from above, like \(1/r^2\), 
while for small \(r\) it will approach negative infinity like \(- 1/ r^3\).
Therefore, it will have a maximum, which is easily computed to lie at \(r = 3M\). 
There, the value of the potential \(V _{\text{eff}}\) is 
%
\begin{align}
l^2 \left( \frac{1}{2 (3M)^2} - \frac{M}{(3M)^3}\right) 
= \frac{l^2}{M^2} \left( \frac{1}{18} - \frac{1}{27} \right) = \frac{l^2}{54M^2}
\,.
\end{align}
%

For photons starting at \(\mathscr{I}^-\), 
this potential barrier might be insurmountable depending on \(M\), \(l\) and \(e\): 
the condition to check for insourmountability is whether 
%
\begin{align}
\frac{e^2}{2} < \frac{l^2}{54 M^2}
\,,
\end{align}
%
which means %
\begin{align}
\frac{l^2}{e^2} > 27M^2
\,.
\end{align}

What is the physical meaning of the ratio \(l/e\)?
it can be shown to correspond 
(up to a \(\pm\) sign, which depends on whether the motion is clockwise or counterclockwise) 
to the impact parameter \(b\), 
the distance between the asymptotic trajectory of the photon and the 
center of the black hole. 
The proof goes as follows: %
\begin{align}
\frac{l}{e} = r^2 \frac{\dot{\varphi}}{\dot{t}} = r^2 \dv{\varphi}{t}
\,,
\end{align}
%
which we can compute at radial infinity: %
\begin{align}
\dv{\varphi}{t} = \dv{\varphi}{r} \underbrace{\dv{r}{t}}_{= -1 } = 
- \dv{r} \arcsin (b / r) \approx - \dv{r} ( \pm b/r) = \mp \frac{b}{r^2}
\,,
\end{align}
%
therefore \(l/e = \mp b\) --- the linear order approximation is justified since
we are only interested in the asymptotic behaviour.

Therefore, photons with an impact parameter \(b\) larger than \(\sqrt{27} M \approx 5.2M\)
will not be able to surmount the potential barrier and "bounce off" it, 
going back to radial infinity, while the ones with smaller \(b\) 
will cross the potential barrier, moving further towards the horizon and then the singularity.

The \(b = \sqrt{27}M\) case is degenerate, as it describes three distinct trajectories: 
photons approaching from \(\mathscr{I}^-\), which spiral forever, 
asymptotically approaching \(r = 3M\); photons stuck in an unstable 
orbit at \(r = 3M\); and finally photons which reach the singularity at
some finite time after having orbited since past infinity. 

\subsection{Exercise 2}

We discuss negative-mass Schwarzschild spacetime, and define a radial coordinate %
\begin{align}
\rho = r + 2M \log (1 - r / 2M) = r - 2 \abs{M} \log (1 + r / 2 \abs{M})
\,.
\end{align}

The \(r \to \infty \) limit corresponds to \(\rho \to \infty\); 
at \(r = -2M\) we have \(\rho = -2M(1 - \log 2) \approx - 0.6 M\) (\(>0\)), 
and finally for \(r \to 0^+\) we get \(\rho \to 0^+\). 
Thus, the range of \(\rho\) is \((0, + \infty )\) just like \(r\)'s, 
although their relation is nonlinear. 

The differential of \(\rho\) will read %
\begin{align}
\dd{\rho } &= \pdv{\rho }{r} \dd{r} = \left( 1 + \frac{2M}{1 - r / 2M} \left(- 2M\right)\right) \dd{r}  \\
&= \left( 1 - \frac{2M}{2M - r} \right) \dd{r} = - \frac{r \dd{r}}{2M - r}
\,,
\end{align}
%
therefore %
\begin{align}
\dd{\rho}^2 = \left(\frac{r}{r-2M}\right)^2 \dd{r}^2 = \left(\frac{r}{r + 2 \abs{M}}\right)^2 \dd{r}^2
\,.
\end{align}

The metric will read as follows, in terms of \(\rho\) (and denoting \(r = r(\rho )\)): 
%
\begin{align}
\dd{s^2} &= - \left( 1 - \frac{2M}{r}\right) \dd{t^2} + \frac{1}{(1 - 2M/r) (r/(r-2M))^2} \dd{\rho^2} 
+ r^2 \dd{\Omega^2}  \\
&= - \left( 1 - \frac{2M}{r}\right) \dd{t^2} + \left( 1 - \frac{2M}{r}\right) \dd{\rho^2} 
+ r^2 \dd{\Omega^2}  \\
&= \underbrace{\left(1 + \frac{2 \abs{M}}{r} \right)}_{>0} \left(- \dd{t}^2 + \dd{\rho }^2\right) + r^2 \dd{\Omega}^2
\,.
\end{align}

We can see that the \((t, \rho)\) section of the metric is conformally related 
to flat spacetime, since the prefactor in front of it is always positive. 

The angular section is not the same as the one of the corresponding flat spacetime, 
since the prefactor in front of it is 
%
\begin{align}
\frac{r^2}{1 - 2M/r} \neq \rho^2
\,,
\end{align}
%
but this only means that we cannot interpret \(\rho\) as the radius defined 
by the area of a sphere: \(\rho \neq \sqrt{A / 4 \pi }\).

The conformal diagram for this spacetime will therefore look the same as Minkowski, 
with the caveat that the prefactor diverges for \(r \to 0\): so, the 
\(r = 0 \) line will not be a reflective boundary in this case. 
Indeed, just like in positive-mass Schwarzschild the Kretschmann scalar diverges there, 
so it is a singularity.



\subsection{Exercise 3}
\subsection{Exercise 4}

The Kerr metric in Boyer-Lindquist coordinates reads %
\begin{align}
\mathrm{d}s^2 = - \frac{\Delta}{\rho^2} 
\left(\mathrm{d}t^2 - a \sin^2 \theta \mathrm{d}\varphi \right)^2
+ \frac{\sin^2 \theta}{\rho^2}
\left( (r^2 + a^2) \mathrm{d}\varphi - a \mathrm{d}t\right)^2
+ \frac{\rho^2}{\Delta } \mathrm{d}r^2 + \rho^2 \mathrm{d}\theta^2
\,,
\end{align}
%
where \(\Delta = r^2 - 2Mr + a^2\) and \(\rho = r^2 + a^2 \cos \theta \).

We have the Killing vectors \(\partial_t\) and \(\partial_\varphi \), which determine two integrals of motion of the trajectory we can associate with energy and angular momentum respectively. 

The angle \(\theta\) is measured from the ``north pole'', so the equatorial plane corresponds to \(\theta = \pi / 2\). Since \(\dot{\theta} = 0\) initially, this will hold for the whole trajectory (by the \(\theta \to \pi - \theta \), \(\varphi \to - \varphi \) symmetry of the metric): therefore, we can use the reduced equatorial metric %
\begin{align}
\mathrm{d}s^2 &= - \frac{\Delta}{\rho^2} 
\left(\mathrm{d}t^2 - a \mathrm{d}\varphi \right)^2
+ \frac{1}{\rho^2}
\left( \rho^2 \mathrm{d}\varphi - a \mathrm{d}t\right)^2
+ \frac{\rho^2}{\Delta } \mathrm{d}r^2   \\
&= 
- \frac{\Delta}{\rho^2} \dd{t^2} 
+ 2 \frac{a \Delta }{\rho^2} \dd{t} \dd{\varphi } 
- \frac{a^2\Delta }{\rho^2} \dd{\varphi^2} 
+ \rho^2 \dd{\varphi^2} 
- 2a \dd{\varphi } \dd{t} + \frac{a^2}{\rho^2} \dd{t}^2 + \frac{\rho^2}{\Delta } \dd{r^2}  \\
&= - \left( \frac{a^2 - \Delta }{\rho^2}\right) \dd{t^2} 
+ 2a \left( \frac{\Delta}{\rho^2} - 1 \right) \dd{t} \dd{\varphi } 
+ \left(\rho^2 - \frac{a^2 \Delta }{\rho^2}\right) \dd{\varphi^2} + \frac{\rho^2}{\Delta } \dd{r^2}
\,,
\end{align}
%
with \(\rho^2 = r^2 + a^2\). 

Denoting the components of the four-velocity \(u^\mu = (\dot{t}, \dot{r}, 0, \dot{\varphi})\), the angular momentum will read %
\begin{align}
l &= u \cdot \partial_\varphi = g_{\varphi \varphi } \dot{\varphi} + g_{t \varphi } \dot{t}  \\
&= \left(\rho^2 - \frac{a^2 \Delta }{\rho^2}\right) \dot{\varphi}
+ a  \left( \frac{\Delta}{\rho^2} - 1 \right) \dot{t}
\,.
\end{align}

We are assuming that this quantity is zero at the start of the trajectory (which means it remains so thereafter). 
This allows us to directly relate \(\dot{t}\) and \(\dot{\varphi}\) at each moment of the trajectory, as long as we know \(r\).
An alternative way to write this is as %
\begin{align}
\dv{\varphi }{t} = \frac{a (  \rho^2 - \Delta )}{\rho^4 - a^2 \Delta } 
= \frac{2Ma}{r^3 + ra^2 + 2Ma^2}
\,.
\end{align}

The physical meaning of this is computation is to figure out by how much the particle is ``dragged along'' by the black hole's rotation. 
As expected, this quantity is concordant in sign with \(a\), and always positive; further, it never diverges, not even for \(r \to 0\). 



\subsection{Exercise 5}
\subsection{Exercise 6}
\subsection{Exercise 7}

\end{document}