\documentclass[main.tex]{subfiles}
\begin{document}

\subsection{No-stress source}

\marginpar{Monday\\ 2021-4-26, \\ compiled \\ \today}

We considered a source in the form \(T_{\mu \nu } = \rho t_\mu t_\nu \), where \(t^{\mu } = (1, \vec{0})\). 

Now we will consider a source in the form 
%
\begin{align}
T_{\mu \nu } = - 2 \rho t_\mu t_\nu + 2 J_{(\mu } t_{\nu )}
\,,
\end{align}
%
where \(J^{\mu } = \rho u^{\mu } = \rho (\gamma , \gamma v^{i} / c)\). 

\todo[inline]{Probably the first 2 is not there.}

The static source from before can be recovered from this expression in the low-velocity limit \(v^{i} / c \to 0\). 
In that case, \(T_{ij} = 0\): we can see that \(T_{ij}\) is of order \(v^2 / c^2\), so to first order they vanish. 

In this situation, we get the system 
%
\begin{align}
\begin{cases}
    \square \overline{h}_{0 \mu } &= - 16 \pi T_{0 \mu } \\
    \square \overline{h}_{ij} &= 0
\,.
\end{cases}
\end{align}

In order to simplify, let us assume that \(\partial_{t} \overline{h}_{ij} = 0\): then, the solution to the second of these becomes \(\nabla^2 \overline{h}_{ij} = 0\), with flat boundary conditions at large distance. 
By linearity, this leads to \(\overline{h}_{ij} = 0\).

\begin{claim}
If we define \(A_{\mu } = - (1/4) \overline{h}_{0 \mu } = - (1/4) \overline{h}_{\mu \nu } t^{\nu }\), then the metric becomes 
%
\begin{align}
g_{00} &= - 1 + 2 A_0  \\
g_{0i} &= 4 A_i  \\
g_{ij} &= (1 + 2 A_0) \delta_{ij}
\,.
\end{align}
\end{claim}

In terms of this \(A_{\mu }\), the Dalambertian equation from before reads 
%
\begin{align}
\square A_{\mu } = - \frac{16}{4} \pi J_\mu = - 4 \pi J_\mu 
\,,
\end{align}
%
which are formally identical to the Maxwell equations! 
Therefore, we can employ known techniques from electromagnetism. 

For example, if \(\partial_{t} A_{\mu } = 0\) then 
%
\begin{align}
\begin{cases}
    A_0 &= - \phi  \\
    A_{i} &= \int \dd[3]{x^{i}} \frac{J_i}{\abs{x - x^{i}}}
\,,
\end{cases}
\end{align}
%
which is the reason why the phenomena which can be described through this formalism are known as gravito-electric and gravito-magnetic effects. 

\begin{claim}
For example, geodesics in a weak-field stationary (no stress) spacetime are described by a Lagrangian 
%
\begin{align}
\mathscr{L} &= - mc \qty(- g_{\mu \nu } \dot{x}^{\mu } \dot{x}^{\nu })^{1/2}  \\
&= - mc^2 \qty(- g_{00} - 2 g_{0i} \frac{v^{i}}{c} - g_{ij} \frac{v^{i} v^{j}}{c^2})^{1/2}  \\
&\approx -mc^2 + \frac{m}{2} v^2 + m \phi + 4 mc A_{i}v^{i}
\,.
\end{align}
\end{claim}

We have a mass term, a kinetic term, a gravitational term, and a contribution to the Lorentz force. 

The corresponding equations of motion read 
%
\begin{align}
\ddot{\vec{x}} = \vec{E} + 4 \vec{v} \times \vec{B}
\,,
\end{align}
%
where \(\vec{E}\) and \(\vec{B}\) are the gravitoelectric and gravitomagnetic fields derived from our \(A_{\mu }\). 
The differences from EM are: the absence of charge, and the factor of 4 before the magnetic term.

An example of a gravito-electromagnetic effect is the Lense-Thirring effect: a magnetic moment \(\vec{s}\) in a magnetic field precesses, according to 
%
\begin{align}
\dv{\vec{s}}{t} &= \vec{s} \times \vec{\Omega}
\qquad \text{where} \qquad
\vec{\Omega} = - \frac{q}{m} \vec{B}_{EM}
\,,
\end{align}
%
so in order to generalize to the precession of a gyroscope in an EM field we need to map \(q \to m\) and \(\vec{B}_{EM} \to 4 \vec{B}\). 

This way, we see for example that \(\Omega _g = - 4 B\). 
A mission called Gravity Probe B measured this effect: they found precession with \(\Omega _g \sim \SI{.22}{arcsec / yr} (R_{\oplus} / r)^3\). 
This is a \SI{20}{\percent} accurate test of GR in the weak field. 

\todo[inline]{What does that mean?}

Another example is \textbf{frame dragging}, which applies in full GR: if we put the gyroscope around a BH a similar effect emerges. 
Around a Kerr BH we have 
%
\begin{align}
g_{0i}^{\text{Kerr}} \sim \Omega_{BH} 
\,,
\end{align}
%
and if the particle is close to the BH a particle is ``locked'' to the BH rotation. 

\section{Gravitational Waves in linear GR}

GW are solutions of weak-field GR in a vacuum.
There, the wave equation reads \(0 = \square _\eta \overline{h}_{\mu \nu }\). What are the properties of the solutions of these equations? 
The simplest thing we can do is look for plane wave solutions. We take a wave vector \(k^{\mu } = (\omega , k^{i})\) and an amplitude \(A_{\mu \nu }\); then 
%
\begin{align}
\overline{h}_{\mu \nu } = A_{\mu \nu } e^{i k_\mu x^{\mu }} = A_{\mu \nu } e^{i (- \omega t + \vec{k} \cdot \vec{x}) }
\,,
\end{align}
%
so \(\partial_{\mu } \overline{h}_{\alpha \beta } =  (i k_\mu ) \overline{h}_{\alpha \beta }\). 

Substituting the plane wave ansatz yields 
%
\begin{align}
0 = \square \overline{h}_{\alpha \beta } = - \eta^{\mu \nu } k_\mu k_\nu \overline{h}_{\alpha \beta }
\,,
\end{align}
%
therefore \(k_{\mu }k^{\mu } = 0\). The wavevector is null. 

This implies that the GW propagates at the speed of light: \(\omega s^2 = \abs{\vec{k}}^2\). 

How do we completely specify a gauge?
Any infinitesimal transformation such that \(\square \xi^{\mu } = 0\) preserves the Hilbert gauge, so we can make a residual gauge transformation. 

The harmonic gauge implies that 
%
\begin{align}
0 = - \partial^{\alpha } \overline{h}_{\mu \alpha } = i k^{\alpha } \overline{h}_{\mu \alpha }
\,,
\end{align}
%
which yields \(k^{\alpha } A_{\alpha \mu } = 0\). This means that GWs are \textbf{transverse} to the propagation direction. 

We know that \(\overline{h}_{\mu \nu }\) maps to \(\overline{h}_{\mu \nu } + 2 \partial_{(\mu } \xi_{\nu )} + \eta_{\mu \nu } \partial_{\alpha } \xi^{\alpha }\). 

Let us use \(\xi^{\mu } = B^{\mu } e^{i k_{\alpha } x^{ \alpha }}\) as an ansatz for our residual gauge transformation, since it automatically harmonic: we get 
%
\begin{align}
A_{\mu \nu } \to A_{\mu \nu } - 2 i k_{(\mu } B_{\nu )} + i \eta_{\mu \nu } k_{\alpha } B^{\alpha }
\,,
\end{align}
%
and since we can pick \(B^{\mu }\) arbitrarily we can impose \(\overline{h} = A^{\mu }_{\mu } = 0\), the \textbf{traceless condition}, as well as \(\overline{h}_{\mu 0} = 0\), the \textbf{transverse condition}.
The second is suggested by the previously found result \(k_\alpha A^{\alpha \beta } = 0\). 

In terms of \(B\), this is a linear algebraic system, and it is invertible.

In summary, we start from 10 variables, we use 4 equations to impose the Hilbert gauge, and 4 more to impose the TT gauge. 
The two degrees of freedom which are left are the true degrees of freedom of a GW. 

More explicitly, if we have \(k^{\mu } = (\omega, 0, 0, k_z)\) this means 
\begin{enumerate}
    \item \(k^2 =0 \) implies \(- \omega = k_z\);
    \item the phase reads \(k_\alpha x^{\alpha } = \omega (t - z)\);
    \item the Hilbert gauge \(k^{\mu } A_{\mu \nu } = 0\) tells us that \(A_{0 \nu } = A_{3 \nu }\); 
    \item the transverse condition tells us that \(A_{0 \mu } = 0\) (so also \(A_{3 \mu } = 0\));
    \item the traceless condition tells us that \(A^{\mu }_{\mu } = 0\).
\end{enumerate}

This leads to the usual formulation 
%
\begin{align}
A_{\mu \nu }^{TT} = \left[\begin{array}{cccc}
0 & 0 & 0 & 0 \\ 
0 & A_+ & A_\times & 0 \\ 
0 & A_\times  & - A_+ & 0 \\ 
0 & 0 & 0 & 0
\end{array}\right]
\,.
\end{align}

Therefore, 
%
\begin{align}
h^{TT}_{\mu \nu } = A_{\mu \nu }^{TT} \exp(i \omega (t - z))
\,.
\end{align}

In TT gauge we have \(\overline{h}_{\mu \nu } = h_{\mu \nu }\) since the trace is zero.
Importantly, the TT gauge can only be defined in vacuo! This is because in that case \(\square \overline{h}_{\mu \nu } \neq 0\), so while we can still exploit gauge freedom we cannot set components to zero inside the source. 

The metric in TT gauge reads 
%
\begin{align}
g &= - \dd{t^2} + \dd{z^2} + (1 + h_+) \dd{x^2} (1 - h_\times ) \dd{y^2} + 2 h_\times \dd{x} \dd{y}
g &= - \dd{t^2} + (\delta_{ij} + h^{TT}_{ij} ) \dd{x^{i}} \dd{x^{j}}
\,.
\end{align}

How do we identify the GW degrees of freedom in general? 
We can impose the TT gauge outside the source (far away from the \(T_{\mu \nu }\)). 

In general, 
%
\begin{align}
h^{TT}_{\mu \nu } = \Lambda_{\mu \nu }{}^{\alpha \beta } \overline{h}_{\alpha \beta }
\,,
\end{align}
%
where \(\Lambda \) is a projection operator, defined as 
%
\begin{align}
\Lambda_{\mu \nu }{}^{\alpha \beta } &= P_{\mu }^{\alpha } P_\nu^{\beta } - \frac{1}{2} P_{\mu \nu } P^{\alpha \beta }  \\
P_{\mu \nu } &= \delta_{\mu \nu } - n_\mu n_\nu 
\,,
\end{align}
%
where \(n^\mu \) is the propagation direction. 

The projection tensor \(P_{\mu \nu }\) is symmetric, it is transverse (\(P_{\mu \nu } n^{\nu } = 0\)), it is idempotent (\(P_{\mu \alpha } P_{\alpha \nu } = P_{\mu \nu }\)), and its trace is equal to \(2\). 

The tensor \(\Lambda_{\mu \nu \alpha \beta } \) is also idempotent, transverse in all indices, traceless in \(\mu \nu \) and \(\alpha \beta \) separately, and symmetric in the swap of \(\mu \nu \) and \(\alpha \beta \). 

In summary, we have found GW solutions, they propagate with \(c\), they are transverse, they have two degrees of freedom. 

Symmetric, Transverse, Trace-Free tensors play an important role in GW theory. They can be used to obtain the \textbf{Multipolar expansion}. 

``Living review of relativity'' (see webpage) describes all the tests of GR. 

\end{document}
