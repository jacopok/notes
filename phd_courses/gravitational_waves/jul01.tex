\documentclass[main.tex]{subfiles}
\begin{document}

\section{Experiments and data analysis}

\marginpar{Thursday\\ 2021-7-1, \\ compiled \\ \today}

We start with a rather historical introduction. 

The first indirect proof for GWs was through pulsars time of arrival, for example the Hulse-Taylor pulsar, through the decay of the pulsar period. 

Resonant bars were also tried, but they are only sensitive to narrow frequency ranges. Each bar can only observe, roughly, a frequency \(\omega \sim c / L _{\text{bar}}\). 
The noise curve only has a dip at a specific frequency. 

Now we are using interferometers: the basic design is the Michelson interferometer, consisting of a laser source, a beam-splitter, two mirrors / test masses, and a photodetector. 

We work at the dark fringe: at equilibrium the photo-diode receives vanishing power. 
The observed signal is \(h \propto \Delta L / L = (L_y - L_x) / L\). 

Construction of ground-based interferometers started in the 1980s, and first light was in 2014 for LIGO. 

Why is it useful to be at the dark fringe? 
On the \(x\) axis we have the photodiode axis, on \(y\) is the power. We see a sort of parabola, and being at the minimum is convenient. 

In 2020 KAGRA joined the team. 
The planned detectors are Cosmic Explorer (from the US), with \(L \sim \SI{30}{km}\), Einstein Telescope (from Europe) with \(L \sim \SI{10}{km}\) and a triangular shape, Laser Interferometric Space Antenna (US + mostly EU) with  \(L \gg \SI{e3}{km}\) (space-based). 

\subsection{Noise in the IFO}

InterFerometric Observatory. 

We start by accounting for seismic motion and gravity gradients. 
We must find a way to distinguish from Earth-made gravity gradients and astrophysical ones. 

In Virgo there are inverse-pendulum suspensions: imagine a pendulum whose fulcrum oscillates with a frequency \(\omega_e \), and which has an intrinsic frequency \(\omega _{\text{proper}}\). 

If the external frequency is much larger than the proper frequency of the pendulum, then the external vibration will be damped. 

Typically, \(S_n^{\text{seism}} (f) \sim f^{-4}\). 

As for thermal noise: the mirrors can be kept at cryogenic temperatures, but condensation can be a problem in this regard. KAGRA implements this. 

Overall, the power spectrum looks like \(S_n^{\text{therm}}(f) \sim f^{-2}\). 

Finally, we have quantum noise: the intrinsic quantum fluctuations of the mirror. 
Radiation pressure looks like \(S_n^{\text{rp}} \sim f^{-4}\), while shot noise is roughly constant, with \(S_n^{\text{shot}} \sim 1\). 

In reality, there are other known noise contributions which must be taken into account. 
Examples are the AC grid, at \SI{50}{Hz}, the sea, and transients like thunderstorms. 

There's lots of noise at low frequencies, say larger than \(\SI{20}{Hz}\). 
At high frequencies we start to see \(\sim f\) noise. 

There is a plateau of sensitivity between \SI{20}{Hz} and \SI{500}{Hz}. 

We check for non-astrophysical transients by looking at coincidences between different detectors, and through matched-filtering. 
In some regards this problem is still not fully solved. 

These interferometers are built in peculiar places. 

\end{document}
