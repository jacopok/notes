\documentclass[main.tex]{subfiles}
\begin{document}

\marginpar{Monday\\ 2021-7-12, \\ compiled \\ \today}

Lecture with Matteo Breschi. 
Last week of class; three classes are scheduled but the topics can fit into two lectures. 

Do we do one extra class for discussion? Extra projects? 
Do we skip it? No to the latter. 

\section{Matched filtering}

We come back to our measured strain \(d(t)= h(t) + n(t)\), signal plus noise.

 
We want something quantifying the signal power. 
So, we define 
%
\begin{align}
S = \int_{- \infty }^{\infty } d(t) K(t) \dd{t}
\,,
\end{align}
%
where \(K(t)\) is an unknown \emph{optimal filter}. 

The noise is defined as the standard deviation of the data assuming \(h(t) = 0\):
%
\begin{align}
N^2 = \qty[ \expval{s^2} - \expval{s}^2]_{h = 0} = \expval{s^2}_{h=0}
\,,
\end{align}
%
since \(\expval{n} = 0\). 

This is then calculated as 
%
\begin{align}
N^2 &= \expval{\int n(t) n(t') K(t) K(t') \dd{t} \dd{t'}}  \\
&= \int \expval{n(t) n(t')} K(t) K(t') \dd{t} \dd{t'}  \\
&= \int \dd{t} \dd{t'} K(t) K(t') \int e^{2 \pi i (f - f')} \underbrace{\expval{\widetilde{n}^{*} (f) \widetilde{n}(f')}}_{(1/2) S_n(f) \delta (f- f')} \dd{f} \dd{f'}  \\
&= \frac{1}{2} \int S_n(f) \abs{\widetilde{K}(f)}^2 \dd{t} 
\,,
\end{align}
%
since, by design, \(\expval{K(t)} = K(t)\), as well as the fact that 
%
\begin{align}
S = \int d(t) K(t) \dd{t} = \int \widetilde{d}^{*} (f) \widetilde{K}(f) \dd{f}
\,.
\end{align}

What we can now do is define the \emph{signal-to-noise ratio}, which quantifies the strength of the signal with respect to the background noise.
Among the possible filters, the one we want maximizes the SNR: the SNR is defined, in the \(h \neq 0\) case, as
%
\begin{align}
\rho^2 &= \frac{\expval{S}^2}{N^2} =
 \frac{\qty(\int \widetilde{h}^{*} K \dd{f})^2}{ \frac{1}{2} \int S_n \abs{\widetilde{K}}^2 \dd{f}}
  \\
 &= \frac{ \qty( \int (\widetilde{h}^{*} / \sqrt{S_n}) (K \sqrt{S_n}) \dd{f})^2 }{\text{stuff}}  \\
 &\leq \int \frac{\abs{\widetilde{h}}^2 / S_n \dd{f} \int \abs{K}^2 S_n\dd{f}}{ \frac{1}{2}\int \abs{K}^2 S_n \dd{f}}  \\
 &= 2 \int_{- \infty }^{\infty } \frac{\abs{\widetilde{h}}^2}{S_n} \dd{f}
\,,
\end{align}
%
therefore the optimal filter is 
%
\begin{align}
K(f) = \frac{\widetilde{h}(f)}{\sqrt{S_n(f)}}
\,.
\end{align}

This corresponds to the signal weighted by the strain spectral density. This can be already be done by plotting the Fourier transform of \(d(f) /\sqrt{S_n(f)} \): if we do so we can see the signal much better.
This is \textbf{whitening}. 

We then have 
%
\begin{align}
\rho^2 = \int \frac{\abs{\widetilde{h}}^2}{S_n} \dd{f}
\,,
\end{align}
%
and this only depends on the model we have for the signal, not on the actual data. 
This is the optimal SNR we could reach, with a noise realization where the noise is precisely zero.
Typically, one does not even really consider the optimal \(\rho \).



We can define 
%
\begin{align}
(a|b) &= \int_{- \infty }^{\infty } \frac{\widetilde{a}^{*}(f) \widetilde{b}(f) + \widetilde{b}^{*}(f) \widetilde{a}(f)}{S_n(f)} \dd{f}  \\
&= 4 \Re \int_{0}^{\infty } \frac{\widetilde{a}^{*}(f) \widetilde{b}(f)}{S_n(f)} \dd{f}
\,.
\end{align}

With this, \(\rho_{\text{opt}} = \sqrt{(h| h)}\), while \(\rho_{MF} = (d|h) / \sqrt{(h|h)}\). 

For next time: could we interpret the "actual" SNR as corresponding to a "number of sigmas" in a hypothesis-testing scenario? Say, we want to compute something akin to a p-value against the null hypothesis of "there is no signal".
I would have thought it was not a probability itself but something like how far we are from the mean of the Gaussian noise.

\end{document}