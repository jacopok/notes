\documentclass[main.tex]{subfiles}
\begin{document}

\marginpar{Wednesday\\ 2021-7-14, \\ compiled \\ \today}

We can write the data as \(d(t) = s(t) + n(t)\); we use \(s\) to denote the \emph{true} signal, while \(h(t)\) is our \emph{model} for the signal. 

The matched-filtering SNR is \((d|h) / \sqrt{(h|h)}\), while the optimal SNR is \(\sqrt{(h|h)}\). 
The issue is that \(h\) is a function of many parameters, 
%
\begin{align}
h = h (t, \vec{\theta} ) 
\qquad \text{where} \qquad
\vec{\theta} = (m_1, m_2, \vec{\chi}_1, \vec{\chi}_2, \Lambda_1 , \Lambda_2 , D_L, \alpha , \delta \dots)
\,,
\end{align}
%
so, how do we estimate \(\vec{\theta}\)?

We need to introduce a probability density \(\mathbb{P}(\theta )\) which encodes the agreement between the observed signal \(d(t)\) and the prediction \(h(t, \vec{\theta})\). 

How do we define \(\mathbb{P}(\theta )\)? 
We start from the residuals, the least squares: we can define the residual as 
%
\begin{align}
\chi^2 = (d-h | d-h)
\,.
\end{align}

The probability density for this reads 
%
\begin{align}
\mathbb{P}(\theta ) \propto e^{- \frac{1}{2} \chi^2}
= \exp(- \frac{1}{2} (h-d | h-d))
\,.
\end{align}
%
which is the probability of having that specific realization of the noise. 

In order to sample this likelihood we need to use some MCMC.



\end{document}