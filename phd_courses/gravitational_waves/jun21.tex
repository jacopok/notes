\documentclass[main.tex]{subfiles}
\begin{document}

\section{Post-Newtonian formalism}

\marginpar{Monday\\ 2021-6-21, \\ compiled \\ \today}

We know that the scaling of the perturbation is typically 
%
\begin{align}
h \sim \frac{R}{D} \frac{GM}{c^2 R} \qty( \frac{v}{c})^2
\,.
\end{align}

The assumptions for this linearized gravity quadrupole formula are: 
\begin{enumerate}
    \item fixed background;
    \item \(v\) is small;
    \item background and source are independent (this is true for non-selfgravitating sources).
\end{enumerate}

However: this formula tells us that the most significant sources of GW are indeed self-gravitating, and not described by it! 

So, how do we move forward, and get a formalism for compactness \(GM / c^2R\) approaching \(1/2\), such as BHs or NSs?

A strongly self-gravitating source will also typically have high velocity, because of the virial theorem: 
%
\begin{align}
\qty( \frac{v}{c})^2 \sim \frac{2GM}{c^2 R} = \frac{R_s}{R}
\,.
\end{align}


The ``background'' for a strongly self-gravitating source is dependent on the source! It is not really a background. 

The idea behind the PN formalism is to iteratively solve the Einstein equations, as an expansion in powers of \(\epsilon = v/c\). 

We will still refer to sources with \emph{compact support} \(r<R\), and assume that \(T^{00} \) is the dominant component: \(\abs{T^{00} / T^{0i}} = \order{\epsilon }\) and \(\abs{T^{00 } / T^{ij}} = \order{\epsilon^2}\). 

The expansion to order \(\order{\epsilon^{2n}} = \order{ (v/c)^{2n}}\) is called the \(n\)-PN order expansion.

So we can talk about 1PN, 1.5 PN, 2.5PN and so on.

A \emph{historical remark}: the PN formalism started with Einstein; Landau and Lifshitz already included it in their book, also De Sitter and Chandrasekhar worked on it.

Einsten-Infeld-Hoffmann developed the 1PN Lagrangian for the motion of \(N\) particles. 
Peter and Hathews (1963). 

The formalism was really nailed down systematically by Blanchet and Damour. 
Also, Will and Wisemann worked on it in the 90s. 

Naîvely, we'd work as such: calculate the motion to \(\epsilon^n\), and then use linear theory to calculate the GWs emitted.
This is wrong: the background is nontrivial, there is a complicated dependence through the Einstein equations. 

WE emission subtracts energy to the motion, and GWs at a certain PN order are sources of GW at higher orders through the Isaacson tensor. 

The equation \(\dv*{E}{t} = P_{gw}\) is an ansatz: this is used but it not guaranteed to hold at each PN order. 

Let us start with the metric at 1PN order: 
%
\begin{align}
g_{00} &= - 1 - 2\phi = - 1 + \frac{2 U}{c^2}  \\
g_{0i} &= 0  \\
g_{ij} &= \delta_{ij}
\,,
\end{align}
%
where \(U\) is the solution to the Newtonian Poisson equation: 
%
\begin{align}
U &= \frac{G}{c^2} \int \dd[3]{y} \frac{T^{00} (y)}{\abs{x-y}}
\,.
\end{align}

Now, we have seen that different components of \(T_{\mu \nu }\) have different orders of \(c\), so different metric coefficients must be expanded up to different powers, determined by how they appear in the EFE. 

If \(g_{00} \) is expanded to order \(\epsilon^n\), then \(g_{0i} \) must be expanded to order \(\epsilon^{n-1}\) because it has a power \(\epsilon \) intrinsically in it, and similarly \(g_{ij} \) must be expanded to the power \(\epsilon^{n-2}\).

\todo[inline]{Understand this better. Look at Maggiore. }

In terms of time reversal symmetry, \(g_{00}\) and \(g_{ij}\) are even while \(g_{0i}\) are odd. 

\begin{figure}
\centering
\begin{tabular}{c|ccc}
Component & Newton (0PN) & 1PN & 2PN \\
\hline
\(g_{00} \) & \(-1 + g_{00}^{(2)}\) & \(g_{00}^{(4)}\) & \(g_{00}^{(6)}\) \\
\(g_{0i}\) & 0 & \(g_{0i}^{(3)}\) & \(g_{0i}^{(5)}\) \\
\(g_{ij}\) & \(\delta_{ij}\) & \(g_{ij}^{(2)}\) & \(g_{ij}^{(4)}\) 
\end{tabular}
\label{tab:PN-orders}
\caption{}
\end{figure}

This is correct up to when we must include radiation reaction, but it turns out that that comes in at 2.5PN order. 

So, in order to write the metric at 1PN order, we can match the EFE at each order taking into account the power counting. 

Derivatives have orders \(\partial_{t} = \order{v} \partial_{i}\), while 
%
\begin{align}
- \frac{1}{c^2} \partial_{tt} + \triangle = \qty[1 + \order{\epsilon }] \triangle  
\,,
\end{align}
%
therefore the retardation effects are ``higher order''. 

Therefore, the lowest PN solutions are typically given in terms of \emph{instantaneous potentials} (solution of Poisson equations).

This is expected: the PN expansion reads 
%
\begin{align}
F(n) = F( t - \frac{r}{c}) = F(t) - \frac{r}{c} \dot{F} (t) + \frac{r^2}{2 c^2} \ddot{F}( t) + \dots 
\,.
\end{align}

In Fourier terms: 
%
\begin{align}
\widetilde{F} &\approx \widetilde{F} \qty(1 - \frac{r \omega }{c} + \frac{r^2\omega^2}{c^2} + \dots)  \\
&= \widetilde{F} \qty(1 - \frac{r}{\lambda } - \frac{r^2}{2 \lambda^2})
\,.
\end{align}
%
\todo[inline]{the lambdas are barred --- characteristic length.}

This means that this is also an expansion for nearby fields: the expansion is in small velocities \(\epsilon = v/c \ll 1\), as well as in the \emph{near-zone} \(r / \lambda  \ll 1\). 

Therefore, the PN expansion is \textbf{not valid to compute GWs far away from the source}! 

Let us define a \textbf{near zone} \(r \ll L\), but \(L \gg R\), such that \(r / \lambda \ll 1\), such that the aforementioned expansion makes sense. 

We also have a \textbf{far zone} \(R < r < \infty \), and an \textbf{overlap zone} \(R < r < L\) in which the near and far zones overlap. 

Back to the 1PN metric: we impose Harmonic gauge, so that \(\partial_{\mu } \qty(\sqrt{-g} g^{\mu \nu }) = 0\), and we start from the 0PN equation \(\triangle U = - (8 \pi G/c^{4}) T^{00}_{(0)}\). 
Then, to 1PN order we get: 
%
\begin{align}
\triangle g_{ij}^{(2)} &= - \frac{8\pi G}{c^{4}} \delta_{ij} T^{00}_{(0)}  \\
\triangle g_{0i}^{(3)} &= \frac{16 \pi G}{c^{4}} T^{0i}_{(1)}  \\
\triangle g_{00}^{(4)} &= \dots
\,,
\end{align}
%
so, assuming that \(T^{\mu \nu }\) is known, these can be solved in terms of Green's functions. 
After all the calculations one finds 
%
\begin{align}
\square_\eta V &= - \frac{4 \pi G}{c^{4}} \underbrace{\qty(T^{00} + T^{ii})}_{\mathclap{\text{active gravitational mass density}}}  \\
\square_\eta V^{i} &= \dots
\,,
\end{align}
%
where we have  the 1PN potential \(V\) which defines the metric: 
%
\begin{align}
g_{00} &= -1 + \frac{2U}{c^2} + \frac{2 V^2}{c^{4}}  \\
g_{0i} &= - \frac{4 V_{i}}{c^3}  \\
g_{ij} &= \delta_{ij} \qty(1 + \frac{2 V}{c^2})
\,.
\end{align}

The Einstein-Infeld-Hoffmann Lagrangian is the \(N\)-particle one for this metric: the corresponding stress-energy tensor is
%
\begin{align}
T^{\mu \nu } = \frac{1}{\sqrt{-g}} \sum _a \dv{\tau_a}{t} m_a \dv{x_a^{\mu }}{t} \dv{x^{\nu }_a}{t} \delta (x - x_a)
\,.
\end{align}

The action reads 
%
\begin{align}
S = - mc^2 \int \dd{t} \sqrt{- g_{00} -2 g_{0i} \frac{v^{i}}{c} - g_{ij} \frac{v^{i} v^{j}}{c^2}} 
\,.
\end{align}

This can be written as \(L= L _{\text{Newt}} + L_{1PN}\). 

If one takes \(N = 1\) they get the two-body problem, the EOM for a relativistic binary at 1PN order. 
This solution can be mapped to a Newtonian problem and solved: this has been done by Damour and Deruelle in 86, which is important since it was applied to the Hulse-Taylor pulsar: the equation they found was 
%
\begin{align}
\expval{\dot{\omega}} = \num{2.11353} \qty(\frac{m_1 + m_2 }{M_{\odot}})^{2/3} \SI{}{deg / yr}
\,.
\end{align}

The orbital decay has been observed very accurately, which can be a way to determine the binary mass. 

What are the difficulties? We get Poisson equations in the form 
%
\begin{align}
\triangle g_{\mu \nu }^{(n)} = \text{matter source} + \text{metric source } g_{\mu \nu }^{(n-1)}
\,.
\end{align}

The second term does not have compact support! The Poisson integrals used so far are not good solutions. 
We need different boundary conditions. 

For example, consider \(\triangle u = \rho = \const\). The Poisson integral is not a solution since \(u \neq 0\) at infinity: it diverges.

However, a solution exists: \(u(r) = - (1/6) \rho r^2\). We need other formal solutions to the Poisson equation. 

How is this solved? We do it by inverting the Poisson equation with boundary conditions by a suitable procedure which involves regularization of the divergences and analytic continuation. 
We will not explore this in detail, but it is good to know that the problem exists. 

Another difficulty is the following: 
the PN expansion of the metric potential is valid in the near zone but it blows up for large \(r \to \infty \).
It is in the form 
%
\begin{align}
F (\epsilon , r) = \sum _{n} c_n (r) \epsilon^n
\,.
\end{align}

This is an expansion with two scales, and the series is \emph{not} uniformly convergent. This is analogous to perturbation theory in quantum mechanics: asymptotic series. 

The residuals eventually blow up, but there is an optimum \(n\) for the expansion. 
The solution is to perform a post-Minkowskian expansion in the far zone: 
%
\begin{align}
F = \sum _{n} G^{n} F_n
\,,
\end{align}
%
and match the PN expansion in the near zone. 

Also, there is the problem of \textbf{backreaction}, GW influence on the motion. 
This breaks the power-counting. 

When does it enter the game? We know that 
%
\begin{align}
E = K + V \approx - \frac{V}{2} + V = \frac{V}{2} = - K = - \frac{M}{2} v^2
\,,
\end{align}
%
therefore \(\dot{E} \sim - M v \dot{v}\). 

Further, we know that 
%
\begin{align}
P_{gw} = \dot{E}_{gw} \approx \frac{GM^2}{c^{5}} \frac{v^{6}}{r}
\,,
\end{align}
%
therefore 
%
\begin{align}
\dot{v} \approx \frac{GM}{r^2} \qty( \frac{v}{c} )^{5}
\,,
\end{align}
%
which gives the correct order of 5, meaning 2.5PN. 

What are the \textbf{Relaxed EFE}? 
We define 
%
\begin{align}
\mathfrak{h}^{\mu \nu  } = \sqrt{-g } g^{\mu \nu } -\eta^{\mu \nu }
\,,
\end{align}
%
a  new field in the full theory. 
We then rewrite the EFE in terms of it, using the gauge \(\partial_{\mu } \mathfrak{h}^{\mu \nu} = 0\) (equivalent to harmonic gauge). 

The EFE then read 
%
\begin{align}
\square \mathfrak{h}^{\mu \nu } &= \frac{16 \pi G}{c^{4}} \tau^{\mu \nu }  \\
&= \frac{16 \pi G}{c^{4}} \qty[-g T^{\mu \nu } + \tau^{\mu \nu }_{u} + \tau^{\mu  \nu }[ \mathfrak{h}^{\alpha \beta }]]
\,,
\end{align}
%
where \(\tau_u \) is the Landau-Liftschitz pseudotensor. 
These are formally similar to the linearized equations, and in the weak-field limit \(\mathfrak{h}^{\mu \nu } = - \overline{h}^{\mu \nu }\). 

All the nonlinearity is still there though, this makes no approximations: they are just made to look that way. 

This \(\tau \) satisfies \(\partial_{\mu } \tau^{\mu \nu } = 0\) (notice the flat derivative). 
Also, these are called ``relaxed'' because a solution of the relaxed equations alone does not imply the conservation of \(\tau^{\mu \nu }\): one needs to add the gauge. 

The formal solution of these REFE is 
%
\begin{align}
\mathfrak{h}^{\mu \nu } = - \frac{4G}{c^2} \int \dd[3]{y} 
\frac{\tau^{\mu \nu }(t - \abs{x-y} / c, y)}{\abs{x-y}}
\,,
\end{align}
%
but we do not truly know \(\tau \); however we can solve this iteratively. 

We start by finding a solution for the REFE in the far region with the multipolar formula: 
%
\begin{align}
\mathfrak{h}^{\mu \nu } = \sum _{n} \mathfrak{h}_{(n)}^{\mu \nu } G^{n}
\,,
\end{align}
%
there for \(r > R\) we have the unspecified source multipole moments \(\mathfrak{h}^{\mu \nu }_{(n)}\). 

Then we find a solution for the REFE in the near zone: 
%
\begin{align}
\mathfrak{h}^{\mu \nu } = \sum _{n } \frac{1}{c^{n}} \mathfrak{h}^{\mu \nu }_{(n)}
\,.
\end{align}

Here, for \(r < L\), we have \(^{(n)} \mathfrak{h}^{\mu \nu }\), multipoles of \(\tau^{\mu \nu }\). 
We match the two in the overlapping zone to determine \(\tau\). 

Finally, we compute the TT solution far away, with a gauge transformation from harmonic to radiative coordinates: from \((I_L, J_L, W_L, \dots)\) to \((U_L,V_L)\), the radiative multipoles of \(h_{ij}^{TT}\).

The good thing is that this can all be done with a finite amount of terms, however in general one does not know how far the expansion can be pushed.  

\todo[inline]{What does the overlap region look like?}



\end{document}
