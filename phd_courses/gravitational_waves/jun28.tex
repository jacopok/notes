\documentclass[main.tex]{subfiles}
\begin{document}

\section{Perturbations and stability}

\marginpar{Monday\\ 2021-6-28, \\ compiled \\ \today}

For the next two-three weeks the lessons will be held by Matteo.
Then, there will be a final lesson on the EOB framework. 

The perturbations we will discuss today could be BH perturbations (Schwarzschild), but they can apply just as well to any spherically symmetric spacetimes, such as perturbations of a spherically symmetric star.  

This framework allows us to discuss the stability of relativistic astrophysical objects. 
Regge and Wheeler discussed this in 1957.

What we end up us the \textbf{Regge-Wheeler-Zerilli} equation:
%
\begin{align}
- \Psi_{tt}^{\ell m} + \Psi^{\ell m}_{r_* r_*} - V_\ell (r_* (r)) \Psi^{\ell m} = S_{\ell m}
\,,
\end{align}
%
where \(r_* = r + R_S \log (r / R_s - 1)\) is the \emph{tortoise coordinate}, which maps \(r \in (2M, \infty )\) to \(r_* \in \mathbb{R}\).

The function \(\Psi_{\ell m} (t, r_*)\) is called the RWZ master function. 
The equation is a wave equation with a potential \(V_\ell\) which is determined by the background metric; also there are source terms \(S_{\ell m}\) which correspond to the perturbed stress-energy tensor \(\delta T_{\mu \nu }\).

The gravitational waves can be recovered as 
%
\begin{align}
h_{+} + i h_{\times} = \frac{G}{c^{4} r} 
\sum _{\ell=2} \sum _{\abs{m} \leq \ell} \sqrt{\frac{(\ell+2)!}{(\ell-2)!}} {}^{(-2)} Y_{\ell m} (\theta , \varphi ) 
\qty(\Psi^{(e)}_{\ell m} + i\Psi^{(o)}_{\ell m}) + \order{1/r^2}
\,.
\end{align}

The harmonics \({}^{(-2)}Y_{\ell m}\) appearing here are spin-weighted spherical harmonics. 

If there is no source, this is very similar to a Schroedinger equation. 
Therefore, this problem can be considered like a quantum-mechanical scattering one. 
    
Starting from an incoming wave, some of it will be transmitted and some will be reflected. 

The reflected wave has a characteristic ``damped oscillatory'' nature. 
These are \emph{quasi}-normal modes since they are complex. 

Davis, Ruffinin and Tiomno in 1971 calculated the gravitational radiation emission from a particle falling radially into a Schwarzschild black hole. 

In this case as well looking at \(\Psi_{20}\) we have a precursor, then a burst, and a ringdown. 

The signal from the collapse of a rotating Neutron Star is also similar! 

The merger itself cannot be predicted by PostNewtonian theory, even qualitatively! 

What is the origin of these quasinormal modes? 
We start with a spherically symmetric manifold with a metric \(g^{(0)}_{\alpha \beta }\), such that \(\mathcal{M} = M^2 \times S^2\), where \(M\) is Lorentzian and 2D while \(S^{2}\) is the 2-sphere.
An example for \(g_{\alpha \beta }^{(0)}\) is Schwarzschild. On top of this metric we consider a perturbation \(h_{\alpha \beta }\), such that 
%
\begin{align}
g_{\alpha \beta }= g_{\alpha \beta }^{(0)} + h_{\mu \nu }
\,.
\end{align}

We can then decompose the perturbation into scalar, vector and tensor spherical harmonics: \(Y_{\ell m}\), \(Z^{\ell m}_{a}\) and \(Z^{\ell m}_{a b}\), and further we can decompose it by even (electric-type) and odd (magnetic-type) parity, so that the two components in 
%
\begin{align}
h_{ \alpha \beta } = h_{\alpha \beta }^{(e)} + h_{\alpha \beta }^{(o)}
\,
\end{align}
%
decouple. 

The even part decomposition reads 
%
\begin{align}
h_{\mu \nu }^{(e)} = \left[\begin{array}{ccc}
H_0 Y_{\ell m} & H_1 Y_{\ell m} & h_{A}^{(e)} Z^{\ell m}_{a} \\ 
 & H_2 Y_{\ell m} & 0 \\ 
 &  & r^2 K Y_{\ell m} \Omega_{ab} + r^2 G Z^{\ell m}_{ab}
\end{array}\right]
\,,
\end{align}
%
where \(\Omega_{ab}\) is the \(S^2\) metric, and the metric coefficients \(H_i\), \(K\), \(G\) should also carry multipolar indices and a sum over \(\ell\), \(m\) is implied. 
The index \(A\) runs over \(0, 1\) while \(a\) runs over \(2, 3\). 

From these metric components we can calculate gauge invariant quantities which do not change under infinitesimal coordinate transformations, among which lie the master functions. 

For each multipole we then have a RW equation 
%
\begin{align}
(\partial_{tt} - \partial_{xx}) \Psi + V_\ell = S_{\ell m}
\,,
\end{align}
%
where \(x\) is the tortoise coordinate corresponding to \(r\). 

A small exercise we can do if we are surprised by the RW equation: consider the equation \(\square_{g^{(0)}} \phi = 0\) for some scalar field \(\phi \) on the Schwarzschild metric \(g^{(0)}\).
This can be written as 
%
\begin{align}
0 = \frac{1}{\sqrt{-g^{(0)}}} \partial_{\mu } 
\qty( \sqrt{-g^{(0)}} g^{(0), \mu \nu } \partial_{\nu } \phi  )
\,,
\end{align}
%
which is a PDE for \(\phi \), since \(g^{(0)}\) is known. 
From this we can then obtain a 3D wave equation for the scalar field. 
Then, we can expand into spherical harmonics: 
%
\begin{align}
\phi (t, r, \theta , \varphi ) = \sum _{\ell=0} \sum _{\abs{m} \leq m} \frac{1}{r} \phi_{\ell m} (t, r) Y_{\ell m} (\theta , \varphi )
\,.
\end{align}
%

If we substitute this expansion in the 3D wave equation we get a RWZ-like equation: 
%
\begin{align}
\partial_{tt} \phi_{\ell m } - \partial_{r_* r_*} \phi_{\ell m} + V_\ell \phi_{\ell m} = 0 
\,,
\end{align}
%
in which the only difference is that \(V_\ell\) here is the effective potential for a scalar field. 
This is a toy model for the linearized Einstein equations. 

Regge and Wheeler only derived the odd-parity equations, the even-parity were done later as they are more complicated.

We can transform the metric into Cartesian components: 
%
\begin{align}
h_{\hat{\mu} \hat{\nu}} e^{\mu }_{\hat{\mu}} e^{\nu }_{\hat{\nu}} h_{\mu \nu }  
\,,
\end{align}
%
with \(e = \diag{e^{a}, e^{-b}, r^{-1}, (r \sin \theta )^{-1} }\).

We can look at the large \(r\) behaviour of the metric coefficients, and see that they do not depend on \(r\) there. 

We can also identify 
%
\begin{align}
\underbrace{\Psi_{\ell m}^{(e)} + i \Psi_{\ell m}^{(o)}}_{r h_{\ell m}} \propto u_{\ell m} - \frac{i}{c} v_{\ell m}
\,.
\end{align}

This admits an expansion in Post-Newtonian theory. 

\subsection{The RWZ problem in vacuo}

How do we set this up as an initial-boundary value problem? 
The potential is zero at radial infinity, therefore there the asymptotic solutions must look like the solutions to \(\Psi_{tt} - \Psi_{xx} = 0\). 

The equation can then be cast into the form 
%
\begin{align}
\dv[2]{\widetilde{\Psi}}{x} + \qty[\omega^2 - V_\ell] \widetilde{\Psi} = 0
\,,
\end{align}
%
but there are no ``bound states'' and the spectrum is continuous: no signals can come out of the horizon, so if the boundary condition at \(x \to - \infty\) is an ingoing wave then the outgoing wave will look like \(\widetilde{\Psi} \sim e^{i \omega x}\) as \(x \to \infty \).

What we find are \emph{damped modes}. 

The solution can be calculated through a Laplace transform. 
The equation can be restated as 
%
\begin{align}
\phi_{xx} - \qty(s^2 + V(x)) \phi = F(s, x) = - s \psi (x) - \psi_t (x)
\,.
\end{align}

In order to solve the corresponding Laplace integral we need to do complex contour integration, which justifies the fact that after the QNM decay there is a powerlaw section. 

One can write down equation 17 (of the slides) for analytic functions and solve it for fixed BCs: this yields the regular normal modes (as opposed to quasi-normal ones).

\textbf{Stability}: rigurous proofs of spacetime stability are hard, we start from linear mode stability, then move to linear stability, and the full result is \emph{nonlinear stability}. 

Nonlinear stability is known for Minkowski and Schwarzschild, but not yet for Kerr. 

As we scatter around the peak, this part of the solution is excited and exponentially decays, while as we scatter against the tail another part of the solution is excited. 
The powerlaw is always there, but it is revealed as the higher amplitude QNMs die down. 

If there was no \(r^{-2}\) decay the solution would go to zero exponentially, as opposed to the powerlaw tail.

We can also do weird things by having frequencies not be able to explore the peak --- in that case we only get the tail. 



\end{document}