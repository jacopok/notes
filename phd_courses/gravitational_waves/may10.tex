\documentclass[main.tex]{subfiles}
\begin{document}

\section{Quadrupole formula}

\marginpar{Monday\\ 2021-5-10, \\ compiled \\ \today}

The self-gravity is denoted as \(\sigma = 2 GM / c^2 R = R_S / R\). 
In order for the quadrupole formula to work, we need \(\sigma \ll 1\) (negligible self-gravity) as well as the weak-field limit.  

Hypothesis 1 is that we are far away from the source, compared to the source's linear scale.

Hypothesis 2 is that the source is slow: 
%
\begin{align}
v \sim \frac{\abs{T_{0i}}}{\abs{T_{00} }} \ll c = 1
\,.
\end{align}

The process is as follows:
\begin{enumerate}
    \item we use the two hypotheses to simplify the Green equation;
    \item we use the conservation law of \(T_{\mu \nu }\) in flat spacetime \(0 = \partial^{\mu } T_{\mu \nu }\) to further express \(T_{ij} \) in the Green equation in terms of \(T_{00} = \rho c^2\)
    \item we project into the TT gauge.
\end{enumerate}

The retarded time is 
%
\begin{align}
u = t - \abs{\vec{x} - \vec{x}'} \approx t- r + \hat{n}\cdot \vec{x}' 
\,.
\end{align}

The far-field approximation is given by expanding \(T_{\mu \nu } (t_r + \hat{n} \cdot \vec{x}' / c)\) around \(t_r\).

As for the slow-velocity expansion, we do a Fourier transform and expand the exponential 
%
\begin{align}
e^{i \omega \qty(t_r + \hat{n} \cdot \vec{x}' /c)} \approx e^{-i \omega t_r} \qty(1 - \frac{i \omega }{c} n_i x^{i} + \dots)
\,.
\end{align}

Then, in the time domain time derivatives correspond to \(i \omega \) terms. 

\end{document}