\documentclass[main.tex]{subfiles}
\begin{document}

\section{GW energy}

\marginpar{Monday\\ 2021-5-17, \\ compiled \\ \today}

Because of the equivalence principle, there can be no local definition of energy density or of a stress-energy tensor for \(g_{\mu \nu }\). 

However, we can define the Landau-Lifshitz pseudotensor (which is actually a tensor density): from the gothic metric 
%
\begin{align}
\mathfrak{g}^{\alpha \beta } = \sqrt{-g } g^{\alpha \beta }
\,
\end{align}
%
we define 
%
\begin{align}
\Lambda^{\alpha \mu \beta \nu } = \mathfrak{g}^{\alpha \beta } \mathfrak{g}^{\mu \nu } - \mathfrak{g}^{\alpha \nu } \mathfrak{g}^{\beta \mu }
\,,
\end{align}
%
a four-tensor density which has the same symmetries as the Riemann tensor. 
Also, we have the properties that 
%
\begin{align}
\partial_{\mu } \partial_{\nu } \Lambda^{\alpha \mu \beta \nu } = 2 (-g) G^{\alpha \beta } + 16 \pi (-g) \tau^{\alpha \beta }_{LL}
\,,
\end{align}
%
where \(G^{\alpha \beta }\) is the Einstein tensor, while \(\tau_{LL}\) (defined to contain whatever remains) is called the Landau-Lifshitz tensor: this can also be expressed as 
%
\begin{align}
\partial_{\mu } \partial_{\nu } \Lambda^{\alpha \mu \beta \nu } =
16 \pi (-g) \qty(T^{\alpha \beta } + \tau_{LL}^{\alpha \beta })
\,.
\end{align}

If we take a further derivative with respect to \(\alpha \), in a vacuum, we find 
%
\begin{align}
\partial_{\alpha } \partial_{\mu } \partial_{\nu } \Lambda^{\alpha \mu \beta \nu } = \partial_{\alpha } \tau_{LL}^{\alpha \beta } = 0
\,,
\end{align}
%
where the last equality is due to the symmetries of \(\Lambda \). Therefore, the tensor \(\tau_{LL}\) is conserved, and we can try to interpret it as the ``energy of the gravitational field''.

In normal coordinates at any given point \(\tau_{LL}\) is identically zero.

A global concept of energy does not exist for certain spacetimes --- see ADM and the Hamiltonian formulation of GR. 

However, asymptotically flat spacetimes have some nice properties.

If we foliate a spacetime \((M, g)\) into 3D spacelike hypersurfaces \(\Sigma \), we say it is asymptotically flat if the reduced metric \(\gamma_{ij} \to f_{ij}\) tends towards the Minkowski metric at radial infinity. 

In this case, we can associate an energy to \(\Sigma \).
Using these concepts, there is hope to identify some quantity which represents GW energy.

We explore directions at infinity which are \textbf{asymptotically null}, and suppose that at \(\Sigma_{t_{1, 2}}\) we have stationary states, with dynamics in between. 

Then, if there is a difference of energy \(\Delta E = E_2 - E_1 \), we expect there to be a flux of ``some \(\tau_{\mu \nu }\)'' from the surface. 

We can work in linearized theory: \(g = f + h\). 
The characteristics we expect a definition of energy to have are, at the very least
\begin{enumerate}
    \item quadratic in \(h\),
    \item generate curvature via a stress-energy tensor \(\tau_{\alpha \beta }\),
    \item gauge invariant under infinitesimal coordinate transformations.  
\end{enumerate}

We start by specifying the \(LL\) tensor to the weak field case. 
The metric reads \(g = \eta + h^{(1)} + h^{(2)} + \order{3}\), where by the indices in parentheses we mean orders in some expansion parameter.

The Ricci reads 
%
\begin{align}
R_{\mu \nu } = 
R_{\mu \nu }^{(0)} + 
R_{\mu \nu }^{(1)} + 
R_{\mu \nu }^{(2)} + 
\order{3}
\,.
\end{align}

Now, \(R_{\mu \nu }^{(0)} = (\text{Ric}[\eta ])_{\mu \nu } \sim \eta \partial^2 \eta  = 0\), but if we had a non-flat background this would not hold. 
On the other hand, 
%
\begin{align}
R^{(1)}_{\mu \nu } = \qty(\text{Ric}^{(1)}[h^{1}])_{\mu \nu } \sim \eta \partial^2 h^{(1)}
\,,
\end{align}
%
where \(\text{Ric}^{(i)}\) is the Ricci operator expanded up to \(i\)-th order. This is relevant in the second-order term: 
%
\begin{align}
R^{(2)}_{\mu \nu } = 
\qty(\text{Ric}^{(1)} [ h^{(2)}])_{\mu \nu }
\qty(\text{Ric}^{(2)} [ h^{(1)}])_{\mu \nu }
\sim
\eta \partial^2 h^{(2)} + h^{(1)} \partial^2 h^{(1)}
\,.
\end{align}

Then, the EFE in vacuo read 
%
\begin{align}
0 = 
R^{(0)}_{\mu \nu } + 
R^{(1)}_{\mu \nu } + 
R^{(2)}_{\mu \nu } +  \order{3}
\,,
\end{align}
%
therefore to order 0 we have the \(\eta \) metric; to order \(1\) we can calculate \(h^{(1)}\), while to order \(2\) we can assume the \(h^{(1)}\) metric is already calculated and solve \(R^{(2)}_{\mu \nu } = 0\) for \(h^{2}\). 

The second order equation reads 
%
\begin{align}
G^{(1)}_{\mu \nu }[h^{(2)}] &= \qty(\text{Ric}^{(1)}[h^{2}] - \frac{1}{2} R^{(1)} [h^{2}]\eta )_{\mu \nu }  \\
&=\qty(- \text{Ric}^{(2)} [h^{1}] + \frac{1}{2} R^{(2)}[h^{1}] \eta )_{\mu \nu }   \\
&\overset{?}{=} 8 \pi \tau_{\mu \nu }
\,,
\end{align}
%
but can this actually be our GW stress-energy tensor? It is symmetric, it is quadratic in \(h\), and due to the Bianchi identities it is also conserved. 
However, it is not gauge invariant, and it is not unique! 

If \(h^{(1)}\) is \textbf{asymptotically flat}, then \(h \sim \order{1/r}\), \(\partial_{i} h \sim \order{1/r^2}\) and \(\partial_{i} \partial_{j} h \sim \order{1/r^3}\).

In this case, \(E = \int_{\Sigma } \dd[3]{x} \tau_{00} \) is both \textbf{gauge invariant} and \textbf{unique}: \(E[h_{\mu \nu } + 2 \partial_{(\mu } \xi_{\nu )}] = E[h_{\mu \nu }]\). 

We can then write an energy flux in the form 
%
\begin{align}
\Delta E = -\int_{S} \dd[2]{y} \tau_{\mu 0} n^\mu   
\,,
\end{align}
%
across a surface \(S\) which determines the spatial boundary of \(\Sigma \) over time.  

Weak-field GR can be interpreted as a field theory on \(\eta \) for the field \(h\). 

Physically speaking, it is true that we can ``eliminate'' the gravitational field at a point, but here we are more interested in the following question:
can the GW in a neighborhood of that point contribute to the curvature via a gauge invariant tensor?  

The issue comes down to: what is the distinction between ``waves'' and background?

One can prove that a suitable average of \(\text{Ric}^{(2)}[h^{(1)}]\) is actually gauge-invariant: this is the \textbf{Isaacson tensor}, 
%
\begin{align}
\tau_{\alpha \beta } = \frac{c^{4}}{32 \pi G} \expval{\partial_{\mu } h_{\alpha \beta } \partial_{\nu } h^{\alpha \beta }}
\,,
\end{align}
%
due to the fact that \(\text{Ric}^{(2)}[h] \sim h \partial \partial h + \partial h \partial h + \partial (h \partial h)\). 

Let us take the derivative \(\partial_{\mu} \tau^{\mu \nu }\): 
%
\begin{align}
\int_{V} \dd[3]{x} \qty(\partial_0 \tau^{00} + \partial_{i} \tau^{0i}) 
&= \dot{E} + \oint_{S} \dd[2]{y} \tau_{0i} n^i  \\
&= \dot{E} + r^2 \oint \dd[2]{y} \expval{\partial^{0} h_{ij} \partial_{r} h^{ij}}
\,,
\end{align}
%
which yields the quadrupole formula: 
%
\begin{align}
\dv{E_{GW}}{t} &= \frac{c^3}{32 \pi G} r^2 \int \dd{\Omega } \expval{\partial_{t} h_{ij}^{TT} \partial_{t} h_{ij}^{TT}}  \\
&= \frac{G}{5 c^{5}}\expval{ \dot{\ddot{Q}}_{ij} \dot{\ddot{Q}}^{ij}}
\,.
\end{align}

Let us look at some dimensional analysis: 
%
\begin{align}
[Q] &= a ML^2  \\
[\dot{\ddot{Q}}] &= a ML^2 T^{-3} \sim a \Omega^3 M L^2 = E^2 T^{-2}\\
[ \frac{G}{c^{5}}] &= T E^{-1}
\,.
\end{align}
%
The factor \(G / c^{5}\) is the inverse of a power: \(c^{5} / G \approx \SI{e52}{W}\), a very large power, and we are dividing by it.  
This suggests a potential rewriting of the formula, which was an idea of Weber's: 
%
\begin{align}
\dot{E} \sim \frac{G}{c^{5}} a^2 \Omega^{6} M^2 R^{4} = \dots
= a^2 \frac{c^{5}}{G} \qty( \frac{v}{c})^{6} \frac{GM}{c^2 R}
\,.
\end{align}

The power is therefore huge if \(v \sim c\) and \(R \sim R_S\)! 

The above formula for \(\dot{E}\) is \emph{correct} for a leading-order description of a binary system, if we take \(a^2 = 32/5\), \(\Omega \) to be the orbital frequency, \(M\) to be \(\mu \) (the reduced mass), and \(R\) the orbital radius. So, 
%
\begin{align}
\dot{E} = \frac{32}{5} \frac{G \mu^2}{c^{5}} R^{4} \Omega^3
\,.
\end{align}

Of course, \(\dot{E} _{\text{orbital energy}} = - \dot{E} _{\text{gw}}\). 

\end{document}
