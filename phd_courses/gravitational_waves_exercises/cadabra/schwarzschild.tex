% This is the bit of LaTeX style information that DataCell.cc needs in
% order to write notebooks out in standalone LaTeX form. It is very
% similar to ../frontend/common/preamble.tex; keep them in sync.

\documentclass[10pt]{article}
\usepackage[scale=.8]{geometry}
\usepackage{setspace}
\usepackage{fancyhdr}
\usepackage{listings}
\usepackage[fleqn]{amsmath}
\usepackage{color}
\usepackage{changepage}
\usepackage[colorlinks=true, urlcolor=black, plainpages=false, pdfpagelabels]{hyperref}
\usepackage{etoolbox}
\usepackage{amssymb}
\usepackage[parfill]{parskip}
\usepackage{graphicx}
%\usepackage{tableaux}
\def\specialcolon{\mathrel{\mathop{:}}\hspace{-.5em}}
\renewcommand{\bar}[1]{\overline{#1}}
\newcommand{\algorithm}[2]{{\tt\Large\detokenize{#1}}\\[1ex]
{\emph{#2}}\\[1ex]
}
\newcommand{\property}[2]{{\tt\Large\detokenize{#1}}\\[1ex]
{\emph{#2}}\\[1ex]
}
\newcommand{\algo}[1]{{\tt \detokenize{#1}}}
\newcommand{\prop}[1]{{\tt \detokenize{#1}}}
\renewcommand{\author}[1]{{\bfseries #1}}
\newcommand{\email}[1]{, {\tt #1}}
%\makeatletter\def\old@comma{,}\catcode`\,=13 \def,{%
%\ifmmode\old@comma\discretionary{}{}{}\else\old@comma\fi}\makeatother

% Math expressions wrapped in \brwrap will get typeset with
% round brackets around them, which have the appropriate size.
% The expression itself can still be broken over multiple lines.

\newcommand\brwrap[3]{%
  \setbox0=\hbox{$#2$}
  \left#1\vbox to \the\ht0{\hbox to 0pt{}}\right.\kern-.2em
  \begingroup #2\endgroup\kern-.15em
  \left.\vbox to \the\ht0{\hbox to 0pt{}}\right#3
}

\renewcommand{\arraystretch}{1.2}
\tolerance=10000
\relpenalty=10
\binoppenalty=10
\hyphenpenalty=10
\raggedright

\lstnewenvironment{python}[1][]{\lstset{language=python,
   columns=fullflexible,
   xleftmargin=1em,
   belowskip=0pt,
   tabsize=3,
   commentstyle={}, % otherwise {#} cadabra arguments look ugly
   breaklines=true,   
   basicstyle=\small\ttfamily\color{blue},
   keywordstyle={}
}}{}
  

\everymath{\displaystyle}

% Page numbers
\pagestyle{fancy}
\fancyhf{} % clear all header and footer fields
\renewcommand{\headrulewidth}{0pt}
\renewcommand{\footrulewidth}{0pt}
\fancyfoot[LE,RO]{{\small\thepage}}
\fancyfoot[LO,RE]{{\tiny\href{http://cadabra.science}{{\tt http://cadabra.science}}}}

% \makeatletter\def\old@comma{,}\catcode`\,=13 \def,{%
% \ifmmode\old@comma\discretionary{}{}{}\else\old@comma\fi}\makeatother

% Ensure that maths broken over multiple lines has a bit of spacing
% between lines.
\lineskiplimit=0mm
\lineskip=1.5ex

% Typesetting Young tableaux, originally in a separate style
% file, now included to avoid path searching problems. 
% Some internals for the typesetting macros below; nothing
% user-servicable here; please read on.

\def\@tabforc#1#2#3{\expandafter\tabf@rc\expandafter#1{#2 \^}{#3}}
\def\tabf@@rc#1#2#3\tabf@@rc#4{\def#1{#2}#4\tabf@rc#1{#3}{#4}}
\long\def\ReturnAfterFi#1\fi{\fi#1}
    \def\tabf@rc#1#2#3{%
      \def\temp@ty{#2}%
      \ifx\@empty\temp@ty
      \else
        \ReturnAfterFi{%
          \tabf@@rc#1#2\tabf@@rc{#3}%
        }%
      \fi
    }%

% Sorry, some global registers for sizes and keeping track of
% measurements.
    
\newdimen\ytsize\ytsize=2mm
\newdimen\ytfsize\ytfsize=4mm
\newcount\repcnt
\newdimen\acchspace
\newdimen\accvspace
\newdimen\raiseh
\newdimen\maxw

\newcommand\phrule[1]{\hbox{\vbox to0pt{\hrule height .2pt width#1\vss}}}

% Typeset a Young tableau with filled boxes. Takes a single 
% argument which is a string of symbols for each row,
% separated by commas. Examples:
%
%   \ftableau{abc,de}
%   \ftableau{ab{d_2},f{g_3}}

\newcommand\ftableau[1]{%
\def\ctest{,}
\def\Ktest{\^}
\acchspace=0ex
\accvspace=0ex
\maxw=0ex
\vbox{\hbox{%
\@tabforc\thisel{#1}{%
 \ifx\thisel\Ktest{%
     \ifnum\maxw=0\maxw=\acchspace\fi%
     \raisebox{\accvspace}{\vbox to \ytfsize{\hbox to
		 0pt{\vrule height \ytfsize\hss}}}\kern\acchspace\kern-\maxw}
 \else\ifx\thisel\ctest
     \ifnum\maxw=0\maxw=\acchspace\fi%
     \raisebox{\accvspace}{\vbox to \ytfsize{\hbox to 0pt{\vrule height \ytfsize\hss}}}%
     \kern\acchspace\acchspace=0ex
	  \advance\accvspace by -\ytfsize
 \else
     \setbox3=\hbox{$\thisel$}%
	  \raiseh=\ytfsize%
	  \advance\raiseh by -1ex%
	  \divide\raiseh by 2%
     \advance\acchspace by-\ytfsize%
     \raisebox{\accvspace}{\vbox to \ytfsize{\hrule\hbox to%
        \ytfsize{\vrule height \ytfsize\hskip.5ex%
         \raisebox{\raiseh}{\tiny$\thisel$}\hss}\vss\phrule{\ytfsize}}}%
 \fi\fi}}}}

% Typeset a Young tableau with unlabelled boxes. Takes a single 
% argument which is a string of numbers, one for the length of
% each row of the tableau. Example:
%
%   \tableau{{10}{8}{3}}
%
% typesets a tableau with 10 boxes in the 1st row, 8 in the 2nd
% and 3 in the 3rd. Curly brackets can be omitted if numbers
% are less than 10.

\newcommand\tableau[1]{%
\def\stest{ }
\def\Ktest{\^}
\acchspace=0ex
\accvspace=0ex
\maxw=0ex
\hbox{%
\@tabforc\thisel{#1}{%
 \ifx\thisel\Ktest{}
 \else
     \repcnt=\thisel%
     \loop{}%
     \advance\acchspace by-\ytsize%
     \raisebox{\accvspace}{\vbox to \ytsize{\hrule \hbox to%
			\ytsize{\vrule height \ytsize\hss}\vss\phrule{\ytsize}}}%
     \advance\repcnt by -1\ifnum\repcnt>1{}\repeat%
     \ifnum\maxw=0\maxw=\acchspace\fi%
     \raisebox{\accvspace}{\vbox to \ytsize{\hbox to 0pt{\vrule height \ytsize\hss}}}%
     \kern\acchspace\acchspace=0ex%
	  \advance\accvspace by -\ytsize%
 \fi}\kern-\maxw}}
 
\begin{document}
\section*{Geodesic deviation and acceleration in Schwarzschild}

This first section is inspired from the example notebook at \url{https://cadabra.science/notebooks/schwarzschild.html}, 
since we need many of the same quantities.
\begin{python}
{r,t,\phi,\theta}::Coordinate;
{\mu,\nu,\rho,\sigma,\lambda,\kappa,\chi,\gamma}::Indices(values={t,r,\phi,\theta}, position=fixed);
\partial{#}::PartialDerivative;
\nabla{#}::Derivative;
g_{\mu\nu}::Metric.
g^{\mu\nu}::InverseMetric.
u^\mu::Depends(\mu);
\end{python}
\begin{adjustwidth}{1em}{0cm}${}\text{Attached property Coordinate to~}\brwrap{[}{r,~ t,~ \phi,~ \theta}{]}.$\end{adjustwidth}
\begin{adjustwidth}{1em}{0cm}${}\text{Attached property Indices(position=fixed) to~}\brwrap{[}{\mu,~ \nu,~ \rho,~ \sigma,~ \lambda,~ \kappa,~ \chi,~ \gamma}{]}.$\end{adjustwidth}
\begin{adjustwidth}{1em}{0cm}${}\text{Attached property PartialDerivative to~}\partial{\#}.$\end{adjustwidth}
\begin{adjustwidth}{1em}{0cm}${}\text{Attached property Derivative to~}\nabla{\#}.$\end{adjustwidth}
\begin{adjustwidth}{1em}{0cm}${}\text{Attached property Depends to~}u^{\mu}.$\end{adjustwidth}
Below is the Schwarzschild metric in standard coordinates. Note how the components are
given in terms of substitution rules, and how the inverse metric is computed. 
The \algo{complete} algorithm adds the rules for the inverse metric to the rules for the metric.
\begin{python}
ss:= { g_{t t} = -(1-2 G M/r),
       g_{r r} = 1/(1-2 G M/r), 
       g_{\theta\theta} = r**2, 
       g_{\phi\phi}=r**2 \sin(\theta)**2
     }.

complete(ss, $g^{\mu\nu}$);
\end{python}
\begin{adjustwidth}{1em}{0cm}${}\brwrap{[}{g_{t t} = -1+2G M {r}^{-1},~ g_{r r} = {\brwrap{(}{1-2G M {r}^{-1}}{)}}^{-1},~ g_{\theta \theta} = {r}^{2},~ g_{\phi \phi} = {r}^{2} {\brwrap{(}{\sin{\theta}}{)}}^{2},~ g^{t t} = {\brwrap{(}{2G M {r}^{-1}-1}{)}}^{-1},~ g^{r r} = -2G M {r}^{-1}+1,~ g^{\phi \phi} = {\brwrap{(}{{r}^{2} {\brwrap{(}{\sin{\theta}}{)}}^{2}}{)}}^{-1},~ g^{\theta \theta} = {r}^{-2}}{]}$\end{adjustwidth}
We can now compute the Christoffel symbols. We give Cadabra the expression for the 
Christoffel symbols in terms of the metric, and then evaluate the components of the
metric using the \algo{evaluate} algorithm.
\begin{python}
ch:= \Gamma^{\mu}_{\nu\rho} = 1/2 g^{\mu\sigma} ( 
                                   \partial_{\rho}{g_{\nu\sigma}} 
                                  +\partial_{\nu}{g_{\rho\sigma}}
                                  -\partial_{\sigma}{g_{\nu\rho}} ):
                          
evaluate(ch, ss, rhsonly=True);
\end{python}
\begin{adjustwidth}{1em}{0cm}${}\Gamma^{\mu}\,_{\nu \rho} = \frac{1}{2}g^{\mu \sigma} \brwrap{(}{\partial_{\rho}{g_{\nu \sigma}}+\partial_{\nu}{g_{\rho \sigma}}-\partial_{\sigma}{g_{\nu \rho}}}{)}$\end{adjustwidth}
\begin{adjustwidth}{1em}{0cm}${}\Gamma^{\mu}\,_{\nu \rho} = \square{}_{\nu}{}_{\rho}{}^{\mu}\brwrap{\{}{\begin{aligned}\square{}_{\phi}{}_{r}{}^{\phi}& = {r}^{-1}\\[-.5ex]
\square{}_{\phi}{}_{\theta}{}^{\phi}& = {\brwrap{(}{\tan{\theta}}{)}}^{-1}\\[-.5ex]
\square{}_{\theta}{}_{r}{}^{\theta}& = {r}^{-1}\\[-.5ex]
\square{}_{r}{}_{r}{}^{r}& = G M {\brwrap{(}{r \brwrap{(}{2G M-r}{)}}{)}}^{-1}\\[-.5ex]
\square{}_{t}{}_{r}{}^{t}& = G M {\brwrap{(}{r \brwrap{(}{-2G M+r}{)}}{)}}^{-1}\\[-.5ex]
\square{}_{r}{}_{\phi}{}^{\phi}& = {r}^{-1}\\[-.5ex]
\square{}_{\theta}{}_{\phi}{}^{\phi}& = {\brwrap{(}{\tan{\theta}}{)}}^{-1}\\[-.5ex]
\square{}_{r}{}_{\theta}{}^{\theta}& = {r}^{-1}\\[-.5ex]
\square{}_{r}{}_{t}{}^{t}& = G M {\brwrap{(}{r \brwrap{(}{-2G M+r}{)}}{)}}^{-1}\\[-.5ex]
\square{}_{\phi}{}_{\phi}{}^{r}& = \brwrap{(}{2G M-r}{)} {\brwrap{(}{\sin{\theta}}{)}}^{2}\\[-.5ex]
\square{}_{\phi}{}_{\phi}{}^{\theta}& =  - \frac{1}{2}\sin\brwrap{(}{2\theta}{)}\\[-.5ex]
\square{}_{\theta}{}_{\theta}{}^{r}& = 2G M-r\\[-.5ex]
\square{}_{t}{}_{t}{}^{r}& = G M \brwrap{(}{-2G M+r}{)} {r}^{-3}\\[-.5ex]
\end{aligned}}{.}
$\end{adjustwidth}
Continuing from here we can compute the Riemann tensor components. Again, we start
by giving this tensor in terms of the Christoffel symbols. We then subsitute the 
Christoffel symbols just found, and work out any remaining component substitions using
\algo{evaluate} (the computation takes a few seconds, essentially because of the round
trips through Sympy).
\begin{python}
rm:= R^{\rho}_{\sigma\mu\nu} = \partial_{\mu}{\Gamma^{\rho}_{\nu\sigma}}
                                  -\partial_{\nu}{\Gamma^{\rho}_{\mu\sigma}}
                                  +\Gamma^{\rho}_{\mu\lambda} \Gamma^{\lambda}_{\nu\sigma}
                                  -\Gamma^{\rho}_{\nu\lambda} \Gamma^{\lambda}_{\mu\sigma};
\end{python}
\begin{adjustwidth}{1em}{0cm}${}R^{\rho}\,_{\sigma \mu \nu} = \partial_{\mu}{\Gamma^{\rho}\,_{\nu \sigma}}-\partial_{\nu}{\Gamma^{\rho}\,_{\mu \sigma}}+\Gamma^{\rho}\,_{\mu \lambda} \Gamma^{\lambda}\,_{\nu \sigma}-\Gamma^{\rho}\,_{\nu \lambda} \Gamma^{\lambda}\,_{\mu \sigma}$\end{adjustwidth}
\begin{python}
substitute(rm, ch)
evaluate(rm, ss, rhsonly=True);
\end{python}
\begin{adjustwidth}{1em}{0cm}${}R^{\rho}\,_{\sigma \mu \nu} = \square{}_{\nu}{}_{\sigma}{}^{\rho}{}_{\mu}\brwrap{\{}{\begin{aligned}\square{}_{t}{}_{t}{}^{r}{}_{r}& = 2G M \brwrap{(}{2G M-r}{)} {r}^{-4}\\[-.5ex]
\square{}_{\theta}{}_{\theta}{}^{r}{}_{r}& = -G M {r}^{-1}\\[-.5ex]
\square{}_{\phi}{}_{\phi}{}^{\theta}{}_{\theta}& = 2G M {\brwrap{(}{\sin{\theta}}{)}}^{2} {r}^{-1}\\[-.5ex]
\square{}_{\phi}{}_{\phi}{}^{r}{}_{r}& = -G M {\brwrap{(}{\sin{\theta}}{)}}^{2} {r}^{-1}\\[-.5ex]
\square{}_{t}{}_{r}{}^{t}{}_{r}& = 2G M {\brwrap{(}{{r}^{2} \brwrap{(}{2G M-r}{)}}{)}}^{-1}\\[-.5ex]
\square{}_{\phi}{}_{\theta}{}^{\phi}{}_{\theta}& = -2G M {r}^{-1}\\[-.5ex]
\square{}_{r}{}_{t}{}^{r}{}_{t}& = 2G M \brwrap{(}{-2G M+r}{)} {r}^{-4}\\[-.5ex]
\square{}_{r}{}_{\theta}{}^{r}{}_{\theta}& = G M {r}^{-1}\\[-.5ex]
\square{}_{\theta}{}_{\phi}{}^{\theta}{}_{\phi}& = -2G M {\brwrap{(}{\sin{\theta}}{)}}^{2} {r}^{-1}\\[-.5ex]
\square{}_{r}{}_{\phi}{}^{r}{}_{\phi}& = G M {\brwrap{(}{\sin{\theta}}{)}}^{2} {r}^{-1}\\[-.5ex]
\square{}_{r}{}_{r}{}^{t}{}_{t}& = 2G M {\brwrap{(}{{r}^{2} \brwrap{(}{-2G M+r}{)}}{)}}^{-1}\\[-.5ex]
\square{}_{r}{}_{r}{}^{\theta}{}_{\theta}& = G M {\brwrap{(}{{r}^{2} \brwrap{(}{2G M-r}{)}}{)}}^{-1}\\[-.5ex]
\square{}_{\theta}{}_{\theta}{}^{\phi}{}_{\phi}& = 2G M {r}^{-1}\\[-.5ex]
\square{}_{r}{}_{r}{}^{\phi}{}_{\phi}& = G M {\brwrap{(}{{r}^{2} \brwrap{(}{2G M-r}{)}}{)}}^{-1}\\[-.5ex]
\square{}_{t}{}_{t}{}^{\phi}{}_{\phi}& = G M \brwrap{(}{-2G M+r}{)} {r}^{-4}\\[-.5ex]
\square{}_{t}{}_{t}{}^{\theta}{}_{\theta}& = G M \brwrap{(}{-2G M+r}{)} {r}^{-4}\\[-.5ex]
\square{}_{\phi}{}_{\phi}{}^{t}{}_{t}& = -G M {\brwrap{(}{\sin{\theta}}{)}}^{2} {r}^{-1}\\[-.5ex]
\square{}_{\theta}{}_{\theta}{}^{t}{}_{t}& = -G M {r}^{-1}\\[-.5ex]
\square{}_{\phi}{}_{r}{}^{\phi}{}_{r}& = G M {\brwrap{(}{{r}^{2} \brwrap{(}{-2G M+r}{)}}{)}}^{-1}\\[-.5ex]
\square{}_{\phi}{}_{t}{}^{\phi}{}_{t}& = G M \brwrap{(}{2G M-r}{)} {r}^{-4}\\[-.5ex]
\square{}_{\theta}{}_{r}{}^{\theta}{}_{r}& = G M {\brwrap{(}{{r}^{2} \brwrap{(}{-2G M+r}{)}}{)}}^{-1}\\[-.5ex]
\square{}_{\theta}{}_{t}{}^{\theta}{}_{t}& = G M \brwrap{(}{2G M-r}{)} {r}^{-4}\\[-.5ex]
\square{}_{t}{}_{\phi}{}^{t}{}_{\phi}& = G M {\brwrap{(}{\sin{\theta}}{)}}^{2} {r}^{-1}\\[-.5ex]
\square{}_{t}{}_{\theta}{}^{t}{}_{\theta}& = G M {r}^{-1}\\[-.5ex]
\end{aligned}}{.}
$\end{adjustwidth}
Let us compute the geodesic deviation $d^\mu$ for an observer described by $\xi^\sigma$ 
which moves along a geodesic with four-velocity $u^\nu$.
\begin{python}
dev := d^{\mu} = R^{\mu}_{\nu \rho \sigma} u^{\nu} u^{\rho} \xi^{\sigma};

velocity := {
	u^{t} = 1 / \sqrt{1 - 2 G M / r},
	u^{r} = 0,
	u^{\phi} = 0,
	u^{\theta} = 0,
};

displacement := {
	\xi^{t} = 0,
	\xi^{r} = h \sqrt{1 - 2 G M / r},
	\xi^{\phi} = 0,
	\xi^{\theta} = 0,
};	

\end{python}
\begin{adjustwidth}{1em}{0cm}${}d^{\mu} = R^{\mu}\,_{\nu \rho \sigma} u^{\nu} u^{\rho} \xi^{\sigma}$\end{adjustwidth}
\begin{adjustwidth}{1em}{0cm}${}\brwrap{[}{u^{t} = {\brwrap{(}{1-2G M {r}^{-1}}{)}}^{ - \frac{1}{2}},~ u^{r} = 0,~ u^{\phi} = 0,~ u^{\theta} = 0}{]}$\end{adjustwidth}
\begin{adjustwidth}{1em}{0cm}${}\brwrap{[}{\xi^{t} = 0,~ \xi^{r} = h \sqrt{1-2G M {r}^{-1}},~ \xi^{\phi} = 0,~ \xi^{\theta} = 0}{]}$\end{adjustwidth}
\begin{python}
evaluate (dev, ss+velocity+displacement, rhsonly=True);
\end{python}
\begin{adjustwidth}{1em}{0cm}${}d^{\mu} = \square{}^{\mu}\brwrap{\{}{\begin{aligned}\square{}^{t}& = h R^{t}\,_{t t r} {\brwrap{(}{\sqrt{\brwrap{(}{-2G M+r}{)} {r}^{-1}}}{)}}^{-1}\\[-.5ex]
\square{}^{r}& = h R^{r}\,_{t t r} {\brwrap{(}{\sqrt{\brwrap{(}{-2G M+r}{)} {r}^{-1}}}{)}}^{-1}\\[-.5ex]
\square{}^{\phi}& = h R^{\phi}\,_{t t r} {\brwrap{(}{\sqrt{\brwrap{(}{-2G M+r}{)} {r}^{-1}}}{)}}^{-1}\\[-.5ex]
\square{}^{\theta}& = h R^{\theta}\,_{t t r} {\brwrap{(}{\sqrt{\brwrap{(}{-2G M+r}{)} {r}^{-1}}}{)}}^{-1}\\[-.5ex]
\end{aligned}}{.}
$\end{adjustwidth}
\begin{python}
acc_modulus := a = \sqrt{ a^{\mu} a^{\nu} g_{\mu \nu}};

substitute (acc_modulus, dev),
substitute (acc_modulus, rm),

evaluate(acc_modulus, ss+velocity+displacement, rhsonly=True);
\end{python}
\begin{adjustwidth}{1em}{0cm}${}a = \sqrt{a^{\mu} a^{\nu} g_{\mu \nu}}$\end{adjustwidth}
\begin{adjustwidth}{1em}{0cm}${}a = \sqrt{\brwrap{(}{{r}^{3} \brwrap{(}{2G M-r}{)} \brwrap{(}{{a^{\phi}}^{2} {\brwrap{(}{\sin{\theta}}{)}}^{2}+{a^{\theta}}^{2}}{)}-{r}^{2} {a^{r}}^{2}+{\brwrap{(}{2G M-r}{)}}^{2} {a^{t}}^{2}}{)} {\brwrap{(}{r \brwrap{(}{2G M-r}{)}}{)}}^{-1}}$\end{adjustwidth}
Let us then take an alternate approach: if we define the covariant derivative, 
we can calculate the four-acceleration of observers which are stationary in Schwarzschild coordinates.

The four-velocity $u^\nu$ is the same one we used before.
\begin{python}
deriv := \nabla_{\mu}{u^{\nu}} = \partial_{\mu}{u^{\nu}} + \Gamma^{\nu}_{\rho \mu} u^\rho;

four_acc := m^{\mu} = u^{\nu} \nabla_{\nu}{u^\mu};

substitute(four_acc, deriv);
substitute(four_acc, ch);
evaluate(four_acc, velocity+ss+ch, rhsonly=True);
\end{python}
\begin{adjustwidth}{1em}{0cm}${}\nabla_{\mu}{u^{\nu}} = \partial_{\mu}{u^{\nu}}+\Gamma^{\nu}\,_{\rho \mu} u^{\rho}$\end{adjustwidth}
\begin{adjustwidth}{1em}{0cm}${}m^{\mu} = u^{\nu} \nabla_{\nu}{u^{\mu}}$\end{adjustwidth}
\begin{adjustwidth}{1em}{0cm}${}m^{\mu} = u^{\nu} \brwrap{(}{\partial_{\nu}{u^{\mu}}+\Gamma^{\mu}\,_{\rho \nu} u^{\rho}}{)}$\end{adjustwidth}
\begin{adjustwidth}{1em}{0cm}${}m^{\mu} = u^{\nu} \brwrap{(}{\partial_{\nu}{u^{\mu}}+\square{}_{\rho}{}_{\nu}{}^{\mu}\brwrap{\{}{\begin{aligned}\square{}_{\phi}{}_{r}{}^{\phi}& = {r}^{-1}\\[-.5ex]
\square{}_{\phi}{}_{\theta}{}^{\phi}& = {\brwrap{(}{\tan{\theta}}{)}}^{-1}\\[-.5ex]
\square{}_{\theta}{}_{r}{}^{\theta}& = {r}^{-1}\\[-.5ex]
\square{}_{r}{}_{r}{}^{r}& = G M {\brwrap{(}{r \brwrap{(}{2G M-r}{)}}{)}}^{-1}\\[-.5ex]
\square{}_{t}{}_{r}{}^{t}& = G M {\brwrap{(}{r \brwrap{(}{-2G M+r}{)}}{)}}^{-1}\\[-.5ex]
\square{}_{r}{}_{\phi}{}^{\phi}& = {r}^{-1}\\[-.5ex]
\square{}_{\theta}{}_{\phi}{}^{\phi}& = {\brwrap{(}{\tan{\theta}}{)}}^{-1}\\[-.5ex]
\square{}_{r}{}_{\theta}{}^{\theta}& = {r}^{-1}\\[-.5ex]
\square{}_{r}{}_{t}{}^{t}& = G M {\brwrap{(}{r \brwrap{(}{-2G M+r}{)}}{)}}^{-1}\\[-.5ex]
\square{}_{\phi}{}_{\phi}{}^{r}& = \brwrap{(}{2G M-r}{)} {\brwrap{(}{\sin{\theta}}{)}}^{2}\\[-.5ex]
\square{}_{\phi}{}_{\phi}{}^{\theta}& =  - \frac{1}{2}\sin\brwrap{(}{2\theta}{)}\\[-.5ex]
\square{}_{\theta}{}_{\theta}{}^{r}& = 2G M-r\\[-.5ex]
\square{}_{t}{}_{t}{}^{r}& = G M \brwrap{(}{-2G M+r}{)} {r}^{-3}\\[-.5ex]
\end{aligned}}{.}
 u^{\rho}}{)}$\end{adjustwidth}
\begin{adjustwidth}{1em}{0cm}${}m^{\mu} = \square{}^{\mu}\brwrap{\{}{\begin{aligned}\square{}^{r}& = G M {r}^{-2}\\[-.5ex]
\end{aligned}}{.}
$\end{adjustwidth}
\begin{python}
four_acc_mod := m = \sqrt{m^\mu m^\nu g_{\mu \nu}};

substitute(four_acc_mod, four_acc);
evaluate(four_acc_mod, ss, rhsonly=True);
\end{python}
\begin{adjustwidth}{1em}{0cm}${}m = \sqrt{m^{\mu} m^{\nu} g_{\mu \nu}}$\end{adjustwidth}
\begin{adjustwidth}{1em}{0cm}${}m = \sqrt{\square{}^{\mu}\brwrap{\{}{\begin{aligned}\square{}^{r}& = G M {r}^{-2}\\[-.5ex]
\end{aligned}}{.}
 \square{}^{\nu}\brwrap{\{}{\begin{aligned}\square{}^{r}& = G M {r}^{-2}\\[-.5ex]
\end{aligned}}{.}
 g_{\mu \nu}}$\end{adjustwidth}
\begin{adjustwidth}{1em}{0cm}${}m = \sqrt{{G}^{2} {M}^{2} {\brwrap{(}{{r}^{3} \brwrap{(}{-2G M+r}{)}}{)}}^{-1}}$\end{adjustwidth}
\begin{python}
\end{python}
\end{document}
