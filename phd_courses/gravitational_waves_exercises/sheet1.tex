\documentclass[main.tex]{subfiles}
\begin{document}

\section{Quadrupole approximation}

The procedure to find the quadrupole approximation 
for the GW emitted by a Newtonian system is in the form: 
\begin{enumerate}
    \item determine the density \(\rho (\vec{x}, t)\);
    \item calculate the trace-free inertia tensor \(Q^{ij}(t)\);
    \item calculate the gravitational wave strain \(h_{ij}\). 
\end{enumerate}

\subsection{Two point particles}

In order to model the point-like nature of the particles we can write an expression for the density as a sum of two delta-functions,
whose locations rotate around an axis --- let us fix it to be
the \(z\) axis for simplicity, 
and also let us suppose that we are on the \(z = 0 \) plane: 
%
\begin{align}
\rho (\vec{x}, t) = m\delta (\vec{x} - \vec{x}_1(t)) + m\delta (\vec{x} - \vec{x}_2 (t))
\,,
\end{align}
%
where 
%
\begin{align}
\vec{x}_1 (t) = r \left[\begin{array}{c}
\cos(\omega t + \phi ) \\ 
- \sin(\omega t + \phi ) \\ 
0
\end{array}\right]
\qquad \text{and} \qquad
\vec{x}_2 (t) = - \vec{x}_1 (t)
\,.
\end{align}

For simplicity we will also assume that the radius \(r\) is constant --- this cannot be precisely the case, since there will be some amount of power lost because of the GW emission, but it is a reasonable approximation.

So, the trace-free inertia tensor will be given by 
%
\begin{align}
Q^{ij} (t) &= \int \rho (\vec{x}, t) \qty( x^{i} x^{j} - \frac{1}{3} \delta^{ij} r^2) \dd[3]{x}  \\
&= m \sum _{k=1, 2} \qty(x_k^{i} x_k^{j} - \frac{1}{3} \delta^{ij} r^2)  \\
&= 2 m r^2 \left[\begin{array}{cc}
\cos^2 - 1/3 & - \cos \sin \\ 
- \cos \sin & \sin^2 - 1/3  
\end{array}\right] 
\,,
\end{align}
%
where we write only the upper-left 2x2 submatrix in \(Q^{ij}\), since the other entries are zero, and we omit the argument of the sines and cosines (which is always \(\omega t + \phi \)). 

The second derivative of this tensor will be given by 
%
\begin{align}
\ddot{Q}^{ij}(t) = 4 m r^2 \omega^2 \left[\begin{array}{cc}
\sin^2 - \cos^2 & 2 \sin \cos \\ 
2 \sin \cos & \cos^2 - \sin^2
\end{array}\right]
\,,
\end{align}
%
 

Now, in order to compute the gravitational wave strain we need the projection tensor \(\Lambda_{ij, kl}\).
If the propagation direction we are interested in is \(\vec{k}\), then the tensor 
%
\begin{align}
P_{ij} = \delta_{ij} - \frac{k_i k_j}{\abs{k}^2}
\,
\end{align}
%
will project a vector onto the subspace orthogonal to \(\vec{k}\); and the tensor 
%
\begin{align}
\Lambda_{ij, kl} = P_{ik} P_{jl} - \frac{1}{2} P_{ij} P_{kl} 
\,,
\end{align}
%
will project a rank-2 tensor onto the corresponding subspace. 



\end{document}
