\documentclass[main.tex]{subfiles}
\begin{document}

\section{Quadrupole approximation}

The procedure to find the quadrupole approximation 
for the GW emitted by a Newtonian system is in the form: 
\begin{enumerate}
    \item determine the density \(\rho (\vec{x}, t)\);
    \item calculate the trace-free inertia tensor \(Q^{ij}(t)\);
    \item calculate the gravitational wave strain \(h_{ij}\). 
\end{enumerate}

\subsection{Two point particles}

The density in this case will be a sum of two delta-functions,
whose locations rotate around an axis --- let us fix it to be
the \(z\) axis for simplicity, 
and also let us suppose that we are on the \(z = 0 \) plane: 
%
\begin{align}
\rho (\vec{x}, t) = m\delta (\vec{x} - \vec{x}_1(t)) + m\delta (\vec{x} - \vec{x}_2 (t))
\,,
\end{align}
%
where 
%
\begin{align}
\vec{x}_1 (t) = \left[\begin{array}{c}
\cos(\omega t + \phi ) \\ 
- \sin(\omega t + \phi ) \\ 
0
\end{array}\right]
\qquad \text{and} \qquad
\vec{x}_2 (t) = - \vec{x}_1 (t)
\,.
\end{align}
%


\end{document}