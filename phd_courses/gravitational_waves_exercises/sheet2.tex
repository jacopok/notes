\documentclass[main.tex]{subfiles}
\begin{document}

\section{GW gauge}

\section{NS surface tidal forces}

We want to estimate the tidal forces on a \SI{1.6}{m} tall person standing on the surface of a nonrotating neutron star with \(M = 1.4 M_{\odot}\) and \(R = \SI{12}{km}\).\footnote{A first sanity check: the corresponding Schwarzschild radius is around \SI{4}{km}.}

We shall do so by comparing three different methods: Newtonian mechanics, weak-field GR and then full GR (still with the spherical, stationary NR assumption).
The weak-field GR formulation will make the various approximations of the Newtonian approach more explicit. 

\subsection{Newtonian mechanics}

Here, the gravitational acceleration \(a (r)\) looks like 
%
\begin{align}
a (r) = - \frac{GM}{r^2}
\,,
\end{align}
%
so the differential acceleration between the head and feet of our test person will be 
%
\begin{align}
\Delta a = - \frac{GM}{(r + h)^2} + \frac{GM}{r^2} \approx 2GM \frac{h}{r^3} + \order{ (h / r)^2}
\,,
\end{align}
%
where the scale of the term we are neglecting is \((h / r)^2 \sim (\num{e-4})^2 = \num{e-8}\) --- definitely acceptable for this rough calculation.

With this expression we get \(\Delta a \approx \SI{3.4e8}{m/s^2}\) (both in the exact case and with the approximation). 

\subsection{Weak-field GR}

In general, the expression for the geodesic deviation \(d^{\mu }\) corresponding to an observer described by a space-like vector \(\xi^{\mu }\) moving along a geodesic with four-velocity \(u^{\mu }\)  is \cite[eq.\ 3.208]{carrollSpacetimeGeometryIntroduction2019}
%
\begin{align}
d^{\mu } = R^{\mu }{}_{\nu \rho \sigma  } u^{\nu } u^{\rho } \xi^{\sigma }
\,.
\end{align}

Our stationary observer on the surface is not moving along a geodesic, but the acceleration due to being stationary in Schwarzschild coordinates is shared by head and feet --- it is the zeroth-order contribution, so to speak. 
The geodesic deviation \(d^{\mu }\) goes directly to the first order. 

In the weak-field approximation, we directly consider the component \(d^{r}\) to estimate \(\abs{\Delta a}\), since we assume (wrongly!) that the metric is not far off from the Minkowski one.
The space-like displacement vector describing the person is \(\xi^{\mu } = h \hat{e}^{r}\), while their four-velocity in this weak-field context is \(u^{\mu } = \hat{e}^{t}\). 

Therefore, the formula simplifies to 
%
\begin{align}
\abs{\Delta a} \approx \abs{d^{r}} = \abs{h R^{r}{}_{ttr} }
\,,
\end{align}
%
and we can approximate this component of the Riemann tensor as such: 
%
\begin{align}
-R^{r}_{ttr} =
+R^{r}_{trt} &= \partial_{r} \Gamma^{r}_{tt} - \underbrace{\partial_t \Gamma^{r}_{rt}}_{\text{zero by stationarity}}
\underbrace{\Gamma^{r}_{r \mu } \Gamma^{\mu }_{tt} - \Gamma^{r}_{t \mu } \Gamma^{\mu}_{r t}}_{\text{second order}}  \\
&\approx \partial_{r} \Gamma^{r}_{tt} = \frac{1}{2} \partial_{r} g^{r \mu } (\underbrace{2\partial_{t} g_{(\mu t)}}_{\text{zero in weak-field}}  - \partial_{\mu } g_{00})  \\
&\approx - \frac{1}{2} \partial_{r} \partial_{r} g_{00} 
\approx \partial_{r} \partial_{r} \Phi  \\
&\approx \partial_{r} \partial_{r} \qty(- \frac{GM}{r}) = \partial_{r} \frac{GM}{r^2} = - \frac{2GM}{r^3}
\,,
\end{align}
%
so our result is exactly the same as the Newtonian one: \(\Delta a = 2GM h / r^3 = \SI{3.4}{m / s^2}\). 

\subsection{Full GR}

Now that we have established the weak-field calculation, let us approach it accounting for the strong-field regime. 

The geodesic deviation equation is still valid, but we must adapt the expressions for \(\xi^{\mu }\) and \(u^{\mu }\): 
we want their moduli to be \(h\) and \(-1\) respectively, and their directions to be \(\hat{e}^{r} \) and \(\hat{e}^{t}\).
In order for this to be true we need to rescale their components as \(\xi^{r} = h / \sqrt{g_{rr}}\) and \(u^{t} = 1/\sqrt{g_{tt}}\). 

Also, we cannot only compute \(d^{r}\): we must consider the full vector \(d^{\mu }\) and compute its modulus \(\sqrt{d^{\mu } d_\mu }\). 
Notice the lowered index: the component \(d^{r}\) appears in the modulus of the acceleration as \(d^{r} \sqrt{g_{rr}}\), and \(\sqrt{g_{rr}} \approx \num{1.23}\) is noticeably different from 1 in this situation. 

This way, we have all the components to plug into a computer algebra system like Cadabra: 
the acceleration comes out to have \(d^{r}\) as its only nonvanishing component, 
%
\begin{align}
d^{r} = \frac{1}{r} \frac{h}{r} \frac{2GM}{r} \qty(1 - \frac{2GM}{rc^2})^{-1/2}
\,,
\end{align}
%
whose modulus is then 
%
\begin{align}
\abs{d} = d^{r} \sqrt{g_{rr}} = \frac{2GMh}{r^3} 
\,,
\end{align}
%
which, incidentally, is proportional to the square root of the Kretschmann scalar \(K = R^{\mu \nu \rho \sigma } R_{\mu \nu \rho \sigma } \propto G^2M^2c^{-4} r^{-6}\). 

This is the same result we found in the weak field regime! 
The factors from the metric cancelled out.

\subsection{Full GR, revised}

There is a problem in the computation we did before: we applied the formula for geodesic deviation despite the fact that our observer is not moving along a geodesic. 

An alternative way to calculate the tidal deformation would be to simply compute the acceleration at the feet and head and subtract them. 
Let us see how this approach goes. 

Both the head and feet are stationary in Schwarzschild coordinates; their four-velocities are therefore time-directed, and normalized like before as \(u^{\mu } = \hat{e}^{t} / \sqrt{g_{tt}}\).

The four-acceleration of an observer is the derivative of the four-velocity along itself (with respect to proper time): 
%
\begin{align}
a^{\mu } = \dv{u^{\mu }}{\tau } = u^{\nu } \nabla_{\nu } u^{\mu } 
\,,
\end{align}
%
whose only nonvanishing component is 
%
\begin{align}
a^{r} = \Gamma^{r}_{tt} \qty( u^{t})^2 = \frac{GM}{r^2}
\,,
\end{align}
%
but, like before, we need to account for the metric component if we want to compute its modulus: 
%
\begin{align}
\abs{a} = a^{r} \sqrt{ g_{rr}} = \frac{GM}{r^2} \frac{1}{\sqrt{1 - 2GM / r}}
\,.
\end{align}

Now, all that is left is to compute the difference between this quantity at \(r\) and \(r + h\): 
%
\begin{align}
\Delta a = 
\frac{GM}{(r + h)^2} \frac{1}{\sqrt{1- \frac{2GM}{c^2(r+h)}}}
- 
\frac{GM}{r^2} \frac{1}{\sqrt{1- \frac{2GM}{c^2r}}}
\,,
\end{align}
%
which could be approximated to linear order like before, 
but we can directly compute the difference: 
it comes out to be \(\Delta a \approx \SI{4.8e8}{m / s^2}\). 

This shows that, accounting for relativistic effects, 
an observer staying still on the surface of the star experiences \(\sim \SI{40}{\percent}\) more tidal acceleration.

\end{document}
