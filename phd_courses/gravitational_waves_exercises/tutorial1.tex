\documentclass[main.tex]{subfiles}
\begin{document}

\marginpar{Thursday\\ 2021-4-22, \\ compiled \\ \today}

We start from a linear approximation on a flat background; 
then we will see the GW energy, the quadrupole approximation, and the multipolar expansion. 

Sources are among BBH, general BCO, hyperbolic encounters, 
self-rotating objects, supernovae, cosmological background... 

\section{Inertia tensor}

Free fall in Newtonian gravity: 
%
\begin{align}
\frac{1}{2} m \dot{r}^2 + \frac{GMm}{r} =0
\,
\end{align}
%
leads to 
%
\begin{align}
\dot{r} = - c \sqrt{ \frac{r_s}{r}}
\,.
\end{align}

This is solved by 
%
\begin{align}
r(t) = r_s^{1/3} (t_0 - t)^{2/3} (3/2)^{2/3} c^{2/3}
\,,
\end{align}
%
so the component \(I_{11} \) can be calculated from the center of the planet: 
%
\begin{align}
I_{11} = m r^2
\,,
\end{align}
%


\end{document}
