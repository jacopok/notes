\documentclass[main.tex]{subfiles}
\begin{document}

\section*{Exercise 4.1}

The detector frame is \emph{locally flat}:
we choose coordinates so that \(g_{\mu \nu } (x_0 ) = \eta_{\mu \nu }\) and \(\Gamma^{\mu }_{\nu \rho } (x_0) = 0\). 

This is a \emph{non-relativistic} frame. 
The corrections around a point will be of the order (distance to that point) squared. 

By making all the approximations we get 
%
\begin{align}
\dv[2]{\xi^{i}}{\tau } = \xi^{\lambda } \partial_{\lambda } \Gamma^{i}_{00} \qty(\dv{x^{0}}{\tau })^2 = 0
\,.
\end{align}

The Riemann tensor is \emph{invariant} under gauge transformations in linearized GR.

In the TT gauge we have 
%
\begin{align}
R_{i0j0} = - \frac{1}{2 c^2} \ddot{h}_{ij}
\,,
\end{align}
%
so 
%
\begin{align}
\ddot{\xi}^{i} - \frac{\ddot{h}_{ij}}{2} \xi_{j} = 0
\,,
\end{align}
%
which is very similar to a Newtonian force, with 
%
\begin{align}
F_{i} = m \ddot{\xi}_{i} = \frac{m}{2} \ddot{h}_{j} \xi_{j}
\,.
\end{align}



\end{document}
