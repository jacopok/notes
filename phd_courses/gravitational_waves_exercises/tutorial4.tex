\documentclass[main.tex]{subfiles}
\begin{document}

\marginpar{Wednesday\\ 2021-5-26, \\ compiled \\ \today}

Circular orbits: there is instability, so the frequency evolves. 

We have an expression for the radiated energy: 
%
\begin{align}
E = \frac{c^{4}}{32 \pi G} \int \expval{ \partial_0 h_{ij} \partial_0 h_{ij}} \dd[3]{x}
\,,
\end{align}
%
and the plus-polarized waveform is 
%
\begin{align}
h_{+} (t) &= \frac{4 G \mu \Omega^2R^2}{r c^{4}} \qty(\frac{1 + \cos^2 \theta }{2})  \cos( 2 \Omega t ) \\
h_{\times } (t) &= \frac{4 G \mu \Omega^2 R^2}{r c^{4}} 
\cos(2 \Omega t)
\,.
\end{align}
%

A good expression is the 00 component of the Isaacson tensor, but one we can also use is 
%
\begin{align}
\frac{ \dd{E}}{ \dd{t} \dd{\Omega }} = \frac{r^2 c^3}{16 \pi G}
\expval{\dot{h}_+^2 + \dot{h}_{\times}^2} = \dv{P}{\Omega } 
\,.
\end{align}

We are in the Newtonian approximation, so at every moment \(\Omega^2 = GM / R^3\) holds. 

The energy of a binary system, in the Newtonian approximation, is simply 
%
\begin{align}
E = - \frac{G m_1 m_2 }{r}
= - \sqrt[3]{G^2 \mathcal{M}^{5} \Omega^2 / 8}
\,,
\end{align}
%
because of the Virial Theorem, where 
%
\begin{align}
\mathcal{M} = \frac{(m_1m_2 )^{3/5}}{(m_1 + m_2 )^{1/5}} = M \nu^{3/5}
\,
\end{align}
%
is the chirp mass, while \(\nu = m_1 m_2 / M^2 = \mu / M\) is the symmetric mass ratio. 

This yields the expression 
%
\begin{align}
\dot{E} = \frac{32}{5} \frac{G}{c^{5}} \mu^2 m^{3/4} \Omega^{10/3}
\,.
\end{align}

The energy balance dictates that \(\dot{E} = - P \): the change in energy of the binary system corresponds to the emitted power. 

Another approximation we need is \(\dot{\Omega} \ll \Omega^2\): each orbit is almost circular, and the evolution is slow. 
This means that we can write 
%
\begin{align}
\dot{E} = \frac{32}{5} \frac{c^{5}}{G} \qty( \frac{G \mathcal{M} \Omega }{c^3})^{10/3}
\,,
\end{align}
%
therefore 
%
\begin{align}
\dot{E} &= \dv{t} \qty[- \qty(\frac{G \mathcal{M}^{5} \Omega^2}{8})^{1/3}]  \\
&= \frac{G^{2/3} \mathcal{M}^{5/3}}{2} \dv{t} \qty( \Omega^{2/3} )
\,,
\end{align}
%
therefore find 

\end{document}
