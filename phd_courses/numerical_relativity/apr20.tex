\documentclass[main.tex]{subfiles}
\begin{document}

\section{Introduction}

What is the problem in NR?
We have the Einstein equation \(G_{ab} = 8 \pi T_{ab}\), in geometric units \(c = G = 1\), and we want to discuss its solutions which cannot be expressed analytically:
\begin{enumerate}
    \item gravitational collapse;
    \item BH or NS collisions;
    \item dynamical stability of some stationary solutions --- for example, Kerr black holes. 
\end{enumerate}

The commonalities among these are: strong gravity, the absence of symmetries (no Killing vectors), and the fact that these are dynamical. 

We need to: 
\begin{enumerate}
    \item  formulate the EFE as a PDE system;
    \item check that it is well-posed;
    \item simulate this using certain numerical algorithms;
    \item extract information: we should use gauge-invariant quantities, such as gravitational waves or to-be-specified ``energies''. 
\end{enumerate}

Regarding the third point, we need to be able to solve nonlinear PDES: these can be elliptic or hyperbolic, deal with space-time decomposition, deal with singularities, and use high-performance computing effectively. 

\subsection{The PDE system}

Let us start by treating the equations in vacuo: they reduce to \(R_{\mu \nu } = 0 \).  The Ricci tensor can be expressed explicitly as 
%
\begin{align}
0 = R_{\mu \nu } &= \underbrace{- \frac{1}{2} g^{\alpha \beta } \partial_{\alpha } \partial_{\beta } g_{\mu \nu } - g_{\alpha (\mu } \partial_{\nu )} H^{\alpha }}_{\text{principal part}} + Q_{\mu \nu } [g, \partial g]  \\
H^{\alpha } &= \partial_{\mu } g^{\alpha \mu } + \frac{1}{2} g^{\alpha \beta } g^{\rho \sigma } \partial_{\beta } g_{\rho \sigma }
\,,
\end{align}
%
where the part denoted as the principal contains the highest derivatives of the metric, while the part denoted as \(Q\) is less important. 

It seems like we have 10 equations for \(g_{\mu \nu }\), so we are done! We actually do not, because of the Bianchi identities: \(\nabla_{a} G^{ab} = 0\). These are four more equations, which we need to consider. 

Our questions are: 
\begin{enumerate}
    \item what type of PDEs are these?
    \item how do we formulate an initial/boundary value problem?
    \item are these PDE problems well-posed?
\end{enumerate}

\begin{definition}[Well-posed PDE problem]
It is a problem in which a unique solution exists, it is continuous, and it depends continuously on the boundary data. 
\end{definition}

\subsection{Maxwell equations in flat spacetime}

Let us start with these as a reference. They are written in terms of the antisymmetric Maxwell-Faraday tensor \(F_{\alpha \beta }\), and they read 
%
\begin{align}
0 = \partial^{\alpha } F_{\alpha \beta } = \partial^{\alpha } \qty( \partial_{\alpha} A_\beta - \partial_{\beta } A_\alpha ) 
\,,
\end{align}
%
where \(A^{\alpha }\) is the vector potential.
We could naively interpret these as four wave-like equations for the vector potential because of the \(\square A\) operator 
--- however, this is not true. 
Let us show it by looking at the \(\beta = 0\) equation: 
%
\begin{align}
0 &= \partial^{\alpha } \partial_{\alpha } A_0 - \partial^{\alpha } \partial_0 A_\alpha = \square A_0 - \partial_0 \partial^{\alpha } A_\alpha   \\
&= - \partial_0^2 A_0 + \partial_{i} \partial^{i} A_0 + \partial_0^2 A_0 - \partial_0 \partial^{i} A_i  \\
&= \partial^{i} \qty(\partial_{i} A_0 - \partial_{i} A_0 ) = \partial^{i} F_{0i} = \partial^{i} E_i = C
\,,
\end{align}
%
where \(E_\alpha = F_{\alpha 0} = F_{\alpha \beta } n^\beta \), where \(n^{\beta }\) is a unit four-vector in the time direction.

Importantly, this equation (\(C=0\)) does not contain second time derivatives (they cancelled out).
Can we get a wave equation from \(\partial^0 C = 0\), maybe? No: the computation goes like 
%
\begin{align}
\partial^{0} C &= \partial^{0} \qty(\partial^{\alpha } \partial_{\alpha } A_0 - \partial_{0} A_\alpha )  \\
&= \partial^{i} \qty[\partial^{\alpha } (\partial_{\alpha } A_i - \partial_{i} A_\alpha )]
\,,
\end{align}
%
where we used the fact that \(\partial^{\alpha } \partial^{\beta } F_{\alpha \beta } = 0\).
The thing we are taking a derivative of is the left-hand side of the Maxwell equations for \(\beta = i\)! 
If \(A\) is a solution, the equation \(\partial^{0} C= 0\) is an \emph{identity}, it does not constrain the solution in any way. 

In summary, the Maxwell equations for \(A^{\alpha }\) are 
3 evolution equations (which contain second time derivatives) 
and the one \(C = 0\) equation, which is a \emph{constraint.}

What this means is that the Maxwell equations are \textbf{undetermined}! 
Three evolution equations, four unknowns. 
They are not well posed according to our mathematical definition. 

However, from a physical point of view this is not a problem: 
we know that there is gauge freedom in electromagnetism,
so the couple \(E_\alpha \), \(B_\alpha \) are calculated from \(A_\alpha \) up to a gauge transformation. 
\(A_\alpha \) and \(A_\alpha + \partial_{\alpha } \phi \) for any scalar field \(\phi \) represent the same electric and magnetic fields.

What we can then do is to fix the gauge: 
we can choose, for example the Lorentz gauge \(\partial_{\alpha } A^{\alpha } = 0\). 

After doing so, the Maxwell equations reduce to \(0 = \partial_{\alpha } \partial^{\alpha } A_\beta =0 \). 
These are 4 dynamical equations (containing \(\partial_0 \partial_0 \)) for 4 unknowns \(A_\alpha \). 
This, then is a well-posed Cauchy problem. 

Does the Lorentz Gauge hold for all time?
Yes, because 
%
\begin{align}
0 = \partial^{\beta } \qty(\square A_\beta ) = \square (\partial^{\beta } A_\beta )
\,,
\end{align}
%
so the quantity in parentheses is zero initially and its derivative is always zero: then, it remains zero. 

What happened to the \(C =0 \) constraint? 
If \(C =0\) initially then the condition \(C = 0\) remains: 
%
\begin{align}
C = \square A_\alpha - \partial^{\alpha } \partial_{0} A_\alpha = \square A_\alpha - \partial_0 \partial^{\alpha } A_\alpha = 0
\,,
\end{align}
%
which is usually stated as:
``\textbf{the constraints are transported along the dynamics}''.

Finally then we can say that the Maxwell equations, in an appropriate gauge, are well-posed.

\subsection{The EFE}

In the EFE case, the Bianchi identities have the same role as our \(C = 0\) constraint. 

If \(n^b  \) is a timelike vector field, then the projection of the Einstein equations 
%
\begin{align}
0 = G_{ab} n^a = C_b [\partial_{i}^2 g, \partial g, g]
\,,
\end{align}
%
are \emph{constraint} equations.
On the other hand, the Bianchi identities guarantee that these are transported along the dynamics. 

If we pick a gauge such that \(0 = C^{\mu } = G^{0 \mu }\), then the Bianchi identities read 
%
\begin{align}
\nabla_{\alpha } G^{\alpha \mu } &= 0  \\
\partial_0 G^{0 \mu } &= - \partial_{k} G^{k \mu } - \Gamma^{\mu }_{\alpha \beta } G^{\alpha \beta } - \Gamma^{\alpha }_{\alpha \rho  } G^{\mu \rho }
\,,
\end{align}
%
and they contain at most \(\partial_0^2 g\). 

Can we write the Einstein equations in a way that makes them explicitly well-posed? 
Let us ignore the term \(Q_{\mu \nu }\) --- the important part is the principal one. 
The first term is already \(g^{\alpha \beta } \partial_{\alpha } \partial_{\beta }\), the Dalambertian. 
The part we do not like is the derivative of \(H^{\alpha }\). 

Can we say that \(H^{\alpha } = 0\)? Yes! This is the Hilbert / Lorentz / Harmonic gauge. 
Then the EFE read 
%
\begin{align}
\square_g g_{\mu \nu } \simeq 0
\,,
\end{align}
%
where the \(\simeq\) sign denotes the fact that this is only considering the principal part. 

\subsection{Causality and globally hyperbolic spacetime}

What we have seen so far is still not enough to show that the PDE system is, as a whole, hyperbolic. For that, we need to consider the eingenvalues of the problem. 

We have initially assumed a ``global motion of time''. 

Note that we do not mean a specific global time: in SR we had such a notion, because of the light-cone structure. 
An equation like \(\square_f \phi = 0\) is always ``clear'', its character is always determined by the lightcone structure --- the condition at a point is affected by the past light cone, and it affects the future lightcone.

In GR this is not the case in general! 
An example are closed timelike curves. 

In order to not have this problem, we need to restrict ourselves to a smaller class of spacetimes. 

\begin{definition}
An achronal set \(S \subset M\) if made of events which are not connected by timelike curves. 
\end{definition}

\begin{definition}
The future domain of dependence \(D_+ (S)\) is the set of events such that every causal curve starting from a point in \(D_+ (S)\) intersects \(S\) in the past.
\end{definition}

\begin{definition}
The future Cauchy horizon \(H_+ (S)\) is the boundary of \(D_+ (S)\). 
\end{definition}

We can make an analogous definition by substituting the past for the future, and the plus for a minus. 

\begin{definition}
The domain of dependence of \(S\) is \(D(S) = D_+ (S) \cup D_- (S)\). 
\end{definition}

\begin{definition}
A \textbf{Cauchy surface} is a hypersurface \(\Sigma \subset M\) of spatial character such that \(D(\Sigma ) = M\), or \(H(\Sigma ) = 0\).  
\end{definition}

A property of Cauchy surface is that every causal curve intersects \(\Sigma \) exactly once. 

Then we can say that 
\begin{definition}
A manifold \((M, g)\) is \textbf{globally hyperbolic} iff there exists a Cauchy surface \(\Sigma \subset M\). 
\end{definition}

This is a key hypothesis in numerical relativity. 
We restrict our class of solutions to globally hyperbolic spacetimes. 

There is a theorem ensuring that the initial value problem \(\square_G \phi  = 0 \) is well-posed in globally hyperbolic spacetimes. 



\end{document}