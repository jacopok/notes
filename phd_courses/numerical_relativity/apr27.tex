\documentclass[main.tex]{subfiles}
\begin{document}

\section{3 + 1 geometry}

\marginpar{Tuesday\\ 2021-4-27, \\ compiled \\ \today}

We take our 4D spacetime \((M, g, \nabla)\) and equip it with a scalar field \(t \colon M \to \mathbb{R}\) such that the vector
%
\begin{align}
(\dd{t})^{a} = g^{ab} (\dd{t})_b
\,
\end{align}
%
is timelike.

This defines 3D spatial hypersurfaces, whose normal vector is \(n^a = - \alpha (\dd{t}^{a})\).

These surfaces are denoted as \(\Sigma _t\), and their tangent vectors \(v^{a} \in T_P(\Sigma _t)\) are all spacelike. 

\textbf{Embedding} is a bijective map \(\phi \) from an \(n-1\) dimensional manifold \(\hat{\Sigma}\) to a subset \(\Sigma \) of an \(n\) dimensional manifold \(M\).

We can move tensor fields from \(\hat{\Sigma}\) to \(M\) and back: these are known as the pullback and pushforward operations.
Not all fields can be moved in this way. 

The idea is to identify \(\Sigma \) with the manifold \(\hat{\Sigma}\) --- however, they are distinct conceptually. 

A metric \(\gamma \) on \(\Sigma \) is given by the pullback of \(g\) on \(M\): this is the \textbf{induced} metric \(\gamma = \phi^{*} g\). 
The components of this metric are calculated as 
%
\begin{align}
\gamma_{ij} = \pdv{x^{\alpha }}{x^{i}} \pdv{x^{\beta }}{x^{j}} g_{\alpha \beta }
\,.
\end{align}

Also, we can define a connection \(D\) and a Riemann tensor \(\mathcal{R}\) on the submanifold; its components are \(\mathcal{R}_{ijkl}\). 
This is the ``internal'', or ``intrinsic curvature'' of the submanifold.

There is another curvature we could define: how does \(\Sigma \) deform into \(M\)? This is the \textbf{extrinsic curvature}. 

For example, a 1D curve has zero intrinsic curvature, but it can bend. 
This is determined by the map 
%
\begin{align}
K \colon T_P (\Sigma ) \times T_P (\Sigma ) \to \mathbb{R}
\,,
\end{align}
%
which acts on two vectors \(u, v \in T_P(\Sigma )\) as \(K (u, v) = K_{ab} u^a v^b = - u_a v^b \nabla_b n^a\).

As an example, take \(M = \mathbb{R}^3\) and \(\Sigma = \mathbb{R}^2\); the metric on \(M\) is the identity and the Riemann tensor vanishes, so we get \(\mathcal{R} = K = 0\). 

If, instead, we take \(\Sigma = C^2\), the surface of a cylinder, we will have \(\mathcal{R}_{ijkl} = 0\) since we can deform it into a plane, while 
%
\begin{claim}
the extrinsic curvature of this objects has a nonvanishing component \(k_{\varphi \varphi } = - a\), where \(a\) is the radius, while the trace reads \(k = k^{i}_i = - 1/a\). 
\end{claim}

On the other hand, if we take a sphere we have both \(\mathcal{R}_{ijkl} \neq 0 \) and \(k_{ij} = 0\); also, the trace reads \(k = - 2 / a\). 

We can use \textbf{projectors} to decompose tensors on \(M\) into tensors on \(\Sigma \) and ``pieces'' along \(n\). 
This is because we can decompose the tangent space of the ambient manifold into 
%
\begin{align}
T_P(M) = V_P (n) \oplus T_P (\Sigma )
\,,
\end{align}
%
so that a vector \(v^{\alpha }\) is written as \(v_\perp n^a + v_\parallel^{a}\), where \(v_\parallel \in T_P (\Sigma )\). 

We can define a projector \(P\) that maps a vector \(v^{a}\) to \(v_\parallel^{a}\). Explicitly, this will look like 
%
\begin{align}
P^{a}{}_b = \delta^{a}{}_b + n^a n_b
\,.
\end{align}

The induced metric can alternatively be written as the projection of the four-metric into \(\Sigma \): 
%
\begin{align}
\gamma_{ab} = P_a^{c} P_b^{d} g_{cd} = g_{ab} + n_a n_b
\,.
\end{align}

This is coordinate independent! 
Also, we can write the projectors as 
%
\begin{align}
P^{a}_{b} = \gamma^{a}_{b} = g^{ac} \gamma_{cb}
\,,
\end{align}
%
therefore we will not write \(P\) anymore --- we can just use \(\gamma \). 

We also need the covariant derivative: schematically, it is written as
%
\begin{align}
D T = P \dots P \nabla T
\,.
\end{align}

Also, the extrinsic curvature reads 
%
\begin{align}
K_{ab} = - \gamma_{a}^{c} \gamma^{d}_{b} \nabla_{(c} n_{d)}
\,.
\end{align}
%

Using \(P\) on every tensor is what ``putting into 3+1 form'' means. 

\subsection{Eulerian observers}

These are observers whose worldlines are defined by \(n\): \(\Sigma _t\) is the set of all events which are simultaneous to the Eulerian observers. 

\begin{definition}
The acceleration of E. observers is defined as 
%
\begin{align}
a_{a} = n^b \nabla_b n_a
\,,
\end{align}
%
therefore \(a_a n^a = 0\). 
\end{definition}

\begin{definition}
The normal evolution vector \(m^a = \alpha n^a\) is defined so that 
%
\begin{align}
\nabla_m t = m^a (\dd{t})_a = +1
\,.
\end{align}
\end{definition}

\begin{claim}
3+1 geometry defines the kinematic of 3+1 GR.
Specifically, the claims we make are:
\begin{enumerate}
    \item The vector \(m\) carries points from \(\Sigma _t\) to \(\Sigma _{t + \delta t}\). 
    \item The lapse function \(\alpha \) relates \(t\) to the proper time of Eulerian observers. 
    \item The Lie derivative along \(m\), \(\mathscr{L}_m\), transports tensors from \(\Sigma _t\) to \(\Sigma _{t + \delta t}\). 
    \item \(\mathscr{L}_m \gamma = - 2 \alpha K\). 
\end{enumerate}
\end{claim}

Let us give some hints: if \(P\) is on \(\Sigma _t\) and \(P'\) is on \(\Sigma _{t + \delta t}\), then 
%
\begin{align}
t(P') = t (P + \delta t \cdot m) = t(p) + \delta t \underbrace{m^a (\dd{t})_a }_{=1} = t(P) + \delta t
\,,
\end{align}
%
while for the second point, the change in proper time reads 
%
\begin{align}
\dd{\tau^2} = - g (\delta t m, \delta t m) = - m^a m_a \dd{t^2} = \alpha^2 \dd{t^2}
\,,
\end{align}
%
which is the reason for the term ``lapse function''. 

The third point follows from the first and the definition of the Lie derivative. 

As for the fourth, we have 
%
\begin{align}
\mathscr{L}_n \gamma_{ab} &= \mathscr{L}_n \qty(g_{ab} + n_a n_b)  \\
&= 2 \nabla_{(a} n_{b)} + n_a \mathscr{L}_n n_b + n_b \mathscr{L}_n n_a  \\
&= 2 \qty[\nabla_{(a} n_{b)} + n_{(a} a_{b)}] = - 2 K_{ab}
\,.
\end{align}

The last passage is left as an exercise: we start by carrying out the projections, then simplify terms in the form \(n n \nabla n\) by substituting the acceleration. 

The Lie derivative along \(n\) can always be written as 
%
\begin{align}
\mathscr{L}_n \gamma_{ab} = \phi^{-1} \mathscr{L}_{\phi n} \gamma_{ab}
\,
\end{align}
%
for any scalar field. 

Now we have three expressions for \(K_{ab}\) --- each of them can be taken to be the definition, and the others can be derived from it. 

What is the physical interpretation of these expressions?
They tell us several things: 
\begin{enumerate}
    \item \(\Sigma _t\) and \(\Sigma _{t + \delta t}\) are identified by the diffeomorphism generated by \(m\);
    \item the spacetime \((M, g)\) is the ``time'' development of \((\Sigma , \gamma )\), where the ``time evolution'' is governed by \(\mathscr{L}_m\);
    \item we can identify \(\gamma \) as the ``main variable'' of (3+1) GR, and also we identify \(K\) as the ``velocity'' of \(\gamma \): the equation \(\mathscr{L}_m \gamma = - 2 \alpha K\) is in the form ``time derivative of variable = velocity'', a kinematic equation.
\end{enumerate}

This all hinges on the possibility to define these non-intersecting hypersurfaces. 


\end{document}
