\documentclass[main.tex]{subfiles}
\begin{document}

\section{Initial Data Problem}

\marginpar{Tuesday\\ 2021-6-1, \\ compiled \\ \today}

It is perhaps the most difficult problem currently. 

The problem is stated as 
%
\begin{align}
C_0 &= R + K^2 - K_{ij} K^{ij} - 16 \pi E = 0  \\
C_i &= D_j K^{j}_{i} - D_i K - 8 \pi P_i = 0
\,,
\end{align}
%
and we want to calculate \(\gamma_{ij}\) and \(K_{ij} \) on \(\Sigma _0\) such that 
\begin{enumerate}
    \item the constraints are satisfied;
    \item the solution is physically meaningful: it really describes a BH, BBH, \(n\)-BH, NS spacetime.
\end{enumerate}

We will assume for today that the matter fields are given or zero --- in general one would have to solve the hydrodynamic equations for them as well.

We have 4 equations for 12 unknowns!
We will need to prescribe 8 quantities.

The problem is split in two: a problem of constrained data (4)
and one of free data (8). 

The choice of free data is guided by a few principles: 
\begin{enumerate}
    \item we need to define our (astro)physical expectations for \(\Sigma_0 \) and verify that they are matched;
    \item we need some heuristic or intuition for the various fields;
    \item mathematical (/ computational?) necessity: the equations \(C_\alpha = 0\) should be as nice as possible --- linearity, decoupling, well-posedness are all desirable. 
\end{enumerate}

The \textbf{Conformal Decomposition} allows us to implement these. 
Within it, there are two main formalisms: one is the ``conformal/transverse traceless'' formalism (CTT), due to York in 1973; the other is the ``conformal/thin sandwich'' (CTS), also due to York in 1999.
Both are based off of early work by Lichnerowicz. 

The CTS formalism is often used to generate binary systems --- it makes it easier to implement the quasi-symmetries there. 

We start from the Lichnerowicz equation (the Hamiltonian constraint): 
%
\begin{align}
C_0 &= \widetilde{D}_i \widetilde{D}^{i} \psi 
- \frac{1}{8} \hat{A}_{ij} \hat{A}^{ij} \psi^{-7} +
\qty(- \frac{K^2}{12} + 2 \pi E) \psi^{5} &= 0
\,,
\end{align}
%
with \(p = -10\). 
Let us solve it as a standalone equation: it is a Poisson-like one, since \(\widetilde{D}_i \widetilde{D}^{i}\) is similar to a Laplacian. 
Then, this is some sort of elliptic operator applied to \(\psi \) equated to some powers of \(\psi \). 

If \(K^2= 0\) the equation simplifies, and the BVP can be studied: there are several known results about the well-posedness of this equation under the hypothesis \(K = \const\).

These are known as Constant Mean Curvature (CMC) spacetimes. 

If we take an Asymptotically Flat, CMC spacetime with \(K = 0\), and with \(E = 0\), then the BVP with the Lichnerowicz equation is \emph{solvable} for a ``large'' class of metrics \(\widetilde{\gamma}_{ij}\) (a ``positive Yamabi class''). 

There is a ``prototype equation'' for the Lichnerowicz one if we set \(K = 0\): 
%
\begin{align}
\triangle u = f^2 u^{p}
\,,
\end{align}
%
in flat spacetime, for some function \(f\).

These theorems are found in somewhat specialized PDE literature.
In the linear case \(p = 1\), and we take \(\eval{u}_{\partial \Omega } = 0\), then the solution, identically zero, is unique. 

In the nonlinear case, the solution is unique iff \(p > 0\) (or specifically, the same sign of the coefficient of the \(f^2 u^{p}\) term). 

So, what about the Lichnerowicz equation? 
The easiest thing to do is to linearize the equation: we write it as 
%
\begin{align}
\widetilde{\triangle} \psi + H (\psi ) = 0
\,.
\end{align}

We write \(\psi = \psi_0 + \epsilon \), and expand 
%
\begin{align}
H(\psi ) = H(\psi_0 ) + \eval{\pdv{H}{\epsilon }}_{0} \epsilon + \order{\epsilon^2}
\,,
\end{align}
%
so we find an equation in the form \(\widetilde{\triangle} \epsilon = f \epsilon \) with: 
%
\begin{align}
f = \frac{1}{8 } \widetilde{R} + \frac{7}{8} \hat{A}_{ij} \hat{A}^{ij} \psi_0^{-8} - 10 \pi E \psi_0^{4}
\,.
\end{align}

This immediately shows us the problem: if \(K =0 \) then \(\widetilde{R} > 0\), but we can check that \(f\) is not positive, because of the ``matter term''! 

However, we insist on solving this equation, and we can do so as long as we \emph{rescale} the matter term. 

If we define \(\widetilde{E} = \psi^{s} E\), then, the term becomes 
%
\begin{align}
- 5 \times 2 \pi E \psi_0^{4} \to - (s \times 5) 2 \pi \widetilde{E} \psi^{4-s}
\,,
\end{align}
%
so this is \(> 0\) for any \(s > 5\). 
This is \emph{not} a trick: in practice, these elliptic equations are solved by iteration --- specifying a guess, and then iterating. 

Specifying the ``correct'' RHS therefore is key if we want to find a solution. 
So, what do we use? 
The best answer turns out to be 8: if we do this, we can express the dominant energy condition in conformal variables: if
%
\begin{align}
\widetilde{E} = \psi^{8} E
\qquad \text{and} \qquad
\widetilde{P}^{i} = \psi^{10} P^{i}
\,,
\end{align}
%
then \(\widetilde{E}^2 \geq \widetilde{P}^2\) implies \(E^2 \geq P^2\). In general this would read 
%
\begin{align}
\psi^{2s} E^2 \geq \psi^{-4} \psi^{10} \psi^{10} P^2
\,.
\end{align}

\subsection{Maximal Slicing}

What does it mean to impose \(K = 0\)? 
As we will see shortly, this is a gauge condition which extremizes the volume of \(\Sigma _t\). 

We just need to remember the definition of the trace: it is an identity that
%
\begin{align}
K = \gamma^{ab} K_{ab} = - \frac{1}{2 \alpha } \gamma^{ij} \mathscr{L}_m \gamma_{ij} = - \frac{1}{2 \alpha } \mathscr{L}_m \log \gamma 
\,,
\end{align}
%
which can be nicely written as 
%
\begin{align}
K = - \frac{1}{\alpha } \mathscr{L}_m \log \sqrt{\gamma } 
\,,
\end{align}
%
so we recover the volume element on \(\Sigma _t\), \(\sqrt{\gamma }\)! 

The volume in a certain region is \(V = \int \sqrt{\gamma } \dd[3]{x}\), so if we perform a variation \(v^{a} = \delta t \qty(\alpha n^a + \beta^{a})\), so that \(\eval{v^{a}}_{S} = 0\) (the deformation is zero at the boundary), the volume changes as such: 
%
\begin{align}
\fdv{V}{t} &= \int \partial_{t} \sqrt{\gamma } \dd[3]{x}
= \int \qty(- \alpha K + D_i \beta^{i}) \dd[3]{x}
\marginnote{Using the kinematic equation.}  \\
&= - \int_V \alpha K \sqrt{\gamma } \dd[3]{x} + \underbrace{\oint_S \beta^{i} S_i}_{= 0}
= - \int \alpha K \sqrt{\gamma } \dd[3]{x}
\,,
\end{align}
%
therefore \(K = 0\) yields \(\delta V = 0\): this is a maximum.

It is the exact same geometric problem as a film of soap. 
If the soap is held at a ring, it stays flat. 

The only difference from that case is that here we are in a Lorentzian geometry: in the Euclidean case we have a minimum, here we have a maximum. 

This is an example of a singularity-avoiding gauge, used by many works. 

\subsection{The Conformal Transverse Traceless formalism}

We use the \(p = -10\) scaling of \(K_{ij}\), and on top of the Lichnerowicz equation we derive an equation from the momentum constraint \(C_i = 0\), for a vector \(x^{i}\) which is obtained with a further decomposition of \(\hat{A}^{ij}\): 
%
\begin{align}
\hat{A}^{ij} = \hat{A}^{ij}_L + \hat{A}^{ij}_{TT}
\,,
\end{align}
%
where the TT part is transverse and traceless: \(\widetilde{\gamma}_{ij} \hat{A}^{ij}_{TT} = 0\) and \(\widetilde{D}_j \hat{A}^{ij}_{TT} = 0\).

The longitudinal part on the other hand is expressed as follows: 
%
\begin{align}
\qty(\widetilde{L} x)^{ij} = \widetilde{D}^{i} x^{j} + \widetilde{D}^{j} x^{i} - \frac{2}{3} \widetilde{D}_k x^{k} \widetilde{\gamma}^{ij}
\,.
\end{align}

This seems like a rather strange expression: \(\widetilde{L}\) is called the \textbf{Conformal Killing operator} on \(x^{i}\). 
It is determined as follows: 
%
\begin{align}
\widetilde{D}_j  \hat{A}^{ij} = 
\widetilde{D}_j  \qty(\widetilde{L} x)^{ij} 
= \widetilde{D}_j  \widetilde{D}^{j} x^{i} 
+ \frac{1}{3} \widetilde{D}^{i} \widetilde{D}_j x^{i} + \widetilde{R}^{i}{}_j x^{i} = \widetilde{\triangle}_L x^{i} 
\,,
\end{align}
%
where \(\widetilde{\triangle}_L x^{i}\) is the \textbf{conformal vector Laplacian operator}. 

There exists a unique L + TT decomposition of \(\hat{A}^{ij}\) iff there exists a unique solution of the conformal Laplacian equation: 
%
\begin{align}
\widetilde{\triangle}_L x^{i} = \widetilde{D}_j \hat{A}^{ij}
\,.
\end{align}

This is useful because there is a theorem by Cantor in 1979 which tells us that if \(\Sigma \) is asymptotically flat and we have \(\partial_{k} \partial_{l} \widetilde{\gamma}_{ij} = \order{r^{-3}}\), 
then existence and uniqueness are guaranteed. 

With this decomposition, we obtain 
%
\begin{align}
\widetilde{\triangle}_L x^{i} - \frac{2}{3} \widetilde{D}^{i} K \psi^{6} - 8 \pi \widetilde{P}^{i} = 0
\,,
\end{align}
%
which can be solved together with the Lichnerowicz equation. 
This allows us to constrain \(\psi\) and \(x^{i}\), as long as we determine the free data: \(\widetilde{\gamma}_{ij}\), \(\hat{A}^{ij}_{TT}\) and \(K\). 

Under maximal slicing, \(K = 0\), the two equations decouple! 
They also partially decouple if we take \(K = \const\), since we can solve one and then the other.

Also, it is nice that the free data are the conformal metric: \(\widetilde{\gamma}_{ij}\) and \(\hat{A}_{TT}^{ij}\) are useful in heuristically determining the GW content of \(\Sigma \). 

Let us give an example of CTT data under the four hypotheses of conformal flatness, asymptotic flatness, and maximal slicing. 

Asymptotic flatness is chosen because we want the solution to exist.
This is the simplest choice of free data: we take \(\hat{\gamma}_{ij} = f_{ij}\) (conformal flatness) and \(K = 0\) (maximal slicing).

Further, we consider \(E = 0 = P^{i}\) (vacuum). 

Without justification we also assume that \(\hat{A}^{ij}_{TT} = 0\). 

Under all these assumptions we have 
%
\begin{align}
\widetilde{D}_i = D_i 
\,,
\end{align}
%
therefore 
%
\begin{align}
\widetilde{D}_i \widetilde{D}^{i} = D_i D^{i} = \triangle
\,
\end{align}
%
is simply the flat elliptic operator, also \(\widetilde{R} = 0\) and \(\widetilde{L} = L\) is the one calculated in the flat case. 

So, the CTT equations become: 
%
\begin{align}
\triangle \psi + \frac{1}{8} \qty(Lx )_{ij} \qty(Lx)^{ij} \psi^{-7} &= 0  \\
\triangle_L x^{i} = \triangle x^{i} + \frac{1}{3} D_j D^{i} x^{j} &= 0
\,,
\end{align}
%
which are decoupled, so they can be solved independently of one another. 
They are ``easy'', since they use the flat Euclidean operators. 
If we want a boundary value problem we also need to specify the boundaries: we specify asymptotic flatness: \(\psi = 1\) and \(x^{i} = 0\) at \(\iota_0\). 

Optionally we can impose strong-field inner boundary conditions: for example we can fix the topology. 

Our \textbf{case 1} is to take \(\Sigma_0 = \mathbb{R}^{3}\) (no inner boundary condition): then, the solution of the second CTT equation is \(x^{i}= 0\), and the first one reads \(\triangle \psi = 0\) together with \(\psi = 1\) at \(\iota_0 \): this implies \(\psi \equiv 1\).
As expected, we recover flat spacetime. 

Can we also get a nontrivial solution? 
For that, we need an inner boundary. 

\end{document}
