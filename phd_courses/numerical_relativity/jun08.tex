\documentclass[main.tex]{subfiles}
\begin{document}

\marginpar{Tuesday\\ 2021-6-8, \\ compiled \\ \today}

We are discussing the initial data problem, and we can adopt two formalisms: CTT and CTS. 

In the CTT formalism, we made a longitudinal-transverse split of  \(\hat{A}^{ij}\). 

We constructed asymptotically flat, conformally flat, time-symmetric initial data for the vacuum. 

The equations almost decouple: they can be solved one after the other. 
We then get the simplified system 
%
\begin{align}
\triangle_L x^{i} &= 0  \\
\triangle \psi + \frac{1}{8} (L x)_{ij} (Lx)^{ij} \psi^{-7} &= 0
\,,
\end{align}
%
with outer boundary conditions \(\psi \to 1\), \(x^{i} \to 0\) as \(r \to \infty \). 
This is a second-order elliptic problem, so we need a new set of boundary conditions: these inner boundary conditions determine the topology of the solution. If we suppose \(\Sigma_0 \sim \mathbb{R}^3\), we get flat spacetime, but its slices are not stationary. 

If we suppose \(\Sigma_0 \sim \mathbb{R}^3 \setminus B\) for some ball \(B\), we must impose some condition on \(S = \partial B\): the simplest choice is to demand \(\eval{x^{i}}_{S} = 0\).
Also, we can fix the conformal factor to 1. 
This yields to a flat \(\Sigma_0 \). 

An alternative is to ask that \(S\) be a closed \emph{minimal} surface:
we ask that \(D_i S^{i} = 0\), where \(S^{i}\) is the normal vector to \(S\). 

This is written as 
%
\begin{align}
0 = \eval{D_i S^{i}}_{S} = \psi^{-6} \widetilde{D}_{i} \qty(\psi^{6} S^{i}) 
= \psi^{-6} \widetilde{D}_{i} \qty(\psi^{4} \widetilde{S}^{i})
\,,
\end{align}
%
where \(\widetilde{s}^{i} = \psi^{-2} s^{i}\) is the unit  normal vector with respect to the conformal metric \(\widetilde{\gamma}\): 
%
\begin{align}
\widetilde{\gamma}(\widetilde{S}, \widetilde{S}) = \psi^{4} \widetilde{\gamma}(S, S) = \gamma (S, S) = 1 = \psi^{-6} \frac{1}{\sqrt{f}} \partial_{i} \qty(\sqrt{f} \psi^{4} \widetilde{S}^{i})
\,.
\end{align}

If we take \(B\) to be a sphere with radius \(r=a\), then \(\widetilde{S}^{i} = (1, 0, 0)\) everywhere, and \(f_{ij} = (1, r^2, r^2 \sin \theta )\). 
Therefore, 
%
\begin{align}
\frac{1}{r^2} \partial_{r} \eval{\qty(r^2 \psi^{4})}_{r=a} = 0
\iff \eval{\qty(\partial_{r} \psi_+ \frac{\psi}{2r})}_{r=a} = 0
\,,
\end{align}
%
which is a boundary condition of mixed Dirichlet-Neumann type for \(\triangle \psi + \dots = 0\).

A solution is \(\psi = 1+ a/r\), but what is the constant \(a\)?
The standard way to determine it is to try to calculate global quantities.

For example, let us try to calculate the ADM mass: 
%
\begin{align}
M _{\text{ADM}} &= - \frac{1}{2 \pi } \lim_{r \to \infty}
\int \pdv{\psi }{r} r^2 \sin \theta \dd[2]{\Omega}  \\
&= 2 a
\,.
\end{align}

Therefore, \(a = M _{\text{ADM}} / 2\). 

Putting things together, 
%
\begin{align}
\gamma_{ij} = \psi \hat{\gamma}_{ij} = \psi f_{ij} = \qty(1 + \frac{M _{\text{ADM}}}{2r}) f_{ij}
\,,
\end{align}
%
Schwarzschild in isotropic coordinates!

This also proves that the Einstein-Rosen bridge is a minimal surface. 

\todo[inline]{???}

We can do a transformation \(r \to r' = M^{4} / 4r\), which sends \([M/2, \infty )\) to \((0, M/2)\). 

The metric is invariant under this transformation: with a bit of abuse of notation, \(\gamma (r, \theta , \varphi ) = \gamma (r', \theta , \varphi )\). This means that the transformation is an isometry. 
This means that we can attach a copy of the upper part of the embedding diagram to its lower part.  

This means that we attach the I and III regions of the Kruskal diagram. 
Here, \(r = 0\) does not correspond to a singularity but instead to the other asymptotically flat end. 

What about the time symmetry condition? The solution we have here is a slice of Schwarzschild in isotropic coordinates. It is characterized by \(\hat{A}_{TT}^{ij} = 0\) and \(x^{i} =0 \). 
If we reconstruct the full \(\hat{A}^{ij}\) we find that it is also zero, and we also know that \(K = 0\) because of maximal slicing; this implies that \(K_{ij} = 0\).

This means that \(\mathscr{L}_m g_{ij} =0\) at \(t = 0\). 
Therefore, we say that \(\Sigma_0 \) is \emph{momentarily static}. 

If we map \(t \to -t\), however, the line element is invariant: this is what we mean by time symmetry. 

We do not have the full Schwarzschild in isotropic coordinates, we only have a slice.  

There could be a nontrivial evolution of \(\Sigma_0 \)!  
An example is when we use geodesic gauge. 

We could insert multiple Black Hole initial conditions by superposition: 
%
\begin{align}
\psi = 1 + \sum _{p=1}^{N} \underbrace{\frac{M_P}{\abs{x^{i} - c_P^{i}}}}_{\psi_{BL}}
\,,
\end{align}
%
where the BL stands for Brill-Lindquist. 
This is only for Schwarzschild BHs, Kerr is much more complicated and will be discussed later. 

More complicated initial data is Misner data. This can be constructed by suitable inner boundary conditions. 

\todo[inline]{He draws ER bridges --- why? do we need the other end of the bridge? }

This construction does not allow for spin, nor does it include boosts. 

Let us move to the work by Bowen and York in 1980. They proposed an ansatz for \(\triangle _L x^{i} + \dots = 0\): 
%
\begin{align}
x^{i} = - \frac{1}{4r} \qty(7 f^{ij} P_j + \frac{1}{r^2} P_j x^{i} x^{j}) - 
\frac{1}{r^3} \epsilon^{ij}_k S_h x^{k}
\,,
\end{align}
%
with six parameters \(P^{i}\) and \(S^{i}\). 

This yields 
%
\begin{align}
\hat{A}^{ij} = (Lx)^{ij} = \frac{3}{2} 
\underbrace{ \frac{1}{r^3} \qty[ x^{i} P^{j} + x^{j} P^{i} + (f^{ij} - \frac{x^{i} x^{j}}{r^2}) P^{k} x_k]}_{\order{1/r^2}} +
\underbrace{\frac{3}{r^{5}} 
\qty( \epsilon^{ik}_\ell S_k x^{\ell} x^{j} + \epsilon^{jk}_\ell S_K x^{\ell} x^{i})}_{\order{1/r^3}}
\,.
\end{align}

\(P^{i}\) and \(S^{i}\) are the ADM momentum and angular momentum components in the isotropic gauge. 

\todo[inline]{Insert sketch of proof for ADM momentum.}

With the BY ansatz one can find nonrotating BHs with \(S^{i} = 0 = P^{i}\); boosted BHs if \(P^{i} \neq 0 \) and \(S^{i} = 0\); spinning BHs wih \(S^{i} \neq 0\) and \(S^{i} = 0\).

Did we find the Kerr solution? No. 
In 2000 Gasat and Price proved that there is no Kerr foliation which
\begin{enumerate}
    \item has axial symmetry;
    \item is conformally flat;
    \item reduces to Schwarzschild in the nonrotating limit \(S^{z} = 0\). 
\end{enumerate}

This is often simply stated as ``there is no conformally flat slicing of Kerr''.

If there is a moment of time symmetry, then the BY data with \(S^{z} \neq 0\) are \emph{nonstationary}! 
The evolution must therefore be nontrivial --- something will happen. 
Typically, during the evolution GWs are produced.  

These are typically called \emph{puncture solutions}: with a BY ansatz and a solution of the Lichnerowicz equation (a numerical one, since the equation is nonlinear in \(\psi \)). 

A recent approach, which is used in many simulations, is called Generalized Bill-Lindquist data (due to Brand and Brügmann in 1997). 

We solve the Lichnerowicz equation on \(\mathbb{R}^3\) by analytically separating the singular behaviour: we take the ansatz \(\psi = \psi_{BL} + u\) where \(u\) is a correction. 

We know that \(\triangle \psi_{BL} = 0\): so, if we plug in the ansats in the nonlinear Lichnerowicz equation we get an equation in the form \(\triangle u + \dots = 0\): specifically, 
%
\begin{align}
\triangle u + \frac{\hat{A}^{ij} \hat{A}_{ij}}{8 \psi_{BL}^{7}} \qty(1 + \frac{u}{\psi_{BL}})^{-7}= 0
\,,
\end{align}
%
with outer boundary conditions \(u \sim 1 + \order{1/r}\). 
Crucially, no inner Boundary Conditions are needed! 

By evaluating the fields near the punctures, one sees that at \(x^{i} = c_P^{i}\) the correction \(u\) satisfies \(\triangle u = 0\): it is regular! 
Thus, it can be solved on \(\mathbb{R}^3\). 

Next, we will discuss CTS and XCTS, which are not well-posed but are useful when discussing stationary slices. 
These 

\end{document}
