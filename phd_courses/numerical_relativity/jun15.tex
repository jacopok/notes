\documentclass[main.tex]{subfiles}
\begin{document}

\marginpar{Tuesday\\ 2021-6-15, \\ compiled \\ \today}

We continue discussing the initial data problem. 

\subsection{Conformal Thin Sandwich}

We can give a different decomposition of \(\widetilde{A}_{ij}\): 
%
\begin{align}
\mathscr{L}_m \widetilde{\gamma}_{ij} &= \partial_{t} \widetilde{\gamma}_{ij} - \mathscr{L}_\beta \widetilde{\gamma}_{ij} = \partial_{t} \widetilde{\gamma}_{ij} + \qty( \widetilde{L} \beta )^{ij} + \frac{2}{3} \widetilde{D}_{k} \beta^{k} \widetilde{\gamma}_{ij}  \\
&= 2 \alpha \hat{A}_{ij} + \frac{2}{3} \widetilde{D}_{k} \beta^{k} \hat{\gamma}_{ij}
\,.
\end{align}

The idea is to combine these two equations in order to express \(\hat{A}_{ij}\) as follows: 
%
\begin{align}
\hat{A}_{ij} = (2\alpha )^{-1} \qty[ \widetilde{\dot{\gamma}}_{ij} + \qty(\widetilde{L} \beta )_{ij}]
\,,
\end{align}
%
and similarly for \(\overline{A}_{ij}\) if we wanted to use the conformal rescaling.

This equation replaces the \(L + TT\) decomposition of \(\hat{A}_{ij}\) in CTS. 
We can plug this into the momentum constraint \(C_i = 0\), and we find 
%
\begin{align}
\widetilde{D}_{j} \qty[ \widetilde{a}^{-1} \qty(\widetilde{L} \beta )^{ij}] + \widetilde{D}_{j} \qty[\widetilde{\alpha}^{-1} \widetilde{\dot{\gamma}}^{ij}] 
- \frac{4}{3} \psi^{6} \widetilde{D}^{i} K 
- 16 \pi \widetilde{P}^{i} = 0
\,.
\end{align}

This must be paired with the Lichnerowicz equation: 
%
\begin{align}
\widetilde{D}^{i} \widetilde{D}_{i} \psi + \dots = 0
\,.
\end{align}

The free data will be \(\widetilde{\gamma}_{ij}\), \(K\), \(\widetilde{\dot{\gamma}}_{ij}\), \(E\) and \(P^{i}\). 
The constrained data can be \(\psi \) and \(\beta^{i}\). 

The difference from before is that we have substituted this TT part with the time derivative of the conformal metric; in some sense specifying \(\widetilde{\dot{\gamma}}\) can help in specifying free data, for example stationary data.

The term \(\widetilde{\alpha} = \psi^{-6} \alpha \) is a conformally rescaled lapse.
The maximal slicing condition \(K=0\) means that the CTS equations decouple.
The momentum constraint for \(P^{i}\) turns out to be \emph{linear} in this case. 

York and Pfeiffer in 2003 proposed a formalism called XCTS, extended CTS, where the idea is to specify an equation for the conformal lapse \(\widetilde{\alpha}\) in order not to have to specify it.

Let us consider 
%
\begin{align}
\mathscr{L}_m K &= \dot{K} - \beta^{i} \widetilde{D}_{i} K   \\
&= - \psi^{-4} \qty(\widetilde{D}_{i} \widetilde{D}^{i} K + 2 \widetilde{D}_{i} \log \psi \widetilde{D}^{i} \alpha ) 
+ \alpha \qty[ \dots]
\,:
\end{align}
%
the piece multiplied by \(\psi^{-4}\) can be written as 
%
\begin{align}
\widetilde{D}_{i} \widetilde{D}^{i} K + 2 \widetilde{D}_{i} \log \psi \widetilde{D}^{i} \alpha =
\psi^{-1} \qty[\widetilde{D}_{i} \widetilde{D}^{i} (\alpha \psi ) + \alpha \widetilde{D}^{i} \widetilde{D}_{i} \psi ]
\,,
\end{align}
%
but using the Hamiltonian constraint \(C_0 = 0\) we can write this as 
%
\begin{align}
\widetilde{D}_{i} \widetilde{D}^{i} \qty(\hat{a} \psi^{7})
- \qty(\hat{a} \psi^{7}) \qty[ \frac{\hat{K}}{8} 
+ \frac{5}{12} K^2 \psi^{4} + \frac{7}{8} \hat{A}_{ij} \hat{A}^{ij} 
\psi^{-8} + 2 \pi \qty( E + 25 \psi^{8}) \psi^{-4}]
+ \qty(\dot{K} - \beta^{i} \widetilde{D}_{i} K ) \psi^{5} = 0
\,.
\end{align}

The XCTS scheme uses this equation, the Lichnerowicz equation and the one previously dubbed equation 1. 

Under \(K = 0\) these do not decouple; however there is a nice feature: free data includes \(\widetilde{\gamma}_{ij} \), \(\widetilde{\dot{\gamma}}\), \(K\) and \(\dot{K}\). Being able to specify these is very convenient. 

There exist examples for which XCTS yields non-unique solutions. 
Let us give a simple example: the solution of these for conformally, asymptotically flat \& zero-derivative vacuum initial data. 

The equation reads 
%
\begin{align}
\triangle \psi + \frac{1}{8} \hat{A}_{ij} \hat{A}^{ij} \psi^{-7} &= 0  \\
D_j \qty[\widetilde{\alpha }^{-1} (L \beta )^{ij} ] &= 0 iln
\triangle \qty(\widetilde{a} \psi^{7}) - \frac{7}{8} \hat{A}_{ij} \hat{A}^{ij} \widetilde{\alpha} \psi^{-7} &= 0
\,.
\end{align}

Asymptotically flat means \(\psi = 1\), \(\beta^{i} = 0\), \(\alpha = 1\) as \(r \to \infty \). 
Then, \(\beta^{i} = 0\) is a solution we can take, together with \(\widetilde{\dot{\gamma}} = 0\), which means \(\hat{A}^{ij} = 0\). 

The equations then read 
%
\begin{align}
\triangle \psi &= 0  \\
\triangle (\hat{\alpha} \psi^{7}) &= 0
\,. 
\end{align}

The inner boundary condition as before is a punctured space: \(\Sigma_0 = \mathbb{R}^3 \setminus 0\). 
This yields \(\psi = 1 + M / 2r\). 

The solution of the second equation is in the same form: \(\phi = \widetilde{\alpha} \psi^{7} = 1 + a / r\) for some constant \(a\). 
Now, \(\alpha = \psi^{6} \widetilde{\alpha} = \phi \psi^{-1}\); therefore the lapse reads 
%
\begin{align}
\alpha = \qty(1 + \frac{a}{r}) \qty(1 + \frac{M}{2r})^{-1} 
= \frac{r + a}{r + M/2}
\,.
\end{align}

What is the constant \(a\), though? Remember that we are solving this equation on the punctured \(\mathbb{R}^3\): \(a\) is fixed by the choice of the value of the lapse \(\alpha \) at a certain radius, say \(r \to 0\). 
The simplest choice is \(\alpha_0 = +1\): this yields \(a = M/2\), which also means that the lapse is identically equal to 1. 

Therefore, we have re-found Schwarzschild in isotropic coordinates and geodesic gauge, 
%
\begin{align}
g = - \dd{t^2} + \psi^{4} \qty(\dd{r^2} + r^2 \dd{\Omega^2})
\,.
\end{align}

What if we set \(\alpha_0 = -1\)? We know that \(\alpha > 0\), but let us explore this crazy condition anyways. 
This yields 
%
\begin{align}
a = - \frac{M}{2}
\,,
\end{align}
%
which implies 
%
\begin{align}
\alpha = \qty(1 - \frac{M}{r}) \qty(1 + \frac{M}{2r})
\,.
\end{align}

This is the lapse in isotropic coordinates. 

We have a moment of time symmetry, but the time development of \(\Sigma_0 \) are different! 
If we take \(\alpha_0 = +1\), \(\Sigma_0 \) evolves in a nontrivial way! 
If, instead, \(\alpha_0 = -1\) then \(\Sigma_0 \) does not evolve, and \(\partial_{t}\) is a Killing vector. 

What do we do with this negative lapse? In some points the \(\vec{n}\) vector will remain future-pointing, but we will be in a weird coordinate system such that the time starts to run backwards at small radii. 
We have to learn to live with it.

Where is XCTS useful? 
Imagine a binary system: here there is no \(\partial_{t}\) timelike Killing vector, and despite the rotation there is no rotational Killing vector \(\partial_{\varphi }\). 

Suppose we are considering a circular orbit with frequency \(\Omega \): 
we can define 
%
\begin{align}
n^a = \qty(\partial_{t})^{a} + \Omega \qty(\partial_{\varphi })^{a}
\,,
\end{align}
%
which is conserved! This is because the orbit looks like a helix in \(t, r, \varphi \) space: the direction of \(n\) is always along this helix. 

This is an idealized, approximate situation, in which there is no radiation.

In a comoving frame with \(\dot{\gamma}_{ij} = 0 = \dot{K}\) we can use a CTT + Bowman-Yosk puncture method.

The other part of the discussion for today is about gauge conditions. 

\subsection{Gauge condition}

We need to make a choice of the foliation and a choice of the spatial coordinates. 
We want our choice to 
\begin{enumerate}
    \item avoid singularities;
    \item enhance symmetries;
    \item minimize grid distortions.
\end{enumerate}

Something useful is to have our slicing be as close as possible to some Kerr coordinates. 
If some Killing vector is present we want to stay close to it.

Let us start with \textbf{slicing} (\(\alpha \)). 
We have already discussed \emph{geodesic slicing} in which \(\alpha = 1\) and \(\beta^{i} = 0\): the coordinates follow freely-falling observers. 

Here, \(n^a = (\partial_{t})^{a}\), and \(\alpha^{a} = D_a \log \alpha \). This also means that \(t = \tau \). 

This is certainly not singularity-avoiding: free-falling observers fall into the singularity. Also, it does not seek symmetries, and it distorts the grid a lot. 

In this gauge, the ADM equation reads 
%
\begin{align}
\partial_{t} K &= K_{ij} K^{ij} + 4 \pi (E - S)  \\
\partial_{t} \log \sqrt{ \gamma } &= - K
\,.
\end{align}

The behaviour of this gauge can be seen from these equations if we think about gravitational collapse: 
we can draw a \(r\) versus \(t\) diagram and see that matter eventually falls into the horizon. 

\(E-S\) gets large, \(\partial_{t} K\) gets large, and \(\partial_{t} \log \sqrt{\gamma }\) gets small. The curvature increases, and the coordinate volume element decreases. These are all characteristics of the \emph{bad gauges}. 

Another gauge we have seen is \textbf{maximal slicing}: \(0 = K\), and we already discussed its geometric meaning.

If we look at \(0 = K = - \nabla_a n^a\), the picture we can draw suggests the motion of an incompressible fluid: the coordinate volume element cannot be squeezed. 

The Maximal Slicing equation can be also written as 
%
\begin{align}
D_i D^{i} \alpha = - \alpha \qty[ 4 \pi (E-S) + K_{ij} K^{ij} ] = 0
\,,
\end{align}
%
to be compared with the previous geodesic slicing ones.
Here the acceleration of an Eulerian observer will be different from 0, and the lapse \(\alpha \) will evolve in a certain way which counteracts the effect of this volume element compression, as well as counteracting the ``singularity seeking'' property of geodesic slicing. 
 
\end{document}
