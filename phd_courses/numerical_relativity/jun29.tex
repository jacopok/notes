\documentclass[main.tex]{subfiles}
\begin{document}

\marginpar{Tuesday\\ 2021-6-29, \\ compiled \\ \today}

Anile and others wrote complicated books on relativistic hydrodynamics. 
With differential forms, in the Carter formalism the proofs become trivial differential geometry identities. 

Today is the last formal full lecture. 
Next, there will be about two weeks for which there will be more hands-on sessions. 

There will be two exercise slots, on Tuesday and Thursday. 

\section{Hyperbolic free-evolution schemes}

We have discussed earlier the fact that the constraints are propagated along the dynamics. 

We want a schema to provide the free evolution of \emph{constraint-satisfying} initial data \(\Sigma_0 \). 
We solve the ``time-dependent'' 3+1 EFE, and monitor the constraints as a check for numerical errors, since theoretically they should always hold.

Other approaches are possible, such as a fully constrained formulation, 
which is not really used --- it maximizes the number of hyperbolic equations. 

Another idea is to do constrained formulations exploiting symmetry. 

The key issue for free-evolution schemas is the well-posedness of the Initial Value Boundary Problem. 

We will discuss only the 3+1 Cauchy approach, for which \(\Sigma \) is always spacelike.
It is not the only options: we could do \emph{characteristic evolution} with null foliations, or hyperboloidal evolution with spacelike foliations reaching null infinity. 

In the strong-field region these lightcones yield singularities, which is why they are not often used; however there are waveform extracting techniques which are based on this kind of thing. 

Let us define hyperbolicity and hyperbolic equations. 
The prototype hyperbolic equations is 
%
\begin{align}
\partial_{tt} \phi - c^2 \triangle \phi = 0
\,,
\end{align}
%
where \(\phi \) is some sort of scalar field. An IBVP for this is imposes constraints in the form \(\phi (t= 0, x) = I(x)\) for \(x \in \Omega \) and \(\phi (t, x) = b(t, x)\) for \(x \in \partial \Omega \).

This can be made first-order in time by introducing an auxilliary field \(\Pi \): 
%
\begin{align}
\begin{cases}
    \partial_{t} \phi &= \Pi   \\
    \partial_{t} \Pi &= c^2 \triangle \phi 
\,.
\end{cases}
\end{align}

There also is a fully first order (in time and space) reduction: 
%
\begin{align}
\begin{cases}
    \partial_{t} \phi &= \Pi  \\
    \partial_{t} \chi_i &= c \partial_{i} \Pi 
    \partial_{t} \Pi &= c \partial_{i} \chi^{i}
    \chi_i &= c \partial_{t} \phi 
\,.
\end{cases}
\end{align}

This can be written in matrix form in terms  of the vector \(u = (\phi , \chi _i, \Pi )\): 
%
\begin{align}
\partial_{t} u + A^{i} (u) \partial_{i} u = s(u)
\,,
\end{align}
%
which is the general form of a hyperbolic equation. 
If \(A\) is a constant, then we can diagonalize the system: \(LAR = R^{-1} A R = \Lambda \), where \(\Lambda \) is diagonal and \(L\), \(R \) are the left and right eigenvalue matrices. 
We can also change variables accordingly: \(w = R u\), which are called \emph{characteristic variables}. 
After doing this, we are left with a system of equations of the form 
%
\begin{align}
\partial_{t} w + \lambda \partial_{x} w = 0
\,,
\end{align}
%
which is simply a translation of the initial profile, according to the eigenvalues of the matrix \(\Lambda \). 

Therefore, the spectrum of \(A^{i}\) plays an important role in determining the solution and well-posedness. 

Another way to see it is as follows: if we consider a Fourier mode 
%
\begin{align}
u = \widetilde{u}_k e^{ik (\vec{x} \cdot \hat{r} - vt)}
\,.
\end{align}

Here \(v = \omega / k\). 
We can plug this in: 
%
\begin{align}
- i k v \widetilde{u}_k + ik A^{j} u_j \widetilde{u}_k &= s  \\
 - v \widetilde{u}_k + A^{j} u_j \widetilde{u}_k &= \frac{s}{ik}
\,,
\end{align}
%
which in the high-frequency limit becomes 
%
\begin{align}
\qty(A^{j} u_j) \widetilde{u}_k &= v \widetilde{u}_k
\,,
\end{align}
%
an eigenvalue system for the \emph{principal symbol } \(P(\hat{n}) = A^{j} u_j\) matrix.
Iff there is a complete set of eigenvectors and eigenvalues, then we are lookiing at a \emph{strongly hyperbolic PDE}. 
If so, the IVP is \emph{well-posed}! 

What does well-posedness mean?
A characterization is as follows: there exists a norm \(\norm{\cdot}\) such that \(\norm{u(t, x)} \leq k e^{\alpha t} \norm{u(0, x)}\), where \(K\) and \(\alpha \) are constants independent of initial deta. 

What this means is: there are no Fourier modes that are growing in time. 

An example of a (non-hyperbolic) non-well-posed system is the inverse heat equation: \(\partial_{t} u = - \partial_{xx} u\). 
If we take \(u = \widetilde{u}_k e^{ikx + \sigma t}\), we get 
%
\begin{align}
\sigma e^{ikx + \sigma t} \widetilde{u}_k - (ik)^2 e^{ikx - \sigma t} \widetilde{u}_k
\,,
\end{align}
%
so \(\sigma = k^2\), and therefore the solution is in the form \(e^{k^2 t + ik x}\). 
The solution grows in a way which depends on the initial data: no universal constant can bound it. 

Another example is the regular heat equation: \(\partial_{t} u = \partial_{xx} u \). Here the Fourier modes satisfy \(\sigma = - k^2\), therefore the system is well-posed. 

Another example is the inverse wave equation: \(\partial_{tt} u = - \partial_{xx} u\). We can prove that this is also ill-posed. 

The definition of hyperbolicity can be stated in several ways: in term of the principal symbol \(P(n^i) = A^{i} n_i\) a PDE system is called 
\begin{enumerate}
    \item weakly hyperbolic iff \(P\) has real eigenvalues but not a complete set of them;
    \item strongly hyperbolic iff \(P\) has a complete set of real eigenvalues;
    \item symmetry hyperbolic iff the symmetrizer \(H\) is independent of \(n^i\), which is equivalent to saying that all the \(A^{i}\) matrices are symmetric. 
\end{enumerate}

The symmetrizer is defined as 
%
\begin{align}
H = \qty(R^{-1})^\dag R^{-1}
\,.
\end{align}

The three conditions imply each other in order, with weak hyperbolicity \(\implies\) strong hyperbolicity \(\implies\) symmetry hyperbolicity. 
Going back to the definition of well-posedness: strong hyperbolicity implies that the system admits a well-posed IVP. 

A symmetry hyperbolic system is also well-posed, and it is typically easier to work with. 

How about GR? Our goal is to get a free evolution scheme which is at least strongly hyperbolic.

A way to do so is using \textbf{Generalized Hyperbolic Gauge} (GHG). 
The EFE in vacuo are 
%
\begin{align}
0 = R_{ab} = \underbrace{- \frac{1}{2} g^{cd} \partial_{c} \partial_{d} g_{ab}}_{\text{principal part}} + \nabla_{(a} \Gamma_{b)} + g^{cd} g^{ef} 
\qty( \partial_{e} g_{ca} \partial_{f} g_{bd} - \Gamma \Gamma )
\,,
\end{align}
%
where the highest-order derivatives are found in the \emph{pprincipal part}, which is the one we must consider when assessing hyperbolicity. 
Here we also defined \(\Gamma_{a} g^{bc} \Gamma_{abc}\). 

The term \(\square_g g_{ab}\) is good (manifestly hyperbolic), the term \(\nabla_{(a} \Gamma_{b)}\) is bad. The idea behind harmonic gauge is to take \(\Gamma_a \equiv 0\). 
The EFE in harmonic gauge are therefore manifestly symmetry hyperbolic. 
If can be easily shown that these coordinates are allowed, since we just need to set \(0 = \square x^{\mu }\). 

This can be generalized: \(\Gamma _a\) can be set to a prescribed function \(H_a\), which can be determined \emph{a priori} or as something satisfying certain hyperbolic equations. 
If this is done carefully, it allows us to preserve the symmetric hyperbolicity. 
The reason to do so is in order to be more flexible with the gauge driver. 

If we define the function \(Z_a = \Gamma _a - H_a\), it is possible to prove that the GHG is equivalent to 
%
\begin{align}
0 = \overline{R}_{ab} = R_{ab} - \nabla_{(a } Z_{b)}
\,,
\end{align}
%
under the algebraic constraint \(0 = Z_a\). 
This is just the Z_4 system for a particular choice of \(Z_a\). 
There is a 2005 paper discussing this. 
GHG was used for 
\begin{enumerate}
    \item the BBH breakthrough by Pretorius (GHG + constraint damping terms);
    \item NR codes based on spectral methods (since the ``patches'' with different Fourier bases must be matched to each other).
\end{enumerate}

What about ADMY? Is it hyperbolic? 
This is way more complicated, but it can be proven under the conditions that 
\begin{enumerate}
    \item \(\beta^{i}\) is prescribed;
    \item \(\alpha \) belongs to the Bona-Masso family or the densitized lapse \(\overline{\alpha} = \alpha / \sqrt{\gamma }\) is prescribed;
    \item \(C_i \equiv 0\) is satisfied along the evolution.
\end{enumerate}

If we drop the third condition, then ADMY is weakly hyperbolic, and no well-posedness result can apply. 

How can we then use ADMY? We can change those equations in order to get a strongly hyperbolic system. 
Consider the ADMY second order equations: 
%
\begin{align}
\partial_{tt} \gamma_{ij} \simeq - 2 \alpha  R_{ij} + \dots \approx \underbrace{\triangle \gamma_{ij}}_{\text{good term}} + \underbrace{\gamma_{ik} \partial_{j} \partial_{l} \gamma^{kl} + \gamma_{jk} \partial_{i} \partial_{l} \gamma^{kl}}_{\text{bad term}}  + \dots
\,,
\end{align}
%
so, without setting the bad term to zero directly we can define the auxilliary variable \(f^{k} = \partial_{l} \gamma^{kl}\), the ``coordinate divergence of \(\gamma \)'' and find an evolution equation for \(f^{k}\) which guarantees hyperbolicity. 
What could be this equation? 
Observe that its time derivative reads 
%
\begin{align}
\partial_{t} f^{k} = \partial_{t} \partial_{l} \gamma^{kl} = \partial_{l} \partial_{t} \gamma^{kl} \sim \partial_{l} k^{kl}  \sim C^{k}
\,,
\end{align}
%
so this looks like the momentum constraint! 

Modern evolution schemas are based on this, but using \emph{conformal ADMY}. 
The conformal Ricci reads 
%
\begin{align}
\widetilde{R}_{ij} \simeq - \frac{1}{2} \widetilde{\gamma}^{kl} \qty(
D_k D_l \widetilde{\gamma}_{ij} + \widetilde{\gamma}_{ik} D_j D_k \widetilde{\gamma}^{kl} + \widetilde{\gamma}_{jk} D_i D_k \widetilde{\gamma}^{kl} 
)  + \dots
\,,
\end{align}
%
so like before we define \(\widetilde{\Gamma}^{i} = - D_j \widetilde{\gamma}^{ij}\), \emph{auxiliary conformal variables}. 
We use the momentum constraint in order to find \(\partial_{t} \widetilde{\Gamma}^{i} = \dots\), an additional equation to add to the ADMY in conformal variables to obtain the BSSN (BSSNOK).

The Z4c formulation is the same as BSSN with an additional equation voor the variable \(\theta = Z^{a} n_a\): \(\partial_{t} \theta \sim C^{0}\), which is convenient because it adds a constraint damping term to BSSN.    

\end{document}