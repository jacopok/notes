\documentclass[main.tex]{subfiles}
\begin{document}

\marginpar{Tuesday\\ 2021-5-4, \\ compiled \\ \today}

The part which is missing from the discussion of the 3+1 formalism are the \textbf{Gauss-Codazzi-Ricci} equations: projections and contractions of the Riemann tensor. 
They connect the four-dimensional Riemann tensor to the three-dimension Riemann and extrinsic curvature tensors. 
They are: 
%
\begin{align}
\gamma^{p}_{a}
\gamma^{q}_{b}
\gamma^{r}_{c}
\gamma^{s}_{d} {}^{(4)}R_{pqrs} 
&= R_{abcd} + K_{ac} K_{bd} - K_{ad} K_{bc}  \\
\gamma^{p}_{a}
\gamma^{q}_{b}
\gamma^{r}_{c}
n^s {}^{(4)}R_{pqrs} 
&= D_a K_{bc} - D_b K_{ac}  \\
\gamma_{ap} n^r &
\,,
\end{align}
%
\todo[inline]{And more, also in contracted versions --- see the notes.}

We will then use these in order to write the EFE in terms of \(\gamma \), \(K\) with spatial covariant derivatives and Lie derivatives along ``time''. 

The final answer will be that the EFE can be into constraints and evolution equations --- the latter contain the time derivative, and can be divided into kinematic and dynamic equations. 

The final equation we will find is commonly called the ADM formulation of the EFE, and we will use an expression of these in a form which is due to York --- we will denote these as ADMY. 

We start by looking at the Hamiltonian constraint: we contract the EFE with temporal vectors and find
%
\begin{align}
^4 G_{ab} n^a n^b &= ^4 R_{ab} n^a n^b - \frac{1}{2} {}^4 R \underbrace{g_{ab} n^a n^b}_{= n^a n^b (\gamma_{ab} - n_a n_b)}  \\
&= ^4 R_{ab} n^a n^b + \frac{1}{2} {}^4 R
\,,
\end{align}
%
which is convenient since what we find is just half of the Gauss equations: using those, we get 
%
\begin{align}
^4 G_{ab} n^a n^b = \frac{1}{2} \qty(R + K^2 - K_{ab} K^{ab})
\,,
\end{align}
%
where \(R\) is the three-dimensional Ricci trace, while \(K\) is the trace of \(K_{ab}\). 
The other side of the equation reads 
%
\begin{align}
T_{ab} n^a n^b &= E 
\,,
\end{align}
%
where \(E\) is the energy density measured by Eulerian observers. 
The constraint then reads 
%
\begin{align}
C_0 = R + K^2 - K_{ab} K^{ab} - 16 \pi E = 0
\,.
\end{align}

This is a scalar equation defined on \(\Sigma \) which only contains spatial quantities and derivatives. 

As for the \textbf{momentum constraint}, we take one spatial and one temporal component: 
%
\begin{align}
{}^4 G_{pq} n^q \gamma_{a}^{p} = {}^4 R_{pq} \gamma^{p}_{a} n^q - \frac{1}{2} {}^4 R g_{qp} \gamma^{p}_a n^q 
\,,
\end{align}
%
where the second term can be simplified with the help of
%
\begin{align}
g_{qp} \gamma^{p}_a n^q = \qty(\gamma_{qp} - n_q n_p) \gamma^{p}_a n^q =0 
\,,
\end{align}
%
since contractions of \(\gamma \) and \(n\) always yield zero. 
We are then left with the four-dimensional Ricci tensor, which is precisely the left-hand side of the contracted Codazzi equations: this means that we can write it as 
%
\begin{align}
{}^4 R_{pq} \gamma^{p}_{a} n^q
= D_a K - D_b K^{b}_a
\,,
\end{align}
%
while for the matter term we get 
%
\begin{align}
T_{pq} \gamma^{p}_a n^q &= P_a 
\,,
\end{align}
%
where \(P_a\) is the momentum density measured by Eulerian observers. 

This constraint can then be written as 
%
\boxalign{
\begin{align}
C_a = D_b K^b_a - D_a K - 8 \pi P_a = 0
\,,
\end{align}}
%
which is called the ADM momentum. 
This is a rank-1 tensor equation in three-space, and it only contains spatial derivatives. 

Let us then move to the dynamical evolution equation: we consider the trace-reversed EFE 
%
\begin{align}
{}^{4} R_{ab} = 8 \pi \qty(T_{ab} - \frac{1}{2} T g_{ab})
\,,
\end{align}
%
which is helpful because it will allow us to use the identities we have. 
Contracting twice in the spatial direction gives us 
%
\begin{align}
{}^{4} R_{pq} \gamma^{p}_a \gamma^{q}_b 
\,,
\end{align}
%
which is the left-hand side of the contracted Ricci equations combined with the contracted Gauss equations.
What is then left is only to deal with the right-hand side: 
%
\begin{align}
T_{pq} \gamma^{p}_{a} \gamma^{q}_b = - S_{ab}
\,,
\end{align}
%
where \(S_{ab}\) is the stress tensor measured by Eulerian observers, a purely spatial term, while the other term is the trace
%
\begin{align}
S = g^{ab} S_{ab}
\,,
\end{align}
%
and we can verify that 
%
\begin{align}
T_{ab} = S_{ab} + 2 n_{(a} P_{b)} + n_a n_b E
\,,
\end{align}
%
whose trace reads \(T = g^{ab} T_{ab} = S - E\).

\todo[inline]{Clarify the minus sign: wouldn't we have 
%
\begin{align}
\gamma^{a}_{p} \gamma^{b}_{q}
\qty( S_{ab} + 2 n_{(a} P_{b)} + n_a n_b E)
= \gamma^{a}_{p} \gamma^{b}_{q} S_{ab} 
\,,
\end{align}
%
with no minus sign?
}

At this point we have all the parts for the kinematic equation: 
%
\begin{align}
\mathscr{L}_m K_{ab} 
&= - D_a D_b \alpha + \alpha \qty{ 
    R_{ab} + K K_{ab} - 2 K_{ac} K^{c}_b + 4 \pi \qty[(S-E) \gamma_{ab} - 2 S_{ab}] }
\,,
\end{align}
%
which could be called \(ADM-K\). It contains \(\mathscr{L}_m\), it is a rank 2 symmetric tensor equation (6 components). 

In this ADMY formalism we then have the equations \(C_0 = 0\) (ADM-H), \(C_a = 0\) (ADM-M), \(\mathscr{L}_m \gamma_{ab} = - 2 \alpha K_{ab}\) (ADM-\(\gamma \)) (kinematic equation), and the equation \(\mathscr{L}_m K_{ab} = \dots\) (a dynamical equation). 
In total we have 6+6 dynamical equations, plus the constaints.

These equations involve \(\gamma \) and \(K\) as fundamental variables, various derivatives (\(\mathscr{L}_m\), \(D\), \(DD\)). 

Also, we have lapse \(\alpha \) and shift \(\beta^{a}\) (the latter is inside the expression for the Lie derivative) appearing in the equations, but there are no equations specifying them: this reflects the gauge freedom for the choice of foliation, and the three spatial coordinates on \(\Sigma \). 
These must be \textbf{prescribed}. 

One choice is to use \textbf{adapted coordinates}: if we take \(x^{\mu } = (t, x^{i})\) we find a natural basis for the tangent space, \(e_\mu = \partial_{\mu } = (\partial_{t}, \partial_{i})\). 

The vector \(m\) is not necessarily aligned with \(\partial_{t}\): if we start from a point \(x^{i} (t) = c^{i}\) and move a point along \(\partial_{t}\) from \(\Sigma _t\) to \(\Sigma _{t + \delta t}\) it will land on \(x^{i} (t + \delta t) = c^{i}\) again. 
\todo[inline]{insert figure}

In general, though, 
%
\begin{align}
\qty(\partial_{t})^{a} = m^a + \beta^{a}
\,,
\end{align}
%
where the vector \(\beta^{a}\) is spatial; if we move from \(x^{i}(t)= c^{i}\) along \(m^a\) we will land on \(x^{i}(t + \delta t) = c^{i} - \beta^{i} \delta t\). 

The norm of this vector will read
%
\begin{align}
(\partial_{t})^{2} = m^2 + \beta^2 = - \alpha^2 + \beta^2
\,,
\end{align}
%
since \(m^a = n^a \alpha \).
Therefore, \(\partial_{t}\) is not necessarily timelike: depending on our choice of \(\alpha \) and \(\beta \) we can make it timelike, null or spacelike. 

What does the possibility to have superluminal shift mean? 
We can make a foliation choice which avoids singularities --- for example, in order to evolve the Schwarzschild metric.
This is not a problem: we can change our \emph{labels} of points arbitrarily fast.

We can specify everything in adapted coordinates: 
%
\begin{align}
\qty(\partial_{t})^{a} &= (1, 0, 0, 0)  \\
\beta^{a } &= (0, \beta^{i})  \\
n^a &= \alpha^{-1} \qty( (\partial_{t})^{a} - \beta^{a}) = (1/\alpha , - \beta^{i} / \alpha )  \\
n_a &= (- \alpha , \vec{0})
\,.
\end{align}

The metric components, on the other hand, read 
%
\begin{align}
g_{00} &= g( \partial_{t}, \partial_{t}) = - \alpha^2 - \beta^2  \\
g_{0i} &= g(\partial_{t}, \partial_{i}) = \beta_{i}  \\
g_{ij} &= g (\partial_{i}, \partial_{j}) = \gamma_{ij}
\,.
\end{align}

Typically, the metric determinants are denoted as \(g = \det g_{\mu \nu }\) and \(\gamma = \det \gamma_{ij}\). 

The inverse component \(g^{00} = - \alpha^{-2} = \det g_{ij} / \det g_{\mu \nu } = \gamma / g\): this is a convenient way to show that \(\sqrt{-g} = \alpha \sqrt{ \gamma }\).

The full inverse metric reads 
%
\begin{align}
g^{\mu \nu } = \left[\begin{array}{cc}
- \alpha^{-2} & \alpha^{-2} \beta^{i} \\ 
\alpha^{-2} \beta^{j} & \gamma^{ij} - \beta^{i} \beta^{j} / \alpha^2
\end{array}\right]
\,.
\end{align}

Note that matrices are not inverted by block: \(g^{ij} \neq \gamma^{ij}\), even though \(g_{ij} = \gamma_{ij}\). 

The derivatives can also be expressed in coordinate form: 
%
\begin{align}
\mathscr{L}_m &= \mathscr{L}_{\partial_{t}} - \mathscr{L}_{\beta } = \partial_{t} - \mathscr{L}_\beta   \\
\mathscr{L}_\beta  \gamma_{ij} &= \beta^{k} D_k \gamma_{ij} + 2 D_{(i} \beta_{j)}
= 2 \partial_{(i} \beta_{j)} - 2 \Gamma_{ij}^{k} \beta_{k} 
\,,
\end{align}
%
while 
%
\begin{align}
\mathscr{L}_\beta K_{ij} = \beta^{k} \partial_{k} K_{ij} + K_{ik} \partial_{j} \beta^{k} + K_{jk}\partial_{i}\beta^{k}
\,.
\end{align}

Now we can finally write the ADMY as PDEs: we make a choice, and write them in \textbf{geodesic gauge}, which means \(\alpha = 1\) and \(\beta^{i} = 0\). 
This means that in this gauge \(\tau = t\), the proper time is the time measured by Eulerian observers, and \(\partial_{t} = m\). 

Recall: the acceleration of Eulerian observers is 
%
\begin{align}
a_{a} = n^b \nabla_{b} n_a = D_a \log \alpha 
\,,
\end{align}
%
which is zero in geodesic gauge, since we set \(\alpha \) to be constant. 
This means that the worldlines of Eulerian observers are geodesics (in geodesic gauge). 

The system is then 
%
\begin{align}
\partial_{t} \gamma_{ij} &= - 2 K_{ij}  \\
\partial_{t} K_{ij} &= R_{ij} + K K_{ij} - 2 K_{ik} K^{k}_{j} + \text{matter term}  \\
C_0 = R + K^2 - K_{ij} K^{ij} - 16 \pi E &= 0  \\
C_{i} = D_j K^{j}_{i} - D_i K - 8 \pi P_i &= 0
\,.
\end{align}

What type of PDEs are these?
The evolution equations are first-order in time and second-order in space, wave-like equations;
if we try to linearize them with \(\gamma_{ij} \approx f_{ij} + h_{ij}\) we get 
%
\begin{align}
\begin{cases}
    \partial_{t} h_{ij} &= - 2 K_{ij}  \\
    \partial_{t} K_{ij} &\sim R_{ij} \sim - \frac{1}{2} \partial_{k} \partial^{k} \gamma_{ij} + \dots
\end{cases}
\,,
\end{align}
%
with some other terms, which we can safely ignore. 
This is the symmetric 2-tensor version of the wave equation! 
The Dalambertian is  \(\square \phi = \qty(- \partial_{t} \partial_{t}  + \partial_{k} \partial^{k}) \phi \), which can be put into the form 
%
\begin{align}
\begin{cases}
    \partial_{t} \phi &= - \pi   \\
    \partial_{t} \pi &= \partial_{k} \partial^{k} \phi 
\,.
\end{cases}
\end{align}

Of course, because of the terms we ignored this is not formal,  but it captures the spirit of the equations. 

We can rewrite the ADMY in geodesic gauge as a ``wave equation'' for \(\gamma_{ij}\), by substituting back \(K_{ij} = - \partial_{t} \gamma_{ij}\).
In principal part (highest order derivatives), this yields 
%
\begin{align}
- \ddot{\gamma}_{ij} + \gamma^{kl} \qty(\partial_{k} \partial_{l} \gamma_{ij} + \partial_{i} \partial_{j} \gamma_{kl} - \partial_{i} \partial_{k} \gamma_{jl} - \partial_{j} \partial_{k} \gamma_{il}) \approx 0 
\,.
\end{align}

\todo[inline]{Is it relevant that this looks similar to the Riemann tensor in Riemann normal coordinates?}

This is not just an exercise: it is important since it is useful in order to classify the PDEs. 
The inverse spatial metric \(\gamma^{kl} = \gamma^{kl} (\gamma_{ij})\) is just some rational polynomial of \(\gamma_{ij}\); 
we then have a second order in both time and space, \textbf{quasilinear} equation system (this means that it is linear in the highest derivative).

If we had written the non-principal part as well, we'd see that it is quadratic in the first order spatial metric derivatives. 

\end{document}
