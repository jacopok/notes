\documentclass[main.tex]{subfiles}
\begin{document}

\marginpar{Tuesday\\ 2021-5-11, \\ compiled \\ \today}

We continue our discussion of 3+1 GR. 

The Euler equations read 
%
\begin{align}
- \ddot{\gamma}_{ij} + \gamma^{kl} (\gamma_{ij}) \qty[
    \partial_{k} \partial_{l} \gamma_{ij}
    + \partial_{i} \partial_{j} \gamma_{kl}
    - 2 \partial_{(i|} \partial_{k} \gamma_{|j) k}
] \simeq 0
\,
\end{align}
%
in principal part, which is a quasilinear hyperbolic equation. 

We start out by discussing what kind of equations the constraints are. 
They read 
%
\begin{align}
0 &= C_0 \approx \gamma^{ik} \gamma^{jl} \partial_{k} \partial_{l} \gamma_{ij} - \gamma^{ij} \gamma^{kl} \partial_{k }\partial_{l} \gamma_{ij}  \\
0 &= C_i = \gamma^{jk} \partial_{j} \dot{\gamma}_{ki} - \gamma^{kl} \partial_{i} \dot{\gamma}_{kl}
\,.
\end{align}

These only contain second spatial derivatives or \(\gamma_{ij}\) or \(\dot{\gamma}_{ij}\). 
The first one kind of looks like a Laplacian, so we might be tempted to say it is elliptic-like, but actually they are of \textbf{no standard type}, although they have been studied and classified this has not been done in this formulation. 

Can we solve them by prescribing some boundary data? This leads to the so-called \emph{initial data problem} in GR. 

We have \(6+6=12\) unknowns, and 4 equations: how can we make this initial data? We must \emph{prescribe} 8 quantities on \(\Sigma_0\), the first surface --- these are called \emph{free data} --- and then solve \(C_\mu = 0\) for the remaining 4 quantities. 

In order to prescribe the free data we will need the \textbf{conformal decomposition}, which will be discussed later. 
Assuming we have a solution, do we need to then solve \(C_\mu = 0\) for later times? \textbf{No}, since the constraints are \textbf{transported along the dynamics} as a consequence of the Bianchi identities \(\nabla^{b} G_{ab}= 0\). 

We will prove this using a simple approach from the \(Z4\) system, which is a bit more elegant than differentiating the constraints and working everything out. 


Consider the following extended theory, which we could introduce from an action but do not need to: 
%
\begin{align}
^4G_{ab} + 2 \nabla_{(a} Z_{b)} - g_{ab} \nabla_{c} Z^{c} = 8 \pi T_{ab} 
\,,
\end{align}
%
which reduces to GR when \(Z^a = 0\) or if \(Z^{a}\) is a Killing vector. 
\todo[inline]{Is \(\nabla_{c} Z^{c}= 0\) for a Killing vector?}

So, we can do a 3+1 split of this \(Z4\) theory: this yields 
%
\begin{align}
\mathscr{L}_m \gamma_{ij} &= - 2 \alpha K_{ij}  \\
\mathscr{L}_m K_{ij} &= (\text{ADM eqs}) + \alpha D_{(i} Z_{j)}  \\
\mathscr{L}_m \theta &= \frac{1}{2} C_0 -- \theta K + D_k Z^k - Z^k D_k \log \alpha   \\
\mathscr{L}_m Z_i &= C_i + D_i \theta - \theta D_i \log \alpha - Z K^{k}_i Z_k 
\,,
\end{align}
%
where \(- \theta = n^a Z_a\). 

Observe that the constraints became evolution equations! 
So, this is just a system of hyperbolic equations with no constraints, and GR is obtained from this extended theory under the \emph{algebraic} constraint \(Z^a = 0\). 

Now the Bianchi identities read 
%
\begin{align}
\nabla^{a} {}^4 G_{ab} = 0 \implies \square 
\,,
\end{align}
%
a wave equation for \(Z^{a}\). 
If \(Z^{a} = 0 = \dot{Z}^{a} = C^{a}\), then \(Z^{a} = 0\) for all times. 
This approach is rather general, and it can be applied for other theories or sets of equations which have a mixed character, with some hyperbolic and some elliptic equations. 

An example is MHD: hydrodynamics with a perfectly conducting fluid. 

In this case we have \(0= \partial_{\mu } T^{\mu \alpha } = \partial_{\mu } \qty(T^{\mu \alpha }_{\text{PF}} + T^{\mu \alpha }_{\text{EM}})\). 

These can be written in terms of the magnetic field \(B^{i}\) as 
%
\begin{align}
\partial_{t} B^{i} + \partial_{k} \qty(B^{k} v^{i} - B^{i} v^{k}) &= 0  \\
C &= \partial_{i} B^{i} = 0
\,.
\end{align}

The question is the same: if we set \(C=0\) initially and then evolve the induction equation, is it guaranteed that the constraint will be preserved? 
In this case we only need a scalar field \(\psi \): 
%
\begin{align}
\partial_{i} B^{i} + \partial_{k} \qty(B^{k}v^{i} - B^{i} v^{k}) + \partial^{i} \psi &= 0  \\
\partial_{t} \psi + \partial_{i} B^{i} = 0 
\,,
\end{align}
%
which like before reduces to MHD if \(\psi = 0\). 

Note that if we take a derivative of the second equation we find \(\partial_{tt} \psi + \partial_{t} \partial_{i} B^{i} = 0\), into which we can substitute the first to get 
%
\begin{align}
\partial_{tt} \psi - \partial_{i} \partial^{i} \psi + \partial_{i} \partial_{k} B^{[k} v^{i]} = - \square \psi = 0
\,.
\end{align}
%

It can be shown that the constraint \(C\) satisfies the same equation. 

\todo[inline]{Show!}

This is a method which can be used to control the divergence of \(B\). 

If we are solving MHD, variations from \(C=0\) can appear due to numerical errors: if we are solving the extended system, we can see that the violation evolves according to the wave equation, so it will propagate away to the domain boundary because of \(\square C = 0\). 
Can we add a \textbf{damping term} to this propagation, so that the evolution reads \(\square C + \delta \dot{C} = 0\) for some small \(\delta > 0\)?

This corresponds to adding a term in the constraint equation with \(\psi \): 
%
\begin{align}
\partial_{t} + \partial_{i} B^{i} + \delta \psi = 0   
\,.
\end{align}

The name of this strategy is the \textbf{divergence cleaning method}, it was proposed in 2012. 

A similar thing can be done in the \(Z4\) term.
This was historically one of the things that allowed for the first successful BBH simulations. 

The Powell method corresponds to the alternative approach of projecting the equation onto the constraint at each step. 
There are general formulations for this. 

We use this specific method because it gives very \emph{uniform} equations. 
PDE theory deals a lot with full hyperbolic systems, and not to mixed systems.
So, it is very convenient to have everything be hyperbolic. 

This, however, is by no means the only way to do things. 
We could also try to have as many elliptic equations as possible.

% The fully constrained formalism is what this 
Nowadays these \textbf{free evolution schemes} are most often used, as opposed to the \textbf{fully constrained schemes} in which elliptic equations are maximized. 

There still must be 2 hyperbolic equations in the fully constrained scheme, since they represent the two physical degrees of freedom of the GW. 

\todo[inline]{What about \(\delta \) choice? It seems like we would prefer to have it as large as possible, are there problems with that?}

\section{ADM Hamiltonian formulation of GR}

The GR action is 
%
\begin{align}
S_{GR} = \int {}^4R \sqrt{-g} + \text{boundary terms}
\,,
\end{align}
%
the boundary terms are important, but they will not be discussed today. 
This can be written in our 3+1 formalism as 
%
\begin{align}
\int \dd{t} \int_{\Sigma _t} \underbrace{\qty(R + K_{ij} K^{ij} - K^2)}_{\text{Lagrangian density}} \alpha \sqrt{\gamma }
\,,
\end{align}
%
so we can introduce conjugate momenta: 
%
\begin{align}
\pi^{ij} = \pdv{L}{\dot{\gamma}_{ij}} = \sqrt{\gamma } \qty(K \gamma^{ij} - K^{ij})
\,,
\end{align}
%
so the Hamiltonian density reads 
%
\begin{align}
\mathcal{H} =\pi^{ij} \dot{\gamma}_{ij} - L =  \\
&= \sqrt{\gamma } \qty[ \alpha C_0 + 2 \beta^{i} C_i + 2 D_j \qty(K \beta^{i} - K^{j}_{i} \beta^{i})]
\,,
\end{align}
%
understood as a function of \(\gamma_{ij}\) and \(\pi^{ij}\). The Hamiltonian is then 
%
\begin{align}
H = \int_{\Sigma_t} \mathcal{H} = \int_{\Sigma _t} \sqrt{\gamma }
\qty(\alpha C_0  + 2 \beta^{i} C_i) 
\,.
\end{align}

The corresponding EoM read 
%
\begin{align}
\dot{\gamma}_{ij} = \pdv{\mathcal{H}}{\pi^{ij}} = - 2 \frac{\alpha}{\sqrt{\gamma }} \underbrace{\qty(\pi^{ij} - \frac{1}{2} \pi \gamma_{ij})}_{\sim K_{ij}} + 2 D_{(i } \beta_{j)}
\,,
\end{align}
%
while 
%
\begin{align}
- \dot{\pi}_{ij} = \pdv{\mathcal{H}}{\gamma_{ij}} = \dots 
\sim \mathscr{L}_m K_{ij} \text{ from ADM } + C_0  
\,,
\end{align}
%
and we can see that lapse and shift appear in these equations: if we differentiate with respect to them we find 
%
\begin{align}
0 = \pdv{\mathcal{H}}{\alpha } = C_0  
\qquad \text{and} \qquad
0 = \pdv{\mathcal{H}}{\beta^{i}} = C_i
\,.
\end{align}

In the original ADM formulation they used \(\gamma_{ij}\) and \(\pi^{ij}\); the York formulation used \(\gamma_{ij}\) and 
%
\begin{align}
K_{ij} = - \frac{1}{\sqrt{\gamma }} \qty(\pi_{ij} - \frac{1}{2} \pi \gamma_{ij})
\,.
\end{align}

Lapse and shift can then simply be understood to be Lagrange multipliers which enforce the constraint. 

Next time we will approach the conformal decomposition of 3+1 GR and the initial data problem: how do we solve it? 
There are some free evolution schemes like BSSNOK or Z4C.

Also, once we get the initial data we will need to solve the Cauchy IVP, discussing hyperbolicity and well-posedness. 

\end{document}
