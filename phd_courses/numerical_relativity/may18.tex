\documentclass[main.tex]{subfiles}
\begin{document}

\section{Conformal decomposition of 3+1 GR}

\marginpar{Tuesday\\ 2021-5-18, \\ compiled \\ \today}

Lichnerowicz in 1944 already used this to deal with the initial data problem. 

York in 1971 showed that the dynamical degrees of freedom of GR (GWs) are encoded in the conformal equivalence class of the three-metric: 
%
\begin{align}
\gamma_{ij} = \psi^{4} \widetilde{\gamma}_{ij}
\,,
\end{align}
%
which can be shown using ADM + the Cotton-York tensor. 
As an example, if \(\widetilde{\gamma}_{ij}\) is the flat 3-metric, we can directly say that there are no GW.

A common choice for conformal metrics is to take the uni-determinant ones: \(\det \gamma_{ij} = \gamma = 1\). 
Now, we know that \(\det (cA) = c^{m} \det A\), therefore 
%
\begin{align}
\det \gamma_{ij} = \gamma = \psi^{12} \widetilde{\gamma}
\,.
\end{align}

Since \(\widetilde{\gamma} = 1\), this means that \(\psi = \gamma^{1/12}\).

With this choice, \(\psi \) is not a scalar! 
Therefore, \(\widetilde{\gamma}_{ij}\) is also not a tensor but instead a tensor density, and it has no Levi-Civita connection associated to it. 

The solution to this issue is to introduce an unphysical background three-metric on \(\Sigma_{t}\), so that 
\begin{enumerate}
    \item \(f_{ij}\) has signature \(+\), \(+\), \(+\) and \(\det f_{ij} = f\);
    \item it is time independent: 
    %
    \begin{align}
    \mathscr{L}_{\partial_{t}} f_{ij} = \pdv{f_{ij}}{t} = 0
    \,,
    \end{align}
    %
    \item it has an inverse: \(f^{ij}f_{jk} = \delta^{i}_{k}\);
    \item it has a Levi-Civita connection \(\mathcal{D}_k f_{ij} = 0\), whose raised-index version is \(\mathcal{D}^{i} = f^{ij} \mathcal{D}_{j}\);
    \item its Christoffel symbols are 
    %
    \begin{align}
    F_{ij}{}^{k} = \frac{1}{2} f^{kl} \qty(\partial_{i} f_{lk} +\partial_{j} f_{il} - \partial_{l} f_{ij})
    \,,
    \end{align}
    %
\end{enumerate}

Now \(\psi = (\gamma / f)^{1/12}\) is a scalar. The conformal metric will be 
%
\begin{align}
\widetilde{\gamma}_{ij} = \psi^{-4} \gamma_{ij} 
\qquad \text{and} \qquad
\widetilde{\gamma}^{ij} = \psi^{4} \gamma^{ij} 
\,,
\end{align}
%
and its determinant will be \(\widetilde{\gamma} = f\). 

This conformal metric is a ``good'' metric. 
What do we use as this metric?
We can take \(f_{ij} = \mathbb{1}_3\), which has vanishing Christoffel symbols; if we are in Cartesian coordinates this is the natural choice.

If, instead, we are in spherical coordinates we might want to use \(f_{ij} = \diag{1, r^2, r^2 \sin^2\theta }\), which has \(f = r^{4} \sin^2 \theta \). 

The idea is to take the simplest flat metric in the coordinates we are using. 

In the weak-field limit, we have \(\psi^{4} = (1-2 \phi )\), so \(\psi \approx 1 - \phi /2\). 
The conformal factor is heuristically analogous to the Newtonian potential. 

As an example, the Schwarzschild metric in isotropic coordinates can be written as 
%
\begin{align}
\dd{s^2} = - \frac{(1 - M / 2r)^2}{(1 + M / 2r)^2} \dd{t^2}
+ \underbrace{\qty(1 + \frac{M}{2r})^{4}}_{\psi^{4}} \underbrace{\qty(\dd{r^2} + r^2 \dd{\Omega^2})}_{f_{ij}}
\,.
\end{align}

Now we have to define the \textbf{conformal connection and the Ricci tensor}, and specifically the relationship between \(\mathcal{D}_i\) and \(\widetilde{\mathcal{D}}_i\), which both live on \(\Sigma _t\). 

The expression reads: 
%
\begin{align}
\mathcal{D}_j T_{\dots}^{\dots} \sim \widetilde{\mathcal{D}}_j T_{\dots}^{\dots} + \Sigma C T_{\dots}^{\dots} - \Sigma C T_{\dots}^{\dots}
\,,
\end{align}
%
where \(C^{k}_{ij} = \Gamma^{k}_{ij} - \widetilde{\Gamma}^{k}_{ij}\). 
Using this, the Riccis are related by 
%
\begin{align}
R_{ij} &= \widetilde{R}_{ij} + \mathcal{\widetilde{D}}C - \widetilde{\mathcal{D}} C + CC - CC  \\
&= \widetilde{R}_{ij} + R^{\psi }_{ij}
\,,
\end{align}
%
where \(R^{\psi }_{ij}\) is a function which contains second covariant derivatives of the conformal factor, something like \(\widetilde{D} \widetilde{D} \psi + \widetilde{D} \psi \widetilde{D} \psi \). 

The Ricci scalar, on the other hand, reads 
%
\begin{align}
R = \gamma^{ij} R_{ij} = \dots = \psi^{-4} \widetilde{R} - 8 \psi^{-5} \widetilde{D}_i \widetilde{D}^{i} \psi 
\,.
\end{align}

\subsection{CD of extrinsic curvature}

We start by performing a trace-traceless decomposition: 
%
\begin{align}
K_{ij} = A_{ij} + \frac{1}{3} K \gamma_{ij}
\,.
\end{align}

Then, we decompose the traceless part \(A^{ij}\): \(A^{ij} = \psi^{p} \overline{A}^{ij}\), and \(\overline{A}\) is called the conformal 
traceless part of \(K^{ij}\). 

There are two possible choices for \(p\): sometimes we take \(p = -4\), which is based on \(\mathscr{L}_m \gamma_{ij}\), which is useful for evolution schemes, and the hyperbolic part of the ADMY equations. 

The alternative is \(p = -10\), based on the constraint equations \(C^{i}= 0\). 

For the first choice, we take the kinematic equation for \(K_{ij}\), and compute 
%
\begin{align}
\mathscr{L}_m \qty(\psi^{4} \widetilde{\gamma}_{ij}) &= - 2 \alpha A_{ij} - \frac{2}{3} \alpha K \psi^{4} \widetilde{\gamma}_{ij}  \\
\psi^{4} \mathscr{L}_m \widetilde{\gamma}_{ij} &= - 4 \psi^{-5} \psi^3 \mathscr{L}_m \widetilde{\gamma}_{ij} - 2 \alpha \psi^{-4} A_{ij} - \frac{2}{3} \alpha K \psi^{-4} \psi^{4} \widetilde{\gamma}_{ij} 
\,.
\end{align}

We trace this equation with \(\widetilde{\gamma}^{ij}\), and get 
%
\begin{align}
\widetilde{\gamma}^{ij} \mathscr{L}_m \widetilde{\gamma}_{ij} &= - 4 \underbrace{\widetilde{\gamma}^{ij} \widetilde{\gamma}_{ij}}_{=3} \mathscr{L}_m \log \psi 
- 2 \alpha \psi^{-4} \widetilde{\gamma}^{ij} A_{ij} - \frac{2}{3} \alpha K \widetilde{\gamma}^{ij} \widetilde{\gamma}_{ij}  \\
&= - 12 \mathscr{L}_m \log \psi - 2 \alpha K
\,.
\end{align}

Alternatively, we can write 
%
\begin{align}
\widetilde{\gamma}^{ij} \mathscr{L}_m \widetilde{\gamma}_{ij} &\sim \Tr \qty(M^{-1} \delta M) = \delta (\log \det M)  \\
&= \mathscr{L}_m \log \widetilde{\gamma} = \qty(\partial_{t} \mathscr{L}_\beta ) \log f  = - \mathscr{L}_\beta \log f   \\
&= - \widetilde{\gamma}^{ij} \mathscr{L}_\beta \widetilde{\gamma}_{ij} 
= - \widetilde{\gamma}^{ij} \qty(\beta^{k} \widetilde{D}_{k} \widetilde{\gamma}_{ij} + 2 \widetilde{\gamma}_{k (i} \widetilde{D}_{j)} \beta^{k})  \\
&= -2 \widetilde{D}_{i} \beta^{i}
\,.
\end{align}

Both ways, we see that 
%
\begin{align}
6 \mathscr{L}_m \log \psi + \alpha K &= \widetilde{D}_i \beta^{i}
\,,
\end{align}
%
so we have conformally decomposed the kinematic equation \(\mathscr{L}_m \gamma = - 2 \alpha K\).
We get 
%
\begin{align}
\mathscr{L}_m \log \psi &= \frac{1}{6} \widetilde{D}_{i} \beta^{i}  - \frac{1}{6} \alpha K  \\
\mathscr{L}_m \widetilde{\gamma}_{ij} &= - 2 \alpha \psi^{-4} A_{ij} - \frac{2}{3} \widetilde{D}_{k} \beta^{k} \widetilde{\gamma}_{ij} 
\,.
\end{align}

A natural choice which simplifies this is \(p=-4\), since it simplifies \(\psi^{-4} A_{ij} \to \widetilde{A}_{ij}\). 
This is commonly denoted as \(\overline{A}\).

The second choice yields equations in the foorm 
%
\begin{align}
D_j A^{ij} &= \widetilde{D}_{j} A^{ij} + C^{i}_{jk} A^{kj} + C^{i}_{jk} A^{ik} = \dots  \\
&= \psi^{-10} \widetilde{D}_{j} \qty(\psi^{10} A^{ij})
\,.
\end{align}

This is commonly denoted as \(\hat{A}\).

With this choice we find 
%
\begin{align}
0 = C^{i} = \widetilde{D}_{j} \hat{A}^{ij}
- \frac{2}{3} \psi^{6} \widetilde{D}^{i} K 
- 8 \pi \psi^{10} p^{i}
\,.
\end{align}

The Hamiltonian constraint, on the other hand, reads 
%
\begin{align}
0 = C^{0} = R + K^2 - K_{ij} K^{ij} - 16 \pi E
\,.
\end{align}

This \(K\) is already a conformal variable! 
As for the other term, we have 
%
\begin{align}
K_{ij} K^{ij} &= \qty(A_{ij} + \frac{1}{2} K \gamma_{ij})
\qty(A^{ij} + \frac{1}{2} K \gamma^{ij})  \\
&= A_{ij} A^{ij} + \underbrace{\frac{2}{3} K \gamma_{ij} A^{ij}}_{= 0} + \frac{1}{3} K^2 
\,.
\end{align}

Putting this together yields, for \(p = -4\), 
%
\begin{align}
0 = C^{0} = \widetilde{D}_{i} \widetilde{D}^{i} \psi 
- \frac{1}{6} \widetilde{R} \psi + 
\qty( \frac{1}{8} \widetilde{A}_{ij} \widetilde{A}^{ij} - \frac{1}{12} K^2 + 2 \pi E) \psi^{5}
\,,
\end{align}
%
and for \(p = -10\): 
%
\begin{align}
0 = C^{0} = \widetilde{D}_{i} \widetilde{D}^{i} \psi - \frac{1}{6}
\widetilde{R} \psi + \frac{1}{8} \hat{A}_{ij} \hat{A}^{ij} \psi^{-7} 
+ \qty( 2 \pi E - \frac{1}{12} K^2 )\psi^{5} 
\,.
\end{align}

The latter is called the \textbf{Lichnerovicz equation}: the Hamiltonian constraint in conformal variables. 
This is a complicated nonlinear equation, but under certain prescriptions we can make the \(\widetilde{D}_{i} \widetilde{D}^{i}\) term look like a Laplacian. 

The other CD ADMY equations are the dynamical ones, we have one for the evolution of \(K\), and one for the evolution of \(\widetilde{A}_{ij}\). 
See the notes for more details on these equations. 

\end{document}
