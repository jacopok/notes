\documentclass[main.tex]{subfiles}
\begin{document}

\section{Asymptotic Flatness \& Global Quantities}

\marginpar{Tuesday\\ 2021-5-25, \\ compiled \\ \today}

We have discussed a sufficient set of formalism, and we are ready to move to more physical topics. 

How do we characterize the spacetime? We'd like to define mass, momentum and such. 

A notion we can give for these works for Asymptotically Flat spacetimes: ADM energy and momentum. 

If the spacetime is not asymptotically flat but we have some symmetries described by Killing vectors, we can define Komar charges. 

There are more details on these topics in the book by Wald. 

\subsection{Asymptotic flatness}

\begin{definition}
A globally hyperbolic spacetime is \textbf{asymptotically flat} iff: 
%
\begin{align}
\forall \Sigma_t \colon \exists f_{ij} \quad \text{``background metric''}
\,,
\end{align}
%
such that 
\begin{enumerate}
    \item \(f_{ij}\) is flat except at most for a compact domain (the strong-field region, typically); 
    \item there exists a chart \(\qty{x^{i}}\) such that \(f_{ij}\) approaches \(\mathbb{1}_3\) for \(r = \sqrt{x^{i} x_i} \to \infty \);
    \item for \(r \to \infty \) the following conditions hold:
    \begin{enumerate}
        \item \(\gamma_{ij} = f_{ij} + \order{r^{-1}}\);\label{item:first-condition-af}
        \item \(\partial_{k} \gamma_{ij} = \order{r^{-2}}\);\label{item:second-condition-af}
        \item \(K_{ij} = \order{r^{-2}}\);
        \item \(\partial_{k} K_{ij} = \order{r^{-3}}\).
    \end{enumerate}
\end{enumerate}
\end{definition}

The region \(r \to \infty \) is called \textbf{spatial infinity}, \(i_0\).

A first example of a spacetime which is asymptotically flat is Schwarzschild; a counterexample on the other hand is a flat spacetime which contains a gravitational wave. 

Let us show this: in this case the spatial metric (in the TT gauge) is 
%
\begin{align}
\gamma_{ij} = f_{ij} + \frac{1}{r} h_{ij}(t-r) + \order{r^{-2}}
\,.
\end{align}

This is compatible with condition \ref{item:first-condition-af}, but the derivative is 
%
\begin{align}
\partial_{k} \gamma_{ij} = - \frac{h'(u)}{r} \frac{x^{k}}{r} - \frac{h_{ij}(u)}{r^2} \frac{x^{k}}{r} + \order{r^{-3}}
\,,
\end{align}
%
so we have an \(\order{r^{-1}}\) term: condition \ref{item:second-condition-af} is broken.

Also, cosmological spacetimes are typically not AF. 

A note: asymptotic flatness depends on the foliation \(\Sigma _t\) (and on \(x^{i}\), which however we can change); the transformation group which preserves these properties is the spin group: rotations, translations and boosts, generically
%
\begin{align}
\Lambda^{i}_{j} + \dots
\,,
\end{align}
%
\todo[inline]{to complete}

If out spacetime is AF we can define ADM energy and mass: the full GR action reads
%
\begin{align}
S_{GR} = \int_{\mathcal{V} \subset \mathcal{M}} {}^4R \sqrt{-g} \dd[4]{x} + \oint_{\partial \mathcal{V}} (Y - Y_0 ) \sqrt{h} \dd[3]{y} 
\,,
\end{align}
%
where the first term is the usual Hilbert action, while the second term is a boundary one, which is needed in order to obtain the correct EFE on the boundary of the region which is considered. 

This would not be needed if we knew that the derivatives of the metric were zero at the boundaries, but this will not be true in general. 

An alternative approach, by Levi-Civita, is to vary with respect to the connection which is not assumed to be metric-compatible.

Here, \(\mathcal{V}\) is our 3D region, \(\partial \mathcal{V}\) is its (timelike) boundary and \(h\) is the induced metric on \(\partial \mathcal{V}\) if we see the boundary as embedded in the larger manifold \((\mathcal{M}, g)\). 

The tensor \(Y_{ab}\) is the extrinsic curvature of \(\partial \mathcal{V}\) in \((\mathcal{M}, g)\); the tensor \(Y_{0, ab}\) is the extrinsic curvature of \(\partial \mathcal{V}\) as embedded in \((\mathcal{M}, \eta )\), and similarly \(h_{0, ab}\) (\(\eta \) is the flat 4D metric). 
 
\(Y\) and \(Y_0 \) are the respective traces.

We will not complete the calculation, but imposing 
%
\begin{align}
\fdv{S}{g^{ab}} = 0 
\,,
\end{align}
%
with \(\eval{\partial g^{ab}}_{\text{boundary}} = 0\) leads to the EFE. 

This way, we can construct the Hamiltonian: 
%
\begin{align}
H = - \int_{\Sigma _t} \sqrt{\gamma } \qty(\alpha C_0 + 2 \beta^{i} C_i) - 2 \oint_{\partial \Sigma _t} \sqrt{q} \qty[\alpha (\kappa - \kappa_0) + \beta^{i} S^{j} (K_{ij} - K \gamma_{ij})]
\,,
\end{align}
%
where we have the lapse \(\alpha \), the Hamiltonian constraint \(C_0 \), the shift \(\beta^{i}\), the momentum constraint \(C_{i}\), while in the other term we find the 2D intersection \(\partial \Sigma _t = \partial \mathcal{V} \cap \Sigma _t\). 

The vector \(S^{i}\) is the normal vector to \(\partial \Sigma _t\); the metric \(q_{ij}\) is the induced metric on \(\partial \Sigma _t\) as embedded in \((\Sigma, \gamma )\). 
On the other hand, \(\kappa_{ij}\) is the extrinsic curvature. 

The quantity \(\kappa_0 \), on the other hand, is the extrinsic curvature of \(\partial \Sigma _t\) as embedded in \((\Sigma, f)\). 

Here we have a key observation: the first term in the Hamiltonian is identically zero on a solution (because the constraints are zero), therefore the boundary terms give us an ``energy''. 

With this, the \textbf{ADM mass} is calculated on \(i_0 \) by taking \(\alpha = 0 = \beta^{i}\): 
%
\begin{align}
M _{\text{ADM}} &= - \frac{1}{8 \pi } \lim_{r \to + \infty } \oint_{\partial \Sigma _t} \sqrt{q} \qty(\kappa - \kappa_0 )  \\
&= + \frac{1}{16 \pi } \lim_{r \to + \infty } 
\oint_{\partial \Sigma _t} \sqrt{q} S^{i} \qty[D^{i} \gamma_{ij} - D_i \qty(f^{kl} \gamma_{kl})]  \\
&= + \frac{1}{16 \pi } \lim_{r \to + \infty } 
\oint \sqrt{q} S^{i} \qty(\partial_{j} \gamma_{ij} - \partial_{i} \gamma^{k}_{k})  \\
&= - \frac{1}{2 \pi  } \lim_{r \to \infty }
\oint \sqrt{q} S^{i} \qty(D_i \psi - \frac{1}{8} D^{j} \widetilde{\gamma}_{ij})
\,.
\end{align}

The second specifies to the background metric, the third specifies to Cartesian coordinates, the last is written in terms of conformal variables. 

If we look at the third expression, asymptotic flatness guarantees that the integral exists! 

Let us give an example, in Cartesian coordinates and for the weak-field metric. 
Here, \(\widetilde{\gamma}_{ij} = f_{ij}\), \(D^{i} \widetilde{\gamma}_{ij} = 0\), and \(\psi = 1 - \phi / 2\), and \(D_i \psi = - D_i \phi /2\). 

Let us calculate with the last definition: 
%
\begin{align}
M _{\text{ADM}} &= \frac{1}{4 \pi } \lim_{r \to \infty }
\oint_{S_r} \sqrt{f} S^{i} \qty(- D_i \phi - 0 )
\,,
\end{align}
%
where the last term vanishes because it is the divergence of the conformal metric, which is flat. 
Now we can use Green's theorem: 
%
\begin{align}
M _{\text{ADM}} &= + \frac{1}{4\pi } 
\int_{\Sigma _t} \sqrt{f} \dd[3]{x} D^{i} D_i \phi  \\
&= \frac{1}{4 \pi } \int_{\Sigma _t} \sqrt{f} \dd[3]{f} 4 \pi \rho = \int_{\Sigma _t} \sqrt{f} \dd[3]{x} \rho  
\,.
\end{align}

Let us try the same for Schwarzschild (in isotropic coordinates): here, the conformal factor is \(\psi = 1 + M/2 r\), the conformal metric is \(\widetilde{\gamma}_{ij} = f_{ij} = \diag{1, r^2, r^2 \sin^2 \theta }\). 
So, the combination we need is 
%
\begin{align}
\sqrt{q} \dd[2]{y} = r^2 \sin \theta \dd{\theta } \dd{\varphi }
\,,
\end{align}
%
which we compute at \(i_0 \), with \(\partial \Sigma _r \sim S_r\). 
We also have \(S^{i} D_i = \partial_{r}\) at \(i_0 \); so 
%
\begin{align}
M _{\text{ADM}} = - \frac{1}{2 \pi } 
\lim_{r \to \infty } \oint_{S_r} \underbrace{\pdv{\psi }{r} r^2}_{- (M / 2r^2) r^2} \sin \theta \dd{\theta } \dd{\varphi }  = M 
\,.
\end{align}

A useful theorem by Schoen and Yau in \('79\) and \('81\), and also by Witten in \('81\) says that is \(T_{ab}\) obeys the dominant energy condition (stating that for Eulerian observers \(E^2 \geq p^2\)) then: 
\begin{enumerate}
    \item \(M _{\text{ADM}}\geq 0\);
    \item \(M _{\text{ADM}} = 0\) iff \(\Sigma _t\) is Minkowski;
    \item \(\dv*{M _{\text{ADM}}}{t} = 0\), which we could have already stated by seeing that the Hamiltonian is not time dependent. 
\end{enumerate}

So, this is a good definition for the energy of the spacetime. 

If we do simulations, we are not extracting the ADM mass at infinity, and we must approximate for some finite radius. 

There is a heuristic derivation: we start from Newton, 
%
\begin{align}
M = \int_{V} \rho = \frac{1}{4 \pi } \int_{V} \Delta \phi  
= \oint_{S} S^{i} \partial_{i} \phi 
\,,
\end{align}
%
and now we must make an assumption to move to GR: \(\phi \sim h_{00} \), and \(\partial \phi \sim \partial h_{ij}\). 

The Hamiltonian constraint \(R + \text{curvature} = \text{energy}\) reads 
%
\begin{align}
\partial \partial h_{ij} + 0 = E \approx r
\,,
\end{align}
%
therefore the integral of \(\phi \) becomes \(\int_{V} R \sim \oint S^{i} \qty(\partial_{j} h_{ij} + \partial_{i} h^{k}_{k})\). 
\todo[inline]{Clarify this derivation}

Can we also have ADM momentum and angular momentum? 
Kind of --- linear momentum, yes, not angular momentum.

Linear momentum is typically associated to spatial translations. 

So, we define the ADM momentum from the boundary term at \(i_0 \) by taking \(\alpha = 0\) and \(\beta^{i} = (\partial_{k})^{i}\): 
%
\begin{align}
P_k^{\text{ADM}} &= \frac{1}{8 \pi }
\lim_{r \to + \infty } \oint_{ \partial \Sigma _t} \sqrt{q} \qty(\partial_{k})^{i} S^{j} \qty(K_{ij} - K \gamma_{ij}) 
\,.
\end{align}

Like in the other case, AF guarantees the existence of these 3 quantities. 

This \(P_k^{\text{ADM}}\) transforms like a 1-form --- it is a vector at \(i_0 \). 

Now, \(P_\alpha^{\text{ADM}} \propto (- M _{\text{ADM}} P^{\text{ADM}}_{k})\) transforms properly as a four-vector under the spin group. 

As for ``ADM angular momentum'', we must look at spatial rotations at \(i_0 \) about the 3 axes. 
They will be in the form \(\phi _x = - z \partial_{y} + y \partial_{z}\) and so on. Note that these scale like \(\order{r}\). 

We can take, at \(i_0 \), \(\alpha = 0\) and \(\beta^{i} = (\phi _k)^{i}\): 
%
\begin{align}
J_k^{\text{ADM}} = \frac{1}{8 \pi } \lim_{r \to \infty } \oint_{\partial \Sigma _t} \sqrt{q} (\phi_k)^{i} S^{j} \qty(K_{ij} - K \gamma_{ij})
\,.
\end{align}

In this case, AF does not guarantee anything: the integrand is \((K_{ij} - K \gamma_{ij}) (\phi_{k})^{j} = \order{r^{-1}}\), but sometimes the contraction with \(S^{i}\) can solve the problem. 

This is what happens in Kerr.

A big problem is the fact that \(J_k^{\text{ADM}}\) does \emph{not} transform like a 1-form at \(i_0 \). 
A working definition of \(\vec{J}\) exists only within a restricted class of transformations --- specifically, we need the stronger decay conditions
%
\begin{align}
\partial_{k} \widetilde{\gamma}_{ij} = \order{r^{-3}} \qquad \text{and} \qquad
K = \order{r^{-3}}
\,.
\end{align}

This amounts to imposing an isotropic gauge condition and an asymptotically maximal gauge.

\subsection{Komar masses}

In GR, \emph{conserved charges} appear when one has symmetries: if \(K^{a}\) is a Killing vector 
%
\begin{align}
\mathscr{L}_K g_{ab} = 2 \nabla_{(a} K_{b)} = 0
\,,
\end{align}
%
then the vector defined as
%
\begin{align}
J^{a} = T^{ab} K_b
\,
\end{align}
%
is a \emph{conserved current}: \(\nabla_a J^{a} = 0\). 
Here \(T^{ab}\) is any symmetric, divergence-free rank-2 tensor, not necessarily the stress-energy one. 
This can be proven in a rather trivial way: 
%
\begin{align}
\nabla_a J^{a} &= \nabla_a \qty(T^{ab} K_b) =  K_b \nabla_a T^{ab} + T^{ab} \nabla_a K_b = 0
\,.
\end{align}

An important current is constructed from the four-Ricci: 
%
\begin{align}
J^{a}_0 = ^4 R^{ab} K_b = \nabla_b \nabla^{a} K^{b} = - \nabla_b \nabla^{b} K^{a} 
\,,
\end{align}
%
where the indices look wrong by they are in fact not --- this does not hold in general, but it does for a Killing vector. The vector \(\nabla^{b} K^{a} = A^{ba}\) is antisymmetric. 

Now, using Stokes' theorem we can write 
%
\begin{align}
\int_{\Sigma _t} \dd[3]{x} \sqrt{\gamma } n_a \nabla_a A^{ab} 
= \oint _{\partial \Sigma _t} \dd[2]{y} \sqrt{q} \qty(S_a n_b - n_a S_b) A^{ab}  = Q_k
\,.
\end{align}

This \(Q_k\) is the conserved Komar charge. 
The vectors \(n_a\) and \(S_b\) are normal and parallel to the surface of \(\Sigma _t\) respectively. 

As an example, in a stationary spacetime \(K^{a} = (\partial_{t})^{a}\) we find the so-called Komar mass:
%
\begin{align}
Q_k = M_k 
\,,
\end{align}
%
and a useful theorem in this regard states that if \(K^{a} = n^a\) at \(i_0 \), then the Komar mass \(M_k\) and the ADM mass \(M _{\text{ADM}}\) are the same. 

In a cosmological context everything there is more messy. We cannot even assume global hyperbolicity.

\end{document}
