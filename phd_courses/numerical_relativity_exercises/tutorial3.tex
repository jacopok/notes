\documentclass[main.tex]{subfiles}
\begin{document}

\section*{Sheet 3}

\subsection{Maximal slicing}

\marginpar{Thursday\\ 2021-5-20, \\ compiled \\ \today}

Why do we want to look at maximal slicing? 
We want to supplement the evolution equation with gauge conditions. 
How does the geometry respond to those choices? 

If we take \(K = 0\) (maximal slicing), then we find an elliptic equation for the lapse function: 
%
\begin{align}
D_i D^i \alpha = \alpha K_{ij} K^{ij}
\,.
\end{align}

We start with Schwarzschild coordinates, and then map \(t \to \overline{t} = t + h(r)\), and choose \(h\) by enforcing \(K =0 \). 

We can write Schwarzschild as 
%
\begin{align}
\dd{s^2} = (1 - \frac{2m}{r}) \qty(- \dd{t^2} + \qty(1 - \frac{2m}{r})^{-2} \dd{r^2}) + r^2 \dd{\Omega^2}
\,.
\end{align}

We define a new coordinate \(r^{*}\) by 
%
\begin{align}
\dd{r*} = ( 1- \frac{2m}{r})^{-1} \dd{r} = \frac{r}{r - 2m} \dd{r}
= \qty(1 + \frac{2m}{r - 2m}) \dd{r}
\,,
\end{align}
%
therefore, up to a constant, 
%
\begin{align}
r^{*} = r + 2m \log \qty( \frac{r}{2m} - 1) 
\,,
\end{align}
%
and in terms of this the metric reads 
%
\begin{align}
\dd{s^2} = (1 - \frac{2m}{r}) \qty(- \dd{t^2} + \dd{r^{*2}}) + r^2 \dd{\Omega^2}
\,.
\end{align}

Now, null geodesics are well described by the coordinates \(u = t - r^{*}\) and \(v = t + r^{*}\). The set \((u, r, \theta , \varphi )\) or \((v, r, \theta , \varphi )\) are Eddington-Finkelstein coordinates. 

The Schwarzschild metric reads 
%
\begin{align}
\dd{s^2} &= - (1 - 2m/r) \dd{v^2} + 2 \dd{v} \dd{r} + r^2 \dd{\Omega^2} \\
\dd{s^2} &= - (1 - 2m/r) \dd{u^2} - 2 \dd{u} \dd{r} + r^2 \dd{\Omega^2}
\,.
\end{align}

The horizon, in these coordinates, is infinitely far away, but we can use the map 
%
\begin{align}
U = - \exp( - \frac{u}{4m}) 
\qquad \text{or} \qquad
V = - \exp( - \frac{v}{4m}) 
\,.
\end{align}

We can calculate \(UV\) and \(U/V\), which relate to \(r\) and \(t\).

The metric is 
%
\begin{align}
\dd{s^2} &= - \frac{32m^3}{r} \exp(- \frac{r}{2m}) \dd{U} \dd{V} 
+ r(U, V)^2 \dd{\Omega^2}
\,.
\end{align}

Now, if we move back to space-time coordinates (as opposed to null) we get 
%
\begin{align}
\dd{s^2} = \frac{32m^3}{r}  \exp(- \frac{r}{2m}) (- \dd{T^2} + \dd{X^2}) + r^2 \dd{\Omega^2}
\,,
\end{align}
%
where \(U = T-X\) and \(V = T+X\). 

\end{document}