\documentclass[main.tex]{subfiles}
\begin{document}

\marginpar{Wednesday\\ 2021-12-1}

Last time we reached the expression for the plane-wave solution to GR. 

Consider a wave propagating along the \(x^{1}\) direction. 
Our equation is \(\square _F \overline{h}_{\mu \nu } = 0 \), to be solved together with our gauge condition \(\partial_{\mu } \overline{h}^{\mu }{}_\nu = 0\). 

The solution will only depend on \(t - x /c\), which we denote as \(w\). We will have 
%
\begin{align}
\pdv{}{x} \overline{h}^{\mu }{}_{\nu } &= - \frac{1}{c} \pdv{}{w } \overline{h}^{\mu }{}_\nu \\
\pdv{}{t} \overline{h}^{\mu }{}_{\nu } &=  \pdv{}{w } \overline{h}^{\mu }{}_\nu 
\,,
\end{align}
%
so, the full condition reads 
%
\begin{align}
\partial_{\mu }  \overline{h}^{\mu }{}_{\nu } = \frac{1}{c} \pdv{}{t} \overline{h}^{0}{}_\nu + \pdv{}{x} \overline{h}^{x }{}_\nu 
= \frac{1}{c} \pdv{}{w} \qty[h^{0}{}_{\nu } - h^{x}{}_{\nu }]
= 0
\,.
\end{align}

The constant the difference of these perturbations are equal to is inessential --- it can be recovered with a rescaling of, say, background time --- so we have 
%
\begin{align}
h^{0}{}_{\nu } = h^{x}{}_{\nu }
\,.
\end{align}

The gauge we chose did not fully determine our transformation: equation \eqref{eq:christoffel-gauge-transformation} will not change if we use an additional \(\xi \) such that \(\square \xi  =0 \). 

It is possible to choose four more conditions thanks to this: 
%
\begin{align}
0 
= \overline{h}^{0}{}_{x}
= \overline{h}^{0}{}_{y}
= \overline{h}^{0}{}_{z}
= \overline{h}^{y}{}_{y} + \overline{h}^{z}{}_{z}
\,.
\end{align}

Together with the ones from before, we find 
%
\begin{align}
0 
= \overline{h}^{x}{}_{x}
= \overline{h}^{x}{}_{y}
= \overline{h}^{x}{}_{z}
= \overline{h}^{0}{}_{0}
\,,
\end{align}
%
which also means \(h = 0 = \overline{h}\) (since \(h = - \overline{h}\)), therefore \(\overline{h}_{\mu \nu } = h_{\mu \nu } \). 

The only two nonvanishing components left are \(h^{y}{}_{y} = - h^{z}{}_z\) and \(h^{x}{}_{y} = h^{y}{}_{x}\). 

We finally have 
%
\begin{align}
h_{\mu \nu } = \left[\begin{array}{cccc}
0 & 0 & 0 & 0 \\ 
0 & 0 & 0 & 0 \\ 
0 & 0 & h_{+} & h_{\times } \\ 
0 & 0 & h_{\times } & -h_{+}
\end{array}\right]
\,.
\end{align}

This is the \textbf{transverse-traceless gauge}.

\subsection{The quadrupole approximation}

We will assume that \(T_{\mu \nu } \neq 0\), but that the source of the GW is all contained within a source such that \(\abs{x'} < \epsilon\), and such that \(\epsilon \ll \lambda_{GW} = 2 \pi c / \omega \). 
This means that \(\omega \epsilon / 2 \pi \sim v _{\text{source}} \ll c\). 

This line of reasoning assumes that \(\omega _{\text{source}} \sim \omega _{\text{GW}} \), which we do not know yet --- we will later find that it is in fact true within a factor 2.

The GW solution reads 
%
\begin{align}
\overline{h}_{\mu \nu }(t, \vec{x}) &= \frac{4 G}{c^{4}}
\int_{V} \frac{T_{\mu \nu } (t - \abs{x - x'} / c, c') \dd[3]{x'}}{\abs{x - x'}}
\,,
\end{align}
%
while in Fourier space we can expand 
%
\begin{align}
T_{\mu \nu } (t, \vec{x}) = \int \widetilde{T}_{\mu \nu } (\omega , \vec{x}) e^{-i \omega t} \dd{\omega }
\,.
\end{align}

We then get 
%
\begin{align}
\int \overline{h}_{\mu \nu } (\omega , \vec{x}) e^{-i \omega t} \dd{\omega } = \frac{4 G}{c^{4}} 
\int_{V} \frac{ \dd[3]{x'}}{\abs{x - x'}} \int T_{\mu \nu } (\omega , x') e^{-i \omega (t - \abs{x-x'} / c) } \dd{\omega }
\,,
\end{align}
%
which means that 
%
\begin{align}
h_{\mu \nu } (\omega , x) = \frac{4G}{c^{4}} \int 
\frac{\dd[3]{x'}}{\abs{x - x'}}
T_{\mu \nu }(\omega , x') e^{i \omega \abs{x-x'} / c}
\,,
\end{align}
%
where we factored out the \(e^{- i \omega t}\) and the integral in \(\dd{\omega }\). 

With our assumption of slow speed we can expand: 
%
\begin{align}
\frac{e^{i \omega \abs{x-x'} / c}}{\abs{x-x'}} \approx 
\frac{e^{i \omega r}}{r} 
\,,
\end{align}
%
where \(r = \abs{x}\). 
This then yields 
%
\begin{align}
h_{\mu \nu }(\omega , r) = \frac{4 G}{c^{4}} \frac{e^{i\omega r}}{r}
\int_{V} T_{\mu \nu } (\omega , x') \dd[3]{x'}
\,,
\end{align}
%
so we can come back to 
%
\begin{align}
\overline{h}_{\mu \nu } (t, r) = \frac{4 G}{c^{4}r} \int T_{\mu \nu } (t- r/c, x') \dd[3]{x'}
\,.
\end{align}

We can simplify this thanks to the expression \(T^{\mu \nu }{}_{, \nu } = 0\), which means we have conservation laws in the form \(\int T^{\mu 0} \dd[3]{x}\). 

We can put these constants to zero (since we are not interested in any non-wavelike behavior, like the stationary Kerr-like metric due to the source). 

We can do an integral of \(\partial_{\mu } T^{\mu \nu }= 0\): 
%
\begin{align}
\frac{1}{c} \pdv{}{t} \int _V T^{n  0} x^{k} \dd[3]{x} &= - \int \pdv{T^{ni}}{x^i} x^{k} \dd[3]{x}  \\
&= \underbrace{\int \dd{S^{i}} (T^{ni} x^{k})}_{ \to 0} + \int T^{nk} \dd[3]{x}  \\
\frac{1}{c} \pdv{}{t} \int x^{k} T^{n0} \dd[3]{x} &= \int T^{nk} \dd[3]{x}
\,.
\end{align}

We can write this as 
%
\begin{align}
\frac{1}{2} \pdv{}{t} \int \qty[ T^{n0} x^{k} + T^{k0} x^{n}] = \int T^{nk} \dd[3]{x}
\,.
\end{align}

The time component of the conservation law can be multiplied by \(x^{n} x^{k}\): 
%
\begin{align}
\frac{1}{c} \pdv{}{t} \int T^{00} x^{n} x^{k} \dd[3]{x} 
&= - \int \pdv{}{x^{i}} T^{0i} x^{n} x^{k} \dd[3]{x}  \\
&= + \int T^{0i} \pdv{}{x^{i}} (x^{n} x^{k}) \dd[3]{x}  \\
&= \int T^{0n} x^{k} + T^{0k} x^{n}  \dd[3]{x} 
\,.
\end{align}

We then take a second derivative: 
%
\begin{align}
\frac{1}{c^2} \pdv[2]{}{t} \int T^{00} x^{n} x^{k} \dd[3]{x} &= 
\frac{1}{c} \pdv{}{t} \int T^{n0} x^{k} + T^{k0} x^{n} \dd[3]{x}  \\
&= 2 \int T^{nk} \dd[3]{x}
\,.
\end{align}

This is the \textbf{virial theorem} in GR. 

We will assume we are working on a \(t = \text{const}\) hypersurface. 
The metric is purely Euclidean there. 

The object 
%
\begin{align}
\frac{1}{c^2} \int T^{00} x^{n} x^{k} \dd[3]{x} = q^{n k} (t) 
\,,
\end{align}
%
the quadrupole tensor. 
We then have 
%
\begin{align}
\frac{1}{2} \ddot{q}^{n k} (t) = \int T^{nk} \dd[3]{x}
\,.
\end{align}

The result is therefore \(h^{\mu 0} = 0\), and 
%
\begin{align}
\overline{h}^{n k} = \frac{2G}{c^{4} r} \ddot{q}^{n k}(t)
\,.
\end{align}



\end{document}