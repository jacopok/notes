\documentclass[main.tex]{subfiles}
\begin{document}

\begin{extracontent}
Exercise: Einstein's carousel (2.2). 

The transformation law to and from flat spacetime is according to the matrix
%
\begin{align}
L^{\mu }_{\nu }= \left[\begin{array}{cccc}
1 & 0 & 0 & 0 \\ 
0 & \cos ( \omega t)  & - \sin ( \omega t)  & 0 \\ 
0 & \sin ( \omega t)  & \cos ( \omega t)  & 0 \\ 
0 & 0 & 0 & 1
\end{array}\right]
\,,
\end{align}
%
so that \(\xi^{\mu } = L^{\mu }_{\nu } x^{\nu }\), and \(x^{\mu } = \widetilde{L}^{\mu }_{\nu } \xi^{\nu }\), where \(\widetilde{L}\) is the inverse of \(L\), which can be obtained by mapping \(t \to -t\) in it. 

Now, we need to compute the derivatives of this transformation in the expression for the Christoffel symbols: 
%
\begin{align}
\Gamma^{\mu }_{\alpha \beta } = \pdv{\xi^{\rho }}{x^{\alpha }}{x^{\beta }} \pdv{x^{\mu }}{\xi^{\rho }}
\,.
\end{align}

The transformation matrix \(\widetilde{L}^{\mu }_{\nu }\) depends on \(\xi \) only through its component \(\xi^{0} = t\):
%
\begin{align}
\pdv{x^{\mu }}{\xi^{\rho }} &= \pdv{}{\xi^{\rho }} \qty(\widetilde{L}^{\mu }_{\nu } \xi^{\nu })  \\
&= \widetilde{L}^{\mu }_{\nu } \delta^{\nu }_{\rho } + \pdv{\widetilde{L}^{\mu }_{\nu }}{\xi^{\rho }} \xi^{\nu }  \\
&= \widetilde{L}^{\mu }_{\rho } + \widetilde{\dot{L}}^{\mu }_{\nu } \xi^{\nu } \delta^{t}_{\rho }
\,,
\end{align}
%
and the reciprocal transformation is quite similar: 
%
\begin{align}
\pdv{\xi^{\rho }}{x^{\beta }} &= \pdv{}{x^{\beta }} \qty(L^{\rho }_{\sigma } x^{\sigma })  \\
&= L^{\rho }_{\beta } + \dot{L}^{\rho }_{\sigma  } \xi^{\sigma } \delta^{t}_{\beta }
\,.
\end{align}

Now, however, we need to take a second derivative of this term: 
%
\begin{align}
\pdv{\xi^{\rho }}{x^{\alpha }}{x^{\beta }} &= \pdv{}{x^{\alpha }} \qty(L^{\rho }_{\beta } + \dot{L}^{\rho }_{\sigma  } \xi^{\sigma } \delta^{t}_{\beta })  \\
&= 
\dot{L}^{\rho }_{\beta } \delta^{t}_{\alpha } 
+ \dot{L}^{\rho }_{\sigma } \delta^{\sigma }_{\alpha } \delta^{t}_{\beta }  
+ \ddot{L}^{\rho }_{\sigma } \delta^{t}_{\alpha } \xi^{\sigma } \delta^{t}_{\beta }  \\
&= 2 \dot{L}^{\rho }_{(\alpha } \delta^{t}_{\beta )} +\ddot{L}^{\rho }_{\sigma } \xi^{\sigma } \delta^{t}_{\alpha } \delta^{t}_{\beta }
\,,
\end{align}
%
which we can combine into the full Christoffel symbol expression, which is (thankfully) manifestly symmetric in \(\alpha \beta \): 
%
\begin{align}
\Gamma^{\mu }_{\alpha \beta } &= \qty(\dot{L}^{\rho }_{\beta } \delta^{t}_{\alpha } 
+ \dot{L}^{\rho }_{\alpha } \delta^{t}_{\beta } + \ddot{L}^{\rho }_{\sigma } \xi^{\sigma } \delta^{t}_{\alpha } \delta^{t}_{\beta }) 
\qty(\widetilde{L}^{\mu }_{\rho } + \widetilde{\dot{L}}^{\mu }_{\nu } \xi^{\nu } \delta^{t}_{\rho })  \marginnote{Terms like \(\dot{L}^{\rho } \delta^{t}_{\rho }\) vanish!}\\
&= 
\dot{L}^{\rho }_{\beta } \delta^{t}_{\alpha } \widetilde{L}^{\mu }_{\rho } 
+ 
\dot{L}^{\rho }_{\alpha } \delta^{t}_{\beta } \widetilde{L}^{\mu }_{\rho } 
+ \ddot{L}^{\rho }_{\sigma } \widetilde{L}^{\mu }_{\rho } \xi^{\sigma } \delta^{t}_{\alpha } \delta^{t}_{\beta } 
\,,
\end{align}
%
which is already starting to look like Coriolis + centrifugal terms.

We can write 
%
\begin{align}
\ddot{L}^{\rho }_{\sigma } = - \omega^2 \left[\begin{array}{cccc}
0 & 0 & 0 & 0 \\ 
0 & \cos(\omega t) & -\sin(\omega t) & 0 \\ 
0 & \sin(\omega t) & \cos(\omega t) & 0 \\ 
0 & 0 & 0 & 0
\end{array}\right] 
= - \omega^2 \qty( L^{\rho }_{\sigma }  
- \delta^{\rho }_{t} \delta^{t}_{\sigma }
- \delta^{\rho }_{z} \delta^{z}_{\sigma })
\,,
\end{align}
%
using which we can rewrite the centrifugal term like 
%
\begin{align}
\ddot{L}^{\rho }_{\sigma } \widetilde{L}^{\mu }_{\rho } \xi^{\sigma } \delta^{t}_{\alpha } \delta^{t}_{\beta }
&= - \omega^2 \qty( \delta^{\mu }_{\sigma } \xi^{\sigma } \delta^{t}_{\alpha } \delta^{t}_{\beta } 
- \delta^{\rho }_{t} \delta^{t}_{\sigma }
\widetilde{L}^{\mu }_{\rho } \xi^{\sigma } \delta^{t}_{\alpha } \delta^{t}_{\beta }
- \delta^{\rho }_{z} \delta^{z}_{\sigma }
\widetilde{L}^{\mu }_{\rho } \xi^{\sigma } \delta^{t}_{\alpha } \delta^{t}_{\beta }
)  \\
&= \omega^2 
\xi^{\mu  } \delta^{t}_{\alpha } \delta^{t}_{\beta } = \omega^2 
L^{\mu }_\nu x^{\nu } \delta^{t}_{\alpha } \delta^{t}_{\beta }
\,.
\end{align}

Let us now move to the Coriolis term: we will get identical results if we swap \(\alpha \) and \(\beta \), and we know that one of them must be \(t\), so let us set \(\alpha = t\): we get 
%
\begin{align}
\qty(\Gamma^{\mu }_{t \beta }) _{\text{Coriolis}} &= \dot{L}^{\rho }_{\beta } 
L^{\mu }_{\rho } \\
&= \omega \left[\begin{array}{cccc}
0 & 0 & 0 & 0 \\ 
0 & -\sin(\omega t) & -\cos(\omega t) & 0 \\ 
0 & -\cos(\omega t) & \sin(\omega t) & 0 \\ 
0 & 0 & 0 & 0
\end{array}\right]_{\beta }^{\rho }
\left[\begin{array}{cccc}
1 & 0 & 0 & 0 \\ 
0 & \cos(\omega t) & \sin(\omega t) & 0 \\ 
0 & -\sin(\omega t) & \cos(\omega t) & 0 \\ 
0 & 0 & 0 & 1
\end{array}\right]^{\mu }_{\rho }  \\
&= \omega \left[\begin{array}{cccc}
0 & 0 & 0 & 0 \\ 
0 & 0 & -1 & 0 \\ 
0 & 1 & 0 & 0 \\ 
0 & 0 & 0 & 0
\end{array}\right]^{\mu }_{\beta }  \\
&= \omega \epsilon^{\mu z}_{t \beta }
\,,
\end{align}
%
where the \(\epsilon \) is the four-dimensional Levi-Civita symbol. 
This also shows that the symmetric \(\beta = t \) term equals \(\omega \epsilon^{\mu z}_{\alpha t}\). 

We can thus compactly write 
%
\begin{align}
\Gamma^{\mu }_{\alpha \beta } = \omega \qty(\epsilon^{\mu z}_{\alpha t} \delta^{t}_{\beta } + \epsilon^{\mu z }_{t \beta } \delta^{t}_{\alpha  }) + \omega^2 L^{\mu }_{\nu }x^{\nu } \delta^{t}_{\alpha } \delta^{t}_{\beta }
\,.
\end{align}

Now, let us write the geodesic equation: 
%
\begin{align}
\ddot{x}^{\mu } &= - \Gamma^{\mu}_{\alpha \beta } \dot{x}^{\alpha } \dot{x}^{\beta }  \\
&= - \qty[\omega \qty(\epsilon^{\mu z}_{\alpha t} \delta^{t}_{\beta } + \epsilon^{\mu z }_{t \beta } \delta^{t}_{\alpha  }) + \omega^2 L^{\mu }_{\nu }x^{\nu } \delta^{t}_{\alpha } \delta^{t}_{\beta }] \dot{x}^{\alpha } \dot{x}^{\beta }
\,.
\end{align}

The sum includes several terms: for example, the \(\alpha = \beta = t\) one reads 
%
\begin{align}
- \omega^2 L^{\mu }_{\nu } x^{\nu } (\dot{x}^{t} )^2
\,.
\end{align}

Let us look at this more concretely: the four-velocity of a timelike trajectory is \(u^{\mu } = \dv*{x^{\mu }}{\tau }\); which in terms of the three-velocity \(\vec{v}\) reads \(u^{\mu } = [\gamma , \gamma \vec{v}]\), where \(\gamma = 1 / \sqrt{1 - \abs{\vec{v}}^2}\). 

In our case, \(\dot{x}^{\mu } = u^{\mu }\). 

Let us then look at the \(\mu = i\), spatial components of this equation: 
%
\begin{align}
\dv{}{\tau } \qty(\gamma v^{i}) &=
- \omega \qty[
    \epsilon^{iz}_{jt} u^{j}
    + \epsilon^{iz}_{tj} u^{j}
    ] u^{t} 
- \omega^2 \xi^{i}
(u^{t})^2  \\
\dv{}{\tau } \qty(\gamma v^{x} ) &= - 2\omega u^{y} u^{t} - \omega^2 \xi^{x} (u^{t})^2 \\
\dv{}{\tau } \qty(\gamma v^{y} ) &= + 2\omega u^{x} u^{t} - \omega^2 \xi^{y} (u^{t})^2 
\,.
\end{align}

\todo[inline]{There's a wrong sign in here somewhere.}

In the nonrelativistic limit \(\gamma \to 1\), \(u^{t} \to 1\), \(\tau = t\), so we just get 
%
\begin{align}
\ddot{x} &= - 2 \omega \dot{y} - \omega^2 x  \\
\ddot{y} &= - 2 \omega \dot{x} - \omega^2 y   
\,.
\end{align}
\end{extracontent}

\begin{extracontent}
Exercise: variation of a vector (1.5). 

We want to show that, when parallel-transporting a vector along a loop, it varies by 
%
\begin{align}
\delta a^{\lambda } = - \frac{1}{2} a^{\rho } R^{\lambda}_{\rho \alpha \beta} \dd{x^{\alpha }} \dd{x^{\beta }}
\,.
\end{align}

We denote the closed curve along which we transport the vector as \(\gamma \). 
Its tangent vector will be \(\dot{\gamma}\), and its parameter will be \(t\) 
(not the time coordinate). 
The parallel transport equation reads 
%
\begin{align}
0 = \dot{\gamma}^{\mu } \nabla_\mu a^{\lambda } = \dot{\gamma}^{\mu } \qty( \partial_{\mu } a^{\lambda } + \Gamma^{\lambda }_{\mu \alpha } a^{\alpha })
\,.
\end{align}

The quantity we want to compute is 
%
\begin{align}
\oint_\gamma \dd{a^{\lambda }} &= \oint_\gamma \dv{a^{\lambda }}{t} \dd{t}  \\
&= \oint_\gamma \dot{\gamma}^{\mu } \partial_{\mu } a^{\lambda } \dd{t}  \\
&= - \oint_{\gamma } \dot{\gamma}^{\mu } \Gamma^{\lambda }_{\mu \alpha } a^{\alpha } \dd{t}  \\
&= - \oint_{\partial \Sigma } \Gamma^{\lambda }_{\mu \alpha  } a^{\alpha } \dd{x^{\mu }}
\marginnote{We identify \(\gamma \) with the border of a region \(\Sigma \).}  \\
&= - \int_\Sigma \partial_{\nu} \qty(\Gamma^{\lambda}_{\mu \alpha } a^{\alpha }) \dd{x^{\mu }} \wedge \dd{x^{\nu }}  \\
&= - \int_\Sigma \partial_\nu \Gamma^{\lambda }_{\mu \alpha } - \Gamma^{\lambda }_{\mu \rho } \Gamma^{\rho }_{\nu \alpha } a^{\alpha } \dd{x^{\mu }} \wedge \dd{x^{\nu }}  \\
&= \int \qty(\partial_{[\mu } \Gamma^{\lambda }_{\nu ] \alpha } + \Gamma^{\lambda }_{\rho [\mu } \Gamma^{\rho }_{\nu ] \alpha }) a^{\alpha } \dd{x^{\mu }} \dd{x^{\nu }}  \\
&= - \int _\Sigma \frac{1}{2} R^{\lambda }_{\alpha \mu \nu } a^{\alpha } \dd{x^{\mu }} \dd{x^{\nu }}
\,.
\end{align}

The result then follows by taking the limit of a small integration region, so that the Riemann tensor inside the expression is roughly constant. 

% Can we do \(\partial_{\nu } a^{\alpha } = - \Gamma^{\alpha }_{\nu \rho } a^{\rho }\) though? That is not true in general, but only along \(\dot{\gamma}^{\nu }\)\dots
\end{extracontent}

\end{document}