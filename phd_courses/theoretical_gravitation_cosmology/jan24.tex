\documentclass[main.tex]{subfiles}
\begin{document}

\marginpar{Monday\\ 2022-1-24}

Sabino \heartsuit. 

We distinguish the \emph{Copernican} principle from the \emph{cosmological} principle. 

Isotropy: symmetry with respect to rotations. 
We observe this only on large enough scales. 

Why do we observe peculiar velocity for the Local Group?
Is there a Great Attractor in the nearby universe, which means the 
Milky Way is moving in a certain direction? 

This does not violate Relativity; matter intrinsically defines 
a specific reference frame. 

In cosmology, the time coordinate \(t\) of comoving observers 
is called ``cosmic time''. 

What is the length scale over which we must smooth in order to 
retrieve homogeneity? 
Initially people thought \SI{5}{Mpc} was enough, 
now people are thinking we need at least a few hundreds of \SI{}{Mpc}. 

Observational cosmology: we define the geometrical model only on the basis 
of what we could observe in principle, that is, our past light-cone. 
That is way more complicated than just taking Robertson-Walker. 

For a lot of the discussion here we will not need the full dynamical spacetime. 

Anti-de Sitter is not physical, but it is theoretically interesting. 

Ellis says that: in the real world we do not have a fluid, but particles instead. 
There is no Ricci, but only Weyl. 
This means that the cosmological, fluid-like description is quite interesting! 

The particle physics people say that \(k\) is made to be zero by inflation,
but this is not the case! \(k\) is a constant of motion. 

Lenticular galaxies vs spiral galaxies. 

Galaxies are rotating: is there vorticity in the universe? 
Kelvin circulation theorem: if there is no vorticity initially, there 
are. 

Galaxies rotate due to tidal torque: properties of a macroscopic gravitating fluid. 

 

\end{document}