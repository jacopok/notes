\documentclass[main.tex]{subfiles}
\begin{document}

\marginpar{Tuesday\\ 2022-1-25}

Elliptical galaxies are basically in equilibrium.

What is the potential composition of DM?
``Baryons'' in cosmology are made to include electrons as well for simplicity. 
Can DM be baryonic matter? 

The idea currently is that DM is made of elementary particles which interact negligibly.
Current candidates include neutralinos and axions, but there seems to be no evidence
for supersymmetry. 
People are also considering more exotic models, including a modification of gravity.

Gauss-Codazzi equation combining intrinsic and extrinsic curvature of a hypersurface. 

The spatial Ricci tensor is proportional to the spatial metric! 
\(\widetilde{R}_{\alpha \beta} = 2k \widetilde{g} _{\alpha \beta }\).  

The form of the SEMT is assumed to be the isotropic one. 

There are other possibilities than a perfect fluid: 
an imperfect fluid only involving bulk viscosity is allowed, as that would 
be isotropic.
Another possibility is to have a scalar field; 
as long as we take the VEV of the field we can get something 
which behaves in the right way. 

Perfect fluid is \emph{defined} by the diagonal SEMT. 

\todo[inline]{Is the VEV of an unperturbed scalar field
actually different from the perfect fluid?} 

The first FE is the ``energy constraint''. 
The second is the ``acceleration equation''. 
The third is the continuity equation. 

We can introduce the volume expansion \(\theta = u^\alpha_{; \alpha }\)
and \(\dot{u}_b = u^a u_{b; a}\), so we get 
%
\begin{align}
( \dot{\rho} + \dot{P}) u_b + \dots
\,,
\end{align}
%
we get an alternative formulation, since \(\theta = 3H\). 

We also get the Euler equation if we project onto spatial surfaces. 

The physical principle behind the non-independence of the Friedmann equations 
is the fact that the Bianchi identities are, indeed, identities. 

There is extra symmetry if \(w\) is a constant! 

\end{document}