\documentclass[main.tex]{subfiles}
\begin{document}

\marginpar{Monday\\ 2022-1-31}

We need a period of accelerated expansion, \(\ddot{a} > 0\). 

If \(Z = a_f / a_i\) is the factor by which the universe inflated, 
how large do we need it to be? 

We can write it as 
%
\begin{align}
Z = \exp( \int_{t_i}^{t_f} \dd{t} H(t)) = \exp( N _{\text{inf}})
\,.
\end{align}

The region which was causally connected at the beginning of inflation, 
\(a(t_i) r_H (t_i)\), can contain the present Hubble radius scaled 
back to the end of inflation \(t_f\). 

This is the minimum inflation: it comes out to be roughly \(N _{\text{inf}} \gtrsim 60\).

History of inflation: 
Starobinski in 1979--80 showed that a de Sitter stage in the early Universe
could be driven by trace anomalies of quantum fields. 
He considered models in which there were quadratic curvature terms 
in the stress-energy tensor. 
This phase would be terminated by the generation of scalar field fluctuations, 
``scalarons''. 

The Starobinski model, with an action like \(R + R^2 / 6M^2\), 
is currently the best fit to the data. 

Guth in 1981 considered inflation caused by a first-order phase transition.

Many groups in 1982--83 discussed density fluctuations' dependence on the 
coupling between the inflaton and matter fields. 

Linde in the 1980s thought about chaotic inflation.  

The energy scale during inflation is quite uncertain; 
it should be between \SI{1}{MeV} and \SI{e16}{GeV}. 

Casimir effect with \(F \sim A / d^4\)! 

What are our options for a field providing vacuum energy?

\(\expval{\phi } \neq 0\)? YES

\(\expval{A_\alpha  } \neq 0\)? NO 

\(\expval{ \psi   } \neq 0\)? NO 

\(\expval{ \overline{\psi} \psi } \neq 0\)? YES 

We could, however, have a small degree of isotropy violation, consistent with 
CMB data. 

People currently are thinking that the Higgs boson might not 
be an elementary particle\dots ???

\end{document}