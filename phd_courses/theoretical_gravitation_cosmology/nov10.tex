\documentclass[main.tex]{subfiles}
\begin{document}

\section{General Relativity basics}

\marginpar{Wednesday\\ 2021-11-10, \\ compiled \\ \today}

Course given by Andrea Maselli.
There's lots of stuff to say, and the course cannot possibly be comprehensive. 

The exam can be two things: there will be a list of ``old'' papers, literature by now, to study in more detail. 
We can prepare a presentation on this, to be shared among each other. 

The alternative is to study more ``book-like'' topics: say, the TOV solution. 

Three standard reference books are 
\begin{enumerate}
    \item Bernard Schultz, ``A first course in GR'';
    \item Clifford and Will, ``Theory and experiments of gravitational physics'', which is more experimentally oriented;
    \item Ferrari, Gualtieri and Pani, ``GR and its applications'', which is very complete. 
\end{enumerate}

\subsection{The equivalence principles}

There is a plural in the title. 
Principia impose constraints which modify the theory, not the other way around. 

The thing to be careful with is whether we are testing \emph{principia} or a specific theory. 
For example, the Pound-Rebka redshift experiment was testing the equivalence principle. 

Alternative theories, such as scalar-tensor theories which were pioneered by Brams and Dicke, could also be written satisfying the same equivalence principle. 

We can then classify alternative theories of gravity according to the principles they satisfy. 

\paragraph{Newton Equivalence Principle} This is also dubbed ``NEP''. 

``The quantity that I mean hereafter by the name of mass\dots''

He distinguishes ``mass'' and ``weight'', and his formulation is not a principle yet. 

\begin{quotation}
In the Newtonian limit, the inertial and gravitational masses are all proportional. 
\end{quotation}

Newton's theory says that a density \(\rho (x(t), t)\) will source a gravitational potential 
%
\begin{align}
\nabla^2 \phi (x(t), t) = 4 \pi G \rho 
\,,
\end{align}
%
and particles will move according to the law 
%
\begin{align}
m \ddot{x} (t) = - m \nabla \phi (x(t), t)
\,.
\end{align}

The density \(\rho \) and the mass \(m\) are conceptually different: \(\rho = \rho _{\text{act}}\) is the \emph{active} gravitational density, the mass \(m = m _{\text{in}}\) multiplying \(\ddot{x}\) is the inertial mass, while the \(m = m _{\text{pass}}\) multiplying \(\nabla \phi \) is the passive gravitational mass. 

We can show that the active and passive gravitational masses are proportional by exploiting the third of Newton's laws: consider two objects 0 and 1, the attraction forces exerted by 0 on 1 and respectively by 1 on 0 will be 
%
\begin{subequations}
\begin{align}
f_1 &= m_1^{\text{in}} a_1 = \frac{G m_0^{\text{act}} m_1^{\text{pass}}}{r^2} \\
f_0 &= m_0^{\text{in}} a_0 = \frac{G m_1^{\text{act}} m_0^{\text{pass}}}{r^2}
\,,
\end{align}
\end{subequations}
%
but the third law imposes \(f_0 = f_1 \), which means 
%
\begin{align}
\frac{m_0^{\text{pass}}}{m_0^{\text{act}}} =
\frac{m_1^{\text{pass}}}{m_1^{\text{act}}}
\,.
\end{align}

Therefore, these are proportional. 
Setting the proportionality constant to 1 is just a matter of the units with which we measure them.
Therefore, we can say \(m _{\text{act}} = m _{\text{pass}} = m _{\text{grav}}\). 

If we have one more body, 2, the force exerted by 0 will be 
%
\begin{align}
f_2 = \frac{G m_0^{\text{act}} m_2^{\text{pass}}}{r^2}
\,,
\end{align}
%
so the accelerations of bodies 1 and 2 will be \(a_1 = f_1 / m_1^{\text{in}}\) and \(a_2 = f_2 / m_2^{\text{in}}\).

If these are equal, then we get \(m_1^{\text{pass}} / m_1^{\text{in}} = m_2^{\text{pass}} / m_2^{\text{in}}\).

A way to quantify discrepancies from this principle is 
%
\begin{align}
\eta = \frac{\frac{m_1^{\text{grav}}}{m_1^{\text{in}}} - \frac{m_2^{\text{grav}}}{m_2^{\text{in}}}}{\frac{m_1^{\text{grav}}}{m_1^{\text{in}}} + \frac{m_2^{\text{grav}}}{m_2^{\text{in}}}}
\,.
\end{align}

We know from experiments that \(\abs{\eta } < \num{e-13}\). 
In the Newtonian language, this is purely a coincidence. 

Consider a nonrelativistic object moving in an external gravitational field, so that 
%
\begin{align}
m _{\text{in}} \ddot{x} = m _{\text{grav}} g + \underbrace{\sum _{k}f_k}_{F}
\,,
\end{align}
%
where \(f_k\) are all other forces affecting this body. 
We are assuming that \(g\) is a constant.

We can make a transformation \(y(t) = x(t) - g t^2 / 2\), \(t' = t\). 
The equation of motion will be 
%
\begin{align}
m _{\text{in}} \ddot{y} = (m _{\text{grav}} - m _{\text{in}} ) g + F
\,.
\end{align}

If the equivalence principle holds, the equation of motion becomes \(m _{\text{in}} \ddot{y} = F\). 
In this reference system, gravity disappears completely. 

We can see hints of the locality of this procedure: we are neglecting any derivatives of \(g\), which will never be truly constant. 
This is the very meaning of a ``local inertial observer''. 

How does this work in GR? 
(we are actually doing this in the simplest possible, SR terms)

\paragraph{Weak Equivalence Principle}

\begin{quotation}
``The trajectory of an uncharged test particle in a point of spacetime is independent of that particle's structure and composition.''
\end{quotation}

What ``test particle'' means is that it has negligible self-gravity, and that it is small enough that it does not couple to the inhomogeneities of the gravitational field. 

The first statement can be quantified by the compactness parameter: \(GM / c^2 R\). 
For the Sun, this is of the order of \num{e-6}. 

The second statement can be quantified by saying that the size of the body should be small compared to the length scale of the curvature of spacetime. 

These two constraints are logically independent.
In GR, the term ``mass'' is quite difficult to deal with. 
Therefore, instead of discussing them we only talk of trajectories. 

``Test experiment'' means that its effects are ``weak''. 

\paragraph{Einstein Equivalence Principle}

There are two alternative formulations.

\begin{quotation}
    ``The outcome of any non-gravitational test experiment is not affected locally, at any point in spacetime, by the presence of the gravitational field.''
\end{quotation}

\begin{quotation}
    ``The outcome of any non-gravitational test experiment is independent of the position of the lab in space and of the velocity of the free-falling apparatus.''
\end{quotation}

The second formulation more explicitly requires invariance under the Poincaré group. 

In the WEP we are only talking about mechanical laws, while here we extend to all the laws of physics. 

This means that we can always find a reference system which cancels gravity. 

This is establishing the connection between local frames in a gravitational field, and frames in the absence of gravity.

\todo[inline]{This sounds a bit circular without a careful definition of a ``gravitational experiment''\dots} 

\paragraph{Strong equivalence principle}

\begin{quotation}
    ``The outcome of any experiment is not affected at any point in spacetime by the presence of the gravitational field.''
\end{quotation}

There is no proof that SEP implies GR, but GR surely satisfies it.

The EEP is what tells us that the theory must be a metric one, since it includes SR, which has the Minkowski tensor. 

There must be (at least one) metric tensor, so that it can be made equal to the Minkowski tensor at each point in spacetime, up to a conformal transformation: 
%
\begin{align}
[g_1, g_2 \dots ] \to [\phi_1 (P) \eta, \phi_2 (p) \eta,  \dots]
\,.
\end{align}

If this is the case, we will be able to rescale \(\phi _i = C_i \phi (P)\), but then we can rescale our units so that \(C_i = 1\), and the metric as \(\overline{g} = \phi^{-1} g\), which finally yields the transformation \(g \to \eta \). 

In the end, then, there must be a reference frame such that  
%
\begin{align}
 g_{\mu \nu } (P) = \eta_{\mu \nu } + \sum _{\alpha } \order{\abs{x^{\alpha } - x^{\alpha } (P)}^2}
\,.
\end{align}

In a Local Lorentz Frame there is a family of preferred curves for \(g_{\mu \nu }\), geodesics, which are straight lines: free-falling trajectories are straight lines for free-falling observers. 

There are purely metric theories, with a single metric tensor, but also ones in which other fields mediate the interaction. 

The presence of the LLF means that there is a frame such that, around a point, \(\dd{s^2} = \eta_{\mu \nu } \dd{y^{\mu }} \dd{y^{\nu }}\). 

What are the EoM for a free-falling particle? 
We know that they read \(\dv[2]{y}{\tau } = 0\) in the LLF. 

Suppose we then move to a different frame, \(x^{\alpha } = x^{\alpha } (y)\): then the interval will transform as 
%
\begin{align}
\dd{s^2} = \underbrace{\eta_{\mu \nu } 
\pdv{y^{\mu }}{x^{\alpha }}
\pdv{y^{\nu }}{x^{\beta   }}}_{g_{\alpha \beta }}
\dd{x^{\alpha }} \dd{x^{\beta }}
\,.
\end{align}

The EoM then reads 
%
\begin{subequations}
\begin{align}
0 = \dv{}{\tau } \qty(\pdv{y^{\alpha }}{x^{\beta }} \dv{x^{\beta }}{\tau }) 
&=  \dv[2]{x^{\beta }}{\tau } \pdv{y^{\beta }}{x^{\beta }} 
+ \pdv[2]{y^{\alpha }}{x^{\beta }}{x^{\gamma }} \dv{x^{\beta }}{\tau } \dv{x^{\gamma }}{\tau }  \\
&= \dv[2]{x^{\beta }}{\tau } \underbrace{\pdv{y^{\beta }}{x^{\beta }}
\pdv{x^{\rho }}{x^{\beta }}}_{ \delta_{\beta }{}^{\rho }} 
+ \underbrace{\qty( \pdv{x^{\rho }}{y^{\alpha }} \pdv[2]{y^{\alpha }}{x^{\beta }}{x^{\gamma }} )}_{ \Gamma^{\rho }_{\beta \gamma }}
\dv{x^{\beta }}{\tau }  \dv{x^{\gamma }}{\tau }  \\
&= \dv[2]{x^{\rho }}{\tau } + \Gamma^{\rho }_{\beta \gamma } \dv{x^{\beta }}{\tau }  \dv{x^{\gamma }}{\tau }
\,.
\end{align}
\end{subequations}

This is the \textbf{Geodesic equation}.
The crucial thing is that \(\Gamma \) vanishes when the spacetime is flat. 

The WEP implies the NEP. 
On the other hand, the NEP does not imply the WEP. 
A theory in which the equations of motion contain weird stuff and not just \(m_I / m_G\) satisfies the NEP and not the WEP. 

% \todo[inline]{But, are we supposed to consider these as ``theory-independent''? for the NEP, maybe not\dots}

\end{document}
