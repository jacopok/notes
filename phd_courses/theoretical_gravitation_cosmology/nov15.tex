\documentclass[main.tex]{subfiles}
\begin{document}

\subsection{Differential geometry}

\marginpar{Monday\\ 2021-11-15}

A point in \(\mathbb{R}^{n}\) is described by the \(n\)-tuple \((x_1, \dots, x_n)\). 
These are a dense set (and also a complete one).

Open sets in \(\mathbb{R}^{n}\) are defined as follows: \(S \subseteq \mathbb{R}^{n}\) is open if for all \(x \in S\) we can make a ball \(B_r(x)\) with \(r > 0\) such that \(B_x \subseteq S\), where a ball is a set 
%
\begin{align}
B_r (x) = \qty{y \in \mathbb{R}^{n} | \abs{x - y} < r}
\,.
\end{align}

A topological space is a collection of open sets which is closed under arbitrary intersections and finite unions. 

A map \(\rho \) between two sets \(M\) and \(N\) takes a point \(x \in M\) to a point \(y = f(x) \in N\). 

If we consider a set \(S \subseteq M\) we can look at the set of the images of \(S\) under \(f\), denoted as \(T = \rho (S) \subseteq N\). 
We can also define the inverse image, \(S = \rho^{-1} (T)\).

\textbf{Composition} of maps between different sets can also be defined: suppose we have three spaces \(M\), \(N\), \(Z\), with some maps \(f \colon M \to N\) and \(g \colon N \to Z\); their composition will be \(g \circ f \colon M \to Z\). 

A map of \(M\) is \textbf{into} \(N\) if all points of \(M\) are mapped to \(N\); a map of \(M\) is \textbf{onto} \(N\) if all points of \(N\) have inverse maps to \(M\). 

A \textbf{continuous} map \(f\) between topological spaces \(M\) and \(N\) is one for which, given any point \(x \in M\) such that \(y = f(x) \in N\), there is an open set of \(N\) containing \(f(x)\), which is the image of an open set of \(M\). 

\todo[inline]{This is less stringent than the \(\epsilon \)-\(\delta \) definition of continuity because there is no size requirement\dots ??}

A differentiable map is defined as follows: a function \(f (x) \), where \(x \in S\) and \(S \subseteq \mathbb{R}^{n}\) is open, is of order \(\mathcal{C}^{k}\) if all partial derivatives of order up to \(k\) exist and are continuous. 

A \textbf{manifold} \(M\) is a collection of points such that each of them has an open neighborhood which has a continuous one-to-one map with an open set of \(\mathbb{R}^{n}\). 
The number \(n\) is called the \textbf{dimension} of the manifold.

\todo[inline]{He talks about the locality of this definition as being about the fact that \(M\) is a subset of a larger space\dots}

A coordinate system or \emph{chart} is a pair \((M, \rho )\), where \(M\) is an open set while \(\rho \) is a map. 

Suppose we have two coordinate systems \((U, f)\) and \((V, g)\), where \(U\) and \(V\) overlap at least at a point (so, in a region as well). 

Consider the region \(f(U \cap V)\): in order to move from it to \(g(U \cap V)\) we need to apply the map \(g \circ f^{-1}\). 
This is, therefore, a \textbf{change of coordinates}, which we will be able to write as \(\vec{y} = \vec{y} (\vec{x})\). 

A \(\mathcal{C}^{k}\) differentiable \textbf{manifold} is one for which 
\begin{enumerate}
    \item each point of \(M\) belongs at least to an open set (and a corresponding chart); 
    \item each chart is \(\mathcal{C}^{k}\) related to other charts it overlaps with. 
\end{enumerate}

These changes of coordinates can be written as \(y^{i} = F^{i}(x^{j})\), for which we can define a Jacobian determinant: 
%
\begin{align}
J = \abs{\pdv{F^{i}}{x^{j}}} 
\,.
\end{align}

If \(J \neq 0\) at a point \(P\), there is an onto, one-to-one map in some neighborhood of \(P\). 

A \textbf{curve} is a path in \(M\) such that we associate a number, called the parameter, to each point in it, and this provides a map between it and a path in \(\mathbb{R}^{n}\), called the image of the curve. 

This is typically denoted as \(\gamma \colon s \in [a, b] \to [x^{1}(s), \dots, x^{\mu }(s)]\). 

The quantities 
%
\begin{align}
\dv{x^{i}}{s} = \left[\begin{array}{ccc}
\displaystyle \dv{x^{1}}{s} 
& \dots &
\displaystyle \dv{x^{n}}{s}
\end{array}\right]
\,
\end{align}
%
are the \emph{components of the tangent vector}. 

Suppose we want to make a coordinate change, \(x'(x)\): how does our tangent vector change? 
By the chain rule, it will read 
%
\begin{align}
\dv{x^{\prime i}}{s} = \pdv{x^{\prime i}}{x^{j}} \dv{x^{j}}{s}
\,.
\end{align}

This kind of vector is called a \textbf{contravariant vector}, since it transforms with the Jacobian matrix. 

We want to show that directional derivatives form a vector space at a point \(P\). 
Consider a curve with its parameter \(\lambda \) and a differentiable function \(\phi (x_1, \dots, x_m)\). 
The derivative 
%
\begin{align}
\dv{\phi }{\lambda } = \pdv{\phi }{x^i} \dv{x^{i}}{\lambda } 
\,.
\end{align}

With this in mind, we can define the \textbf{directional derivative operator} 
%
\begin{align}
\dv{}{\lambda } = \dv{x^{i}}{\lambda } \pdv{}{x^{i}}
\,.
\end{align}

However, we already know that the \(\dv*{x^{i}}{\lambda }\) are the \emph{components of the tangent vector}.

If we have two curves, \(x^{i}(\lambda )\) and \(x^{i} (s)\), we can define a directional derivative for each of them.
We can then sum these, by defining the sum as 
%
\begin{align}
\dv{}{\lambda } + \dv{}{s} = \qty(\dv{x^{i}}{\lambda } + \dv{x^{i}}{s}) \pdv{}{x^{i}}
\,.
\end{align}

This vector will still be tangent to the curve, and we will be able to find a third parameter \(\mu \) such that 
%
\begin{align}
\dv{}{\mu } = \dv{x^{i}}{\mu } \pdv{}{x^{i}}
\,.
\end{align}

We can also scale the directional derivative operator: 
%
\begin{align}
a \dv{}{\lambda } = \underbrace{\qty(a \dv{x^{i}}{\lambda })}_{\dv*{x^{i}}{\sigma }} \pdv{}{x^{i}}
\,.
\end{align}

This proves (?) that the space of directional derivatives is a vector space. 
This space is denoted as \(T_P\), where \(\dv*{}{\lambda }\) is a vector.

The crucial idea here is that vectors at different points belong to different vector spaces, and cannot be directly compared. 

Coordinate lines are the ones for which all but one of the coordinates we are using remain constant. 
What this means is that directional derivatives along these read 
%
\begin{align}
\dv{}{x^{i}} = \dv{x^{j}}{x^{i}} \pdv{}{x^{j}} = \delta^{j}_{i} \pdv{}{x^{j}} = \pdv{}{x^{i}}
\,.
\end{align}

This also means that any directional derivatives can be expressed (uniquely) as linear combinations of these \(\partial_{i}\), which are therefore a basis for the tangent space at each point.

We have a one-to-one connection between the tangents of curves at a point \(P\) and the derivatives along a curve at \(P\). 
Because of this, we associate the directional derivative \(\dv*{}{\lambda }\) to a tangent vector to a curve \(x^{i}(\lambda )\).

We denote the \(i\)-th basis vector as \(\vec{e}_{(i)} = \pdv{}{x^{i}}\).
The number between round brackets enumerates the vectors in the basis, it is not a spatial index. 

We can then express any vector in a more conventional notation as \(\vec{A} = A^{i}\vec{e}_{(i)}\).
If we change coordinates, the basis will shift like 
%
\begin{align}
\vec{e}_{k'} = \widetilde{\Lambda}^{j}_{k'} \vec{e}_j
\,,
\end{align}
%
where \(\widetilde{\Lambda}\) is the inverse of the Jacobian. 

\paragraph{One-forms}

A one-form \(\widetilde{q}\) is a real-valued, linear function of vectors: 
%
\begin{align}
\widetilde{q}(a \vec{V} + \vec{W}) = a \widetilde{q}(\vec{V}) + \widetilde{q}(\vec{W})
\,.
\end{align}

We can sum one-forms by summing their action, and multiply them by scalars by multiplying the result of their application by a scalar. 
These are then a vector space, which is called the \emph{dual}: \(T_P^{*}\).

The reason for the name is the dual symmetry: taking \(\widetilde{q}(\vec{V}) \simeq \vec{V} (\widetilde{q})\). 

The basis for the dual is called the conjugate basis \(\widetilde{\omega}\), and it is convenient to select it so that \(\widetilde{\omega}^{(i)} (\vec{V}) = V^{i}\). 
Alternatively, we can write this as \(\widetilde{\omega}^{(i)} (\vec{e}_{(j)}) = \delta^{i}_{j}\). 

Any 1-form can also be written in terms of the basis covectors, \(\widetilde{q} = q_{i} \widetilde{\omega}^{(i)}\). 

We can make the manipulation 
%
\begin{align}
\widetilde{q} (\vec{V}) = 
\widetilde{q} (V^{i} \vec{e}_{(i)}) =
V^{i} \widetilde{q} (\vec{e}_{(i)}) =
\widetilde{\omega}^{(i)}(\vec{V}) \widetilde{q} (\vec{e}_{(i)}) = q^{i} V_i
\,.
\end{align}
%


\end{document}
