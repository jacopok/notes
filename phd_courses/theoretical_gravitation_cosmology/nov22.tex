\documentclass[main.tex]{subfiles}
\begin{document}

\subsection{Covariant differentiation}

\marginpar{Monday\\ 2021-11-22}

Consider a vector \(\vec{V} = V^{\alpha } \vec{e}_\alpha \). 

How do we compute \(\pdv*{\vec{V}}{x^{\mu }}\)? It will read 
%
\begin{align}
\pdv{\vec{V}}{x^{\beta }} = \pdv{V^{\alpha }}{x^{\beta }} \vec{e}_{(\alpha )} + V^{\alpha } \pdv{\vec{e}_{(\alpha )}}{x^{\beta }} 
\,.
\end{align}

What we need to argue now is that the second term is a vector just like the first one: 
we need to use the equivalence principle. 
We can move to a reference in which \(g_{\mu \nu } = \eta_{\mu \nu }\),
such that \(\vec{e}_{(\alpha )}\) also become constant. 

The new and old will be related through transformation matrices: 
%
\begin{align}
\vec{e}_{(\alpha ')} = \Lambda^{\mu }_{\alpha'} \vec{e}_{\mu }
\,,
\end{align}
%
so we will get 
%
\begin{align}
\pdv{}{x^{\beta }} \vec{e}_{(\alpha )} = \pdv{}{x^{\beta }} \qty[ \Lambda^{\mu '}_\alpha ] \vec{e}_{(\mu ')} 
\,.
\end{align}

This object can be expressed as a linear combination of the 
basis vectors in the new basis, and is therefore a vector. 
We call it 
%
\begin{align}
\pdv{\vec{e}_{(\alpha )}}{x^{\beta} } = \Gamma^{\rho }_{\alpha \beta } \vec{e}_{(\rho )}
\,.
\end{align}

This is an \emph{affine connection}.
With this expression, we write 
%
\begin{align}
\pdv{\vec{V}}{x^{\beta }} &= \pdv{V^{\alpha }}{x^{\beta }} + V^{\alpha } \Gamma^{\rho }_{\alpha \beta } \vec{e}_{(\rho )} 
\,.
\end{align}

The formalism is as follows: 
%
\begin{align}
V^{\alpha }{}_{, \mu } = \partial_{\mu } V^{\alpha } &= \pdv{V^{\alpha }}{x^{\mu }} \\
V^{\alpha }{}_{; \mu } = \nabla_\mu V^{\alpha } &= \pdv{V^{\alpha }}{x^{\mu }} + V^{\beta } \Gamma^{\alpha }_{\mu \beta }
\,.
\end{align}

The  covariant derivative can also be written as 
%
\begin{align}
\nabla \vec{V} = \qty[V^{\alpha }{}_{; \mu } \vec{e}_{(\alpha )}] \otimes \widetilde{\omega}^{(\mu )}
\,.
\end{align}

Therefore, \(\nabla \vec{V}\) is a \((1, 1)\) tensor. 

In a Local Inertial Frame, covariant and component derivatives are equal, since \(\Gamma^{\mu }_{\nu \rho } = 0\). 
Also, the covariant derivative of a scalar is equal to its 
partial derivative. 

We can also differentiate a one-form: this will be a \((0, 2)\) tensor. Its components will read 
%
\begin{align}
q_{\alpha ; \beta } = \partial_\beta q_\alpha - \Gamma^{\rho }_{\alpha \beta } q_\rho
\,.
\end{align}

When taking the covariant derivative of a tensor with any amount of indices we need to include a Christoffel term for all indices. 

Let us write out a mixed tensor's derivative:
%
\begin{align}
\nabla_\beta T^{\mu }_\nu = T^{\mu }_{\nu , \beta } 
+ \Gamma^{\mu }_{\rho \beta }T^{\rho }_\nu  
- \Gamma^{\rho }_{\nu  \beta }T^{\mu }_\rho   
\,.
\end{align}

In GR, we always have \(g_{\mu \nu ; \rho } = 0\). 
Why is this? If we move to a LIF, it equals \(\eta_{\mu \nu , \rho } = 0\).

The Christoffel symbols are also symmetric: \(\Gamma^{\rho }_{\alpha \beta } = \Gamma^{\rho }_{(\alpha \beta )}\). 
There is an explicit expression for them in terms of the metric: 
%
\begin{align}
\Gamma^{\rho }_{\alpha \beta } = \frac{1}{2} g^{\rho \sigma } \qty(g_{\sigma \alpha , \beta } + g_{\sigma \beta , \alpha } - g_{\alpha \beta , \sigma })
\,.
\end{align}

Are the Christoffel symbols the components of a tensor? No. 
They vanish in the LIF, so if they were a tensor they would always be zero. 
If one tries to compute their transformation law, it comes out to depend on the mixed second derivatives of the coordinate change. 

The fact that in the LIF \(g_{\mu \nu , \alpha } = 0 \) and \(\Gamma^{\rho }_{\alpha \mu } = 0\) is the mathematical representation of the equivalence principle. 

\paragraph{Parallel transport}

It helps us understand the intrinsic geometry of the manifold. 
We will do it for a vector along a path. 
We ask that, for each infinitesimal displacement, the displaced vector remains parallel to itself and does not change magnitude.

This means that we cannot, in general, define a \emph{global} constant vector field, or ``comb a hairy ball''. 

Suppose we have a curve parametrized by \(\lambda \), whose tangent vector is 
%
\begin{align}
u^{\mu } = \dv{x^{\mu }}{\lambda }
\,.
\end{align}

We want to transport a vector \(\vec{V}\) along it. 
How do we do it? We can move to a LIF, with coordinates \(\xi^{\alpha }\). 
If we move \(\vec{V}\) by \(\dd{\lambda }\) along the curve, we will change it by 
%
\begin{align}
\dv{V^{\mu }}{\lambda } = \dv{V^{\mu }}{\xi^{\alpha }} \dv{\xi^{\alpha }}{\lambda } = u^{\alpha }\partial_{\alpha } V^{\mu }
\,.
\end{align}

However, \(\partial = \nabla\) in a LIF, so this non-tensorial expression is equal in this reference frame to the tensorial expression \(u^{\alpha } \nabla_\alpha V^{\mu }\), which we set to zero in order to impose that the vector does not change.

This is often also written like \(\nabla_{\vec{u}} \vec{V} = 0\), or 
%
\begin{align}
\nabla_{\vec{u}} V^{\alpha } = \dv{V^{\alpha }}{\lambda } + V^{\mu } u^{\beta } \Gamma^{\alpha }_{\mu \beta } = 0
\,.
\end{align}

In the presence of a curved spacetime, the components of the vector do change, depending on the curve! 

\textbf{Geodesics} are curves which parallel-transport their own tangent vector: \(u^{\mu } \nabla_{\mu } u^{\nu } = 0\). 
This means that 
%
\begin{align}
u^{\beta } u^{\alpha }_{; \beta } = \dv{x^{\beta }}{\lambda } \qty[ \pdv{x^{\alpha }}{x^{\beta }} + \Gamma^{\alpha }_{\beta \rho } u^{\rho }] = \dv[2]{x^{\alpha }}{\lambda }
+ \Gamma^{\alpha }_{\beta \rho } \dv{x^{\beta }}{\lambda } \dv{x^{\rho }}{\lambda  }
\,.
\end{align}

Any affine transformation \(\lambda \to a \lambda + b\) leaves this unchanged. 

\todo[inline]{Can't we do any monotone differentiable transformation?}

\subsubsection{The curvature tensor}

It is the only four-index tensor which can be built only from derivatives of the metric. 

A way to do it is to try and build an object which transforms like a tensor from the (derivatives of the) Christoffel symbols. 

The Riemann tensor is written as 
%
\begin{align}
R^{\lambda }_{\mu \nu \kappa } = - \qty( \partial_{\alpha } \Gamma^{\lambda }_{\mu \nu } - \partial_{\nu } \Gamma^{\lambda }_{\mu \kappa } + \Gamma^{\lambda }_{\kappa \eta } \Gamma^{\eta }_{\mu \nu } - \Gamma^{\lambda }_{\nu \eta } \Gamma^{\eta }_{\mu \kappa }) 
\,.
\end{align}

It can be schematically represented as \(R \sim \partial \Gamma + \Gamma \Gamma \). 
In the LIF, \(\Gamma \Gamma= 0\) but the first part does not vanish; this allows it to be tensorial. 

In that case, it is also written as 
%
\begin{align}
R^{\alpha }_{\beta \mu \nu } = \frac{1}{2} g^{\alpha \sigma } \qty(g_{\sigma \nu , \beta \mu } - g_{\sigma \mu , \beta \nu } + g_{\beta \mu , \sigma \nu } - g_{\beta \nu , \sigma \mu })
\,.
\end{align}

Let us write out its symmetries in the \((0, 4)\) version: 
%
\begin{align}
R_{\alpha \beta \mu \nu } = - R_{\beta \alpha \mu \nu } = - R_{\alpha \beta \nu \mu } = R_{\mu \nu \alpha \beta }
\,,
\end{align}
%
and also 
%
\begin{align}
R_{\alpha \beta \mu \nu } + R_{ \alpha \nu \beta \mu } + R_{\alpha \mu \nu \beta } = 0
\,.
\end{align}

These reduce the number of independent components: it would have \(4^{4} = 256\), but the symmetries reduce that number to 20. 

If we parallel-transport a vector along a loop of coordinate directions 1 and 2 it changes by 
%
\begin{align}
\delta V^{\alpha } = R^{\alpha }_{\beta 12} V^{\beta }
\,.
\end{align}

So, we can see that the spacetime is globally flat iff \(R^{\alpha }_{\beta \mu \nu } = 0\). 

It can also be defined as 
%
\begin{align}
\qty[\nabla_\mu, \nabla_\nu ] V^{\alpha } = R^{\alpha }_{\beta \mu \nu } V^{\beta }
\,.
\end{align}

We will also need the Bianchi identities: 
%
\begin{align}
R_{\alpha \beta [ \mu \nu ; \lambda ]} = 0
\,.
\end{align}

\end{document}
