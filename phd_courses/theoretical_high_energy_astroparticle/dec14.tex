\documentclass[main.tex]{subfiles}
\begin{document}

\marginpar{Tuesday\\ 2021-12-14}

The equations of motion we found read 
%
\begin{align}
m \gamma \dv{v_x}{t} &= \frac{q}{c} \left( v_y B_0 - v_z \delta B_y\right) \\
m \gamma \dv{v_y}{t} &= \frac{q}{c} \left( v_x B_0 - v_z \delta B_y\right) \\
m \gamma \dv{v_z}{t} &= \frac{q}{c} \left( v_x \delta B_y - v_y \delta B_x \right) 
\,,
\end{align}
%
and we found 
%
\begin{align}
\dv{\mu }{t} \sim \cos \left(\Omega t \mp kz \pm \omega t \mp \varphi \right)
\,.
\end{align}

We then neglected \(\omega t \sim k v_A t\) compared to  \(kz = k v \mu t\). 

We found \(\expval{ \Delta \mu \Delta \mu } = \Omega^2 (\delta B / B_0 )^2 \Delta t (\pi / v \mu ) \delta (k \mp \Omega / v \mu )\). 
The \(\Delta \mu^2 \sim \Delta t\) dependence tells us that it is \emph{diffusive} motion, as opposed to \emph{ballistic} motion, which would have \(\Delta \mu \sim \Delta t\). 

The diffusive motion is then described as 
%
\begin{align}
D_{\mu \mu } = \frac{1}{2} \expval{\frac{\Delta \mu \Delta \mu }{t}} = \frac{\pi}{2} \Omega (1 - \mu^2) k _{\text{res}} F (k _{\text{res}}) 
\,,
\end{align}
%
where \(F (k _{\text{res}}) = \delta B^2( k _{\text{res}}) / B_0^2\). 

Diffusion in pitch angle means diffusion in space! Once the angle changes enough, the particle changes direction completely.
The timescale for diffusion by \(\pi /2\) is 
%
\begin{align}
\tau \sim \frac{1}{\Omega k _{\text{res}} F( k _{\text{res}})}
\,.
\end{align}

We will have a diffusion length scale of the order of \(\lambda _d \approx v \tau \), and the diffusion coefficient in space will be 
%
\begin{align}
D_{zz} = \frac{1}{3} v \lambda _d = \frac{1}{3} v^2 \tau 
\,,
\end{align}
%
so we get 
%
\begin{align}
D &= \frac{1}{3} v^2 \frac{1}{\Omega k _{\text{res}} F(k _{\text{res}})}  \\
&= \frac{1}{2} r_L (p) \frac{v}{k _{\text{res}} F(k _{\text{res}})}
\,.
\end{align}

This is \emph{Bohm diffusion}! 
This theoretical result allows us to compute a timescale for diffusion, which with some reasonable numbers is 
%
\begin{align}
\tau = \frac{H^2}{D} \approx \num{25e20} \times \text{resonant power} \overset{!}{=} \text{million years}
\,,
\end{align}
%
so adding a very small \(k _{\text{res}} F( k _{\text{res}}) \sim \num{e-6}\) perturbation allows for strong diffusive motion. 

\todo[inline]{Do calculation with units}

It's easy to understand how a particle gets to \(\mu = 0\), but crossing that threshold is complicated. 
Slightly nonlinear theories of this problem allow for the crossing of this threshold more easily.

If we include the electric field, the energy can change a bit! 
Proceeding in this way allows us to compute a corrected diffusion coefficient for the momentum: 
%
\begin{align}
\expval{\frac{\Delta p \Delta p}{\Delta t}} &\approx (m v_A \gamma )^2 \expval{\frac{\Delta \mu \Delta \mu }{\Delta t}}  \\
&\approx (mc \gamma )^2 \Omega k _{\text{res}} F( k _{\text{res}}  )  \\
&\approx (mc \gamma )^2 \frac{v_A^2}{c^2} \frac{1}{\tau } = p^2 \left( \frac{v_A}{c}\right)^2 \frac{1}{\tau }
\,,
\end{align}
%
therefore 
%
\begin{align}
\tau _{\text{acceleration}} \approx \left(\frac{c}{v_A}\right)^2 \tau 
\,,
\end{align}
%
it takes \(\sim \num{e10}\) times longer to accelerate/decelerate a particle than to make it diffuse over time. 

There can be situations in which \(v_A\) is not so small compared to \(c\)! 
For example, in a GRB the plasma is relativistic, and even diffusive processes will be fast. 

The effect of diffusion is to isotropize the distribution of the particles! 
The net speed of these relativistic particles will roughly become \(v_A\)! 

We will now try to understand this quantitatively. 
Once we get the result of this equation, the rest of the course will be applications. 

There are two ways to get this result: we will use the Fokker-Planck approach, which is less heavy in terms of mathematics.
The alternative would be to look at the zeroth-order term in the Vlasov equation which would describe collisions, a big average on the right. 
We do have a way to express it: we write the \(\delta f\) for cosmic rays, the \(\delta B\) for Alfvén waves, and compute away.

Here is the alternative approach:
we need to introduce a distribution function in phase space for the cosmic rays. 

We will need to introduce the probability that a certain cosmic ray momentum \(\vec{p}\) changes to \(\vec{p} + \Delta \vec{p} \): \(\Psi (\vec{p}, \Delta \vec{p})\), which will satisfy 
%
\begin{align}
\int \Psi (\vec{p}, \Delta \vec{p}) \dd{ \Delta \vec{p}} = 1
\,.
\end{align}

We will then have the following:
%
\begin{align}
f(\vec{x} + \vec{v} \Delta t, \vec{p}, t + \Delta \vec{t}) 
= \int \dd{ \Delta \vec{p}} f(\vec{x}, \vec{p} - \Delta \vec{p} , t) \Psi (\vec{p} - \Delta \vec{p}, \Delta \vec{p})
\,,
\end{align}
%
which we can expand:
%
\begin{align}
f(\vec{x} + \vec{v} \Delta t, \vec{p}, t + \Delta t) []&=
f + \vec{v} \cdot \pdv{}{\vec{x}} f \Delta t + \pdv{}{t} f \Delta t  \\
f(\vec{x}, \vec{p} - \Delta \vec{p}, t) &=  
f - \pdv{f}{\vec{p}} \Delta \vec{p} + \frac{1}{2} \pdv{}{p} \pdv{}{p} f \Delta \vec{p} \Delta \vec{p}  \\
\Psi (\vec{p} - \Delta \vec{p}, \Delta \vec{p}) &= \Psi  - \Delta \vec{p} \pdv{\Psi }{\vec{p}} + \frac{1}{2} \pdv{}{\vec{p}} \pdv{}{\vec{p}} \Psi \Delta \vec{p} \Delta \vec{p}
\,.
\end{align}

Inserting these terms in the integral equation yields 
%
\begin{align}
f + \pdv{f}{t} \Delta t + \vec{v} \cdot \pdv{f}{\vec{x}}  \Delta t
&= \int \dd{ \Delta \vec{p}} 
\left(
    f - 
    \pdv{f}{\vec{p}} \Delta \vec{p} 
    + \frac{1}{2} \pdv[2]{f}{p}
    \Delta p^2
\right)
\left(
    \Psi - \pdv{\Psi }{p} \Delta p
    + \frac{1}{2} \Delta p^2 \pdv[2]{\Psi }{p}
\right)  \\
&= \int \dd{ \Delta p}
\left(
    f \Psi - f \pdv{\Psi }{p} \Delta p
    + \frac{1}{2} f \pdv[2]{\Psi }{p}
    - \Psi \pdv{f}{p} \Delta p 
    + \pdv{f}{p} \pdv{\Psi }{p} \Delta p^2
    + \frac{1}{2} \pdv[2]{f}{p} \Delta p^2
\right)  \\
&= f - \int \dd{ \Delta \vec{p}} \Delta \vec{p}  \pdv{}{p} (f \Psi )
+ \frac{1}{2} \pdv{}{p} \pdv{}{p} \left(
    \dd{ \Delta p} \Delta \vec{p} \Delta \vec{p} f \Psi 
\right)
\,,
\end{align}
%
where we need the result 
%
\begin{align}
\pdv[2]{(f \Psi )}{p} = \pdv[2]{f}{p } \Psi + \pdv[2]{\Psi }{p} f + 2 \pdv{f}{p} \pdv{\Psi }{p}
\,.
\end{align}

We also need to define 
%
\begin{align}
\expval{ \frac{\Delta \vec{p}}{\Delta t}} &= \frac{1}{\Delta t} \int \Psi (\Delta \vec{p}) \Delta \vec{p} \dd{ \Delta \vec{p}} \\
\expval{ \frac{\Delta \vec{p} \Delta \vec{p}}{\Delta t}} &= \frac{1}{\Delta t} \int \Psi (\Delta \vec{p}) \Delta \vec{p} \Delta p \dd{ \Delta \vec{p}} 
\,.
\end{align}

Then, we get 
%
\begin{align}
\pdv{f}{t} + \vec{v} \cdot \pdv{f}{\vec{x}} 
= - \pdv{}{\vec{p}} \left(
    f \expval{\frac{\Delta p}{\Delta t}}
\right)
+ \frac{1}{2} \pdv{}{p} \pdv{}{p} \left(
    f \expval{\frac{\Delta p \Delta p}{\Delta t}}
\right)
\,.
\end{align}

We also need to introduce an assumption: a detailed balance principle,
%
\begin{align}
\Psi (\vec{p}, - \Delta \vec{p}) &= \Psi (\vec{p} - \Delta \vec{p} , \Delta p)  \\
\Psi (\vec{p}, - \Delta \vec{p}) &= 
\Psi (\vec{p}, \Delta \vec{p}) - \Delta \vec{p} \pdv{\Psi }{\vec{p}} + \frac{1}{2} \pdv[2]{\Psi }{p} \Delta p \Delta p  \\
\pdv{}{p} \expval{\frac{\Delta \vec{p}}{\Delta t}} &=
\frac{1}{2} \pdv{}{p} \pdv{}{p} 
\expval{\frac{\Delta \vec{p} \Delta \vec{p}}{\Delta t}}
\,,
\end{align}
%
meaning that the quantity 
%
\begin{align}
\expval{\frac{\Delta \vec{p}}{\Delta t}} -
\frac{1}{2} \pdv{}{p} 
\expval{\frac{\Delta \vec{p} \Delta \vec{p}}{\Delta t}}
\,
\end{align}
%
is a constant, which we set to zero. 
We will verify \emph{a posteriori} that this is correct, but it kind of makes sense. So, 
%
\begin{align}
\underbrace{\expval{\frac{\Delta \vec{p}}{\Delta t}}}_{R} =
\underbrace{\frac{1}{2} \pdv{}{p} 
\expval{\frac{\Delta \vec{p} \Delta \vec{p}}{\Delta t}}}_{\pdv*{}{p} (D_{pp})}
\,.
\end{align}

The equation then becomes 
%
\begin{align}
\pdv{f}{t} + \vec{v} \cdot \pdv{f}{\vec{x}} =
 - \pdv{f}{p} R 
 - f \pdv{R}{p}
 + \pdv{}{p} \pdv{}{p} (f
 D_{pp} ) = \pdv{}{p} \left( D_{pp} \pdv{f}{p} \right)
\,.
\end{align}

In one dimension, and neglecting interactions, this simply becomes 
%
\begin{align}
\pdv{f}{t} + v \mu \pdv{f}{z} = \pdv{}{\mu } \left(D_{\mu \mu } \pdv{f}{\mu } \right) 
\,,
\end{align}
%
à la Fokker-Planck. 
The beautiful thing is that this is valid without assuming slow velocities or isotropy! 

This exactly describes the diffusion of the beam we were considering. 
It's a mess to solve this. 

Let us try to integrate this between \(-1\) and \(1\): we apply the operator \(I = (1/2) \int_{-1}^{1} \dd{\mu } \). 
We define the distribution function \(M(z, t) = I(f)\). 

The equation then becomes 
%
\begin{align}
\pdv{M}{t} + \pdv{}{z} I( v \mu f) = 0
\,,
\end{align}
%
since \(D_{\mu \mu }\) vanishes at \(\pm 1\). 

The term \(J = I(v \mu f)\) looks like a current: we then have the conservation equation
%
\begin{align}
\pdv{M}{t} + \pdv{J}{z} = 0
\,.
\end{align}

We can rewrite \(J\) using \(\mu = -(1/2) \partial_{\mu } (1 - \mu^2)\): 
%
\begin{align}
J &= - \frac{1}{2} v \int \dd{\mu } \pdv{}{\mu } (1 - \mu^2) f  \\
&= + \frac{v}{4} \int_{-1}^{1} \dd{\mu } (1 - \mu^2) \pdv{f}{\mu }
\,.
\end{align}

If we integrate up to an arbitrary \(\mu \) instead, we get 
%
\begin{align}
\pdv{}{t} \int_{-1}^{\mu } \dd{\mu '} f + v \int_{-1}^{\mu } \dd{\mu '} \mu ' \pdv{f}{z} &=  D_{\mu \mu} \pdv{f}{\mu }  \\
(1 - \mu^2)  \pdv{f}{\mu } &= \frac{(1 - \mu^2)}{D_{\mu \mu }}
\pdv{}{t} \int_{-1}^{\mu } \dd{\mu '} f 
+ \frac{v (1 - \mu^2)}{D_{\mu \mu }} \int_{-1}^{\mu } \dd{\mu '} \mu ' \pdv{f}{z}
\,,
\end{align}
%
but we are close to isotropy, so 
%
\begin{align}
(1 - \mu^2) \pdv{f}{\mu } = \frac{(1 - \mu^2)}{D_{\mu \mu }} (\mu +1) \pdv{M}{t} + \frac{v (1- \mu^2)}{D_{\mu \mu }} \frac{1}{2} (M^2 - 1) \pdv{M}{z} 
\,,
\end{align}
%
so we can multiply everything by \(v/4\) and integrate in \(\mu \) between \(-1\) and \(1\): 
%
\begin{align}
J = \int_{-1}^{1} \dd{\mu } \left(
\frac{(1 - \mu^2)}{D_{ \mu \mu }} (1 + \mu )
\pdv{M}{t} 
- \frac{v^2 (1 - \mu^2)^2}{8 D_{\mu \mu }} \pdv{M}{z}
\right)
\,,
\end{align}
%
so we can define 
%
\begin{align}
k_{tt} &= \frac{v}{4} \int_{-1}^{1} \dd{\mu } \frac{(1 - \mu^2) ( 1 + \mu )}{D_{\mu \mu } }\\ 
k_{zz} &= \frac{v^2}{8} \int_{-1}^{1} \dd{\mu } \frac{(1 - \mu^2)^2}{D_{\mu \mu }}
\,,
\end{align}
%
therefore 
%
\begin{align}
J = k_{tt } \pdv{M}{t} - k_{zz} \pdv{M}{z}
\,.
\end{align}

Together with the continuity equation we find 
%
\begin{align}
\pdv{M}{t} = - \pdv{}{z} \left( k_{tt} \pdv{M}{t} - k_{zz} \pdv{M}{z}\right)
\,,
\end{align}
%
but \(f\) will be close to isotropy --- the simplest thing is a dipole, \(f_0 (1 + \delta \mu )\), where \(\delta \) is a number \(\ll 1\): 
%
\begin{align}
J = \frac{1}{2} \int_{-1}^{1} \dd{\mu } f_0 v \mu (\delta \mu ) = \frac{f_0 v \delta }{3}
\,.
\end{align}

If \(\delta \) is small, then the \(k_{tt}\) part will be small: therefore, we find 
%
\begin{align}
\pdv{M}{t} = \pdv{}{z} \left( k_{zz} \pdv{M}{z} \right)
\,.
\end{align}

\todo[inline]{Clarify this. The first term is \(\sim v M\), the second is \(\sim v^2 M / z\). }

\end{document}