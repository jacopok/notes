\documentclass[main.tex]{subfiles}
\begin{document}

\marginpar{Wednesday\\ 2021-12-22}

Last time we derived the transport equation in pitch angle: 
%
\begin{align}
\pdv{f}{t} + v \mu \pdv{f}{z} = \pdv{}{\mu } \left(D_{\mu \mu } \pdv{f}{\mu }\right)
\,,
\end{align}
%
where \(D_{\mu \mu} = (1/2) \expval{\Delta \mu \Delta \mu / \Delta t}\). 

This already contains a lot of information about what scattering does to cosmic rays. 

If we have the classic system with a ``gun'' of particle emitting them 
into a plasma with Alfvén waves, those will be isotropized. 

Because of this, after some time we expect to see the bulk velocity of the Alfvén waves and the cosmic rays to be the same. 

This is not realistic in every situation: if the plasma is relativistic, the assumption of the particle velocity being much higher than that of the waves breaks down. 
This can be treated, but we will not discuss it in the course; the professor can provide the necessary references upon request. 

After integrating the aforementioned equation between \(-1\) and \(1\), as well as between \(-1\) and \(\mu \), we found 
%
\begin{align}
\pdv{M}{t} &= \pdv{}{z} \left( k_{zz} \pdv{M}{z} \right)  \\
k_{zz} &= \frac{v^2}{8} \int \dd{\mu } \frac{(1 - \mu )^2}{D_{\mu \mu }} 
\,,
\end{align}
%
where the first equation tells us that \(M\) changes in time due to a process of \emph{spatial} diffusion, which is caused (as the second equation tells us) by diffusion in pitch angle.

The second-order nature of this equation is a trademark of a diffusive process.
With a heuristic line of reasoning we got a typical time for diffusion by \SI{90}{\degree}, called 
%
\begin{align}
\tau_{90} = \frac{1}{D_{\mu \mu }} 
\,,
\end{align}
%
which allows us to define \(\lambda = v \tau_{90}\), 
while the correct expression is the one for \(k_{zz }\) above. 

One thing we are still missing is the effect of the plasma! 
That is currently only described through the fact that the particles are diffusing in pitch angle due to Alfvén waves. 

Can we not ignore the bulk velocity if we move to the correct frame?
Yes! But, we cannot do the same if there are velocity \emph{gradients}. 

Suppose we have a nonrelativistic plasma velocity \(u(z)\). 
The change in the particle velocity term \(v \mu \) can be approximated as \(v \mu \to u + v \mu \).

\todo[inline]{Risky! \(u + v \mu \) could be larger than \(c\)! Wouldn't a better approximation be \(v \mu \)? }
 
What about the momentum? We transform it as 
%
\begin{align}
\left[\begin{array}{c}
E' \\ 
p_z'c
\end{array}\right]
=
\left[\begin{array}{cc}
\gamma  & -\beta \gamma  \\ 
- \beta \gamma  & \gamma 
\end{array}\right]
\left[\begin{array}{c}
E \\ 
p_zc
\end{array}\right]
\,.
\end{align}

So, the \(z\)-axis momentum \(p_z\) changes to \(p_z'c = - \beta \gamma E + \gamma p_z c\), where \(\beta = u(z) / c\). 

The derivative of the phase space distribution changes to 
%
\begin{align}
\pdv{f}{z} \to \pdv{f'}{z'} &= \pdv{f'}{p'_z} \pdv{p'_z}{z'}  \\
&= \pdv{f'}{p_z'} \left( - \dv{u(z)}{c} \right) \frac{E'}{c}   \\
v \mu \pdv{f}{z} &\to - (u + v \mu ) \left( \pdv{f'}{p'} \frac{E'}{c} \dv{u(z)}{c} \right)  \\
&= (u + v \mu ) \pdv{f'}{z'} - (u + v \mu ) \pdv{f'}{p'} \frac{E'}{c^2} \dv{u}{z'}  \\
&\approx (u + v \mu ) \pdv{f'}{z'} - v \mu \pdv{f'}{p'} \frac{E'}{c^2} \dv{u}{z'}
\marginnote{ \(u \ll v \mu \)}
\,.
\end{align}

We neglect the derivative of \(\gamma \) since \(\gamma = 1 + \order{\beta^2}\), and we are working to first order around 0 in the \(\beta \) corresponding to the plasma. 

The change of coordinates formulas read 
%
\begin{align}
\dd{\mu } &= \frac{1 - \mu^2}{p} \dd{p_z} - \mu (1 - \mu^2)^{1/2} \dd{p_\perp }  \\
\dd{p} &= \mu \dd{p_z} + (1 - \mu^2)^{1/2} \dd{p_\perp}
\,,
\end{align}
%
which means we get 
%
\begin{align}
\pdv{f}{p_z} = \pdv{f}{p} \mu + \pdv{f}{\mu } \frac{1 - \mu^2}{p}
\,,
\end{align}
%
therefore
%
\begin{align}
v \mu \pdv{f}{z} &\approx (u + v \mu ) \pdv{f}{z}- v \mu \frac{E'}{c^2} \dv{u}{z} \left( \pdv{f}{\mu } p + \pdv{f}{p} \frac{1 - \mu^2}{p} \right)
\,.
\end{align}

Now we can apply the exact same approach as before!
We just added the term \(u \pdv*{f}{z}\). 

The term \(\pdv*{f}{\mu }\) can be neglected, because \(f\) is assumed to be close to isotropy. If we do not neglect it, we only get higher order corrections. 

This yields 
%
\begin{align}
\frac{v E'}{c^2} \pdv{\mu }{p } \frac{1}{2} \int_{-1}^{1} \dd{\mu } \mu^2 = \frac{1}{3} \frac{ v E'}{c^2} \pdv{\mu}{p} 
\,.
\end{align}

The object in front can be written in terms of the Lorentz factor of the particle \(\Gamma \): 
%
\begin{align}
\frac{v E'}{c^2} = \frac{v m_p c^2 \Gamma }{c^2} = p
\,,
\end{align}
%
so the final result is 
%
\begin{align}
\pdv{M}{t} + u \pdv{M}{z} - \frac{1}{3} \dv{u}{z} p \pdv{M }{p }
= \pdv{}{z} \left( k_{zz} \pdv{M}{z} \right)
\,.
\end{align}

This is the transport equation for any non-thermal particles. 
The assumption of those being non-thermal is implicit in the assumption that \(v \gg u\). 
This is conceptually important, since we need to answer the mechanism by which these particles emerge out of the thermal distribution. 

The \emph{injection problem} is about how we go from a thermal distribution to a thermal distribution with non-thermal particles. 

The effect of the particles on the plasma is also something which we should consider. 
This would be a dependence of \(u\) and \(k_{zz}\) on the distribution of non-thermal particles. 

The term \(p \pdv*{M}{p}\) is sensitive to the \emph{spectrum} of the particles! 
This can change the distribution of particles in \(p\). 

\begin{extracontent}
    Exercise:  suppose we have a system with particles injected and removed from surfaces orthogonal to \(k\). 
    Further, suppose that \(k_{zz}\) is very small: this means that there is a lot of diffusion, since \(D_{\mu \mu }\) is very large.
    
    This basically means that the particles are ``glued'' to the plasma. 
    Therefore, we also get \(\pdv*{M}{t} = 0\). 
    
    What happens to the distribution function in this case? 
    Suppose there is a linear gradient in the plasma velocity, 
    %
    \begin{align}
    u = u_0 + \frac{\Delta u}{L} z
    \,.
    \end{align}    
    
    Hint: using the method of characteristics. 
    
    The expected result is to get adiabatic compression. 
    If the velocity decreases we get energy increasing, and vice versa. This is the same as cosmological redshift.   
\end{extracontent}

The way this works is that since there is a velocity gradient, in some reference frame we need to transform the magnetic field, therefore we also get an electric field which does the work. 

\subsection{Particle transport}

In nature there are basically two acceleration mechanisms:
\begin{enumerate}
    \item when \(\expval{\vec{E}} \neq 0\) --- \emph{regular processes};
    \item when \(\expval{\vec{E}} = 0\) but \(\expval{E^2} = 0\) --- \emph{stochastic processes}. 
\end{enumerate}

Unipolar inductors are when we have a magnet spinning very fast generating an electric field. 
This might happen, for example, near a pulsar! 
In the ideal MHD approximation \(E\) is orthogonal to \(B\), but near a pulsar this can be broken! 
Then, we get a parallel component.
Another way this can happen is the accretion of a black hole. 

Near a neutron star the magnetic field is roughly dipolar; 
if \(E \parallel B\) we get an electric field which is about 8 orders of magnitude larger than the gravitational field. 
So, particles feeling this field are stripped off. 

When this happens, the particles still feel the very strong magnetic field. 
So, electrons undergoing this process with a tilted dipolar magnetic field emit curvature radiation (bremsstrahlung). 

We need to give a quantum description of such a magnetic field; 
the virtual photons of this field can do pair production. 

This process can repeat for \(\num{e4}\) to \num{e6} times: electrons stripped from the star are multiplied, and we get a lot of positrons as well. 
These particles form a plasma of their own!

This will not be really discussed in this course, but there will be a short course about it. 

Ideal MHD is locally violated, giving rise to regular processes, in plasma reconnection as well. 

Magnetic flux is not conserved anymore, and we get heating of particles. 

The rest of this course will mainly be about \textbf{stochastic} processes. 
We will discuss \emph{second order Fermi acceleration}, the first investigation of particle acceleration historically. 

Fermi, after getting the Nobel in '38, moved to Chicago and then worked on the Manhattan Project. 

After the war, he started to work on many new problems. 
The professor has met the last student of Fermi, Simpson. 

At uni Chicago there is a quadrangle, which in 1948 was used to hold important dinners. 
Enrico Fermi was not there a few minutes from midnight on New Year's, 
at which point he came running and sweaty, having understood how cosmic rays were accelerated. 
The idea came to Fermi after Alfvén gave a talk about waves. 

The model by Fermi does not really work, but it was important historically. 

The momentum of the particles in an Alfvén wave will change by \(\Delta p / p \sim v_A / c\), but this will happen stochastically with 
%
\begin{align}
D_{pp} = \expval{\frac{\Delta p \Delta p}{\Delta t}} = \frac{p^2}{T^2} \left( \frac{v_A}{c}\right)^2
\,,
\end{align}
%
so the acceleration timescale is \(\tau _{\text{acc}} =p^2 / D_{pp} = (c / v_A)^2 T\). 
This is too long to actually be the origin of cosmic rays, but in 1948  this was not clear. 

This is called second-order because of two reasons: the square in \((c / v_A)\), and the fact that the diffusive motion can both increase and decrease the momentum. 

Suppose we have regions of plasma moving in different directions. 
Also suppose we have an electric field \(E\) in the reference frame of the laboratory. 
A particle will sometimes ``interact'' with one of these magnetized regions, in a purely elastic way. 

We introduce \(\beta \) and \(\gamma \) for the plasma clouds, with \(\beta \ll 1\). 

After the scattering with the cloud, the energy of the particle in the reference frame of the cloud will become 
%
\begin{align}
E' = \gamma E + \beta \gamma p \mu 
\,.
\end{align}

Suppose that the scattering is purely elastic: \(\mu \) is simply reversed. The cloud is acting like a mirror. 
This is not really necessary, it is just a simplifying assumption. 

The final energy in the lab frame will be 
%
\begin{align}
E'' &= \gamma E' + \beta \gamma p'_z \mu  
\marginnote{The momentum is reversed, hence the plus sign.}  \\
&= \gamma^2 E \left( 1 + p^2 + 2 \beta \mu  \frac{p}{E} \right)
\,,
\end{align}
%
where \(p / E\) is the velocity of the particle. 

The quantity we are interested in is 
%
\begin{align}
\frac{E'' - E}{E} &= \gamma^2 \left(  1 + 2 p v \mu + \beta^2 \right) - 1  \\
&\approx 2 \beta^2 + 2 \beta^2 + 2 \beta v \mu 
\marginnote{In the relativistic approximation, \(\beta \ll 1\) but \(v \sim 1\).}
\,.
\end{align}

What is the mean value of this? 
It can attain positive and negative values, depending on the scattering angle \(\mu \). 
This is the reason this process does not work! 

If there is a situation for which \(\mu \) is bound to only allow acceleration, that can work. 

We need to average this, weighed by the probability to have an interaction with a given \(\mu \). 
The probability will be proportional to the relative velocity, 
\todo[inline]{Why? If the particle is chasing the cloud it will catch it!}
%
\begin{align}
\mathbb{P }(\mu ) = A v _{\text{rel}} = A \frac{\beta \mu + v}{1 + v \beta \mu } \approx A (1  + \beta \mu )
\,,
\end{align}
%
and \(A\) must be \(1/2\) for normalization. 
We can then compute 
%
\begin{align}
\expval{\frac{\Delta E}{E}} = \int_{-1}^{-1} \dd{\mu } \frac{1}{2} \left( 2 \beta^2 + 2 \beta^2 + 2 \beta v \mu  \right) \left(1 + \beta \mu \right) = \frac{8}{3} \beta^2
\,,
\end{align}
%
so we have a mechanism for acceleration, but it is \(\propto \beta^2\). 
The typical velocity, as we know now, is of a few tens of \SI{}{km/s}, which means we have something of the order \num{e-9}.

If the plasma is relativistic, even second order processes can be important, and actually for long gamma ray bursts they are believed to be the main mechanism. 

The problem of injection, as mentioned by Fermi, is about the fact that we have nuclei as well as protons in cosmic rays. 
How do we deal with ionization losses? 
There is no solution to this problem, the mechanism does not really work. 

This was understood about 30 years later.
The thing we need to discuss are explosions and shock waves.

\end{document}
