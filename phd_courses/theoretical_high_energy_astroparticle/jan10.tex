\documentclass[main.tex]{subfiles}
\begin{document}

\marginpar{Monday\\ 2022-1-10}

We have seen how Enrico Fermi thought that there could be 
second-order acceleration in a plasma. 

The problems yet to be resolved are 
\begin{enumerate}
    \item injection: how does one plasma particle become nonthermal? 
    \item the timescale for energy gain in second-order processes seems close to the timescale for energy loss through ionization, which can hamper the process as a whole.
\end{enumerate}

Fermi recognized the ``problem of injection'' in the abstract of his first paper, meaning the second of these.

These problems are only now being solved with particle-in-a-cell simulations.

The people who studied these problems in the early days were the ones involved with the Manhattan Project: they were interested in the physics of bombs. 

When such an explosion occurs, shock fronts form! 
These are ``event horizons'', preventing the transmission of information in one direction. 

At the beginning of the course, from the Vlasov equation we derived mass and momentum conservation, while neglecting energy conservation since we were thinking of adiabatic processes. 

The continuity equation reads 
%
\begin{align}
\pdv{\rho }{t} + \nabla \cdot \left( \rho \vec{u} \right) = 0
\,,
\end{align}
%
which in a stationary situation translates to \(\nabla \cdot \left( \rho \vec{u} \right) = 0\). 
In one spatial dimension, this simply means \(\rho u = \text{const}\). 

The momentum conservation equation reads 
%
\begin{align}
\rho \pdv{\vec{u}}{t} + \rho \vec{u} \cdot \vec{\nabla} \vec{u} = - \vec{\nabla} P
\,,
\end{align}
%
which in 1D and without time-dependence reads 
%
\begin{align}
\rho u \pdv{u}{x} = - \pdv{P}{x}
\,,
\end{align}
%
but because \(\rho u\) is constant we get 
%
\begin{align}
\pdv{}{x} \left( \rho u^2 + P \right) = 0
\,,
\end{align}
%
the conservation of momentum, or Bernoulli's law. 

The conservation of entropy for our isentropic system reads 
%
\begin{align}
u \pdv{}{x} \left( P \rho^{-\gamma } \right) &= 0  \\
\pdv{P}{x} &= \gamma \frac{P}{\rho } \pdv{\rho }{x} 
\,,
\end{align}
%
but the derivative of \(P\) can be expressed in terms of \(\rho u^2\): 
%
\begin{align}
\pdv{}{x} \left( \rho u^2\right) = - \gamma \frac{P}{\rho } \pdv{\rho }{x} 
\,,
\end{align}
%
which we can manipulate and integrate by parts: 
%
\begin{align}
u \pdv{}{x} \left( \rho u^2\right) = + \gamma \frac{P}{\rho } \rho  \pdv{u}{x} = \gamma P \pdv{u}{x}
\,.
\end{align}

Manipulating this further, we find 
%
\begin{align}
u \pdv{}{x} \left( \rho u^2\right) &= \gamma \pdv{}{x} (P u) - \gamma u \pdv{u^2}{x}  \\
u \pdv{}{x} (\rho u^2) (1 - \gamma ) &= \gamma \pdv{}{x} (P u )  \\
\pdv{}{x} \left(
    \frac{1}{2} \rho u^3 + \frac{\gamma }{\gamma -1} P u
\right)
&= 0
\,.
\end{align}

This is analogous to energy conservation (``the flux of energy is constant'').

Through these equations we can connect two points in a solution; a trivial solution will be \(u_1 = u_2 \) and so on, but there may be another nontrivial solution. 

We can define the sound speed as \(c_s^2 = \gamma P / \rho \), the Mach number is \(M = u / c_s\);
it can be proven that if \(M < 1\) only the trivial solution exists, while if \(M > 1\) at least on one side of the boundary we are looking at there is a solution 
%
\begin{align}
\frac{\rho_2}{\rho_1 } = \frac{u_1}{u_2 } &= \frac{(\gamma + 1) M_1^2}{(\gamma - 1) M_1^2 + 2}  \\
\frac{P_2}{P_1 } &= \frac{2 \gamma M_1^2}{\gamma + 1} - \frac{\gamma - 1}{\gamma + 1}
\,.
\end{align}

This is a \emph{shock front} solution of the plasma equations.
If the shock is \emph{strong}, meaning that \(M_1 \gg 1\), the compression ratio \(\rho_2 / \rho_1 \) approaches \((\gamma + 1) / (\gamma -1 )\), which for the \(\gamma = 5/3\) of a monoatomic gas means we have a compression factor \(r = \rho_2 / \rho_1  = 4\). 

What happens to the pressure? That tends to infinity! 
The plasma is heated in such a shock front. 

What is happening is we are converting bulk motion \(\rho u^2\) into compression and heat \(P\), in the Bernoulli equation!

The plasma is slowed down and heated. Since the pressure is \(P \propto N k_B T\), the ratio of the pressures is also a ratio of temperatures! 

The new temperature will be 
%
\begin{align}
T_2 = \frac{2 \gamma (\gamma - 1)}{(\gamma + 1)^2} \frac{u_1^2}{\gamma} \frac{\rho_1}{P_1} T_1 
\,,
\end{align}
%
but \(P_1 \propto T_1 / m_p\), so we get 
%
\begin{align}
T_2 \sim \underbrace{\frac{2 (\gamma - 1)}{(\gamma + 1)^2}}_{\text{order-1 normalization}} m_p u_1^2
\,,
\end{align}
%
so the kinetic energy is being converted to thermal energy! 
The shock isotropizes the motion of the particles. 

This is quite different in the atmosphere and in astrophysics. 
In the atmosphere the collision path length is quite short, while in astrophysics the cross-sections are way too small; still, we can observe these shocks, and they are quite thin! 

Mainly, these are \emph{collisionless} shocks. 
Instead of particle-particle interaction, we have interactions mediated by local electromagnetic fields. 

The averaged approach we are using allows us to neglect the microphysics, but the price is that we don't know what the microphysics is! 

Let us look at a shock at \(x = 0\), in the reference frame in which this shock is stationary. 

The \(x< 0\) region is called \emph{upstream}: a fast and cold plasma, whose Mach number is large (while the Mach number below the shock is forced to be \(< 1\)). 

The \(x > 0\) region is called \emph{downstream}: slow and warm plasma, with velocity \(u_2\).

The relative velocity between the two sides is \(u_1 - u_2 \). 

If you are downstream, the shock is moving towards you; but so if you are upstream! 

The idea is that photons coming from the other side will \emph{always} be blueshifted! 
Maybe we got rid of the situations in which a particle could lose energy?

We are neglecting the magnetic field in that \(B^2 / 8 \pi \) is small compared to \(P\) or to \(\rho u^2\), however magnetic fields can be there! 
It is customary to define an Alfvén Mach number \(M_{A, 1} = u_1 / v_{A, 1}\). Typically, the sonic Mach number is of order 100, the Alfvén Mach number is of the order of 1000. 

Notice that \(\rho v_A^2 = B^2 / 4 \pi \): the Alfvén velocity is determined by the energy density of the magnetic field, while \(\rho c_s^2 = \gamma P \). 
The assumption that the magnetic field energy is small is not very good, really.

Particles diffuse slowing to \(\sim v_A\) in the upstream reference, but that reference is approaching the shock with a speed larger than \(v_A\), so particles will always come back to the shock eventually. 

This holds true even for particles initially moving away from it! 

Suppose we have a particle coming towards the shock with a pitch angle \( 0 \leq \mu < 1\); its energy in the downstream frame will be \(E_d = \gamma (1 + \beta \mu )\), where \(\beta = (u_1 - u_2 ) / c\) and \(\gamma= 1 / \sqrt{1 -\beta^2}\). 

The particle, in the downstream, has a chance of diffusing away to infinity, but also a chance to come back to the shock. 

The energy the upstream sees is \(E_u = \gamma^2 E (1 + \beta \mu ) (1 - \beta \mu ')\), where \(-1 \leq \mu ' < 0\). 

Therefore, \(E_u > E\) necessarily. 
In each crossing, \emph{every single particle} which goes upstream-downstream-upstream gains energy! 

But, what is the probability that a particle will have a certain pitch angle \(\mu \)? 
The flux will read 
%
\begin{align}
\Phi = \int_{\mu = 0}^{\mu = 1} \dd{\Omega } \frac{N}{4 \pi } v \mu = \frac{Nv}{4} 
\,,
\end{align}
%
therefore 
%
\begin{align}
P(\mu ) = \frac{4 A N v \mu}{Nv} = 2 \mu 
\,,
\end{align}
%
and \(P(\mu ')\) will be exactly the same. 

Therefore, we can compute 
%
\begin{align}
\expval{\frac{E_u - E}{E}} = \int_0^{1} \dd{\mu } \int_{-1}^0 \dd{\mu '} 2 \mu 2 \mu ' 
\left(
    \gamma^2 \left(1 + \beta \mu \right)
    \left(1 - \beta \mu '\right)
    -1 
\right)
= \frac{4}{3} \beta = \frac{4}{3} \frac{u_1 - u_2 }{c}
\,.
\end{align}

This is very nice! We have something to first order in the shock velocity, which is large, larger than the Alfvén speed by a couple orders of magnitude. 

This mechanism is called DSA, Diffusive Shock Acceleration, or ``first order Fermi'' even though Fermi didn't develop this. 

We assumed there is diffusion, but we don't care at all about the mechanism! 
The problem with diffusion is that it must be fast, fast enough that a particle can do several passes through. 

The probability of returning from upstream is 1 or close to it, the thing that can happen for it to leave is that its Larmor radius gets so large it grows out of the system. 

The incoming flux to the downstream will be 
%
\begin{align}
\varphi _{\text{in}} = \int_{- u_2 / c}^{1} \dd{\mu } f_0 (u_2 + c \mu )
= f_0 \frac{1}{2} \left(1 + \frac{u_2}{c}\right)^2
\,,
\end{align}
%
where now we must be careful to only consider particles which are actually moving downstream. 

The particles which are reapproching the shock to get out will be 
%
\begin{align}
\varphi _{\text{out}} &= \int_{-1}^{-u_2/c} \dd{\mu } f_0 (u_2 + c \mu )= f_0 \frac{1}{2} \left(1 + \frac{u_2}{c}\right)^2
= f_0 \frac{1}{2} \left(1 - \frac{u_2}{c}\right)^2
\,.
\end{align}

Therefore, the probability is 
%
\begin{align}
P _{\text{return}} = \frac{\varphi _{\text{out}}}{ \varphi _{\text{in}}} = \frac{(1 - u_2 / c)^2}{(1 + u_2 / c)^2} \approx 1 - 4 \frac{u_2}{c}
\,.
\end{align}

This is quite sizeable, although it is not 1. 

\todo[inline]{Counterintuitive scaling with \(u_2 \)! }

Suppose we start with \(N_0 \) particles with energy \(E_0 \): after one cycle their energy will be \(E_1 = E_0 (1 + \Delta E / E)\), which comes out to 
%
\begin{align}
E_1 = E_0 \left( 1 + \frac{4}{3} \beta \right)
\,,
\end{align}
%
and \(N_1 = N_0 P _{\text{ret}}\) particles will have at least this energy. 

\(N_2 = N_0 P _{\text{ret}}^2 \) particles will have at least \(E_2 = E_0 (1 + 4 \beta /3)^2\) and so on. 

We have 
%
\begin{align}
\log \frac{E_k}{E_0 } = k \log \left(1 + \frac{4}{3} \beta \right)
\,,
\end{align}
%
so 
%
\begin{align}
k = \frac{\log E_k /E_0}{\log (1 + 4\beta /3)} = \frac{\log N_k / N_0 }{(1 - 4 u_2 / c)}
\,.
\end{align}

Taylor expanding, we find 
%
\begin{align}
\log \frac{N_k}{N_0 } = - \alpha \log \frac{E_k}{E_0}
\,,
\end{align}
%
where \(\alpha = 3 / (r-1)\), with \(r = u_1 / u_2 \). 

The powerlaw only depends on \(r\)! 
For a strong shock, we get \(r = 4\), so \(\alpha = 1\).

This is the \emph{integral spectrum}, while the differential spectrum \(n(E) \dd{E}\) will scale like \(n(E) \propto N_k / E \propto E^{- (r+2) / (r-1)}\), which means \(E^{-2}\) with \(r = 4\). 

This only holds for relativistic particles! 
For non-relativistic ones, this does not hold as we shall see.
 
The fact that this produces a powerlaw is great! We observe many powerlaws for non-thermal particles.

The energy of these high-energy particle scales like  \(\mathcal{E} \sim \int_{E_0 }^{E _{\text{max}}} \dd{E} A (E/E_0)^{-2} E \sim \log E _{\text{max}} / E_0 \).

If \(E _{\text{max}}\) is large enough, this quantity can become comparable with \(\rho u^2\), therefore the non-thermal particles will contribute a significant amount to the overall physics of the shock, invalidating the theory we used. 
This is the crux of the issue behind the \emph{nonlinear physics} of shock acceleration. 

What we need to do now is to get a connection with microphysics, which will give us a hint about where to go to deal with nonlinear theory. 

Properly speaking, \(u_1 \) and \(u_2 \) should be replaced with \(u_1 \pm v_{A, 1}\) and so on. 
This is relevant because the compression factor inside the spectrum is really 
%
\begin{align}
r = \frac{u_1 \pm v_{A, 1}}{u_2 \pm v_{A, 2}}
\,,
\end{align}
%
making the spectrum a bit harder or steeper. 

\end{document}