\documentclass[main.tex]{subfiles}
\begin{document}

\marginpar{Tuesday\\ 2022-1-11}

Anywhere we have a supersonic plasma, we will form a shock front. 
The conditions at the shock are called Rankine-Hugoniot equations. 

What the kinetic energy is dissipated into can vary: accelerated particles are a possibility, but we can also have heat, magnetic fields and so on. 

The transport equation for the distribution function in one dimension reads 
%
\begin{align}
\pdv{f}{t} + u \pdv{f}{x} = \pdv{}{x} \left(
    D \pdv{f}{x}
\right)
+ \frac{1}{3}
\dv{u}{x}
p 
\pdv{f}{p}
+ Q(x, p)
\,,
\end{align}
%
where \(u\) is the velocity of the plasma, \(D\) the diffusion coefficient, while \(Q(x, p)\) is the injection term: without it the equation is linear, so unless we put particles in there will not be any. 

This distribution in phase space will be roughly isotropic, so that \(f(p) \dd{p} = 4 \pi p^2 f(p) \dd{p}\). 

Conceptually, there is a neat separation between the plasma with its velocity \(u\) and the nonthermal particles with their distribution \(f\). 

The price to pay by inserting \(Q\) is to split the thermal and non-thermal particles. 

The form of \(u(x)\) near the shock looks like a step function. 
What happens to the nonthermal particles? 

We can define these as those for which the shock seems to be infinitesimally small. 

The thickness of the shock will be roughly of the order of the Larmor radius of the thermal particles, since the shock forms through EM interactions (while collisions are negligible). 

A practical example is a SN explosion: it leads to a plasma moving at about \SI{10000}{km/s}. 

The magnetic field in the field is roughly \(B \sim \SI{1}{\micro G}\); 
the Larmor radius is something like \(\num{e9}\) to \SI{e10}{cm}. 

This is tiny compared to the scale of the remnant.
The nonthermal particles are then chosen to be the ones such that their Larmor radius is much larger than the width of the shock. 

This implies that \(f\), the phase space distribution of the \emph{nonthermal} particles, must be continuous across the shock: 
they move so fast that their properties don't have time to change discontinuously across the shock.

The distribution function of the thermal particles downstream will be thermal, where the mean particle will have a Larmor radius on the order of magnitude of the shock thickness. 

The injected particles are those at the upper tail of the distribution, 
whose Larmor radii are roughly larger than the shock width, and which are then amplified. 

The fraction of the particles which are accelerated is tiny! 

The velocity gradient is modelled as 
%
\begin{align}
\dv{u}{x} = (u_2 - u_1 ) \delta (x)
\,.
\end{align}

The injection term can also be modelled as 
%
\begin{align}
Q (x, p) = Q_0 \delta (x) \delta (p - p _{\text{in}}) \frac{1}{4 \pi p _{\text{in}}^2}
\,,
\end{align}
%
since the part of the tail which is high enough to inject but still has particles in it is quite low. 

We will assume stationarity, thereby neglecting the \(\partial_t f\) term (and also taking \(Q\) not to be function of time as well).  

The approach is to solve the equation upstream and downstream, using the shock as a boundary condition. 

Upstream, we have 
%
\begin{align}
u_1 \pdv{f}{x} &= \pdv{}{x} \left( D \pdv{f}{x} \right)  \\
\pdv{}{x} \left(
    \underbrace{u_1 f}_{\text{advective}} - \underbrace{D \pdv{f}{x} }_{\text{current}}
\right) &= 0
\,,
\end{align}
%
where \(D \partial f\) is the current term. 
This equation is basically the conservation of the total flux of the particles! 

The source is downstream: upstream infinity is fast and cold, so there, in the interstellar medium we have \(f = \pdv*{f}{x} = 0\): therefore, we can say that that term vanishes, so 
%
\begin{align}
u_1 f = D \pdv{f}{x}
\,.
\end{align}

We take 
%
\begin{align}
f(x, p) &= A e^{\alpha x}  = f_0 (p) \exp(\alpha x)  \\
\pdv{f}{x} &= \alpha f_0 (p) \exp(\alpha x)
\,,
\end{align}
%
therefore we have \(u_1 f_0 = D \alpha f_0 \), so \(\alpha = u_1 / D\). 

Upstream for us is \(x < 0\); the distribution is therefore a decreasing exponential. 
The diffusion length is \(1 / \alpha  = D(p) / u_1 \). 

What this is telling us is that particles cannot leave!
They could leave if their Larmor radius became larger than the system, but this would lead to a peaked spectrum! 

Downstream the distribution is weird: if there were gradients, it could not be stationary since we also have convection. 
So, we have \(\pdv*{f}{x} = 0\) downstream. 

Let us integrate the function in \([- \epsilon , + \epsilon ]\) around 0: the term \(u \pdv*{f}{x}\) must vanish, since the distribution is continuous. 

We get 
%
\begin{align}
D \eval{\pdv{f}{x}}_{2} -
D \eval{\pdv{f}{x}}_{1}
+ 
\frac{1}{3} (u_2 - u_1 ) p \pdv{f_0 }{p} + k \delta (p - p _{\text{in}}) = 0
\,.
\end{align}

We know that \(D \pdv*{f}{x}\) downstream is zero, so we get 
%
\begin{align}
- D \eval{\pdv{f}{x}}_{1} = - D \frac{u_1}{D} f_0 
\,,
\end{align}
%
therefore 
%
\begin{align}
-u_1 f_0 + \frac{1}{3} (u_2 - u_1 ) p \pdv{f_0 }{p} = 0
\,
\end{align}
%
everywhere but at \(p _{\text{in}}\) (that will only serve towards normalization): 
%
\begin{align}
\frac{ \dd{f_0 }}{f_0 } = \frac{3 u_1 }{u_2 - u_1 } \frac{\dd{p}}{p}
\,,
\end{align}
%
or 
%
\begin{align}
f_0 = k p^{-3 u_1 / (u_2 - u_1 )}
\,.
\end{align}

We find the same powerlaw as yesterday! Also, the index is precisely \(3 r / (r -1 )\). 
Here, however, the result is in momentum, and it also holds for non-relativistic particles. 
For a strong shock we find \(f \sim p^{-4}\); the spectrum we found yesterday was 
%
\begin{align}
n(E) \dd{E} = 4 \pi p^2 f(p) \dd{p} 
\,,
\end{align}
%
meaning that in the relativistic regime \(E \sim p\) we get 
%
\begin{align}
n(E) \propto E^2 E^{-4} = E^{-2}
\,.
\end{align}

In the non-relativistic case, we have \(E \sim p^2 / 2m\), so 
%
\begin{align}
n(E) \propto p^2 p^{-4} \dv{p}{E}  \propto E E^{-2} \frac{1}{p} \propto E^{-3/2}
\,.
\end{align}

In momentum, the spectrum is always \(p^{-4}\); in energy we must distinguish between the two cases.

Here, we do not have any suppression at high \(p\): the powerlaw keeps going. 

There are problems: the energy is infinite if we take a strong shock with \(r = 4\). 
Even if we put a \(E _{\text{max}}\) by hand, we are completely neglecting the loss of energy by the plasma. 
The available energy is at most \(\rho u^2\)! 

The proper way to deal with this is a \emph{nonlinear extension}; we will just try to give a rough idea. 

The mass conservation equation reads \(\rho_0 u_0 = \rho_1 u_1 \); now we need to add a new term for the nonthermal particles to the momentum conservation equation, which now reads 
%
\begin{align}
\rho_0 u_0^2 + P_0 = \rho_1 u_1^2 + P_1 + P _{\text{CR}}
\,.
\end{align}

We do not add this to the mass conservation since their mass is tiny. 
Further, we assume that the shock is strong, therefore the gas pressure is negligible: 
%
\begin{align}
\rho_0 u_0^2 &= \rho_1 u_1^2 + P _{\text{CR}}  \\
1 &= \frac{u_1}{u_0 } + \frac{P_{\text{CR}}}{\rho_0 u_0^2}
\,. 
\end{align}
%
The term \(P _{\text{CR}} / \rho_0 u_0^2\) is the efficiency of CR production, \(\xi _{\text{CR}}\): the velocity at the shock is damped when we add cosmic rays. 
The plasma slows down when it accelerates cosmic rays. 

The velocity upstream near the shock will be smaller; the compression factor ``seen'' by the particles which remain close to the shock (since they have smaller Larmor radii) will be smaller, meaning that the spectrum is harder at smaller radii. 

The backreaction makes the problem of energy divergence even harder! 
With the stationarity assumption the system is doomed to break, if we don't use a nonlinear theory. 

This thing works relatively well if we have an efficiency \(\lesssim \SI{10}{\percent}\).

Is this process able to accelerate particles up to the very high energies we detect cosmic rays at? 

The problems in raising the energy is in the lifetime of the source, or its scale. 
The conditions are \(\tau _{\text{acc}} \lesssim \tau _{\text{age}}\) and \(D / u_1 \lesssim R _{\text{source}}\). 
These two roughly give the same bound. 

A rough estimate of \(D\) in the galaxy can be given through the abundance of spallation products; it comes out to be 
%
\begin{align}
D \approx \SI{3e28}{cm^2 / s}  \left( \frac{E}{\SI{}{GeV}}\right)^{1/2}
\,.
\end{align}

What is \(u_1 \)? It is the velocity of the shock (which is also the velocity of the ISM in the shock frame). 
This is typically \(v \sim \SI{e9}{cm /s}\) (or smaller). 

The scale of the source is typically the velocity of the shock times its timescale: \(\tau _{\text{source}} \approx \SI{3e10}{s}\) (1 thousand years). 

From here we get 
%
\begin{align}
\left( \frac{E}{\SI{1}{GeV}}\right) < \left( \frac{\tau}{\SI{1000}{yr}}\right)^{2}
\,.
\end{align}

If this is the case, we need a thousand years to get to a \SI{}{GeV}! 
We need to get to a million \SI{}{GeV}. This mechanism seems to be way too slow! 

Once we expressed the diffusion coefficient as 
%
\begin{align}
D = \frac{1}{3} r_L v \frac{1}{\mathscr{F}(k)}
\,,
\end{align}
%
so we can say that this quantity should be 
%
\begin{align}
\frac{1}{3} r_L v \frac{1}{\mathscr{F}(k)} \leq u_1^2 \tau _{\text{age}}
\,,
\end{align}
%
meaning that the power at the resonant frequency should be 
%
\begin{align}
\mathscr{F}(k) > \frac{1}{3} \frac{r_L}{R _{\text{source}}} \frac{v}{u_1 }
\,,
\end{align}
%
so for the Larmor radius of a \SI{1}{PeV} particle we get \(r_L = pc / eB \approx \SI{e19}{cm}\), while the source's scale is a few parsecs, so we get 
%
\begin{align}
\mathscr{F} > \frac{1}{3} \frac{\SI{e19}{cm}}{\SI{3e19}{cm}} \frac{\SI{3e10}{cm/s}}{\SI{e9}{cm/s}} \gtrsim 1
\,,
\end{align}
%
meaning that there must be a \emph{lot} of resonant power! 

This will also mean that the perturbative approach to particle diffusion will be broken.

Something very different must be happening directly upstream! 
The accelerating particles are the only thing which can be doing something upstream of the shocks: are their own magnetic fields strengthening the diffusion?

\end{document}
