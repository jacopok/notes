\documentclass[main.tex]{subfiles}
\begin{document}

\section{Particle transport}

\marginpar{Tuesday\\ 2022-1-18}

In the last couple of lectures, we have looked at how the 
presence of shocks affects cosmic rays. 

Today we smoothly transition to the problem of particle transport. 

Supernova remnants seem to be the most likely candidate for the production. 

Cosmic rays in the galaxy: 
we model them taking the galaxy as a thin disk
(in reality it's \SI{30}{kpc} in diamater, a few hundred parsecs in height). 
We say that its height is \(2h = \SI{300}{pc}\). 

All around it, we think of a region extending at least \(H=\) a few of \SI{}{kpc} where the density is very low, the plasma is completely ionized and magnetized. 
This region is called the \emph{halo} of the galaxy. 

We use this simple galactic model which neglects spiral arms and such,
since it still is able to give a rough idea of the mechanism. 

Let us assume that a source goes off in the galaxy; as we know due to Alfvén waves' resonance the cosmic rays will diffuse. 

Let us call the distribution function of boron nuclei \(f_B (E)\): this will satisfy 
%
\begin{align}
f_B(E) = \text{rate of production} \times \text{confinement time}
\,,
\end{align}
%
but what is the rate of production? It will be given by nuclear fission of heavier elements --- let's assume it is only due to Carbon, so
%
\begin{align}
\text{rate of production} = f _C (E) n _{\text{gas}} \sigma _{\text{sp}} c
\,,
\end{align}
%
where we take the speed to be \(c\) since the Carbon nuclei are assumed to be relativistic. Then, 
%
\begin{align}
\frac{f_B (E)}{f_C(E)} = n _{\text{gas}} \sigma _{\text{sp}} c \tau _{\text{confinement}} = \frac{\tau _{\text{confinement}}}{\tau _{\text{spallation}}}
\,.
\end{align}

Why do we say the energy is the same? 
It's not! however, what we mean here is the 
\emph{energy per nucleon}, which is conserved in the process \(\ce{C} + \ce{N} \to \ce{B} + x\). 

We can actually measure this ratio! 
It's about \(1/3\) around \SI{10}{GeV} per nucleon. 

\todo[inline]{which means that one Carbon every four is spallated}

The mean value of the density in the disk and outside is about 
%
\begin{align}
\overline{n} = \frac{n_d \pi r_d^2 2 h + n_H r_d^2 \pi 2H}{2 \pi r_d H} \approx n_d \frac{h}{H}
\,.
\end{align}

The halo's only effect is to have the particle diffusing, 
while (almost) only when it passes through the galaxy it 
can interact. 

The energy density of the gas at the Sun, which we take to be representative of the disk, is about 
%
\begin{align}
\omega _{\text{CR}} \approx \SI{1}{eV / cm^3}
\,,
\end{align}
%
therefore the total energy available is \(E _{\text{CR}} = \omega _{\text{CR}} \pi r_d^2 2 H \). 

Let us say that this ratio is \(E _{\text{CR}} = \xi _{\text{CR}} E _{\text{source}} R _{\text{source}} \tau _{\text{confinement}}\), 
where \(\xi _{\text{CR}}\) is the rate at which energy is transformed into cosmic rays. 

This is then also 
%
\begin{align}
E _{\text{CR}} = \frac{1}{3} \xi _{\text{CR}} R _{\text{source}} E _{\text{source}} \frac{H}{n_d h \sigma _{\text{sp}} c}
\,,
\end{align}
%
so we get 
%
\begin{align}
\xi _{\text{CR}} = \frac{3 \omega _{\text{CR}} \pi r_d^2 n_d h \sigma _{\text{sp}} c}{E _{\text{source}} R _{\text{source}}}
\,,
\end{align}
%
of which we know everything except for \(\xi _{\text{CR}}\); 
the idea is that for various source types we can estimate their energy and rate, and see what \(\xi \) is required. 

For supernovae, we have \(E _{\text{source}} \approx \SI{e51}{erg}\), while \(R _{\text{source}} \approx 1/30\unit{yr^{-1}}\). 
This means we require \(\xi _{\text{CR}} \approx \SI{3}{\percent}\) if \(r_d \approx \SI{30}{kpc}\). 

This does not suffice, of course, but it does indicate that if an acceleration mechanism can work then we are good! 

We now define 
%
\begin{align}
X(E) = n_d \frac{h}{H} m_p \tau _{\text{confinement}}(E) \approx \frac{1}{3} \frac{m_p}{\sigma _{\text{sp}}}
\,:
\end{align}
%
this is called the \emph{grammage}, the total amount of matter ``experienced'' by the CR before leaving the galaxy. 
It is roughly \SI{10}{g / cm^2}. 

The confinement time, on the other hand, is on the order of \SI{100}{Myr}. 

The main problem with this paradigm is the maximum energy; 
recent investigations showed that typical remnants can achieve energies on the order of \SI{100}{TeV}, but not more! 

There is a large zoo of supernovae; 
the type Ia and type II are the most common. 

Type I are old stars: they exhibit no hydrogen lines. 

The collapse of a white dwarf happens when it starts accreting material from its companion, through Roche lobe overflow. 

Type II supernovae are completely different: they are core collapse supernovae. 

For both of these, the typical energy is a foe. 
What is the velocity of the ejecta? It turns out to be about 
%
\begin{align}
v _{\text{ej}} = \SI{e9}{cm/s} \left(\frac{E}{\SI{e51}{erg}}\right)^{1/2}
\left(\frac{M _{\text{ej}}}{M_{\odot}}\right) \ll c
\,.
\end{align}

This must be compared with the sound speed of the ISM: 
%
\begin{align}
c_s = \sqrt{\frac{\gamma P}{\rho }} = \sqrt{\gamma \frac{n k_B T}{n m_p}}
\approx \SI{e6}{cm/s} \left(\frac{T}{\SI{e4}{K}}\right)^{1/2}
\,,
\end{align}
%
meaning that the Mach number of the shock is on the order of \(M \approx 1000\). 

The Alfvén speed, on the other hand, is on the order of 
%
\begin{align}
v_A = \frac{B}{\sqrt{4 \pi \rho }} \approx \SI{2e5}{cm/s} \left(\frac{n}{\SI{1}{cm^{-3}}}\right)^{-1/2} \left(\frac{B}{\SI{1}{\micro \gauss}}\right)
\,,
\end{align}
%
so the plasma is moving super-Alfvenically as well in the reference frame of the shock. 

We can see the shock front with X-ray observations, 
and we can measure its velocity by looking at it after years. 

As mentioned before, both the sonic and Alfvenic Mach numbers are \(\gg 1\). 

It is important to know \(D\) and \(u\), which give us a characteristic length \(D / u\) and a characteristic time \(\sim D / u^2\). 

The shock will slow down as it passes through the ISM. 
This process happens in two stages. 

\paragraph{Ejecta dominated phase}
Here, the shock velocity is roughly constant, so the shock radius is roughly \(R = v _{\text{sh}} t\). 

The mass flux is \(\rho _{\text{ISM}} v _{\text{sh}}\), so the accumulated mass is 
%
\begin{align}
M _{\text{accumulated}} &= \rho _{\text{ISM}} v _{\text{sh}} 4 \pi R^2 _{\text{sh}} t = 4 \pi \rho _{\text{ISM}} v _{\text{sh}}^3 t^3  \\
&= \SI{6.2e26}{g} n_1 E^{3/2}_{51} M _{\text{ej}}^{-5/2} t^3
\,.
\end{align}

When this becomes comparable to the mass of the ejecta, the adiabatic (Sedov-Taylor) phase starts: 
%
\begin{align}
t _{\text{st}} \approx \SI{150}{yr} E_{51}^{-1/2} M _{\text{ej}}^{5/6} n_I^{-1/3}
\,.
\end{align}

A radius can also be associated to this transition: \(R _{\text{st}} = v _{\text{eq}} t _{\text{st}} \sim \SI{1.5}{pc}\).

Then, the scaling will look like 
%
\begin{align}
\frac{1}{2} \left(M _{\text{ej}} + M _{\text{acc}}\right) v _{\text{sh}}^2 = E _{\text{SN}}
\,,
\end{align}
%
so, without looking at the transition, let us assume that the accelerated mass is dominant right away: 
%
\begin{align}
\frac{1}{2} \rho _{\text{ISM}}  v _{\text{sh}} 4 \pi R _{\text{sh}}^2 t v _{\text{sh}}^2 =  E _{\text{SN}}
\,,
\end{align}
%
so we will have \(R _{\text{sh}} \sim t^{\alpha }\), \(v _{\text{sh}} \sim t^{\alpha - 1}\) for some \(\alpha \): then the equation looks like 
%
\begin{align}
t^{5 \alpha - 2} \sim E = \text{constant}
\,,
\end{align}
%
so \(\alpha = 2 /5 \). 
The radius increases like \(t^{2/5}\), the velocity decreases like \(t^{-3/5}\).

We can do the same setup for SN II: sometimes they emit a lot of stellar wind with a certain \(\dot{M}\) which can be taken to be roughly constant, in which case then \(\rho _{\text{wind}} = \dot{M} / 4 \pi r^2 v _{\text{wind}}\) near the massive star. 

\begin{extracontent}
    Do the same exercise assuming the profile of the gas is this one. 
\end{extracontent}

At the beginning of the \(ST\) phase the reverse shock is hitting the center. 

The best time for CR acceleration is the transition into the ST phase. 

Between the shocks, there is a ``contact discontinuity'' where the shocked ejecta become subsonic. 
This is typically a turbulent region. 
1D hydrodyamic approximation can model this. 

So, the galactic model.

The equation is 
%
\begin{align}
\pdv{f}{t} + u \pdv{f}{z} = \frac{1}{3} \dv{u}{z} p \pdv{f}{p} 
+ \pdv{}{z} \left(
    D \pdv{f}{z}
\right)
+ Q
\,.
\end{align}

The velocity \(u\) is the one of the ISM, there are situations in which it's nonvanishing but we will take it to be uniformly zero. 

Then, we kill those two terms. 
The timescale for confinement is quite a bit smaller than galactic evolution timescales, so we assume stationarity; 
then, the equation is simply 
%
\begin{align}
\pdv{}{z} \left( D \pdv{f}{z}\right) = -Q
\,,
\end{align}
%
but what are the sources \(Q\)? 

Suppose each source produces a spectrum \(N(p) \sim p^{-4}\): then we will have 
%
\begin{align}
Q = N(p) R _{\text{sources}} \frac{\delta (z)}{\pi r_d^2}
\,.
\end{align}

But, we also need to account for the fact that particles leave at a certain point? 
We can do so by imposing a \emph{free escape boundary condition}: \(f (\abs{z} = H) = 0\) (we are neglecting the lateral boundaries). 

This is even easier than the equation for the shock: 
for \(z \neq 0\) we have \(\partial_z (D \partial_z f) = 0\), which is solved by a linear equation in the form \(f= kz/D + f_0 \). 

However, the system is symmetric: what happens if we integrate around \(z=0\)? 
%
\begin{align}
2 \eval{D \pdv{f}{z}}_{+} = - \frac{N(p) R _{\text{source}}}{\pi r_d^2}
\,.
\end{align}

The distribution is therefore 
%
\begin{align}
f(z) = - \frac{N(p) R _{\text{source}}}{2 \pi r_d^2 D} \abs{z} + f_0 
\,.
\end{align}
%

Further, we need 
%
\begin{align}
\frac{k}{D} H + f_0 = 0
\,,
\end{align}
%
so the full solution reads 
%
\begin{align}
f(z) = - \frac{N(p) R _{\text{source}}}{2 \pi r_d^2 D} \abs{z} + \frac{N(p) R _{\text{source}}}{2 \pi r_d^2 } \frac{H}{D}
\,.
\end{align}

The density in the galaxy is then just 
%
\begin{align}
f_0 = \underbrace{\frac{N(p) R _{\text{source}}}{2 \pi r_d^2 H}}_{\text{injection phase space density rate}} \underbrace{\frac{H^2}{D}}_{\text{diffusion timescale}}
\,.
\end{align}

Since  \(D(p)\) is a function of \(p\), the injection spectrum is not the same one which is observed!

If \(D(p) \sim p^{\delta }\) and \(N(p) \sim p^{-\gamma }\), 
we will always measure \(p^{- \delta - \gamma }\). 

We need a different channel to figure out what \(\delta \) is. 

The flux of escaping particles is the source flux: \(D \partial_z f \sim N(p)\), so outside the galaxy the spectrum is the source one. 

For protons, the loss timescale is about a \SI{}{Gyr}. 
For nuclei, this line of reasoning does not hold, as we shall see next week.

\end{document}