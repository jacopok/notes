\documentclass[main.tex]{subfiles}
\begin{document}

\marginpar{Wednesday\\ 2022-1-19}

Yesterday we started looking at the transport of cosmic rays in the galaxy through a simple cylindrical model. 

Because we have CR protons leaving the galaxy, we get an electric current! 
This generates a turbulent magnetic field. 

Now we will look at spallation products in more detail. 
The ISM basically only reflects Big Bang Nucleosynthesis (Hydrogen and \SI{25}{\percent} of Helium)
and the products of stellar fusion. 

In stars, Boron Lithium and Beryllium are destroyed more efficiently than they are formed. 

Let us write the simplified transport equation we found yesterday, but accounting for the losses underwent by a primary nucleus (which is not formed much by spallation) denoted as \(p\): 
%
\begin{align}
-\pdv{}{z} \left( D \pdv{f_p}{z} \right) = \frac{N_p R}{\pi r_d^2} \delta (z) 
- f_p n_d \sigma _{\text{sp}} c 2 h \delta (z)
\,,
\end{align}
%
since this nucleus undergoes reactions in the form 
%
\begin{align}
\ce{C} + \ce{p} _{\text{ISM}} \to \ce{B} + X
\,.
\end{align}
%
where we only write Boron but the cross-section \(\sigma _{\text{sp}}\) refers to all possible spallation reactions. 

We assumed that everything is relativistic; if we do not assume this there are additional energy losses --- specifically, ionization. 

Further, like yesterday we assumed that the plasma velocity \(u\) vanishes; for the high energies we are looking at (more than \SI{10}{GeV} per nucleon) this is a good assumption. 
for smaller energies this it not a good assumption. 

The Alfvén waves are moving and they are not isotropic (since the CR distribution is not isotropic), but their velocity is relatively small --- less than \SI{10}{km/s}.

Advection is important when \(H/v_A < H^2/  D\), which comes out to be around \SI{10}{GeV}. 

The galactic rotation velocity could be relevant at these energies. 

Below \SI{10}{GeV} we have trouble measuring cosmic rays, because of solar modulation. 

Let's solve the equation! 
When \(z \neq 0\), we get again \(f(z, p) = A + Bz = f_0 + Bz = f_0 ( 1- \abs{z} / H)\). 

But, what happens when we integrate around 0?


%
\begin{align}
-2 D \eval{\pdv{f}{z}}_{0^{+}} &= \frac{N_p R}{\pi r_d^2}
- f_0 n_d \sigma _{\text{sp}} 2 hc  \\
2 D \frac{f_0 }{H} &= \frac{N_p R}{\pi r_d^2}
- f_0 n_d \sigma _{\text{sp}} 2 hc  \\
f_0 \left( \frac{2D}{H} + 2h c n_d \sigma  _{\text{sp}} \right) &= \frac{N_p R}{ \pi r_d^2}  \\
f_0 \left( 1 + n_d \frac{h H}{D} c \sigma _{\text{sp}} \right) &= 
\frac{N_d R H}{2 \pi r_d^2 D}
\,,
\end{align}
%
so we have 
%
\begin{align}
f_0 (E) = \frac{N_p R H}{2 \pi r_d^2 D} \frac{1}{1 + n_d \frac{1}{H} \frac{H^2}{D} c \sigma _{\text{sp}}}
\,,
\end{align}
%
which we can write ass a function of the grammage \(X = n_d (h / H) m_p (H^2/ D)\): 
%
\begin{align}
f_0 (E) = \frac{N_p R H}{2 \pi r_d^2 D} \frac{1}{1 + X(E) / X _{\text{crit}}}
\,,
\end{align}
%
where \(X _{\text{crit}} = m_p / \sigma _{\text{sp}}\). 

So, if the grammage is small compared to this critical amount then the spectrum is basically unchanged since the interactions are weak. 
The grammage scales inversely with energy (because of its inverse dependency on \(D\)), so this happens at high energies. 

Very energetic nuclei behave basically as a proton. 
On the other hand, less energetic nuclei with \(X \gg X _{\text{crit}}\) 
have 
%
\begin{align}
f_0 (E) = \frac{N_p R H}{2 \pi r_d^2 D} \frac{m_p / \sigma _{\text{sp}}}{n_d (h / H) m_p (H^2/ D) c} = \underbrace{\frac{N_p(E) R}{2 \pi r_d^2 H}}_{\text{injection rate per unit volume}} \underbrace{\frac{1}{n_d \sigma _{\text{sp}} (h/H) c}}_{\text{interaction timescale}}
\,.
\end{align}

The spectrum is the same as the production one! 
The ratio \(h / H\) came out naturally. 

If the production spectrum is \(N_p \sim p^{- \gamma }\) while the diffusion looks like \(D \sim E^{ \delta }\), we get \(f_0 \sim E^{-\gamma - \delta }\) for little interaction, while \(f_0 \sim E^{-\gamma }\) for much interaction. 

\begin{extracontent}
    Exercise: take the diffusion coefficient to be different in the galaxy and the halo. 
    However, we need to impose continuity. 
    We should find that the diffusion coefficient in the disk is irrelevant. 
\end{extracontent}

The presence of neutral gas is very effective on the diffusion coefficient; the neutral gas kills Alfvén waves. 

We still need to understand what is happening to the 
spallation products. 
This will apply to the stable ones, for the unstable ones we need to account for the decay, which we will do tomorrow. 

Let us consider the distribution of a spallation product --- really, we would need to consider all the web of possible reactions. 

We will neglect reacceleration of spallation products, and only include a source term accounting for the spallation. 
The equation then reads 
%
\begin{align}
- \pdv{}{z} \left( D \pdv{f_s }{z}\right) = f_p n_d \sigma _{\text{ps}} c 2 h \delta  (z) - f_s n_d \sigma _{\text{s}} c 2 h \delta (z)
\,,
\end{align}
%
where \(\sigma _{\text{ps}}\) is the cross-section for the primary to go into the secondary, while \(\sigma _{\text{s}}\) is the branching ratio for the secondary to spallate again.

At high energies, it looks like the spallation cross-section is roughly 
%
\begin{align}
\sigma _{\text{spall}} \approx \SI{45}{mb} A^{0.7}
\,.
\end{align}

People were interested in this in the 60s and 70s and then they stopped; a lot of the limitations of CR physics come from the uncertainties in these old measurementes. 

For \(z \neq 0\) again we get 
%
\begin{align}
f_s (z) = f_0 \left( 1 - \frac{\abs{z}}{H} \right)
\,.
\end{align}

Again, we get 
%
\begin{align}
2 D \frac{f_0}{H} = f_{p, 0} n_d \sigma _{\text{ps}} c 2 h  - f_{s, 0} n_d \sigma _{\text{s}} c 2 h
\,.
\end{align}

This means 
%
\begin{align}
f_{0, s}  \left( 
    \frac{2D}{H} + n_d \sigma _s 2 h c
\right)
&= f_{p, 0} 
n_d \sigma _{\text{ps}} c  
2h  \\
f_{s, 0} \left(1 + \frac{n_d h H^2\sigma _s c}{H D}\right) &= f_{p, 0} n_d \frac{c}{m_p} \frac{h}{H} \frac{H^2}{D} m_p  \\
f_{s, 0} &= f_{p, 0} \frac{ X(E) / X _{\text{crit, ps}}}{1 + X / X _{\text{crit, s}}}
\,,
\end{align}
%
which means that 
%
\begin{align}
\frac{\text{secondary}}{\text{primary}} = \frac{X(E) / X _{\text{crit, ps}}}{1 + X(E) / X _{\text{crit, s}}}
\,.
\end{align}

Suppose that \(X(E) \ll X _{\text{crit, s}}\). 
Then, the ratio is simply \(X(E) / X _{\text{crit, ps}} \propto H/D\):
the energy dependence of the diffusion coefficient is known in this way. 

This tells us a lot about the microphysics! 
We don't know \(H\) very well, but we do have an estimate for it 
within a few: therefore, we know
%
\begin{align}
D(E) \approx \SI{3e28}{cm^2 / s} \left( \frac{E}{\SI{}{GeV}}\right)^{1/2}
\,.
\end{align}

If we compare this to the expression we had found a long time ago, 
%
\begin{align}
D \sim \frac{1}{3} \frac{r_L c}{\mathscr{F}(k)}
\,,
\end{align}
%
which tells us, if we put the numbers in, that we need extremely small fluctuations! 

Secondary over primary is flat at low energies, and goes down with \(E^{-\gamma }\) as the energy increases. 

At low energies, we expect the carbon (primary) spectrum to look like \(E^{-\gamma -\delta }\), while the boron spectrum would look like \(E^{-\gamma - 2 \delta }\); 
on the other hand if Boron is produced at the source it looks as if it was a primary. 

If the spectrum at \(x_0 \) in the shock is \(E^{- \gamma - 2 \delta}\), what happens? 

What about antiprotons? 
They can be produced in \(2\ce{p} \to 3\ce{p} + \overline{\ce{p}}\). 
The kinematic threshold is quite high, while the production of pions is much easier. 

Let us neglect the dependence of cross-section on the energy for simplicity. 

The equation we computed for the secondaries is the same one we should use for antiprotons! 
Their energy losses are negligible, so we get \(\overline{p} / p \sim h / D(E)\). 

Really, we measure an almost-flat spectrum for it. 

There are two things which are not well measured: the cross-section, and the spectrum. 
However, within experimental uncertainties the measurements (antiprotons and boron) are incompatible. 

Tomorrow we will break this degeneracy with unstable Beryllium-10: 
this is nice since it introduces a new timescale. 

The three isotopes are Beryllium 7, 9 and 10: in lab they are produced in the same amounts, while in CRs we see almost no Beryllium. 
This gives us an estimate of the time the Beryllium remained there. 

Since that flux is low, we need a very good experiment with a very good magnetometer. 
AMS hoped to do that; unfortunately the magnet had a problem and was replaced with a weaker permanent magnet. 
This could not distinguish the isotopes, however it could measure Beryllium over Boron: 
%
\begin{align}
\frac{\ce{Be}}{\ce{B}} = \frac{^{7} \ce{Be} + ^{9} \ce{Be} + ^{10} \ce{Be} + }{^{10} \ce{B}}
\,,
\end{align}
%
which is still relatively good since \(^{10} \ce{Be}\) decays into \(^{10} \ce{B}\). 

For leptons we have a \emph{huge} anomaly. 



\end{document}
