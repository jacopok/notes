\documentclass[main.tex]{subfiles}
\begin{document}

\marginpar{Thursday\\ 2022-1-20}

Yesterday we solved the transport equation for stable nuclei, 
both primary and secondary; now we need to move to unstable ones.

This is useful since the secondary over primary ratio in that case scales like 
\(S/P \propto H/D(E)\), while unstable ones give us a new timescale. 

For very high energies, the lifetime scales like \(\gamma \)! 
long lifetimes can be multiplied. 

Let us look at Beryllium. It will not have a source term in a shock; 
we get an equation like 
%
\begin{align}
\pdv{}{z} \left(D \pdv{f_s}{z}\right) 
= - f_p n_d c \sigma _{\text{ps}} 2 h \delta (z) 
+ f_S n_d c \sigma _s 2 h \delta (z) + \underbrace{\frac{f_s}{\tau _d}}_{\text{new decay loss term}}
\,,
\end{align}
%
where \(\tau _{d} = \gamma \tau _{\text{1/2}}\) is the lifetime in the lab frame; 
in the \(z \neq 0\) region the equation simplifies to 
%
\begin{align}
D \pdv[2]{f_s}{z} = \frac{f_s}{\tau _d}
\,,
\end{align}
%
therefore the solution reads 
%
\begin{align}
f_s &= A \exp(- \beta z) + B \exp(\beta z)  \\
\pdv{f_s}{z} &= -\beta A \exp(-\beta z) + B \beta \exp(\beta z) \\
\pdv[2]{f_s}{z} &= \beta^2 A \exp(-\beta z) + B \beta^2 \exp(\beta z)
\,,
\end{align}
%
which we can use to show that \(DA \beta^2 = A / \tau _d\), which implies \(\beta = 1/ \sqrt{D \tau  _d}\). 

In the galaxy we have \(A+B = f_0\), while at the boundary \(z=H\) we have \(A \exp(- \beta H) + B \exp(\beta H) = 0\). 
The exponential is just a number: \(\exp(\beta H) = \exp(H / \sqrt{D \tau _d}) = g\). 

Therefore, we get 
%
\begin{align}
f_0 (p) &= - f_0 \frac{g^2}{1 - g^2} \exp(- \beta z) + \frac{f_0}{1 - g^2} \exp( \beta z)
\,,
\end{align}
%
an implicit form which we can integrate over the galactic disk: 
%
\begin{align}
2 D \eval{\pdv{f}{z}}_{0^+} = 
- f_{p0} n_d \sigma _{\text{ps}} 2h 
+ f_{s0} n_d c \sigma _s 2h
\,.
\end{align}

Plugging the solution into this, we get 
%
\begin{align}
\eval{\pdv{f}{z}}_{0^+} = \frac{g^2 f_0 \beta }{1 - g^2}
+ \frac{f_0 \beta }{1 - g^2} = 
\frac{f_0 \beta }{1-g^2} (1 + g^2)
\,,
\end{align}
%
and putting this into the integration around zero we get 
%
\begin{align}
f_{s0 } \left(
    \sqrt{ \frac{D}{\tau _d}} \frac{1 + g^2}{1-g^2}
    - n_d \sigma _s c h 
\right) 
= - f_{p0} n_d \sigma _{\text{ps}} c h
\,,
\end{align}
%
which is the full solution! 

Let us look at its limits: \(\tau _d \to \infty\) is a stable nucleus or an unstable one in the limit of very high energy; 
in that case, the exponential looks like 
%
\begin{align}
g = \exp( \frac{H}{\sqrt{D \tau _d}}) 
\,,
\end{align}
%
so the term is 
%
\begin{align}
\frac{1 + g^2}{1- g^2} \approx \frac{2}{1 - 1- H / \sqrt{D \tau _d}}
\,.
\end{align}

In the solution then \(\sqrt{\tau _d}\) cancels (as it should), so we have
%
\begin{align}
f_{s0} \left( 1 + \frac{n_d \sigma _s c h H}{D}  \right)
= f_{p0} n_d \sigma _s c \frac{h}{H} \frac{H^2}{D}  
\,,
\end{align}
%
therefore again the solution is 
%
\begin{align}
f_{s0} = f_{p0} \frac{X / X _{\text{crit}}}{1 + X / X _{\text{sp}}}
\,.
\end{align}

What happens, instead, when \(\tau _d\) is very small? 
\(g\) approaches infinity, so we get \((1+g^2) / (1 - g^2) \to \infty\). 
The equation is then 
%
\begin{align}
f_{s0} \left( \sqrt{ \frac{D}{\tau _d}} + n_d \sigma_s c h  \right)
= f_{p0} n_d \sigma _{\text{ps}} c h
\,,
\end{align}
%
so in the end we find 
%
\begin{align}
f_{s0} = f_{p0} \frac{n_d \sigma _{\text{ps}} c h}{n_d \sigma _s h + \sqrt{D / \tau _d}} \propto \sqrt{ \frac{\tau _d}{D}}
\,.
\end{align}

If we are able to measure this, we get an independent way to measure \(H\)! 
This yields a relatively large halo, on the order of  \(H \gtrsim \SI{5}{kpc}\). 

The experiments we currently have are not able to distinguish the specific isotopes of Beryllium, but we can measure overall Beryllium over Boron, which is already useful.

What we said so far applies at high energies, where we can neglect the Alfvén speed. 

In the equations of MHD there is a term like \(- \nabla P\); 
in our case this will also include the pressure of cosmic rays! 

Consider the bulk of the distribution: its distribution looks like \(f = f_0 (1 - \abs{z} / H)\), so there is a gradient in 
%
\begin{align}
P_c = \frac{1}{3} \int \dd{p} 4 \pi p^2 p \delta (p) f(p, z)
\,.
\end{align}

So, assume that there is dark matter, with a NFW profile (\(1/r\) up to \SI{10}{kpc}, \(1/r^3\) after). 

At \(r = 0\), consider (on the axis of the galaxy) the comparison between the gravitational force by DM and the pressure term \(- \nabla P\). 

If the pressure gradient is high enough, we can get CR-driven winds! 

There is currently very weak evidence that our own galaxy has winds. 

These winds are really ``fountains'', with the plasma later falling back to the galaxy. 

The aspects discussed up to now are in great agreement with experiment; 
the big anomaly is with leptons. 

There, the transport equation reads 
%
\begin{align}
\pdv{}{z} \left(D \pdv{f}{z}\right) = - Q + \frac{f}{\tau _L}
\,,
\end{align}
%
where \(f\) is the distribution of electrons, \(Q\) is a source term while \(\tau _L\) is the timescale for losses. 

We need to solve this equation for electrons just like for protons, with the difference that electrons lose a lot of energy through synchrotron and inverse Compton. 
These processes have \(\dv*{E}{t} \propto E^2\) (at least in the Thompson regime). 

Therefore, \(\tau _L \sim E / \dot{E} \propto E^{-1}\). 

The approach is otherwise the same as before, taking the source term to have the form \(Q = N(E) R \delta (z) / \pi r_d^2\). 

We get the same solution as before: 
%
\begin{align}
f(z, p)= A \exp(- \beta z) + B \exp(\beta z)
\,,
\end{align}
%
so again we get 
%
\begin{align}
f = \frac{f_0 g^2}{g^2 - 1} \exp(- \beta z) + \frac{f_0 }{g^2-1} \exp(\beta z)
\,. 
\end{align}

The distribution at \(z=0\) can be found by integrating around it as usual: 
%
\begin{align}
f_0 (E) = \frac{N_e R}{2 \pi r_d^2} \sqrt{ \frac{\tau_L}{D}} \frac{g^2- 1}{g^2 + 1}
\,.
\end{align}

If \(\tau _L\) is very large, we get 
%
\begin{align}
f_0 (E) = \frac{N_e R}{ 2 \pi r_d^2} \sqrt{\frac{\tau _L}{D}} \frac{H}{\sqrt{D \tau _L}} = \frac{N_e R}{2 \pi r_d^2 H} \frac{H^2}{D}
\,.
\end{align}

Again if \(\tau _L \) is very small, we find 
%
\begin{align}
f_0 (E) = \underbrace{\frac{N_e(E) R }{2 \pi r_D^2} \frac{1}{\sqrt{D \tau _L}}}_{\text{source}} \tau _L
\,,
\end{align}
%
which makes sense since the electrons can only reach as far as \(\sqrt{D \tau _L}\) from the disk, and we know that \(\sqrt{D \tau _L} \ll H\). 
The volume they can fill is therefore simply \(2 \pi r_d^2 \sqrt{D \tau _L}\). 

If we put the numbers in, we see that for most energies electrons are loss dominated. 

For a field of \SI{3}{\micro \gauss}, inverse Compton with CMB photons and synchrotron with the galactic magnetic field have the same magnitude. 

Around \SI{10}{GeV}, \(\tau _L\) is shorter than \(H^2 / D\). 

The overall spectrum should therefore be 
%
\begin{align}
E^{-\gamma -1 - (\delta /2 - 1) } = E^{-\gamma - 1/2 - \delta /2}
\,.
\end{align}

This spectrum is slightly steeper than that of protons. 

The ``leaky box'' model applies only if there are no energy losses. 

Protons in the ISM sometimes interact with CR protons, producing pions, muons, neutrinos, electrons and positrons. 

These new electrons and positrons add to the leptonic spectrum! 

Typically, a proton with energy \(E_p\) produces an electron with energy \(E_e \approx E_p / 20\). 

In order to model this, we add a source term in the form 
%
\begin{align}
\frac{f_p (E / \xi )}{\xi } \sigma _{\text{pp}} c n_d 2 h \delta (z)
\,.
\end{align}

We take the source function to be in the form 
%
\begin{align}
q(E_{\pm } ) \dd{E_{ \pm}} =  f_p( E_{\pm} / \xi ) \underbrace{\dd{E_p}}_{= \dd{E_{\pm}} / \xi } \sigma _{pp} c n_d
\,,
\end{align}
%
where \(E_{\pm}\) is the energy of the positron or electron, while here we need to account for the different energy of the protons, with \(\xi \approx 1/20\). 

This new term is tiny for electrons, it is very relevant for positrons though! 
The density of protons is known, so we can apply the same rule of thumb as before! 

The density of positrons, as long as losses are \emph{not} important, will look like 
%
\begin{align}
f_+ (E_+) \sim \underbrace{\left(\frac{E_+}{\xi  }\right)^{\gamma } \left(\frac{E_+}{\xi }\right)^{-\delta }}_{\text{source}} H^2 E_+^{-\delta }
\,,
\end{align}
%
where the source term is the proton term, just like the electron term. 
Thus, if losses are unimportant the spectrum of positrons scales such that 
%
\begin{align}
\frac{f_+}{f _{\text{primary} e^-}} \propto E^{-\delta }
\,.
\end{align}

If losses are important, the spectrum of the primary electrons is 
%
\begin{align}
f_- = \frac{N_e (E) R \tau _L}{2 \pi r_d^2 \sqrt{D \tau _L}} 
\,,
\end{align}
%
while (secondary) positrons will look like 
%
\begin{align}
f_+ = \underbrace{\frac{N_p (E /\xi ) R /\xi }{2 \pi r_d^2 H} \frac{H^2}{D(E / \xi )}}_{\text{spectrum of primary protons}} \frac{\tau _L}{\sqrt{D \tau _L} 2 \pi r_d^2} 
\,,
\end{align}
%
so overall 
%
\begin{align}
\frac{f _{\text{primary electrons}}}{f _{\text{secondary protons}}} 
\propto E^{- \delta }
\,.
\end{align}

This is the same result we got in the case in which losses were unimportant! 

If we measure this \emph{positron fraction}, it \emph{increases} with energy! 
(technically we are computing \(e^- / (e^- + e^+)\), but positrons are few). 

Pulsars are a factory of electron-positron pairs. 

Perhaps there is a new mechanism for the production of positrons, like neutron stars? 



\end{document}
