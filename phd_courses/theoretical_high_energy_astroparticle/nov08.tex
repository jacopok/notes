\documentclass[main.tex]{subfiles}
\begin{document}

\marginpar{Monday\\ 2021-11-8, \\ compiled \\ \today}

Pasquale Blasi gives this part of the course. 
He might ask us to do exercises at the blackboard. 

It is a theory course, but he wants to push the physical interpretation
a lot. 

In general, in order to produce high-energy particles we need 
acceleration mechanisms and transport mechanisms. 

These are non-thermal particle sources. 
The spectra from these phenomena seem to be powerlaws, non 
the exponentially suppressed tail of a Maxwellian.

Things need to happen \emph{faster} than the collisions between the particles. 
The system must be, to a fist, approximation, collisionless. 

We typically see charged particles making up cosmic rays, 
and the (interstellar, intergalactic) medium they move in is mostly ionized. 

These will then move under the action of electromagnetic fields. 
These charged particles also produce EM fields, so they 
``feel what the others are doing''. 

There is a deep connection between the micro and the macro physics. 
David N.\ Schramm called this an ``inner world --- outer world'' connection. 

We need both particle physics and plasma physics! 

Things to study are: 
\begin{enumerate}
    \item high-energy cosmic rays;
    \item astrophysical \(\gamma \) rays and neutrinos;
    \item \dots
\end{enumerate}

A brief history lesson. We start with electroscopes, which discharge: 
at the end of the 1800s the explanation was thought to be radioactivity. 

Pacini explored electroscope behaviour underwater: decreasing. 
Wulf explored it on a balloon: it was increasing above \SI{2}{km} of elevation! 

Millikan proposed them as the ``birth cry'' of elements: \(\gamma \) rays with energies
corresponding to the mass defects of certain nuclides. 
He used Dirac's theory of Compton scattering. He was wrong.

Bruno Rossi put lead blocks between Geiger counters: these are not \(\gamma \) rays, since
they would be absorbed. 

They have to be charged: their flux is influenced by the Earth's magnetic field, so 
the one from the East and the West is different. 

In the 1930s there started a boom of discoveries. 
At the end of the 30s Auger measured \SI{100}{TeV} cosmic rays. 

At the end of the 60s the CMB was discovered: 
there should be a cutoff around \SI{e20}{eV}. 
This is because the cosmic ray ``sees'' CMB photons as \(\gamma \) rays, so  
it can undergo production of pions, losing a lot of energy. 

There are peaks in the spectral power of the discovery of fossils: 
a \SI{62}{Myr} period corresponds to the period of the oscillation of the Solar system 
with respect to the galactic disk. 

Astroparticle physics takes up a lot of stuff, many fields. 

There's lots of hydrogen and helium in the ISM, 
while almost no Beryllium, Boron and Lithium, 
while up from Carbon are formed in stars. 

On the other hand, in cosmic rays there is a much higher amount of 
Be, B, Li.

The cross-section for \emph{spallation} is \(\sigma \sim 45 A^{2/3} \SI{}{mb}\). 

The motion needs to be ``diffusive''. If this is happening, 
the timescale is quadratic in the distance, and matches with our observations. 

But, it cannot be collisions! 
So, what is it? 

There is Ballmer emission in the shockwave from SNe. 

The Larmor radius at knee-level \(E \sim \SI{3e15}{eV}\) is on the order of \SI{3}{pc}. 

The loss length for radiation decreases with energy: first we can do pair production when
a proton interacts with a CMB photon, and then pion production can start. 

The \emph{goal of the course} is to understand the behaviour of 
charged particles in a magnetic field they generate themselves. 

Diffusive transport  takes into account charged particles, as well 
as ordered and turbulent \(B\) fields, but 
plasma instabilities are what happen when charged particles interact with a 
turbulent magnetic field. 

There are many experiments measuring this kind of stuff! 

The plan is 
\begin{enumerate}
    \item basics of plasma physics;
    \item basics of MHD;
    \item basics of transport of charged particles;
    \item basic aspects of the supernova paradigm for cosmic rays;
    \item test particle theory of diffusive shock acceleration;
    \item cosmic ray transport in the galaxy;
    \item some non-linear aspects of particle transport;
    \item a taste of advanced topics: 
        recent findings, positrons, end of galactic CR, UHECR\dots
\end{enumerate}



\end{document}
