\documentclass[main.tex]{subfiles}
\begin{document}

\marginpar{Monday\\ 2021-11-8, \\ compiled \\ \today}

Pasquale Blasi gives this part of the course. 
He might ask us to do exercises at the blackboard. 

\subsection*{Introduction}

This is a theory course, but the professor wants to have lots of connections to the physical interpretation.

In general, in order to produce high-energy particles we need \emph{acceleration} and \emph{transport} mechanisms. 

The particles we observe are \emph{non-thermal}: 
the spectra from these phenomena seem to be powerlaws, not 
the exponentially suppressed tail of a Maxwellian.

Therefore, the system must be, to a fist, approximation, collisionless: if things happen \emph{faster} than the collisions between the particles, they do not have enough time to thermalize.

We typically see charged particles making up cosmic rays, 
and the (interstellar, intergalactic) medium they move in is mostly ionized. 
Thus, they will move under the action of electromagnetic fields. 
Each of these charged particles also produces EM fields of its own: they all ``feel what the others are doing''.

There is a deep connection between the micro and the macro physics. 
David N.\ Schramm called this an ``inner world --- outer world'' connection. 

In order to understand the behaviour of cosmic rays, we need both particle physics and plasma physics! 

Things to study are: 
\begin{enumerate}
    \item high-energy cosmic rays;
    \item astrophysical \(\gamma \) rays and neutrinos;
    \item \dots
\end{enumerate}
\todo[inline]{What was this list about?}

\subsection*{Electroscope discharge: a brief history of the discovery of cosmic rays}

What follows is a brief history lesson, roughly based on the work of \textcite[]{deangelisAtmosphericIonizationCosmic2014}.

In the late 1700s Coulomb discovered that an electroscope, left on its own, will discharge. 
This phenomenon seemed to decrease in speed as the pressure decreased: the air was being ionized, but by what?

In the late 1800s, after the discovery of the emission of charged particles by radioisotopes (which definitely managed to discharge electroscopes), ambient radioactivity was blamed for the phenomenon as a whole. 

Domenico Pacini explored the variation of the rate of ionization as one moved underwater: it was decreasing. 
So, it seemed likely that the origin was extraterrestrial! 

Later explorations, especially on balloons, showed a slight decrease followed by a sharp increase as altitude went above a couple kilometers.
It took a while for the scientific community to accept this, but the origin of the discharge of electroscopes was not Earth-bound: it was cosmic.
 
Millikan though they might be the ``birth cry'' of elements: \(\gamma \) rays with energies
corresponding to the mass defects of certain nuclides. 
He used Dirac's theory of Compton scattering. 
He was wrong, but he was also the first to use the term ``cosmic rays''.

Bruno Rossi put lead blocks between Geiger counters: these are not \(\gamma \) rays, since
they would be absorbed. 

They have to be charged: their flux is influenced by the Earth's magnetic field, so 
the one from the East and the West is different. 

In the 1930s there started a boom of discoveries. 
At the end of the 30s Auger measured \SI{100}{TeV} cosmic rays. 

At the end of the 60s the CMB was discovered: 
therefore, in the cosmic ray spectrum there should be a cutoff around \SI{e20}{eV}: the \textbf{GZK feature} \cite[sec.\ 5.1]{aloisioSelectedTopicsCosmic2018}. 
This is because the cosmic ray ``sees'' CMB photons as \(\gamma \) rays, so  
it can undergo production of pions, losing a lot of energy, in a process like: 
%
\begin{align}
\ce{N} + \gamma \to \ce{N} + \pi^{0}
\,.
\end{align}

Specifically, the average CMB photon has an energy of \(E \sim \SI{2.7}{K} \approx \SI{245}{\micro \eV}\); in order for it to be blueshifted up to the mass of a pion, \(m_{\pi^{+}} \approx \SI{140}{MeV}\) it needs a \(\gamma \) factor of about \(\gamma \sim \num{6e11}\); this is achieved when a proton has an energy of \(E_p \approx \SI{6e11}{GeV}\). 

The threshold for pair production, \(\sim \SI{1}{MeV}\), is lower: ``only'' \(E_p \approx \SI{4e9}{GeV}\). 
These are not hard thresholds: they are computed from the 
average temperature of CMB photons, but these are distributed according to a thermal distribution, so there is a tail at high energies, albeit exponentially suppressed. 

There are peaks in the spectral power of the discovery of fossils: 
a \SI{62}{Myr} period corresponds to the period of the oscillation of the Solar system 
with respect to the galactic disk. 

% Astroparticle physics takes up a lot of stuff, many fields. 

There's lots of hydrogen and helium in the ISM, 
almost no Beryllium, Boron and Lithium, 
while elements heavier than Carbon, which are formed in stars, are found in decent amounts.

On the other hand, in cosmic rays there is a much higher amount of Be, B, Li.
This is due to \emph{spallation}, or \(x\)-process nucleosynthesis: a process by which a cosmic ray hits a nucleus, thereby splitting it into lighter components. 
The cross-section for spallation is \(\sigma \sim 45 A^{2/3} \SI{}{mb}\), where \(A\) is the mass number of the nuclide. 

The motion needs to be ``diffusive''.
If this is happening, the timescale is quadratic in the distance, and matches with our observations; if the motion
was ``ballistic'' (straight on, basically) the timescale would be linear in the distance travelled.

But, it cannot be collisions, otherwise the distribution would be thermal! So, what is it?
Magnetic fields. 

There is Ballmer emission in the shockwave from SNe. 

The Larmor radius is in general given by 
%
\begin{align}
r _{\text{Larmor}} \approx \frac{\gamma m v_\perp}{\abs{q} \abs{B}}
\,,
\end{align}
%
where the magnitude of the galactic \(B\)-field is of the order of \SI{100}{pT}.

\todo[inline]{Is this a typical value, an underestimate maybe?}

This means that even at knee-level, \(E \sim \SI{3e15}{eV}\), the gyroradius is on the order of \SI{3}{pc}: extremely \emph{small}, in terms of the scale of the galaxy, \SI{10}{kpc} or more.

The loss length for radiation decreases with energy: first we can do pair production when
a proton interacts with a CMB photon, and then pion production can start. 

The \emph{goal of the course} is to understand the behaviour of 
charged particles in a magnetic field they generate themselves. 

Diffusive transport  takes into account charged particles, as well 
as ordered and turbulent \(B\) fields, but 
plasma instabilities are what happen when charged particles interact with a 
turbulent magnetic field. 

There are many experiments measuring this kind of stuff! 

\subsubsection{Course outline}

The plan is to discuss:
\begin{enumerate}
    \item basics of plasma physics;
    \item basics of MHD;
    \item basics of transport of charged particles;
    \item basic aspects of the supernova paradigm for cosmic rays;
    \item test particle theory of diffusive shock acceleration;
    \item cosmic ray transport in the galaxy;
    \item some non-linear aspects of particle transport;
    \item a taste of advanced topics: 
        recent findings, positrons, end of galactic CR, UHECR\dots
\end{enumerate}



\end{document}
