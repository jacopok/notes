\documentclass[main.tex]{subfiles}
\begin{document}

\marginpar{Wednesday\\ 2021-11-10, \\ compiled \\ \today}

Regarding books, the course is a synthesis of several topics, so many books cover them but they also include lots of other material.

For plasma physics, ``Plasma physics for astrophysics'' by Russel M.\ Kulsrud.

For transport, ``Astrophysics of cosmic rays''.

\section{Basics of plasma physics}

From the microphysical point of view, cosmic rays are just electric charges moving in a plasma, which they in turn  affect. 

Loosely, a plasma is ionized gas; however the interstellar medium is also at temperatures of \SI{e4}{K} to \SI{e6}{K}, this is also true for the intergalactic medium, which has a much lower density and similar temperatures, 

This also applies to the medium in clusters, which has a higher temperature, of the order of \SI{e8}{K} but lower densities, \SI{e-3}{cm^{-3}}.

Magnetic fields are sourced by currents, which cosmic rays affect. 
Electric fields, on the other hand, are ``short-circuited'' since the conductivity is very large. 

If it is difficult to have an electric field, how can particles be accelerated? 
We will simply assume that cosmic rays, which are non-thermal, exist. 

First, though, we will try to understand how the plasma works, then we will look at the non-thermal particles, and finally we will put them together.

For simplicity, let us consider a plasma made of protons (with density \(n_i\)) and electrons (with density \(n_e\)). 
Maxwell' equations will read 
%
\begin{subequations}\label{eq:maxwell}
\begin{align}
\vec{\nabla} \cdot \vec{E} &= 4 \pi \zeta  \\
\vec{\nabla} \cdot \vec{B} &= 0  \\
\vec{\nabla} \times \vec{E} &= - \frac{1}{c} \pdv{\vec{B}}{t}  \\
\vec{\nabla} \times \vec{B} &= \frac{4 \pi }{c} \vec{J} + \frac{1}{c} \pdv{\vec{E}}{t}
\,,
\end{align}
\end{subequations}
%
where \(\zeta = n_i e - n_e e\), while the current reads \(\vec{J} = n_i e \vec{v}_i - n_e e \vec{v}_e\). 

The interactions between these will be Coulomb ones. 
At thermal equilibrium, we will have charge neutrality, so \(n_i = n_e = n_0\). 

\paragraph{Screening}

Suppose we add a positive charge in a neutral medium, at \(\vec{r} = 0\). 
Then, the divergence of \(\vec{E}\) will be 
%
\begin{align} \label{eq:electric-field-perturbation}
\vec{\nabla} \cdot \vec{E} = 4 \pi n_i e - 4 \pi n_e e + 4 \pi e \delta (\vec{r})
\,,
\end{align}
%
where the tilde means that the densities are perturbed. 

We will assume that the perturbation induced by this is small. The potential energy drop of a particle at a distance \(d\) from another is \(e^2 / d\), so if we assume \(e^2 / d \ll k_B T\) the thermal background remains fixed, and is not ``broken'' by Coulomb interactions. 
The typical distance between particles will be \(d \sim n_0^{-1/3}\), so the weak perturbation condition will read \(e^2 n_0^{1/3} \ll k_B T\). 

What we want to do now is to solve this equation \eqref{eq:electric-field-perturbation} for the electric field.

In order to do so, we need to discuss the dependence of the phase space distribution on the energy of each particle, which we call \(\epsilon \). 
At zeroth order, we will have 
%
\begin{align}
\widetilde{n}_i = n_0 \exp(- \frac{\epsilon}{k_B T})
\,,
\end{align}
%
which we assume, on mesoscopic scales, to average out to \(\expval{e^{- \epsilon / k_B T}} = 1\).
In the following discussion we apply this mesoscopic approximation by setting \(\epsilon \equiv k_B T\), but we do leave in the external perturbation to the potential. 

The external potential will perturb the energy \(\epsilon \) by \(e \varphi \) where \(\varphi \) is the electric potential, so 
%
\begin{subequations}
\begin{align}
\widetilde{n}_i &= n_0 \exp(\frac{- \epsilon + e \varphi }{k_B T})  \\
\widetilde{n}_e &= n_0 \exp(\frac{- \epsilon - e \varphi }{k_B T})
\,,
\end{align}
\end{subequations}
%
and we can use \(\vec{E} = - \vec{\nabla} \varphi \): then, for \(r \neq 0\) we will have
%
\begin{align}
\vec{\nabla}^2 \varphi = 
- 4 \pi e n_0 \exp(- \frac{\epsilon }{k_B T}) 
\exp( - \frac{e \varphi }{k_B T})
+ 4 \pi e n_0 \exp(- \frac{\epsilon }{k_B T}) \exp( \frac{e \varphi }{k_B T})
\,,
\end{align}
%
which we can expand up to linear order: 
%
\begin{subequations}
\begin{align}
\vec{\nabla}^2 \varphi &= 
- 4 \pi e n_0 \exp(- \frac{\epsilon }{k_B T}) 
\qty( \exp(- \frac{e \varphi }{k_B T}) - \exp(\frac{e \varphi }{k_B T}))  \\
&\approx 
- 4 \pi e n_0 \exp(- \frac{\epsilon }{k_B T}) 
\qty(1 - \frac{e \varphi }{k_BT} - 1 - \frac{e \varphi }{k_B T})  \\
&=
8 \pi e \underbrace{n_0 \exp(- \frac{\epsilon }{k_B T})}_{ = n_0 } 
\frac{e }{k_B T} \varphi
\,.
\end{align}
\end{subequations}

The prefactor had the dimensions of an inverse square length; further, we include the exponential \(e^{- \epsilon / k_BT}\) into the unperturbed density \(n_0 \).
We define the \textbf{Debye length}
%
\begin{align}
\lambda _D = \qty(\frac{k_B T}{8 \pi n_0 e^2})^{1/2}
\,.
\end{align}

The equation reads 
%
\begin{align}
\frac{1}{r^2} \pdv{r} \qty(r^2 \pdv{\varphi }{r}) 
= \frac{1}{\lambda _D^2} \varphi 
\,,
\end{align}
%
which is solved by looking at \(f = r \varphi \): then, 
%
\begin{align}
\dv[2]{f}{r} = \frac{f}{\lambda _D^2}
\,,
\end{align}
%
which means \(f = A \exp(- r / \lambda _D)\) (we discard the unphysical, exponentially diverging solution). 
Inserting back our particle boundary condition to fix \(A\), we get 
%
\begin{align}
\varphi = \frac{e}{r} \exp(- \frac{r}{\lambda _D})
\,.
\end{align}

This is physically meaningful: there is a \textbf{screening effect} on charges, on length scales of \(\lambda _D\). 

For this to happen, however, we need to have enough charges to screen the inserted one: the number of particles in the Debye volume, \(\sim n_0 \lambda _D^3\), must be much larger than 1. 
In a way, this is also a mesoscopic statistical requirement. 

This can be written as 
%
\begin{align}
n_0 \frac{(k_BT)^{3/2}}{( 8 \pi e^2)^{3/2} n_0^{3/2}} 
= \frac{(k_BT)^{3/2}}{( 8 \pi e^2)^{3/2} n_0^{1/2}} 
\gg 1
\,,
\end{align}
%
which shows that, counter-intuitively, this condition is easier to fulfill for under-dense plasmas. 

The path length for Coulomb scattering, \(\lambda _C\), should be much larger than both \(\Delta r \approx n^{-1/3}\) (the separation between particles) and \(\lambda _D\). 

It can be estimated by 
%
\begin{align}
\lambda _C = \frac{1}{n_0 \sigma _C} = \frac{(k_B T)^2}{n_0 e^{4}} \gg n^{-1/3} 
\,,
\end{align}
%
where \(e^2 / b = k_B T\) gives us a limit for the Coulomb interaction length, therefore \(\sigma _C \approx b^2 \approx (e^2 / k_B T)^2\). 


It can be shown that \(\lambda _C \gg \lambda _D\) is equivalent to the condition that many particles should be contained in a single Debye length: the condition reads 
%
\begin{align}
\lambda _C = \frac{(k_B T)^2}{n_0 e^{4}} &\gg \sqrt{\frac{k_B T}{8 \pi n_0 e^2}} = \lambda _D   \\
(8 \pi)^2 n_0 \qty(\frac{k_B T}{8 \pi n_0 e^2})^2 &\gg  \qty( \frac{k_BT}{ 8 \pi n_0 e^2})^{1/2}  \\
\lambda _D^3 &\gg \frac{1}{(8 \pi )^2 n_0 }
\,.
\end{align}

Collective effects dominate the dynamics of a plasma, while the effect of a single charge quickly becomes negligible.

\paragraph{Propagation modes in a plasma}

We will now do an exercise in perturbation theory: which perturbations are allowed, beyond electromagnetic waves? 
Certain modes are ``allowed'', in that they do not die out. 

A lot of interesting physics come from the fact that each particle interacts with the collection of all the others. 

Perturbations which are unstable are interesting, but they break our perturbative approach. 

The fields in Maxwell's equations \eqref{eq:maxwell} are assumed to start out at zero in the unperturbed configuration. 

The current \(J\) is given by the Generalized Ohm's law: 
%
\begin{align}
J_r = \sigma_{rs} E_s 
\,,
\end{align}
%
where the proportionality constant \(\sigma _{rs}\) is the \emph{conductivity tensor} (whose components are, roughly, 1 over resistance). 

It is also convenient to define the displacement current \(\vec{D}\) by:
%
\begin{align}
\frac{4 \pi \vec{J}}{c} + \frac{1}{c} \pdv{\vec{E}}{t} = \frac{1}{c} \pdv{\vec{D}}{t}
\,,
\end{align}
%
which means that the current can be recovered by 
%
\begin{align}
\vec{J} = \frac{1}{4 \pi } \qty(\pdv{\vec{D}}{t} + \pdv{\vec{E}}{t})
\,.
\end{align}

The idea is to decompose the perturbation in Fourier modes:
%
\begin{align}
 \vec{E} \to \widetilde{E}_k (\vec{k}, \omega ) \exp(- i \omega t + i \vec{k} \cdot \vec{r})
\,.
\end{align}

The full expression for the electric field is a superposition of these modes, but we can look at them just one at a time. 
This simplifies things: time derivatives become \(i \omega \), curls become \(\vec{k} \times \) and so on. 

The current reads, with this as well as Ohm's law,
%
\begin{align}
\widetilde{J} _r = \frac{- i \omega }{4 \pi } \qty(\widetilde{D}_r - \widetilde{E}_r) = \sigma_{rs} E_s
\,.
\end{align}

The displacement field is therefore 
%
\begin{align}
\widetilde{D}_r
= \frac{4 \pi }{ i \omega } \qty(\frac{i \omega }{ 4 \pi } \widetilde{E}_r - \sigma_{rs} E_s) = \widetilde{E}_r
+ i \frac{4 \pi }{\omega } \sigma_{rs} \widetilde{E}_s 
= \mathbb{K}_{rs} \widetilde{E}_s
\,,
\end{align}
%
where \(\mathbb{K}_{rs} = \delta_{rs} + (4 \pi i / \omega ) \sigma_{rs}\) is called the \textbf{Dielectric tensor}.

What we are trying to do is to write a \emph{dispersion relation}, \(F(\vec{k}, \omega ) E = 0\). 

Now we move to Lenz's law: we take another curl, to get 
%
\begin{subequations}
\begin{align}
\vec{\nabla} \times (\vec{\nabla} \times \vec{E}) &= - \frac{1}{c} \pdv{}{t} \qty(\vec{\nabla} \times \vec{B})   \\
&= - \frac{1}{c} \pdv{}{t} \qty( \frac{1}{c}\pdv{\vec{D}}{t})  \\
&= - \frac{1}{c^2} \pdv[2]{\vec{D}}{t}  \\
[- \vec{k} \times (\vec{k} \times \widetilde{E})]_r &= + \frac{\omega^2}{c^2} \widetilde{D}_r = \frac{\omega^2}{c^2} \mathbb{K}_{rs} \widetilde{E}_s
\,.
\end{align}
\end{subequations}

We are almost done: the only issue here is that \(\mathbb{K}_{rs}\) contains the conductivity tensor \(\sigma_{rs}\). 

Supposing for simplicity  that our plasma is non-relativistic, we have the EoM 
%
\begin{align}
m_e \dv{v_e}{t} = - e E \implies 
- i \omega m_e \widetilde{v}_e = - e \widetilde{E}
\,,
\end{align}
%
but the current \(\vec{J}\) is \(\vec{J} = - e n_e \vec{v}_e\), so \(\widetilde{J}_r = - (n e^2 / \omega m_e) \widetilde{E}_r\), therefore the conductivity reads 
%
\begin{align}
\sigma_{rs} = \frac{i n e^2}{\omega m_e} \delta_{rs} 
\,,
\end{align}
%
so we can write out the dielectric tensor explicitly: 
%
\begin{align}
\mathbb{K}_{rs} = \delta_{rs} \qty[ 1 - \qty(\frac{\omega_p}{\omega })^2]
\qquad \text{where} \qquad
\omega_p = \sqrt{\frac{ 4 \pi n_e e^2}{m_e}}
\,
\end{align}
%
is called the plasma frequency.
% This is a \textbf{high-pass filter}!

We have found our dispersion relation: 
%
\begin{align}
\vec{k} \times \qty(\vec{k} \times \vec{E})
+ \qty(\frac{\omega^2 - \omega _p^2}{c^2}) \widetilde{E} = 0
\,.
\end{align}

Let us separate out \emph{longitudinal perturbations}, which have \(\vec{k} \propto \vec{E}\), from \emph{transverse} ones, since we are always able to write \(\vec{E} = \vec{E}_\parallel + \vec{E}_\perp\). 

In the \textbf{longitudinal case},
%
\begin{align}
(\omega^2 - \omega _p^2) E_{\parallel} = 0
\,,
\end{align}
%
which means that the propagation must happen exactly at the plasma frequency: these are called \textbf{Langmir waves}, or plasma waves. 
In the transverse case, we get 
%
\begin{align}
\qty[ - k^2 c^2 + (\omega^2 - \omega _p^2)] E_\perp = 0
\,.
\end{align}

The pulsation must be 
%
\begin{align}
\omega^2 = \omega _p^2 + c^2 k^2
\qquad \text{or} \qquad
k^2 = \frac{\omega^2 - \omega^2_p}{c^2}
\,.
\end{align}

If \(\omega > \omega _p\), then \(k^2 > 0 \): these are allowed perturbations, which indeed exhibit oscillatory behavior.

Actually, we can compute their group velocity 
%
\begin{align}
v_g = \pdv{\omega }{k} = c \qty(1 - \frac{\omega_p^2}{\omega^2})^{1/2}
\,,
\end{align}
%
which shows that if \(\omega \gg \omega _p\) the speed is close to \(c\). 

If \(\omega < \omega _p\), on the other hand, \(k\) is imaginary.
The solution corresponding to exponential growth is unphysical (for one, there is no mechanism providing the energy for the magnitude of the oscillation to exponentially grow), so we only look at the exponentially damped solutions. 

They are exponentially damped over a scale \(\abs{\vec{k}}^{-1} = c / \omega _p\).
This is the \textbf{skin depth} of the plasma, the largest distance a perturbation can penetrate in the plasma if it oscillates too slowly.

This characteristic wavelength's ratio to the Debye length reads 
%
\begin{align}
\frac{\lambda _{\text{skin depth}}}{ \lambda _{\text{Debye}}} 
= \sqrt{\frac{m_e c^2}{4 \pi n_e e^2} \frac{8 \pi n_e e^2}{k_B T}}
= \sqrt{\frac{2 m_e c^2}{k_B T}} \gg 1
\,,
\end{align}
%
since the plasmas we are considering are typically non-relativistic. 

\paragraph{An example: fast radio bursts}

Fast radio bursts are like \(\gamma \)-ray bursts in the radio band.
These are relatively high-frequency, around \SI{1.4}{GHz}. 
We do not really know what their sources are; some were false positives due to microwave ovens making lunch, but others were certified to be true detections.

We know of one which came from a galaxy \(L \sim \SI{1}{Gpc}\) away, and which had a dispersion of about \(\Delta \omega \sim \SI{300}{MHz}\). 

Photons of different frequencies arrived at different times, with a spread of about \SI{300}{ms}. 

We are in the second case (\(\vec{k} \perp \vec{E}\)) since they are EM waves. 
The group velocity is, again
%
\begin{align}
v_g = c \qty[ 1 - \qty(\frac{\omega _p}{\omega})^2]^{1/2}
\,,
\end{align}
%
and we can assume (we will check \emph{a posteriori}) that \(\omega \gg \omega _p\). 

The time difference will be 
%
\begin{subequations}
\begin{align}
\Delta t &= \frac{L}{v_g (\omega)} - \frac{L}{v_g (\omega + \Delta \omega )} 
\approx \frac{L}{v_g(\omega )} \frac{1}{v_g(\omega )}\pdv{v_g}{\omega } \Delta \omega  \\
&\approx \frac{L}{c^2} \underbrace{c \frac{\omega _p^2}{\omega^3}}_{\approx \pdv*{v_g}{\omega }} \Delta \omega \\
\omega _p &\approx \sqrt{\frac{ c \omega^3 \Delta t}{L \Delta \omega }}
\,,
\end{align}
\end{subequations}
%
but we know that 
%
\begin{align}
\omega _p^2 = \frac{ 4 \pi n_e e^2}{m_e}
\,.
\end{align}

This allows us to measure \(\omega _p \sim \SI{5.2}{Hz}\), and therefore also the density, which comes out to be \(n_e \sim \SI{8.2e-09}{cm^{-3}}\).
% This corresponds with the \(\Omega _b\) from CMB observations (?)

How well does this match the baryon density computed from the CMB? 
The computation goes  
%
\begin{align}
\underbrace{\underbrace{\frac{\omega _p^2 m_e}{4 \pi e^2}}_{n_e} m_p}_{\rho _{\text{plasma}}} \times \frac{1}{\Omega _{0b} \rho _c} \approx 0.03
\,.
\end{align}

Therefore, from this observation we can estimate that about \SI{3}{\percent} of baryonic matter is in the ISM. 

This can be generalized a bit: maybe, not all the path \(L\) from the source to here had this much plasma in it. 
Suppose, for simplicity's sake, that a fraction \(\alpha \) of the path was constituted by uniform-density plasma. 

Then, our estimate for \(\omega _p\) will be multiplied by a factor \(\alpha^{-1/2}\), while our estimate for \(n_e\) will be multiplied by \(\alpha^{-1}\). 
On the other hand, in the estimate for \(\rho _{\text{plasm}}\), averaged over the whole universe, will need to be shifted by some factor. 

If the path we were looking at is a fair sample for the population (not a given, but let's approximate as such), then about a fraction \(\alpha \) of the universe is filled with this plasma --- this precisely cancels the correction to our estimate, so it all works out. 



\end{document}