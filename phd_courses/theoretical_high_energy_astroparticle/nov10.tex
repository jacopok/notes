\documentclass[main.tex]{subfiles}
\begin{document}

\marginpar{Wednesday\\ 2021-11-10, \\ compiled \\ \today}

Regarding books, the course is a synthesis of several topics, so many books cover them but they also include lots of other material.

For plasma physics, ``Plasma physics for astrophysics'' by Russel Co...,
a bit Shukuchicken ``Kinetic theory...?''

For transport, ``Astrophysics of cosmic rays..''

\section{Basics of plasma physics}

From the microphysical point of view, cosmic rays are just electric charges moving in a medium. 
They, however, interact with the medium. 
More interesting, this medium is a plasma. 

Loosely, a plasma is ionized gas; however the interstellar medium is also at temperatures of \SI{e4}{K} to \SI{e6}{K}, this is also true for the intergalactic medium, which has a much lower density and similar temperatures, 

\todo[inline]{and this also applies to the medium in clusters, which has a higher temperature}

Magnetic fields are sourced by currents, which cosmic rays affect. 
Electric fields, on the other hand, are ``short-circuited'' since the conductivity is very large. 

If it is difficult to have an electric field, how can particles be accelerated? 

For now, we will assume that cosmic rays, which are non-thermal, exist. 

First, we will try to understand how the plasma works, then we will look at the non-thermal particles, and finally we will put them together. 

For simplicity, let us consider a plasma made of protons (with density \(n_i\)) and electrons (with density \(n_e\)). 
Maxwell' equations will read 
%
\begin{align}
\vec{\nabla} \cdot \vec{E} &= 4 \pi \zeta  \\
\vec{\nabla} \cdot \vec{B} &= 0  \\
\vec{\nabla} \times \vec{E} &= - \frac{1}{c} \pdv{\vec{B}}{t}  \\
\vec{\nabla} \times \vec{B} &= \frac{4 \pi }{c} \vec{J} + \frac{1}{c} \pdv{\vec{E}}{t}
\,,
\end{align}
%
where \(\zeta = n_i e - n_e e\), while the current reads \(\vec{J} = n_i e \vec{v}_i - n_e e \vec{v}_e\). 

The interactions between these will be Coulomb ones. 
At thermal equilibrium, we will have charge neutrality, so \(n_i = n_e = n_0\). 

Suppose we add a positive charge in a neutral medium, at \(\vec{r} = 0\). 
Then, the divergence of \(\vec{E}\) will be 
%
\begin{align} \label{eq:electric-field-perturbation}
\vec{\nabla} \cdot \vec{E} = 4 \pi \vec{n}_i e - 4 \pi \vec{n}_e e + 4 \pi e \delta (\vec{r})
\,,
\end{align}
%
where the tilde means that the densities are perturbed. 

We will assume that the perturbation induced by this is small. The change in the energy of a particle is \(e^2 / d\), so if we assume \(e^2 / d \ll k_B T\) the thermal background remains fixed. 
The typical distance between particles will be \(d \sim n_0^{-1/3}\), so the weak perturbation condition will read \(e^2 n_0^{1/3} \ll k_B T\). 

What we want to do now is to solve \eqref{eq:electric-field-perturbation}. 
At zeroth order, we will have 
%
\begin{align}
\widetilde{n}_i = n_0 \exp(- \frac{\epsilon}{k_B T})
\,,
\end{align}
%
but the external potential will perturb the energy \(\epsilon \) by \(e \varphi \) where \(\varphi \) is the electric potential, so 
%
\begin{align}
\widetilde{n}_i &= n_0 \exp(\frac{- \epsilon + e \varphi }{k_B T})  \\
\widetilde{n}_e &= n_0 \exp(\frac{- \epsilon - e \varphi }{k_B T})
\,,
\end{align}
%
and we can use \(\vec{E} = - \vec{\nabla} \varphi \): then, for \(r \neq 0\) we will have
%
\begin{align}
\vec{\nabla}^2 \varphi = 
- 4 \pi e n_0 \exp(- \frac{\epsilon }{k_B T}) 
\exp( - \frac{e \varphi }{k_B T})
+ 4 \pi e n_0 \exp(- \frac{\epsilon }{k_B T}) \exp( \frac{e \varphi }{k_B T})
\,,
\end{align}
%
which we can expand up to linear order: 
%
\begin{align}
\vec{\nabla}^2 \varphi &= 
- 4 \pi e n_0 \exp(- \frac{\epsilon }{k_B T}) 
\qty( \exp(- \frac{e \varphi }{k_B T}) - \exp(\frac{e \varphi }{k_B T}))  \\
&= 
- 4 \pi e n_0 \exp(- \frac{\epsilon }{k_B T}) 
\qty(1 - \frac{e \varphi }{k_BT} - 1 - \frac{e \varphi }{k_B T})  \\
&=
8 \pi e \underbrace{n_0 \exp(- \frac{\epsilon }{k_B T})}_{ \to n_0 } 
\frac{e }{k_B T} \varphi
\,.
\end{align}

The prefactor had the dimensions of an inverse square length; further, we include the exponential \(e^{- \epsilon / k_BT}\) into the unperturbed density \(n_0 \).
We define the \textbf{Debye length}
%
\begin{align}
\lambda _D = \qty(\frac{k_B T}{8 \pi n_0 e^2})^{1/2}
\,.
\end{align}

The equation reads 
%
\begin{align}
\frac{1}{r^2} \pdv{r} \qty(r^2 \pdv{\varphi }{r}) 
= \frac{1}{\lambda _D^2} \varphi 
\,,
\end{align}
%
which is solved by looking at \(f = r \varphi \): then, 
%
\begin{align}
\dv[2]{f}{r} = \frac{f}{\lambda _D^2}
\,,
\end{align}
%
which means \(f = A \exp(- r / \lambda _D)\). 
Inserting back our particle boundary condition to fix \(A\), we get 
%
\begin{align}
\varphi = \frac{e}{r} \exp(- \frac{r}{\lambda _D})
\,.
\end{align}

This is physically meaningful: there is a \textbf{screening effect} on charges, on length scales of \(\lambda _D\). 

For this to happen, however, we need to have enough charges to screen the inserted one: the number of particles in the Debye volume, \(\sim n_0 \lambda _D^3\), must be much larger than 1. 
In a way, this is also a mesoscopic statistical requirement. 

This can be written as 
%
\begin{align}
n_0 \frac{(k_BT)^{3/2}}{( 8 \pi e^2)^{3/2} n_0^{3/2}} \propto n_0^{-1/2} \gg 1
\,,
\end{align}
%
which shows that, counter-intuitively, this condition is easier to fulfill for under-dense plasmas. 

The path length for Coulomb scattering, \(\lambda _C\), should be much larger than both \(\Delta r\) (the separation between particles) and \(\lambda _D\). 

It can be estimated by 
%
\begin{align}
\lambda _C = \frac{1}{n_0 \sigma _C} = \frac{(k_B T)^2}{n_0 e^{4}} \gg n^{-1/3} 
\,,
\end{align}
%
where \(e^2 / b = k_B T\) gives us a limit, therefore \(\sigma _C = b^2 \approx (e^2 / k_B T)^2\). 

\todo[inline]{
It can be shown that \(\lambda _C \gg \lambda _D\) is equivalent to the condition that many particles should be contained in a single Debye length.}

Collective effects dominate the dynamics of a plasma. 

We will now do an exercise in perturbation theory: which perturbations are allowed, beyond electromagnetic waves? 
Certain modes are ``allowed'' in that they do not die out. 

\end{document}
