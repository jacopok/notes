\documentclass[main.tex]{subfiles}
\begin{document}

\subsection{Statistical descriptions of plasmas}

\marginpar{Thursday\\ 2021-11-18}

If the length scales we consider are larger than the Debye length, we can safely neglect the effect of the Coulomb potential of each individual particle. 
This does not mean that there are no electric fields, but the electric fields are mesoscopic or larger. 

This is ``desirable'', in that we'd like to use statistical descriptions of the plasma.
Such a statistical description will necessarily work in phase space. 

Let us start out in the nonrelativistic approximation: if we only have one particle, we can write its phase space distribution function as 
%
\begin{align}
N(\vec{x}, \vec{v}, t) = \delta (\vec{x} - \vec{X}(t)) \delta (\vec{v} - \vec{\dot{X}}(t))
\,.
\end{align}

If we have several particles, this can be readily generalized: 
%
\begin{align}
N(\vec{x}, \vec{v}, t) =
\sum _{i} \delta (\vec{x} - \vec{X}_i(t)) \delta (\vec{v} - \vec{\dot{X}}_i(t))
\,,
\end{align}
%
but this only describes a single particle species: we know that at the very least we will have electrons and protons, in order to preserve charge neutrality. 
So, let us call the quantity defined above \(N_s\), where \(s\) is an index spanning \(\qty{e, i}\) for electrons and ions respectively. 

We want to describe the evolution of this quantity: its \textbf{total} time derivative can be computed through the chain rule, and if we set it to zero we find that \(\dv*{N_s}{t} = 0\) is equivalent to:
%
\begin{align}
\pdv{N_s}{t} (\vec{x}, \vec{v}, t) = 
- \sum _{i} \nabla_x \delta (\vec{x} - \vec{X}_i(t)) \vec{\dot{X}} \delta (\vec{v} - \vec{V}_i (t) ) 
- \sum _{i} \delta (\vec{x} - \vec{X}_i(t)) \nabla_v \delta (\vec{v} - \vec{V}_i(t)) \vec{\dot{V}}_i(t)
\,.
\end{align}

Now, these charges will evolve under the actions of the electric and magnetic fields: but we also need to describe what is the source of these fields. 

The action of these fields will be described by the Lorentz force, 
%
\begin{align}
m_s \vec{\dot{V}}_i (t) = q_s \vec{E}^{\text{microscopic}} (\vec{x}) + \frac{q_s}{c} \vec{v}_i \times \vec{B}^{\text{microscopic}} (\vec{x}, t)
\,.
\end{align}

This acceleration term will then be put into the aforementioned evolution equation. 
We can already understand why this problem will be hard: the fields acting on a particle will be sourced by all the others. 

The microscopic EM fields will satisfy Maxwell's equations: 
%
\begin{align}
\vec{\nabla} \cdot \vec{E}^{\text{micro}} (\vec{x}, t) 
&= 4 \pi \zeta^{\text{micro}}  \\
\vec{\nabla} \cdot \vec{B}^{\text{micro}} &= 0  \\
\vec{\nabla} \times E^{\text{micro}} &= - \frac{1}{c} \pdv{\vec{B}^{\text{micro}}}{t}  \\
\vec{\nabla} \times \vec{B}^{\text{micro}} &= \frac{4 \pi }{c} \vec{J}^{\text{micro}} + \frac{1}{c} \pdv{\vec{E}^{\text{micro}}}{t} 
\,,
\end{align}
%
where the density and current density read 
%
\begin{align}
\zeta^{\text{micro}} (\vec{x}, t) &= \sum _{s=i, e} q_s 
\int \dd[3]{\vec{v}} N_s (\vec{x}, \vec{v}, t)  \\
\vec{J}^{\text{micro}} &= \sum _{s=i, e} q_s 
\int \dd[3]{\vec{v}} \vec{v} N_s (\vec{x}, \vec{v}, t)
\,.
\end{align}

The Boltzmann equation plus the Lorentz one can be more compactly written as 
%
\begin{align}
\pdv{N_s}{t} 
= - \vec{v} \cdot \vec{\nabla}_x N_s 
- \sum _{i=1}^{N} \frac{q_s}{m_s} 
\qty[ \vec{E}^{\text{micro}} + \frac{1}{c} \vec{v} \times \vec{B}^{\text{micro}}] \cdot \vec{\nabla}_v \qty[\delta (\vec{x} - \vec{X}_i(t)) \delta (\vec{v} - \vec{\dot{X}}_i(t))]
\,,
\end{align}
%
where we have done a nontrivial step: the gradient \dots

\todo[inline]{That's wrong! Or maybe not? it's because the velocity dependence is mediated through the delta-function\dots }

The equation then becomes 
%
\begin{align}
\pdv{N_s}{t} + \vec{v} \cdot \vec{\nabla}_x N_s 
&= - \frac{q_s}{m_s} \qty[\vec{E}^{\text{micro}} + \frac{\vec{v} \times \vec{B}^{\text{micro}}}{c}] \cdot \nabla_v N_s
\,.
\end{align}

This equation is called the Klimontovich-Dupree equation.\footnote{As many things done during the Cold War, it was developed by military personnel independently in the two blocks. } 

This equation, however, is basically useless, unless we do a mean-field approximation. 

Let us introduce the quantity \(f_s (\vec{x}, \vec{v}, t)\), which we want to compute through average on mesoscopic scales: 
%
\begin{align}
f_s (\vec{x}, \vec{v}, t) = \expval{N_s (\vec{x}, \vec{v}, t)}_{\Delta V}
\,.
\end{align}

The true phase space density will then be 
%
\begin{align}
N_s (\vec{x}, \vec{v}, t) = f_s (\vec{x}, \vec{v}, t) + \delta f_s
\,,
\end{align}
%
where the fluctuations are assumed to average to zero. 
We can write a similar expression for the electric and magnetic fields: 
%
\begin{align}
\vec{E}^{\text{micro}} &= \vec{E} + \delta \vec{E} \\
\vec{B}^{\text{micro}} &= \vec{B} + \delta \vec{B}
\,.
\end{align}

The averaged KD equation then becomes: 
%
\begin{align}
\dv{f_s}{t} + \vec{v} \cdot \nabla_x f_s + \frac{q_s}{m_s} 
\qty[\vec{E} + \frac{1}{c} \vec{v} \times \vec{B}] \vec{\nabla}_v f_s = - \frac{q_s}{m_s} \expval{
    \qty( \delta \vec{E} + \frac{1}{c} \vec{v} \times \delta \vec{B} ) \cdot \vec{\nabla}_v \delta f_s
}
\,.
\end{align}

The interaction and collision terms are \emph{quadratic} in the fluctuations. 
If we neglect this term (which is typically called a ``correlation'' term), we get the \textbf{Vlasov} equation: 
%
\begin{align}
\pdv{f_s}{t} + \vec{v} \cdot \vec{\nabla}_x f_s + \frac{q_s}{m_s} \qty[ \vec{E} + \frac{\vec{v} \times \vec{B}}{c}] \vec{\nabla}_v f_s = 0
\,.
\end{align}

We can do the same thing to the Maxwell equations, using an averaged version of the charge and current densities.

We will need to generalize to the relativistic case: we move to \((\vec{x}, \vec{p})\) phase space. 
For relativistic particles, the Lorentz force reads 
%
\begin{align}
\vec{\dot{p}} = q_s \qty[\vec{E} + \frac{\vec{v} \times \vec{B}}{c}]
\,.
\end{align}

The source terms in the Maxwell equations will be integrated in \(\dd[3]{p}\), but for the charge current we will have an integral \(\int \dd[3]{p} \vec{v} f_s\). 

The relativistic Vlasov equation is then readily derived with minor modifications, and reads 
%
\begin{align}
\pdv{f_s}{t} + \vec{v} \cdot \vec{\nabla}_x f_s + q_s \qty[ \vec{E} + \frac{\vec{v} \times \vec{B}}{c}] \vec{\nabla}_p f_s = 0 
\,.
\end{align}

Let us now move to the nonrelativistic plasma again, and assume we are at zero temperature. 
We will then make a small perturbation: the ions will be stationary in first approximation. 
Will the Vlasov equation contain the plasma waves we derived earlier? 

We only write it for electrons, so 
%
\begin{align}
\pdv{f}{t} + \vec{v} \cdot \vec{\nabla}_x f_s 
- \frac{e}{m_e} \qty[\vec{E} + \frac{\vec{v} \times \vec{B}}{c}]_\alpha \pdv{f}{v_\alpha } = 0
\,,
\end{align}
%
where in the second term we are adopting the Einstein convention, summing over \(\alpha \). 

We will assume that there is no magnetic field perturbation: this is the same thing we did in the plasma waves, specifically in assuming that \(\mathbb{K}\) is diagonal.

\todo[inline]{is this correct?}

In the stationary configuration there is \(\vec{E} = 0\). 
To linear order, the perturbed equation will read 
%
\begin{align}
\pdv{ \delta f}{t} + \vec{v} \cdot \nabla \delta f + \frac{e}{m_e} \nabla \varphi _\alpha \pdv{f}{v_\alpha } = 0
\,.
\end{align}

\todo[inline]{But isn't \(f = \text{const}\) at zeroth order?}
We want to assume that the plasma is cold! 

The electric potential \(\varphi \) will satisfy 
%
\begin{align}
- \nabla^2 \varphi = - 4 \pi e \int \dd[3]{v} \delta f
\,.
\end{align}

We then move to Fourier space: 
%
\begin{align}
- i \omega \widetilde{\delta f} + 
i \vec{k} \cdot \vec{v} \widetilde{\delta f} 
+ \frac{e}{m_e} i k_\alpha  \widetilde{\varphi} \pdv{f}{v_\alpha } = 0
\,,
\end{align}
%
but the electric potential will satisfy 
%
\begin{align}
k^2 \widetilde{\varphi} = - 4 \pi e \int \dd[3]{v} \widetilde{\delta f} 
\,,
\end{align}
%
so we get 
%
\begin{align}
\widetilde{\delta f} \qty[ i \vec{k} \cdot \vec{v} - i \omega ] &= 
- \frac{e}{m_e} \widetilde{\varphi} k_\alpha \pdv{f}{v_\alpha }  \\
\widetilde{\delta f} &= - \frac{e}{m_e} \widetilde{\varphi} \frac{k_\alpha \pdv*{f}{v_\alpha }}{\vec{k} \cdot \vec{v} - \omega }  
\,,
\end{align}
%
which we substitute into the integral for \(\widetilde{\varphi}\): 
%
\begin{align}
\widetilde{\varphi} = \frac{4 \pi e^2}{k^2m_e} \widetilde{\varphi} k_\alpha \int \dd[3]{v}  \pdv{f}{v_\alpha } \frac{1}{\vec{k} \cdot \vec{v} - \omega }
\,,
\end{align}
%
so the allowed perturbations are those which make this equation true (for arbitrary \(\varphi \)). 
For now we have not assumed that the electrons are cold: this will enter in how we write the unperturbed \(f\). 

The assumptions of the electrons being cold can be modelled as \(f = n_0 \delta (\vec{v})\): therefore, we need to integrate by parts. 

Suppose that the \(z\) axis is along \(k\): then, the integrand reads 
%
\begin{align}
\int \dd{v_x} \dd{v_y} \dd{v_z} \pdv{f}{v_\alpha } \frac{1}{k v_z - \omega } &=
\int \dd{v_x} \dd{v_y} \dd{v_z} f \frac{1}{(k v_z - \omega)^2 }  \\
&= \int \dd{v_x} \dd{v_y} \delta (v_x) \delta (v_y) n_0 \frac{k}{\omega^2} = \frac{n_0 k }{ \omega^2}
\,,
\end{align}
%
so the equation just reads 
%
\begin{align}
1 - \frac{4 \pi e^2 n_0 }{k^2 m_e} \frac{k^2}{\omega^2} \implies \omega^2 = \frac{4 \pi e^2 n_0 }{m_e} = \omega_p^2
\,.
\end{align}

At the very least, this more complicated approach allows us to recover the results we expected. 

% \subsection{}
\section{Alfvén waves}

Next time, we will add one complication: a global, ordered magnetic field. 

We know that this happens, for example, in spiral galaxies like our own: we observe large-scale magnetic fields.  

Much of the physics of the transport of non-thermal particles will be affected by these magnetic fields. 

The perturbation of the two coupled Vlasov equations under the effect of this external \(\vec{B}\) field will yield what are called \textbf{Alfvén waves}.

We will also find a simplification of the 

\end{document}
