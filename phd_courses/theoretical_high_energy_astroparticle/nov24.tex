\documentclass[main.tex]{subfiles}
\begin{document}

\marginpar{Wednesday\\ 2021-11-24}

Each of the particle species in our plasma satisfy a Vlasov equation, 
%
\begin{align}
\pdv{f_\alpha }{t} + \vec{v} \cdot \nabla_x f_\alpha + q_\alpha \qty[ \vec{E} + \frac{\vec{v}}{c} \times \vec{B}]_\beta  \pdv{f_\alpha }{p_\beta } = 0
\,,
\end{align}
%
where \(\alpha \) runs over electrons, ions, cosmic rays and so on: any collisionless particles; while the index \(\beta \) is a spatial index, to be interpreted in the Einstein summation convention.

The \(\vec{E}\) and \(\vec{B}\) fields do \emph{not} have an index \(\alpha\): they are produced by all the components of the plasma. 

An astrophysically significant situation is one in which there is a ``baseline'' magnetic field, which adds to the one generated within the plasma. 

We align our axes so that \(\vec{B}_0 = B_0  \hat{z}\). 
The baseline electric field is \(\vec{E}_0 = 0\). 

\begin{enumerate}
    \item We ask that \(\vec{B}_0\) is ordered: its variations should be on much larger scales than the scales of the plasma. 
    \item The equilibrium phase space density, \(f_{0, \alpha }\), is assumed to be known: it could, for example, be a Maxwellian, or a delta-function on \(\vec{v} = 0\). 
\end{enumerate}

Again, we will perturb this system in order to find a dispersion relation \(F(k, \omega ) \delta = 0\), where \(\delta \) is our perturbation.

\paragraph{Perturbing the Vlasov equation}

The perturbed components in our fields will read 
%
\begin{align}
f_\alpha &= f_{0, \alpha } + \delta f_\alpha  \\
\vec{B} &= \vec{B}_0 + \delta \vec{B}  \\
\vec{E} &= \delta \vec{E}
\,.
\end{align}

The perturbed equation, to first order, reads 
%
\begin{align}
\pdv{ \delta f_\alpha }{t} + \vec{v} \cdot \vec{\nabla} \delta f_\alpha + q_\alpha \delta \vec{E}_\beta \pdv{f_{0 \alpha }}{p_\beta } + q_\alpha \qty( \frac{\vec{v}}{c} \times \vec{B}_0)_\beta \pdv{ \delta f_\alpha }{p_\beta } 
+ q_\alpha \qty( \frac{\vec{v}}{c} \times \delta \vec{B}) \pdv{f_{0 \alpha }}{t} = 0
\,,
\end{align}
%
and, as usual, we will move to Fourier space, without tildes for simplicity: 
%
\begin{align} \label{eq:perturbed-fourier-space-vlasov}
- i \omega \delta f_\alpha 
+ i \vec{k} \cdot \vec{v} \delta f_\alpha 
+ \hlc{violet}{q_\alpha \delta \vec{E}_\beta \pdv{f_{0 \alpha }}{p_\beta }} 
+ \hlc{orange}{q_\alpha \qty( \frac{\vec{v}}{c} \times \vec{B}_0)_\beta \pdv{ \delta f_\alpha }{p_\beta }}
+ \hlc{yellow}{q_\alpha \qty( \frac{\vec{v}}{c} \times \delta \vec{B}) 
\pdv{f_{0 \alpha }}{p_\beta }} = 0
\,.
\end{align}

\paragraph{Perturbed Maxwell equations}

The divergence of the electric field perturbation reads 
%
\begin{align}
\vec{\nabla} \cdot \delta \vec{E} = 4 \pi \zeta = 4 \pi \sum _{\alpha } q_\alpha \int \dd[3]{p} \delta f_\alpha 
\,,
\end{align}
%
where \(\zeta \) denotes the charge density (which is purely a perturbation, since its unperturbed value is zero due to charge neutrality).

The divergence of \(\vec{B}\) vanishes, but so does the divergence of the constant \(\vec{B}_0\), so we have \(\vec{\nabla} \cdot \delta \vec{B} = 0\) as well. 

The other two equations read 
%
\begin{align}
\vec{\nabla} \times \delta \vec{E} &= - \frac{1}{c} \pdv{ \delta \vec{B} }{t}  \\
\vec{\nabla} \times \delta \vec{B} &= \frac{4 \pi \vec{J}}{c} + \frac{1}{c} \pdv{ \delta \vec{E}}{t}
\,.
\end{align}

In Fourier space, these read 
%
\begin{align}
i \vec{k} \cdot \delta \vec{E} &= 4 \pi q_\alpha \int \dd[3]{p} \delta f_\alpha  \\
i \vec{k} \cdot \delta \vec{B} &= 0  \\
i \vec{k} \times \delta \vec{E} &= \frac{i \omega }{c} \delta \vec{B}  \\
i \vec{k} \times \delta \vec{B} &= \frac{4 \pi }{c} \sum _{\alpha } q_\alpha \int \dd[3]{p} \vec{v} \delta f_\alpha - \frac{i \omega }{c} \delta \vec{E}
\,.
\end{align}

We are now in a position to make a further simplifying assumption.

\begin{enumerate}[resume]
    \item We restrict ourselves to perturbations moving parallel to the magnetic field, \(\vec{k} \parallel \vec{B}_0\). 
\end{enumerate}

Boron, Beryllium and Lithium are absent in the ISM but present in cosmic rays: there must be spallation, but for them to have significant interaction chances they must move very slowly. Something must be deflecting them. 

Since \(\vec{k} \cdot \delta \vec{B} = 0\), the magnetic field perturbation must be in the form \(\delta \vec{B} = (\delta B_x, \delta B_y, 0)\). 
From Faraday's law, we find \(\omega \delta \vec{B} = c \vec{k} \times \delta \vec{E}\): 
%
\begin{align}
\frac{\omega}{c} \left[\begin{array}{c}
\delta B_x \\ 
\delta B_y \\ 
0
\end{array}\right]
= \det \left[\begin{array}{ccc}
\hat{x} & \hat{y} & \hat{z} \\ 
0 & 0 & k \\ 
\delta E_x & \delta E_y & \delta E_z
\end{array}\right]
= \left[\begin{array}{c}
- k \delta E_y \\ 
k \delta E_x \\ 
0
\end{array}\right]
\,.
\end{align}

Therefore, \(\delta B_x = - (kc / \omega ) \delta E_y\) and \(\delta B_y = (kc / \omega ) \delta E_x\). 

\paragraph{Manipulating the Lorentz terms in Vlasov}

Let us now try to simplify the last three terms in the perturbed, Fourier space Vlasov equation \eqref{eq:perturbed-fourier-space-vlasov}: they read 
%
\begin{align}
\hlc{violet}{q_\alpha \delta \vec{E}_\beta \pdv{f_{0 \alpha }}{p_\beta } }
+ \hlc{orange}{q_\alpha \qty( \frac{\vec{v}}{c} \times \vec{B}_0)_\beta \pdv{ \delta f_\alpha }{p_\beta }}
+ \hlc{yellow}{q_\alpha \qty( \frac{\vec{v}}{c} \times \delta \vec{B}) 
\pdv{f_{0 \alpha }}{p_\beta }}
\,,
\end{align}
%
so, explicitly: 
%
\begin{align}
\begin{split}
&\phantom{=}\  \hlc{yellow}{\frac{q_\alpha}{c} \det \left[\begin{array}{ccc}
\hat{x} & \hat{y} & \hat{z} \\ 
v_x & v_y & v_z \\ 
\delta B_x & \delta B_y & 0
\end{array}\right]_\beta 
\pdv{f_{0 \alpha }}{p_\beta }} \\
&+ \hlc{orange}{\frac{q_\alpha}{c} \det \left[\begin{array}{ccc}
\hat{x} & \hat{y} & \hat{z} \\ 
v_x & v_y & v_z \\ 
0 & 0 & B_0 
\end{array}\right]_\beta 
\pdv{ \delta f_\alpha }{p_\beta }}
+ \hlc{violet}{q_\alpha
\qty(\delta E_x \pdv{f_{0 \alpha }}{p_x} + \delta E_y \pdv{f_{0 \alpha }}{p_y})} =
\end{split}  \\
\begin{split}
&= \hlc{yellow}{\frac{q_\alpha}{c} \qty( - v_z \delta B_y \pdv{f_{0 \alpha }}{p_x} + v_z \delta B_x \pdv{f_{0 \alpha }}{p_y} + \qty(v_x \delta B_y - v_y \delta B_x) \pdv{f_{0\alpha} }{p_z})} 
\\ 
& + \hlc{orange}{\frac{q_\alpha}{c} \qty(
    v_y B_0 \pdv{ \delta f_\alpha }{p_x} - v_x B_0 \pdv{ \delta f_\alpha }{p_y}
)}
+ \hlc{violet}{q_\alpha \qty(
    \delta E_x \pdv{f_{0 \alpha }}{p_x} +
    \delta E_y \pdv{f_{0 \alpha }}{p_y}
)}
\,.
\end{split}
\end{align}

\paragraph{Moving to a circular polarization basis}

We can always do the calculations for a perturbation in terms of circularly polarized waves, and then add them together. 
We will write this as \(\delta E_y = \pm i \delta E_x\). 
The idea is then to rewrite everything in terms of \(\delta E_x\), which we can then just denote as \(\delta E\).  
Thanks to this, we get: 
%
\begin{align}
\delta E_x &= \delta E  &
\delta E_y &= \pm i \delta E  \\
\delta B_x &= \mp \frac{kc}{\omega } \delta E  &
\delta B_y &= \frac{kc}{\omega } \delta E
\,.
\end{align}
%
%  \(\delta B_x = \mp (k c / \omega ) \delta E\) and  \(\delta B_y = (kc / \omega ) \delta E\). 

With these substitutions, the three terms read
%
\begin{align}
\begin{split}
& \hlc{yellow}{\frac{q_\alpha}{c} \qty( -v_z \frac{kc}{ \omega } \delta E \pdv{f_{0 \alpha }}{p_x}
-  v_z i \frac{kc}{\omega } \delta E
\pdv{f_{0 \alpha }}{p_y} 
+ \qty[v_x \frac{ck}{\omega } \delta E \pm i v_y \frac{ck}{\omega } \delta E] \pdv{f_{0 \alpha }}{p_z})} +
\\
&\phantom{=}\ 
+ \hlc{violet}{q_\alpha \delta E \pdv{ f_{0 \alpha} }{p_x} \pm q_\alpha i \delta E \pdv{f_{0 \alpha }}{p_y}} 
+ \hlc{orange}{\frac{q_\alpha}{c} 
B_0 \qty[v_y \pdv{f_{0 \alpha }}{p_x} - v_x \pdv{ \delta f_\alpha }{p_y}]}
\,.
\end{split}
\end{align}

\paragraph{Moving to cylindrical coordinates}

The unperturbed distribution function \(f_{0 \alpha }\) cannot depend on the angle around the direction of the magnetic field \(\vec{B}_0 \)
because of the cylindrical symmetry: it will be written like \(f_{0 \alpha } = f_{0 \alpha } (p_{\parallel}, p_\perp)\), therefore it is useful to move to cylindrical coordinates. 

From \(p_x, p_y, p_z\) we move to \(p_\parallel\), \(p_\perp\), \(\varphi \), where 
%
\begin{align}
p_x &= p_\perp \cos \varphi \\
p_y &= p_\perp \sin \varphi \\
p_z &= p_\parallel
\,.
\end{align}

One can invert the Jacobian in order to get 
%
\begin{align}
\dd{p_\perp} &= \cos \varphi \dd{p_x} + \sin \varphi \dd{p_z}  \\
\dd{\varphi } &= - \frac{1}{p_\perp} \sin \varphi \dd{p_x} + \frac{1}{p_z} \cos \varphi \dd{p_z}  \\
\dd{p_\parallel} &= \dd{p_z}
\,.
\end{align}

Therefore, 
%
\begin{align}
\pdv{ f_{0 \alpha }}{p_x} = \pdv{ f_{0 \alpha }}{p_\perp} \cos \varphi 
\qquad \text{and} \qquad
\pdv{ f_{0 \alpha }}{p_y} = \pdv{ f_{0 \alpha }}{p_\perp} \sin \varphi 
\,,
\end{align}
%
since the term containing a derivative with respect to \(\varphi \) vanishes. 

With this, the term reads 
%
\begin{align}
\begin{split}
& \frac{q_\alpha }{c} v_z \frac{ck}{\omega } \delta E \qty(
    \pdv{f_{0 \alpha }}{p_\perp} \cos \varphi 
    \pm i \pdv{f_{0 \alpha }}{p_\perp} \sin \varphi 
)
+ \frac{q_\alpha}{c} \frac{ck}{\omega } \delta E
\qty(v_\perp \cos \varphi \pm i v_\perp \sin \varphi )
\pdv{f_{0 \alpha }}{p_\parallel} 
\\
&\phantom{=}\ 
+ q_\alpha \delta E \pdv{f_{0 \alpha }}{p_\perp} \qty(\cos \varphi \pm i \sin \varphi )  
- \frac{q_\alpha}{c} B_0 \frac{v_\perp}{p_\perp} \pdv{ \delta f_\alpha }{p}
\end{split}  \\
&= \frac{q_\alpha }{c} v_z \frac{ck}{\omega } \delta E 
    \pdv{f_{0 \alpha }}{p_\perp} e^{\pm i \varphi }
+ \frac{q_\alpha}{c} \frac{ck}{\omega } \delta E
v_\perp e^{\pm i \varphi }
\pdv{f_{0 \alpha }}{p_\parallel} 
+ q_\alpha \delta E \pdv{f_{0 \alpha }}{p_\perp} e^{\pm i \varphi } 
- \frac{q_\alpha}{c} B_0 \frac{v_\perp}{p_\perp} \pdv{ \delta f_\alpha }{\varphi }
\,.
\end{align}

\paragraph{Recombining the Vlasov equation}

Now we just need to add in the two remaining terms of the Vlasov equation: 
%
\begin{align}
\begin{split}
&
- i \omega \delta f_\alpha 
+ i \vec{k} \cdot \vec{v} \delta f_\alpha + 
\frac{q_\alpha }{c} v_z \frac{ck}{\omega } \delta E 
    \pdv{f_{0 \alpha }}{p_\perp} e^{\pm i \varphi }
+ \frac{q_\alpha}{c} \frac{ck}{\omega } \delta E
v_\perp e^{\pm i \varphi }
\pdv{f_{0 \alpha }}{p_\parallel} 
+ q_\alpha \delta E \pdv{f_{0 \alpha }}{p_\perp} e^{\pm i \varphi } 
\\
&\phantom{=}\ 
- \frac{q_\alpha}{c} B_0 \frac{v_\perp}{p_\perp} \pdv{ \delta f_\alpha }{\varphi } = 0
\,.
\end{split}
\end{align}

The last term includes a factor in which we can recognize the cyclotron frequency corresponding to particle species \(\alpha \) moving in the magnetic field \(\vec{B}_0\): 
%
\begin{align}
\frac{q_\alpha}{c} B_0 \frac{v_\perp}{p_\perp} = \frac{q_\alpha B_0 }{c} \frac{v_\perp}{m_\alpha v_\perp \gamma } = \Omega _{\text{gyr, }\alpha }
\,.
\end{align}

\paragraph{Making a rotating ansatz}

If \(\delta f_\alpha \) was in the form \(\delta g_\alpha e^{\pm i \varphi }\), then the equation would automatically be satisfied, if \(g_\alpha \) does not depend on the phase, since in that case the last derivative would read \(\mp i \delta g_\alpha e^{\pm i \varphi }\). 

With this substitution all the phase factors would cancel, and we would find 
%
\begin{align}
\delta g_\alpha = \frac{\frac{q_\alpha}{c} \delta E \qty[v_z \frac{ck}{\omega } \pdv{f_0 }{p_\perp} - v_\perp \frac{ck}{\omega } \pdv{f_0 }{p_\perp} - \pdv{f_0 }{p_\perp}]}{- i \omega + i k v_z \mp i \Omega _{\text{gyr}}}
\,.
\end{align}

\paragraph{Closing the system with the Ampére-Maxwell law}

We have one step left in order to close the system: 
we need a relation between \(\delta f_\alpha \) and \(\delta E\); we know that 
%
\begin{align}
i \vec{k} \times \delta \vec{B} = \frac{4 \pi }{c} \sum _{\alpha } q_\alpha \int \dd[3]{p } \vec{v} \delta f_\alpha 
- i \frac{\omega}{c} \delta \vec{E}
\,.
\end{align}
%
However, we already know how to connect \(\delta \vec{B}\) and \(\delta \vec{E}\): 
%
\begin{align}
- i k \frac{ck}{\omega } \delta E = 
\frac{4 \pi }{c} \sum _{\alpha } q_\alpha \int \dd{p_\perp} p_\perp \int \dd{p_\parallel} \int \dd{\varphi } v_\perp \cos \varphi 
- i \frac{\omega}{c} \delta E
\,,
\end{align}
%
where we are considering the \(x\) component only, which was what we called \(\delta E\). 

If we decompose \(\delta f_\alpha  = \delta g_\alpha e^{\pm i \varphi }\), the integral in \(\varphi \) reads 
%
\begin{align}
\int_0^{2 \pi } \dd{\varphi } \cos \varphi e^{\pm i \varphi } = \pi 
\,,
\end{align}
%
so we get, substituting in what we know for \(\delta g_\alpha \): 
%
\begin{align}
\frac{- i k^2 c}{\omega } \delta E = \frac{4 \pi }{c} 
\sum _{\alpha } q_\alpha \int \dd{p_\perp} p_\perp v_\perp \int \dd{p_\parallel} 
\frac{\frac{q_\alpha}{c} \delta E \qty[v_z \frac{ck}{\omega } \pdv{f_0 }{p_\perp} - v_\perp \frac{ck}{\omega } \pdv{f_0 }{p_\perp} - \pdv{f_0 }{p_\perp}]}{- i \omega + i k v_z \mp i \Omega _{\text{gyr, } \alpha }}
- i \frac{\omega}{c} \delta E
\,,
\end{align}
%
which looks like a dispersion relation: we just need to collect the generic \(\delta E\), to get 
%
\begin{align}
\frac{k^2 c^2}{\omega^2} = 1 - \sum _{\alpha } \frac{4 \pi^2 q_\alpha^2}{\omega } \int \dd{p_\parallel} \int \dd{p_\perp} \qty( p_\perp v_\perp \qty(\frac{v_\parallel k}{\omega } - 1) \pdv{p_{0 \alpha }}{p_\perp} - \frac{k v_\perp}{\omega } \pdv{f_{0 \alpha }}{p_\parallel}) \qty[ \omega - k v_\parallel \pm \Omega _\alpha ]^{-1}
\,.
\end{align}

We can write an alternative version of this using \(v_\parallel = v \mu \) and \(v_\perp (1 - \mu^2)^{1/2} v\), where \(\mu = \cos \theta \): 
\begin{align}
\frac{k^2 c^2}{\omega^2} = 1 - \sum _{\alpha } \frac{4 \pi^2 q_\alpha^2}{\omega } \int \dd{p} \int \dd{\mu } \frac{p^2 v (1 - \mu^2)}{\omega - kv \mu \pm \Omega _\alpha }
\qty[- \pdv{f_{0 \alpha }}{p} + \frac{1}{p} \pdv{f_{0 \alpha }}{\mu } \qty(\mu - \frac{kv}{\omega })]
\,.
\end{align}

\paragraph{Application: a cold electron-proton plasma}

This is very general, and we can specify it to the case in which we only have electrons and protons, at \(T = 0\). 

This means that \(f_{0 \alpha } = \delta (p) n_\alpha / 4 \pi p^2\) for some constant. 
This way, we have 
%
\begin{align}
\int_0^{\infty } \dd{p} 4 \pi p^2 \frac{n_\alpha}{4 \pi p^2} \delta (p) = n_\alpha 
\,.
\end{align}

This will be true for both protons and electrons, and their densities will be equal to preserve neutrality: \(n_e = n_p\). 
There is no dependence on \(\mu \), so we are left with 
%
\begin{align}
\frac{k^2 c^2}{\omega^2} &= 
1 - \sum _{\alpha } \frac{4 \pi^2 q_\alpha^2}{\omega } \int \dd{p} \int \dd{\mu } 
\frac{p^2 v (1 - \mu^2)}{\omega - kv \mu \pm \Omega _\alpha }
\qty(- \pdv{}{p} \frac{n_\alpha \delta (p)}{4 \pi p^2})  \\
&= 1 - \frac{4 \pi e^2 n}{\omega } \int \dd{p} \dd{\mu }
\frac{ \delta (p)}{4 \pi p^2} \pdv{}{p}  \qty[\frac{p^2 v (1 - \mu^2)}{\omega - k v \mu \pm \Omega _\alpha }]
\,.
\end{align}

Let us look at the integral with respect to momentum, using \(p = m_\alpha v\): 
%
\begin{align}
\begin{split}
&\phantom{=}\ \int \dd{p} \dd{\mu }
\frac{ \delta (p)}{4 \pi p^2} \pdv{}{p}  \qty[\frac{p^2 v (1 - \mu^2)}{\omega - k v \mu \pm \Omega _\alpha }]= \\
&= \int \dd{p} \frac{ \delta (p)}{4 \pi p^2} \qty( \frac{3 p^2}{m_\alpha } (\omega - k v \mu \pm \Omega _\alpha ) + \frac{k_\mu }{m_\alpha } \frac{p^3}{m_\alpha }) \qty(\omega - k v \mu \pm \Omega_\alpha^2)^{-1} 
\end{split}  \\
&= \frac{3}{4 \pi m_\alpha } \frac{1}{\omega \pm \Omega _\alpha }
\,.
\end{align}

The integral in \(\mu \) is just \(\int_{-1}^{1} \dd{\mu } (1 - \mu^2) = 4/3\). 
This finally yields 
%
\begin{align}
\frac{k^2c^2}{\omega^2} &= 1 
- \frac{4 \pi e^2 n }{\omega m_p} \frac{1}{\omega \pm e B_0 / m_p c} 
- \frac{4 \pi e^2 n }{\omega m_e} \frac{1}{\omega \mp e B_0 / m_e c}  \\
&= 1 
\mp 
\frac{4 \pi e^2 n}{\omega m_p} \frac{m_p c}{e B_0} \frac{1}{1 \pm \omega / \Omega _p} 
\pm 
\frac{4 \pi e^2 n}{\omega m_e} \frac{m_e c}{e B_0} \frac{1}{1 \mp \omega / \abs{\Omega _e}} 
\marginnote{We can replace \(m_e / m_e\) by \(m_p / m_p\). }
\,,
\end{align}
%
so if we consider the case in which \(\omega \ll \Omega _p \ll \abs{\Omega _e}\), we get 
%
\begin{align}
\frac{k^2 c^2}{\omega^2} &= 1 \mp 
\frac{4 \pi e^2 n}{\omega m_p} \frac{m_p c}{e B_0 } \qty( 1 \mp \frac{\omega }{\Omega _p}) 
\pm \frac{4 \pi e^2n}{\omega m_p} \frac{1}{\Omega _p}  \\
&= 1 + \frac{4 \pi e^2 n }{\omega m_p } \frac{\omega}{\Omega _p^2}  \\
&= 1 + \frac{4 \pi e^2 n }{m_p } \frac{m_p^2 c^2}{e^2 B_0^2}  \\
&= 1 + 4 \pi \underbrace{n m_p }_{\rho } \frac{c^2}{B_0^2}  \\
&= 1 + 4 \pi \rho \frac{c^2}{B_0^2} = 1 + \qty( \frac{c}{v_A})^2
\,,
\end{align}
%
where we introduced 
%
\begin{align}
v_A = \frac{B_0 }{\sqrt{4 \pi \rho }}
\,,
\end{align}
%
so in the galaxy, where \(B_0 \sim \SI{}{\micro G}\) and \(\rho \sim m_p / \SI{}{cm^3}\), we have \(v_A \sim \SI{2}{km/s}\). 

This means that 
%
\begin{align}
\frac{k^2c^2}{\omega^2} = \frac{c^2}{v_A^2} \implies \pm k v_A = \omega 
\,.
\end{align}

These perturbations travel very slowly! 
They are called Alfvén waves. They move with \(\vec{k}\) along \(\vec{B}_0 \) (well, we assumed so), and they have oscillating electric and magnetic fields, both perpendicular to the driving magnetic fields. 

Without these, cosmic rays would be basically travelling at the speed of light all the time. 
They also create a small, induction-generated electric field, which is interesting! 

\end{document}
