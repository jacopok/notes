\documentclass[main.tex]{subfiles}
\begin{document}

\section{Magneto-HydroDynamics}

\marginpar{Thursday\\ 2021-11-25}

We are going to briefly treat a topic which is often discussed in whole books. 

The Vlasov equation has a large amount of information in it, way more than what we can actually measure for astrophysical plasmas. 

So, we would like to treat our plasma like a magnetized fluid. 
Further, we know that astrophysical plasmas are extremely conductive. 
Therefore, the only significant electric field we can have is the one associated with induction. 

Still, we must retain a kinetic description (with the full phase space distribution) for anything which is not the background plasma, like the nonthermal particles. 
We only want to simplify the treatment of the background. 

We always start with the Vlasov equation, which we can write in index notation: 
%
\begin{align}
\pdv{f}{t} + \vec{v} \cdot \vec{\nabla} f 
+ \frac{q}{m} \qty[\vec{E} + \frac{\vec{v}}{c} \times \vec{B}] \cdot \vec{\nabla}_v f &= 0  \\
\pdv{f}{t} + v_r \pdv{f}{x_r} 
+ \frac{q}{m} \qty[E_r + \frac{1}{c} \epsilon_{rst} v_s B_t] \pdv{f}{v_r} &= 0
\,.
\end{align}

Now, we need to define local mean values with respect to the phase space distribution of any function: 
%
\begin{align}
\expval{\varphi } = \frac{\int \dd[3]{v} \varphi f}{\int \dd[3]{v} f} = \frac{1}{n} \int \dd[3]{v} \varphi f
\,.
\end{align}

Suppose the function we are averaging is only dependent on the velocity, which we denote as \(\psi (\vec{v})\), we can multiply it by the Vlasov equation and integrate over \(\dd[3]{v}\): the various terms become
%
\begin{align}
\int \pdv{}{t} \psi (\vec{v}) f \dd[3]{v} &= \pdv{}{t} \int \dd[3]{v} \psi f = \pdv{}{t} \qty[n \expval{\psi }]  \\
\int \dd[3]{v} \psi (\vec{v}) v_r \pdv{f}{x_r} &= \pdv{}{x_r} \dd[3]{v} \psi v_r f = \pdv{}{x_r} \qty[n \expval{\psi v_r}]  \\
\frac{q}{m} \int \dd[3]{v} E_r \psi (\vec{v}) \pdv{f}{v_r} &= - \frac{q}{m} \int \dd[3]{v} f \dv{\psi }{v_r} E_r = - \frac{q}{m} E_r n \expval{ \dv{\psi }{v_r}} 
\marginnote{We integrate by parts, and at the boundary \(\abs{v} \to \infty \) all quantities vanish.}  \\
\frac{q}{mc} \int \dd[3]{v} \psi \epsilon_{rst} v_s B_t \pdv{f}{v_r} &= - \frac{q}{mc} \int \dd[3]{v} f \epsilon_{rst} B_t \pdv{}{v_r} \qty( v_s \psi )  \\
&= - \frac{q}{mc} \epsilon_{rst} B_t n \expval{v_s \pdv{\psi }{v_r}}
\,.
\end{align}

We brought \(v_s\) outside the derivative since the term it yields is symmetric but contracted with the Levi-Civita symbol. 

If we take \(\psi \equiv 1\), we get 
%
\begin{align}
\dv{n}{t} + \pdv{}{x_r} \qty(n \expval{v_r})  = 0
\,,
\end{align}
%
since all the derivatives of \(\psi \) vanish. 
We found the continuity equation!
Keep in mind that for now this only holds for one component of the fluid. 

Let us take \(\psi = v_r\). We then get 
%
\begin{align}
\pdv{}{t} \qty[n \expval{v_r}] + 
\pdv{}{x_s} \qty[n \expval{v_r v_s}] 
- \frac{q}{m} E_r n  
- \frac{q}{mc} n \epsilon_{rst} B_t \expval{v_s} = 0
\,.
\end{align}

If the fluid is \emph{at rest} in our frame its mean velocity is zero. 
Let us be more general than this: suppose there is a bulk motion \(u_r = \expval{v_r}\). 
Then, the equation reads 
%
\begin{align}
\pdv{}{t} \qty[n u_r] + 
\pdv{}{x_s} \qty[n \expval{v_r v_s}] 
- \frac{q}{m} E_r n  
- \frac{q}{mc} n \epsilon_{rst} B_t u_s = 0
\,.
\end{align}

We have one of these equations for the electrons and one for the protons. 
We have a few component masses in the denominator, but if we multiply through my \(m\) the densities in the first two terms just become mass densities. 

Let us define the total density \(\rho = n_p m_p + n_e m_e\). 
The density of charges will read \(\zeta = (n_p - n_e) e\), 

For now, the bulk motions of electrons and protons may differ, so we will have \(u_{p, r}\) and \(u_{e, r}\).
The current density will then read  
%
\begin{align}
J_r = \qty(n_p u_{p, r} - n_e u_{e, r}) e
\,.
\end{align}

We can define a global, mass-averaged bulk velocity: 
%
\begin{align}
U_r = \frac{n_p m_p u_{p, r} + n_e m_e u_{e, r}}{\rho }
\,.
\end{align}

We still need to describe internal energy through fluctuations over this mean: the peculiar velocity will read 
%
\begin{align}
w_{p, r} &= v_{p, r} - U_r \\
w_{e, r} &= v_{e, r} - U_r 
\,.
\end{align}

Let us now multiply the continuity equation for protons by the proton mass, and similarly for the electrons: 
%
\begin{align}
\pdv{}{t} (n_p m_p) + \pdv{}{x_r} \qty( n_p m_p u_{p, r}) &= 0  \\
\pdv{}{t} (n_e m_e) + \pdv{}{x_r} \qty( n_e m_e u_{e, r}) &= 0 
\,,
\end{align}
%
which we can add together: we get a global mass conservation equation
%
\begin{align}
\pdv{\rho }{t} + \pdv{}{x_r} \qty(\rho U_r) = 0
\,.
\end{align}

Applying a similar procedure to the \(\psi = v_r\) equation we find 
%
\begin{align}
\pdv{}{t} \qty(n_p m_p u_r) + \pdv{}{x_s} \qty(n_p m_p \expval{v_r v_s}) - e E_r n_p - \frac{e}{c} n_p \epsilon_{rst} B_t n_s &= 0
\,,
\end{align}
%
which we can add to the corresponding equation for the electrons. 
First, though, let us look at the \(\expval{v_r v_s}\) term: 
%
\begin{align}
\expval{v_r v_s} 
&= \expval{(w_r + U_r) (w_s + U_s)}   \\
&= \expval{w_r w_s} + \expval{w_r} U_s + \expval{w_s} U_r + U_r U_s  \\
&=  \expval{w_r w_s} + (u_r - U_r) U_s + (u_s - U_s) U_r + U_r U_s   \\
&= \expval{w_r w_s} + u_r U_s + u_s U_r - U_r U_s
\,.
\end{align}

Let us now introduce a stress (pressure) tensor:
%
\begin{align}
P_{rs, p} = m_p \int \dd[3]{v} f w_{p, r} w_{p, s} = n m_p \expval{w_{p, r} w_{p_s}}
\,,
\end{align}
%
which yields 
%
\begin{align}
\expval{v_r v_s} = \frac{P_{rs}}{mn} + u_r U_s + u_s U_r - U_r U_s
\,.
\end{align}

We can now sum the two components: 
%
\begin{align}
\begin{split}
&\phantom{=}\ 
n_p m_p \qty[\frac{P_{rs, p}}{m_p n_p} + u_{r, p} U_s + u_{s, p} U_r - U_r U_s] + 
n_e m_e \qty[\frac{P_{rs, e}}{m_e n_e} + u_{r, e} U_s + u_{s, e} U_r - U_r U_s] =\\
&= P_{rs} + \rho U_r U_s + \rho U_r U_s - \rho U_r U_s 
= P_{rs} + \rho U_r U_s  
\,.
\end{split}
\end{align}

The full equation now looks like: 
%
\begin{align}
\dv{t}(\rho U_r) + \pdv{}{x_s} \qty[P_{rs} + \rho U_r U_s]
- \zeta E_r 
- \frac{1}{c} \epsilon_{rst} J_s B_t = 0
\,.
\end{align}

We can expand the derivatives to get something which looks more like an equation of motion, and simplify through the continuity equation:  
%
\begin{align}
\underbrace{\pdv{\rho }{t} U_r}_{U_r \times \text{continuity}} + \rho \dv{U_r}{t} + \pdv{P_{rs}}{x_s}
+ \underbrace{\pdv{}{x_s} \qty(\rho U_s) U_r}_{U_r \times \text{continuity}} 
+ \pdv{}{x_s} \qty(\rho U_r) U_s + \dots
\,,
\end{align}
%
so the equation reads 
%
\begin{align}
\underbrace{\rho \pdv{U_r}{t} + \rho U_s \pdv{U_r}{x_s} }_{\rho D_t U}
= - \pdv{P_{rs}}{x_s} + \zeta E_r
+ \frac{1}{c} \epsilon_{rst} J_s B_t
\,.
\end{align}

Now this system does not know about protons and electrons anymore: it is just a fluid. 

In vector terms, 
%
\begin{align}
\rho \qty[ \pdv{\vec{U}}{t} + \vec{U} \cdot \vec{\nabla} U]
= - \vec{\nabla} P + \zeta \vec{E} + \frac{1}{c} \vec{J} \times \vec{B}
\,.
\end{align}

We haven't done the energy conservation equation since it will not be useful for our purposes, while being very lengthy. 
Still, in principle we should use it. 

We will now make the \textbf{ideal MHD approximation}, in which the conductivity is assumed to be very high. 
Still, there are situations in which the resistivity becomes high locally. 
This can happen through a phenomenon called \emph{reconnection}, for example. 

Suppose we have an electric field in the lab frame: in the frame comoving with the plasma, it will look like \(\vec{E} \to \vec{E} + \vec{U} \times \vec{B} / c\). 

This can be described by Ohm's law, 
%
\begin{align}
\vec{E} + \frac{1}{c} \vec{U} \times \vec{B} = \eta \vec{J}
\,.
\end{align}

Using the Ampére-Maxwell equation we get that the current reads 
%
\begin{align}
\vec{J} = \frac{c}{4 \pi } \qty[\vec{\nabla} \times \vec{B} - \frac{1}{c} \pdv{\vec{E}}{t}]
\,.
\end{align}

We assume that \(\eta \) is very small. 
In the \(\eta = 0 \) case, in the lab reference frame we get 
%[]
\begin{align}
\vec{E} = - \frac{1}{c} \vec{U} \times \vec{B}
\,.
\end{align}

What is the order of magnitude of \(\vec{J} \times \vec{B}  / c\)? 
%
\begin{align}
\frac{1}{4 \pi } \qty[\vec{\nabla} \times \vec{B} - \frac{1}{c} \pdv{\vec{E}}{t} ] \times \vec{B} \sim \frac{B^2}{UT} + \frac{U}{c^2} \frac{B^2}{T}
\,,
\end{align}
%
The first term has scale \(B^2 / L\), the second has scale \((U/c^2) (B^2 /T)\).

We can then see that under this low-resistivity assumption the second term can be neglected. 

Then, the equation just reads 
%
\begin{align}
\rho \qty[ \pdv{\vec{U}}{t} + \vec{U} \cdot \vec{\nabla} U]
= - \vec{\nabla} P + \frac{1}{4 \pi } (\vec{\nabla} \times \vec{B}) \times \vec{B}
\,.
\end{align}

There is an interesting implication to this equation known as \textbf{flux freezing}. 

The assumption of ideal MHD is that there is no dissipation: the field lines cannot cross; we can ``squeeze'' them, but the flux will be preserved.

Let us take a surface \(S_0 \), and compute the magnetic field flux across it at \(t = t_0 \): 
%
\begin{align}
\Phi_0 = \int \dd{\vec{S}_0} \vec{B} (\vec{x}, t_0 )
\,,
\end{align}
%
which we want to compare to the flux across the same surface, transported along the fluid motion: 
%
\begin{align}
\Phi_1 &= \int \dd{S_1} \vec{B} (\vec{x} + \vec{v} \dd{t}, t_0 + \dd{t})   \\
&= \int \dd{S_1} \qty[
    \vec{B} (\vec{x} + \vec{v} \dd{t}, t_0 )
    + \pdv{\vec{B}}{t} (\vec{x} + \vec{v} \dd{t}, t_0 ) \dd{t}
]
\,.
\end{align}

If we include \(S_0 \), \(S_1 \) as well as the boundary \(S_2 \) we will get a closed surface. 
The integral over \(S = - S_0 + S_1  + S_2 \) will be 
%
\begin{align}
\oint_S \vec{B} \cdot \dd{S} = 0 
\,,
\end{align}
%
where we write \(- S_0 \) since the normal to that surface must be oriented outward. 

From this equation we get that 
%
\begin{align}
\Phi_0 = \int \dd{S_1} B + \int \dd{S_2} B
\,.
\end{align}

What is \(\dd{S_2}\)? It can be written as \(\dd{\ell} \wedge \vec{v} \dd{t}\), since we are asking that we move along the flux lines of the fluid. 

We know that 
%
\begin{align}
\Phi_1 
&= \Phi_0 - \int \dd{\vec{S}_1} \cdot \vec{B} 
 + \dd{t} \int \dd{\vec{S}_1} \pdv{\vec{B}}{t} (\vec{x} + \vec{v} \dd{t} , t_0 )  \\
&= \Phi_0 - \dd{t} \int (\dd{\vec{\ell}} \times \vec{c}) \cdot \vec{B} 
 + \dd{t} \int \dd{\vec{S}_1} \pdv{\vec{B}}{t} (\vec{x} + \vec{v} \dd{t} , t_0 )  \\
&= \Phi_0 - \int \dd{t} (\vec{v} \times \vec{B}) \cdot \dd{\ell} + \dd{t} \int \pdv{\vec{B}}{t} \dd{S}  \\
&= \Phi_0 - \dd{t} \int \dd{S} \qty[ \vec{\nabla} \times (\vec{\nabla} \times \vec{B}) + \pdv{\vec{B}}{t}]
\,,
\end{align}
%
but since 
%
\begin{align}
\vec{E} = - \frac{1}{c} \vec{v} \times \vec{B}
\qquad \text{and} \qquad
\pdv{\vec{B}}{t} = - c \vec{\nabla} \times \vec{E}
\,,
\end{align}
%
we get that the integrand vanishes. 


\end{document}
