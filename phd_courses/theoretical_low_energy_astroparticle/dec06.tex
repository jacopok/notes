\documentclass[main.tex]{subfiles}
\begin{document}

\section{Neutrino physics}

\marginpar{Monday\\ 2021-12-6}

Neutrinos are important because:
\begin{enumerate}
    \item being neutral, they can trace sources; \todo[inline]{What does that mean?}
    \item they are connected with new physics;
    \item there are open questions about their nature (Dirac or Majorana);
    \item there could be a CP violation in the lepton sector;
    \item they affect cosmology, with Baryon Acoustic Oscillations in the CMB, 
    as well as the large scale structure.
\end{enumerate}

\url{nu.to.infn.it}, ``Neutrino Unbound''; Giunti et al, Giunti and Kim. 

Outline of the course: 
\begin{enumerate}
    \item Dirac equation; 
    \item gauge theories;
    \item standard EW model; 
    \item fermion masses and mixing;
    \item neutrino oscillations in vacuo;
    \item neutrino oscillations in matter;
    \item current neutrino phenomenology;
    \item extra on statistics and data analysis. 
\end{enumerate}

We know that neutrino masses are less than an \SI{}{eV}; 
there are (at least) three flavours: \(\nu _e\), \(\nu _\mu \) and \(\nu _\tau \), 
typically organized in doublets with the corresponding charged lepton. 

These neutrinos can have charged interactions with a \(W^{\pm}\) boson, 
as well as neutral interactions with a \(Z\) boson
(which is connected with photon production). 

Looking at these decays tells us about the number of (interacting) neutrino families: \(N_\nu = 3\). 

There are also bounds from cosmology: both in the CMB spectrum and in primordial nucleosynthesis. 

Neutrinos are chiral in nature: only left-handed neutrinos (with negative helicity) seem to interact. 

We know that a left-handed particle has a right-handed component of order \(\beta \sim m/E\). 

At order \(m^2 / E\) there are neutrino flavor oscillations. 

\subsection{The Dirac equation}

We start from Lorentz transformations: linear transformations which preserve the Minkowski metric. 
They are rotations and boosts. 
They can be written in terms of infinitesimal transformations: for example, 
%
\begin{align}
\left[\begin{array}{c}
y' \\ 
z'
\end{array}\right]
= \left[\begin{array}{cc}
\cos \theta  & \sin \theta  \\ 
- \sin \theta  & \cos \theta 
\end{array}\right]
\left[\begin{array}{c}
y \\ 
z
\end{array}\right]
\approx \left(
    \left[\begin{array}{cc}
    1 & 0 \\ 
    0 & 1
    \end{array}\right]
    + 
    \left[\begin{array}{cc}
    0 & \theta  \\ 
    - \theta  & 0
    \end{array}\right]
\right) \left[\begin{array}{c}
y \\ 
z
\end{array}\right]
\,,
\end{align}
%
therefore  
%
\begin{align}
\Lambda = \mathbb{1} + i \dd{\theta } \left[\begin{array}{cc}
0 & -i \\ 
i & 0
\end{array}\right] + \order{\theta^2} = e^{i \theta J_1 } + \order{\theta^2}
\,.
\end{align}

The same holds for the boosts: 
%
\begin{align}
\Lambda = \left[\begin{array}{cc}
\gamma  & \beta \gamma  \\ 
\beta \gamma  & \gamma 
\end{array}\right] 
= e^{i u k_1 }
\qquad \text{where} \qquad
K_1 = \left[\begin{array}{cc}
0 & -i \\ 
i & 0
\end{array}\right]
\,.
\end{align}

With these generators we can make any transformation we want: 
a Lorentz transformation will be in general given by the composition
of a rotation around \(\vec{\omega} = \theta \hat{n}\) and a boost in \(\vec{u} = u \hat{v}\). 
This will then read 
%
\begin{align}
\Lambda = \exp(i \left(\vec{\omega } \cdot \vec{J} + \vec{u} \cdot \vec{K}\right))
\,.
\end{align}

In general these do not commute: 
\([J_i, J_k] \neq 0\), \([K_i, K_j] \neq 0\), \([J_i, K_j] \neq 0\). 
This, however, is a closed algebra: all these commutators are given in terms of other \(J\), \(K\) matrices.

This construction is not only useful if we need to make complicated transformations; 
it is also useful as a theoretical mean to parametrize a general transformation. 

Pauli was trying to generalize the Schrödinger equation for a spin-\(1/2\) particle. 
It can be found by mapping \(\vec{p} \to -i \vec{\nabla}\) and \(E \to i \partial_t\) in the 
eigenvalue equation for a classical kinematic Hamiltonian \(H = p^2 / 2m + V\). 

This is classical, so we forget boosts: let us try to at least have the 4D \(J_i\) rotation algebra 
\([J_i, J_j] = i \epsilon_{ijk}J_k\)
for two-component vectors: this is also obeyed by the 2D matrices \(\sigma_i / 2\), where 
%
\begin{align}
\sigma_1 &= \left[\begin{array}{cc}
0 & 1 \\ 
1 & 0
\end{array}\right]  \\
\sigma_2 &= \left[\begin{array}{cc}
0 & -i \\ 
i & 0
\end{array}\right]  \\
\sigma _3 &= \left[\begin{array}{cc}
1 & 0 \\ 
0 & -1
\end{array}\right]
\,.
\end{align}

So, we could think to have a spinor transform under 
%
\begin{align}
\xi ' = \lambda \xi  
\qquad \text{where} \qquad
\lambda = \exp(i \vec{\omega} \cdot \frac{\vec{\sigma}}{2})
\,.
\end{align}

In order for the momentum \(\vec{p}\) to act on 2D objects we can use \(\vec{p} \cdot \vec{\sigma}\). 

The idea is to use minimal coupling: \(\vec{p} \to \vec{p} - q \vec{A}\), and \(\vec{V} \to \vec{V} + q \Phi \). 

How can we generalize this to boosts? 
Not only is it possible to do, but there are two ways to do it. 

The Lorentz group, to which the matrices \(\Lambda \) in \(x' = \Lambda x \) belong, satisfies 
%
\begin{align}
\Lambda &= \exp(i \left(\vec{\omega} \cdot \vec{J} + \vec{u} \cdot \vec{K}\right)) \\
[J_i, J_j] &= i \epsilon_{ijk} J_k \\
[K_i, K_j] &= - i \epsilon_{ijk} J_k \\
[J_i, K_j] &= i \epsilon_{ijk} K_k 
\,.
\end{align}

Spinors will transform with 
%
\begin{align}
\xi ' &= \Lambda_\xi \xi  \\
\Lambda _\xi &= \exp(i \frac{\vec{\sigma}}{2} \cdot \vec{J} + ?)  \\
\left[\frac{\sigma_i}{2}, \frac{\sigma _j}{2}\right] &= i \epsilon_{ijk} \frac{\sigma _k}{2}
\,,
\end{align}
%
so what do we need to add? We can do either \(\vec{k} = \pm i \vec{\sigma} /2\), so we get 
%
\begin{align}
\Lambda _\xi = \exp(i \frac{\vec{\sigma}}{2} \cdot \vec{\omega} \pm \vec{u} \cdot \frac{\vec{\sigma}}{2})
\,.
\end{align}

The plus sign is for right-handed spinors, the minus sign is for left-handed ones.
If we have a spinor at rest, we cannot determine its helicity. 
If we boost it, we still cannot determine it! 

But, it cannot be in a superposition of things which transform in different ways. 

The idea is then to have a direct sum, an object which contains both the left and right-handed components: 
%
\begin{align}
\xi = \left[\begin{array}{c}
\xi _L \\ 
\xi _R
\end{array}\right]
\,.
\end{align}

These will be given by boosting the rest-frame spinor \(\xi \) in two different ways: 
%
\begin{align}
\xi'_R &= \exp( \vec{u} \cdot \frac{\vec{\sigma}}{2}) \xi  \\
\xi'_L &= \exp( - \vec{u} \cdot \frac{\vec{\sigma}}{2}) \xi 
\,.
\end{align}

The Dirac equation is a relation between the left- and right-handed components of a free spinor. 
In order to derive it, we need some relations: 
%
\begin{align}
\exp(\vec{u} \cdot \frac{\vec{\sigma}}{2}) &= \cosh (u /2 ) + \hat{u} \cdot \vec{\sigma} \sinh (u/2)  \\
(\vec{\sigma} \cdot \vec{a}) (\vec{\sigma} \cdot \vec{b}) &= \vec{a} \cdot \vec{b} + i \vec{\sigma} \cdot \left(\vec{a} \times \vec{b}\right)  \\
\exp(i \vec{\omega } \cdot \frac{\vec{\sigma}}{2}) &= \cos(\theta /2) + i \hat{u} \cdot \vec{\sigma} \sin (\theta /2)
\,.
\end{align}

With these, we get 
%
\begin{align}
\exp(\pm \vec{u} \cdot \frac{\vec{\sigma}}{2}) &= \frac{E + m \pm \vec{p} \cdot \vec{\sigma}}{\sqrt{2 m (E+m)}}  \\
\xi_{R, L} &= \frac{E + m \pm \vec{p} \cdot \vec{\sigma}}{\sqrt{2 m (E+m)}} \xi  \\
\xi &= \frac{E + m \mp \vec{p} \cdot \vec{\sigma}}{\sqrt{2 m (E+m)}} \xi_{R, L}
\,,
\end{align}
%
therefore we can put these equations together into 
%
\begin{align}
- m \xi_{R, L} + \left(E \pm \vec{p} \cdot \vec{\sigma} \right) \xi_{L, R} = 0
\,,
\end{align}
%
or 
%
\begin{align}
\left[\begin{array}{cc}
-m & E + \vec{p} \cdot \vec{\sigma} \\ 
E - \vec{p} \cdot \vec{\sigma} & -m
\end{array}\right]
\left[\begin{array}{c}
\xi _R \\ 
\xi _L
\end{array}\right]
= 0
\,.
\end{align}


If \(m = 0\), these decouple into two equations: 
%
\begin{align}
(p \pm \vec{p} \cdot \vec{\sigma}) \xi_{L, R} = 0
\,.
\end{align}

Therefore, for a massless particle 
%
\begin{align}
\left(\hat{p} \cdot \sigma \right) = \pm \xi _{R, L}
\,.
\end{align}

This is not the case when the particle is massive. 

Helicity is the expectation value of \(\hat{p} \cdot \vec{\sigma}\),
chirality is ``being \(\xi_{L}\) or \(\xi_R\)''. 

This is all in momentum space, but it can also be written in position space
by switching the momentum to a derivative: we introduce the Dirac matrices 
%
\begin{align}
\gamma^{0} = \left[\begin{array}{cc}
0 & \mathbb{1}_2 \\ 
\mathbb{1}_2 & 0
\end{array}\right]
\qquad \text{and} \qquad
\vec{\gamma} = \left[\begin{array}{cc}
0 & -\vec{\sigma} \\ 
\vec{\sigma} & 0
\end{array}\right]
\,,
\end{align}
%
so that the Dirac equation becomes 
%
\begin{align}
(\gamma^{\mu }p_\mu - m ) \psi = 0
\,,
\end{align}
%
where \(\psi = (\xi _L, \xi _R)^{\top}\). 

This can then also be written in terms of derivatives as \((i \gamma^{\mu } \partial_{\mu } - m) \psi = 0\).

Left- and right-handed spinors transform in the same way under rotations; they differ under boosts. 

It is useful to introduce 
%
\begin{align}
\gamma^{5} = i \gamma^{0} \gamma^{1} \gamma^{2} \gamma^{3}
\,:
\end{align}
%
if we introduce 
%
\begin{align}
P_{L, R} = \frac{1 \mp \gamma^{5}}{2}
\,
\end{align}
%
this will select the left- or right-handed component of a spinor.

We have derived this result using the Weyl basis; Dirac used a different one. 

If the spinor transforms with \(\psi \to T \psi \) the Dirac matrices transform with \(\gamma^{\mu } = T \gamma^{\mu } T^{-1}\). 

In the Dirac basis, which is useful when one studies non-relatvistic particles, 
the matrices look like 
%
\begin{align}
\gamma^{0} = \left[\begin{array}{cc}
\mathbb{1}_2 & 0 \\ 
0 & -\mathbb{1}_2
\end{array}\right]
\qquad \text{and} \qquad
\vec{\gamma} = \left[\begin{array}{cc}
0 & \vec{\sigma} \\ 
- \vec{\sigma} & 0
\end{array}\right]
\,,
\end{align}
%
while 
%
\begin{align}
\gamma^{5} = \left[\begin{array}{cc}
0 & \mathbb{1} \\ 
\mathbb{1} & 0
\end{array}\right]
\,.
\end{align}

In the nonrelativistic limit \(\vec{p} \to 0\) the equation decouples into 
%
\begin{align}
\left[\begin{array}{cc}
E-m & 0 \\ 
0 & E+m
\end{array}\right]
\left[\begin{array}{c}
\xi  \\ 
\eta 
\end{array}\right] 
= 0 
\,,
\end{align}
%
so we have \(E = \pm m\) solutions. 
Note that \(\xi \) and \(\eta \) are not left- and right-handed, this is a different basis.
In Dirac's time this was seen as a tragedy. 

The fact that these are actually particles and antiparticle can be properly explained in QFT; 
in the end, the four degrees of freedom correspond to the left- and right-handed components of
particles and antiparticles. 

We can write the particle and antiparticle solutions as 
%
\begin{align}
\psi _p = \left[\begin{array}{c}
\xi  \\ 
\frac{\vec{\sigma} \cdot \vec{p}}{E+m} \xi 
\end{array}\right]
e^{-i p_\mu x^{\mu }}
\qquad \text{and} \qquad
\psi _A = \left[\begin{array}{c}
\frac{\vec{\sigma} \cdot \vec{p}}{E+m} \eta  \\
\eta   \\ 
\end{array}\right]
e^{+i p_\mu x^{\mu }}
\,.
\end{align}

It is convenient to introduce a charge-conjugation operator: 
%
\begin{align}
C(\psi ) = \psi^{c} = i \gamma^{2} \psi^{*}
\,.
\end{align}

This can be seen by proving that \(C( \psi _P) = \psi _A\). 
Also, if we look at the minimally-coupled Dirac equation for a particle 
in an external EM field, and we apply the conjugation operator we find that 
the particle will obey the same equation, but with an opposite charge. 

The last thing to introduce here is the adjoint spinor: we know how a 4-spinor \(\psi \) behaves
under a Lorentz transformation. 
What is the object which transforms with the inverse transformation? If we denote it as \(\overline{\psi}\), 
we will be able to write \emph{invariant} objects like \(\overline{\psi}\psi  \). 

It takes some time to prove, but it comes out to be 
%
\begin{align}
\overline{\psi} &= \psi^{\dag} \gamma^{0}
\,.
\end{align}

We can make invariant objects like \(\overline{\psi} \psi\), 
or objects like \(\overline{\psi} \gamma^{\mu } \psi \): it can be proven that the latter is 
\emph{divergenceless}, \(\partial_{\mu } (\overline{\psi} \gamma^{\mu } \psi ) = 0\), so it can be used 
to describe currents.

\end{document}