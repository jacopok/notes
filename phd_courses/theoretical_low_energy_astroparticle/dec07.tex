\documentclass[main.tex]{subfiles}
\begin{document}

\section{The standard electroweak model}

\subsection{Gauge theories}

\marginpar{Tuesday\\ 2021-12-7}

Yesterday we wrote the equation for a free fermion: 
%
\begin{align}
\xi_{R, L} ' = \exp(- i \vec{\omega} \cdot \frac{\vec{\sigma}}{2} \pm \vec{u} \cdot \frac{\vec{\sigma}}{2}) \xi _{R, L}
\,.
\end{align}

The Dirac equation tells us that these are coupled: 
%
\begin{align}
\left[\begin{array}{cc}
-m & E + \vec{p} \cdot \vec{\sigma} \\ 
E - \vec{p} \cdot \vec{\sigma} & m
\end{array}\right]
\left[\begin{array}{c}
\xi _R \\ 
\xi _L
\end{array}\right]
= 0 
\,.
\end{align}

For nonzero mass these components are coupled, while in the zero-mass case they are decoupled and helicity is equal to chirality. 

Helicity is the eigenvalue under \(\hat{p} \cdot \vec{\sigma}\), chirality is the eigenvalue under \(\gamma^{5}\). 

\todo[inline]{What is the meaning of \(\gamma^{5}\) in position space? Is it a parity transformation?}

The coupling of a fermion to an external EM field can be represented with \(\partial_{\mu } \to \partial_{\mu } + iq A_\mu \). 

The SM Lagrangian can be written on a mug as  
%
\begin{align}
\mathscr{L} = - \frac{1}{4} F^{\mu \nu } F_{\mu \nu } 
+ i \overline{\psi} \slashed{D} \psi + \text{h.c.}
+ i \psi_i y_{ij} \psi_j \phi  + \text{h.c.}
+ \abs{\slashed{D} \phi }^2 - V(\phi )
\,.
\end{align}

We have scalars \(\phi \), fermions \(\psi \) and spin-1 gauge fields \(A_\mu \). 

From a Lagrangian \(L (q, \dot{q})\) we get Lagrange equations 
%
\begin{align}
\dv{}{t} \pdv{L}{\dot{q}} - \pdv{L}{q} = 0
\,.
\end{align}

If the Lagrangian is, say, \(L = m \dot{q}^2 / 2 - V(q)\) we get Newton's law \(m \ddot{q} = - \vec{\nabla} V = F\). 

This is classical; in field theory we have Lagrangians in the form 
%
\begin{align}
\partial_{\mu } \pdv{\mathscr{L}(\phi , \partial \phi )}{ \partial_{\mu } \phi _i} - \pdv{\mathscr{L}}{\phi _i} = 0
\,.
\end{align}

We will write the Yukawa Lagrangian with terms like \(\psi \phi \), and the QED Lagrangian with terms like \(\psi A_\mu \). 

The equation of motion for a scalar field is the Klein-Gordon equation: \((\partial_{\mu } \partial^{\mu } + m^2) \phi = 0\). 
The Lagrangian giving rise to this is 
%
\begin{align}
L = \frac{1}{2} \partial_{\mu } \phi \partial^{\mu } \phi - \frac{1}{2} m \phi^2
\,.
\end{align}

We have written the EOM for a free fermion, the Dirac equation \(\left(i \gamma^{\mu } \partial_{\mu } - m\right) \psi = 0\); the Lagrangian giving it is the Dirac Lagrangian 
%
\begin{align}
\mathscr{L} = \overline{\psi} \left(i \gamma^{\mu } \partial_{\mu } - m\right) \psi 
\,.
\end{align}

Again, we have a mass-like term and a kinetic-like term.

The term \(m \overline{\psi} \psi\) can be expanded into left and right components: 
%
\begin{align}
m \overline{\psi} \psi = m \left( \overline{\psi}_R \psi _L + \overline{\psi}_L \psi _R\right)
\,.
\end{align}

What about the EM field? \(A_\mu \) is gauge-dependent, but 
%
\begin{align}
F_{\mu \nu } = 2 \partial_{[\mu  } A_{\nu ]}
\,
\end{align}
%
is not. The Lagrangian giving Maxwell's equations in a vacuum (so, \(\square A^{\mu } = 0\) in the appropriate gauge) is 
%
\begin{align}
\mathscr{L} = - \frac{1}{4} F_{\mu \nu } F^{\mu \nu }
\,.
\end{align}

The absence of a mass term means that the photon is massless. 

What about interactions? 
We start from the Yukawa interaction between a fermion and a scalar. 
The Lagrangian, for starters, must contain the free terms for both: 
%
\begin{align}
\mathscr{L} = \underbrace{\overline{\psi} \left(i \gamma^{\mu } \partial_{\mu } - m_\psi \right) \psi + \frac{1}{2} \left( \partial_{\mu } \phi \partial^{\mu } \phi -  m_\phi  \phi^2 \right) }_{\text{free terms}} - \underbrace{g \overline{\psi} \psi \phi}_{\text{interaction}} 
\,.
\end{align}

The pictorial way to represent this is to draw quadratic terms for fermions like straight lines with an arrow, scalar fields as dashed lines, and cubic interaction terms like vertices. 

What do fermion-photon interactions look like?  
%
\begin{align}
\mathscr{L} _{\text{QED}} = \overline{\psi} \left(i \gamma^{\mu } \partial_{\mu } - m_\psi \right) \psi - \frac{1}{4} F^{\mu \nu } F_{\mu \nu } - q \overline{\psi} \gamma^{\mu } \psi A_\mu 
\,.
\end{align}

The EOM for the electromagnetic field are  Maxwell's equations with an external current \(j^{\mu } = q \overline{\psi} \gamma^{\mu } \psi \). 

This is the same Lagrangian we get if we do a minimal coupling substitution \(\partial \to \partial + iqA\). 
It is invariant under gauge transformations. 

These diagrams can be used to compute scattering amplitudes perturbatively. 

The way to get a cross-section is to multiply the square modulus of the amplitude by the phase space term. 
The idea to get decay rates is similar. 

The guiding principle to describe EW interactions is gauge invariance. 
The Yukawa Lagrangian has global phase invariance; what happens if we try to make a transformation \(\psi \to e^{- i q \alpha (a)} \psi \)?
The term \(q\) is only introduced here for later convenience.

The timezone analogy for local gauge invariance! 
If we have the freedom to choose a phase locally, we must have carriers of information moving at the maximum possible speed, otherwise processes would be disrupted.

We start with \(U(1)\) gauge invariance, as written above, but we will also need \(SU(2)\) invariance, written as 
%
\begin{align}
\exp(- ig \vec{\theta} \cdot \vec{T})
\,,
\end{align}
%
where \(\vec{T} = \vec{\sigma} / 2\), but we write them differently to not confuse them with spacetime rotations. 

Introducing gauge invariance for a term \(\overline{\psi} \left(i \slashed{\partial} - m \right) \psi \) will take us to the QED Lagrangian.

The mass term is already invariant, the problem is the kinetic one. 
The trick is to introduce a covariant derivative 
%
\begin{align}
\text{D}_\mu = \partial_{\mu } + iq A_\mu 
\,.
\end{align}

If \(\psi \to e^{-iq \alpha (x)} \psi \), also \(\text{D}_\mu \psi \to e^{-iq \alpha (x)} \text{D}_\mu \psi \), as long as \(A_\mu \) also transforms like \(A_\mu \to A_\mu + \partial_{\mu } \alpha \).

Then, the Lagrangian 
%
\begin{align}
\mathscr{L} = \overline{\psi} \left( i \slashed{\text{D}} - m \right) \psi  - \frac{1}{4} F^{\mu \nu } F_{\mu \nu }
\,.
\end{align}

A term like \(m^2 A_{\mu } A^{\mu }\) would violate gauge invariance, so this principle also gives an explanation as to why the photon is massless if we accept the gauge invariance principle. 

Protons and neutrons were thought to be part of a doublet under isospin SU(2) transformations. 

But, we have interactions like protons and neutrons interacting with electrons and neutrinos. 
So, one might think that electrons + neutrinos have isospin-like SU(2) symmetry as well. 

We want to make the doublet \(\psi_1 , \psi_2 \) invariant under SU(2) transformations as written above: how do we do it? 
Again, we redefine the derivative: 
%
\begin{align}
\partial_{\mu } \to \text{D}_\mu - ig \vec{T} \cdot \vec{A}_\mu 
\,,
\end{align}
%
so we need to introduce three fields \(\vec{A}_\mu \). These are the gauge bosons related to SU(2) gauge invariance. 

These now transform like 
%
\begin{align}
\vec{A}_\mu \to \vec{A}_\mu - \partial_{\mu } \vec{\theta}(x) + g \vec{\theta} \times \vec{A}_\mu 
\,.
\end{align}

The presence of this coupling means that the field is charged. 

The field tensor here is 
%
\begin{align}
\vec{F}_{\mu \nu } = 2 \partial_{[\mu } A_{\nu ]} + g \vec{A}_\mu \times \vec{A}_\nu 
\,.
\end{align}

We then get a Lagrangian like 
%
\begin{align}
\mathscr{L} = \overline{\psi} \left( i \gamma^{\mu } \text{D}_\mu -m\right) \psi - \frac{1}{4} \vec{F}^{\mu \nu } \cdot \vec{F}_{\mu \nu }
\,.
\end{align}

In the \(\psi \) term we have \(\psi \psi \) terms, as well as \(\psi \psi A\) interactions.  
In this \(F^2\) term we have quadratic, trilinear and quadrilinear terms in \(A\)! 

The actual group for the EW theory is \(SU(2)_L \otimes U(1)_Y\). 
The simplest EW world contains 
\begin{enumerate}
    \item 1 massive electron, with both right- and left-handed components;
    \item 1 massless \(\nu _e\), with only the left-handed component;
    \item EM interactions (we want to have a diagram describing how the electron interacts with a massless photon); 
    \item weak chiral interactions like \(\nu _L e_L W^{\pm}\), where \(W^{\pm}\) is massive.
\end{enumerate}

The Higgs mechanism accomplishes this, and it also gives mass to fermions. 
If neutrinos have Majorana masses, the Higgs mechanism cannot give them mass. 

Let us define a doublet \(L=(e_L, \nu _{e, L})\), and a singlet \(R = e_R\). 
The theory must then be invariant if we redefine 
%
\begin{align}
L &\to L' = \exp(-i g \vec{\theta}(x) \cdot \vec{T}) L  \\
R &\to R' = R
\,.
\end{align}

The doublet \(L\) corresponds to the \(T_3 = \pm 1/2\) quantum numbers, while \(T_3 = 0\) for the singlet. 

Could the three bosons be some combinations  of the photon and the two \(W^{\pm}\) bosons? We know that the charge operator \(Q\) has eigenvalues \(0\), \(-1\)  for the doublet, but \(\Tr Q \neq 0\) while \(\Tr T_i = 0\) for all \(T_i\). 

Why would \(Q\) need to be able to be written as a function of the \(T_i\)? 
The proper answer lies in Nöther's theorem. 

We therefore introduce a further hypercharge symmmetry: 
%
\begin{align}
L' &= \exp(- i g' \beta (x) \frac{Y}{2}) L \\
R' &= \exp(- i g' \beta (x) \frac{Y}{2}) R
\,,
\end{align}
%
so that this generator commutes with the \(T_i\), and indeed \([T_i, Y] = 0\). The only way to have this is \(Y \propto Q - T_3\), and indeed \(Y = 2 (Q - T_3)\). 

In the end, the Lagrangian must be chargeless. 

The full transformation is therefore 
%
\begin{align}
L' = \exp(-i g \vec{\theta}(x) \cdot \vec{T} - ig' \beta (x) \frac{Y}{2}) L \\
R' = \exp( - ig' \beta (x) \frac{Y}{2}) R
\,.
\end{align}

The procedure is then like before: we need to redefine the derivative, as 
%
\begin{align}
\text{D}_\mu L =\left(  \partial_{\mu } + i g \vec{T} \cdot \vec{A}_\mu - i g' \frac{Y}{2} B_\mu  \right) L
\,,
\end{align}
%
so we have four fields, the three \(\vec{A}_\mu \) with their \(\vec{F}_{\mu \nu }\) and \(B_\mu \) with its \(G_{\mu \nu }\). 

The Lagrangian will then read 
%
\begin{align}
\mathscr{L} = 
\overline{L} i \gamma^{\mu } \text{D}_\mu L 
+ 
\overline{R} i \gamma^{\mu } \text{D}_\mu R + \text{kinetic}
\,,
\end{align}
%
but we cannot write mass terms like \(m \overline{\psi} \psi \), which would be \(\overline{L} R\) or \(L \overline{R}\): the matrix dimensions don't match up! 

So, everything's massless: the Higgs mechanism comes to the rescue. 
It gives masses to all the bosons except the photon, as well as giving mass to the leptons. 

Suppose we have a \(U(1) \otimes U(1)\) symmetry, spontaneously broken to \(U(1)\). The thing we think of is a pencil about to fall on a table. 



\end{document}
