\documentclass[main.tex]{subfiles}
\begin{document}

\marginpar{Thursday\\ 2021-12-9}

We have discussed the Dirac equation and SU(2) gauge invariance; 
now we will look at the Higgs mechanism and SSB, with the goal in mind
to understand the Lagrangian written on the CERN mug. 

We study this model in the context of an \(L\) doublet \((\nu _{eL}, e_L)\) and an \(R\) singlet \((e_R)\). 

The Higgs mechanism yields neutral currents as a ``bonus''. 

So far, we have written the gauge field Lagrangian: 
%
\begin{align}
\mathscr{L} _{\text{GF}} = - \frac{1}{4} \vec{F}_{\mu \nu } \cdot \vec{F}^{\mu \nu } - \frac{1}{4} G_{\mu \nu } G^{\mu \nu }
\,,
\end{align}
%
and the lepton kinetic Lagrangian: 
%
\begin{align}
\mathscr{L} _{\text{F}} &= \overline{L} (i \gamma^{\mu } \text{D}_\mu ) L + \overline{R} (i \gamma^{\mu } \text{D}_\mu ) R  \\
\text{D}_\mu L &= \left(\partial_{\mu } - ig \vec{T} \cdot \vec{A}_\mu  - ig' \frac{Y}{2} B_\mu \right)L \\
\text{D}_\mu R &= \left(\partial_{\mu } - ig' \frac{Y}{2} B_\mu \right)R
\,.
\end{align}

We want to have a breaking like \(\text{SU}(2)_L \otimes \text{U}(1)_Y \to \text{U}(1)_Q\), where we want the ground state \(Q\) to be chargeless (since it represents the photon).

In order to have a Dirac-like mass term, \(m \overline{\psi}_L \psi _R\), we need to saturate a doublet with a singlet: it cannot be done! 
So, we need a new field \(\phi \) which is at least a doublet, and then we can make a term \(\overline{L} \phi  R\). 

A complex field \(\phi \in \mathbb{C}^2\) works, and it must be chargeless for things to work.  
The Yukawa Lagrangian will then be 
%
\begin{align}
\mathscr{L} _{\text{YUK}} = - y_e \left(\overline{L} \phi R + \overline{R} \phi ^\dag L\right)
\,.
\end{align}

A mass term for the field \(\phi \) in the form \(m^2 \phi ^\dag \phi \) would be stable, but we want to make it \emph{un}stable at \(\phi = 0\), o we need a \(\phi ^\dag \phi \) term with a negative coefficient and a \((\phi ^\dag \phi )^2\) term with a positive coefficient: 
%
\begin{align}
V(\phi ) = - \mu^2 (\phi \phi ^\dag) + \lambda (\phi \phi ^\dag)^2
\,.
\end{align}

The minimum of this potential is reached when \(\phi \phi ^\dag = (1/2 ) \mu^2 / \lambda = (1/2) v^2\).

\todo[inline]{Why \(\phi \phi ^\dag\)? That's 2x2\dots}

So far, we have introduced the couplings \(g\) and \(g'\), the parameters \(\mu \) and \(v\) for the Higgs field potential, and the Yukawa coupling \(y_e\). 

The connection to experiment will be to write in terms of these parameters \(e\), \(G_F\), \(M_Z\), \(M_H\), \(m_e\). 

\todo[inline]{So a 5 parameter fit for 5 measurements?}

The \(\phi \) field near the minimum can be parametrized as 
%
\begin{align}
\phi = \left[\begin{array}{c}
0 \\ 
\frac{v + H}{\sqrt{2}}
\end{array}\right]
\,.
\end{align}

The upper, charged part is equal to zero. 
In the Yukawa Lagrangian we find 
%
\begin{align}
\mathscr{L} _{\text{YUK}} &= - y_e \left[ 
    \left[\begin{array}{cc}
    \overline{\nu}_{eL} & e_L
    \end{array}\right]
    \left[\begin{array}{c}
    0 \\ 
    \frac{v + H}{\sqrt{2}}
    \end{array}\right]
    e_R 
    + 
    \overline{e}_R 
    \left[\begin{array}{cc}
    0 & \frac{v + H}{\sqrt{2}}
    \end{array}\right]
    \left[\begin{array}{c}
    \nu_{eL} \\ 
    e_L
    \end{array}\right]
 \right]  \\
&= - \frac{y_e}{\sqrt{2}} \left(v + H\right)
\left( \overline{e}_L e_R + \overline{e}_R e_L\right)  \\
&= - m_e \overline{e} e - \frac{y_e v}{\sqrt{2}} H \overline{e} e
\,.
\end{align}

We have not obtained any mass for the neutrino, as we expected by not having a \(\nu _R\) term. 

The proportionality of the Yukawa coupling to the masses has been seen experimentally. 

In general the degrees of freedom of the \(\phi \) can be written as 
%
\begin{align}
\phi = \exp(i \vec{T} \cdot \frac{\vec{\sigma}}{2}) \left[\begin{array}{c}
0 \\ 
\frac{v + H}{\sqrt{2}}
\end{array}\right]
\,,
\end{align}
%
so we can use SU(2) gauge invariance to select a ground state; the three remaining degrees of freedom become Goldstone bosons and are ``eaten'' by the three \(W^{\pm}_{\mu }\), \(Z^{0}_{\mu }\) bosons to give them a third, longitudinal polarization. 

The Higgs sector of the Lagrangian becomes 
%
\begin{align}
\mathscr{L} _{\text{H}} = \left( \text{D}_\mu \phi ^\dag\right)
\left( \text{D}_\mu \phi \right) - V(\phi )
\,,
\end{align}
%
which contains the bosons. 
What we get if we diagonalize this Lagrangian is terms written in terms of the new fields
%
\begin{align}
W^{\pm}_{\mu } = \frac{1}{\sqrt{2}} \left( A^{1}_{\mu } \pm A^{2}_{\mu } \right)
\,,
\end{align}
%
and 
%
\begin{align}
\left[\begin{array}{c}
Z_\mu  \\ 
A_\mu 
\end{array}\right]
= \left[\begin{array}{cc}
\cos \theta _W & - \sin \theta _W \\ 
\sin \theta _W & \cos \theta _W
\end{array}\right]
\left[\begin{array}{c}
A^{3}_{\mu } \\ 
B_\mu 
\end{array}\right]
\,,
\end{align}
%
where \(\theta _W\) is the Weinberg angle, chosen such that \(\tan \theta _W = g' / g\). 

The masses of these bosons come out to be 
%
\begin{align}
M_W^2 = \frac{v^2}{4} g^2 \qquad \text{and} \qquad
M_Z = \frac{v^2}{4} (g^2 + g^{\prime 2})
\,,
\end{align}
%
while \(m_\gamma = 0\). 

We get cubic couplings like \(HZZ\) but also quartic ones like \(HHZZ\) or \(HHWW\). 

The mass terms for the Higgs is \(m_H = \sqrt{2} \mu \). 

The photon does not couple directly to the Higgs, but we can have a \(H \to \gamma \gamma \) process through loops.

When we rewrite the gauge field Lagrangian in terms of the physical fields \(A_\mu \), \(W_\mu \) and \(Z_\mu \), we get quadratic, cubic and quartic terms in the gauge terms. 

The terms we get are \(\gamma WW\), \(Z WW\), as well as the quartic \(\gamma \gamma WW\), \(WWWW\), \(ZZWW\), \(\gamma ZWW\). 

What about the fermion field Lagrangian?
We expect to get terms in the form \(X^{\mu } J_\mu \), where \(X^{\mu }\) is a gauge field while \(J_\mu \) is a fermion current. 

The result is 
%
\begin{align}
\mathscr{L} _{\text{F}} = \mathscr{L} _{\text{electromagnetic}} + \mathscr{L} _{\text{charged current}} + \mathscr{L} _{\text{neutral current}}
\,.
\end{align}

The electromagnetic term is 
%
\begin{align}
\mathscr{L} _{\text{electromagnetic}} = \underbrace{\frac{gg'}{\sqrt{g^2 + g^{\prime 2}}}}_{e} J^{\mu }_{EM} A_\mu 
\,,
\end{align}
%
where \(e\) is the electric charge; the other terms read 
%
\begin{align}
\mathscr{L} _{\text{charged current}} &= \frac{g}{\sqrt{2}} J^{\mu}_{\pm} W^{\mp} \\
\mathscr{L} _{\text{neutral current}} &= \frac{g}{\cos \theta _W} J^{\mu}_{NC} Z_\mu  
\,.
\end{align}

The electromagnetic current will be \(J^{\mu }_{EM} = \overline{e} \gamma^{\mu } Q e\) (by construction: we made the theory to reproduce Maxwell's equations). 

The charged current is \(J^{\mu }_{+} = \overline{e}_L \gamma^{\mu } \nu_{eL}\) and \(J^{\mu }_{-} = \overline{\nu}_{eL} \gamma^{\mu } e_L\). 

These were the already-observed parts, while the new one is 
%
\begin{align}
J^{\mu }_{NC} = \overline{\nu}_{eL} \gamma^{\mu } T_3 \nu_{eL} 
+ \overline{e}_L \gamma^{\mu } \left( T_3 - Q s^2_W  \right) e_L 
+ \overline{e}_R \gamma^{\mu } (-  Q s^2 _W ) e_R
\,.
\end{align}

This term is carrying a current \(T_3 - Q s^2_W\), where \(s^2_W = \sin^2 \theta _W\).
The generator \(T_3\) is just half of the \(\sigma _3\) Pauli matrix. 

The phenomenology of this prediction is quite rich. 

We have couplings in the form \(\gamma^{\mu } P_{L, R}\), there is specific jargon to describe these interactions. 
\begin{enumerate}
    \item Vector (\(V\)) currents are those in the form \(\overline{\psi} \gamma^{\mu } \psi \), which do not change sign under a parity transformation;
    \item Axial (\(A\)) currents are those in the form \(\overline{\psi} \gamma^{\mu } \gamma^{5} \psi \), which \emph{do} change sign under a parity transformation.
\end{enumerate}

EM interactions are of type \(V\), charged currents are of type \(V - A\) (or, \(1 \pm \gamma^{5}\)), while neutral currents are of the form \(g_V V + g_A A\). 

We can define left-handed couplings in the neutral current \(g_L = T_3 - Q s^2_W\), as well as right-handed ones like \(g_R = - Q s^2_W\) (since the action of \(T_3\) on an \(R\) singlet is zero). 

The vector and axial couplings are then defined as 
%
\begin{align}
g_V &= g_L + g_R = T_3 - 2 Q s_W^2  \\
g_A &= g_L - g_R = T_3
\,.
\end{align}

When a short-range gauge boson mediates an interaction, its contribution will be approximately \(1/M^2\) at low energies. 

So, both processes which are mediated by \(W^{\pm}\) and \(Z\) must give the correct low-energy limit.

Specifically, we find 
%
\begin{align}
G_F \sim \frac{g^2}{M^2} \qquad \text{or} \qquad
\frac{g^2}{8 M_Z^2 \cos^2 \theta _W} = \frac{g^2}{8 M_W^2} = \frac{G_F}{\sqrt{2}}
\,.
\end{align}

In the notes there are qualitative considerations about the amplitudes in \(\beta \) decay, getting the energy spectrum of the electron. 

A charged current process is \(\mu \to \nu _\mu \overline{\nu}_e e\). The Fermi constant is best estimated through this process, since it is very well-known both theoretically and experimentally. 

The decay \(\pi \to \ell \overline{\nu}_e\) probes the \(V-A\) nature of the interaction. 

Another interesting process is \(\nu _\mu e\) scattering, which allows us to estimate \(s_W^2\). 

We can have the recoil of an entire nucleus with a low-energy neutrino, which is mediated by a \(Z\) boson, Coherent Elastic \(\nu \) Nucleus Scattering, CE\(\nu \)NS. 

This process turns out to be proportional to the number of neutrons squared, which goes up quickly --- this is the only tabletop neutrino detector. 

The idea is that the neutrino's wavelength is very long, so it cannot see the specific nucleons or quarks. 

We separate out the parameters \(g\), \(g'\), \(\mu \) and \(\nu \) from the Yukawa coupling of the electron --- the first four remain the same, while we will need to introduce more Yukawa coupling for the different families. 

The data are \(e\), \(G_F\), \(M_H\), \(M_Z\), as well as the mass of the electron. 

The relations are 
%
\begin{align}
e &= \frac{gg'}{\sqrt{g^2 + g^{\prime 2}}}  \\
G_F &= \frac{1}{\sqrt{2} v^2} \\
M_H &= \sqrt{2} \mu  \\
M_Z &= \frac{v}{2} \sqrt{g^2  + g^{\prime 2}}
\,,
\end{align}
%
as well as 
%
\begin{align}
m_e = \frac{y_e v}{\sqrt{2}}
\,.
\end{align}

These are all computed at tree-level; at more loops there can be corrections of the order of a few parts in a hundred. 
There are proposals to replace \(e\) with \(M_W\), so that all the terms refer to the same energy scale \(\sim \SI{100}{GeV}\). 

In natural units, 
%
\begin{align}
e &= \num{.3}  \\
G_F &= \SI{1.17e-5}{GeV^{-2}}  \\
M_H &= \SI{125}{GeV} \\
M_Z &= \SI{91}{GeV} \\
m_e &= \SI{.51e-3}{GeV}
\,,
\end{align}
%
\todo[inline]{\(e = .3\) corresponds to \(e^2 / 4 \pi = 1/137 = \alpha \); why does this not match with 
% (ac.e.gauss.value * u.cm**(3/2) * u.g**(1/2) / u.s * ac.hbar**(-1/2) * ac.c**(-1/2)).si
the astropy code? 
}
so 
%
\begin{align}
g &= \num{.64}  \\
g' &= \num{.34}  \\
\mu &= \SI{88}{GeV}  \\
v &= \SI{246}{GeV}  \\
y_e &= \num{2.9e-6} 
\,,
\end{align}
%
so 
%
\begin{align}
\lambda &= \num{.13}  \\
\sin^2 \theta_W &= \num{.22}  \\
M_W &= \SI{80}{GeV}
\,.
\end{align}

In terms of naturalness, it is weird that \(y_e \ll 1\), which tells us that \(m_e \ll \SI{100}{GeV}\). 



\end{document}