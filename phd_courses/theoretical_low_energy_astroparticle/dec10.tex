\documentclass[main.tex]{subfiles}
\begin{document}

\subsection{Fermion masses and mixing}

\marginpar{Friday\\ 2021-12-10}

The full symmetry group of the standard model is 
\(\text{SU}(3)_c \otimes \text{SU}(2)_L \otimes \text{U}(1)_Y\). 

The building blocks are doublets like \((\nu_{eL}, e_L)\) with charges respectively
\(Q = 0\) and \(Q= -1\), as well as \(T_3 = +1/2\) and \(T_3 = -1/2\). 

The right-handed singlets, on the other hand, also have \(Q = 0\) and \(Q=-1\) 
but \(T = 0\). 

Any right-handed neutrinos would be \emph{sterile}, since they do not interact. 

This also holds for muonic and tauonic flavors, 
as well as for quark doublets as long as we add \(2/3\) to all charges.

The relation between the quantum numbers is \(Y = 2 (Q-T_3)\). 

In general the doublets look like 
%
\begin{align}
\left[\begin{array}{c}
U^{\alpha } \\ 
D^{\alpha }
\end{array}\right]_L
\,,
\end{align}
%
where \(\alpha \) is a generation index.

The most general mass term looks like \(m \overline{\psi}_L \psi _R\) --- we should also 
consider its Hermitian conjugate, but it works analogously. 

As usual, we insert the Higgs field to saturate the doublets and diagonalize.
We will need to look at the effect of this change of basis on the mixing matrix.

The most general currents look like 
%
\begin{align}
J^{\mu }_{EM} &= \sum _{\alpha } \overline{U}^{\alpha } \gamma^{\mu } U^{\alpha } + (U \to D)  \\
J^{\mu }_{NC} &= \sum _{\alpha } \overline{U}^{\alpha }_{L} \gamma^{\mu } (T_3 - Q s^2_W) U^{\alpha }_{L} 
+ \sum _{\alpha } \overline{U}^{\alpha }_{R} \gamma^{\mu } (- Q s^2_W) U^{\alpha }_{R}  + (U \to D)  \\
J^{\mu }_{-} &= \sum _{\alpha} \overline{U}^{\alpha }_L \gamma^{\mu } D^{\alpha }_{L} (U \to D \text{ yields } J^{\mu }_{+})
\,.
\end{align}

The coupling terms then look like \(J^{\mu }_{EM} A_\mu \), \(J^{\mu }_{NC} Z_\mu \) and \(J^{\mu }_{\mp} W^{\pm}\). 

The various components will transform according to some unitary matrices: 
%
\begin{align}
D^{\alpha }_{R} &= W^{\alpha \beta } D^{\beta }_{R} 
\qquad \text{and} \qquad
D^{\alpha }_{L} = S^{\alpha \beta } D^{\beta }_{L}  \\
U^{\alpha }_{R} &= T^{\alpha \beta } D^{\beta }_{R} 
\qquad \text{and} \qquad
U^{\alpha }_{L} = R^{\alpha \beta } D^{\beta }_{L}  
\,.
\end{align}

Most of the currents are left unchanged by these transformations;
the charged current however is the one which changes: 
%
\begin{align}
J^{\mu }_{-} = \overline{U}_L \gamma^{\mu } \underbrace{R ^\dag S}_{\neq 1 \text{ in general}} D_L
\,.
\end{align}

The Yukawa Lagrangian will then include a term like 
%
\begin{align}
\mathscr{L} \ni 
\underbrace{\sum _{\alpha \beta } y^{\alpha \beta }_{D} \left[\begin{array}{cc}
\overline{U}^{\alpha } & \overline{D}^{\alpha }
\end{array}\right]_L
\left[\begin{array}{c}
0 \\ 
v / \sqrt{2}
\end{array}\right]
D^{\beta }_{R}}_{\text{D-type couplings}}
+ \underbrace{\sum _{ \alpha \beta } y^{\alpha \beta }_{U} 
\left[\begin{array}{cc}
\overline{U}^{\alpha } & \overline{D}^{\alpha }
\end{array}\right]_L
\left[\begin{array}{c}
v / \sqrt{2} \\ 
0
\end{array}\right]
U^{\beta }_R}_{\text{U-type couplings}}
\,.
\end{align}

We define a mass matrix 
%
\begin{align}
M^{\alpha \beta }_{U, D} = y^{\alpha \beta }_{U, D} \frac{v}{\sqrt{2}}
\,,
\end{align}
%
so that 
%
\begin{align}
\mathscr{L} \ni \overline{D}_L M_D D_R + \overline{U}_L M_U U_R + 
\text{h. c. }
\,.
\end{align}

How can we put the Higgs vacuum on the upper component? 
We can define \(\widetilde{\phi} = i \sigma _2 \phi^{*}\). 
For this field, the 1 component has zero charge while the 2 component has \(-1\) charge. 

If we introduce this field, the charge of the Yukawa Lagrangian is zero. 

Each of the \(M_{U, D}\) matrices will be diagonalized like \(S ^\dag M T = M _{\text{diag}}\), where \(S\) and \(T\) are both unitary. 

We then have, for down-type fields: \(\overline{D}_L M_D D_R\) going into 
%
\begin{align}
D_R \to W D_R \qquad D_L \to S D_L \qquad M_D \to M_D^{\text{diag}}
\,,
\end{align}
%
while \(\overline{U}_R M_U U_R\) becomes
%
\begin{align}
U_R \to T U_R \qquad U_L \to R U_L \qquad M_U \to M_U^{\text{diag}}
\,,
\end{align}
%


In general, these will be different matrices for leptons and for quarks. 

The charged current then reads \(J^{\mu }_{-} = \overline{U}_L \gamma^{\mu } V D_L\), where \(V = R ^\dag S\). 

More explicitly, we get 
%
\begin{align}
J^{\mu }_{-} \overline{U}^{\alpha }_{L} \gamma^{\mu } V^{\alpha \beta } D^{\beta }_{L}
\,.
\end{align}

The interaction of a \(\overline{U}^{\alpha }_L\) and \(D^{\beta }_{L}\) is then modulated by a factor \((g/\sqrt{2}) V^{\alpha \beta }\). 

If \(V\) were 2x2 it would be a real matrix, if it were 3x3 it would be a complex matrix.

For quarks, the values on the diagonal are close to 1, while the largest values off-diagonal are of the order \num{.22} for the CKM matrix, the mass-mixing matrix for quarks. 

For example, in \(\beta \) decay we get a modulation by a factor \(V_{ud}\), which is slightly less than 1. 

For leptons, the matrix is called PMNS. 

In the usual construction of the Standard Model, there were no mass terms for neutrinos or right-handed \(\nu_{\alpha R}\). 
Therefore, in this case there could be no mixing among leptons, since the mixing matrix only ever appears ``sandwitched''. 

The basic definition of neutrino flavor is ``which lepton does it have charged current interactions with''. 

For example, \(\nu _e\) is the \(\nu \) produced in \(\beta^{+} \) decay, and \(\overline{\nu}_e\) is the \(\nu \) produced in \(\beta^{-}\) decay. 

Let us then consider \(\nu _e \to \nu _\mu  \) oscillations. 
We always need to consider the amplitudes for production, propagation, detection. 

However, it is convenient to isolate the propagation into the matrix 
%
\begin{align}
\left[\begin{array}{c}
\nu _e \\ 
\nu _\mu  \\ 
\nu _\tau 
\end{array}\right]
= 
\left[\begin{array}{ccc}
U_{e1} & U_{e2} & U_{e3} \\ 
U_{\mu 1} & U_{\mu 2} & U_{\mu 3} \\ 
U_{\tau 1} & U_{\tau 2} & U_{\tau 3} 
\end{array}\right]
\,.
\end{align}

Forgetting the fact that this is a simplification can lead to paradoxes;
for example, ether leptons do not oscillate.

There is a basis for the Dirac equation for which \(i \gamma^{\mu }\) is real --- this leads to the fact that \(\psi = \psi^{*}\). 

This cannot work for electrons and positrons, since they are charged.
A neutral fermion, on the other hand, might be its own antiparticle.

If we have a \(\beta^{+}\) decay, we will produce a \(\nu _e\) which is left-handed. 
In a \(\beta^{-}\) decay, on the other hand, we will produce a right-handed \(\overline{\nu}_e\). 

If the neutrino is massless, this cannot change.
Formally, the Dirac equation decouples. 

This ``Weyl'' case is not completely excluded by current phenomenology; the lightest of the neutrinos could be completely massless. 

If we have massive neutrinos, though, there will be an \(\order{m_\nu / E}\) component with the opposite chirality. 

However, the neutrino and antineutrino are still distinguishable.

If they are Majorana, on the other hand, \(\nu_e = \overline{\nu}_e\); this must mean that they are neutral not just for electric charge but for all charges. 

We have observed inverse beta decay:
%
\begin{align}
\nu _e + \ce{n} \to \ce{p} + e^{-} 
\qquad \text{and} \qquad
\overline{\nu} _e + \ce{p} \to \ce{n} + e^{+}
\,,
\end{align}
%
but not the same reaction for \(\nu _e \leftrightarrow \overline{\nu}_e\). 
Is this an indication that neutrinos are indeed Dirac? 

Well, if neutrinos are indeed Majorana these reactions are possible, but suppressed at order \(m_\nu / E\)! 

The experiment which saw the largest number of neutrinos collected about 2 million events\dots

The way to have good statistics is to look at decays, specifically neutrinoless double-beta decay. 

The decay looks like 
%
\begin{align}
\{ 2 n \} \to
\{ 2 p \} + 2 e^{-} 
\,.
\end{align}

The decay of the first proton looks like the emission of a right-handed \(\overline{\nu}_e\) with an \(e^{-}\), through a charged-current interaction. 

IF \(m_\nu \neq 0\), the neutrino has a left-handed component (not possible if the neutrino is Weyl); 
if \(\nu = \overline{\nu}\), this left-handed component is a left-handed \(\nu _e\) (not possible if the neutrino is Dirac)! 

At the second proton, the left-handed \(\nu _e\) is absorbed and an \(e^{-}\) is emitted. 

This process is suppressed by \(m_\nu / E\): it probes absolute neutrino mass. 

Current neutrinoless double beta decay are probing lifetimes of the order of \(\num{e26}\) to \(\SI{e27}{yr}\). 

It can be shown that if this process takes place, then neutrinos must be Majorana. 

Pictorially, one can join the legs of the \(0 \nu \beta \beta \) diagram to find a diagram which turns a \(\nu _e\) into a \(\overline{\nu}_e\). 

\end{document}

