\documentclass[main.tex]{subfiles}
\begin{document}

\subsection{Mass terms for neutrinos}

\marginpar{Monday\\ 2021-12-13}

We will discuss the difference between Weyl, Dirac and Majorana neutrinos for 
the single-generation case. 
Remember that under a Lorentz transformation \(\Lambda (\vec{\omega}, \vec{n}) \) they change as 
%
\begin{align}
\phi_R \to \exp( i \vec{\omega} \cdot \frac{\vec{\sigma}}{2} + \vec{n} \cdot \frac{\vec{\sigma}}{2}) \phi _R \\
\phi_L \to \exp( i \vec{\omega} \cdot \frac{\vec{\sigma}}{2} - \vec{n} \cdot \frac{\vec{\sigma}}{2}) \phi _L
\,.
\end{align}

It can be shown that if \(\phi _R\) is right-handed, \(i \sigma_2 \phi_R^{*}\) is left-handed 
(and similarly with \(L \leftrightarrow R\)). 

We can therefore parametrize our full spinor in terms of two right-handed spinors, as 
%
\begin{align}
\Psi = \left[\begin{array}{c}
\phi _R \\ 
\phi _L
\end{array}\right]
=
\left[\begin{array}{c}
u \\ 
i \sigma_2 v^{*}
\end{array}\right]
\,.
\end{align}

Recall also the projectors 
%
\begin{align}
\psi_{L, R} = P_{L, R} \psi 
\qquad \text{where} \qquad
P_{L, R} = \frac{1 \mp \gamma^{5}}{2}
\,.
\end{align}

We also have \(\overline{\psi} = \psi ^\dag \gamma^{0}\) and \(\psi^{c} = C(\psi ) = i \gamma^{2} \psi^{*}\). 

The convention for the order of operations is as follows: 
\begin{enumerate}
    \item projectors \(P_{L, R}\) act before
    \item charge conjugation \(C( \cdot)\) which acts before
    \item conjugation \(\overline{\cdot}\).
\end{enumerate}

In terms of our parametrization \(\psi (u, v)\) the charge conjugate swaps \(u\) and \(v\):
%
\begin{align}
\psi^{c} = \left[\begin{array}{c}
v \\ 
i \sigma_2 u^{*}
\end{array}\right]
\,.
\end{align}

The Dirac case is the one in which \(u \neq v\), the Majorana case is the one in which \(u \simeq v\). 
The equality in the Majorana case can actually be relaxed including an arbitrary phase: 
\(\psi = \psi^{c} e^{i \phi }\) for some fixed \(\phi \), called the \emph{Majorana phase}.

The Weyl case is the one in which \(m = 0\), meaning that \(\psi = \psi _R\) or \(\psi = \psi _L\). 
The spinor just has two degrees of freedom. 

The Weyl-Majorana case would then be \(\psi = \psi _R + \psi _R^{c} = \psi^{c}\). 

The Dirac case is the general, massive, 4 degrees of freedom one. 

A Dirac mass term looks like \(m \overline{\psi} \psi\), 
a Majorana one would look like \((1/2) m \overline{\psi} \psi \).

In the Dirac case, this would read 
%
\begin{align}
\overline{\psi} \psi = \overline{\psi}_L \psi_R + \overline{\psi}_R \psi _L
\,,
\end{align}
%
while in the Majorana, left-handed case we would have 
%
\begin{align}
\overline{\psi} \psi = \overline{\psi}_L \psi _L^{c} + \overline{\psi}_L^{c} \psi _L
\,,
\end{align}
%
and the right-handed one is analogous. 

The most general Lagrangian would then include a term 
%
\begin{align}
\mathscr{L} &\ni m_D 
\left( 
    \overline{\psi}_L \psi _R + \overline{\psi}_R \psi _L
\right)
+ 
\frac{1}{2} m_L 
\left(
    \overline{\psi}_L^{c} \psi _L + \overline{\psi}_L \psi _L^{c}
\right)
+ 
\frac{1}{2} m_R
\left(
    \overline{\psi}_R^{c} \psi _R + \overline{\psi}_R \psi _R^{c}
\right)  \\
&= \frac{1}{2} \left[\begin{array}{cc}
\overline{\psi}_L + \overline{\psi}_L^{c}, &
\overline{\psi}_R + \overline{\psi}_R^{c}
\end{array}\right]
\left[\begin{array}{cc}
m_L & m_D \\ 
m_D & m_R
\end{array}\right]
\left[\begin{array}{c}
\psi _L + \psi_L^{c} \\ 
\psi _R + \psi _R^{c}
\end{array}\right]
\,,
\end{align}
%
which we can diagonalize! 
The eigenvectors will in general be linear combinations of Majorana fields. 

What does the Standard Model tell us about these masses? 
The \(m_D\) term is allowed, is comes from the SSB of the Yukawa terms as long as we have right-handed neutrinos. 

Specifically, it will be \(m_D \sim y_\nu v / \sqrt{2}\) where \(v \approx \SI{200}{GeV}\). 
In order for this to work we must have \(y_\nu \lesssim \num{e-11}\). 

In the SM we cannot have Majorana mass terms \(m_L\), since they break the
electroweak symmetry. 

\todo[inline]{Clarify this point}

The term \(m_R\), on the other hand (\(m_R \overline{\psi}_R \psi _R^{c} + \text{h. c.}\)), is allowed by the SM gauge group, but the \(m_R\) term is a new mass scale of the theory. 

The mass matrix can be thought to look like 
%
\begin{align}
\left[\begin{array}{cc}
m_L & m_D \\ 
m_D & m_R
\end{array}\right]
= 
\left[\begin{array}{cc}
0 & \sim v \\ 
\sim v &  \sim \Lambda 
\end{array}\right]
\,,
\end{align}
%
where \(\Lambda \) is a scale from Beyond-the-Standard-Model physics. 

These ideas came out when people were considering very large gauge groups. 

The diagonalization of that matrix yields a light neutrino \(\nu \sim \nu _L + \nu _L^{c}\) with mass of the order \(\order{v^2 / \Lambda } \ll v\), as well as a massive neutrino \(\nu \sim \nu _R + \nu _R^{c}\) with mass of the order \(\order{\Lambda }\). 

This is the see-saw mechanism. 

A technical detail: one mass comes out negative, but if \(\psi \) has a negative mass \(\gamma^{5} \psi \) has a positive mass. 

What happens if we have more than one neutrino generation? 

In the SM, we have determined that the flavor eigenstates are linear combinations of the mass eigenstates with a unitary mixing matrix \(U\). 

We can also have a number \(N_s\) of sterile neutrino states which do not couple to the \(W^{\pm}\) and \(Z\) bosons. 
The number \(N_s\) is not limited to the right-handed components of the three neutrinos, there could be more. The full mixing matrix will then be 
%
\begin{align}
\left[\begin{array}{cc}
3 \times 3 & 3 \times N_s \\ 
N_s \times 3 & N_s \times N_s
\end{array}\right]
\sim
\left[\begin{array}{cc}
\sim U _{\text{PMNS}} & \sim \order{v / \Lambda } \\ 
\sim \order{v / \Lambda } & U_{N \times N}
\end{array}\right]
\,.
\end{align}

A phenomenologically motivated situation is already the \(N_s = 1\) one, 
with a fourth sterile neutrino with mass \(\sim \SI{1}{eV}\). 
At present, though, the evidence for a sterile neutrino is very controversial. 

The process of leptogenesis, by some process which satisfies Sakharov's conditions, is relevant for this discussion. 

We need both C and CP violation. 
If we can find CP violation at low energy scales, it would be a good indication! 

CP violation in the quark sector, \(V _{\text{CKM}} = V^{*} _{\text{CKM}}\), cannot explain matter-antimatter asymmetry.

It is possible that a CP violating mechanism in the weak sector may give an imprint in the primordial GW background. 

If the CP violating decay of heavy neutrinos is to blame, there might be a connection between the low- and high-energy manifestations of CP violation. 

\subsubsection{The effect of neutrino masses and mixing in propagation}

We have 
%
\begin{align}
\left[\begin{array}{c}
\nu _e \\ 
\nu _\mu  \\ 
\nu _\tau 
\end{array}\right]
= 
U _{\text{PMNS}} 
\left[\begin{array}{c}
\nu_1   \\ 
\nu_2  \\ 
\nu_3 
\end{array}\right]
\,,
\end{align}
%
where \(U _{\text{PMNS}} U _{\text{PMNS}} ^\dag = 1\), while in general \(U \neq U ^\dag\). 
We will assume \(E \gg m \). 

The convention used for this matrix is as follows, since it can be shown that it contains three angles and a CP-violating phase:
%
\begin{align}
\left[\begin{array}{ccc}
U_{e1} & U_{e2} & U_{e3} \\ 
U_{\mu 1} & U_{\mu 2} & U_{\mu 3} \\ 
U_{\tau 1} & U_{\tau 2} & U_{\tau 3} 
\end{array}\right]
= 
\underbrace{\left[\begin{array}{ccc}
1 & 0 & 0 \\ 
0 & c_{23} & s_{23} \\ 
0 & -s_{23} & c_{23}
\end{array}\right]}_{R(\theta_{23})}
\underbrace{\left[\begin{array}{ccc}
c_{13}  & 0 & s_{13} e^{-i \delta } \\ 
0 & 1 & 0 \\ 
s_{13} e^{i \delta } & 0 & c_{13} 
\end{array}\right]}_{R(\theta_{13})} 
\underbrace{\left[\begin{array}{ccc}
c_{12}  & s_{12}  & 0 \\ 
-s_{12}  & c_{12}  & 0 \\ 
0 & 0 & 1
\end{array}\right]}_{R(\theta_{12})}
\underbrace{\left[\begin{array}{ccc}
1 & 0 & 0 \\ 
0 & e^{i \psi ' } & 0 \\ 
0 & 0 & e^{i \psi ''}
\end{array}\right]}_{\text{Majorana}}
\,.
\end{align}

If we have Majorana neutrinos the spinor is equal to its conjugate up to a phase, but only two of the phases are physically meaningful. 
However, these play no role in oscillations.

We have measured these angles: for \(\theta_{23}\) there is (almost) maximal mixing, \(\theta_{23} \approx \pi /4\) or \(s^2_{23} \approx \num{.5}\). 
This is called \emph{octant ambiguity}, since we do not know the precise value. 

The value of \(s^2_{12}\) is approximately \(\num{.3}\). 

The value of \(s^2_{13}\) is approximately \(\num{.02}\). 

What about the mass spectrum? 
Well, the energy is approximately 
%
\begin{align}
E = \sqrt{p^2 + m^2} 
\approx p + \frac{m^2}{2 p}
\approx p + \frac{m^2}{2E}
\,.
\end{align}

The masses are numbered \(1\), \(2\) and \(3\); 
by convention we always take \(m^2_2 > m^2_1\). 

We have \(\delta m^2 = m_2^2 - m_1^2 \approx \SI{7.5e-5}{eV^2}\), while \(\abs{\Delta m^2} = \abs{m_3^2 - m_1^2} \approx \SI{2.5e-3}{eV^2}\). 
We do not know the sign of \(\Delta m^2\)! 

As long as we deal with three neutrino masses and mixing, we have 5 known quantities and 5 unknowns: we know 
\begin{enumerate}
    \item \(\delta m^2\);
    \item \(\abs{\Delta m^2}\);
    \item \(s^2_{23}\);
    \item \(s^2_{12}\);
    \item \(s^2_{13}\),
\end{enumerate}
%
while we do not know some parameters which are hard or impossible to measure with oscillations:
\begin{enumerate}
    \item the value of the CP-violating \(\delta \) --- but we do have some weak \(\sim 2 \sigma \) indications that it might be nonzero;
    \item the sign of \(\Delta m^2\);
    \item whether \(\theta_{13} \lessgtr \pi /4 \); 
    \item the absolute mass scale (we only have some bounds);
    \item whether neutrinos are Dirac or Majorana.
\end{enumerate}

There is a search for right-handed neutrinos at many scales, even at LHC.

Tomorrow we will start studying neutrino oscillations, dealing with wavefunctions as opposed to spinors, approximating time with position \(t \sim x\); and using the fact that \(E \approx p + m^2 / 2E\). 

The Schrödinger equation will read 
%
\begin{align}
i \dv{}{t} \left[\begin{array}{c}
\nu_1  \\ 
\nu_2  \\ 
\nu_3 
\end{array}\right]
= H 
\left[\begin{array}{c}
\nu_1  \\ 
\nu_2  \\ 
\nu_3 
\end{array}\right]
\,,
\end{align}
%
where, in a vacuum, \(H\) is diagonal, with the three energies of the neutrinos.
There will be corrections in matter. 

We will have 
%
\begin{align}
H = p + \frac{1}{2E} \left[\begin{array}{ccc}
m_1^2  & 0 & 0 \\ 
0 & m_2^2 & 0 \\ 
0 & 0 & m_3^2
\end{array}\right]
\,.
\end{align}



\end{document}