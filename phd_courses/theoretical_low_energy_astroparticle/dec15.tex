\documentclass[main.tex]{subfiles}
\begin{document}

\subsection{Matter effects in neutrino oscillations}

\marginpar{Wednesday\\ 2021-12-15}

This is also called the MSW effect. 
Neutrinos interact weakly with matter, meaning that they are rarely absorbed;
however, neutrinos may feel the presence of matter even without changing direction. 

Through some interaction, a propagating neutrino may feel the presence of background fermions (matter).
This idea is called ``coherent forward scattering''. 

The full Hamiltonian will include a neutrino ``interaction energy potential''
%
\begin{align}
H = \frac{1}{2E} U M^2 U ^\dag + \left[\begin{array}{ccc}
V_{ee} & V_{e \mu } & V_{e \tau } \\ 
V_{\mu e} & V_{\mu \mu } & V_{\mu \tau } \\ 
V_{\tau e} & V_{\tau \mu } & V_{\tau \tau }
\end{array}\right]
+ \text{terms proportional to } \mathbb{1}_3
\,.
\end{align}

Pictorially, this matrix \(V\) has a diagonal term driven by neutral current interactions,
with any neutrino interacting through a \(Z\) boson. 
The coupling of the \(Z\) does not depend on the flavor, so that component is 
proportional to \(\mathbb{1}\), meaning it does not affect oscillations. 

A second term is the one corresponding to the \(W\) boson: for this, we only have
an effect for electron neutrinos, in the \(V_{ee}\) component. 

At tree level, this is proportional to the Fermi constant \(G_F\). 
Also, it will linearly depend on the electron number density. 

The calculation yields 
%
\begin{align}
V_{ee} &= \sqrt{2} G_F N_e 
\qquad \text{for neutrinos}  \\
V_{ee} &= -\sqrt{2} G_F N_e 
\qquad \text{for antineutrinos}  
\,.
\end{align}

This effect is quite analogous to the propagation of light within a medium with a 
non-1 index of refraction. 
It alters the oscillation pattern both in amplitude and in frequency.

The main oscillation channel for atmospheric neutrinos is \(\nu _\mu \to \nu _\tau \), 
meaning that these matter effect did not need to be included to study it. 

It is convenient to introduce a term \(A = 2 E V_{ee} = 2 \sqrt{2} G_F N_e E\); then the Hamiltonian reads 
%
\begin{align}
H = \frac{1}{2E} \left( U M^2 U ^\dag + \left[\begin{array}{ccc}
A & 0 & 0 \\ 
0 & 0 & 0 \\ 
0 & 0 & 0
\end{array}\right] \right)
\,.
\end{align}
%

Qualitatively, we can understand that this term will be important when \(A\) 
is \(\gtrsim \delta m^2\) or \(\Delta m^2\). 

It turns out that 
%
\begin{align}
\frac{A}{\Delta m^2_{ij}} \approx \num{1.526e-7} \left( \frac{N_e}{\SI{}{mol/cm^3}}\right) 
\left( \frac{E}{\SI{}{MeV}}\right)
\left( \frac{\SI{}{eV^2}}{\Delta m^2_{ij}} \right)
\,.
\end{align}

For the Sun, in the core, \(N_e \sim \SI{e2}{mol/cm^3}\), so we get that as an order of magnitude \(A \sim \delta m^2\). 

Since the number density of neutrinos depends on position, we need to integrate the Schrödinger equation numerically. 

Integrating these oscillatory functions, however, can be tricky! 
Errors accumulate. 

For the crust of the Earth, say, the LHC to LNGS beam, approximating the density as constant can work quite well. 

Let us consider a two-neutrino case: 
%
\begin{align}
H = \frac{1}{2E} U \left[\begin{array}{cc}
m_1^2 & 0 \\ 
0 & m_2 ^2
\end{array}\right]
U ^\dag
+ \frac{1}{2E}
\left[\begin{array}{cc}
A & 0 \\ 
0 & 0
\end{array}\right]
= \frac{1}{2E} \widetilde{U} \widetilde{M}^2 \widetilde{U} ^\dag
\,,
\end{align}
%
so we find a corrected diagonal matrix \(\widetilde{M}^2\) as well as corrected mixing angles! 
They read:
%
\begin{align}
\sin 2 \widetilde{\theta}_{12} &= \frac{\sin 2 \theta_{12} }{\sqrt{(\cos 2 \theta_{12} - A / \delta m^2)^2 + \sin^2 2 \theta_{12} }}  \\
\widetilde{\delta m}^2 &= \delta m^2 \frac{\sin 2 \theta_{12} }{\sin 2 \widetilde{\theta}_{12}}
\,,
\end{align}
%
which looks like a Breit-Wigner resonance.

The corrected \(\sin 2 \widetilde{\theta}_{12}\) has a maximum when \(\cos 2 \theta_{12} = A / \delta m^2 \), at which point we also have a minimum for \(\widetilde{\delta m}^2\). 

This resonance does not happen for antineutrinos, which have a different sign for \(A\). 

For the Sun, we make two approximations: the density is slowly changing, \(\dv*{N_e}{x} \) is nonzero but small, and there are many oscillations. 

The transition is ``adiabatic'', so everything will only depend on the initial and final mixing angles, not on \(\delta m^2\). 

The result is 
%
\begin{align}
P_{ee}^{2 \nu } ( \text{solar}) \approx \cos^2 \widetilde{\theta}_i \cos^2 \theta + \sin^2 \widetilde{\theta}_i \sin^2 \theta 
\,.
\end{align}

This probability has been tested! 
It looks a bit like a decreasing ``sigmoid'', with a transition at few \SI{}{MeV}. 

This oscillation probability is no longer octant symmetric! 
It solves the degeneracy of the mixing angles in the KamLand experiment. 

Uncertainties in the solar model do not really create problems in this regard; our models work well enough. 

\end{document}