\documentclass[main.tex]{subfiles}
\begin{document}

\marginpar{Thursday\\ 2021-12-16}

The last lecture for this course! 

The \textbf{normal ordering} for neutrino masses is \(m_1 < m_2 < m_3\), 
while \textbf{inverted ordering} is \(m_3 < m_1 < m_2\). 

We always denote the lightest neutrino mass as \(m_0 \) --- it could be either \(m_1 \)
or \(m_3 \).
It could be zero! that is not excluded by current data. 

We can plot \(m_i\) as a function of \(m_0 \), with the constraint of 
fixing \(\delta m^2\) and \(\Delta m^2\). 
We need to precisely define what the large \(\Delta m^2\) is: 
the convention used by  the professor is \(\Delta m^2 = (\Delta m^2_{31} + \Delta m^2_{32}) / 2\). 

Absolute neutrino mass observables are: 
\begin{enumerate}
    \item endpoint of \(\beta \) decay spectrum;
    \item precision cosmology;
    \item neutrinoless double beta decay (iff neutrinos are Majorana);
    \item time-of-flight: neutrino masses cause a delay of the order
    %
    \begin{align}
    \Delta t \approx D \frac{1}{2} \left( \frac{m_i}{E} \right)^2
    \,.
    \end{align}
\end{enumerate}

Unfortunately, the time-of-flight can be done with supernovae only constraining 
down to \(\sim \SI{100}{eV}\) since the signal is extended in time. 

What about \(\beta \) decay? 
Looking at the endpoint of the spectrum could indicate what the mass looks like. 
However, there are problems from both energy resolution and low statistics. 
The experiment KATRIN is trying to measure this!

Which mass would such an experiment be measuring? 
There is no ``mass of the electron neutrino''!

One would theoretically expect three different kinks corresponding to the
three different neutrino masses. 
The leading-order net effect of the superposition of these is 
%
\begin{align}
m^2_\beta = \sum _{i} \abs{U_{e1}}^2 m^2_i
\,.
\end{align}

We can rewrite this in terms of the mixing angles; this also changes 
according to normal and inverted ordering. 

What about cosmology? 
The net effect measured there is \(\Sigma = m_1 + m_2 + m_3 \). 

The way to measure this is to look at CMB anisotropies and the Large Scale Structure. 

The contribution of neutrinos is peculiar in that they are ``free-streaming''. 

The \(\Lambda \)CDM model is rather unsatifactory!
Dark matter and dark energy are just parametrized ignorance. 

Depending on how much one trusts the data (and ignores the inconsistencies 
such as the measurement of \(H_0 \)) the bound comes out to be \(\Sigma \lesssim \SI{100}{meV}\)
or a few hundreds. 

Cosmology is reaching into the region where the masses are quite degenerate. 

The decay amplitude for \(0 \nu \beta \beta \) is written like 
%
\begin{align}
\Gamma \propto \abs{\sum _{i} U^2_{ei} e^{i \phi _i} m_i}^2
\,,
\end{align}
%
so the three phases can make the terms interfere! 
One can then define an effective neutrino mass
%
\begin{align}
m_{\beta \beta  } = \abs{ \abs{U_{e1}}^2 m_i + \abs{U_{e2}}^2 e^{i \alpha } m_2 + \abs{U_{e3}}^2 e^{i \beta } m_3}
\,.
\end{align}

One can compute an allowed band for \(m_{\beta \beta }\), both in normal and inverted ordering. 
The idea is to make these kinds of plots for \(m_\beta (m_0 )\), \(m_{\beta \beta } (m_0 )\) and 
\(\Sigma (m_0 )\) for NO and IO; however the alternative is to plot these against each other! 

We can add exclusion bands to these. 

The possibility of having a fourth, almost-sterile neutrino with \(\Delta M^2 \sim \SI{1}{eV}\) can be probed with short baseline experiments, where \(\Delta M^2 L / 4E\) is close to 1. 

There is some controversial evidence for this happening. 
Theory kind-of tells us that these effects will be too small to be observable. 

If the oscillation probability for \(\Delta M^2\) is of significant magnitude, we can approximate the others as zero, so we get 
%
\begin{align}
P_{\alpha \alpha } &= 1 - 4 \abs{U_{\alpha 4}}^2 \left( 1-  \abs{U_{\alpha 4}}^2\right) \sin^2 \left(\frac{\Delta M^2 L}{4E}\right)  \\
P_{\alpha \beta } &= 4 \abs{U_{\alpha 4}}^2 \abs{U_{\beta 4}}^2 \sin^2 \left(\frac{\Delta M^2 L}{4E}\right)
\,.
\end{align}

However, this must be small, since the \(U_{\alpha 4} \sim \epsilon \) terms of the mixing matrix are small! If they were large we would observe large deviations from unitarity in the PMNS matrix. 

The expectation is therefore that appearance effects are \(\order{\epsilon^2}\) while disappearance effects are \(\order{\epsilon }\), while current data seem to be measuring the same effect size for both. 

What about a possible new flavor changing vertex \(\nu _\alpha + \text{fermions} \to \nu _\beta + \text{fermions}\), proportional to \(\epsilon_{\alpha \beta } G_F\)? 

It would produce some kind of new matter effect, \(V = \sqrt{2} \epsilon_{\alpha \beta } G_F N_f\) where \(N_f\) is the charged fermion density. 

The bounds on the parameters \(\epsilon_{\alpha \beta }\) are rather weak, of the order of \num{e-1} to 1. 
These are ``non-standard interactions''. 

The last lecture, in the notes, is about two more advanced topics.

The first is how one can probe the mass ordering: in order to find out the sign of \(\Delta m^2\), one must make it interfere with something which has a known sign: \(\delta m^2\), or matter effects \(Q = \sqrt{2} G_F N_e\), or the more exotic \(Q = \sqrt{2}G_F N_\nu \) --- this can be measurable only in supernovae. 

Juno probes the first: it works with a medium baseline of \(L \sim \SI{50}{km}\) and a very good energy resolution. 

One can probe the full \(P_{ee} (\delta m^2, \pm \Delta m^2, \theta_{12}, \theta_{13})\), and by the precise phase of the \(\Delta m^2\) short oscillations.

Long baseline experiments, on the other hand, probe \(P_{\mu e}\) considering matter effects. There is an additional dependence on \(\delta \) and \(V\). 
DUNE tries to do this. 

The Romans used to say ``nomen omen''. 
The name ``neutrino'' was invented by Enrico Fermi, and it means
``little neutral one''. 
It is inspired by the Italian ``neutro'', which in turn comes from 
the Latin ne-uter, which includes ``uter'' which means ``either''. 

Some other words which have the root ``uoter'' are in Greek, German (whether, weather). 

The old Ionic Greek word Koteros means ``which of the two''. 

The old Sanskrit word Kataras also means ``which of the two''. 

Many Indo-European words began with Ka, and Kwa. 
Also ``Quantus'', ``how much'', originates from this. 

This prefix is a sort of interrogative case, a ``question mark''.

The destiny of neutrinos is to raise questions! 

\url{eligio.lisi@ba.infn.it}. 

 

\end{document}