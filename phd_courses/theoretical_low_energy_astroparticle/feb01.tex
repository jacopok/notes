\documentclass[main.tex]{subfiles}
\begin{document}

\marginpar{Tuesday\\ 2022-2-1}

We were discussing the Boltzmann equation for a set of \(N\) particles \(\chi _i\),
which satisfy a collective Boltzmann equation for \(n = \sum _{i} n_i\). 

The ratio of the equilibrium density of each species to the overall equilibrium density 
appears: what will be its value? 
%
\begin{align}
\frac{n_i^{\text{eq}}}{n^{\text{eq}}} 
&= \frac{
    g_i (M_i / M_1) ^{3/2} \exp(- M_i / T) \exp(M_1 / T)
}{
    \sum _{j} g_j (M_i / M_j)^{3/2} \exp(- M_j / T) \exp(M_1 / T)
}  \\
&= 
\frac{
    g_i (1 + \Delta _i)^{3/2} \exp(- M_1 \Delta _i / T )
}{
    \sum _{j} g_j (1 + \Delta _i)^{3/2} \exp(- M_1 \Delta _j / T)
}
\,,
\end{align}
%
where we added the exponential of \(M_1 / T\) since we want to 
get factors of \(\Delta _i = (M_i - M_1 ) / M_1\). 

We can call the denominator \(g _{\text{eff}}\): then, 
%
\begin{align}
\expval{\sigma _{\text{eff}} v} &= \sum _{i, j} \expval{\sigma _{ij} v} \frac{n_i^{\text{eq}} n_j^{\text{eq}}}{(n^{\text{eq}})^2}   \\
&= \sum _{ij} \expval{\sigma_{ij} v} \frac{g_i g_j}{g _{\text{eff}}^2} 
(1 + \Delta _i)^{3/2} (1 + \Delta _j)^{3/2} 
\exp(- \frac{M_1}{T} (\Delta _i + \Delta _j))
\,.
\end{align}

Suppose we only have two particles. 
Then, we can denote 
%
\begin{align}
g _{\text{eff}} = g_1 + g_2  (1 + \Delta )^{3/2} \exp(- \frac{M_1}{T} \Delta )= g_1 (1 + w )  
\,.
\end{align}

The annihilation cross-sections will read 
%
\begin{align}
\sigma_{22} = \alpha \sigma_{21} = \alpha^2 \sigma_{11} 
\,.
\end{align}

Then, we get 
%
\begin{align}
\sigma _{\text{eff}} = \sigma_{11} \frac{g_1^2}{g_1^2 (1 + w)^2}
+ 2 \sigma_{12} \frac{w}{(1+ w)^2} + \sigma_{22} \frac{w^2}{(1+w)^2} 
\approx \sigma_{11} \frac{(1 + \alpha w)^2}{(1 + w)^2}
\,.
\end{align}

In the limit where \(\Delta \to 0\), we get \(w \to g_2 / g_1 \) and so 
%
\begin{align}
\sigma _{\text{eff}} \to \sigma_{11} \frac{(1 + \alpha (g_2 /g_1)^2}{(1 + g_2 / g_1 )}
\,,
\end{align}
%
therefore, assuming \(\Delta \sim 1/20\), if \(\alpha \gtrsim 1\) we reduce \(\Omega \), while if \(\alpha \lesssim 1\) we enhance it! 

Consider the case of a pure Higgsino: \(\widetilde{\chi}^0_{1, 2}\) and \(\widetilde{\chi}^+\). 

Some considerations are then made regarding the possibility of the lightest 
DM particle being a Bino or a Higgsino. 

Indirect detection is disfavored in this case, since today we only have \(\sigma_0 v\) 
for the lightest state.

Could we have the opposite scenario, \(\sigma_0 v \gg \expval{\sigma _{\text{eff}} v} \)? 

This is the case for Sommerfeld enhancement. 

Now, the cross-section is computed in the context of empty space! 
If we compare the ``size'' of the particle, \(\sim \sqrt{\sigma }\), to the separation
of the particles, \(1/\sqrt[3]{n}\), we still get that they are quite 
far away from each other.

In the regime where long-range forces become active, the range of interaction 
will not be \(\sqrt{\sigma }\) anymore! 

The Coulomb potential, for example, can be written as \(V = - \alpha _{\text{em}} / r\). 
It has an infinite range! 

Really, we should solve the Schrödinger equation in the non-relativistic limit, 
%
\begin{align}
\left( \frac{1}{2m_\chi  } \dv[2]{}{r} 
+ \frac{\abs{k}^2}{2 m_\chi }
- V_\phi (r) \right)
\psi _k (r) = 0
\,.
\end{align}

There is then an enhancement for the annihilation rate in the form 
%
\begin{align}
S = \frac{\sigma _{\text{ann}}}{\sigma_0 } = \frac{\abs{\psi _k(0)}^2}{\abs{\psi _k^{(0)}(0)}}
\,,
\end{align}
%
where \(V_\phi = - \alpha / r\) and the denominator in \(S\) is in the case 
where we have no field.

In the Coulomb case, this can be computed exactly: 
%
\begin{align}
S = \frac{\pi \alpha _\phi }{v} \frac{1}{1 - \exp(- \pi \alpha _\phi / v)}
\,,
\end{align}
%
where \(v\) is the relative velocity. 

In the case where \(m_\phi \neq 0\) we get a Yukawa interaction, 
suppressed by \(\exp(- m_\phi r)\); there, we can also compute \(S\) analytically. 
%
\begin{align}
S = \frac{\pi \alpha _\phi }{v} \frac{\sinh (12 v / \pi \alpha _\phi \xi )}{\cosh (12v / \pi \alpha _\phi \xi )
- \cos(2 \pi \sqrt{6 / \pi^2\xi - (6 v / \pi^2 \alpha _\phi \xi )^2})}
\,,
\end{align}
%


If \(v \) is large, or if \(m_\phi \gtrsim m_\chi \), 
we can expand in \(1 / \xi \). 

If, on the other hand, \(\xi \ll 1\), then \(S = \pi \alpha _\phi / v\). 

There are resonances when \(\xi = 6 / \pi^2 n^2\): 
then, \(S = \alpha _\phi^2 / v^2 n^2\), for \(n \in \mathbb{Z}\). 

We can plot different curves for \(\log S\) as a function of \(\xi \), with different 
values of \(v\). 

\subsection{Feebly Interacting Massive Particles}

This is DM which is not in thermal equilibrium in the Early Universe. 

For example, these could be sterile neutrinos. 
These would be SM singlets, but which mix with SM neutrinos. 

Let us look at the 1+1 configuration: we have \(\ket{\nu _\alpha }\) 
with \(\alpha \in \left\lbrace e, \mu , \tau \right\rbrace\), plus a 
\(\ket{\nu _s}\). 
Further suppose that 
%
\begin{align}
\ket{\nu _\alpha } &= \cos \theta _\alpha \ket{\nu _1} + \sin \theta _\alpha \ket{\nu _2} \\
\ket{\nu _s } &= -\sin \theta _\alpha \ket{\nu _1} + \cos \theta _\alpha \ket{\nu _2}
\,,
\end{align}
%
and we assume that \(m_2 \gg m_1\). 
Since we want these neutrinos on the order of the \SI{}{keV} at least, 
but we know that the masses of SM neutrinos are very small, 
therefore we need \(\theta _\alpha \ll 1\). 

How do neutrino oscillations work in the early universe? we get 
a transition amplitude in the form 
%
\begin{align}
A(\nu \alpha \to \nu _s) = c_s \bra{\nu _s} e^{-i \frac{m_i^2t}{2 E_i}} \ket{\nu _\alpha }
\,.
\end{align}

Therefore, the final probability reads 
%
\begin{align}
\mathbb{P}(\nu _\alpha \to \nu _s ) = \sin^2(2 \theta _\alpha )
\sin^2 (\frac{\Delta m^2}{2E_i} \frac{L}{2}) 
\,.
\end{align}

We can write a similar expression in the Early Universe 
with \(t_\alpha = 2 E_\nu / \Delta m^2\). 

The evolution can be described by a Hamiltonian  
%
\begin{align}
H = U \operatorname{diag} (\frac{m_1^2}{2 E_\nu }, \frac{m_j^2}{2 E_\nu }) U ^\dag + V _{\text{int}}
\,.
\end{align}

The interaction potential only has a single nonzero component \(V_{\alpha \alpha}\), 
which accounts for the interactions with the heat bath. 

The relevant interaction rate is \(\Gamma \sim \SI{100}{MeV}\), 
so we get 
%
\begin{align}
V_{\alpha \alpha } \approx C_\alpha \times 25 G_F^2 T^4 E_\nu 
\,,
\end{align}
%
where \(C_e \approx 3.5\). 

The probability of a \(\nu _\alpha \) becoming a \(\nu _s\) is therefore 
%
\begin{align}
P(\nu _\alpha \to \nu _s) = \sin^2(2 \theta _\alpha^m) \sin^2(t / 2 t_\alpha^m)
\,,
\end{align}
%
where 
%
\begin{align}
t_\alpha^m = \frac{t_\alpha^{\text{vac}}}{\sqrt{\sin^2 2 \theta _\alpha + (\cos(2 \theta _\alpha ) - V_{\alpha \alpha } t_\alpha^{\text{vac}})^2}}
\,,
\end{align}
%
and 
%
\begin{align}
\sin^2(2 \theta _\alpha^m) &= \left(\frac{t_\alpha^m}{t_\alpha^{\text{vac}}} \right)^2
\sin^2 (2 \theta _\alpha )
\,.
\end{align}



\end{document}