\documentclass[main.tex]{subfiles}
\begin{document}

\section{Misalignment production for ultra-light scalars}

\marginpar{Thursday\\ 2022-2-3}

We have some scalar \(\phi \) which is spatially constant over some length scale. 

We will evolve this classical configuration, ignoring perturbations, in the \(o\)-mode. 
However, we will account for the FLRW background. 

The Klein-Gordon equation reads 
%
\begin{align}
\ddot{\phi}_0 (t) + 3H \dot{\phi}_0 + \pdv{V}{\phi_0 } = 0
\,.
\end{align}

We will assume that 
%
\begin{align}
V(\phi_0 ) \approx \frac{1}{2} m_\phi^2 \phi_0^2
\,,
\end{align}
%
neglecting all corrections, where \(m_\phi \) is assumed to be tiny.
If this field is associated with some phase transition, the parameter \(m_\phi \) 
might be time-dependent: \(m_\phi (t)\). 

The equation then reads 
%
\begin{align}
\ddot{\phi}_0 (t) + 3H \dot{\phi}_0 + m_\phi^2 \phi_0  = 0
\,.
\end{align}

We need to distinguish different regimes: in the friction-dominated
regime, \(H \gg m_\phi \), so \(\phi_0 \sim \text{const}\).
 
Since \(H \sim \sqrt{g_*} T^2 / m_P\), this will happen at early times. 

On the other hand, when \(H \ll m_\phi \) we get oscillations 
with frequency \(m_\phi \). 

In between, at time \(t_*\) and temperature \(T_*\), we will have \(H(t_*) \approx m_P(t_*)\). 

At that time, the energy density of the field reads 
\(\rho _\phi = m_\phi^2 \phi_0^2 / 2 + \dot{\phi}^2/2\). 
How does that change with time?

Its derivative reads 
%
\begin{align}
\dot{\rho}_\phi = m_\phi^2 \phi_0 \dot{\phi}_0 
+ m_\phi \dot{m}_\phi \phi_0^2 + \dot{\phi}_0 \ddot{\phi}_0
\,,
\end{align}
%
where we recognize some terms which were appearing in the EoM, so we can 
replace them: 
%
\begin{align}
\dot{\rho}_\phi = 
-3H \dot{\phi}^2_0 
+ m_\phi \dot{m}_\phi \phi_0^2 
\,.
\end{align}

If we have fast oscillations, we will have \(\expval{V} = \expval{K} = \expval{\rho _\phi } / 2\):
the expectation value of the derivative of the energy density, therefore, will be 
%
\begin{align}
\expval{ \dot{\rho}_\phi } &= - 3 H \expval{ \dot{\phi}_0^2} + \frac{\dot{m}_\phi }{m_\phi }
\expval{m_\phi^2 \phi_0^2}  \\
&= \expval{\rho _\phi } \left(\frac{\dot{m}_\phi }{m_\phi } - 3H\right)
\,,
\end{align}
%
which we can solve: \(\expval{ \rho_\phi } \propto m_\phi / a^3\) works. 
When \(m_\phi \) is constant, this is just the equation of state for matter. 

When the transition happens from \(t_*\) to the regime in which \(H \ll m_\phi \), 
we can use the WKB approximation and find a solution in the form 
%
\begin{align}
\phi_0 (t) = A \cos \alpha 
\,,
\end{align}
%
where 
%
\begin{align}
A(t) \approx \phi_0^* \sqrt{\frac{m_\phi^* a^3_*}{m_\phi (t) a^3(t)}}
\,,
\end{align}
%
while 
%
\begin{align}
\alpha (t) = \int_{t_*}^t \dd{t'} m_\phi (t')
\,.
\end{align}

Then, the energy density reads 
%
\begin{align}
\expval{\rho _\phi } = \frac{m_\phi^2}{2} A^2
\,,
\end{align}
%
while the pressure reads 
%
\begin{align}
\expval{P_\phi } = \expval{ \frac{\dot{\phi}^2}{2} - \frac{m_\phi^2}{2} \phi_0^2} = \frac{\dot{A}^2}{2}
\,,
\end{align}
%
which is much smaller than \(\expval{\rho _\phi }\). 

Why is this dubbed ``misalignment'' production? 
The potential energy the field had in the beginning is now translated
to kinetic and potential energy in the oscillations. 

What is the relic density for this kind of DM? 
%
\begin{align}
\Omega _\phi = \frac{1}{\rho _c (t_0 )} \frac{m_\phi^2}{2} A^2
= \sqrt{\frac{m_\phi^2}{m_\phi^* \SI{}{eV}}} \left(\frac{\phi_0^*}{\SI{e11}{GeV}}\right)^2
\,.
\end{align}

The reason for that denominator is that \(H(T_*) = 1/2t_* = m_\phi^*\) in the
radiation domination phase.
Also, \(a_* \propto \sqrt{t_*} \propto (m_\phi^*)^{-1/2}\).

We are assuming that \(t_* < t _{\text{eq}}\): this means that \(m_\phi^* > \SI{e-28}{eV}\). 

So, this is an ultralight \emph{classical} scalar; 
we need to worry about quantum corrections, specifically to the mass. 

We need the mass to be protected by a symmetry.
\(\phi \) might be the Goldstone boson of some SSB. 

In that case, we could have 
%
\begin{align}
V(\phi ) \approx m_\phi^2 F_\phi^2 (1 - \cos(\phi_0^* / F_\phi ))
\,,
\end{align}
%
where \(F_\phi \) is some breaking scale. 

This field will also have some interactions: 
%
\begin{align}
\mathscr{L}_\phi \sim \frac{1}{F_\phi } J_\mu \partial^{\mu } \phi_0 
\,,
\end{align}
%
where \(J_\mu \) is some Nöther current associated with the breaking. 

The most popular example in this class of models is the QCD axion. 

This comes in connection to the \emph{strong CP} problem: 
the QCD Lagrangian, in general, looks like 
%
\begin{align}
\mathscr{L} _{\text{QCD}} = \sum \overline{q} \left(
    i \slashed{D} - m_q e^{i \theta q}
\right)
q
+ \frac{1}{4}
G_{\mu \nu }^{a} G^{\mu \nu a} 
+ \theta \frac{g^2_s}{32 \pi^2} G_{\mu \nu }^a \widetilde{G}^{\mu \nu a}
\,,
\end{align}
%
where \(\widetilde{G} = * G\) is the Hodge dual of the gluon field strength tensor. 

CP violation comes in from the phase added to the mass, 
as well as in the \(G \widetilde{G}\) term. 
This term is in the form \(\vec{E} \cdot \vec{B}\): it chooses a direction. 

We can write this term as a total derivative: \(G_{\mu \nu }^{a} \widetilde{G}^{\mu \nu a} = \partial_\mu K^\mu \), 
where 
%
\begin{align}
K^\mu = \epsilon^{\mu \nu \alpha \beta } \left(
    A_\alpha^a G^a_{\beta \gamma } 
    - \frac{g_s}{3} f^{abc} A_\alpha^a A_\beta^b A_\gamma^c
\right)
\,.
\end{align}

Therefore, at the perturbative level this term does not come in: 
when integrating out we get something like \(\int \dd{\sigma _\mu } K^\mu \)
on a boundary surface; but can we take \(A^a_\mu  = 0\) at infinity? 

A gauge transformation for \(A\) looks like 
%
\begin{align}
A_\mu \to U^{-1} A_\mu U + i g_s^{-1} U^{-1} \partial_\mu U
\,,
\end{align}
%
where the second term might not vanish at the boundary. 

The integral 
%
\begin{align}
\frac{g_s^2}{32 \pi^2} \int \dd[4]{x} G_{\mu \nu}^a \widetilde{G}^{\mu \nu a}
\,
\end{align}
%
takes on different integer values called \emph{winding numbers}, 
which identify various vacua.

To map between these vacua, we can use 
a chiral transformation in the form \(q \to e^{i \gamma _5 \alpha } q \), 
which means \(\theta _q \to \theta _q + 2 \alpha \). 
We might then simply eliminate the CP-violating term! 

This transformation, however, is associated to a current in the form 
%
\begin{align}
J_\mu^5 = \overline{q} \gamma _\mu \gamma _5 q
\,,
\end{align}
%
which however is not conserved in the full Lagrangian: 
%
\begin{align}
\partial^\mu J_\mu^5 = \frac{g_s^2}{32 \pi^2}  G_{\mu \nu}^a \widetilde{G}^{\mu \nu a}
\,,
\end{align}
%
which is still negligible at the perturbative level. 

Therefore, we define 
%
\begin{align}
\overline{\theta} = \theta + \arg \det (Y_u Y_d)
\,.
\end{align}

The neutron electric dipole moment can be exploited to 
measure this CP violation: it is a term in the Hamiltonian in the form
%
\begin{align}
H = d_n \vec{E} \cdot \vec{S}
\,,
\end{align}
%
where \(\vec{E}\) is the external electric field and \(\vec{S}\) is the spin of the neutron. 

Experimentally, neutrons don't seem to care about the electric field at all, 
so we can put a bound like 
%
\begin{align}
\abs{d_n } \lesssim \SI{3e-26}{cm} e
\,.
\end{align}

From a microphysical point of view, this comes from a
CP-violating operator in the form 
%
\begin{align}
\mathscr{L} = d_n \frac{i}{2} \overline{n} \sigma_{\mu \nu} \gamma _5 n
\,,
\end{align}
%
where 
%
\begin{align}
\sigma_{\mu \nu } = \frac{i}{2} \qty{\gamma _\mu , \gamma _\nu } 
= \frac{i}{2} \left(
    \gamma _\mu \gamma _\nu + \gamma _\nu \gamma _\mu 
\right)
\,.
\end{align}

With this, we get 
%
\begin{align}
d_n \approx \frac{\overline{\theta}}{(4 \pi )^2} e \frac{m_\pi }{m_n^2} 
\approx \SI{2.4e-16}{cm } e \overline{\theta}
\,,
\end{align}
%
which means that \(\overline{\theta} \lesssim \num{e-10}\). 
This is the \emph{strong CP problem}, because physicists 
like some numbers better than others. 

The idea of the Peccei-Quinn mechanism is to have a symmetry in the form 
%
\begin{align}
q_L &\to e^{i \beta } q_L  \\
q_R &\to e^{-i \beta } q_R
\,,
\end{align}
%
which gets spontaneously broken with a Goldstone boson \(a(x)\):
the \emph{axion}! 
The axion transforms like \(a(x) \to a(x) + \beta F _{\text{PQ}}\). 

This sources a term 
%
\begin{align}
\mathscr{L}_a = \frac{g_s^2}{32 \pi^2} \frac{a(x)}{F _{\text{PQ}}} G_{\mu \nu}^a \widetilde{G}^{\mu \nu a}
\,
\end{align}
%
in the Lagrangian. 

At this level, the axion is massless, but depending on the 
implementation of the symmetry is might gain mass: 
after the QCD phase transition we gain a potential for the axion in the form 
%
\begin{align}
V(a) = g^2 _{\text{PQ}} m_a^2 \left(1 - \cos(a / F _{\text{PQ}})\right)
\,.
\end{align}

The calculation of the axion mass is similar to that of the pion mass, 
and it comes out to be 
%
\begin{align}
m_a^2 = \frac{m_u m_d}{(m_u + m_d)^2} \frac{m_\pi^2 f_\pi^2}{F _{\text{PQ}}^2}
\approx \frac{\SI{e12}{GeV}}{F _{\text{PQ}}} \SI{5}{ \micro eV}
\,.
\end{align}

The axion field is then expected to fall into its minimum, 
and the mean value \(\expval{a(x)}\) is then a parameter which can tune 
\(\overline{\theta}\) and explain the strong CP problem. 

The variation of the axion mass at \(T _{\text{QCD}}\) is difficult to model, 
since it involves nonperturbative effects which need to be accounted for. 
An approach which works is that of lattice QCD. 
%
\begin{align}
f _{\text{PQ}}^2 m_a^2(T) \approx f _{\text{PQ}}^2 
\left( 
    \beta m_a(T_0 ) \left(\frac{T _{\text{QCD}}}{T}\right)^\gamma 
\right)^2
\,,
\end{align}
%
where \(\gamma \approx 4 \) while \(\beta \approx 0.02\). 

There are two cases for the misalignment \(\Omega _a\): 
a possibility is that \(f _{\text{PQ}} > \max (H_I, T _{\text{rh}})\)
in which case 
\(U(1)_{PQ}\) is broken at the end of inflation, 
and it is not restored by reheating. 

A single mode for the axion is therefore stretched to a huge size. 

The axion mass increases to its final value around \(T _{\text{QCD}}\), which occurs right after \(T_*\). 
At the phase transition, it is 
%
\begin{align}
m_a (T_*) &= \beta m_a (t_0 ) \left(\frac{T _{\text{QCD}}}{T_*}\right)^{\gamma }  \\
&\approx H(T_*)
\,,
\end{align}
%
so we get 
%
\begin{align}
T_*^{2+\gamma } \left(\frac{\beta T _{\text{QCD}}^\gamma M_P}{1.66 \sqrt{g_*}}\right) m_a (t_0 )
\,.
\end{align}

Using this formula, together with 
%
\begin{align}
\Omega _a = \frac{m_a n_a (t_0 )}{\rho _c (t_0 )}
\,,
\end{align}
%
yields, for a patch in the universe:
%
\begin{align}
\Omega _a = \num{.1} \left(\frac{\theta_a^*}{\num{2.15}}\right)^2 \left( \frac{F_{PQ}}{\SI{e12}{GeV}} \right)^{1+1/(2+\gamma )} 
\left(\frac{60}{g_*}\right)^{1/2 - 1/(4 + 2 \gamma )}
\,.
\end{align}

The axion mass comes out to a few \SI{}{\micro eV}. 
There is fine-tuning, but we have isocurvature, adiabatic perturbations \(\delta \rho _a = 0\)! 

In the opposite regime, \(F_{PQ} < \max(H_I, T _{\text{rh}})\): 
then, we are either restoring the PQ symmetry during reheating 
or it was never broken in the first place. 

Then, \(\theta _\alpha \in [- \pi , \pi ]\), and we will get the same formula but 
%
\begin{align}
\Omega _a \approx \int_{-\pi }^{\pi } \dd{\theta _a^*} \Omega _a^{\text{1 patch}}
\,,
\end{align}
%
so the expression is as if we only had a single averaged angle
%
\begin{align}
\expval{\theta _a^{*2}} = \frac{1}{2 \pi } \int \dd{\theta _a^*} \theta _a^{* 2} 
\,,
\end{align}
%
so we get \(\pi^2 /3\), and we cannot play around with the angle anymore. 


\end{document}