\documentclass[main.tex]{subfiles}
\begin{document}

\section{Dark Matter}

\marginpar{Tuesday\\ 2022-1-25}

This part of the course is given by Piero Ullio \url{ullio@sissa.it}. 

We know that \(\Omega _{\text{DM}} h^2 = \num{.1200(12)}\). 
This is about \(26.7\%\) of the critical density and \(84.4\%\) of the total matter density. 

Bertone and Hooper wrote a good historical review of early DM work. 

Zwicky was the first to write about the problem of DM: the virial mass of the Coma cluster did not match its luminous mass. 

The mass of the Coma cluster came out to be around \(M \sim \num{5e14} M_{\odot}\). 

Current estimates give \(M/L \sim 160 M_{\odot} / L_{\odot}\). 

We can go further and measure the baryonic mass through X-rays! 
Assuming the gas is in hydrostatic equilibrium, it will emit through bremsstrahlung. 
We get the electron number density and the electron temperature, with this we then get the mass profile. 

In the Abell 2029 cluster we get a baryonic mass fraction of the order of \SI{14}{\percent}. 

We can get better estimates through strong gravitational lensing. 

We can do inference on the mass distribution which causes lensing. 

\todo[inline]{How does one do inference on such a high-dimensional space?}

Rotation curves in spiral galaxies are further evidence of DM. 
In order to explain them, we need DM with \(\rho \propto 1/ r^2 \). 

Now we are using more sophisticated galactic dynamics, with a set of tracers (observable galaxies, stars etc.) in an underlying potential
with a distribution function \(f(\vec{x}, \vec{v}, t)\). 

We write a collisionless Boltzmann equation for the conservation of the phase-space probability of this problem. 

This can be solved directly assuming stationarity and some symmetries: in terms of the total density \(\nu \), the average velocity \(\overline{v}_i\) and the average stress tensor \(\overline{\sigma}_{ij} = \overline{v_i v_j} - \overline{v}_i \overline{v}_j\): 
the equation is called the Jeans equation 
%
\begin{align}
\nu \pdv{\overline{v}_j}{t} + \nu \overline{v}_i
\pdv{\overline{v}_j}{x_i} = - \nu \pdv{\phi }{x_j} - \pdv{(\nu \sigma^2_{ij})}{x_j}
\,.
\end{align}

The collisionless Boltzmann equation reads 
%
\begin{align}
\pdv{f}{t} + \vec{v} \cdot \pdv{f}{\vec{x}} - \pdv{\phi }{\vec{x}} \cdot \pdv{f}{\vec{v}} = 0
\,,
\end{align}
%
where \(\phi \) is the potential the galaxies are moving in. 

The Jeans equation is the first moment of the Boltzmann equation. 
The Euler equation in fluid dynamics, 
%
\begin{align}
\rho \dv{\vec{v}}{t} = - \rho \vec{\nabla} \phi - \vec{\nabla} P
\,,
\end{align}
%
is analogous to it. 

A typical example is measuring the density of dark matter 
in pressure-supported systems such as dwarf galaxies. 

For spherical symmetry we get the Spherical Jeans equation. 

These dwarf spheroidal galaxies are nice DM laboratories: they have radii of just about \SI{1}{kpc}. 

There is a general circular velocity versus radius relation. 

In the inner part: mapping how the gas is moving, with the \SI{21}{cm} line. 
Near us, the way is to finely track the velocities of local object. 

Just outside, we mostly measure in the vertical component of the oscillation around the galactic disk. 
The systematics start to grow at larger radii. 

Far from the center, we look at halo stars; we can measure the velocity dispersion of those objects just like with dwarf spheroidal galaxies. 

DM is included as an ``ingredient'' in the \(\Lambda \)CDM model.
Together with the \(\Omega \)s, we have perturbation and couplings, 
such as the baryon-photon fluid before decoupling, with Compton and Coulomb scattering. 

The formation of LSS requires Dark Matter: 
%
\begin{align}
\frac{ \delta \rho _b}{\rho _b} \propto 1 + z
\,,
\end{align}
%
therefore it would not have had time to reach the collapse \(\delta \rho \sim \rho \). 

There are a few glitches in the concordance with \(\Lambda \)CDM; 
one is the \(H_0 \) tension, then we also have the \(\sigma _8\) tension: power spectrum normalization from the CMB and weak lensing.

The EDGES detection of the \SI{21}{cm} line seem to measure a larger absorption than was thought possible.

\end{document}