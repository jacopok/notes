\documentclass[main.tex]{subfiles}
\begin{document}

\marginpar{Wednesday\\ 2022-1-26}

The density profile of Dark Matter is uncertain: 
the NFW profile gives a \(1/r\) scaling at low radii, the Einasto profile 
is smoother in that region. 

This is the cusp-core problem, a mismatch between the results of simulations
and observations. 

For Dark Matter we introduce only gravity: there is, therefore, no characteristic 
scale, we expect self-similarity; a Harrison-Zel'dovich spectrum, once the 
system is virialized. 

This is the root of the missing satellite problem, 
the number of substructures in CDM simulations
is much larger than the satellites observed in the Milky Way. 

The too-big-to-fail problem is a reframing of this issue: the 
satellites of the Milky Way are not massive enough, the tail of the 
distribution. 

There seems to be too much power in \(\Lambda \)CDM on small scales. 
A possibility is also that baryonic components are not being properly treated. 

Reformulating the DM problem in terms of elementary particles is an \emph{assumption}; 
there are other possibilities, such as primordial black holes for examples. 

The dilute limit is the one in which 2-body interactions dominate over multi-body 
ones. 

How do we build a Lagrangian for DM? We would like to have its mass, its 

We have \textbf{5 golden rules} which we know for sure cannot be (strongly) violated: 
\begin{enumerate}
    \item it is \textbf{optically dark}:
    \begin{enumerate}    
        \item it does not couple to photons before recombination;
        \item it does not contribute significantly to background radiation at any frequency;
        \item it cannot cool by radiating photons (like baryons do when they form galaxies);
    \end{enumerate}
    \item it is \textbf{collisionless};
    \item it is in a \textbf{fluid limit}; 
    \item it is \textbf{classical};
    \item it is \textbf{not hot} (or, cold).
\end{enumerate}

Suppose we build a Lagrangian where the DM particle \(\chi \) couples to the photon 
with a milli-charge \(\epsilon e\): \(L_{\gamma \chi } = i \epsilon e \overline{\chi} \gamma^\mu \chi A_\mu\). 
This leads to a cross-section for interaction (Compton-like) with the photon, proportional to \(\epsilon^4 / m_\chi^2\),
and with baryons (Coulomb-like) proportional to \(\epsilon^2 / \mu^2_\chi\). 

Since we know that in the CMB DM cannot behave like baryons, these cross-sections 
are bound to be small, so we find \(\epsilon \lesssim \num{e-7}\). 
\todo[inline]{Does one run a whole analysis of the CMB peaks with a different likelihood?}

For other couplings we even have laboratory constraints. 

The fact that DM is optically dark means that it is dissipation-less. 
It cannot dissipate into visible photons, but not into any other dark particle either! 

We have simulations of tidal effects stripping apart satellite galaxies.

The collisionless nature of DM is shown very well in the Bullet cluster.
A mapping of gravitational lensing and one from X-rays show a displacement
between the DM and the baryons. 

\todo[inline]{How do we measure the peculiar velocity in the tangential direction?}

From it, we can infer a limit on the self-interaction cross-section for DM: 
%
\begin{align}
\frac{\sigma }{m} < \SI{1.25}{cm^2 / g}
\,.
\end{align}

Another limit can be computed from the fact that dwarf galaxies survive their
orbit around their host. 

We also have constraints from ellipticity: 
self-interactions tend to isotropize DM halos, so when we measure triaxial 
halos we can use that as a constraint. 

In particle physics units, we get a rather loose bound:
%
\begin{align}
\frac{\sigma}{m} \lesssim \SI{2e-24}{cm^2} \frac{m_\chi }{\SI{}{GeV}}
\,.
\end{align}

If DM were made of particles they would have accreted and re-emitted energy, 
from these effects we can find that \(M _{\text{DM}} \lesssim 43 M_{\odot}\). 
If DM were accreting we would see it in microlensing events. 

This kind of DM would be called MACHO, Massive compact halo object. 

We have upper limits for these going from \(\num{e-6} M_{\odot}\) up to 
a few solar masses. 

There is still a window for primordial black holes, 
there is the possibility to detect them with future GW observatories. 

We know that DM is confined on galactic scales, of \SI{1}{kpc}, 
with densities like a \SI{}{GeV/cm^3}, and velocities like \SI{100}{km/s}. 

If DM were bosonic, its De Broglie wavelength must be smaller than \SI{1}{kpc}, so 
%
\begin{align}
m \gtrsim \frac{h}{\SI{1}{kpc} v} \approx \SI{e-22}{eV}
\,.
\end{align}

Fuzzy DM is an ultralight scalar, approaching this limit. 

For fermions, we must require that the phase space density does not 
exceed that of a completely degenerate Fermi gas, which should also 
not have a maximal momentum larger than the escape velocity of the 
gravitational well, \(v _{\text{esc}} = \sqrt{2 G_N M / R}\). 
This  yields 
%
\begin{align}
m > \sqrt[4]{\frac{9 \pi }{2^{5/2} M^{1/2} R^{3/2} G_N^{5/2}}}
\,.
\end{align}

The relation we get this from is 
%
\begin{align}
\frac{M}{4/3 \pi R^3} \frac{1}{m} < \frac{g}{(2 \pi )^3} \frac{4 \pi }{3} p^3 < \frac{g}{(2 \pi )^3} \frac{4 \pi }{3} (m v _{\text{esc}})^3
\,.
\end{align}

A more sophisticated way to do this is the Gunn-Tremaine bound. 
In a collisionless system, the phase-space density is conserved: 
supposing that the DM was thermal initially with \(f = g/h^3\) we get 
%
\begin{align}
m > \SI{101}{eV} \left(\frac{\SI{100}{km/s}}{g \sigma }\right)^{1/4} \left(\frac{\SI{1}{kpc}}{r_e}\right)^{1/2}
\,.
\end{align}

This already allows us to reject SM neutrinos as DM candidates. 

Why is DM cold? If the kinetic term of the system were not negligible gravitational collapse
would not work. 

The free-streaming scale comes out to be 
%
\begin{align}
\lambda _{\text{FS}} = \SI{0.4}{Mpc} \frac{\SI{}{keV}}{m_X} \frac{T_X}{T} \left(1 +  \log (t _{\text{eq}} / t _{\text{NR}}) / 2\right)
\,.
\end{align}

Here \(t _{\text{eq}}\) is the time when the DM becomes non-relativistic. 

\todo[inline]{Why is that the definition of the free-streaming scale?}

The 5 golden rules imply that baryonic DM and hot DM are excluded, 
on the other hand non-baryonic CDM is favored. 

We can remove power on small scales by introducing a new length scale: 
\begin{enumerate}
    \item a free-streaming scale, with warm DM;
    \item a self-interaction scale;
    \item a ``quantum'' scale with fuzzy DM or DM forming a Bose-Einstein condensate;
    \item a (yet inaccessible) DM-baryon or DM-photon interaction scale.
\end{enumerate}

People are trying to reproduce these with N-body simulations. 

Particle physics people have mostly explored the possibilities of 
\begin{enumerate}
    \item DM as a \textbf{thermal relic}, thermal production and a freeze-in mechanism (WIMPs and sterile neutrinos, for example);
    \item DM as a \textbf{condensate} (axions and ALPs);
    \item DM generated \textbf{at a large temperature} (very massive candidates).
\end{enumerate}

MOND is falsified by 
\begin{enumerate}
    \item we have evidence for DM from other sources (e.g. cosmology);
    \item some galaxies do not have a flat rotation curve;
    \item the bullet cluster. 
\end{enumerate}



\end{document}
