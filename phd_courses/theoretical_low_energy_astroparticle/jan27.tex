\documentclass[main.tex]{subfiles}
\begin{document}

\section{The WIMP paradigm}

\marginpar{Thursday\\ 2022-1-27}

We will be assuming homogeneity and isotropy, without considering 
perturbations at any point. 

The fluids used in the standard model of cosmology can be written in terms of 
their microstates; in order to this however we need a very large number of particles. 

Because of this, instead of a microphysical description we will use a thermodynamic one.

The thermodynamics here will, however, be quite different from what we use classically, 
in lab conditions: for the universe, there can be no concept of an infinite thermalizer,
or of global equilibrium. 

However, we can talk of \emph{local} equilibrium: we introduce a notion of entropy \(S\),
and discuss when it reaches a stable maximum. 
The physical interpretation of this is to say that the local patch of microstates ``talk''
to each other more effectively than the patches' ``change rate''. 

The collision rate is denoted as \(\Gamma = \sigma v n\), and it defines a collision timescale \(t_c = 1 / \Gamma \). 
Also, we have an expansion timescale \(t_U = H^{-1}\). 

We will have local equilibrium when \(t_c \ll t_U\), or \(\Gamma \gg H\). 

Suppose we have a particle \(\chi \) with a mass \(m_\chi \), 
a number of internal degrees of freedom \(g_\chi \),
and a set of statistics, either Fermi-Dirac or Bose-Einstein. 

The phase space distribution function at equilibrium will be given by 
%
\begin{align}
\dv[3]{n^{\text{eq}}_\chi }{k} = f^{\text{eq}}_\chi (x^\mu , k^\mu) = f^{\text{eq}}_\chi (t, E)
\,,
\end{align}
%
where \(E^2 = m^2_\chi + \abs{\vec{k}}^2\). 
This will be given by 
%
\begin{align}
f^{\text{eq}}_\chi (t, E) = \frac{g_\chi }{(2 \pi )^3} \left(
    \exp(\frac{E_\chi - \mu }{T}) \mp 1
\right)^{-1}
\,,
\end{align}
%
where we have a \(-\) for bosons, and a \(+\) for fermions. 
Also, \(T(t)\) is the temperature, and \(\mu _\chi (t)\) is the chemical potential for \(\chi \). 

Since there can be reactions in the form \(e^- + \gamma \to e^- + 2 \gamma \), the chemical 
potential of the photon must be zero.

If we were to observe \(\mu _\gamma \) to be different from zero (for example, in the CMB)
we should think that the assumption of local equilibrium is the one which was violated. 

The number density as a function of time is given by 
%
\begin{align}
n^{\text{eq}}_\chi  (t) = \int \dd[3]{k} f^{\text{eq}}_\chi 
\,.
\end{align}

The integral can be computed analytically in the \(m_\chi \ll T\) and \(m_\chi \gg T\) limits.
If \(m_\chi \ll T \), we have 
%
\begin{align}
n^{\text{eq}}_\chi = f \frac{\zeta (3)}{\pi^2} g_\chi T^3
\,,
\end{align}
%
where the factor \(f\) is equal to 1 for bosons, and \(3/4\) for fermions. 

If, on the other hand, \(m_\chi \gtrsim T\), as well as larger than \(\abs{\mu _\chi - T}\), we get 
%
\begin{align}
n^{text{eq}}_\chi = g_\chi \left(\frac{m_\chi T}{2 \pi }\right)^{3/2} \exp(- \frac{m_\chi - \mu _\chi }{T})
\,.
\end{align}

For the energy density we integrate \(\int \dd[3]{k} E f\), so we get 
%
\begin{align}
\rho^{\text{eq}}_\chi (t) = 
\begin{cases}
f_2 \frac{\pi^2}{30} g_\chi T^4  \qquad \text{ if } m_\chi \ll T \\
n^{\text{eq}}_\chi (m_\chi + \frac{3}{2} T) \qquad \text{ if } m_\chi \gg T
\,.
\end{cases}
\end{align}

The coefficient \(f_2\) is \(1\) for bosons and \(7/8\) for fermions. 

The mean value of the energy per particle, \(\expval{E}\), in the relativistic case 
is about \(2.70\) (bosons) or \(3.15\) (fermions) times \(T\). 

Within the Standard Model, we expect \(\mu _\gamma = 0\) and that is indeed what we see. 

\begin{extracontent}
A little exercise: consider any SM state, charged under any group with a structure constant \(\alpha \),
and look at its relativistic limit. 

From dimensional analysis, we can show that local equilibrium happens when 
%
\begin{align}
T \lesssim \frac{\alpha^2 M _{\text{Pl}}}{g_*^{1/2}}
\,.
\end{align}

The idea is to use \(t_U = H^{-1}\), where \(H = \sqrt{8 \pi G \rho / 3 M _{\text{Pl}}^2}\), 
where \(\rho \sim g_* T^4\), with 
%
\begin{align}
g_* = \sum _{i \in \text{BE}} g_i + \frac{7}{8} \sum _{i \in \text{FD}} g_i
\,.
\end{align}

The sum is taken over all relativistic degrees of freedom. 

For the SM at the highest temperatures we find \(g_* \approx 106.75\). 
\end{extracontent}

In practice, we do not use the full phase space distribution function but 
only describe its moments, which account for its normalization and shape. 

We will work in the \emph{dilute limit}: only considering 2-body scatterings. 
Really, we should be thinking of ensembles of particles. 

We think of a situation in which we have a stable DM particle \(\chi \), and 
some SM particle \(p\) in a ``thermal bath'' state. 
Stability means we do not have processes in the form \(\chi \to 2 p\). 

Instead, we will have processes in the form \(\chi + \overline{\chi} \leftrightarrow p + \overline{p} \)
as well as \(\chi + p \leftrightarrow \chi + p\). 

The first are number-changing for \(\chi \), the second are not. 
The first, therefore, affect \emph{chemical equilibrium},
while the second only affect the energy and the \emph{kinetic equilibrium}. 

Equilibrium for the first process is corresponds to \(\Gamma _{\text{ann}} > H\), 
equilibrium for the second corresponds to \(\Gamma _{\text{scatt}} > H\). 

The cross-sections for scattering and annihilation could be different, 
but crossing symmetry makes them similar; on the other hand in the scattering
rate the number density will account for all the particles in the heat bath, 
therefore we typically have \(\Gamma _{\text{scatt}} \gg \Gamma _{\text{ann}}\). 

This implies that chemical equilibrium is lost a long time before 
kinetic equilibrium is. 

\subsection{The Boltzmann equation}

Its basic structure, for a particle \(\chi \) with all its properties, reads
%
\begin{align}
\hat{L} [f_\chi ] = \hat{C} [f_\chi ]
\,.
\end{align}

The Liouville operator \(\hat{L}\) describes the free, geodesic propagation of 
the particle: it will read 
%
\begin{align}
\hat{L} = k^\alpha \pdv{}{x^\alpha } \underbrace{- \Gamma^\alpha_{\beta \gamma} k^\beta k^\gamma}_{= \dv*[2]{x^\alpha }{\tau }} \pdv{}{k^\alpha }
\,.
\end{align}

Since we are only looking at the time dependence we only need the \(\alpha = 0\) component, 
and the relevant Christoffel symbols read 
%
\begin{align}
\Gamma^0_{ij} = \frac{\dot{a}}{a} \gamma_{ij}
\,,
\end{align}
%
where \(\gamma_{ij}\) is a flat 3D metric. 

Then, the Liouville operator applied to \(f_\chi \) reads 
%
\begin{align}
\hat{L}[f_\chi ] = E \pdv{f_\chi }{t} - H \abs{\vec{k}}^2 \pdv{f_\chi }{E}
\,.
\end{align}

In order to track chemical equilibrium we just need the normalization 
of this equation: 
%
\begin{align}
n_\chi = \int \dd[3]{k} f_\chi  
\,,
\end{align}
%
and the kinetic integral 
%
\begin{align}
T_\chi = \frac{2}{3} \frac{1}{n_\chi } \int \dd[3]{k} \frac{\abs{\vec{k}}^2}{2 m_\chi } f_\chi 
\,.
\end{align}

We will always assume that this temperature is equal to the overall temperature 
of the thermal bath \(T\). 

Integrating the Boltzmann equation over \(\dd[3]{k}\) we find 
%
\begin{align}
\dv{n_\chi }{t} - H \int \dd[3]{k} \frac{k^2}{E} \pdv{f_\chi }{E}  
\,,
\end{align}
%
and we can integrate by parts to get, since \(E \dd{E} = k \dd{k}\):
%
\begin{align}
4 \pi \int \dd{k} k^2 \frac{k^2}{E} \pdv{f_\chi }{E} &= 4 \pi \int \dd{E} k^3 \pdv{f_\chi }{E}  \\
&= \eval{4 \pi k^3 f_\chi }_{k=0}^{k \to \infty} - \int \dd{E} f_\chi 3 k^2 \pdv{k}{E}  \\
&= - 3 H n_\chi 
\,.
\end{align}

Therefore, the equation reads 
%
\begin{align}
\dv{n_\chi }{t} + 3 H n_\chi = \int \frac{ \dd[3]{k}}{E} \hat{C}[f_\chi ]
\,.
\end{align}

If there are no collisions then \(n_\chi a^3\), the number of particles in a comoving volume, 
is conserved.

The collisional term is written as 
%
\begin{align}
\int \frac{\dd[3]{k}}{E} \hat{C}[f_\chi ] 
= - \int \dd{\Pi _\chi } \dd{\Pi _{\overline{\chi}} }
\dd{\Pi _p } \dd{\Pi _{\overline{p}} }
\left(
    \abs{M_{ \text{forward} }}^2  h_\chi h_{\overline{\chi}} (1 \mp h_p) (1 \mp h_{\overline{p}})
    - \abs{M_{\text{reverse}}}^2 h_p h_{\overline{p}} (1 \mp h_\chi ) (1 \mp h_{\overline{\chi}})
\right)
(2 \pi )^{4} \delta^{(4)} (k_\chi^{\mu } + k_{\overline{\chi}}^{\mu } - k_p^\mu - k_{\overline{p}}^\mu )
\,.
\end{align}

Here, \(h_i = f_i / (g_i / (2 \pi )^3)\), while 
%
\begin{align}
\dd{\Pi _i} = \frac{g_i}{(2 \pi )^3} \frac{ \dd[3]{k_i}}{2 E_i}
\,.
\end{align}

The minus signs refer to fermions, the plus signs to bosons. 

Why to include two different matrix elements? 
Well, if we have CP symmetry we will get 
%
\begin{align}
\abs{M_{ \text{forward}}}^2 = \abs{M _{\text{backward}}}^2
\,.
\end{align}

It is not a given that this will be the case.

In the dilute limit, besides considering only two body interactions we also assume 
\(h_i \ll 1\), therefore \(1 \pm h_i \approx 1\). 

We can approximate these \(h_i\) as 
%
\begin{align}
h_i^{\text{eq}} \approx \exp(- \frac{E_i - \mu _i}{T})
\,,
\end{align}
%
so with all this we get 
%
\begin{align}
\int \frac{ \dd[3]{k}}{E} \hat{C}[f_\chi ] - \int \left\lbrace \dd{\Pi _i}\right\rbrace_i
\abs{M}^2 \left( h_\chi h_{\overline{\chi}} - h_p^{\text{eq}} h_{\overline{p}}^{\text{eq}} \right)
(2 \pi )^{4} \delta^{(4)} (\dots)
\,.
\end{align}

Further, we will have kinetic equilibrium, thereby 
%
\begin{align}
h_\chi (E, t) = \frac{n_\chi (t) }{n_\chi^{\text{eq}} (t)}  h_\chi^{\text{eq}} (t, E)
\,:
\end{align}
%
the shape of the distribution is the same at all times. 

The product of these two distribution functions will read 
%
\begin{align}
h_p^{\text{eq}}
h_{\overline{p}}^{\text{eq}}
= 
\exp(- \frac{E_p + E_{\overline{p}}}{T}) = \exp(- \frac{E_\chi + E_{\overline{\chi}}}{T}) = 
h_\chi^{\text{eq}}
h_{\overline{\chi}}^{\text{eq}}
\,.
\end{align}

Thanks to this, we get the full Boltzmann equation: 
%
\begin{align}
\dv{n_\chi }{t} + 3H n_\chi = - \expval{\sigma v}_T \left(n_\chi n_{\overline{\chi}} - n_\chi^{\text{eq}} n_{\overline{\chi}}^{\text{eq}}\right) 
\,, 
\end{align}
%
where 
%
\begin{align}
\expval{\sigma v}_T = \frac{1}{n^{\text{eq}}_\chi n^{\text{eq}}_{\overline{\chi}}} 
\int \left\lbrace \dd{\Pi _i}\right\rbrace_i
\abs{M}^2 
\exp(- \frac{E_\chi + E_{\overline{\chi}}}{T})
(2 \pi )^{4} \delta^{(4)} (\dots)
\,.
\end{align}

\end{document}