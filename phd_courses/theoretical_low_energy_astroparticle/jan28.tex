\documentclass[main.tex]{subfiles}
\begin{document}

\marginpar{Friday\\ 2022-1-28}

We reached an expression for the evolution of the number density of particle \(\chi \).

We now want to heuristically follow it up until the moment of decoupling. 

When \(\mu _\chi = 0\), we get 
%
\begin{align}
n_\chi^{\text{eq}} - n_{\overline{\chi}}^{\text{eq}} = 2 g_\chi 
\left(\frac{m_\chi T}{2 \pi }\right)^{3/2}
\exp( - \frac{m_\chi}{T}) \sinh(\frac{\mu_\chi }{T})
\,.
\end{align}

If \(\mu _\chi = \mu _{\overline{\chi}}\), we get \(n_\chi = n_{\overline{\chi}}\). 

The situation in which \(\mu _\chi \neq 0\) is called \emph{asymmetric DM}. 

With this requirement, our equation simplifies to 
%
\begin{align}
\dv{n_\chi }{t}
+ 3 H n_\chi 
= - \expval{\sigma v}_T 
\left( n_\chi^2 - \left(n_\chi^{\text{eq}}\right)^2\right)
\,.
\end{align}

We want to remove the expansion rate of the universe from the equation: 
therefore, we try to normalize \(n_\chi \) to something which 
is conserved in a comoving volume. 

It's convenient to normalize to the entropy density: 
%
\begin{align}
s 
= \frac{\rho + P}{T} 
\approx \frac{\rho _{\text{rel}} + P _{\text{rel}}}{T} 
= \frac{4}{3} \frac{ \rho _{\text{rel}}}{T}
= \frac{4}{3} \frac{\pi^2 }{45} g_* T^3 
\,.
\end{align}

We can compute this at any time: right now we have \(s_0 \approx \SI{2900}{cm^{-3}}\). 

The quantity is called \(Y_\chi = n_\chi /s\), whose derivative is 
%
\begin{align}
\dv{Y_\chi }{t} = \frac{1}{s} \dv{n_\chi }{t} - \frac{1}{s^2} n_\chi \dv{s}{t}
\,.
\end{align}

Since \(a^3s\) is constant, we get 
%
\begin{align}
\dv{s}{t} = - 3H ss
\,.
\end{align}

This  means that 
%
\begin{align}
\dv{Y_\chi }{t} = \frac{1}{s} \left( \dv{n_\chi }{t} + 3H n_\chi \right)
\,.
\end{align}

The variation of \(s\) with temperature reads 
%
\begin{align}
\dv{s}{T} = \frac{3s}{T} \left(1 + \frac{1}{3} \frac{T}{g_*} \dv{g_*}{T} \right) = \frac{3s}{TZ}
\,,
\end{align}
%
where \(Z \approx 1 \) takes account of the correction by the variation of \(g_*\). 

Therefore, the variation of \(Y_\chi \) with temperature reads 
%
\begin{align}
\dv{Y_\chi }{T} 
= \dv{Y_\chi }{t} \dv{s}{T} \left(\dv{s}{t}\right)^{-1}
= \frac{\expval{\sigma v}_T}{s H T} \left( n_\chi^2 - \left( n_\chi^{\text{eq}}\right)^2 \right)
\,.
\end{align}

In the end, therefore, the Boltzmann equation reads 
%
\begin{align}
\dv{Y_\chi }{T} 
&= \frac{ \expval{\sigma v}_T s}{HT} \left( Y_\chi^2 - \left( Y_\chi^{\text{eq}}\right)^2\right) \\
\frac{T}{Y_\chi^{\text{eq}}}  \dv{Y_\chi }{T} &= \frac{ \expval{\sigma v}_T n_\chi^{\text{eq}}}{HT} \left( \frac{Y_\chi^2}{(Y_\chi^{\text{eq}})^2} - 1\right)  \\
&= \frac{\Gamma}{H} \left( \frac{Y_\chi^2}{(Y_\chi^{\text{eq}})^2} - 1\right)
\,.
\end{align}

This clarifies what is meant by \(\Gamma \) when comparing \(\Gamma \) and \(H\): 
\(\Gamma (T) = \expval{\sigma v}_T n_\chi^{\text{eq}}\). 

On the other hand, 
%
\begin{align}
H = \sqrt{\frac{8 \pi }{3 m _{\text{Pl}}^2} g_* \frac{\pi^2}{30} T^4}
\,.
\end{align}

While the species is relativistic, \(\Gamma \propto T^3\) while \(H \propto T^2 / M _{\text{Pl}}\). 
Therefore, there will be a configuration at high \(T\) such that \(\Gamma (T) \gg H(T)\). 

The opposite is true at low temperatures. 
So, at high temperatures \(Y_\chi \approx Y_\chi^{\text{eq}}\). 
On the other hand, there will be a freeze-out temperature \(T _{\text{freeze}}\) such that 
\(\Gamma \approx H\), 
and at smaller \(T\) we will remain at \(Y_\chi (T) \approx Y_\chi(T _{\text{freeze}})\). 

This will also be the case up until the current time. 

We can use this to compute the current relic density \(\Omega _\chi \) for a cold relic,
such that \(T _{\text{freeze}} \lesssim m_\chi \). 

Cold relic is not a synonym for CDM: that means that \(\chi \) is non-relativistic at the time 
of matter-radiation equality, which is much smaller than the freezeout temperature typically. 

Cold relic implies CDM, but not the other way around. 

We get: 
%
\begin{align}
\Omega _\chi h^2 &= \frac{\rho _\chi (T_0 )}{\rho _{\text{crit}}(T_0 ) h^{-2}} \\
&= \frac{m_\chi n_\chi (T_0 )}{\rho _{\text{crit}}^0 h^{-2}}  \\
&= \frac{m_\chi Y_\chi (T_0 ) s_0 }{\rho _{\text{crit}} h^{-2}}
\,.
\end{align}

However, we also know that \(\Gamma (T _{\text{freeze}}) = \expval{\sigma v}_T Y_\chi^{\text{eq}} (T _{\text{freeze}}) s( T _{\text{freeze}}) \). 
This all yields 
%
\begin{align}
\Omega _\chi h^2 &= \frac{m_\chi s_0 }{\rho _{\text{crit}}^0 h^{-2}}   \\
&\approx \frac{m_\chi s_0}{\rho _{\text{crit}}^0} \frac{H (T _{\text{freeze}})}{s(65536T _{\text{freeze}}) \expval{\sigma v}_T}  
\,.
\end{align}

This comes out to be 
%
\begin{align}
\Omega _\chi h^2 = \frac{1}{g_*^{1/2} (T _{\text{freeze}})} \frac{m_\chi}{T _{\text{freeze}}}
\frac{\SI{e-27}{cm^3 s^{-1}}}{\expval{\sigma v}_{T _{\text{freeze}}}}
\,.
\end{align}

What is the value of the thermally-averaged cross-section \(\expval{\sigma v}\)? 

We get 
%
\begin{align}
\exp( \frac{m_\chi}{T _{\text{freeze}}}) \left(\frac{m_\chi }{T _{\text{freeze}}}\right)^{-1/2}
= K = \frac{3 \sqrt{5}}{4 \sqrt{2} \pi^3} \frac{g_\chi }{g_*^{1/2}} \sigma_0 m_\chi m _{\text{Pl}} 
\,.
\end{align}

This can be solved iteratively. 
For a mass \(m_\chi \approx \SI{100}{GeV}\), and a cross-section \(\sigma_0 \approx \SI{2.5e-27}{cm^3 s^{-1}}\), 
with \(g_* \approx 60\), we find \(m_\chi / T _{\text{freeze}} \approx 20\).

This yields 
%
\begin{align}
\Omega _\chi h^2 \approx \frac{\SI{3e-27}{cm^3 s^{-1}}}{\expval{\sigma v}_{T _{\text{freeze}}}}
\,.
\end{align}

That's the WIMP miracle: we get \(\Omega _{\text{DM}} h^2 \approx 0.1\)
with a value for \(\sigma v\) compatible with the weak interaction. 

The recipe for WIMPs is 
\begin{enumerate}
    \item we need a massive particle, which is non-relativistic at freeze-out;
    \item we need it to be stable (within the current lifetime of the universe);
    \item we need it to have a weak-interaction-like coupling to the SM.
\end{enumerate}

This third point also implicitly constrains the mass scale: 
a lower limit comes from the fact that the reaction \(\chi \overline{\chi} \leftrightarrow F \overline{F}\)
must be at equilibrium for some SM particle. 
This typically does not work within the SM unless \(m_\chi \gtrsim \SI{1}{GeV}\);
if there is a beyond-the-standard model thermalizer, we could go down to \(m_\chi \gtrsim \SI{1}{MeV}\). 

There is also an upper limit, 
which comes from unitarity when making a partial wave decomposition 
of the cross-section \(\sigma = \sum _{J} \sigma _J\). 
These are bounded by 
%
\begin{align}
\sigma _J \lesssim \frac{\pi (2 J + 1)}{p_i^2}
\qquad \text{where} \qquad
p_i \approx m_\chi \frac{v _{\text{rel}}}{2}
\,,
\end{align}
%
therefore 
%
\begin{align}
(\sigma _J v _{\text{rel}}) _{\text{max}} \approx \frac{4 \pi (2J+1)}{m_\chi^2 v _{\text{rel}}}
\,.
\end{align}

The relative velocity is fixed by equipartition; 
%
\begin{align}
\frac{1}{3} m_\chi v _{\text{rel}}^2 = 3T
\,,
\end{align}
%
which comes out to be \(v _{\text{rel}} \approx \sqrt{ 6 T _{\text{freeze}}/ m_\chi } \approx \sqrt{6/20}\).

The various \(\sigma _J\) with \(J > 0\) are smaller by a factor \(v^2 _{\text{rel}}\) each;
this is an upper bound on \(\expval{\sigma v}\), 
combining which with what we know about DM abundance 
leads to a lower limit on the mass of \(\chi \) in the form 
%
\begin{align}
m_\chi \lesssim \SI{90}{TeV}
\,.
\end{align}

\subsection{The Higgs portal for a scalar singet}

Consider the SM plus a single real scalar field \(S\), 
which is a SM singlet but which couples to the SM Higgs. 
This is based on early work by Silvera and Zee in 1985. 

This is the most minimal model outside the SM.

The Lagrangian includes some interactions, which we choose so that \(S\) is stable: 
%
\begin{align}
\mathscr{L} = \mathscr{L} _{\text{SM}} + \frac{1}{2} \partial_\mu S \partial^\mu S 
- \frac{m_S^2}{2} S^2
- \frac{\lambda_s}{4} S^4
- \lambda S^2 (H ^\dag H)
\,,
\end{align}
%
where we introduce a \(\mathbb{Z}_2\) symmetry: 
we write a Lagrangian which is symmetric under \(S \to - S\). 

A trick to make a renormalizable Lagrangian is to not have any term with a 
dimension which is larger than 4 in the field. 

What is the potential for \(S\)? 
In the unitarity gauge we can write \(H ^\dag  = (h, 0 ) / \sqrt{2}\), 
so we get 
%
\begin{align}
V = \frac{m_0^2}{2} S^2 + \lambda _s S^4 + \frac{\lambda}{2} S^2 h^2
+ \lambda _h (h^2 - v_{EW}^2)^2
\,.
\end{align}

We need to impose that the minimum of \(V\) is the same as the minimum of \(V _{\text{SM}}\): 
the configuration \(h = v _{EW}\).
This is used to put constraints on the various couplings. 

After EW symmetry breaking, and mapping \(h \to h + v_{EW}\) we get 
%
\begin{align}
V = V _{\text{SM}} + \frac{1}{2} (m_{0}^2 + \lambda v_{EW}^2) S^2
+ \frac{\lambda_S}{4} S^4 
+ \frac{\lambda}{2} h^2 S^2
+ \lambda v_{EW} h S^2
\,.
\end{align}

Exam: oral or presentation on a given topic.

\end{document}