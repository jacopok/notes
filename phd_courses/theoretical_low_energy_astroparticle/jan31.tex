\documentclass[main.tex]{subfiles}
\begin{document}

\marginpar{Monday\\ 2022-1-31}

The Lagrangian therefore reads 
%
\begin{align}
\mathscr{L} = \mathscr{L} _{\text{SM}} + \frac{1}{2} \partial_\mu S \partial^\mu S - \frac{1}{2} m_S^2 S^2 - \lambda v _{\text{EW}} h S^2 - \frac{\lambda}{2} h^2 S^2
\,.
\end{align}

We have a coupling \(\lambda \) between a Higgs \(h\) and two \(S\)s, 
then this \(h\) can couple to two fermions like \(h F \overline{F}\)
with a coupling constant \(y_F\).

What happens with thermal equilibrium? 
We need to consider the cases \(T \gg m_h\) and \(T \ll m_h\). 

If \(T \gg m_h\), the Higgs mass is irrelevant, so 
when we estimate \(\Gamma = n_T \sigma _{\text{ann}} v\) we get something proportional to \(\Gamma \propto T^3 \times \lambda^2 / T^2 \times 1 = \lambda^2 T\). 

On the other hand, if \(T \ll m_H\) we get \(\Gamma \sim T^3 \lambda^2 T^2/m_h^4 \times 1 \sim \lambda^2 T^5 / m_h^4\). 

The velocity is still relativistic, therefore we are assuming that \(m_S \ll T \ll  m_h\). 

The expansion of the Universe scales like 
%
\begin{align}
H \sim g_*^{1/2} \frac{T^2}{m _{\text{Pl}}}
\,.
\end{align}

We get decoupling when \(\max (\Gamma /H) \gtrsim 1\): 
this means 
%
\begin{align}
\eval{\frac{\Gamma}{H} }_{T \sim m_P} \sim \lambda^2 \frac{M_p}{m_h} g_*^{-1/2}
\,.
\end{align}

This requires \(\lambda \gtrsim g_*^{1/4} \sqrt{ m_h / M_p} \sim \num{e-8}\).

Therefore, we can have thermalization as long as \(\lambda \gtrsim \num{e-8}\) --- the coupling can be rather weak while still yielding thermalization.

We now want to compute the relic density: we get 
%
\begin{align}
\Omega _S h^2 \sim \frac{\SI{3e-27}{cm^3 s^{-1}}}{\expval{\sigma _s v}_T}
\,.
\end{align}

Let us make an estimate for this thermally averaged cross-section: 
%
\begin{align}
\expval{\sigma _S v} = \frac{g \lambda^2 v _{\text{EW}}^2}{(4 m_S^2 - m_h^2)^2 + m_h^2 \Gamma^2_h} F_h(m_s)
\,,
\end{align}
%
where we split the initial Feynman diagram in two. 

The width for the Higgs transition is roughly \(\Gamma _h \sim \SI{4.2}{MeV}\). 

The factor \(F_h(m_s)\) is given by 
%
\begin{align}
F_h (m_s) = \sum _{X} \lim_{m_{\widetilde{h}} \to 2 m_s} \frac{\Gamma (\widetilde{h} \to X )}{m_{ \widetilde{h}}}
\,,
\end{align}
%
where \(X\) is any allowed final state. 

Since \(m_S \ll m_h\), we will have 
%
\begin{align}
\sigma _s v \sim \frac{\lambda^2 m_s^2}{m_h^2}
\,.
\end{align}

Since this is constrained by observation of \(\Omega _S\), we must have \(\lambda m_s\) equal to some fixed value. 

When, however, we reach \(m_s \approx m_h/2\) we get a huge enhancement due to resonance. 

In a \(\log \lambda \) against \(\log m_s\) plot this corresponds to a large \emph{dip}, since we need a much smaller value of \(\lambda \) in order to balance such a natively high cross-section. 

The width of this dip will not just be \(\Gamma _h \sim \SI{4.2}{MeV}\) because of Doppler broadening. 

After the dip, more final states will open up. 
So, if \(m_s \gg m_h\) then the term \(\lambda h^2S^2\) in the Lagrangian also starts to become relevant. 

Then, we get \(\sigma _S v \approx \lambda^2 / m_S^2\). 

This leads to a high-energy regime where, roughly, \(\lambda / m_s\) is constant. 
Before that, we get a region of roughly constant \(\lambda \). 

This curve corresponds to a fixed value of \(\Omega _s h^2\); above it we get under-production and below it we get over-production.

Can we get other constraints? 
If \(m_S < m_h / 2\), we could have \(h \to SS\) decay: this would contribute to the ``invisible width'' in Higgs decay. 

The rest of the invisible width is accounted for with neutrinos. 

This excludes a portion of the parameter space in the low \(m_s\) region. 

Most of the rest of the line is excluded by other reasons, what remains as a viable candidate is only the resonance dip. 

\subsection{Supersymmetric WIMPs}

Just a sketch. 

A supersymmetric version of the Standard Model can be given, for example, by the Minimal SUSY SM, in which to any SM degree of freedom
we associate one degree of freedom in such a way that the setup is symmetric under the exchange of fermionic and bosonic degrees of freedom. 

R-parity is in the form \(R = (-1)^{3(B-L) + 2S}\). 
SM particles have \(R = +1\), and the SUSY partners have \(R = -1\). 

SUSY particles appear in pairs in scatterings and annihilations, 
the result of this is that the lightest SUSY particle is stable. 

Is this LSP a WIMP DM candidate? 
We want it to have no electric charge, no \(SU(3)_c\) charge, and no \(SU(2)_L\) charge. 

One possibility is the neutrino's couterpart: this has been excluded by direct detection. 

Another possibility is to look at \(SU(2)_L \times U(1)_Y\) gauge, Higgs with 0 electric charge.
The gluino, on the other hand, is excluded because it would be too strongly interacting. 

We could have the SUSY \(U(1)_Y\) partner \(\widetilde{B}\), 
the neutral partner of the \(SU(2)\) gauge bosons \(\widetilde{W}^3\), 
the SUSY partner of the Higgs doublet \(\widetilde{H}_u\) or \(\widetilde{H}_d\). 

These are 4 spin \(1/2\) particles which are Majorana fermions and 
which are collectively called \emph{neutralinos}. 

There must be some soft SUSY breaking, since we didn't observe any of them yet. 
So, there must be a mass matrix with off-diagonal terms. 

Depending on the hierarchy between these masses, the LSP could be 
the Higgsino (an \(SU(2)_L\) doublet),
the Wino (an \(SU(2)_L\) triplet), or
the Bino (an \(SU(2)_L\) singlet).

The states \(\widetilde{\chi}^0\) and \(\widetilde{\chi}^{\pm}\) are nearly degenerate in mass, but it's hard to measure this at LHC. 

These particles \(\chi _i\) are all sharing a quantum number, so 
they all ``talk'' to each other. 
Suppose that the masses are ordered such that \(M_1 < M_2 < \dots\); 
then if the temperature of freeze-out is given by 
%
\begin{align}
T _{\text{FO}} \sim \frac{M_1}{20}
\,
\end{align}
%
the particles with \(M_i - M_1 \lesssim T _{\text{FO}} \) will remain non-relativistic until they decouple. 

There are many processes causing a change in \(\chi _i\) abundance. 

The full Boltzmann equation will read 
%
\begin{align}
\dv{n_i}{t} + 3 H n_i = - \sum _{j} \expval{\sigma _{ij} v} \left(
    n_i n_j - n_i^{\text{eq}} n_j^{\text{eq}}
\right)
- \sum _{j, P, P'} \expval{\sigma_{ij}' v} \left(
    n_i n_P^{\text{eq}} 
    - n_j n_{P'}^{\text{eq}}
\right)
- \sum _{j< i} \Gamma_{i \to j} n_i 
+ \sum _{j > i} \Gamma_{j \to i} n_j
\,.
\end{align}

This will be the case for any \(i\) between \(1\) and \(N\). 
At late times, however, all the \(i > 1\) will have decayed into \(1\):  
therefore, the relic density will be given by \(n _{\text{tot}} = \sum _{i} n_i\), which constrains \(\Omega _1\). 

The result is then 
%
\begin{align}
\dv{n}{t} + 3 H n = - \sum _{i, j} \expval{\sigma _{ij} v} \left(
    n_i n_j - n_i^{\text{eq}} n_j^{\text{eq}}
\right)
\,.
\end{align}

We still have the individual densities on the right, but we 
can give an estimate for \(\Gamma \sim n_j \expval{ \sigma_{ij} v}\), while \(\Gamma _{\text{scatt}} \sim n_P^{\text{eq}} \expval{\sigma_{ij}' v}\). 

Since \(n_i^{\text{eq}} \gg n_j\), we expect \(n_i / n\) to scale like \(n_i^{\text{eq}} / n^{\text{eq}}\). 

This then yields an equation in the form 
%
\begin{align}
\dv{n}{t} + 3 H n &= - \expval{\sigma _{\text{eff}} v} \left(
    n^2 - n^2 _{\text{eq}}
\right)  \\
\expval{\sigma _{\text{eff}} v} &= \sum _{ij} \expval{\sigma_{ij} v} \frac{n_i^{\text{eq}} n_j^{\text{eq}}}{(n^{\text{eq}})^2}
\,.
\end{align}



\end{document}
