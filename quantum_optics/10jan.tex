\documentclass[main.tex]{subfiles}
\begin{document}

\section*{Fri Jan 10 2020}

The energy of the states \(\ket{n}\) is well defined, but there are still issues to sort out: for instance, the expectation value of the electric field is zero \emph{at each point}, 
%
\begin{align}
\bra{n} E_x(z, t) \ket{n} \propto \bra{n} \hat{a} \ket{n} + \bra{n} \hat{a}^\dag  \ket{n} = 0 
\,.
\end{align}

However, the expectation value of the \emph{square} of the electric field is nonzero: 
%
\begin{align}
\expval{E^2_{x} (z, t) } = 2 \mathcal{E}_{0}^2 \sin^2(kz) \qty(n+ \frac{1}{2})
\,.
\end{align}

This is in accordance with the uncertainty principle, since the operator \(\hat{n} \) does not commute with the electric field operator \(\hat{E}_{x}\). We can write the undetermination relation 
%
\begin{align}
\Delta n \Delta E \geq \frac{1}{2} \mathcal{E}_{0} \abs{\sin(kz)} \abs{\expval{\hat{a}^\dag - \hat{a}}}
\,.
\end{align}

We expect to be able to find a notion of phase such that there is a number-phase uncertainty relation, similarly to the time-energy uncertainty relation. 

The time evolution of the electric field operator is given by 
%
\begin{align}
E_{x} = \mathcal{E}_{0} \qty(\hat{a} e^{-i \omega t}  + \hat{a}^\dag e^{i \omega t} )\sin(kz)
\,,
\end{align}
%
and we define the quadrature operators: 
%
\begin{align}
X_{1} = \frac{1}{2} \qty(\hat{a} + \hat{a}^\dag) \qquad \text{and} \qquad
X_{2} = \frac{1}{2i} \qty(\hat{a} - \hat{a}^\dag)
\,.
\end{align}

We have effectively decomposed the electric field into two oscillating parts, out of phase with each other by \SI{90}{\degree}. We have the commutator \([\hat{X}_{1}, \hat{X}_{2}] = i/2\).
Even for the vacuum state the fluctuations of these operators are nonzero. 

We will now distinguish two different kinds of radiation: blackbody radiation and coherent (laser-like) radiation. 

For the first case, we know that the distribution of the energy levels is given by the Boltzmann distribution,
%
\begin{align}
P(n) = \frac{1}{Z} \exp(- \frac{E_n}{k_B T})
\,,
\end{align}
%
where \(Z\) is a normalization factor. 
We now give an intuitive justification. 

Let us suppose that we have a system of four particles, with three quanta of energy, which we  write as \(\Delta E\). How can this energy be distributed? 

\begin{table}[h]
\centering
\begin{tabular}{ccccc|c}
0 & \(\Delta E\) & \(2 \Delta E\)& \(3 \Delta E\)& \(4 \Delta E\)  & Possibilities\\
3 & 0 & 0 & 1 & 0 & 4 \\
2 & 1 & 1 & 0 & 0 & 12  \\
1 & 3 & 0 & 0 & 0 & 4 
\end{tabular}
\end{table} 

So, for \(0 \Delta E\) we have \(12+24+4 = 40\) total possibilities, for \(1 \Delta E\) we have \(12 +12 = 24\) total possibilities, for \(2 \Delta E\) we have \(12\) possibilities, for \(3 \Delta E\) we have \(4\) possibilities. The total is then 80. 

This kiind of looks like an exponential decrease, I guess if we were to do a more precise calculation we would get an exponential exactly. 

In a quantum setting, we will have a density matrix looking like 
%
\begin{align}
\rho = \frac{1}{\tr \exp(- \hat{H} / k_B T)} \exp(- \hat{H} / k_B T)
\,;
\end{align}
%
which gives a familiar result: 
%
\begin{align}
\rho = \sum _{n} \frac{\exp(- E_n / k_B T)}{Z} \dyad{n} 
\,.
\end{align}

The average number of photons can be found to be given by 
%
\begin{align}
\expval{n} = \frac{1}{\exp(\hbar \omega / k_B T) - 1}
\,.
\end{align}

In the limits \(\hbar \omega / k_B T\) going to either infinity or zero we get \(\expval{n} \rightarrow \hbar \omega / k_B T\) or its inverse. 

We can write the relation 
%
\begin{align}
\exp(- \hbar \omega /k_B T) = \frac{\overline{n}}{1 + \overline{n}
}
\,,
\end{align}
%
where \(\overline{n} = \expval{n}\). Then we will have 
%
\begin{align}
\rho = \frac{1}{1 + \overline{n}} \sum _{n} \qty(\frac{\overline{n}}{1 + \overline{n}})^{n} \dyad{n} 
\,.
\end{align}

It can be shown that 
%
\begin{align}
\expval{\hat{n}^2 } = \overline{n}  +2 \overline{n}^2
\,,
\end{align}
%
which implies 
%
\begin{align}
\Delta n = \qty(\overline{n} + \overline{n}^2)^{1/2}
\,,
\end{align}
%
so we have \(\Delta n \sim \overline{n} + \frac{1}{2}\) asymptotically. Therefore, there never is a well-defined number of photons in the box. 

We can write an expression for the average energy density \(U(\omega )\): 
%
\begin{align}
U(\omega ) = \frac{\hbar \omega^3}{\pi^2c^3} \frac{1}{\exp(\hbar \omega / k_B T) - 1}
\,.
\end{align}

The average energy of these photons is given by \(\hbar
 \omega \overline{n}\). 

From these expressions we can recover Wien's law and Stefan-Boltzmann's law. 

How do we represent a plane wave in a QFT? 
If we want a nonzero electric field we need a superposition of number states differing by \(\pm 1\). 

Another way to put it is: are there eigenstates \(\ket{\alpha }\) of the annihilation operator \(\hat{a}\)? 

They will look like 
%
\begin{align}
\ket{\alpha } = C_0 \sum _{n} \frac{\alpha^{n}}{\sqrt{n!}} \ket{n}
\,.
\end{align}

By normalization, \(C_0 = \exp( - \abs{\alpha }^2 / 2)\). This gives us a coherent state, which like we wanted has a nonzero expected electric field. It looks precisely like a plane wave: 
%
\begin{align}
\expval{\hat{E}_{x}}_{\alpha } = i \sqrt{\frac{\hbar \omega }{2 \epsilon_0 V }} \qty(\alpha \exp(i \vec{k} \cdot \vec{x} - i \omega t) - \alpha^{*} \exp(-i \vec{k} \cdot \vec{x} + i \omega t))
\,.
\end{align}

We find also the expectation value of the \emph{square} of the electric field 
%
\begin{align}
\expval{E^2_{x} }_{\alpha } = \frac{\hbar \omega }{2 \epsilon_0 V} \qty(1 + 4 \abs{\alpha }^2 \sin^2 (\omega t - \vec{k} \cdot \vec{r} - \theta ))
\,,
\end{align}
%
where \(\alpha = \abs{\alpha } \exp(i \theta )\). 

This means that not even in the vacuum state we can have a zero expected electric field. The vectors \(\ket{\alpha }\) are \emph{over-complete}, since they are bidimensional while a one-dimensional continuous basis would be enough to span the Hilbert space.

The average of the number of photons \(\hat{n}\) for an eigestate \(\ket{\alpha }\) is quickly calculated to be \(\abs{\alpha }^2\): then we can see that \(\abs{\alpha }^2 = \overline{n}\). 

So, with these states we have \(\expval{\hat{n}^2}_{\alpha } = \overline{n}^2 + \overline{n}\). Therefore \(\Delta n = \sqrt{\overline{n}}\). The probability of detecting \(n\) photons is given by \(\abs{\braket{n}{\alpha }}^2\): 
%
\begin{align}
P_{\alpha }(n) = \exp(- \overline{n} ) \frac{\overline{n}^{n}}{n!}
\,,
\end{align}
%
a Poissonian distribution. If the number of photons gets large then the Poissonian approaches a Gaussian. 



\end{document}