\documentclass[main.tex]{subfiles}
\begin{document}

\section*{Mon Jan 13 2020}

The distribution for a thermal source is more irregular than the Poissonian we get for the coherent light. 

Recall: our coherent states are given by 
%
\begin{align}
\ket{\alpha } = \exp(- \frac{1}{2} \abs{\alpha}^2) \sum _{n} \frac{\alpha^{n}}{\sqrt{n!}} \ket{n}
\,
\end{align}
%
where \(\ket{n}\) are the number eigenstates. 

We can define the displacement operator: 
%
\begin{align}
\hat{D} (\alpha ) = \exp( \alpha \hat{a}^\dag - \alpha^{*} \hat{a})
\,,
\end{align}
%
which, it can be proven, can give us the state \(\ket{\alpha }\) starting from the vacuum \(\ket{0}\): \(\hat{D} (\alpha ) \ket{0} = \ket{\alpha }\). 

We discuss interactions with an electromagnetic field: we have the interaction term in the Hamiltonian: 
%
\begin{align}
\hat{V}(t) = \int \dd[3]{r} \vec{j} (\vec{r},t) \cdot \hat{A} (\vec{r}, t)
\,,
\end{align}
%
where \(\hat{A}\) is the operator corresponding to the vector potential, and is given by 
%
\begin{align}
\hat{A} = 
\,.
\end{align}

In certain cases, we can only consider the dipole contribution since on the length scales of the problem the field is approximately constant. 

\dots

We can find an expression for \(\alpha \) by integrating a coupling term. 
The coherent states are not orthonormal: we have 
%
\begin{align}
\braket{\beta }{\alpha } = \exp( \frac{1}{2} \qty(\beta ^{*} \alpha - \beta \alpha^{*} -\abs{\beta -\alpha }^2 ))
\,,
\end{align}
%
which can never be zero: its square norm is \(\exp(-\abs{\beta- \alpha }^2)\) there are no orthogonal vectors here. Nevertheless, we are in a Hilbert space so we can write a completeness relation, even though we do \emph{not} have a basis. 

We have a distribution in the space of \(\alpha \): it is constrained by the uncertainty principle. We can use \emph{homodyne} detectors to select a quadrature to ``squeeze'' and get more resolution on. \emph{Heterodyne} means we split the signal and use both. 

We can use our QFT of light to solve the problem of the interaction of an EM field with an atom: the transformed Hamiltonian is 
%
\begin{align}
H' = \frac{1}{2m} \qty(\vec{P} + e \vec{A})^2 + V - e \Phi 
\,,
\end{align}
%
which is an operator equation, even though I omit the hats. 

If we consider both the Hilbert spaces, we will have a transition from an initial state \(\ket{a} \ket{n}\)  to either \(\ket{b} \ket{n+1}\) if a photon is emitted or to \(\ket{b} \ket{n-1}\) if a photon is absorbed. 

We compute the probability amplitudes of these by sandwitching the interaction Hamiltonian. 

The zero-point energy cannot be harvested for a transition: however, spontaneous emission can happen by interaction with the vacuum of the EM field. 

The interaction Hamiltonian is separable: its EM part is either \(\hat{a} \) or \(\hat{a}^\dag\), so we immediately get the result that 
%
\begin{align}
\frac{\abs{\expval{\text{emission}}}^2}{\abs{\expval{\text{absorption}}}^2} = \frac{n+1}{n}
\,,
\end{align}
%
so emission is slightly more probable. 

In the interaction Hamiltonian we need to consider the actual shapes of the orbitals of the atoms: it is a difficult search. 

\subsection{A crash course in LASER}

For this part, see the Saleh-Teich. 

Let us say we have two states with energies \(E_{1, 2}\) with \(E_2 - E_1 = \hbar \omega \). 
We can either have emission, absorption or stimulated emission, as we were discussing. 

The dipole term is approximately constant. 
The probability density of thhe deexcitation is given by 
%
\begin{align}
w_{i} = \frac{\overline{n}  }{t _{\text{sp}}} 
\,,
\end{align}
%
where \(t _{\text{sp}}\) is such that \(\mathbb{P} = 1 / t _{\text{sp}}\). 
\todo[inline]{What? what are the units here?}

We'd like to have amplification of the emission between the states 2 and 1: however the emission is always more likely than the absorption, asymptotically they have the same probability. 

The way to solve this is introducing more states: the easy way to do it is to introduce two of them, call them 0 and 3, one below and one above our 1 and 2. These are still excitation states of a certain atom. 

The \emph{pumping} is the temporal and spatial density of the \(0 \to 3\) transitions, then the state descends through 2, 1, and finally 0. 

We need to choose the atom appropriately, with a good probability of the atom absorbing the pumping, and a high probability of doing \(3 \to 2\). The \(2 \to 1\) transition must have a reasonably low probability. 
Neodymium is a good candidate. For it, the characteristic time of state \(2\) is of the order of the hundreds of \SI{}{\micro \second}. 

Recall the law of Boltzmann statistics: 
%
\begin{align}
\frac{N_2 }{N_1 } = \exp(-\frac{E_2 - E_1 }{k_B T})
\,.
\end{align}

The construction we made creates an \emph{inversion} of this population: the population of state \(2\) becomes larger than that of state \(1\). 

We are interested in \(N = N_2 - N_1 \): in the thermodynamical equilibrium case this is almost \(-N_1 \) since \(N_2 \) is negligible. 
In our case, instead, it becomes \(N = W t _{\text{sp}}\). 

It is useful to introduce the concept of \emph{optical gain}: 
let's say we have a cavity of length \(d\), then the frequency corresponding to the lowest mode is \(\nu_{f} = c / 2d \) and we can consider modes like \(\nu = q \nu_{f}\). 

The number flux is 
%
\begin{align}
\Phi = \frac{I}{h \nu }
\,,
\end{align}
%
where \(I\) is the light intensity; 
 we are interested in \(\dv*{\Phi }{z}\), where \(z \) is the coordinate along our cavity. 
 It will look like: 
 %
 \begin{align}
 \dv{\Phi }{z} = \gamma(\nu ) \Phi(z) 
 \,,
 \end{align}
 %
where \(\gamma = N \sigma (\nu )\), \(N\) being the one from above, \(N_2 - N_1 \). If this is positive, then we have optical gain. 

We want our laser to create a \emph{ray} of light: this is not trivial, the simplest thing is to create a cavity where one of the mirrors actually has a non-1 reflectivity, so that a certain portion of light escapes. 

This can be summarized with the average permanence time of a photon in the cavity: 
if the intensity in terms of the position looks like \(I(z)= I_0 \exp(-\alpha_{r} z)\) for some coefficient \(\alpha_{r}\), then we can also write \(\alpha_{r} = 1 / c \tau \), where we introduced the characteristic time \(\tau \).  

The rule will then be \(\gamma > \alpha_{r}\): if this is the case, then the radiation is amplified more than it gets out. 
If this is not the case, we basically have a thermal source. 

\end{document}
