\documentclass[main.tex]{subfiles}
\begin{document}

\marginpar{Friday\\ 2020-4-3, \\ compiled \\ \today}

Why is it important to choose the measurements randomly? 
If we already know what the measurement bases will be, then we can create a hidden variable model which will give the correct results. 

\section{Continuous variables in QM}

If we have a single mode, then we have 
%
\begin{align}
a ^\dag_{\lambda, \epsilon , \vec{k}} \overset{\text{def}}{=} a ^\dag
\qquad \text{where} \qquad
\qty[a, a  ^\dag] = 1
\,,
\end{align}
%
with which we can generate \(n\)-photon states as 
%
\begin{align}
\ket{n} = \frac{\qty(a ^\dag)^{n}}{\sqrt{n!}} \ket{0}
\,,
\end{align}
%
where our Hilbert space must be infinite-dimensional since we can have as many photons as we want. 

So, we can define 
%
\begin{align}
a = \hat{X}_{1} + i \hat{X}_{2}
\,,
\end{align}
%
where we have 
%
\begin{align}
\qty[\hat{X}_{1}, \hat{X}_{2}] = \frac{i}{2}
\,,
\end{align}
%
so this real-imaginary decomposition is like a position-momentum decomposition. 
These operators are called \emph{quadratures}, they are self-adjoint.

We are in second quantization, so the wavefunction is written as 
%
\begin{align}
\ket{\psi } = \int \dd[3]{k} \int \dd{\lambda } \sum_{\epsilon } a ^\dag_{\lambda, \epsilon , \vec{k}} \ket{0}
\,.
\end{align}

\subsection{P-function}

Any state can be written as 
%
\begin{align}
\rho = \sum _{n, m} \rho_{n, m} \ketbra{n}{m}
\,,
\end{align}
%
since the single-particle states are a basis. 
If we write a coherent state 
%
\begin{align}
\ket{\alpha } = e^{- \abs{\alpha }^2/2} \sum _{n} \frac{(a ^\dag)^{n}}{n!} \ket{0}
\,,
\end{align}
%
we can use the expression 
%
\begin{align}
\int \frac{ \dd[2]{\alpha }}{\pi } \dyad{\alpha } = \mathbb{1}
\,,
\end{align}
%
which allows us to write 
%
\begin{align}
\rho = \frac{1}{\pi } \int \dd[2]{\alpha} P(\alpha ) \dyad{\alpha }
\,.
\end{align}

Note that we have \(\dyad{\alpha }\), not \(\ketbra{\alpha }{\beta }\).
This can be done since the coherent states are not linearly independent, they are an overcomplete basis. 

This must be normalized, so that 
%
\begin{align}
\int \dd[2]{\alpha } P(\alpha ) = 1
\,.
\end{align}

If this \(P(\alpha )\) was always positive, then we would have classical states, a statistical mixture. 

The over-completeness relation is found by expanding the integral 
%
\begin{align}
\bra{n} \qty[ \int \frac{ \dd[2]{\alpha }}{\pi } \dyad{\alpha }] \ket{m} = \delta_{nm}
\,.
\end{align}

\subsection{Husini-function}

Given the \(P\)-function, we have 
%
\begin{align}
\hat{O} = \int \frac{\dd[2]{\alpha}}{\pi } P_O (\alpha ) \dyad{\alpha }
\,
\end{align}
%
for any operator \(\hat{O}\), so we will have 
%
\begin{align}
\expval{\hat{O}}_{\rho } = \Tr(\hat{O}\rho ) \int \frac{ \dd[2]{\alpha }}{\pi  } P_O(\alpha ) \ev{\rho }{\alpha }
\,,
\end{align}
%
so we define 
%
\begin{align}
\frac{1}{\pi } \ev{\rho }{\alpha } = Q(\alpha )
\,,
\end{align}
%
which is called the Husini function.

The generating function \(C(k)\) of a pdf \(p(x)\) is defined as 
%
\begin{align}
C(x) = \int e^{ikx} p(x) \dd{x}
\,,
\end{align}
%
so we will have 
%
\begin{align}
\eval{\dv[n]{C(k)}{x}}_{k=0} = i^{n} \expval{x^{n}}_{p}
\,.
\end{align}

We can do a similar thing in the quantum realm. 
We can define 
%
\begin{align}
\chi (\eta ) = \Tr[\rho \underbrace{\exp(\eta \hat{a} ^\dag - \eta^{*} \hat{a})}_{D(\eta )}]
\,,
\end{align}
%
which is somewhat like a Fourier transform. The operator \(D(\eta )\) is called the displacement operator: its action is 
%
\begin{align}
D(\alpha ) \ket{0} = \ket{\alpha }
\,,
\end{align}
%
so it generates the coherent states. 
In phase space, it translates from the vacuum to the coherent state \(\ket{\alpha }\). We can write it as 
%
\begin{align}
D(\alpha ) = \exp(\alpha a ^\dag - \alpha^{*} a) 
= \exp(- \frac{\abs{\alpha }^2}{2}) \exp(\alpha a ^\dag) \exp(- \alpha^{*} a)
\,,
\end{align}
%
but \(e^{- \alpha^{*} a} \ket{0} = \ket{0}\), since \(a \ket{0} = 0\). 

We also define the operators 
%
\begin{align}
\chi_{N} (\eta ) =
\Tr[\rho e^{\eta a ^\dag} e^{- \eta^{*} a}]
\qquad \text{and} \qquad
\chi_{A} (\eta ) =
\Tr[\rho  e^{- \eta^{*} a} e^{\eta a ^\dag}]
\,,
\end{align}
%
where the \(N\) stands for normal-ordering; these are 
%
\begin{align}
\chi_{N} (\eta ) = e^{ \frac{1}{2} \abs{\eta }^2} \chi (\eta )
\qquad \text{and} \qquad
\chi_{A} (\eta ) = e^{- \frac{1}{2} \abs{\eta }^2} \chi (\eta )
\,.
\end{align}

So, with these, we can define 
%
\begin{align}
P(\alpha ) = \frac{1}{\pi } \int \dd[2]{\eta } e^{ \overline{\eta} \alpha - \eta \overline{\alpha}} \chi_{N}(\eta ) \qquad \text{and} \qquad
Q(\alpha ) = \frac{1}{\pi } \int \dd[2]{\eta } e^{ \overline{\eta} \alpha - \eta \overline{\alpha}} \chi_{A}(\eta )
\,,
\end{align}
%
while 
%
\begin{align}
W(\alpha ) = \frac{1}{\pi } \int \dd[2]{\eta } e^{ \overline{\eta} \alpha - \eta \overline{\alpha}} \chi(\eta )
\,,
\end{align}
%
the latter being the Wigner function. 
These are all equivalent, we can use whichever is more convenient, which usually is the Wigner function. 

Similarly to the classical case, we have 
%
\begin{align}
\expval{a ^{\dag m} a^{n}}_{\rho } &= \pdv[n]{\eta } \pdv[m]{\overline{\eta}} \chi_{N} (\eta ) \\
\expval{a^{n} a ^{\dag m} }_{\rho } &= \pdv[n]{\overline{\eta} } \pdv[m]{\eta} \chi_{A} (\eta )\\
\expval{ \qty{ a^{n}, a ^{\dag m}} }_{\rho } &= \pdv[n]{\overline{\eta} } \pdv[m]{\eta} \chi (\eta ) \\
\,.
\end{align}

For coherent states we have 
%
\begin{align}
\ev{a}{\alpha } = \alpha 
\,,
\end{align}
%
while 
%
\begin{align}
\ev{\alpha }{\Delta X^2_{1}} =
\ev{\alpha }{\Delta X^2_{2}} 
= \frac{1}{4}
\,.
\end{align}

The Wigner function of a coherent state is 
%
\begin{align}
\ket{\beta } \rightarrow W(\alpha ) = \frac{2}{\pi } e^{-2 \abs{\alpha - \beta }^2}
\,.
\end{align}

Usually we will calculate probability densities like 
%
\begin{align}
\ev{x_1}{\rho } = p(x_1 ) = \int \dd{x_2 } W(x_1 , x_2 )
\,.
\end{align}

While \(W(\alpha )\) can be both negative and positive, its integral along any line in the complex plane will be \(\geq 0\), since we can define 
%
\begin{align}
X_{\theta } = \frac{1}{2} \qty(a e^{-i \theta } + a ^\dag e^{+i \theta })
\,.
\end{align}

In this coherent state, we will have 
%
\begin{align}
Q(\alpha ) = \abs{ \braket{\alpha }{\beta }}^2 = \frac{1}{\pi } e^{- \abs{\alpha - \beta }^2}
\,,
\end{align}
%
while 
%
\begin{align}
P(\alpha ) = \delta (\alpha - \beta )
\,.
\end{align}

How do we measure these? We need a type of measure called \textbf{homodyne detection}.

We have a beamsplitter, on one side we have our state \(\rho \), on the other we have a laser: an almost-classical state  \(\ket{\beta }\) with \(\beta \gg 1\). 

Outside the BS we then measure \(I_c - I_d\), which is the average on \(\rho \) of \(X_{\theta }\), where \(\beta = \abs{\beta } e^{i \theta }\). 

This is because \(c ^\dag = b ^\dag + i a ^\dag\) and \(d ^\dag = a ^\dag + i b ^\dag\). So, we are measuring 
%
\begin{align}
I_c - I_d \propto c ^\dag c - d ^\dag d &= 2 i a ^\dag b - 2 i b ^\dag a
\,,
\end{align}
%
but since we need to do this on the state \(\rho \otimes \dyad{\beta }\) we get 
%
\begin{align}
2 i \beta \expval{a ^\dag}_{\rho } - 2 i \beta^{*} \expval{a}_{\rho } 
\,,
\end{align}
%
which is precisely 
%
\begin{align}
2 i \abs{\beta } \expval{\qty[i e^{i \theta } a ^\dag - i e^{-i \theta }a]}_{\rho }
\,,
\end{align}
%
which corresponds to the Wigner function integrated on a line at an angle \(\theta \). 

This is a macroscopic measurement! We measure bulk currents, however if we measure the variance 
%
\begin{align}
\expval{\Delta I^2_{cd}} = 4 \abs{\beta }^2 \expval{\Delta X_{\theta }^2}_{\rho } + \order{ \abs{\beta} }
\,.
\end{align}

The noise here is intrinsically quantum. 
If \(\beta \gg 1 \) we can do this approximation. 

We can then do a sort of tomography on our state. 

A different measurement is called a double homodyne, or heterodyne. 

Now our detector is split in two, we use the same laser but in two beam splitters, with a respective phase of \(\pi / 2\). We are effectively using a POVM 
%
\begin{align}
\Pi_{\beta } = \frac{1}{\pi } \dyad{\beta }
\,,
\end{align}
%
where 
%
\begin{align}
\beta = X_1 + i X_2 
\,.
\end{align}

Then we will have \(\Tr (\rho \Pi_{\beta }) = \pi^{-1} \ev{\beta }{\rho } = Q(\beta )\). 

In classical communication, this is called a coherent detector. 

\section{Squeezing}

Coherent states are really classical states. 
We know that 
%
\begin{align}
\Delta X_1 \Delta X_2 \geq \frac{1}{4} 
\qquad \text{and} \qquad
\expval{ \Delta X^2_{\theta }} = \frac{1}{4} \forall \theta 
\,,
\end{align}
%
but this constraint can also be satisfied by ``squeezed'' states: these have large variances in one direction and small in the other. 

These are written using the squeezing operator: 
%
\begin{align}
S(\xi ) = \exp( \frac{1}{2} \qty(\xi^{*} a ^{\dag 2} - \xi a^2))
\,.
\end{align}

Since we have the squares of the operators, we have different behaviour from before. We can write \(S\) as  
%
\begin{align}
S(\xi ) = \frac{1}{\sqrt{\mu }} \exp( - \frac{1}{2} \frac{\nu }{\mu } \alpha ^{\dag 2}) \mu^{- a ^\dag a} \exp( \frac{1}{2} \frac{\overline{\nu}}{\mu } a^2)
\,,
\end{align}
%
where 
%
\begin{align}
\mu = \cosh r, \nu = \sinh r e^{i \varphi } \qquad \text{where} \qquad
\xi = r e^{i \varphi }
\,.
\end{align}
%
so we have 
%
\begin{align}
\ket{\xi } = S(\xi ) \ket{0} 
&= \frac{1}{\sqrt{\mu }} \exp(- \frac{1}{2} \tanh r e^{i \varphi } a ^{\dag 2})    \\
&= \frac{1}{\sqrt{\mu }} \sum _{n} \qty(\dots) \frac{a ^{\dag 2}}{\sqrt{n!}} \ket{0}
\,,
\end{align}
%
so this state will have only an even number of photons: we will have 
%
\begin{align}
\sqrt{\expval{\Delta X_{\varphi }}} = \frac{1}{2} e^{-r} \qquad \text{and} \qquad
\sqrt{\expval{\Delta X_{r}}} = \frac{1}{2} e^{+r}
\,,
\end{align}
%
so \(\varphi \) tells us along which direction we are squeezing, while \(r\) tells us by how much. 

This was applied on the vacuum, but then we can do 
%
\begin{align}
\ket{\xi , \alpha } = D(\alpha ) S( \xi ) \ket{0}
\,.
\end{align}

This only changes the average, the variances are kept. 
The operators \(S\) and \(D\) do not commute.

This can be experimentally generated using nonlinear crystals. 
The first term in the expansion of \(S(\xi )\) is precisely the first term in the spontaneous parametric down-conversion. 

The Hamiltonian in SPDC is 
%
\begin{align}
H = b a^{\dag 2} - b ^\dag a^2
\,,
\end{align}
%
so we will actually never have an odd number of photons. 

\end{document}
