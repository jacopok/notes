\documentclass[main.tex]{subfiles}
\begin{document}

\marginpar{Friday\\ 2020-4-10, \\ compiled \\ \today}

% Bin da 100 e 60 picosecondi. 

\section{Quantum Random Number Generators}

Random number: generated by an unpredictable process. 

Random numbers are crucial in several applications: 
\begin{enumerate}
    \item information technology and security;
    \item scientific simulations;
    \item games and lotteries.
\end{enumerate}

Classically, we imagine a coin toss or a dice roll. 
However, these are deterministic processes: if we know the initial conditions we can determine the outcome. 

We have Pseudo RNGs: they are simple and fast, but they have a period, they exhibit non-uniformity, (another thing). (predictability?)

RANDU was used by IBM from the sixties to the nineties. 

We start from a seed \(V_0 \), and then compute 
%
\begin{align}
V_{k+1} = 65539 \times V_k \mod 2^{31}
\,.
\end{align}

This gives us numbers in \([0, 2^{31}-1]\) which we can normalize to \([0, 1)\). 

However there are several flaws in PNRGs. 

We can use QRNG: for instance, a diagonal polarization sent through a polarizing beam splitter. 

Even if the initial state is perfectly known, the outcome is not predictable. 

\subsection{Fully trusted QRNG}

We have a fully trusted source \(\ket{\psi }\), and a fully trusted measurement. The probability of outcome \(\ket{z}\) is 
%
\begin{align}
P_z = \abs{\braket{z}{\psi }}^2
\,.
\end{align}

The way to quantify this is the classical min-entropy: 
%
\begin{align}
H_{ \infty } (Z) = - \max_{z} \qty(\log_2 P_z )
\,.
\end{align}

The issue is that our starting state may not be pure; instead it is a mixed state, which is pure when considered together with the eavesdropper's state. 

It is hard to evaluate the min-entropy in the presence of an eavesdropper. 

In general, we will have an entropy source, which is a physical system plus a measurement, from which we extract raw bits, which we then do post-processing to, and finally we read-out the random sequence. 

We have three scenarios, with decreasing speed and simplicity, and increasing security: 
\begin{enumerate}
    \item Trusted QRNG;
    \item Semi device independent QRNG;
    \item Fully device independent QRNG. 
\end{enumerate}

\subsection{Trusted QRNG}

We suppose that the starting state is indeed pure, so the min-entropy is indeed the classical one. 
We can do single-photon measurements, or we can do time-of-arrival calculations, or we can count the photons measured in a certain time. 

We can use vacuum fluctuations, we measure the quadratures as we saw last time: they have a defined variance, so we can subdivide the pdf of the homodyne measurement into equal-probability regions.

\subsection{Semi device independent QRNG}

How do we distinguish the randomness of a quantum superposition from the randomness of a mixed state? We can rotate the basis.

If we do this, then we can tell whether the state we are seeing is truly pure. 
The asymptotic mean entropy is lower in the two bases: the state was not truly pure. 

We can also have a measurement-device independent measurement. 

We can also do dimension-witness QRNG: we only assume that the state is a qubit. 

\subsection{Fully device independent QRNG}

We certify the entanglement by the violation of Bell's inequalities. 

LAB: tomogaphy + QKD with entangled photons. 
We will estabilsh the dates at the end of April, depending on whether the labs can reopen or not. 

\end{document}
