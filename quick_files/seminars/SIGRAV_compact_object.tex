\documentclass[main.tex]{subfiles}
\begin{document}

\section{Physics of compact objects in GR and beyond}

\marginpar{Monday\\ 2021-2-1, \\ compiled \\ \today}

Lecture by Gualtieri from Milan Bicocca.
We will use geometrized units \(G = c = 1\) (note that this is not the same as the particle-physics \(c = \hbar = 1\)).

References: Misner-Thorne-Wheeler, Weinberg, Poisson-Will (``Gravity''), Wald (``GR''), Chandrasekhar (``The mathematical theory of BH''), Ferrari-Gualtieri.

The compactness in general is defined to be \(C = M / R\), and we ask that it is \(\lesssim 1\) in order to be discussing a compact object.
The known compact objects are neutron stars (possibly quark stars) and black holes. 
Exotic compact objects are also a possibility.

Observables for black holes include: 
\begin{enumerate}
    \item Accretion disks: EM emission, mostly in the X-ray range;
    \item BH shadow: EM emission;
    \item motion of stars around supermassive BH;
    \item BBH coalescence: GW emission, ground-based detection for small (\(5 \divisionsymbol 100 M_{\odot}\)) BHs, space-based detection for massive (\(\num{e6} \divisionsymbol \num{e7} M_{\odot}\)) BHs;
    \item stochastic background.
\end{enumerate}

For NSs we have 
\begin{enumerate}
    \item accretion disks;
    \item binary pulsars: EM emisssion (radio);
    \item BNS coalescence with ground-based GW detectors;
    \item pulsars as sources of continuous GWs.
\end{enumerate}

We consider a \textbf{stationary, isolated} body in GR. 
``Isolated'' means that the spatial support of its stress-energy tensor \(T_{\mu \nu }\) is limited: it is inside a world-tube. 
The spacetime will be \emph{asymptotically flat}: a rather informal definition is the existence of a coordinate \(r\) such that 
%
\begin{align}
\lim_{r \to \infty } g_{\mu \nu } = \eta_{\mu \nu }
\,.
\end{align}

If \(R\) is the scale of the support of the stress-energy tensor, then we know that \(r \gg R\) is the far-field region, in which the field is \emph{weak}: \(g_{\mu \nu } = \eta_{\mu \nu } + h_{\mu \nu }\), with \(\abs{h_{\mu \nu }} \ll 1 \) and \(\abs{h_{\mu \nu , \rho } \ll 1/R}\).
In general, we will expand in orders of \(1/R\). 

It can be shown that it is always possible to write the far-field metric in the form 
%
\begin{align}
\dd{s^2} = - \qty(1 - \frac{2M}{r}) \dd{t^2}
+ \qty(1 + \frac{2M}{r}) \dd{r^2} + r^2 \qty( \dd{\theta^2} + \sin^2 \theta \dd{\varphi^2}) - \frac{4 J }{r} \sin^2 \theta \dd{\theta } \dd{\varphi } + \order{ \qty( \frac{1}{r})^2}
\,.
\end{align}

In the weak field limit we have, in terms of the trace-reversed \(\overline{h}_{\mu \nu } = h_{\mu \nu } - (1/2) \eta_{\mu \nu }\):
%
\begin{align}
\square_b \overline{h}_{\mu \nu } = - 16 \pi G_{\mu \nu }
\,,
\end{align}
%
and we can calculate this trace-reversed metric perturbation by integrating:
%
\begin{align}
\overline{h}_{\mu \nu } (x) = 4 \int_{V} \frac{T_{\mu \nu } (x') \dd[4]{x'}}{\abs{\vec{x} - \vec{x}'}}
\,.
\end{align}

This integral can be approximated through the multipole expansion: 
%
\begin{align}
\frac{1}{\abs{\vec{x} - \vec{x}'}} = \frac{1}{r} + \frac{\vec{x} \cdot \vec{x}^{\prime}}{r^2} + \order{r^{-3}}
\,,
\end{align}
%
so the metric perturbation will read 
%
\begin{align}
\overline{h}_{\mu \nu } = \frac{4}{r} \int_{V} T_{\mu \nu } \dd[3]{x} 
+ \frac{4}{r^2} \vec{x} \cdot
\int_{V} T_{\mu \nu } \vec{x}' \dd[3]{x} 
+ \dots
\,,
\end{align}
%
but this last term is just the angular momentum of the object.

In the end one finds 
%
\begin{align}
h_{00} &= \frac{2M}{r} + \order{r^{-3}}  \\
h_{0i} &= \frac{2}{r^2} \epsilon_{ijk} x^{j} J^{k} + \order{r^{-3}}  \\
h_{ij} &= \frac{2M}{r} \delta_{ij} + \order{r^{-3}}
\,.
\end{align}

The metric is then the one we wrote as long as \(\vec{J} = (0, 0, J)\). 

In the strong-field limit we cannot use Newtonian integrals, however we can write the equation \(\tensor{\overline{h}}{^{\mu }_{\nu , \mu} } = 0\). It turns out that the solution is the same, but \(M\) and \(J\) do not necessarily correspond to the mass and angular momentum. 

We can define a stress-energy \emph{pseudotensor} such that 
%
\begin{align}
(-g) \qty(T^{\mu \nu } + t^{\mu \nu }) = \frac{1}{16} \fdv{}{x^{\mu } } \fdv{}{x^{\nu }} \qty(g^{\mu \nu } g^{\nu \rho } - g^{\mu })
\,,
\end{align}
%
\todo[inline]{Fix equation, unreadable}

The pseudo-momentum 
%
\begin{align}
P^{\mu } = \int \dd[3]{x} (-g) \qty(T^{\mu \nu } + t^{\mu \nu })
\,.
\end{align}

The expression for the pseudo-stress-energy tensor is a total divergence, so we can perform the integration on the boundary. We then find \(P^{0}= M\) and \(J = \epsilon_{ijk} \int \dd[3]{x} (-g) x^{i} \qty(T + t)\). 
\todo[inline]{Still unintelligible.}

We can measure the mass from Kepler's laws, seeing the stars orbiting around the object. 
The angular momentum can be measured by looking at the precession of a gyroscope orbiting around the central body.

The angular momentum is described by the intrinsic spin \(S^{\mu }\), which satisfies \(u^{\alpha } S_{\alpha } = 0\) and evolves according to the equation \(u^{\alpha } S^{\mu }_{; \alpha } = 0\): this means 
%
\begin{align}
\dv{S^{i}}{\tau } = \epsilon_{ijk} \omega ^{j} S^{k}
\,,
\end{align}
%
where \(\omega^{k} = (1/2) \epsilon^{kij} h_{0i,j}\). 
The expression depends on the metric perturbation: we have precession with angular velocity 
%
\begin{align}
\vec{\omega} = \frac{1}{r} \qty(- \vec{J} + 3\frac{\vec{J} \cdot \vec{x}}{r^2} \vec{x})
\,.
\end{align}

So far we have not measured the spin of a compact object in this manner, but we have managed to do it with the Earth.  

The multipoles can be approximated as constant for a compact objects. 
In Newtoninan gravity we have 
%
\begin{align}
\nabla^2 \phi = 4 \pi \rho  
\,,
\end{align}
%
and we can write the field solution of this equation as 
%
\begin{align}
\phi (\vec{x}) &= \sum_{\ell, m} \frac{1}{r^{\ell + 1}} \frac{4 \pi }{2 \ell + 1} I_{\ell m} Y_{\ell m} (\theta , \varphi )   \\
&= \sum _{\ell = 0}^{\infty } \frac{1}{r^{\ell+1}} \frac{(2 \ell-1)!!}{\ell!} I^{<i_1 \dots i_\ell>} n^{<i_1} \dots n^{i_\ell>}
\,,
\end{align}
%
where 
%
\begin{align}
I_{\ell m} (t) = \int \rho (t, \vec{x}) r^{\ell} Y_{\ell m} (\theta , \varphi ) \dd[3]{x}
\,.
\end{align}

The angular parentheses mean symmetrization without trace: for example,
%
\begin{align}
n^{<i} n^{j>} = n^{i} n^{j} - \frac{1}{3} \delta^{ij}
\,.
\end{align}

These are the Symmetric Trace-Free moments.
The quadrupole moment is just the \(\ell=2\) term in this expansion: 
%
\begin{align}
Q^{ij} = I^{ij} = \int \rho (t, \vec{x}) \qty(x^{i} x^{j} - \frac{1}{3} \delta^{ij}) \dd[3]{x}
\,.
\end{align}

These Newtonian results can be extended to GR: Thorne has shown that in Asymptotically Cartesian Mass-Centered coordinates the metric can be written in a form like 
%
\begin{align}
g_{00} =-1 + \frac{2M}{r} + \sum _{\ell\geq2} \frac{1}{r^{\ell+1}}
\qty(
    2 \frac{(2 \ell-1)!!}{\ell!} M^{<i_1 \dots i_\ell} n^{<i_1} \dots n^{i_\ell} + \dots
)
\,.
\end{align}

We can define the mass multipole moments and the current multipole moments.

\end{document}