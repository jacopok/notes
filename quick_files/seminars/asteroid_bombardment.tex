\documentclass[main.tex]{subfiles}
\begin{document}

\section{The earliest asteroidal bombardment of the Earth-Moon system}

A talk by Simone Marchi from the Southwest Research Institute in Boulder, CO.

\subsection{Bombardment}

From the Earth's accretion to the Moon forming impact: 90Myr, no samples from this period.
From 90Myr to 600Myr: \emph{late accretion}. The oldest rocks we have started at \(>\SI{600}{Myr} \) after the initial accretion.

On the moon, we have samples on the surface from 90Myr after the Earth's accretion (will omit ``after the Earth's accretion'' from now on).
This is due to the lack of geological activity.

There are 2 orders of magnitude more large (r than \(\SI{50}{km}\)) impact craters on the Moon than on the Earth.

Late heavy bombardment of the moon: there seems to be a spike in the number of asteroids flux.

Siderophilic elements: stuff that is found in the core, there should be no more of it after the Moon's formation: we see more of it than expected, maybe it got here later?

There are siderophilic elements in asteroids.

Spike in Iridium corresponding to the K-T  extinction event.

(then I got tired of taking notes)

\end{document}
