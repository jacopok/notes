\documentclass[main.tex]{subfiles}
\begin{document}

\section{Exploring the universe with gravitational waves}

Eugenio Coccia, Gran Sasso Science Institute \& INFN.
\emph{Rettore} at GSSI.
Member of the VIRGO collaboration.

Idea for a demonstration: two masses anchored to the surface of a sphere, with a spring between them. As they move perpendicular to the spring, the spring will be contracted by the geodesic ``force''.
If the masses stop, the spring extends back.
``But why can't they stop in GR?'': the norm of the four-velocity is fixed. 

The Einstein equations as a spring:
%
\begin{equation}
  G_{\mu \nu } = \frac{8 \pi G }{c^{4}} T_{\mu \nu } \iff
  F = -kx
\,,
\end{equation}
%
identifying the stress-energy tensor with the force, the Einstein tensor with the position. Then, the elastic constant is \(c^{4}/ (8 \pi G) \approx \SI{e45}{kg s^{-2}}\): very stiff! 

GW: Einstein 1916. If \(g_{\mu  \nu }= \eta_{ \mu \nu } + h_{\mu \nu }\), with small \(h_{\mu \nu }\), then the field equations become \(\square h_{\mu \nu }=0\).

They are associated with massless spin 2 particles: gravitons. No dipole radiation is emitted gravitationally by conservation of momentum (\(\nabla_{\mu T^{\mu i}}=0\)?)

They have two polarization states.
Kelvin was not ok with \(E\) and \(B\) being orthogonal to the propagation of EM radiation.

Circularly positioned masses will be deformed into ellipses (?).
Einstein thought that GW could never be practically measured: the displacement is of the order \(\Delta L / L \approx \num{e-18}\).
The luminosity depends on the quadrupole moment: a 1000 ton steel rotor at \SI{4}{Hz} would emit \SI{e-30}{W} in GW (this was what Einstein was thinking of), while \(1.4 M_{\odot}\) neutron stars emit \SI{e52}{W} in GW.

The objectives are 

\begin{enumerate}
    \item test GR;
    \item explore dark regions for astrophysical purposes;
    \item explore beyond the CMB for cosmological purposes.
\end{enumerate}

We can detect supernovae, spinning neutron stars, coalescing binaries, and the background. All of these have typical shapes, signatures.

The turning point in GW astronomy: 1957, the ``Chapel Hill meeting''.
Debates on the covariance of GWs.

Pirani points out the equivalence of the equation of geodesic deviation with the Newton equation, identifying \(R_{a 0 a 0 } \sim \partial_{a } \partial_b V\), where \(V\) is the newtonian potential (?)

Indirect evidence was found by fitting experimental data of the period shift in time for binaries based on energy lost to GW: it precisely matched the GR prediction.

The instrument is a Michelson-Morley interferometer.

The relative velocity reaches \(0.5c\) (!!).
No light is emitted from a BH merger: only a perturbation of spacetime.

Convolving a prepared waveform with the noisy signal we can see the signal even if it is lower than the noise.

We have a limit on the graviton mass, on the order of \SI{e-23}{eV} --- lower than the photon bound!
This is found adding a dispersive term \(E^2 = p^2c^2+ A p^{a} c^{a}\)\dots

\todo[inline]{How does this work? What is the vector \(c^{a}\)?

I'd expect \(E^2 - p^2 c^2 = m^2 c^{4}\)\dots
}

We can do localization by comparing the data from different detectors.

After the GW alarm we can spot the place they were coming from, and look at the varying spectrum: we can see \emph{heavy metal lines} (after iron).

New perspectives: underground GW detectors to reduce seismic noise; LISA and eLISA in space to detect lower frequencies.

\end{document}
