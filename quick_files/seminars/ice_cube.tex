\documentclass[main.tex]{subfiles}
\begin{document}

\section{Ice Cube}

Hansen.

Nobody has seen anything above \SI{100}{TeV} in astronomy: 
photons with that high of an energy interact with CMB photons to do pair production, so we cannot detect them

Neutrinos do not exhibit this behaviour: they propagate through space essentially unabsorbed.

The flux of cosmic rays decreases as a power law. To generate photons with \SI{e20}{eV} we'd need a LHC with the radius of the orbit of Mercury.

To generate photons with that high of an energy we can have extreme gravitational phenomena, whose energy is converted into kinetic energy. 

Neutrinos are produced from the decay of \(\pi \) particles, which come from proton beams, we have \(\pi \rightarrow \mu + \nu_{\mu } \) and \(\mu \rightarrow e + \nu_{e}\). 

Every neutrino has a corresponding \(\gamma \) with twice its energy.

Ice cube detects a neutrino every \SI{5}{min}. 
The atmospheric neutrino flux also is a powerlaw: we are working at around \SI{100}{TeV}, at the edge of the known area. 

The method for detection is to have a big volume of transparent material, the neutrinos come through and when they decay we detect the Cherenkov radiation using photomultipliers. 

At the geographic south pole, below \SI{1.35}{km} of ice the \emph{ultra-transparent} ice start. 
\num{e11} atmospheric muons, \num{e5} astrophysical neutrinos decaying to muons, \num{e1} to \num{e2} cosmic neutrinos decaying to muons. 

The ice is melted going down, and then the photomultipliers are lowered and we wait for the ice to refreeze, this is slow since ice is an insulator. 

For now the data are compatible with a 1:1:1 neutrino composition. 

A \(\tau \) neutrino has a travel time of \(\sim \SI{50}{ns}\) at the energies of IceCube. 

\todo[inline]{What is the angular resolution of the determination of the position?}

Right now \SI{.2}{\degree} to \SI{.4}{\degree}, they aim for \SI{.1}{\degree}.

The neutrino radiation seems to be almost uniform throughout the sky. 

A shocking conclusion: the energy density of neutrinos and that of gamma rays is the same. 

Within less than a minute of the detection the information is sent to the Fermi satellite, which can look at it. 

IceCube is on \(> 99.5\%\) of the time. 

\textbf{Neutrino astrophysics exists}. 

\end{document}
