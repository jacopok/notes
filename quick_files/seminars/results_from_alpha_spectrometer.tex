\documentclass[main.tex]{subfiles}
\begin{document}

\section{The Latest Results from the Alpha Magnetic Spectrometer on the International Space Station}

\marginpar{Wednesday\\ 2021-11-10, \\ compiled \\ \today}

A lecture by Nobel Laureate Samuel C.\ C.\ Ting.

The lecture starts with a history of AMS. 

It measures several properties of incoming cosmic rays. 
The largest systematic error is the uncertainty on the absolute scale of the momentum. 

They quantify the momentum uncertainty as \(1/ \abs{p} - 1 / E\), which is of the order of \SI{30}{TeV^{-1}}. 

\todo[inline]{Why is it expressed in this way?}

AMS searches for DM by looking for annihilations, like \(\chi + \chi \to e^{+} + p + \gamma \dots\). 

The rest of the talk discusses the interpretation of the various spectra. 

Pulsars produce positrons but not antiprotons, yet we observe them in similar amounts. 

There is a cutoff in energy in the positron spectrum, at \(4 \sigma \). 

``AMS will provide the definitive answer on the nature of dark matter''.

\todo[inline]{What is the way they are doing non-uniform binning? Are they bins with the same amount of events inside them? Might be because of the \(E^3\) multiplication\dots}

He puts Fluorine together with B, Be, Li as spallation products\dots is this correct? 

They always work in terms of rigidity, \(R = \text{momentum} / \text{charge}\).

They measure increasing flux over the years.
Improvement in the instrument? What's up?

\end{document}
