\documentclass[main.tex]{subfiles}
\begin{document}


\section{Svelare l'universo}

\section*{Thu Oct 17 2019}

``L'eccellenza padovana nella ricerca di frontiera''.

Parla Flavio Seno, nostro nuovo direttore. 
``Eccellenza'' non è un termine utilizzato a sproposito: dei 980 dipartimenti nazionali, 180 sono stati selezionati, e finanziati con cifre cospicue, dell'ordine di \num{e7}€.

Il lavoro non è stato in gran parte suo, bensì della Soramel, sua precorritrice.

Galileo nell'ottobre 1604 ha scritto l'equazione del moto di una particella in caduta libera, e ha capito che la \emph{stella nova} che aveva osservato non era più vicina alla Terra della Luna.

Abbiamo degli speaker molto importanti, la cui attività di ricerca è strettamente collegata a ciò che si fa in DFA.

Si inizia. C'è anche MicMap fra gli speaker!

\subsection{Astrofisica a molti messaggeri}

di Marica Branchesi, GSSI L'Aquila: lei ha cercato di collegare i mondi della fisica e dell'astronomia. È stata selezionata come una delle 10 persone più influenti nel mondo della Fisica, Time l'ha inserita fra le 100 persone più influenti al mondo.

I Gamma Ray Bursts sono classificati in ``corti'' e lunghi: durata \(\lessgtr \SI{2}{s}\).  

Si menzionano molte cose in velocità: Supermassive Black Holes, GW Astronomy.

A Padova ci sono stati i gruppi che hanno studiato i sistemi binari di buchi neri e stelle di neutroni?

Insieme alla rilevazione di onde gravitazionali bisogna osservare in altri modi: è stato visto un segnale estremamente lungo, e contemporaneamente Fermi l'ha visto nel range \(\gamma \) e INTEGRAL l'ha confermato nei raggi X.

\todo[inline]{Qual è la scala temporale della risposta? quanto ci mettiamo a puntare Fermi et al.\ nella direzione giusta?}

I telescopi (infrarossi, cileni) hanno visto un nuovo oggetto che rapidamente diminuiva in temperatura (con una distribuzione di radiazione circa di corpo nero, parrebbe).

Adalberto Giazotto, padre di VIRGO: ``Rivelare le onde gravitazionali, nessuna idea era più folle di questa''.

MicMap risponde a ``cosa succede a Padova''?
Padova è stata attiva per lo sviluppo di AURIGA, predecessore di VIRGO.

Ci sono tre finanziamenti europei indipendenti, IRC, riguardo lo studio delle binarie: formazione, caratterizzazione.
C'è anche molto riguardo l'astronomia multimessaggio: nello studio che ha menzionato Marika nella controparte EM del binary neutron star merger c'era anche PD, nello specifico nello studio dei dati di Fermi.

Il DFA è anche coinvolto nello studio delle osservazioni di neutrini.

\subsection{Osservando la luce dell'universo più lontano}

Licia Verde, ICREA \& ICC-UB-IEEC Barcelona. 
Tesista (magistrale, perlomeno) di Sabino \(\heartsuit\).

Ci dev'essere una teoria del tutto.
Jim Peebles la settimana scorsa ha preso il Nobel: lui ha contribuito alla formazione del modello \(\Lambda \)CDM (?).

La legge di Hubble è \(cz = v = H_0 d\).

1998: l'espansione dell'universo è accelerata, Nobel 2011. 

Bullet Cluster: un ``applauso cosmico''. La materia oscura, però, non interagisce se non gravitazionalmente.

Scoperta della CMB. Appariva uniforme, circa a \SI{3}{K}, ma furono poi notate delle piccole anisotropie.
È stato emesso quando l'universo aveva \SI{3.8e5}{yr}.

``Planck potrebbe misurare la temperatura di un coniglio sulla Luna dalla Terra''.

\todo[inline]{In termini di risoluzione angolare o spettrale? La temperatura del coniglio è abbastanza diversa da quella della Luna, no?

Dal sito dell'ESA: si parla di risoluzione spettrale, e l'affermazione è che Planck riesce a vedere il \SI{}{\micro\kelvin}.
Temperatura media di un coniglio: da \num{310} a \SI{312}{K} circa, temperatura della luna super variabile, da \SI{100}{K} a \SI{400}{K} circa.
}

Molte domande senza risposta: perché l'universo si espande etc. 

\subsection{Exoplanets}

Willy Benz, University of Bern, Switzerland.

We use Doppler shift to measure the motion of stars due to the presence of exoplanets. 861 exoplanets have been found, to date. This allows us to measure the mass \(M\).

Another method is the transit measurement. This allows us to measure the radius \(R\).

Empirically, the number of known exoplanets is increasing exponentially. There are however huge peaks in 2014 and 2016. 

\todo[inline]{why?}

We can plot the mean density: \(\rho = 3 M / (4 \pi R^3 )\). We can distinguish the giant gas planets from the rocky ones.
Right now, however, there are huge error bars on the densities.

We are building the CHEOPS mission for this. It is on schedule and on budget! 

The angular size of the star we are interested in is \SI{0.77}{\arcsecond}.

Most stars in a 10 parsec ball around here are small M type. As star brightness decreases, it gets easier to detect exoplanets indirectly but harder to image them spectroscopically.

People are doing high contrast spectroscopy: actually \emph{seeing} the exoplanet. Wow!

Good question: how do we measure the atmospheric pressure of an exoplanets?

\end{document}
