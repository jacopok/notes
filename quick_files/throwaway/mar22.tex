\documentclass[main.tex]{subfiles}
\begin{document}

We define the 4 events as in the figure, and impose: 
%
\begin{align}
\operatorname{dist} \qty(A_1, B_2 ) &= 0 \\ 
\operatorname{dist} \qty(B_1, A_2  ) &= 0 \\ 
\operatorname{dist} \qty(A_1, A_2  ) &=  
\operatorname{dist} \qty(B_1, B_2  ) = - \tau 
\marginnote{the timelines are time-like, and we are using the mostly plus signature}
\,.
\end{align}

The velocities are approximated as \(u^{\mu }_A = u^{\mu }_{B} = (1, \vec{0}) \overset{\text{def}}{=} u^{\mu }\). So, for the \(2\) positions we have (to linear order): 
%
\begin{align}
x^{\mu }_{i, 2} = x^{\mu }_{i, 1} + \tau u^{\mu }
\,,
\end{align}
%
where \(i = A, B\). 
Also, we are assuming \(x^{\mu }_{B, 1} = x^{\mu }_{A, 1} + \dd{x^{\mu }}\). So, we get 
%
\begin{align}
\qty(x^{\mu }_{B, 2} - x^{\mu }_{A, 1})^2 &= \qty(x^{\mu }_{A, 1} + \dd{x^{\mu }} + \tau u^{\mu } - x^{\mu }_{A, 1})^2  \\
&= \qty(\dd{x^{\mu }} + \tau u^{\mu })^2  \\
&= \dd{x^{\mu }} \dd{x_{\mu }} + 2 \dd{x^{\mu }} \tau u_{\mu } + \tau^2 u^{\mu } u_{\mu } =0
\,.
\end{align}

Now, for the other distance we have instead 
%
\begin{align}
\qty(x^{\mu }_{A, 2} - x^{\mu }_{B, 1})^2 &=
\qty(x^{\mu }_{B, 1} - \dd{x^{\mu }} + \tau u^{\mu } - x^{\mu }_{B, 1})^2  \\
&= \qty(-\dd{x^{\mu }} + \tau u^{\mu })^2  \\
&= \dd{x^{\mu }} \dd{x_{\mu }} - 2 \dd{x^{\mu }} \tau u_{\mu } + \tau^2 u^{\mu } u_{\mu } =0
\,,
\end{align}
%
so subtracting the two we get 
%
\begin{align}
4 \dd{x^{\mu }} u_{\mu } \tau = 0
\,,
\end{align}
%
which must hold for any (small) \(\tau \), so we must have \(u_{\mu }\dd{x^{\mu }}=u^{\mu } g_{\mu \nu } \dd{x^{\nu }} =  0\), therefore \(g_{0 \mu } \dd{x^{\mu }}= 0\).


\end{document}